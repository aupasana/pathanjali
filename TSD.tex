\titleformat*{\section}{\large\bfseries }
	\renewcommand*\contentsname{\Large विषयानुक्रमणिका}
	\title{॥तर्कसङ्ग्रहदीपिका॥}\par
	\author{अन्नम्भट्टः \\ \\
		सम्पादकः - श्रीनिवास कारन्तः}
	\date{}
	\maketitle
	\frontmatter
	\tableofcontents
	\mainmatter
	\renewcommand{\thepage}{\devanagarinumeral{page}}
\section*{मङ्गळवादः}
\addcontentsline{toc}{section}{१. मङ्गळवादः}
	\begin{center}
		तर्कसङ्ग्रहः\\
		निधाय हृदि विश्वेशं विधाय गुरुवन्दनम्~।\\
		बालानां सुखबोधाय क्रियते तर्कसङ्ग्रहः~॥\\
	 दीपिका\\
		विश्वेश्वरं साम्बमूर्तिं प्रणिपत्य गिरां गुरुम्~।\\
		टीकां शिशुहितां कुर्वे तर्कसङ्ग्रहदीपिकाम् ~॥
	\end{center}
		चिकीर्षितस्य ग्रन्थस्य निर्विघ्नपरिसमाप्त्यर्थं शिष्टाचारानुमितश्रुतिबोधितकर्तव्यताकम् इष्टदेवतानमस्कारात्मकं मङ्गलं शिष्यशिक्षायै ग्रन्थतो निबध्नंश्चिकीर्षितं प्रतिजानीते- निधायेति~। ननु मङ्गलस्य समाप्तिसाधनत्वं नास्ति~। मङ्गले कृतेऽपि कादम्बर्यादौ समाप्त्यदर्शनात्, मङ्गलाभावेऽपि किरणावल्यादौ समाप्तिदर्शनाच्च अन्वयव्यतिरेकव्यभिचारादिति चेत् न~। कादम्बर्यादौ विघ्नबाहुल्यात्समाप्त्यभावः~। किरणावल्यादौ ग्रन्थाद्बहिरेव मङ्गलं कृतमतो न व्यभिचारः~। ननु मङ्गलस्य कर्तव्यत्वे किं प्रमाणमिति चेत् न~। शिष्टाचारानुमितश्रुतेरेव प्रमाणत्वात्~। तथा हि - मङ्गलं वेदबोधितकर्तव्यताकम्, अलौकिकाविगीतशिष्टाचारविषयत्वात्~, दर्शादिवत्~। भोजनादौ व्यभिचारवारणाय अलौकिकेति~। रात्रिश्राद्धादौ व्यभिचारवारणाय अविगीतेति~। शिष्टपदं स्पष्टार्थम्~। {\bfseries न कुर्यान्निष्फलं कर्म} इति जलताडनादेरपि निषिद्धत्वात्~। तर्क्यन्ते प्रतिपाद्यन्ते इति तर्काः - द्रव्यादिपदार्थास्तेषां सङ्ग्रहः - सङ्क्षेपेण स्वरूपकथनं क्रियत इत्यर्थः~। कस्मै प्रयोजनायेत्यत आह - सुखबोधायेति~। सुखेनानायासेन यो बोधः पदार्थज्ञानं, तस्मा इत्यर्थः~। ननु बहुषु तर्कग्रन्थेषु सत्सु किमर्थमपूर्वोऽयं ग्रन्थः क्रियत इत्यत आह~। बालानामिति~। तेषामतिविस्तृतत्वाद्बालानां बोधो न जायत इत्यर्थः~। ग्रहणधारणपटुर्बालः न तु स्तनन्धयः~। किं कृत्वा क्रियत इत्यत आह - निधायेति~। विश्वेश्वरं जगन्नियन्तारम्~। हृदि निधाय - नितरां स्थापयित्वा सदा तद्ध्यानपरो भूत्वेत्यर्थः~। गुरूणां विद्यागुरूणां वन्दनं, नमस्कारं विधाय कृत्वेत्यर्थः ~॥
\section*{अथ पदार्थनिरूपणम्}
\addcontentsline{toc}{section}{२. पदार्थनिरूपणम्}
	{\bfseries द्रव्यगुणकर्मसामान्यविशेषसमवायाऽभावाः सप्त पदार्थाः ~॥}\par
		पदार्थान् विभजते - द्रव्येति~। पदस्यार्थः पदार्थः इति व्युत्पत्त्याऽभिधेयत्वं पदार्थसामान्यलक्षणम् इति लभ्यते~। ननु विभागादेव सप्तत्वे सिद्धे सप्तग्रहणं व्यर्थमिति चेत्, न~। अधिकसङ्ख्याव्यवच्छेदार्थत्वात्~। नन्वतिरिक्तः पदार्थः प्रमितो न वा~। नाऽद्यः, प्रमितस्य निषेधायोगात्~। न द्वितीयः, प्रतियोगिप्रमितिं विना निषेधानुपपत्तेरिति चेत्, न~। पदार्थत्वं द्रव्यादिसप्तान्यतमत्वव्याप्यमिति व्यवच्छेदार्थत्वात्~। ननु सप्तान्यतमत्वं सप्तभिन्नभिन्नत्वमिति वक्तव्यम्~। सप्तभिन्नस्याप्रसिद्ध्या सप्तान्यतमत्वम् कथम्~? इति चेत्, न~। द्रव्यादिसप्तान्यतमत्वं द्रव्यादिभेदसप्तकाभाववत्त्वमिति तदर्थत्वात्~। एवमग्रेऽपि द्रष्टव्यम् ~॥\\[10pt]
	{\bfseries तत्र द्रव्याणि पृथिव्यप्तेजोवाय्वाकाशकालदिगात्ममनांसि नवैव ~॥}\par
		द्रव्याणि विभजते - तत्रेति~। तत्र - द्रव्यादिमध्ये~। द्रव्याणि नवैवेत्यन्वयः~। कानि तानि इत्यत आह - पृथिवीति~। ननु तमसो दशमद्रव्यस्य विद्यमानत्वात्कथं नवैव द्रव्याणीति~। तथा हि~। नीलं तमश्चलतीत्यबाधितप्रतीतिबलान्नीलरूपाधारतया, क्रियाधारतया च तमसो द्रव्यत्वं तावत् सिद्धम्~। तत्र तमसो नाकाशादिपञ्चकेऽन्तर्भावः रूपवत्त्वात्~। अत एव न वायौ, स्पर्शाभावात् सदागतिमत्त्वाभावाच्च~। नापि तेजसि, भास्वरूपाभावात् उष्णस्पर्शाभावाच्च~। नापि जले, शीतस्पर्शाभावात् नीलरूपवत्त्वाच्च~। नापि पृथिव्यां, गन्धाभावात् स्पर्शरहितत्वाच्च~। तस्मात्तमो दशमद्रव्यमिति चेत्, न~। तमसस्तेजोऽभावरूपत्वात्~। तथा हि~। तमो न रूपिद्रव्यमालोकाऽसहकृतचक्षुर्ग्राह्यत्वादालोकाभाववत्~। रूपिद्रव्यचाक्षुषप्रमायामालोकस्य कारणत्वात्~। तस्मात् प्रौढप्रकाशकतेजस्सामान्याभावस्तमः~। तत्र नीलं तमश्चलति इति प्रत्ययो भ्रमः~। अतो नव द्रव्याणीति सिद्धम्~। द्रव्यत्वजातिमत्त्वं गुणवत्त्वं वा द्रव्यसामान्यलक्षणम्~। लक्ष्यैकदेशावृत्तित्वमव्याप्तिः~। यथा गोः कपिलत्वम्~। अलक्ष्ये लक्षणस्य वर्तनमतिव्याप्तिः~। यथा गोः श्रृङ्गित्वम्~। लक्ष्यमात्रावृत्तित्वमसम्भवः~। यथा गोरेकशफवत्त्वम्~। एतद्दूषणत्रयरहितो धर्मो लक्षणम्~। स एवासाधारणधर्म इत्युच्यते~। लक्षयतावच्छेदकसमनियतत्वमसाधारणत्वम्~। व्यावर्तकस्यैव लक्षणत्वे व्यावृत्तावभिधेयत्वादौ चातिव्याप्तिरतस्तद्वारणाय तद्भिन्नत्वं धर्मविशेषणं देयम्~। व्यवहारस्यापि लक्षणप्रयोजनत्वे तन्न देयम्~, व्यावृत्तेरपि व्यवहारसाधनत्वात्~। ननु गुणवत्त्वं न द्रव्यलक्षणम्, आद्यक्षणावच्छिन्नघटे उत्पन्नविनिष्टघटे चाव्याप्तेरिति चेत्, न~। गुणसमानाधिकरणसत्ताभिन्नजातिमत्त्वस्य विवक्षितत्वात्~। नन्वेवमपि एकं रूपं रसात्पृथक् इति व्यवहाराद्रूपादावतिव्याप्तिरिति चेत्, न~। एकार्थसमवायादेव तादृशव्यवहारोपपत्तौ गुणे गुणानङ्गीकारात्~।\\[10pt]
	{\bfseries रूप-रस-गन्ध-स्पर्श-सङ्ख्या-परिमाण-पृथक्‍त्व-संयोग-विभाग-परत्वा-ऽपरत्व-गुरुत्व-द्रवत्व-स्‍नेह-शब्द-बुद्धि-सुख-दुःखेच्छा-द्वेष-प्रयत्‍न-धर्माधर्म-संस्काराः चतुर्विंशतिर्गुणाः ~॥}\par
		गुणान्विभजते- रूपेति~। द्रव्यकर्मभिन्नत्वे सति सामान्यवान् गुणः~। गुणत्वजातिमान्वा~। ननु लघुत्वकठिनत्वमृदुत्वादीनां विद्यमानत्वात् कथं चतुर्विंशतिर्गुणा इति चेत्, न~। लघुत्वस्य गुरुत्वाभावरूपत्वान्मृदुत्वकठिनत्वयोः अवयवसंयोगविशेषत्वात्~।\\[10pt]
	{\bfseries उत्क्षेपणापक्षेपणाकुञ्चनप्रसारणगमनानि पञ्च कर्माणि ~॥}\par
		कर्म विभजते~। उत्क्षेपणेति~। संयोगभिन्नत्वे सति संयोगासमवायिकारणं कर्म कर्मत्वजातिमद्वा~। भ्रमणादीनामपि गमनेऽन्तर्भावान्न पञ्चत्वविरोधः ~॥\\[10pt]
	{\bfseries परमपरं चेति द्विविधं सामान्यम् ~॥}\par
		सामान्यं विभजते~। परमिति~। परमधिकदेशवृत्ति~। अपरं न्यूनदेशवृत्ति~। सामान्यादिचतुष्टये जातिर्नास्ति ~॥\\[10pt]
	{\bfseries नित्यद्रव्यवृत्तयो विशेषास्त्वनन्ता एव ~॥}\par
		विशेषं विभजते~। नित्येति~। पृथिव्यादिचतुष्टयपरमाणवः आकाशादिपञ्चकं च नित्यद्रव्याणि~।\\[10pt]
	{\bfseries समवायस्त्वेक एव ~॥}
		समवायस्य भेदो नास्तीत्याह- समवायस्त्विति~।\\[10pt]
	{\bfseries अभावश्चतुर्विधः - प्रागभावः, प्रध्वंसाभावः, अत्यन्ताभावः, अन्योन्याभावश्चेति ~॥}\par
		अभावं विभजते - प्रागभावेति~। \section*{अथ द्रव्यनिरूपणम्}
\addcontentsline{toc}{section}{३. द्रव्यनिरूपणम्}
	{\bfseries तत्र गन्धवती पृथिवी~। सा द्विविधा, नित्याऽनित्या च~। नित्या परमाणुरूपा~। अनित्या कार्यरूपा~। पुनस्त्रिविधा शरीरेन्द्रियविषयभेदात्~। शरीरमस्मदादीनाम्~। इन्द्रियं गन्धग्राहकं घ्राणम् , नासाग्रवर्ति~। विषयो मृत्पाषाणादिः ~॥}\par
		तत्रोद्देशक्रमानुसारात् प्रथमं पृथिव्या लक्षणमाह - तत्रेति~। नाम्ना पदार्थसङ्कीर्तनमुद्देशः~। उद्देशक्रमे च सर्वत्रेच्छैव नियामिका~। ननु सुरभ्यसुरभ्यवयवारब्धे द्रव्ये परस्परविरोधेन गन्धानुत्पादादव्याप्तिः~। न च तत्र गब्धप्रतीत्यनुपपत्तिरिति वाच्यम्~। अवयवगन्धस्यैव तत्र प्रतीतिसम्भवेन चित्रगन्धानङ्गीकारात्~। किं चोत्पन्नविनष्टघटादावव्याप्तिरिति चेत् न~। गन्धसमानाधिकरणद्रव्यत्वापरजातिमत्त्वस्य विवक्षितत्वात्~। ननु जलादावपि गन्धप्रतीतिरिति चेत् न~। अन्वयव्यतिरेकाभ्यां पृथिवीगन्धस्यैव तत्र भानाङ्गीकारात्~। ननु कालस्य सर्वाधारतया सर्वेषां लक्षणानां कालेऽतिव्याप्तिरिति चेत् न~। सर्वाधारताप्रयोजकभिन्नसम्बन्धेन लक्षणत्वस्य विवक्षितत्वात्~। पृथिवीं विभजते सा द्विविधेति~। नित्यत्वं ध्वंसाप्रतियोगित्वम्~। अनित्यत्वं ध्वंसप्रतियोगित्वम्~। प्रकारान्तरेण विभजते - पुनरिति~। आत्मनो भोगायतनं शरीरम्~। यदवच्छिन्नात्मनि भोगो जायते तद्भोगायतनम्~। सुखदुःखान्यतरसाक्षात्कारो भोगः~। शब्देतरोद्भूतविशेषगुणानाश्रयत्वे सति ज्ञानकारणमनःसंयोगाश्रयत्वमिन्द्रियत्वम्~। शरीरेन्द्रियभिन्नो विषयः~। एवं च गन्धवच्छरीरं पार्थिवशरीरम्~, गन्धवदिन्द्रियं पार्थिवेन्द्रियम्~, गन्धवान् विषयः पार्थिवविषयः इति तत्तल्लक्षणं बोध्यम्~। पार्थिवशरीरं दर्शयति - शरीरमिति~। इन्द्रियं दर्शयति - इन्द्रियमिति~। गन्धग्राहकमिति प्रयोजनम्~। घ्राणमिति संज्ञा~। नासाग्रेत्याश्रयोक्तिः~। एवमुत्तरत्र ज्ञेयम्~। पार्थिवविषयं दर्शयति - मृत्पाषाणादीति~।\\[10pt]
	{\bfseries शीतस्पर्शवत्यः आपः~। ता द्विविधाः, नित्या अनित्याश्च~। नित्याः परमाणुरूपाः~। अनित्याः कार्यरूपाः~। पुनस्त्रिविधाः शरीरेन्द्रियविषयभेदात्~। शरीरं वरुणलोके~। इन्द्रियं रसग्राहकं रसनं जिह्वाग्रवर्ति~। विषयः सरित्समुद्रादिः~॥}\par
		अपां लक्षणमाह - शीतेति~। उत्पन्नविनष्टजलेऽव्याप्तिवारणाय शीतस्पर्शसमानाधिकरणद्रव्यत्वापरजातिमत्वं विवक्षितम्~। "शीतं शिलातलम्" इत्यादौ जलसम्बन्धादेव शीतस्पर्शभानमिति नातिव्याप्तिः~। अन्यत्सर्वं पूर्वरीत्या व्याख्येयम्~।\\[10pt]
	{\bfseries उष्णस्पर्शवत् तेजः~। तच्च द्विविधं, नित्यमनित्यं च~। नित्यं परमाणुरूपम्~। अनित्यं कार्यरूपम्~। पुनः त्रिविधं शरीरेन्द्रियविषयभेदात्~। शरीरमादित्यलोके प्रसिद्धम्~। इन्द्रियं रूपग्राहकं चक्षुः कृष्णताराग्रवर्ति~। विषयः चतुर्विधः, भौम-दिव्य-औदर्य-आकरजभेदात्~। भौमं वह्न्यादिकम्~। अबिन्धनं दिव्यं विद्युदादि~। भुक्तस्य परिणामहेतुरौदर्यम्~। आकरजं सुवर्णादि~॥}\par
		तेजसो लक्षणमाह - उष्णस्पर्शवदिति~। "उष्णं जलम्" इति प्रतीतेस्तेजः संयोगानुविधायित्वान्नातिव्याप्तिः~। विषयं विभजते भौमेति~। ननु सुवर्णं पार्थिवं, पीतत्वाद्गुरुत्वात् हरिद्रावत् इति चेत् न~। अत्यन्तानलसंयोगे सति घृतादौ द्रवत्वनाशदर्शनेन, जलमध्यस्थघृतादौ तन्नाशादर्शनेन च असति प्रतिबन्धके पार्थिवद्रवत्वनाशाग्निसंयोगयोः कार्यकारणभावावधारणात्सुवर्णस्य अत्यन्तानलसंयोगे सत्यनुच्छिद्यमानद्रवत्वाधिकरणत्वेन पार्थिवत्वानुपपत्तेः, पीतद्रव्यद्रवत्वनाशप्रतिबन्धकतया द्रवद्रव्यान्तरसिद्धौ नैमित्तिकद्रवत्वाधिकरणतया जलत्वानुपपत्तेः रूपवत्तया वाय्वादिष्वनन्तर्भावात्तैजसत्वसिद्धिः~। तस्योष्णस्पर्शभास्वररूपयोरुपष्टम्भकपार्थिवरूपस्पर्शाभ्यां प्रतिबन्धादनुपलब्धिः~। तस्मात् सुवर्णं तैजसमिति सिद्धम्~।\\[10pt]
	{\bfseries रूपरहितः स्पर्शवान् वायुः~। स द्विविधः नित्यः अनित्यश्च~। नित्यः परमाणुरूपः~। अनित्यः कार्यरूपः~। पुनः त्रिविधः शरीरेन्द्रियविषयभेदात्~। शरीरं वायुलोके~। इन्द्रियं स्पर्शग्राहकं त्वक् सर्वशरीरवर्ति~। विषयो वृक्षादिकम्पनहेतुः~। शरीरान्तः सञ्चारी वायुः प्राणः~। स च एकोऽपि उपाधिभेदात् प्राणापानादिसंज्ञां लभते~॥}\par
		वायुं लक्षयति रूपरहितेति~। आकाशादावतिव्याप्तिवारणाय स्पर्शवानिति~। पृथिव्यादावतिव्याप्तिवारणाय रूपरहितेति~। प्राणस्य कुत्रान्तर्भाव इत्यत आह - शरीरेति~। स च इति - एक एव प्राणः स्थानभेदात्प्राणापानादिशब्दैः व्यवह्रियत इत्यर्थः~। स्पर्शानुमेयो वायुः~। तथा हि - योऽयं वायौ वाति सति अनुष्णाशीतस्पर्श उपलभ्यते स क्वचिदाश्रितः गुणत्वाद्रूपवत्~। न चास्य पृथिवी आश्रयः, उद्भूतस्पर्शवतः पार्थिवस्योद्भूतरूपवत्त्वनियमात्~। न जलतेजसी अनुष्णाशीतस्पर्शवत्वात्~। न विभुचतुष्टयं सर्वत्रोपलब्धिप्रसङ्गात् न मनः परमाणुस्पर्शस्यातीन्द्रियत्वात्~। तस्माद्यः प्रतीयमानस्पर्शाश्रयः स वायुः~। ननु वायुः प्रत्यक्षः, प्रत्यक्षस्पर्शाश्रयत्वात्~, घटवदिति चेत् न~। उद्भूतरूपस्योपाधित्वात्~। "यत्र द्रव्यत्वे सति बहिरिन्द्रियजन्यप्रत्यक्षत्वं, तत्र उद्भूतरूपवत्वम्" इति घटादौ साध्यव्यापकत्वम्~। "यत्र प्रत्यक्षस्पर्शाश्रयत्वं तत्र उद्भूतरूपवत्त्वं नास्ति" इति पक्षे साधनाव्यापकत्वम्~। न चैवं तप्तवारिस्थतेजसोऽपि अप्रत्यक्षत्वापत्तिः, इष्टत्वात्~। तस्माद्रूपरहितत्वाद्वायुरप्रत्यक्षः~। इदानीं कार्यरूपपृथिव्यादिचतुष्टयस्योत्पत्तिविनाशक्रमः कथ्यते~। ईश्वरस्य चिकीर्षावशात्परमाणुषु क्रिया जायते~। ततः परमाणुद्वयसंयोगे द्व्यणुकमुत्पद्यते~। त्रिभिरेव द्व्यणुकैस्त्र्यणुकमुत्पद्यते~। एवं चतुरणुकादिक्रमेण महती पृथिवी, महत्य आपः; महत्तेजः; महान्वायुरुत्पद्यते~। एवमुत्पन्नस्य कार्यद्रव्यस्य सञ्जिहीर्षावशात् परमाणुषु क्रिया, क्रियया परमाणुद्वयविभागे द्व्यणुकनाशः~। इत्येवं महापृथिव्यादिनाशः~। असमवायिकारणनाशात् द्व्यणुकनाशः~। समवायिकारणनाशात् त्र्यणुकनाश इति सम्प्रदायः सर्वत्रासमवायिकारणनाशात् द्रव्यनाश इति नवीनाः~। परमाणुसद्भावे किं प्रमाणम्~। उच्यते, जालसूर्यमरीचिस्थं सूक्ष्मतमं यद्रजो दृश्यते तत्सावयवं चाक्षुषद्रव्यत्वात् घटवत्~। त्र्यणुकावयवोऽपि सावयवः महदारम्भकत्वात्कपालवत्~। यो द्व्यणुकावयवः स एव परमाणुः~। स च नित्यः, तस्यापि कार्यत्वे अनवस्थाप्रसङ्गात्~। तथा च मेरुसर्षपयोरपि तुल्यपरिमाणापत्तेः~। सृष्टिप्रलयसद्भावे {\bfseries धाता यथापूर्वमकल्पयत्} इति श्रुतिरेव प्रमाणम्~। सर्वकार्यद्रव्यध्वंसोऽवान्तरप्रलयः, सर्वभावकार्यध्वंसो महाप्रलयः इति विवेकः~।\\[10pt]
	{\bfseries शब्दगुणकमाकाशम्~। तच्चैकं विभु नित्यञ्च~॥}\par
		आकाशं लक्षयति शब्दगुणकमिति~। ननु किमाकाशमपि पृथिव्यादिवन्नाना~?
		नेत्याह तच्चैकमिति~। भेदे प्रमाणाभावादित्यर्थः~। एकत्वादेव सर्वत्रोपलब्धेर्विभुत्वमङ्गीकर्तव्यमित्याह विभ्विति~। सर्वमूर्तद्रव्यसंयोगित्वं विभुत्वं~। मूर्तत्वं परिच्छिन्नपरिमाणवत्वं, क्रियावत्वं वा~। विभुत्वादेव आत्मवन्नित्यमित्याह - नित्यं चेति~।\\[10pt]
	{\bfseries अतीतादिव्यवहारहेतुः कालः~। स चैको विभुर्नित्यश्च~॥}\par
		कालं लक्षयति - अतीतेति~। सर्वाधारः कालः सर्वकार्यनिमित्तकारणम्~।\\[10pt]
	\noindent
	{\bfseries प्राच्यादिव्यवहारहेतुर्दिक्~। सा चैका विभ्वी नित्या च~॥}\par
		दिशो लक्षणमाह - प्राचीति~। दिगपि कार्यमात्रे निमित्तकारणम्~।\\[10pt]
	{\bfseries ज्ञानाधिकरणमात्मा~। स द्विविधः जीवात्मा परमात्मा च~। तत्रेश्वरः सर्वज्ञः परमात्मैक एव~। जीवात्मा प्रतिशरीरं भिन्नो विभुर्नित्यश्च ~॥}\par
		आत्मनो लक्षणमाह - ज्ञानेति~। आत्मानं विभजते - स द्विविध इति~। परमात्मनो लक्षणमाह - तत्रेति~। नित्यज्ञानाधिकरणत्वमीश्वरत्वम्~। नन्वीश्वरसद्भावे किं प्रमाणम्~। न तावत्प्रत्यक्षम्~। तद्धि बाह्यमान्तरं वा~। नाद्यः अरूपिद्रव्यात्~। न द्वितीयः आत्मसुखादिव्यतिरिक्तत्वात्~। नाप्यनुमानं लिङ्गाभावादिति चेत् न~। अङ्कुरादिकं सकर्तृकं कार्यत्वाद्घटवत् इत्यनुमानस्यैव प्रमाणत्वात्~। उपादानगोचरापरोक्षज्ञानचिकीर्षाकृतिमत्वं कर्तृत्वम्~। उपादानं समवायिकारणम्~। सकलपरमाण्वादिसूक्ष्मदर्शित्वात्सर्वज्ञत्वम्~। {\bfseries यः सर्वज्ञः स सर्ववित्} इत्यागमोऽपि तत्र प्रमाणम्~। जीवस्य लक्षणमाह - जीव इति~। सुखाद्याश्रयत्वं जीवलक्षणम्~। ननु मनुष्योऽहं, ब्राह्मणोऽहमित्यादौ सर्वत्राहम्प्रत्यये शरीरस्यैव विषयत्वाच्छरीरमेवात्मेति चेत् न~। शरीरस्यात्मत्वे करपादादिनाशे शरीरनाशादात्मनोऽपि नाशप्रसङ्गात्~। नापीन्द्रियाणामात्मत्वम् , "योऽहं घटमद्राक्षं सोऽहमिदानीं स्पृशामि" इत्यनुसन्धानाभावप्रसङ्गात्~, अन्यानुभूतेऽर्थे अन्यस्यानुसन्धानायोगात्~। तस्माद्देहेन्द्रियव्यतिरिक्तो जीवः सुखदुःखादिवैचित्र्यात् प्रतिशरीरं भिन्नः~। स च न परमाणुः सर्वशरीरव्यापिसुखाद्यनुपलब्धिप्रसङ्गात्~। न मध्यमपरिमाणवान् तथा सति अनित्यत्वप्रसङ्गेन कृतहानाकृताभ्यागमप्रसङ्गात्~। तस्मान्नित्यो विभुर्जीवः~।\\[10pt]
	{\bfseries सुखाद्युपलब्धिसाधनमिन्द्रियं मनः~। तच्च प्रत्यात्मनियतत्वादनन्तं परमाणुरूपं नित्यं च~॥}\par
		मनसो लक्षणमाह - सुखेति~। स्पर्शरहितत्वे सति क्रियावत्वं मनसो लक्षणम्~। तद्विभजते तच्चेति~। एकैकस्यात्मनः एकैकं मन इत्यात्मनामनेकत्वान्मनसोऽप्यनेकत्वमित्यर्थः~। ननु मनो विभु, स्पर्शरहितत्वे सति द्रव्यत्वादाकाशादिवदिति चेत् न~। मनसो विभुत्वे आत्ममनःसंयोगस्याऽसमवायिकारणस्याभावाज्ज्ञानानुत्पत्तिप्रसङ्गः~। न च विभुद्वयसंयोगोऽस्त्विति वाच्यम्~। तत्संयोगस्य नित्यत्वेन सुषुप्त्यभावप्रसङ्गात्, पुरीतद्व्यतिरिक्तस्थले आत्ममनःसंयोगस्य सर्वदा विद्यमानत्वात्~। अणुत्वे तु यदा मनः पुरीतति नाड्यां प्रविशति तदा सुषुप्तिः, यदा निस्सरति तदा ज्ञानोत्पत्तिरित्यणुत्वसिद्धिः~। \section*{अथ गुणनिरूपणम्}
\addcontentsline{toc}{section}{४. गुणनिरूपणम्}
	{\bfseries चक्षुर्मात्रग्राह्यो गुणो रूपम्~। तच्च शुक्लनीलपीतरक्तहरितकपिशचित्रभेदात्सप्तविधम्~। पृथिवीजलतेजोवृत्ति~। तत्र पृथिव्यां सप्तविधम्~। अभास्वरशुक्लं जले~। भास्वरशुक्लं तेजसि~॥}\par
		रूपं लक्षयति - चक्षुरिति~। सङ्ख्यादावतिव्याप्तिवारणाय मात्रपदम्~। रूपत्वेऽतिव्याप्तिवारणाय गुणपदम्~। नन्वव्याप्यवृत्तिनीलादिसमुदाय एव चित्ररूपमिति चेत् न~। रूपस्य व्याप्यवृत्तित्वनियमात्~। ननु चित्रपटे अवयवरूपस्यैव प्रतीतिरिति चेत् न~। रूपरहितत्वेन पटस्याप्रत्यक्षत्वप्रसङ्गात्~। न च रूपवत्समवेतत्वं प्रत्यक्षत्वप्रयोजकं गौरवात्~। तस्मात्पटस्य प्रत्यक्षत्वान्यथानुपपत्या चित्ररूपसिद्धिः~। रूपस्याश्रयमाह - पृथिवीति~। आश्रयं विभज्य दर्शयति तत्रेति~।\\[10pt]
	{\bfseries रसनग्राह्यो गुणो रसः~। स च मधुराम्ललवणकटुकषायतिक्तभेदात् षड्विधः~। पृथिवीजलवृत्तिः~। तत्र पृथिव्यां षड्विधः~। जले मधुर एव~॥}\par
		रसं लक्षयति - रसनेति~। रसत्वेऽतिव्याप्तिवारणाय गुणपदम्~। रसस्याश्रयमाह पृथिवीति~। आश्रयं विभज्य दर्शयति - तत्रेति~।\\[10pt]
	{\bfseries घ्राणग्राह्यो गुणो गन्धः~। स द्विविधः सुरभिरसुरभिश्च~। पृथिवीमात्रवृत्तिः~॥}\par
		गन्धं लक्षयति - घ्राणेति~। गन्धत्वेऽतिव्याप्तिवारणाय गुणपदम्~।\\[10pt]
	{\bfseries त्वगिन्द्रियमात्रग्राह्यो गुणः स्पर्शः~। स त्रिविधः, शीतोष्णानुष्णाशीतभेदात्~। पृथिव्यप्तेजोवायुवृत्तिः~। तत्र शीतो जले~। उष्णस्तेजसि~। अनुष्णाशीतः पृथिवीवाय्वोः~॥}\par
		स्पर्शं लक्षयति - त्वगिति~। स्पर्शत्वेऽतिव्याप्तिवारणाय गुणपदम्~। संयोगादावतिव्याप्तिवारणाय मात्रपदम्~।\\[10pt]
	{\bfseries रूपादिचतुष्टयं पृथिव्यां पाकजमनित्यं च~। अन्यत्र अपाकजं नित्यमनित्यं च~। नित्यगतं नित्यम्~, अनित्यगतमनित्यम्~॥}\par
		पाकजमिति - पाकस्तेजःसंयोगः~। तेन पूर्वरूपं नश्यति रूपान्तरमुत्पद्यत इत्यर्थः~। तत्र परमाणुष्वेव पाको न द्व्यणुकादौ~। आमपाकनिक्षिप्ते घटे परमाणुषु रूपान्तरोत्पत्तौ श्यामघटनाशे पुनर्द्व्यणुकादिक्रमेण रक्तघटोत्पत्तिः~। तत्र परमाणवः समवायिकारणम्~। तेजः संयोगोऽसमवायिकारणम्~। अदृष्टादिकं निमित्तकारणम्~। द्व्यणुकादिरूपे कारणरूपमसमवायिकारणमिति पीलुपाकवादिनो वैशेषिकाः~। पूर्वघटस्य नाशं विनैव अवयविनि अवयवेषु च परमाणुपर्यन्तेषु युगपद्रूपान्तरोत्पत्तिरिति पिठरपाकवादिनो नैयायिकाः~। अत एव पार्थिवपरमाणुषु रूपादिकमनित्यमित्यर्थः~। अन्यत्र - जलादावित्यर्थः~। नित्यगतमिति - परमाणुगतमित्यर्थः~। अनित्यगतमिति - द्व्यणुकादिगतमित्यर्थः~। रूपादिचतुष्टयं उद्भूतं प्रत्यक्षम्~, अनुद्भूतमप्रत्यक्षम्~। उद्भूतत्वं प्रत्यक्षत्वप्रयोजको धर्मः, तदभावोऽनुद्भूतत्वम्~।\\[10pt]
	{\bfseries एकत्वादिव्यवहारहेतुः सङ्ख्या~। सा नवद्रव्यवृत्तिः एकत्वादिपरार्धपर्यन्ता~। एकत्वं नित्यमनित्यं च~। नित्यगतं नित्यम्~। अनित्यगतमनित्यम्~। द्वित्वादिकं तु सर्वत्राऽनित्यमेव~॥}\par
		सङ्ख्यां लक्षयति - एकेति~।\\[10pt]
	{\bfseries मानव्यवहारासाधारणकारणं परिमाणम्~। नवद्रव्यवृत्ति~। तच्चतुर्विधम्~। अणु महत् दीर्घं हृस्वं चेति~॥}\par
		परिमाणं लक्षयति - मानेति~। परिमाणं विभजते - तच्चेति~। भावप्रधानो निर्देशः~। अणुत्वं, महत्त्वं, दीर्घत्वं, ह्रस्वत्वं चेत्यर्थः~।\\[10pt]
	{\bfseries पृथग्व्यवहारासाधारणकारणं पृथक्त्वम्~। सर्वद्रव्यवृत्ति~॥}\par
		पृथक्त्वं लक्षयति - पृथगिति~। इदमस्मात्पृथक् इति व्यवहारकारणं पृथक्त्वमित्यर्थः~।\\[10pt]
	{\bfseries संयुक्तव्यवहारहेतुः संयोगः~। सर्वद्रव्यवृत्तिः~॥}\par
		संयोगं लक्षयति - संयुक्तेति~। इमौ संयुक्तौ इति व्यवहारहेतुरित्यर्थः~। सङ्ख्यादिलक्षणे सर्वत्र दिक्कालादावतिव्याप्तिवारणाय असाधारणेति विशेषणीयम्~। संयोगो द्विविधः, कर्मजः संयोजश्चेति~। आद्यो हस्तक्रियया हस्तपुस्तकसंयोगः~। द्वितीयो हस्तपुस्तकसंयोगात्कायपुस्तकसंयोगः~। अव्याप्यवृत्तिः संयोगः~। स्वात्यन्ताभावसमानाधिकरणत्वमव्याप्यवृत्तित्वम्~।\\[10pt]
	{\bfseries संयोगनाशको गुणो विभागः~। सर्वद्रव्यवृत्तिः~॥}\par
		विभागं लक्षयति - संयोगेति~। कालादावतिव्याप्तिवारणाय गुण इति~। रूपादावतिव्याप्तिवारणाय संयोगनाशक इति~। विभागोऽपि द्विविधः, कर्मजः विभागजश्चेति~। आद्यो हस्तपुस्तकविभागः~। द्वितीयो हस्तपुस्तकविभागात्कायपुस्तकविभागः~।\\[10pt]
	{\bfseries परापरव्यवहारासाधारणकारणे परत्वापरत्वे~। पृथिव्यादिचतुष्टयमनोवृत्तिनी~। ते द्विविधे दिक्कृते कालकृते च~। दूरस्थे दिक्कृतं परत्वम्~। समीपस्थे दिक्कृतमपरत्वम्~। ज्येष्ठे कालकृतं परत्वम्~। कनिष्ठे कालकृतमपरत्वम्~॥}\par
		परत्वापरत्वयोर्लक्षणमाह - परापरेति~। परव्यवहारासाधारणकारणं परत्वम्~। अपरव्यवहारासाधारणकारणमपरत्वमित्यर्थः~। ते विभजते - ते द्विविधे इति~। दिक्कृतयोरुदाहरणमाह - दूरस्थ इति~। कालकृते उदाहरति - ज्येष्ठ इति~।\\[10pt]
	{\bfseries आद्यपतनासमवायिकारणं गुरुत्वम्~। पृथिवीजलवृत्ति~॥}\par
		गुरुत्वं लक्षयति - आद्येति~। द्वितीयादिपतनस्य वेगासमवायिकारणत्वाद्वेगेऽतिव्याप्तिवारणाय आद्येति~।\\[10pt]
	{\bfseries आद्यस्यन्दनासमवायिकारणं द्रवत्वम्~। पृथिव्यप्तेजोवृत्ति~। तद्द्विविधं, सांसिद्धिकं नैमित्तिकं च~। सांसिद्धिकं जले~। नैमित्तिकं पृथिवीतेजसोः~। पृथिव्यां घृतादावग्निसंयोगजं द्रवत्वम्~। तेजसि सुवर्णादौ~॥}\par
		द्रवत्वं लक्षयति - आद्येति~। स्यन्दनं प्रस्रवणम्~। तेजःसयोगजं नैमित्तिकम्~। तद्भिन्नं सांसिद्धिकम्~। पृथिव्यां नैमित्तिकमुदाहरति - घृतादाविति~। तेजसि तदाह - सुवर्णादाविति~।\\[10pt]
	{\bfseries चूर्णादिपिण्डीभावहेतुर्गुणः स्नेहः~। जलमात्रवृत्तिः~॥}\par
		स्नेहं लक्षयति - चूर्णेति~। कालादावतिव्याप्तिवारणाय गुणपदम्~। रूपादावतिव्याप्तिवारणाय पिण्डीभावेति~।\\[10pt]
	{\bfseries श्रोत्रग्राह्यो गुणः शब्दः~। आकाशमात्रवृत्तिः~। स द्विविधः, ध्वन्यात्मकः वर्णात्मकश्च~। तत्र ध्वन्यात्मकः भेर्यादौ~। वर्णात्मकः संस्कृतभाषादिरूपः~॥}\par
		शब्दं लक्षयति - श्रोत्रेति~। शब्दत्वेऽतिव्याप्तिवारणाय गुणपदम्~। रूपादावतिव्याप्तिवारणाय श्रोत्रेति~। शब्दस्त्रिविधः - संयोगजः, विभागजः, शब्दजश्चेति~। तत्र आद्यो भेरीदण्डसंयोगजन्यः~। द्वितीयो वंशे पाट्यमाने दलद्वयविभागजन्यश्चटचटाशब्दः~। भेर्यादिदेशमारभ्य श्रोत्रदेशपर्यन्तं द्वितीयादिशब्दाः शब्दजाः~।\\[10pt]
	{\bfseries सर्वव्यवहारहेतुर्गुणो बुद्धिर्ज्ञानम्~। सा द्विविधा स्मृतिरनुभवश्च~॥}\par
		बुद्धेर्लक्षणमाह - सर्वव्यवहारेति~। कालादावतिव्याप्तिवारणाय गुण इति~। रूपादावतिव्याप्तिवारणाय सर्वव्यवहार इति~। जानामीत्यनुव्यवसायगम्यं ज्ञानमेव लक्षणमिति भावः~। बुद्धिं विभजते - सेति~।\\[10pt]
	{\bfseries संस्कारमात्रजन्यं ज्ञानं स्मृतिः~॥}\par
		स्मृतेर्लक्षणमाह - संस्कारेति~। भावनाख्यः संस्कारः~। संस्कारध्वंसेऽतिव्याप्तिवारणाय ज्ञानमिति~। घटादिप्रत्यक्षेऽतिव्याप्तिवारणाय संस्कारजन्यमिति~। प्रत्यभिज्ञायाम् अतिव्याप्तिवारणाय मात्रपदम्~।\\[10pt]
	{\bfseries तद्भिन्नं ज्ञानमनुभवः~। स द्विविधः, यथार्थोऽयथार्थश्च~॥}\par
		अनुभवं लक्षयति तद्भिन्नमिति~। स्मृतिभिन्नं ज्ञानमनुभव इत्यर्थः~। अनुभवं विभजते स द्विविध इति~।\\[10pt]
	{\bfseries तद्वति तप्रकारकोऽनुभवो यथार्थः~। यथा रजते इदं रजतमिति ज्ञानम्~। सैव प्रमेच्युते~॥}\par
		यथार्थानुभवस्य लक्षणमाह तद्वतीति~। ननु घटे घटत्वमिति प्रमायामव्याप्तिः, घटत्वे घटाभावादिति चेत् न, यत्र यत्सम्बन्धोऽस्ति तत्र तत्सम्बन्धानुभवः इत्यर्थाद्घटत्वे घटसम्बन्धोऽस्तीति नाव्याप्तिः~। सैवेति - यथार्थानुभव एव शास्त्रे प्रमेत्युच्यते इत्यर्थः~।\\[10pt]
	{\bfseries तदभाववति तत्प्रकारकोऽनुभवोऽयथार्थः~। यथा शुक्ताविदं रजतमिति~। सैव अप्रमा इत्युच्यते~॥}\par
		अयथार्थानुभवं लक्षयति तदभाववतीति~। नन्विदं संयोगीति प्रमायामतिव्याप्तिरिति चेत् न~। यदवच्छेदेन यत्सम्बन्धाभावस्तदवच्छेदेन तत्सम्बन्धज्ञानस्य विवक्षितत्वात्~। संयोगाभावावच्छेदेन संयोगज्ञानस्य भ्रमत्वात्संयोगावच्छेदेन संयोगसम्बन्धस्य सत्त्वान्नातिव्याप्तिः~।\\[10pt]
	{\bfseries यथार्थानुभवः चतुर्विधः, प्रत्यक्षानुमित्युपमितिशाब्दभेदात्~॥}\par
		यथार्थानुभवं विभजते - यथार्थेति~।\\[10pt]
	{\bfseries तत् करणमपि चतुर्विधं प्रत्यक्षानुमानोपमानशब्दभेदात्~॥}\par
		प्रसङ्गात्प्रमाकरणं विभजते - तत्करणमपीति~। प्रमाकरणमित्यर्थः~। प्रमाकरणं प्रमाणमिति प्रमाणसामान्यलक्षणम्~।\\[10pt]
	{\bfseries व्यापारवदसाधारणं कारणं करणम्~॥}\par
		करणलक्षणमाह - असाधारणेति~। दिक्कालादावतिव्याप्तिवारणाय असाधारणेति~।\\[10pt]
	{\bfseries कार्यनियतपूर्ववृत्ति कारणम्~॥}\par
		कारणलक्षणमाह - कार्येति~। पूर्ववृत्तिकारणमित्युक्ते रासभादावतिव्याप्तिः स्यादतो नियतेति~। तावन्मात्रे कृते कार्येऽतिव्याप्तिरतः पूर्ववृत्तीति~। ननु तन्तुरूपमपि पटं प्रति कारणं स्यादिति चेत् न, अनन्यथासिद्धत्वे सतीति विशेषणात्~। अनन्यथासिद्धत्वमन्यथासिद्धिरहितत्वम्~। अन्यथासिद्धिः त्रिविधा - येन सहैव यस्य यं प्रति पूर्ववृत्तित्वमवगम्यते तं प्रति तेन तदन्यथासिद्धम्~। यथा तन्तुना तन्तुरूपं तन्तुत्वं च पटं प्रति~। अन्यं प्रति पूर्ववृत्तित्वे ज्ञात एव यस्य यं प्रति पूर्ववृत्तित्वमवगम्यते तं प्रति तदन्यथासिद्धम्~। यथा शब्दं प्रति पूर्ववृत्तित्वे ज्ञात एव पटं प्रत्याकाशस्य~। अन्यत्र क्लृप्तनियतपूर्ववतिर्न एव कार्यसम्भवे तत्सहभूतमन्यथासिद्धम्~। यथा पाकजस्थले गन्धं प्रति रूपप्रागभावस्य~। एवञ्च अनन्यथासिद्धनियतपूर्ववृत्तित्वं कारणत्वम्~।\\[10pt]
	{\bfseries कार्यं प्रागभावप्रतियोगि~॥}\par
		कार्यलक्षणमाह - कार्यमिति~।\\[10pt]
	{\bfseries कारणं त्रिविधम् - समावाय्यसमवायिनिमित्तभेदात्~॥}\par
		कारणं विभजते कारणमिति~।\\[10pt]
	{\bfseries यत् समवेतं कार्यमुत्पद्यते तत् समवायिकारणम्~। यथा तन्तवः पटस्य, पटश्च स्वगत\-रूपादेः~॥}\par
		समवायिकारणस्य लक्षणमाह - यत्समवेतमिति~। यस्मिन् समवेतमित्यर्थः~। असमवायिकारणं लक्षयति - कार्येणेति~। कार्येणेत्येतदुदाहरति तन्तुसंयोग इति~।\\[10pt]
	{\bfseries कार्येण कारणेन वा सह एकस्मिन्नर्थे समवेतत्वे सति यत्कारणं तदसमवायिकारणम्~। यथा तन्तुसंयोगः पटस्य, तन्तुरूपं पटरूपस्य~॥}\par
		कार्येण पटेन सह एकस्मिंस्तन्तौ समवेतत्वात्तन्तुसंयोगः पटस्यासमवायिकारणमित्यर्थः~। कारणेनेत्येतदुदाहरति - तन्तुरूपमिति~। कारणेन पटेन सह एकस्मिंस्तन्तौ समवेतत्वात्तन्तुरूपं पटरूपस्यासमवायिकारणमित्यर्थः~।\\[10pt]
	{\bfseries तदुभयभिन्नं कारणं निमित्तकारणम्~। यथा तुरीवेमादिकं पटस्य~॥}\par
		निमित्तकारणं लक्षयति - तदुभयेति~। समवाय्यसमवायिभिन्नकारणं निमित्तकारणमित्यर्थः~।\\[10pt]
	{\bfseries तदेतत्त्रिविधकारणमध्ये यदसाधारणं कारणं तदेव करणम् ~॥}\par
		करणलक्षणमुपसंहरति - तदेतदिति~। \subsection*{अथ प्रत्यक्षपरिच्छेदः}
\addcontentsline{toc}{subsection}{४.१. प्रत्यक्षपरिच्छेदः}
	{\bfseries तत्र प्रत्यक्षज्ञानकरणं प्रत्यक्षम्~॥}\par
		प्रत्यक्षलक्षणमाह - तत्रेति~। प्रमाणचतुष्टयमध्ये इत्यर्थः~।\\[10pt]
	{\bfseries इन्द्रियार्थसन्निकर्षजन्यं ज्ञानं प्रत्यक्षम्~। तद्द्विविधं निर्विकल्पकं सविकल्पकं चेति~। तत्र निष्प्रकारकं ज्ञानं निर्विकल्पकम्~। सप्रकारकं ज्ञानं सविकल्पकं यथा डित्थोऽयं, ब्राह्मणोऽयं, श्यामोऽयं, पाचकोऽयमिति ~॥}\par
		प्रत्यक्षज्ञानस्य लक्षणमाह - इन्द्रियेति~। इन्द्रियं चक्षुरादिकम्, अर्थो घटादिः, तयोः सन्निकर्षः संयोगादिः, तज्जन्यं ज्ञानमित्यर्थः~। तद्विभजते - तद्द्विविधमिति~। निर्विकल्पकस्य लक्षणमाह - निष्प्रकारकमिति~। विशेषणविशेष्यसम्बन्धानवगाहिज्ञानमित्यर्थः~। ननु निर्विकल्पके किं प्रमाणमिति चेत् न, गौरिति विशिष्टज्ञानं विशेषणज्ञानजन्यं, विशिष्टज्ञानत्वात्, दण्डीति ज्ञानवत् इत्यनुमानस्य प्रमाणत्वात्~। विशेषणज्ञानस्यापि सविकल्पकत्वेऽनवस्थाप्रसङ्गान्निर्विकल्पकसिद्धिः~। सविकल्पकं लक्षयति - सप्रकारकमिति~। नामजात्यादिविशेष्यविशेषणसम्बन्धावगाहिज्ञानमित्यर्थः~। सविकल्पकमुदाहरति - यथेति~।\\[10pt]
	{\bfseries प्रत्यक्षज्ञानहेतुरिन्द्रियार्थसन्निकर्षः षड्विधः~। संयोगः, संयुक्तसमवायः, संयुक्तसमवेतसमवायः, समवायः, समवेतसमवायः, विशेषणविशेष्यभावश्चेति~॥}\par
		इन्द्रियार्थसंनिकर्षं विभजते - प्रत्यक्षेति~।\\[10pt]
	{\bfseries चक्षुषा घटप्रत्यक्षजनने संयोगः सन्निकर्षः~॥}\par
		संयोगसन्निकर्षमुदाहरति - चक्षुषेति~। द्रव्यप्रत्यक्षे सर्वत्र संयोगः संनिकर्षः~। आत्मा मनसा संयुज्यते, मन इन्द्रियेण, इन्द्रियमर्थेन, ततः प्रत्यक्षज्ञानमुत्पद्यते इत्यर्थः~।\\[10pt]
	{\bfseries घटरूपप्रत्यक्षजनने संयुक्तसमवायः संनिकर्षः~। चक्षुःसंयुक्ते घटे रूपस्य समवायः~॥}\par
		संयुक्तसमवायमुदाहरति - घटरूपेति~। तत्र युक्तिमाह - चक्षुःसंयुक्त इति~।\\[10pt]
	{\bfseries रूपत्वसामान्यप्रत्यक्षे संयुक्तसमवेतसमवायः सन्निकर्षः~। चक्षुःसंयुक्ते घटे रूपं समवेतं, तत्र रूपत्वस्य समवायात्~॥}\par
		संयुक्तसमवेतसमवायमुदाहरति - रूपत्वेति~।\\[10pt]
	{\bfseries श्रोत्रेण शब्दसाक्षात्कारे समवायः सन्निकर्षः~। कर्णविवरवर्त्याकाशस्य श्रोत्रत्वात् , शब्दस्याकाशगुणत्वात् , गुणगुणिनोश्च समवायात्~॥}\par
		समवायमुदाहरति - श्रोत्रेणेति~। तदुपपादयति - कर्णेति~। ननु दूरस्थशब्दस्य कथं श्रोत्रसम्बन्ध इति चेत् न, वीचीतरङ्गन्यायेन कदम्बमुकुलन्यायेन वा शब्दान्तरोत्पत्तिक्रमेण श्रोत्रदेशे जातस्य श्रोत्रेण सम्बन्धात्प्रत्यक्षसम्भवः~।\\[10pt]
	{\bfseries शब्दत्वसाक्षात्कारे समवेतसमवायः सन्निकर्षः~। श्रोत्रसमवेते शब्दे शब्दत्वस्य समवायात्~॥}\par
		समवेतसमवायमुदाहरति - शब्दत्वेति~।\\[10pt]
	{\bfseries अभावप्रत्यक्षे विशेषणविशेष्यभावः सन्निकर्षः~। घटाभाववद्भूतलमित्यत्र चक्षुःसंयुक्ते भूतले घटाभावस्य विशेषणत्वात्~॥}\par
		विशेषणविशेष्यभावमुदाहरति - अभावेति~। तदुपपादयति घटाभाववदिति~। भूतले घटो नास्तीत्यत्र अभावस्य विशेष्यत्वं द्रष्टव्यम्~। एतेन अनुपलब्धेः प्रमाणान्तरत्वं निरस्तम्~। यद्यत्र घटोऽभविष्यत्तर्हि भूतलमिवाद्रक्ष्यत, दर्शनाभावान्नास्तीति तर्कितप्रतियोगिसत्वविरोध्यनुपलब्धिसहकृतेन्द्रियेणैव अभावज्ञानोत्पत्तौ अनुपलब्धेः प्रमाणान्तरत्वासम्भवात्~। अधिकरणज्ञानार्थमपेक्षणीयेन्द्रियस्यैव करणत्वोपपत्तावनुपलब्धेः करणत्वायोगात्~। विशेषणविशेष्यभावो विशेषणविशेष्यस्वरूपमेव, नातिरिक्तः सम्बन्धः~।\\[10pt]
	{\bfseries एवं सन्निकर्षषट्कजन्यं ज्ञानं प्रत्यक्षम्~। तत्करणमिन्द्रियम्~। तस्मादिन्द्रियं प्रत्यक्षप्रमाणमिति सिद्धम्~॥}\par
		प्रत्यक्षज्ञानमुपसंहरंस्तस्य करणमाह - एवमिति~। असाधारणकारणत्वादिन्द्रियं प्रत्यक्षज्ञानकरणमित्यर्थः~। प्रत्यक्षप्रमाणमुपसंहरति तस्मादिति~। \subsection*{अथानुमानपरिच्छेदः}
\addcontentsline{toc}{subsection}{४.२. अनुमानपरिच्छेदः}
	{\bfseries अनुमितिकरणमनुमानम्~॥}\par
		अनुमानं लक्षयति - अनुमितिकरणमिति~।\\[10pt]
	{\bfseries परामर्शजन्यं ज्ञानमनुमितिः~॥}\par
		अनुमितिं लक्षयति - परामर्शेति~। ननु संशयोत्तरप्रत्यक्षेऽतिव्याप्तिः, स्थाणुपुरुषसंशयानन्तरं पुरुषत्वव्याप्यकरादिमानयम् इति परामर्शे सति, पुरुष एव इति प्रत्यक्षजननात्~। न च तत्रानुमितिरेवेति वाच्यम्~। पुरुषं साक्षात्करोमि इत्यनुव्यवसायविरोधादिति चेत् न~। पक्षतासहकृतपरामर्शजन्यत्वस्य विवक्षितत्वात्~। सिषाधयिषाविरहविशिष्टसिद्ध्यभावः पक्षता~। साध्यसिद्धिरनुमितिप्रतिबन्धिका~। सिद्धिसत्त्वेऽपि अनुमिनुयाम् इतीच्छायामनुमितिदर्शनात् सिषाधयिषोत्तेजिका~। ततश्चोत्तेजकाभावविशिष्टमण्यभावस्य दाहकारणत्ववत्सिषाधयिषाविरहविशिष्टसिद्ध्यभावस्याप्यनुमितिकारणत्वम्~।\\[10pt]
	{\bfseries व्याप्तिविशिष्टपक्षधर्मताज्ञानं परामर्शः~। यथा वह्निव्याप्यधूमवानयं पर्वत इति ज्ञानं परामर्शः~। तज्जन्यं पर्वतो वह्निमानिति ज्ञानमनुमितिः~॥}\par
		परामर्शं लक्षयति - व्याप्तीति~। व्याप्तिविषयकं यत्पक्षधर्मताज्ञानं स परामर्श इत्यर्थः~। परामर्शमभिनीय दर्शयति - यथेति~। अनुमितिमभिनीय दर्शयति - तज्जन्यमिति~। परामर्शजन्यमित्यर्थः~। व्याप्तिलक्षणमाह - यत्रेति~।\\[10pt]
	{\bfseries यत्र यत्र धूमस्तत्र तत्राग्निरिति साहचर्यनियमो व्याप्तिः~॥}\par
		यत्र धूमस्तत्राग्निरिति व्याप्तेरभिनयः~। साहचर्यनियमः इति लक्षणम्~। साहचर्यं सामानाधिकरण्यं, तस्य नियमः~। हेतुसमानाधिकरणात्यन्ताभावाप्रतियोगिसाध्यसामानाधिकरण्यं व्याप्तिरित्यर्थः~।\\[10pt]
	\noindent
	{\bfseries व्याप्यस्य पर्वतादिवृत्तित्वं पक्षधर्मता~॥}\par
		पक्षधर्मतास्वरूपमाह व्याप्यस्येति~।\\[10pt]
	{\bfseries अनुमानं द्विविधं - स्वार्थं परार्थं च~॥}\par
		अनुमानं विभजते अनुमानमिति~।\\[10pt]
	{\bfseries तत्र स्वार्थं स्वानुमितिहेतुः~। तथाहि - स्वयमेव भूयोदर्शनेन यत्र यत्र धूमस्तत्र तत्राग्निरिति महानसादौ व्याप्तिं गृहीत्वा पर्वतसमीपं गतः, तद्गते चाग्नौ सन्दिहानः पर्वते धूमं पश्यन् व्याप्तिं स्मरति - यत्र यत्र धूमस्तत्र तत्राग्निरिति~। तदनन्तरं वह्निव्याप्यधूमवानयं पर्वत इति ज्ञानमुत्पद्यते~। अयमेव लिङ्गपरामर्शः इत्युच्यते~। तस्माद् पर्वतो वह्निमानिति ज्ञानमनुमितिः उत्पद्यते~। तदेतत् स्वार्थानुमानम्~॥}\par
		स्वार्थानुमानं दर्शयति - स्वयमेवेति~। ननु पार्थिवत्वलोहलेख्यत्वादौ शतशः सहचारदर्शनेऽपि मण्यादौ व्यभिचारोपलब्धेर्भूयोदर्शेनेन कथं व्याप्तिग्रह इति चेत् न~। व्यभिचारज्ञानविरहसहकृतसहचारज्ञानस्य व्याप्तिग्राहकत्वात्~। व्यभिचारज्ञानं निश्चयः शङ्का च~। तद्विरहः क्वचित्तर्कात्, क्वचित्स्वतः सिद्धः एव~। धूमाग्न्योर्व्याप्तिग्रहे कार्यकारणभावभङ्गप्रसङ्गलक्षणस्तर्को व्यभिचारशङ्कानिवर्तकः~। ननु सकलवह्निधूमयोरसन्निकर्षात्कथं व्याप्तिग्रह इति चेत् न~। वह्नित्वधूमत्वरूपसामान्यप्रत्यासत्त्या सकलवह्निधूमज्ञानसम्भवात्~। तस्मादिति - लिङ्गपरामर्शादित्यर्थः~।\\[10pt]
	{\bfseries यत्तु स्वयं धूमादग्निमनुमाय परं प्रति बोधयितुं पञ्चावयववाक्यं प्रयुज्यते, तत् परार्थानुमानम्~। यथा पर्वतो वह्निमान्~, धूमवत्वात्~, यो यो धूमवान् स स वह्निमान् यथा महानसम्~। तथा चायम्~। तस्मात्तथेति~। अनेन प्रतिपादितात् लिङ्गात् परोऽप्यग्निं प्रतिपद्यते ~॥}\par
		परार्थानुमानमाह - यत्त्विति~। यच्छब्दस्य ’तत्परार्थानुमानम्’ इति तच्छब्देनान्वयः~। पञ्चावयववाक्यमुदाहरति यथेति~।\\[10pt]
	{\bfseries प्रतिज्ञा-हेतु-उदाहरण-उपनय-निगमनानि पञ्चावयवाः~। पर्वतो वह्निमानिति प्रतिज्ञा~। धूमवत्वात् इति हेतुः~। यो यो धूमवान् स स वह्निमान् ,यथा महानसम् इत्युदाहरणम्~। तथा च अयमिति उपनयः~। तस्मात्तथेति निगमनम्~॥}\par
		अवयवस्वरूपमाह - प्रतिज्ञेति~। उदाहृतवाक्ये प्रतिज्ञादिविभागमाह - पर्वतो वह्निमानिति~॥ साध्यवत्तया पक्षवचनं प्रतिज्ञा~। पञ्चम्यन्तं लिङ्गप्रतिपादकं हेतुः~। व्याप्तिप्रतिपादकं उदाहरणम्~। व्याप्तिविशिष्टलिङ्गप्रतिपादकं वचनमुपनयः~। हेतुसाध्यवत्तया पक्षप्रतिपादकं वचनं निगमनम्~। पक्षज्ञानं प्रतिज्ञाप्रयोजनम्~, लिङ्गज्ञानं हेतुप्रयोजनम्~, व्याप्तिज्ञानमुदाहरणप्रयोजनम्~, पक्षधर्मज्ञानमुपनयप्रयोजनम्~, अबाधितत्वादिकं निगमनप्रयोजनम्~।\\[10pt]
	{\bfseries स्वार्थानुमितिपरार्थानुमित्योः लिङ्गपरामर्श एव करणम्~। तस्मात् लिङ्गपरामर्शोऽनुमानम् ~॥}\par
		अनुमितिकरणमाह - स्वार्थेति~। ननु व्याप्तिस्मृतिपक्षधर्मताज्ञानाभ्यामेव अनुमितिसम्भवे विशिष्टपरामर्शः किमर्थमङ्गीकर्तव्यः इति चेत् न, ’वह्निव्याप्यधूमवानयम्’ इति शाब्दपरामर्शस्थले परामर्शस्यावश्यकतया लाघवेन सर्वत्र परामर्शस्यैव कारणत्वात्~। लिङ्गं न कारणम्~, अतीतादौ व्यभिचारात्~। ’व्यापारवत्कारणं करणम्’ इति मते परामर्शद्वारा व्याप्तिज्ञानं करणम्~। तज्जन्यत्वे सति तज्जन्यजनको व्यापारः~। अनुमानमुपसंहरति - तस्मादिति~।\\[10pt]
	{\bfseries लिङ्गं त्रिविधम्~। अन्वयव्यतिरेकि, केवलान्वयि, केवलव्यतिरेकि चेति~। अन्वयेन व्यतिरेकेण च व्याप्तिमदन्वयव्यतिरेकि~। यथा वह्नौ साध्ये धूमवत्वम्~। यत्र धूमस्तत्राग्निर्यथा महानसम् इत्यन्वयव्याप्तिः~। यत्र वह्निर्नास्ति तत्र धूमोऽपि नास्ति, यथा महाहृद इति व्यतिरेकव्याप्तिः ~॥}\par
		लिङ्गं विभजते लिङ्गमिति~। अन्वयव्यतिरेकिणं लक्षयति अन्वयेति~। हेतुसाध्ययोर्व्याप्तिरन्वयव्याप्तिः~।\\[10pt]
	{\bfseries अन्वयमात्रव्याप्तिकं केवलान्वयि~। यथा घटोऽभिधेयः प्रमेयत्वात् पटवत्~। अत्र प्रमेयत्वाभिधेयत्वयोः व्यतिरेकव्याप्तिर्नास्ति, सर्वस्यापि प्रमेयत्वात् अभिधेयत्वाच्च ~॥}\par
		केवलान्वयिनो लक्षणमाह - अन्वयेति~। केवलान्वयिसाध्यकं लिङ्गं केवलान्वयि~। अत्यन्ताभावाप्रतियोगित्वं केवलान्वयित्वम्~। ईश्वरप्रमाविषयत्वं सर्वपदाभिधेयत्वं च सर्वत्रास्तीति व्यतिरेकाभावः~।\\[10pt]
	{\bfseries व्यतिरेकमात्रव्याप्तिकं केवलव्यतिरेकि, यथा पृथिवीतरेभ्यो भिद्यते गन्धवत्वात्~। यदितरेभ्यो न भिद्यते न तद्गन्धवत् यथा जलम्~। न चेयं तथा~। तस्मान्न तथेति~। अत्र यद्गन्धवत् तदितरभिन्नम् इत्यन्वयदृष्टान्तो नास्ति, पृथिवीमात्रस्य पक्षत्वात् ~॥}\par
		केवलव्यतिरेकिणो लक्षणमाह - व्यतिरेकेति~। तदुदाहरति - यथेति~। नन्वितरभेदः प्रसिद्धो न वा~। आद्ये यत्र प्रसिद्धस्तत्र हेतुसत्त्वे अन्वयित्वम्~, असत्वे असाधारण्यम्~। द्वितीये साध्यज्ञानाभावात्कथं तद्विशिष्टानुमितिः~। विशेषणज्ञानाभावे विशिष्टज्ञानानुदयात्प्रतियोगिज्ञानाभावाद्व्यतिरेकव्याप्तिज्ञानमपि न स्यादिति चेत् न~। जलादित्रयोदशान्योन्याभावानां त्रयोदशसु प्रत्येकं प्रसिद्धानां मेलनं पृथिव्यां साध्यते~। तत्र त्रयोदशत्वावच्छिन्नभेदात्मकसाध्यस्यैकाधिकरणवृत्तित्वाभावान्नान्वयित्वासाधारण्ये~। प्रत्येकाधिकरणप्रसिद्ध्या साध्यविशिष्टानुमितिः व्यतिरेकव्याप्तिनिरूपणं चेति~।\\[10pt]
	{\bfseries सन्दिग्धसाध्यवान् पक्षः~। यथा धूमवत्त्वे हेतौ पर्वतः~॥}\par
		पक्षलक्षणमाह - सन्दिग्धेति~। ननु श्रवणानन्तरभाविमननस्थले अव्याप्तिः~। तत्र वेदवाक्यैरात्मनो निश्चितत्वेन सन्देहाभावात्~। किञ्च प्रत्यक्षेऽपि वह्नौ यत्रेच्छयानुमितिस्तत्राव्याप्तिरिति चेत् न, उक्तपक्षताश्रयत्वस्य पक्षलक्षणत्वात्~।\\[10pt]
	{\bfseries निश्चितसाध्यवान् सपक्षः~। यथा तत्रैव महानसम्~॥}\par
		सपक्षलक्षणमाह - निश्चितेति~।\\[10pt]
	{\bfseries निश्चितसाध्याऽभाववान् विपक्षः~। यथा तत्रैव महाह्रदः~॥}\par
		विपक्षलक्षणमाह - निश्चितेति~।\\[10pt]
	{\bfseries सव्यभिचारविरुद्धसत्प्रतिपक्षासिद्धबाधिताः पञ्च हेत्वाभासाः~॥}\par
		एवं सद्धेतून्निरूप्य असद्धेतून्निरूपयितुं विभजते सव्यभिचारेति~। अनुमितिप्रतिबन्धकयथार्थज्ञानविषयत्वं हेत्वाभासत्वम्~।\\[10pt]
	{\bfseries सव्यभिचारोऽनैकान्तिकः~। स त्रिविधः साधारणासाधारणानुपसंहारिभेदात्~॥}\par
		सव्यभिचारं विभजते - स त्रिविध इति~।\\[10pt]
	{\bfseries तत्र साध्याभाववद्वृत्तिः साधारणोऽनैकान्तिकः, यथा पर्वतो वह्निमान् प्रमेयत्वात् इति~। प्रमेयत्वस्य वह्न्यभाववति हृदे विद्यमानत्वात्~॥}\par
		साधारणं लक्षयति - तत्रेति~। उदाहरति - यथेति~।\\[10pt]
	{\bfseries सर्वसपक्षविपक्षव्यावृत्तः पक्षमात्रवृत्तिः असाधारणः~। यथा शब्दो नित्यः शब्दत्वात् इति~। शब्दत्वं हि सर्वेभ्यो नित्येभ्योऽनित्येभ्यश्च व्यावृत्तं शब्दमात्रवृत्ति~॥}\par
		असाधारणं लक्षयति - सर्वेति~।\\[10pt]
	{\bfseries अन्वयव्यतिरेकदृष्टान्तरहितोऽनुपसंहारी~। यथा सर्वमनित्यं प्रमेयत्वादिति~। अत्र सर्वस्यापि पक्षत्वात् दृष्टान्तो नास्ति~॥}\par
		अनुपसंहारिणो लक्षणमाह - अन्वयेति~।\\[10pt]
	{\bfseries साध्याभावव्याप्तो हेतुर्विरुद्धः~। यथा शब्दो नित्यः कृतकत्वादिति~। कृतकत्वं हि नित्यत्वाभावेनाऽनित्यत्वेन व्याप्तम्~॥}\par
		विरुद्धं लक्षयति - साध्येति~।\\[10pt]
	{\bfseries यस्य साध्याभावसाधकं हेत्वन्तरं विद्यते स सत्प्रतिपक्षः~। यथा शब्दो नित्यः श्रावणत्वात् शब्दत्ववत्~। शब्दोऽनित्यः कार्यत्वात् घटवत्~॥}\par
		सत्प्रतिपक्षं लक्षयति - यस्येति~।\\[10pt]
	{\bfseries असिद्धस्त्रिविधः - आश्रयासिद्धः, स्वरूपासिद्धो, व्याप्यत्वासिद्धश्चेति~॥}\par
		असिद्धं विभजते - असिद्ध इति~।\\[10pt]
	{\bfseries आश्रयासिद्धो यथा गगनारविन्दं सुरभि, अरविन्दत्वात् , सरोजारविन्दवत्~। अत्र गगनारविन्दमाश्रयः~। स च नास्त्येव ~॥}\par
		आश्रयासिद्धमुदाहरति - गगनेति~।\\[10pt]
	{\bfseries स्वरूपासिद्धो यथा शब्दो गुणश्चाक्षुषत्वात्~। अत्र चाक्षुषत्वं शब्दे नास्ति शब्दस्य श्रावणत्वात्~॥}\par
		स्वरूपासिद्धमुदाहरति यथेति~।\\[10pt]
	{\bfseries सोपाधिको हेतुः व्याप्यत्वासिद्धः~। साध्यव्यापकत्वे सति साधनाव्यापकत्वम् उपाधिः~। साध्यसमानाधिकरणान्त्यन्ताभावाप्रतियोगित्वं साध्यव्यापकत्वम्~। साधनवन्निष्ठात्यन्ताभावप्रतियोगित्वं साधनाव्यापकत्वम्~। पर्वतो धूमवान्वह्निमत्वादित्यत्र आर्द्रेन्धनसंयोगः उपाधिः~। तथाहि - यत्र धूमस्तत्रार्द्रेन्धनसंयोग इति साध्यव्यापकता~। यत्र वह्निस्तत्रार्द्रेन्धनसंयोगो नास्ति, अयोगोलके आर्द्रेन्धनसंयोगाभावादिति साधनाव्यापकता~। एवं साध्यव्यापकत्वे सति साधनाव्यापकत्वादार्द्रेन्धनसंयोगः उपाधिः~। सोपाधिकत्वात् वह्निमत्वं व्याप्यत्वासिद्धम्~॥}\par
		व्याप्यत्वासिद्धस्य लक्षणमाह - सोपाधिक इति~। उपाधेर्लक्षणमाह - साध्येति~। उपाधिश्चतुर्विधः - केवलसाध्यव्यापकः, पक्षधर्मावच्छिन्नसाध्यव्यापकः, साधनावच्छिन्नसाध्यव्यापकः, उदासीनधर्मावच्छिन्नसाध्यव्यापकश्चेति~। आद्यः आर्द्रेन्धनसंयोगः~। द्वितीयो यथा वायुः प्रत्यक्षः प्रत्यक्षस्पर्शाश्रयत्वात् इत्यत्र बहिर्द्रव्यत्वावच्छिन्नप्रत्यक्षत्वव्यापकमुद्भूतरूपवत्वम्~। तृतीयो यथा - प्रागभावो विनाशी जन्यत्वादित्यत्र जन्यत्वावच्छिन्नानित्यत्वव्यापकं भावत्वम्~। चतुर्थो यथा - प्रागभावो विनाशी प्रमेयत्वात् इत्यत्र जन्यत्वावच्छिन्नानित्यत्वव्यापकं भावत्वम्~।\\[10pt]
	{\bfseries यस्य साध्याभावः प्रमाणान्तरेण निश्चितः स बाधितः~। यथा वह्निरनुष्णो द्रव्यत्वात् जलवत्~। अत्रानुष्णत्वं साध्यं तदभाव - उष्णत्वं स्पार्शनप्रत्यक्षेण गृह्यत इति बाधितत्वम्~॥}\par
		बाधितस्य लक्षणमाह - यस्येति~। अत्र बाधस्य ग्राह्याभावनिश्चयत्वेन, सत्प्रतिपक्षस्य विरोधिज्ञानसामग्रीत्वेन साक्षादनुमितिप्रतिबन्धकत्वम्~। इतरेषां तु परामर्शप्रतिबन्धकत्वम्~। तत्रापि साधारणस्याव्यभिचाराभाववत्तया, विरुद्धस्य सामानाधिकरण्याभाववत्तया, व्याप्यत्वासिद्धस्य विशिष्टव्याप्त्यभाववत्तया, असाधारणानुपसंहारिणोः व्याप्तिसंशयोपधायकत्वेन व्याप्तिज्ञानप्रतिबन्धकत्वम्~। आश्रयासिद्धस्वरूपासिद्धयोः पक्षधर्मताज्ञानप्रतिबन्धकत्वम्~। उपाधिस्तु व्यभिचारज्ञानद्वारा व्याप्तिज्ञानप्रतिबन्धकः~। सिद्धसाधनं तु पक्षताविघटकतया आश्रयासिद्धावन्तर्भवतीति प्राञ्चः~। निग्रहस्थानान्तरमिति नवीनाः~। \subsection*{अथ उपमानपरिच्छेदः}
\addcontentsline{toc}{subsection}{४.३. उपमानपरिच्छेदः}
	{\bfseries उपमितिकरणमुपमानम्~। संज्ञासंज्ञिसम्बन्धज्ञानमुपमितिः~। तत्करणं सादृश्यज्ञानम्~। अतिदेशवाक्यार्थस्मरणमवान्तरव्यापारः~। तथा हि कश्चिद्गवयशब्दार्थमजानन् कुतश्चित् आरण्यकपुरुषाद्गोसदृशो गवय इति श्रुत्वा वनं गतो वाक्यार्थं स्मरन् गोसदृशं पिण्डं पश्यति~। तदनन्तरमसौ गवयशब्दवाच्य इत्युपमितिरुत्पद्यते~॥}\par
		उपमानं लक्षयति - उपमितिकरणमिति~। \subsection*{अथ शब्दपरिच्छेदः}
\addcontentsline{toc}{subsection}{४.४. शब्दपरिच्छेदः}
	{\bfseries आप्तवाक्यं शब्दः~। आप्तस्तु, यथार्थवक्ता~। वाक्यं पदसमूहः~। यथा गामानयेति~। शक्तं पदम्~। अस्मात्पदात् अयमर्थो बोद्धव्य इतीश्वरसङ्केतः शक्तिः~॥}\par
		शब्दं लक्षयति - आप्तेति~। वाक्यलक्षणमाह - वाक्यमिति~। पदलक्षणमाह - शक्तमिति~। अर्थस्मृत्यनुकूलपदपदार्थसम्बन्धः शक्तिः~। सा च पदार्थान्तरमिति मीमांसकाः~। तन्निरासार्थमाह - अस्मादिति~। डित्थादीनामिव घटादीनामपि सङ्केत एव शक्तिः, न तु पदार्थान्तरमित्यर्थः~। ननु गवादिपदानां जातावेव शक्तिः, विशेषणतया जातेः प्रथममुपस्थितत्वात्~। व्यक्तिलाभस्तु आक्षेपादिति केचित्~। तत् न, गामानयेत्यादौ वृद्धव्यवहारेण सर्वत्रानयनादेर्व्यक्तावेव सम्भवेन, जातिविशिष्टव्यक्तावेव शक्तिकल्पनात्~। शक्तिग्रहश्च वृद्धव्यवहारेण~। व्युत्पित्सुर्बालो ’गामानय’ इत्युत्तमवृद्धवाक्यश्रवणानन्तरं मध्यमवृद्धस्य प्रवृत्तिमुपलभ्य, गवानयनं च दृष्ट्वा, मध्यमवृद्धप्रवृत्तिजनकज्ञानस्यान्वयव्यतिरेकाभ्यां वाक्यजन्यत्वं निश्चित्य, ’अश्वमानय, गां बधान’ इति वाक्यान्तरे आवापोद्वापाभ्यां गोपदस्य गोत्वविशिष्टे शक्तिः, अश्वपदस्य अश्वत्वविशिष्टे शक्तिरिति व्युत्पद्यते~। ननु सर्वत्र कार्यपरत्वाद्व्यवहारस्य, कार्यपरवाक्य एव व्युत्पत्तिर्न सिद्धे इति चेत् , न~। ’काश्यां त्रिभुवनतिलको भूपतिः’ इत्यादौ सिद्धेऽपि व्यवहारात्, ’विकसितपद्मे मधूनि पिबति मधुकरः’ इत्यादौ प्रसिद्धपदसमभिव्याहारात्सिद्धेऽपि मधुकरादिव्युत्पत्तिदर्शनाच्च~। लक्षणापि शब्दवृत्तिः~। शक्यसम्बन्धो लक्षणा~। गङ्गायां घोष इत्यत्र गङ्गापदवाच्यप्रवाहसम्बन्धादेव तीरोपस्थितौ तीरेऽपि शक्तिर्न कल्प्यते~। सैन्धवादौ लवणाश्वयोः परस्परसम्बन्धाभावान्नानाशक्तिकल्पनम्~। लक्षणा त्रिविधा - जहल्लक्षणा, अजहल्लक्षणा, जहदजहल्लक्षणा चेति~। यत्र वाच्यार्थस्यान्वयाभावः तत्र जहल्लक्षणा~। यथा मञ्चाः क्रोशन्तीति~। यत्र वाच्यार्थस्याप्यन्वयः तत्र अजहदिति~। यथा छत्रिणो गच्छन्तीति~। यत्र वाच्यैकदेशत्यागेन एकदेशान्वयः तत्र जहदजहदिति~। यथा तत्वमसीति~। गौण्यपि लक्षणैव लक्ष्यमाणगुणसम्बन्धस्वरूपा यथा अग्निर्माणवकः इति~। व्यञ्जनापि शक्तिलक्षणान्तर्भूता, शब्दशक्तिमूला अर्थशक्तिमूला च~। अनुमानादिना अन्यथासिद्धा~। तात्पर्यानुपपत्तिर्लक्षणाबीजम्~। तत्प्रतीतीच्छयोच्चरितत्वं तात्पर्यम्~। तात्पर्यज्ञानञ्च वाक्यार्थज्ञाने हेतुः नानार्थानुरोधात्~। प्रकरणादिकं तात्पर्यग्राहकम्~। द्वारमित्यादौ पिधेहीति शब्दाध्याहारः~। ननु अर्थज्ञानार्थत्वाच्छब्दस्यार्थमविज्ञाय शब्दाध्याहारासम्भवादर्थाध्याहार एव युक्त इति चेत् न~। पदविशेषजन्यपदार्थोपस्थितेः शाब्दज्ञाने हेतुत्वात्~। अन्यथा ’घटः कर्मत्वमानयनं कृतिः’ इत्यत्रापि शाब्दज्ञानप्रसङ्गात्~। पङ्कजादिपदेषु योगरूढिः~। अवयवशक्तिर्योगः~। समुदायशक्ती रूढिः~। नियतपद्मत्वादिज्ञानार्थं समुदायशक्तिः~। अन्यथा कुमुदेऽपि प्रयोगप्रसङ्गात्~। ’इतरान्विते शक्तिः’ इति प्राभाकराः~। अन्वयस्य वाक्यार्थतया भानसम्भवादन्वयांशेऽपि शक्तिर्न कल्पनीया इति गौतमीयाः~।\\[10pt]
	{\bfseries आकाङ्क्षा योग्यता संनिधिश्च वाक्यार्थज्ञाने हेतुः~। पदस्य पदान्तरव्यतिरेकप्रयुक्तान्वयाननुभावकत्वम् आकाङ्क्षा~। अर्थाबाधो योग्यता~। पदानामविलम्बेनोच्चारणं संनिधिः~॥}\par
		आकाङ्क्षेति - आकाङ्क्षादिज्ञानमित्यर्थः~। अन्यथा आकाङ्क्षादिभ्रमाच्छाब्दभ्रमो न स्यात्~। आकाङ्क्षां लक्षयति पदस्येति~। योग्यतालक्षणमाह - अर्थेति~। सन्निधिलक्षणमाह पदानामिति~। अविलम्बेनपदार्थोपस्थितिः सन्निधिः~। उच्चारणं तु तदुपयोगितयोक्तम्~।\\[10pt]
	{\bfseries आकाङ्क्षादिरहितं वाक्यमप्रमाणम्~। यथा गौरश्वः पुरुषो हस्तीति न प्रमाणमाकाङ्क्षाविरहात्~। अग्निना सिञ्चेदिति न प्रमाणं योग्यताविरहात्~। प्रहरे प्रहरेऽसहोच्चारितानि गामानयेत्यादिपदानि न प्रमाणं सांनिध्याभावात्~॥}\par
		गौरश्व इति~। घटः, कर्मत्वमित्यप्यनाकाङ्क्षोदाहरणं द्रष्टव्यम्~।\\[10pt]
	{\bfseries वाक्यं द्विविधम्~। वैदिकं लौकिकं च~। वैदिकमीश्वरोक्तत्वात्सर्वमेव प्रमाणम्~। लौकिकं त्वाप्तोक्तं प्रमाणम्~। अन्यदप्रमाणम्~॥}\par
		वाक्यं विभजते - वाक्यमिति~। वैदिकस्य विशेषमाह वैदिकमीश्वरोक्तमिति~। ननु वेदस्यानादित्वात्कथमीश्वरोक्तमिति चेत् न~। ’वेदः पौरुषेयः, वाक्यसमूहत्वात् भारतादिवत्’ इत्यनुमानेन पौरुषेयत्वसिद्धेः~। न च स्मर्यमाणकर्तृकत्वमुपाधिः, गौतमादिभिः शिष्यपरम्परया वेदेऽपि सकर्तृत्वस्मरणेन साधनव्यापकत्वात्~। {\bfseries तस्मात्तेपानात्त्रयो वेदा अजायन्त} इति श्रुतेश्च~। ननु वर्णा नित्याः, स एवायं गकार इति प्रत्यभिज्ञाबलात्~। तथा च कथं वेदस्यानित्यत्वमिति चेत् न, ’उत्पन्नो गकरो विनष्टो गकारः’ इत्यादिप्रतीत्या वर्णानामनित्यत्वात्, ’सोऽयं गकारः’ इति प्रत्यभिज्ञायाः ’सेयं दीपज्वाला’ इतिवत्साजात्यालम्बनत्वात्~। वर्णानां नित्यत्वेऽप्यानुपूर्वीविशिष्टवाक्यस्यानित्यत्वाच्च~। तस्मादीश्वरोक्ता वेदाः~। मन्वादिस्मृतीनामाचाराणां च वेदमूलकतया प्रामाण्यम्~। स्मृतिमूलवाक्यानामिदानीमनध्ययनात्तन्मूलभूता काचिच्छाखोचछिन्नेति कल्प्यते~। ननु पठ्यमानवेदवाक्योत्सादस्य कल्पयितुमशक्यतया विप्रकीर्णवादस्यायुक्तत्वान्नित्यानुमेयो वेदो मूलमिति चेत् न, तथा सति कदापि वर्णानामानुपूर्वीज्ञानासम्भवेन बोधकत्वासम्भवात्~।\\[10pt]
	{\bfseries वाक्यार्थज्ञानं शाब्दज्ञानम्~। तत्करणं शब्दः~॥}\par
		ननु एतानि पदानि स्मारितार्थसंसर्गवन्ति आकाङ्क्षादिमत्पदकदम्बकत्वात् मद्वाक्यवत् इत्यनुमानादेव संसर्गज्ञानसम्भवाच्छब्दो न प्रमाणान्तरमिति चेन्न~। अनुमित्यपेक्षया विलक्षणस्य शाब्दज्ञानस्य ’शब्दात्प्रत्येमि’ इत्यनुव्यवसायसाक्षिकस्य सर्वसम्मतत्वात्~। नन्वर्थापत्तिरपि प्रमाणान्तरमस्ति ’पीनो देवदत्तो दिवा न भुङ्क्ते’ इति दृष्टे श्रुते वा पीनत्वान्यथानुपपत्या रात्रिभोजनमर्थापत्या कल्प्यत इति चेन्न~। ’देवदत्तो रात्रौ भुङ्क्ते दिवाऽभुञ्जानत्वे सति पीनत्वात्’ इत्यनुमानेनैव रात्रिभोजनस्य सिद्धत्वात्~। शते पञ्चाशदिति सम्भवोऽप्यनुमानमेव~। ’इह वटे यक्षस्तिष्ठति’ इत्यैतिह्यमपि अज्ञातमूलवक्तृकः शब्द एव~। चेष्टापि शब्दानुमानद्वारा व्यवहारहेतुरिति न प्रमाणान्तरम्~। तस्मात्प्रत्यक्षानुमानोपमानशब्दाश्चत्वार्येव प्रमाणानि~। \subsection*{अथ प्रामाण्यनिरूपणम्}
\addcontentsline{toc}{subsection}{४.५. प्रामाण्यनिरूपणम्}
		ज्ञानानां तद्वति तत्प्रकारकत्वं स्वतो ग्राह्यं परतो वेति विचार्यते~। तत्र विप्रतिपत्तिः - ज्ञानप्रामाण्यं तदप्रामाण्याग्राहकयावज्ज्ञानग्राहकसामग्रीग्राह्यं न वा इति~। अत्र विधिकोटिः स्वतस्त्वम्~। निषेधकोटिः परतस्त्वम्~। अनुमानादिग्राह्यत्वेन सिद्धसाधनवारणाय यावदिति~। ’इदं ज्ञानमप्रमा’ इति ज्ञानेन प्रामाण्याग्रहाद्बाधवारणाय - अप्रामाण्याग्राहकेति~। इदं ज्ञानमप्रमा इत्यनुव्यवसायनिष्ठप्रामाण्यग्राहकस्यापि अप्रामाण्याग्राहकत्वाभावात् स्वतस्त्वं न स्यादतस्तदिति~। तस्मिन्ग्राह्यप्रामाण्याश्रयेऽप्रामाण्याग्राहकेत्यर्थः~। उदाहृतस्थले व्यवसाये अप्रामाण्यग्राहकस्याप्यनुव्यवसाये तदग्राहकत्वात्स्वतस्त्वसिद्धिः~। ननु स्वत एव प्रामाण्यं गृह्यते, ’घटमहं जानामि’ इत्यनुव्यवसायेन घटघटत्वयोरिव तत्सम्बन्धस्यापि विषयीकरणात् व्यवसायरूपप्रत्यासत्तेस्तुल्यत्वात्~। पुरोवर्तिनि प्रकारसम्बन्धस्यैव प्रमात्वपदार्थत्वादिति चेत् न~। स्वतःप्रामाण्यग्रहे ’जलज्ञानं प्रमा न वा’ इत्यनभ्यासदशायां प्रमात्वसंशयो न स्यात्~। अनुव्यवसायेन प्रामाण्यस्य निश्चितत्वात्~। तस्मात्स्वतो ग्राह्यत्वाभावात्परतो ग्राह्यत्वमेव~। तथाहि~। प्रथमं जलज्ञानानन्तरं प्रवृत्तौ सत्यां जललाभे सति, पूर्वोत्पन्नं जलज्ञानं प्रमा, सफलप्रवृत्तिजनकत्त्वात्~, यन्नैवं तन्नैवं, यथा अप्रमा इति व्यतिरेकिणा प्रमात्वं निश्चीयते~। द्वितीयादिज्ञानेषु पूर्वज्ञानदृष्टान्तेन तत्सजातीयत्वलिङ्गेनान्वयव्यतिरेकिणापि गृह्यते~। प्रमाया गुणजन्यत्वमुत्पत्तौ परतस्त्वम्~। प्रमासाधारणकारणं गुणः, अप्रमासाधारणकारणं दोषः~। तत्र प्रत्यक्षे विशेषणवद्विशेष्यसंनिकर्षो गुणः~। अनुमितौ व्यापकवति व्याप्यज्ञानम्~। उपमितौ यथार्थसादृश्यज्ञानम्~। शाब्दज्ञाने यथार्थयोग्यताज्ञानम्~। इत्याद्यूहनीयम्~। पुरोवर्तिनि प्रकाराभावस्यानुव्यवसायेनानुपस्थितत्वादप्रमात्वं परत एव गृह्यते~। पित्तादिदोषजन्यत्वमुत्पत्तौ परतस्त्वम्~। ननु सर्वेषां ज्ञानानां यथार्थत्वादयथार्थज्ञानमेव नास्तीति~। न च ’शुक्ताविदं रजतम्’ इति ज्ञानात्प्रवृत्तिदर्शनादन्यथाख्यातिसिद्धिरिति वाच्यम्~। रजतस्मृतिपुरोवर्तिज्ञानाभ्यामेव प्रवृत्तिसम्भवात्~। उपस्थितेष्टभेदाग्रहस्यैव सर्वत्र प्रवर्तकत्वेन ’नेदं रजतम्’ इत्यादौ अतिप्रसङ्गाभावादिति चेत् न~। सत्यरजतस्थले पुरोवर्तिविशेष्यकरजतत्वप्रकारकज्ञानस्य लाघवेन प्रवृत्तिजनकतया शुक्तावपि रजतार्थिप्रवृत्तिजनकत्वेन विशिष्टज्ञानस्यैव कल्पनात्~।\\[10pt]
	{\bfseries अयथार्थानुभवस्त्रिविधः संशयविपर्ययतर्कभेदात्~॥}\par
		अयथार्थानुभवं विभजते - अयथार्थ इति~। स्वप्नस्य मानसविपर्ययरूपत्वान्न त्रैविध्यविरोधः~।\\[10pt]
	{\bfseries एकस्मिन्धर्मिणि विरुद्धनानाधर्मवैशिष्ट्यावगाहि ज्ञानं संशयः~। यथा स्थाणुर्वा पुरुषो वेति~॥}\par
		संशयलक्षणमाह - एकेति~। घटपटाविति समूहालम्बनेऽतिव्याप्तिवारणाय एकेति~। ’घटो द्रव्यम्’ इत्यादावतिव्याप्तिवारणाय विरुद्धेति~। घटत्वविरुद्धपटत्ववान् इत्यत्र अतिव्याप्तिवारणाय नानेति~।\\[10pt]
	{\bfseries मिथ्याज्ञानं विपर्ययः~। यथा शुक्तौ इदं रजतमिति~॥}\par
		विपर्ययलक्षणमाह - मिथ्येति~। तदभाववति तत्प्रकारकनिश्चय इत्यर्थः~।\\[10pt]
	{\bfseries व्याप्यारोपेण व्यापकारोपस्तर्क~। यथा यदि वह्निर्न स्यात्तर्हि धूमोऽपि न स्यादिति~॥}\par
		तर्कं लक्षयति - व्याप्येति~। यद्यपि तर्को विपर्ययेऽन्तर्भवति, तथापि प्रमाणानुग्राहकत्वाद्भेदेन कीर्तनम्~।\\[10pt]
	{\bfseries स्मृतिरपि द्विविधा~। यथार्थाऽयथार्था च~। प्रमाजन्या यथार्था~। अप्रमाजन्याऽयथार्था~॥}\par
		स्मृतिं विभजते - स्मृतिरिति~।\\[10pt]
	{\bfseries सर्वेषामनुकूलतया वेदनीयं सुखम्~॥}\par
		सुखं लक्षयति सर्वेषामिति~। ’सुख्यहम्’ इत्याद्यनुव्यवसायगम्यं सुखत्वादिकमेव लक्षणम्~। यथाश्रुतं तु स्वरूपकथनमिति द्रष्टव्यम्~।\\[10pt]
	{\bfseries सर्वेषां प्रतिकूलतया वेदनीयं दुःखम्~॥}\par
	{\bfseries इच्छा कामः~। क्रोधो द्वेषः~। कृतिः प्रयत्नः~। विहितकर्मजन्यो धर्मः~। निषिद्धकर्मजन्योऽधर्मः~॥}\par
	{\bfseries बुद्ध्यादयोऽष्टावात्ममात्रविशेषगुणाः~। बुद्धीच्छाप्रयत्ना द्विविधाः~। नित्या अनित्याश्च~। नित्या ईश्वरस्य अनित्या जीवस्य ~॥}\par
	{\bfseries संस्कारस्त्रिविधः~। वेगो भावना स्थितिस्थापकश्चेति~। वेगः पृथिव्यादिचतुष्टयमनोवृत्तिः~। अनुभवजन्या स्मृतिहेतुर्भावना, आत्ममात्रवृत्तिः~। अन्यथाकृतस्य पुनस्तदवस्थापादकः स्थितिस्थापकः, कटादिपृथिवीवृत्तिः~॥}\par
		संस्कारं विभजते - संस्कार इति~। संस्कारत्वजातिमान्संस्कारः~। वेगस्याश्रयमाह - वेग इति~। वेगत्वजातिमान्वेगः~। भावनां लक्षयति अनुभवेति~। अनुभवध्वंसेऽतिव्याप्तिवारणाय स्मृतिरिति~। आत्मादावतिव्याप्तिवारणाय अनुभवेति~। स्मृतेरपि संस्कारजनकत्वं नवीनैरुक्तम्~। स्थितिस्थापकं लक्षयति - अन्यथेति~। सङ्ख्यादयोष्टौ नैमित्तिकद्रवत्ववेगस्थितिस्थापकाः सामान्यगुणाः~। अन्ये रूपादयो विशेषगुणाः~। द्रव्यविभाजकोपाधिद्वयसमानाधिकरणावृत्तिजातिमत्वं विशेषगुणत्वम्~।\\
\section*{अथ कर्मनिरूपणम्}
\addcontentsline{toc}{section}{५. कर्मनिरूपणम्}
	{\bfseries चलनात्मकं कर्म~। ऊर्ध्वदेशसंयोगहेतुरुत्क्षेपणम्~। अधोदेशसंयोगहेतुरपक्षेपणम्~। शरीरसंनिकृष्टसंयोगहेतुराकुञ्चनम्~। विप्रकृष्टसंयोगहेतुः प्रसारणम्~। अन्यत् सर्वं गमनम्~। पृथिव्यादिचतुष्टयमनोवृत्ति~॥}\par
		कर्मणो लक्षणमाह - चलनेति~। उत्क्षेपणादीनां कार्यभेदमाह - ऊर्ध्वेति~। शरीरेति - वक्रतासम्पादकमाकुञ्चनम्~। ऋजुतासम्पादकं प्रसारणमित्यर्थः~।\\
\section*{अथ सामान्यनिरूपणम्}
\addcontentsline{toc}{section}{६. सामान्यनिरूपणम्}
	{\bfseries नित्यमेकमनेकानुगतं सामान्यं द्रव्यगुणकर्मवृत्ति~। तद्विविधं पराऽपरभेदात्~। परं सत्ता~। अपरं द्रव्यत्वादि~॥}\par
		सामान्यं लक्षयति नित्यमिति~। संयोगेऽतिव्याप्तिवारणाय नित्यमिति~। जलपरमाणुगतरूपेऽतिव्याप्तिवारणाय एकेति~। परमाणुपरिमाणादावतिव्याप्तिवारणाय अनेकेति~। अनुगतत्वं समवेतत्वम्~। तेन नाभावादवतिव्याप्तिः~।\\[10pt]
\section*{अथ विशेषनिरूपणम्}
\addcontentsline{toc}{section}{७. विशेषनिरूपणम्}{\bfseries नित्यद्रव्यवृत्तयो व्यावर्तका विशेषाः~॥}\par
		विशेषं लक्षयति - नित्येति~।\\
\section*{अथ समवायनिरूपणम्}
\addcontentsline{toc}{section}{८. समवायनिरूपणम्}
	{\bfseries नित्यसम्बन्धः समवायः~। अयुतसिद्धवृत्तिः~। ययोर्द्वयोर्मध्ये एकमविनश्यदपराऽश्रितमेवावतिष्ठते तावयुतसिद्धौ, यथा अवयवावयविनौ, गुणगुणिनौ, क्रियाक्रियावन्तौ, जातिव्यक्ती, विशेषनित्यद्रव्ये चेति~॥}\par
		समवायं लक्षयति - नित्येति~। संयोगेऽतिव्याप्तिवारणाय नित्येति~। आकाशादावतिव्याप्तिवारणाय संबन्धेति~। अयुतसिद्धलक्षणमाह - ययोरिति~। ’नीलो घटः’ इति विशिष्टप्रतीतिर्विशेषणविशेष्यसम्बन्धविषया, विशिष्टबुद्धित्वात्, दण्डीति विशिष्टबुद्धिवत् इति समवायसिद्धिः~। अवयवावयविनाविति - द्रव्यसमवायिकारणमवयवः जन्यद्रव्यमवयवि~।\\
\section*{अथ अभावनिरूपणम्}
\addcontentsline{toc}{section}{९. अभावनिरूपणम्}
	{\bfseries अनादिः सान्तः प्रागभावः~। उत्पत्तेः पूर्वं कार्यस्य~॥}\par
		प्रागभावं लक्षयति - अनादिरिति~। प्रतियोगिसमवायिकारणवृत्तिः, प्रतियोगिजनको भविष्यतीति व्यवहारहेतुः प्रागभावः~।\\[10pt]
	{\bfseries सादिरनन्तः प्रध्वंसः~। उत्पत्यनन्तरं कार्यस्य ~॥}\par
		ध्वंसं लक्षयति सादिरिति~। घटादावतिव्याप्तिवारणाय अनन्तेति~। आकाशादावतिव्याप्तिवारणाय सादिरिति~। प्रतियोगिजन्यः, प्रतियोगिसमवायिकारणवृत्तिर्ध्वस्त इति व्यवहारहेतुः ध्वंसः~।\\[10pt]
	{\bfseries त्रैकालिकसंसर्गावच्छिन्नप्रतियोगिताकोऽत्यन्ताभावः~। यथा भूतले घटो नास्तीति ~॥}\par
		अत्यन्ताभावं लक्षयति त्रैकालिकेति~। अन्योन्याभावेऽतिव्याप्तिवारणाय संसर्गेति~। ध्वंसप्रागभावयोरतिव्याप्तिवारणाय त्रैकालिकेति~।\\[10pt]
	{\bfseries तादात्म्यसम्बन्धावच्छिन्नप्रतियोगिताकोऽन्योन्याभावः~। यथा घटः पटो नेति~॥}\par
		अन्योन्याभावं लक्षयति - तादात्म्येति~। प्रतियोगितावच्छेदकारोप्यसंसर्गभेदादेकप्रतियोगिकयोरत्यन्तान्योन्याभावयोर्भिन्नत्वम्~। 'केवलदेवदत्ताभावात् दण्ड्यभाव' इति प्रतीत्या विशिष्टाभावः, ’एकसत्वेऽपि द्वौ न स्तः’ इति प्रतीत्या द्वित्वावच्छिन्नाभावः, संयोगेन घटवति समवायेन घटाभावः, तत्तद्घटाभावाद्घटत्त्वावच्छिन्नप्रतियोगिताकसामान्याभावश्चातिरिक्तः~। एवमन्योन्याभावोऽपि~। घटत्वेन पटो नास्तीति व्यधिकरणधर्मावच्छिन्नाभावो नाङ्गीक्रियते~। पटे घटत्वं नास्तीति तदर्थः~। अतिरिक्तत्वे स केवलान्वयी~। सामयिकाभावोऽप्यत्यन्ताभाव एव समयविशेषे प्रतीयमानः~। घटाभाववति घटानयने अत्यन्ताभावस्य अन्यत्र गमनाभावेऽपि अप्रतीतेः, घटापसरणे सति प्रतीतेः, भूतलघटसंयोगप्रागभावध्वंसयोर्घटात्यन्ताभावप्रतीतिनियामकत्वं कल्प्यते~। घटवति तत्संयोगप्रागभावध्वंसयोरसत्त्वादेवात्यन्ताभावस्याप्रतीतिः~। घटापसरणे तु संयोगध्वंसस्य सत्त्वात्प्रतीतिरिति~। केवलाधिकरणेनैव नास्तीति व्यवहारोपपत्तौ अभावो न पदार्थान्तरमिति गुरवः~। तन्न~। अभावानङ्गीकारे कैवल्यस्य निर्वक्तुमशक्यत्वात्~। अभावाभावो भाव एव~। नातिरिक्तः, अनवस्थाप्रसङ्गात्~। ध्वंसप्रागभावः, प्रागभावध्वंसश्च प्रतियोग्येवेति प्राञ्चः~। अभावाभावोऽतिरिक्त एव तृतीयाभावस्य प्रथमाभावरूपत्वान्नानवस्थेति नवीनाः~।\\[10pt]
	{\bfseries सर्वेषां पदार्थानां यथायथमुक्तेष्वन्तर्भावात् सप्तैव पदार्थाः इति सिद्धम्~॥}\par
		ननु {\bfseries प्रमाणप्रमेयसंशयप्रयोजनदृष्टान्तसिद्धान्तावयवतर्कनिर्णयवादजल्पवितण्डाहेत्वाभास\-- च्छलजातिनिग्रहस्थानानां तत्त्वज्ञानान्निःश्रेयसाधिगमः} इति न्यायसूत्रे षोडशपदार्थानामुक्तत्वात्कथं सप्तैवेत्यत आह- सर्वेषामिति~। सर्वेषां सप्तस्वेवान्तर्भाव इत्यर्थः~। ’आत्मशरीरेन्द्रियार्थबुद्धिमनःप्रवृत्तिदोषप्रेत्यभावफलदुःखापवर्गास्तु प्रमेयम्’ इति द्वादशविधं प्रमेयम्~। प्रवृत्तिर्धर्माधर्मौ~। रागद्वेषमोहा दोषाः~। राग इच्छा~। द्वेषो मन्युः~। मोहः शरीरादौ आत्मत्वभ्रमः~। प्रेत्यभावो मरणम्~। फलं भोगः~। अपवर्गो मोक्षः~। स च स्वसमानाधिकरणदुःखप्रागभावासमानकालीनदुःखध्वंसः~। प्रयोजनं सुखप्राप्तिः दुःखहानिश्च~। दृष्टान्तो महानसादिः~। प्रामाणिकत्वेन अभ्युपगतोऽर्थः सिद्धान्तः~। निर्णयो निश्चयः~। स च प्रमाणफलम्~। तत्त्वबुभुत्सोः कथा वादः~। उभयसाधनवती विजिगीषुकथा जल्पः~। स्वपक्षस्थापनहीना (अपरपक्षविदळनमात्रावसाना) वितण्डा~। कथा नाम नानावक्तृकपूर्वोत्तरपक्षप्रतिपादकवाक्यसन्दर्भः~। अभिप्रायान्तरेण प्रयुक्तस्यार्थान्तरं परिकल्प्य दूषणं छलम्~। असदुत्तरं जातिः~। साधर्म्यवैधर्म्योत्कर्षापकर्षवर्ण्यावर्ण्यविकल्पसाध्यप्राप्त्यप्राप्तिप्रसङ्गप्रतिदृष्टान्तानुत्पत्तिसंशयप्रक\-- रणाहेत्वर्थापत्यविशेषोपपत्त्युपलब्ध्यनुपलब्धिनित्यानित्यकार्याकार्यसमा जातयः~। वादिनोऽपजयहेतुर्निग्रहस्थानम्~। तच्च प्रतिज्ञाहानिः, प्रतिज्ञान्तरम्, प्रतिज्ञाविरोधः, हेत्वन्तरम्, अर्थान्तरम्, निरर्थकम्, अविज्ञातार्थकम्, अपार्थकम्, अप्राप्तकालम्, न्यूनम्, अधिकम्, पुनरुक्तम्, अननुभाषणम्, अज्ञानम्, अप्रतिभा, विक्षेपः, मतानुज्ञा, पर्यनुयोज्योपेक्षणम्, निरनुयोज्यानुयोगः, अपसिद्धान्तः, हेत्वाभासाश्च~। शेषं सुगमम्~।
		\begin{center} {\bfseries काणादन्यायमतयोर्बालव्युत्पत्तिसिद्धये~।\\ अन्नम्भट्टेन विदुषा रचितस्तर्कसङ्ग्रहः~॥\\[10pt]॥ इति श्रीमहामहोपाध्याय-अन्नम्भट्टविरचित-तर्कसङ्ग्रहः समाप्तः ॥}\end{center}
\subsection*{अथ शक्तिनिराकरणम्}
\addcontentsline{toc}{subsection}{९.१. शक्तिनिराकरणम्}
		ननु करतलानलसंयोगे सत्यपि, प्रतिबन्धके सति दाहानुत्पत्तेः शक्तिः पदार्थान्तरमिति चेत् न~। प्रतिबन्धकाभावस्य कार्यमात्रे कारणत्वेन शक्तेरनुपयोगात्~। कारणत्वस्यैव शक्तिपदार्थत्वात्~। ननु भस्मादिना कांस्यादौ शुद्धिदर्शनादाधेयशक्तिरङ्गीकार्येति चेत् न~। भस्मादिसंयोगसमानकालीनास्पृश्यस्पर्शप्रतियोगिकयावदभावसहितभस्मादिसंयोगध्वंसस्य शुद्धिपदार्थत्वात्~। स्वत्वमपि न पदार्थान्तरम्~, यथेष्टविनियोगयोग्यत्वस्य स्वत्वरूपत्वात्~। तदवच्छेदकं च प्रतिग्रहादिलब्धत्वमेवेति~।
\subsection*{अथ विधिनिरूपणम्}
\addcontentsline{toc}{subsection}{९.२. विधिनिरूपणम्}
		अथ विधिर्निरूप्यते~। प्रयत्नजनकचिकीर्षाजनकज्ञानविषयो विधिः~। तत्प्रतिपादको लिङादिर्वा~। कृत्यसाध्ये प्रवृत्त्यदर्शनात् कृतिसाध्यताज्ञानं प्रवर्तकम्~। न च विषभक्षणादौ प्रवृत्तिप्रसङ्गः, इष्टसाधनतालिङ्गककृतिसाध्यताज्ञानस्य काम्यस्थले विहितकालशुचिजीवित्वनिमित्तज्ञानजन्यस्य नित्यनैमित्तिकस्थले प्रवर्तकत्वात्~। न चाननुगमः स्वविशेषणवत्ताप्रतिसन्धानजन्यत्वस्यानुगतत्वादिति गुरवः~। तत् न~। लाघवेन कृतिसाध्येष्टसाधनताज्ञानस्यैव चिकीर्षाद्वारा प्रयत्नजनकत्वात्~। न च नित्यनैमित्तिकस्थले इष्टसाधनत्वाभावादप्रवृत्तिप्रसङ्गः, तत्रापि प्रत्यवायपरिहारस्य पापक्षयस्य च फलत्वकल्पनात्~। तस्मात्कृतिसाध्येष्टसाधनत्वमेव लिङाद्यर्थः~। ननु {\bfseries ज्योतिष्टोमेन स्वर्गकामो यजेत} इत्यत्र लिङा स्वर्गसाधनमपूर्वं कार्यं प्रतीयते~। यागस्याशुतरविनाशिनः कालान्तरभाविस्वर्गसाधनत्वायोगात् तद्योग्यं स्थायिकार्यमपूर्वमेव लिङाद्यर्थः~। कार्यं कृतिसाध्यम्~। कृतेः सविषयत्वाद्विषयाकाङ्क्षायां यागो विषयत्वेनान्वेति~। ’कस्य कार्यम्’ इति नियोज्याकाङ्क्षायां स्वर्गकामपदं नियोज्यपरतयान्वेति~। कार्यबोद्धा नियोज्यः~। तेन ’ज्योतिष्टोमनामकयागविषयकं स्वर्गकामस्य कार्यम्’ इति वाक्यार्थः सम्पद्यते~। वैदिकलिङ्त्वात् {\bfseries यावज्जीवमग्निहोत्रं जुहुयात्} इति नित्यवाक्येऽप्यपूर्वमेव वाच्यं कल्प्यते~। ’आरोग्यकामो भेषजपानं कुर्यात्’ इत्यादिलौकिकलिङः क्रियाकार्ये लक्षणा इति चेत् न~। यागस्याप्ययोग्यतानिश्चयाभावेन इष्टसाधनतया प्रतीत्यनन्तरं तन्निर्वाहार्थम् - अवान्तरव्यापारतयाऽपूर्वकल्पनात्~। कीर्तनादिना नाशश्रुतेर्न यागध्वंसो व्यापारः~। लोकव्युत्पत्तिबलात्क्रियायामेव कृतिसाध्येष्टसाधनत्वं लिङा बोध्यत इति लिङ्त्वेन विध्यर्थकत्वम्, अाख्यातत्वेन यत्नार्थकत्वम्~। पचति, पाकं करोतीति विवरणदर्शनात्, किं करोतीति प्रश्ने पचतीत्युत्तराच्चाख्यातस्य प्रयत्नार्थकत्वात्~। रथो गच्छतीत्यादौ अनुकूलव्यापारे लक्षणा~। देवदत्तः पचति, देवदत्तेन पच्यते तण्डुलः, इत्यादौ कर्तृकर्मणोर्नाख्यातार्थत्वम्~। किं तु तद्गतैकत्वादीनामेव~। तयोराक्षेपादेव लाभः~। प्रजयतीत्यादौ धातोरेव प्रकर्षे शक्तिः~। उपसर्गाणां द्योतकत्वमेव न तत्र शक्तिः~।
\subsection*{अथ मोक्षनिरूपणम्}
\addcontentsline{toc}{subsection}{९.३. मोक्षनिरूपणम्}
		पदार्थतत्वज्ञानस्य परमप्रयोजनं मोक्षः~। तथाहि ’आत्मा वाऽरे द्रष्टव्यः श्रोतव्यो मन्तव्यो निदिध्यासितव्यः’ इति श्रुत्या श्रवणादीनाम् आत्मसाक्षात्कारहेतुत्वबोधनात् श्रुत्या देहादिविलक्षणात्मज्ञाने सत्यप्यसम्भावनानिवृत्तेः युक्त्यनुसन्धानरूपमननसाध्यत्वात् मननोपयोगिपदार्थनिरूपणद्वारा शास्त्रस्यापि मोक्षोपयोगित्वम्~। तदनन्तरं श्रुत्युपदिष्टयोगविधिना निदिध्यासने कृते, तदनन्तरं देहादिविलक्षणात्मसाक्षात्कारे सति, देहादावहमित्यभिमानरूपमिथ्याज्ञाननाशे, दोषाभावात् प्रवृत्त्यभावे, धर्माधर्मयोरभावे, जन्माभावे, पूर्वधर्माधर्मयोरनुभवेन नाशे, चरमदुःखध्वंसलक्षणमोक्षो जायते~। ज्ञानमेव मोक्षसाधनम्~। मिथ्याज्ञाननिवृत्तेर्ज्ञानमात्रसाध्यत्वात्~। {\bfseries तमेव विदिवाऽतिमृत्युमेति नान्यः पन्था विद्यतेऽयनाय} इति श्रुत्या साधनान्तरनिषेधाच्च~। ननु {\bfseries तत्प्राप्तिहेतुर्विज्ञानं कर्म चोक्तं महामुने} इति कर्मणो मोक्षसाधनत्वस्मरणाज्ज्ञानकर्मणोः समुच्चय इति चेत् न~। {\bfseries नित्यनैमित्तिकैरेव कुर्वाणो दुरितक्षयम्~। ज्ञानं च विमलीकुर्वन्नभ्यासेन च पाचयेत्~। अभ्यासाच्च क्वचिज्ज्ञानात्कैवल्यं लभते नरः~।} इत्यादिना कर्मणो ज्ञानसाधनत्वप्रतिपादनाज्ज्ञानद्वारैव कर्मणो मोक्षसाधनत्वं न साक्षात्~। तस्मात्पदार्थतत्त्वज्ञानस्य मोक्षः परमप्रयोजनमिति सर्वं रमणीयम्~॥
	\begin{center}
	{\bfseries~॥ इति श्रीमदन्नम्भट्टोपाध्यायविरचिता तर्कसङ्ग्रहदीपिका समाप्ता~॥}\\[10pt]
	\end{center}
	
