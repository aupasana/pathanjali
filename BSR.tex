\titlespacing{\section}{0pt}{*-5}{*-4}
\titleformat*{\section}{\large\bfseries }
\renewcommand*\contentsname{\Large विषयानुक्रमणिका}
\frontmatter
\title{वैयाकरणभूषणसारः}
\author{कौण्डभट्टः}
\date{}
 \begin{titlepage}
 \vfill
 \vfill
 \centering
 \maketitle
 \end{titlepage}
\thispagestyle{empty}
\tableofcontents
\fancyhead[RE,LO]{वैयाकरणभूषणसारः}
\mainmatter
\renewcommand{\thepage}{\devanagarinumeral{page}}
\section*{\begin{center} मङ्गलम् \end{center}}\addcontentsline{toc}{section}{मङ्गलम्}
\fancyhead[LE,RO]{मङ्गलम्}
\begin{center}
 श्रीलक्ष्मीरमणं नौमि गौरीरमणरूपिणम्~।\newline
 स्फोटरूपं यतः सर्वं जगदेतद् विवर्त्तते ॥१॥\\[10pt]
 अशेषफलदातारं भवाब्धितरणे तरिम्~।\\
 शेषाशेषार्थलाभार्थं प्रार्थये शेषभूषणम् ॥२॥\\[10pt]
 पाणिन्यादिमुनीन् प्रणम्य पितरं रङ्गोजिभट्टाभिधं\\
 द्वैतध्वान्तनिवारणादिफलिकां पुम्भाववाग्देवताम्~।\\
 ढुण्ढिं गौतमजैमिनीयवचनव्याख्यातृभिर्दूषितान्\\
 सिद्धान्तानुपपत्तिभिः प्रकटये तेषां वचो दूषये ॥३॥\\[10pt]
 नत्वा गणेशपादाब्जं गुरूनथ सरस्वतीम्~।\\
 श्रीकौण्डभट्टः कुर्वेऽहं वैयाकरणभूषणम् ॥४॥
\end{center}
 प्रारीप्सितप्रतिबन्धकोपशमनाय कृतं श्रीफणिस्मरणरूपं मङ्गलं शिष्यशिक्षार्थं निबध्नन् चिकीर्षितं प्रतिजानीते-
\begin{center}{\bfseries फणिभाषितभाष्याब्धेः शब्दकौस्तुभ उद्धृतः~।\\
 तत्र निर्णीत एवाऽर्थः सङ्क्षेपेणेह कथ्यते ॥१॥}
\end{center}
 उद्धृत इति~।
अत्र `अस्माभिः' इति शेषः~।
`भाष्याब्धेः शब्दकौस्तुभ उद्धृतः' इत्युक्तिस्तु शब्दकौस्तुभोक्तानामर्थानामाधुनिकोत्प्रेक्षितत्वनिरासाय~।
अन्यथा तन्मूलकस्याऽस्य ग्रन्थस्याप्याधुनिकोत्प्रेक्षितासारत्वापत्तौ पाणिनीयानामनुपादेयतापत्तेः~।`तत्र निर्णित' इत्युक्तिरितोऽप्यधिकं जिज्ञासुभिः शब्दकौस्तुभे द्रष्टव्यमिति ध्वनयितुम् ॥१॥
\cleardoublepage
\section*{\begin{center} अथ धात्वर्थनिर्णयः\end{center}}\addcontentsline{toc}{section}{धात्वर्थनिर्णयः}
\fancyhead[LE,RO]{धात्वर्थनिर्णयः}
प्रतिज्ञातमाह-
\begin{center}
{\bfseries फलव्यापारयोर्धातुराश्रये तु तिङः स्मृताः~।\\
 फले प्रधानं व्यापारस्तिङर्थस्तु विशेषणम् ॥२॥}
\end{center}
धातुरित्यत्र स्मृत इति वचनविपरिणामेनाऽन्वयः~।
फलम् = विक्लित्त्यादि~।
व्यापारस्तु भावनाभिधा = साध्यत्वेनाभिधीयमाना क्रिया~।
उक्तञ्च वाक्यपदीये -
 \begin{center}`यावत् सिद्धमसिद्धं वा साध्यत्वेनाभिधीयते~।\\
 आश्रितक्रमरूपत्वात् सा क्रियेत्यऽभिधीयते ॥
\end{center}
इति~।
 न च साध्यत्वेनाऽभिधाने मानाभावः, `पचति, पाकः, करोति, कृतिः' इत्यादौ धात्वर्थावगमाविशेषेऽपि क्रियान्तराकाङ्क्षानाकाङ्क्षयोर्दर्शनस्यैव मानत्वात्~।
तथा च क्रियान्तराकाङ्क्षाऽनुत्थापकतावच्छेदकरूपं साध्यत्वम्~।
तद्रूपवत्त्वमसत्त्वभूतत्वम्~।
एतदेवाऽऽदाय-
\begin{center} `असत्त्वभूतो भावश्च तिङ्पदैरभिधीयते~।'\end{center}
इति वाक्यपदीयमिति द्रष्टव्यम्~।
 अयञ्च व्यापारः फूत्कारत्वाधःसन्तापनत्वयत्नत्वादितत्तद्रूपेण वाच्यः, पचतीत्यादौ तत्तत्प्रकारकबोधस्याऽनुभवसिद्धत्वात्~।
न च नानार्थतापत्तिः, `तदादि'न्यायेन बुद्धिविशेषादेः शक्यतावच्छेदकानामनुगमकस्य सत्त्वात्~।
आख्याते क्रियैकत्वव्यवस्थाऽपि अवच्छेदकबुद्धिविशेषैक्यमादायैव~।
उक्तञ्च वाक्यपदीये -
\begin{center} `गुणभूतैरवयवैः समूहः क्रमजन्मनाम्~।\\
 बुद्धया प्रकल्पिताऽभेदः क्रियेति व्यपदिश्यते ॥'\end{center}
 इति~। धात्वर्थं निरूप्य तिङर्थमाह-
आश्रये त्विति - फलाश्रये, व्यापाराश्रये चेत्यर्थः~।
फलाश्रयः कर्म, व्यापाराश्रयः कर्ता~।
तत्र फलव्यापारयोर्धातुलभ्यत्वान्न तिङस्तदंशे शक्तिः, अन्यलभ्यत्वात्~।
शक्यतावच्छेदकं चाश्रयत्वं तत्तच्छक्तिविशेषरूपमिति सुबर्थनिर्णये वक्ष्यते~।
नन्वनयोराख्यातार्थत्वे किं मानम्~।
प्रतीतेर्लक्षणया आक्षेपात् प्रथमान्तपदाद्वा सम्भवादिति चेत् -
अत्रोच्यते - `लः कर्मणि च भावे चाऽकर्मकेभ्यः'\footnote{३.४.६९} इति सूत्रमेव मानम्~।
अत्र हि चकारात् 'कर्तरि कृत्' इति सूत्रोक्तं कर्तरीत्यनुकृष्यते~।
बोधकतारूपां तिबादिशक्तिं तत्स्थानित्वेन कल्पिते लकारे प्रकल्प्य लकाराः कर्मणि कर्तरि चानेन विधीयन्ते, नकारविसर्गादिनिष्ठां कर्मकरणादिबोधकताशक्तिमादाय शसादिविधानवत्~।
न च सूत्रे कर्तृकर्मपदे कर्तृत्वकर्मत्वपरे~।
तथा च कर्तृत्वम् - कृतिः, कर्मत्वञ्च फलमेवार्थोऽस्त्विति शङ्क्यं, फलव्यापारयोर्धातुलभ्यत्वेन लकारस्य पुनस्तत्र शक्तिकल्पनाऽयोगात्~।
अथ दर्शनान्तरीयरीत्या व्यापारस्य धात्वर्थत्वाभावात्तत्र लकारविधिः स्यादिति चेत् , तर्हि कृतामपि कर्तृकर्मादिवाचित्वं न सिद्ध्येत् , `कर्तरि कृत्' इति च `लः कर्मणि' इत्यनेन तुल्ययोगक्षेमम्~।
अपि च मीमांसकानां कृतामिवाऽऽख्यातानामपि कर्तृवाचित्वमस्तु , भावनाया एवाऽऽक्षेपेण कृदादिवत् प्रतीतिसम्भवे वाच्यत्वं माऽस्तु~।
 तथा सति प्राधान्यं तस्या न स्यादिति चेन्न, `घटमानय' इत्यादावाक्षिप्तव्यक्तेरपि प्राधान्यवदुपपत्तेः~।
पचतीत्यादौ पाकं करोतीति भावनाया विवरणदर्शनाद्वाच्यत्वमिति चेन्न, पाकाऽनुकूलव्यापारवतः कर्तुरपि विवरणविषयत्वाऽविशेषात्~।
न च कर्तुर्विवरणं तात्पर्याऽर्थविवरणम् , पाकं करोतीत्यशब्दाऽर्थकर्मत्वविवरणवत् , इतरेतरयोगद्वन्द्वे समुच्चयांशविवरणवद्वा न तदर्थनिर्णायकमिति वाच्यम् , भावनायामपि तुल्यत्वात्~।
किञ्च, `पचति देवदत्तः' इत्यत्राऽभेदान्वयदर्शनात्तदनुरोधेन कर्तुर्वाच्यत्वमावश्यकम् `पक्ता देवदत्तः' इतिवत्~।
न चाऽभेदबोधे समानविभक्तिकत्वं नियामकम् , तच्चात्र नास्तीति वाच्यम् , `सोमेन यजेत' `स्तोकं पचति' `राजपुरुषः' इत्यादावप्यभेदबोधाऽनापत्तेः~।
न च लक्षणया कर्तुरुक्तत्वात् सामानाधिकरण्यम् , पिङ्गाक्ष्यादियौगिकानामपि द्रव्यवाचित्वाऽनापत्तेः~।
एवं वैश्वदेवीत्यादितद्धितानामपि, `अनेकमन्यपदार्थे', `साऽस्य देवता' इत्यनुशासनेन पिङ्गे आक्षिणी यस्याः, विश्वे देवा देवता अस्या इति विग्रहदर्शनात् प्रधानषष्ठ्यर्थे एव अनुशासनलाभात्~।
तथाच `अरुणया पिङ्गाक्ष्यैकहायन्या सोमं क्रीणाति' इति वाक्ये द्रव्याऽनुक्तेरारुण्यस्य स्ववाक्योपात्तद्रव्ये एवाऽन्वयप्रतिपादकाऽरुणाधिकरणोच्छेदाऽऽपत्तिः, द्रव्यवाचकत्वसाधकमूलयुक्तेः सामानाधिकरण्यस्योक्तरीत्योपपत्तेरिति प्रपञ्चितं विस्तरेण बृहद्वैयाकरणभूषणे~।
तिङः इति~।
बोधकतारूपा शक्तिस्तिङ्क्ष्वेवेत्यभिप्रेत्येदम्~।
पदार्थं निरूप्य वाक्यार्थं निरूपयति - फले इत्यादि~।
विक्लित्त्यादि फलं प्रति~।
तिङर्थः - कर्तृकर्मसंख्याकालाः~।
तत्र कर्तृकर्मणी फलव्यापारयोर्विशेषणे~।
संख्या कर्तृप्रत्यये कर्तरि, कर्मप्रत्यये कर्मणि, समानप्रत्ययोपात्तत्वात्~।
तथा च `आख्यातार्थसंख्याप्रकारकबोधं प्रति आख्यातजन्यकर्तृकर्मोपस्थितिर्हेतुः' इति कार्यकारणभावः फलितः~।
नैयायिकादीनामाख्यातार्थसङ्ख्यायाः प्रथमान्तार्थे एवाऽन्वयात् `आख्यातार्थसंख्याप्रकारकबोधे प्रथमान्तपदजन्योपस्थितिर्हेतुः' इति कार्यकारणभावो वाच्यः~।
सोऽपि `चन्द्र इव मुखं दृश्यते', `देवदत्तो भुक्त्वा व्रजति' इत्यादौ चन्द्रक्त्वार्थयोराख्यातार्थानन्वयादितराऽविशेषणत्वघटित इत्यतिगौरवम्~।
इदमपि कर्तृकर्मणोराख्यातार्थत्वे मानमिति स्पष्टं भूषणे~।
कालस्तु व्यापारे विशेषणम् , तथाहि - `वर्तमाने लट्'\footnote{३-२-१२३} इत्यत्राऽधिकाराद् धातोरिति लब्धम्~।
तच्च धात्वर्थं वदत् प्राधान्याद् व्यापारमेव ग्राहयतीति तत्रैव तदन्वयः~।
न च सङ्ख्यावत् कर्तृकर्मणोरेवान्वयः शङ्क्यः,अतीतभावनाके कर्तरि पचतीत्यापत्तेः, अपाक्षीदित्यनापत्तेश्च~।
पाकानारम्भदशायां कर्तृसत्त्वे पक्ष्यतीत्यनापत्तेश्च~।
नापि फले तदन्वयः,फलाऽनुत्पत्तिदशायां व्यापारसत्त्वे, पचतीत्यनापत्तेः, पक्ष्यतीत्यापत्तेश्चेत्यवधेयम्~।
न चाऽऽमवातजडीकृतकलेवरस्योत्थानाऽनुकूलयत्नसत्त्वादुत्तिष्ठतीति प्रयोगाऽऽपत्तिः, परयत्नस्याज्ञानादप्रयोगात्~।
किञ्चिच्चेष्टादिना तदवगतौ च, अयमुत्तिष्ठति, शक्त्यभावात् फलन्तु न जायते इति लोकप्रतीतेरिष्टत्वात्~।
एवञ्च तिङर्थो विशेषणमेव, भावनैव प्रधानम्~।
यद्यपि प्रकृतिप्रत्ययार्थयोः प्रत्ययार्थस्यैव प्राधान्यमन्यत्र दृष्टम् , तथाऽपि `भावप्रधानमाख्यातं सत्त्वप्रधानानि नामानि' इति\footnote{नि.अ.२ ख.२} निरुक्तात् , भूवादिसूत्रादिस्थक्रियाप्राधान्यबोधकभाष्याच्च धात्वर्थभावनाप्राधान्यमध्यवसीयते~।
अपि च, आख्यातार्थप्राधान्ये तस्य देवदत्तादिभिः सममभेदान्वयात् प्रथामान्तस्य प्राधान्यापत्तिः~।
तथा च, `पश्य मृगो धावति' इत्यत्र भाष्यसिद्धैकवाक्यता न स्यात् , प्रथमान्तमृगस्य धावनक्रियाविशेष्यस्य दृशिक्रियायां कर्मत्वापत्तौ द्वितीयापत्तेः~।
न चैवमप्रथमासामानाधिकरण्याच्छतृप्रसङ्गः, एवमपि द्वितीयाया दुर्वारत्वेन `पश्य मृगः' इत्यादिवाक्यस्यैवाऽसम्भवापत्तेः~।
न च पश्येत्यत्र तमिति कर्माध्याहार्य्यम् ,वाक्यभेदप्रसङ्गात् , उत्कटधावनक्रियाविशेषस्यैव दर्शनकर्मतयाऽन्वयस्य प्रतिपिपादयिषितत्वात् , आध्याहारेऽनन्वयापत्तेश्च~।
एवञ्च, `भावनाप्रकारकबोधे प्रथमान्तपदजन्योपस्थितिः कारणम्' इति नैयायिकोक्तं नादरणीयम्~।
किन्तु `आख्यातार्थकर्तृप्रकारकबोधे धातुजन्योपस्थितिर्भावनात्वावच्छिन्नविषयतया कारणम्' इति कार्यकारणभावो द्रष्टव्यः~।
भावनाप्रकारकबोधं प्रति तु कृज्जन्योपस्थितिवद् धात्वर्थभावनोपस्थितिरपि हेतुः, `पश्य मृगो धावति', `पचति भवति' इत्याद्यनुरोधादिति दिक्~।
इत्थञ्च पचतीत्यत्रैकाऽऽश्रयिका पाकाऽनुकूला भावना~।
पच्यते इत्यत्रैकाऽऽश्रया या विक्लित्तिस्तदनुकूला भावनेति बोधः~।
देवदत्तादिपदप्रयोगे त्वाख्यातार्थकर्त्रादिभिस्तदर्थस्याऽभेदाऽन्वयः~।
`घटो नश्यति' इत्यत्र घटाऽभिन्नाऽऽश्रयको नाशाऽनुकूलो व्यापार इति बोधः~।
स च व्यापारः प्रतियोगित्वविशिष्टनाशसामग्रीसमवधानम्~।
अत एव तस्यां सत्यां नश्यति, तदत्यये 'नष्टः' तद्भावित्वे नङ्क्षयतीति प्रयोगः~।
`देवदत्तो जानाति', इच्छतीत्यादौ च देवदत्ताऽभिन्नाऽऽश्रयको ज्ञानेच्छाद्यनुकूलो वर्त्तमानो व्यापार इति बोधः~।
स चान्तत आश्रयतैवेतिरीत्योह्यम् ॥२॥\par
नन्वाख्यातस्य कर्तृकर्मशक्तत्वे `पचति' इत्यत्रोभयबोधापत्तिः~।
कर्तृमात्रबोधवत् कर्ममात्रस्यापि बोधापत्तिरित्यतस्तात्पर्यग्राहकमाह-
\begin{center}
{\bfseries फलव्यापारयोस्तत्र फले तङ्यक्चिणादयः~।\\
 व्यापारे शप्श्नमाद्यास्तु द्योतयन्त्याश्रयाऽन्वयम् ॥३॥}
\end{center}
 तङादयः फले आश्रयाऽन्वयं द्योतयन्ति, फलाऽन्वय्याश्रयस्य कर्मत्वात्~।
तद्द्योतकाः = कर्मद्योतकाः~।
व्यापाराऽन्वय्याश्रयस्य कर्तृत्वात्~।
तद्द्योतकाः = कर्तृद्योतका इति समुदायार्थः~।
द्योतयन्ति = तात्पर्यं ग्राहयन्ति ॥३॥\par
नन्वेवं `क्रमादमुं नारद इत्यबोधि स' इत्यादौ, पच्यते ओदनः स्वयमेव' इत्यादौ च व्यभिचारः, कर्मणः कर्तृत्वविवक्षायां कर्तरि लकारे सति, `कर्मवत् कर्मणा तुल्यक्रियः'\footnote{३.१.८७} इत्यतिदेशेन यगात्मनेपदचिण्चिण्वदिटामतिदेशाद्यगादिसत्त्वेऽपि कर्तुरेव बोधात् , व्यापारे एवाश्रयान्वयाच्च~।
अबोधीत्यत्रापि बुध्यतेः कर्तरि लुङ्, तस्य "दीपजन"\footnote{३-१-६१} इति चिण् , `चिणो लुक्' {६-४-१०४} इति तस्य लुग् इति साधनादित्याशङ्कायामाह-
\begin{center}
{\bfseries उत्सर्गोऽयं कर्मकर्तृविषयादौ विपर्ययात्~।\\
 तस्माद् यथोचितं ज्ञेयं द्योतकत्वं यथागमम् ॥४॥}
\end{center}
 कर्मकर्तृविषयादौ = पच्यते ओदनः स्वयमेव, इत्यादौ~।
अत्र ह्येकौदनाभिन्नाश्रयकः पाकानुकूलो व्यापार इति बोधः~।
`क्रमात्' इति आदिपदग्राह्यम्~।
अत्र `सामान्यविशेषज्ञानपूर्वक-एकनारदविषयकज्ञानाऽनुकूलः कृष्णाभिन्नाश्रयकोऽतीतो व्यापारः' इति बोधः~।
यथोचितमिति = सकर्मकधातुसमभिव्याहृतभावसाधारणविधिविधेयचिण्यगादि कर्मद्योतकमिति भावः ॥४॥\par
एवं सूचीकटाहन्यायेन सोपपत्तिकं वाक्यार्थमुपवर्ण्य `फलव्यापारयोः' इति प्रतिज्ञातं धातोर्व्यापारवाचित्वम् , लडाद्यन्ते भावनाया अवाच्यत्वं वदतः प्राभाकरादीन् प्रति व्यवस्थापयति-
\begin{center}
{\bfseries व्यापारो भावना सैवोत्पादना सैव च क्रिया~।\\
कृञोऽकर्मकतापत्तेर्न हि यत्नोऽर्थ इष्यते ॥५॥}
\end{center}
पचति = पाकमुत्पादयति, पाकानुकूला भावना, तादृश्युत्पादना इति विवरणाद् विव्रियमाणस्यापि तद्वाचकतेति भावः~।
व्यापारपदं च फूत्कारादीनामयत्नानामपि फूत्कारत्वादिरूपेण वाच्यतां ध्वनयितुमुक्तम्~।
अत एव पचतीत्यत्र अधःसन्तापनत्व-फूत्कारत्व-चुल्ल्युपरिधारणत्वयत्नत्वादिभिर्बोधः सर्वसिद्धः~।
न चैवमेषां शक्यताऽवच्छेदकत्वे गौरवापत्त्या कृतित्वमेव तदवच्धेदकं वाच्यम् , `रथो गच्छति' जानातीत्यादौ च व्यापारत्वादिप्रकारकबोधो लक्षणयेति नैयायिकरीतिः साध्वी, शक्यताऽवच्छेदकत्वस्यापि लक्ष्यताऽवच्छेदकत्ववद् गुरुणि सम्भवात् , तयोर्वैषम्ये बीजाभावात्~।
न च पचति = पाकं करोतीति यत्नार्थककरोतिना विवरणाद् यत्न एवाऽऽख्यातार्थ इति वाच्यम् ,
`रथो गमनं करोति', ‘बीजादिना अङ्कुरः कृतः' इति दर्शनात् कृञो यत्नार्थकताया असिद्धेः~।
किञ्च, भावनाया अवाच्यत्वे घटं भावयतीत्यत्रेव घटो भवतीत्यत्रापि द्वितीया स्यात्~।
न चात्र घटस्य कर्तृत्वेन तत्संज्ञया कर्मसंज्ञाया बाधान्न द्वितीयेति वाच्यम् , अनुगतकर्तृत्वस्य त्वन्मते दुर्वचत्वेन घटस्याकर्तृत्वात्~।
कृत्याश्रयत्वस्य, कारकचक्रप्रयोक्तृत्वस्य वा घटादावभावात् , धात्वर्थानुकूलव्यापाराश्रयत्वस्य च कारकमात्रातिव्यापकत्वात्~।
अपि च भावनाया अवाच्यत्वे धातूनां सकर्मकत्वाकर्मकत्वविभाग उच्छिन्नः स्यात्~।
`स्वार्थफलव्यधिकरणव्यापारवाचकत्वम्', `स्वार्थव्यापारव्यधिकरणफलवाचकत्वम्' वा सकर्मकत्वं भावनाया वाच्यत्वमन्तरेणासम्भवि~।
`अन्यतमत्वं तत्त्वम्' इति चेन्न, एकस्यैवार्थभेदेनाकर्मकत्व-सकर्मकत्वदर्शनात्~।
तदेतदभिसन्धायाह - कृञ इति~।
अयं भावः - व्यापाराऽवाच्यत्वपक्षे फलमात्रमर्थ इति फलितम्~।
तथा च करोतीत्यादौ यत्नप्रतीतेस्तन्मात्रं वाच्यमभ्युपेयम्~।
तथा च `यति प्रयत्ने' इतिवत् फलस्थानीययत्नवाचकत्वाविशेषादकर्मकतापत्तिरुक्तरीत्या दुर्वारेति~।
तथा च `न हि यत्नः' इत्यत्र फलस्थानीयत्वेनेति शेषः~।
कृञ इति धातुमात्रोपलक्षणम् , सर्वेषामप्यकर्मकता, सकर्मकता वा स्यादिति भावः~।
 अथवा, `व्यापारो भावना' इत्यर्धेन व्यापारस्य वाच्यत्वं प्रसाध्य फलांशस्यापि तत् साधयन् नैयायिकाभ्युपगतं जानाति-करोत्यादेः केवलज्ञान-यत्नादिक्रियामात्रवाचित्वं दूषयति - कृञ इति~।
अयं भावः - फलांशस्यावाच्यत्वे व्यापार एव धात्वर्थः स्यात्~।
तथाच स्वार्थफलव्यधिकरणव्यापारवाचित्वादिरूपसकर्मत्वोच्छेदापत्तिः~।
न च कृञादौ सकर्मकत्वव्यवहारो भाक्त इति नैयायिकोक्तं युक्तम् , व्यावहारस्य भाक्तत्वेऽपि कर्मणि लकारासम्भवात्~।
नहि तीरे गङ्गापदस्य भाक्तत्वेऽपि तेन स्नानादिकार्य्यं कर्तुं शक्यम्~।
एवञ्च `नहि यत्नः' इत्यत्र यत्नमात्रमित्यर्थः ॥५॥\par
अत एवाह-
\begin{center} 
{\bfseries किन्तूत्पादनमेवातः कर्मवत् स्याद् यगाद्यपि~।\\
 कर्मकर्तर्य्यन्यथा तु न भवेत् तद् दृशेरिव ॥६॥}
\end{center}
उत्पादनम् = उत्पत्तिरूपफलसहितं यत्नादि कृञर्थ इत्यर्थः~।
फलस्य वाच्यत्वे युक्त्यन्तरमाह- अत इत्यादि~।
यतः कृञो यत्नमात्रमर्थो नेष्यते, अतः `कर्मवत् स्यात्' इतिपदेन `कर्मवत् कर्मणा तुल्यक्रियः'\footnote{३.१.८७} इति सूत्रं लक्ष्यते~।
अयमर्थः- यत एवास्योत्पादनार्थकता, अतः पच्यते ओदनः स्वयमेवेतिवत् , क्रियते घटः स्वयमेवेति यगादयोऽप्युपपद्यन्ते~।
अन्यथा यत्नस्य कर्मनिष्ठत्वाभावात् तन्न स्यात् , दृशिवत्~।
यथा दृश्यते घटः स्वयमेवेति न, दर्शनस्य घटावृत्तित्वात् यथा यत्नस्यापि, इति तथा प्रयोगानापत्तेरिति ॥६॥\par
नन्वेवं कृञादेरिव जानातीत्यादेरपि विषयावच्छिन्नावरणभङ्गादिफलवाचित्वमावश्यकम्, अन्यथा सकर्मकतानापत्तेः~।
तथाच ज्ञायते घटः स्वयमेवेति किन्न स्यात्~।
एवं ग्रामो गम्यते स्वयमेवेत्याद्यपीत्याशङ्कां मनसि कृत्वाह-
\begin{center}{\bfseries निर्वर्त्ये च विकार्य्ये च कर्मवद्भाव इष्यते~।\\
 न तु प्राप्ये कर्मणीति सिद्धान्तोऽत्र व्यवस्थितः ॥७॥}
\end{center}
 ईप्सितं कर्म त्रिविधम् - निर्वर्त्यम् , विकार्यम् , प्राप्यञ्च~।
तत्राद्ययोः कर्मवद्भावः, नान्त्ये~।
प्राप्यत्वञ्च - क्रियाकृतविशेषानुपलभ्यमानत्वमिति सुबर्थनिर्णये वक्ष्यते~।
न ह्ययं घटः केनचिद् दृष्टो ग्रामोऽयं केनचिद् गत इति शक्यं कर्मदर्शनेनाऽवगन्तुम्~।
घटं करोतीति निर्वर्त्त्ये, सोमं सुनोतीति विकार्य्ये च तज्ज्ञातुं शक्यमिति न तत् प्राप्यम्~।
तथाच घटादेर्दृश्यादौ प्राप्यकर्मत्वान्नोक्तातिप्रसङ्ग इति भावः~।
धातूनां फलावाचकत्वे त्यजिगम्योः पर्य्यायतापत्तिः, क्रियावाचकत्वाविशेषात्~।
फलस्योपलक्षणत्वेऽप्येकक्रियाया एव पूर्वदेशविभागोत्तरदेशसंयोगजनकत्वादुक्तदोषतादवस्थ्यमित्यपि वदन्ति~।
तस्मादावश्यकं सकर्मकाणां फलवाचकत्वम्~।
अकर्मकाणां तु तन्निर्विवादमेव, "भू सत्तायाम्" इत्याद्यनुशासनाच्च~।
अत एव `द्व्यर्थः पचिः' इति भाष्यं सङ्गच्छते इति दिक् ॥७॥\par


एवं सिद्ध्यतु फलव्यापारयोर्वाच्यत्वम् , किन्तु आख्यातवाच्यैव सा भावना, न धातोः, प्राधान्येन प्रतीयमानस्य व्यापारस्य धात्वर्थतायाः 'प्रकृतिप्रत्ययार्थयोः प्रत्ययार्थस्य प्राधान्यम्' इतिन्यायविरुद्धत्वात् `तदागमे हि दृश्यते' इति न्यायविरुद्धत्वाच्च~।
एवञ्च `स्वायुक्ताख्यातार्थव्यापारव्यधिकरणफलवाचकत्वं सकर्मकत्वम् ,आख्यातार्थव्यापाराश्रयत्वञ्च कर्तृत्वं वाच्यम्' इत्यादिवदन्तं मीमांसकम्मन्यं प्रत्याह-
\begin{center}{\bfseries तस्मात् करोतिर्धातोः स्याद् व्याख्यानं नत्वसौ तिङाम्~।\\
पक्ववान् कृतवान् पाकं किं कृतं पक्वमित्यपि ॥८॥}
\end{center}
 तस्मात् = अभिप्रायस्थहेतोः, स चेत्थम्-फलमात्रस्य धात्वर्थत्वे `ग्रामो गमनवान्' इति प्रयोगापत्तिः, संयोगाश्रयत्वात्~।
फलानुत्पाददशायां व्यापारसत्त्वे `पाको भवति' इत्यनापत्तिः, व्यापारविगमे फलसत्तवे पाको विद्यत इत्यापत्तिश्च~।
 यत्तु, भावप्रत्ययस्य घञादेरनुकूलव्यापारवाचकत्वान्नानुपपत्तिरिति, तन्न, कर्त्राख्यातवत् `कर्तरि कृत्' इत्यत एव तद्विधानलाभे भावे विधायकाऽनुशासनवैयर्थ्यापत्तेः, तद्विरोधापत्तेश्च~।
अथ व्यापारोऽपि धात्वर्थ इत्यभ्युपेयमिति चेत्तर्हि धातुत एव सकलव्यापारलाभसम्भवेनाख्यातस्य पृथक्शक्तिकल्पने गौरवमिति~।
पचतीत्यस्य पाकं करोतीति विवरणात्मा करोतिर्धातोरेव व्याख्यानम्=विवरणम्, अतस्तदपि नाख्यातार्थत्वसाधकमिति भावः~।
मीमांसकोक्तं बाधकमुद्धरँस्तन्मतं दूषयति-नत्वित्यादिना~।
 नाऽसौ तिङां व्याख्यानम्, `पक्ववान्' इत्यादावनन्वयापत्तेः~।
 अयं भावः-"प्रकृतिप्रत्ययौ सहार्थं ब्रूतस्तयोः प्रत्ययार्थः प्रधानम्" इत्यत्र हि `विशेष्यतया प्रकृत्यर्थप्रकारकबोधे{बोधम्प्रति)तदुत्तरप्रत्ययजन्योपस्थितिर्हेतुः' इति कार्य्यकारणभावः फलितः~।
तथाच पक्ववानित्यत्र पाकः कर्मकारकम्, क्तवतुप्रत्ययार्थः कर्तृकारकम्~।
तयोश्चारुणाधिकरणोक्तरीत्या, वक्ष्यमाणाऽस्मद्रीत्या चान्वयासम्भव इति प्रकृतिप्रत्ययार्थयोरन्वयनियमस्यौवाऽभावे क्व प्राधान्यबोधक उक्तकार्य्यकारणभावः~।
न च
\begin{center}
 सम्बन्धमात्रमुक्तञ्च श्रुत्या धात्वर्थभावयोः~।\\
 तदेकांशनिवेशे तु व्यापारोऽस्या न विद्यते॥
\end{center}
 इति भट्टपादोक्तरीत्या सम्बन्धसामान्येन कारकाणामन्वयः शङ्कयः, योग्यताविरहात् अन्वयप्रयोजकरूपवत्त्वस्यैव तथात्वात्~।
 क्रियात्वमेव हि कारकान्वयिताऽऽवच्छेदकमिति वक्ष्यते~।
तदेतदाविष्कर्तुंविवरणेन धात्वर्थक्तवत्वर्थयोः कर्मत्व-कर्तृत्वे दर्शयति-`कृतवान् पाकम्' इति~।
 वस्तुतस्तु `प्रत्ययार्थः प्रधानम्'इत्यस्य, यः प्रधानां स प्रत्ययार्थ एवेति वा, यः प्रत्ययार्थः स प्रधानमेवेति वा नार्थः, `अजा' `अश्वा' छागीत्यत्र स्त्रीप्रत्ययार्थे स्त्रीत्वस्यैव प्राधान्यापत्तेः, छाग्यादेरनापत्तेश्च~।
किन्तूत्सर्गोऽयम्~।
विशेष्यत्वादिना बोधस्तु तथा व्युत्पत्त्यनुरोधात्~।
अत एव नैयायिकानां प्रथमान्तविशेष्यक एव बोधः~।
लक्षणायामालङ्कारिकाणां शक्यतावच्छेदकप्रकारक एव बोधः, न नैयायिकादीनाम्~।
घटः, कर्मत्वम्, आनयनं कृतिरित्यादौ विपर्य्ययेणापि व्युत्पन्नानां नैयायिकनव्यादीनां बोधः, नतु तद्व्युत्पत्तिरहितानाम्~।
अन्येषां तन्निराकाङ्क्षमेवेत्यादीकं सङ्गच्छते~।
 अत एव " प्रधानप्रत्ययार्थवचनमर्थस्यान्यप्रमाणत्वात्"\footnote{१-२-५६} इत्याह भगवान् पाणिनिः~।
 `प्रधानं प्रत्ययार्थः' इति वचनम् {`न'इत्यनुवर्त्य) न कार्य्यम्, अर्थस्य=तथाबोधस्य, अन्यप्रमाणत्वात्=व्युत्पत्त्यनुसारित्वादिति हि तदर्थः~।
एवं सत्यपि नियामकापेक्षणे च "भावप्रधानमाख्यातम्" इति वचनमेव गृह्यतामिति सुधीभिरूह्यम्~।

 `तदागमे हि' इति न्यायो विवरणञ्चऽतिव्याप्तमित्याह-किं कृतं पक्वमिति~।
कृञा विवरणं प्रतीतिश्च पक्वमित्यत्रापि, इति तत्रापि भावना वाच्या स्यादिति भावः~।
 नन्वस्तु तिङ इव कृतामपि भावना वाच्येत्यत आह-अपीति-तथाचोभयसाधारण्येन तत्प्रतीतेरुभयसाधारणो धातुरेव वाचक इति भावः~।
भवद्रीत्या प्रत्ययार्थत्वात् प्राधान्यापत्तिश्चेति द्रष्टव्यम्॥८॥\par
 साधकान्तरमाह-
\begin{center}{\bfseries किं कार्य्यं पचनीयं चेत्यादि दृष्टं हि कृत्स्वपि~।\\
किञ्च क्रियावाचकतां विना धातुत्वमेव न॥९॥}
\end{center}
 `कार्य्यम्'इत्यत्र `ऋहलोर्ण्यत्'\footnote{३-१-१८४} इति कर्मणि ण्यत्~।
`पचनीयम्' इत्यादौ चानीयर्~।
आदिना 'ज्योतिष्टोमयाजी' इत्यादौ करणे उपपदे कर्तरि णिनिः~।
एते च क्रिययोगमन्तरेणासन्तस्तद्वाच्यतां बोधयन्ति, विना क्रियां कारकत्वासम्भवेन तद्वाचकप्रत्ययस्याप्यऽसम्भवात्~।
नच `गम्यमानक्रियामादाय कारकयोगः' इति भाट्टरीतिर्युक्ता, आख्यातेऽपि तथात्वापत्तौ तत्रापि भावनाया वाच्यत्वासिद्धयापत्तेः~।
 अथ लिङ्गसंख्यान्वयानुरोधात् कर्तुर्वाच्यत्वमावश्यकमिति तेनाऽक्षेपाद् भावनाप्रत्ययः स्यादिति मतम्, तर्हि संख्यान्वयोपपत्तये आख्यातेऽपि कर्ता वाच्यः स्यात्~।
`पक्ववान्' इत्यादौ कालकारकान्वयानुरोधाद् भावनाया अपि वाच्यत्वस्याऽऽवश्यकत्वाच्चेति भावः~।
 अपिना हेत्वन्तरसमुच्चयः~।
तथाहि- नखैर्भिन्नः `नखभिन्नः', हरिणा त्रातः `हरित्रातः' इत्यादौ, "कर्तृकरणे कृता बहुलम्"\footnote{२-१-३२} इति समासो न स्यात्, `पुरुषो राज्ञो भार्य्या देवदत्तस्य' इतिवद्, असामर्थ्यात्~।
नचाध्याहृतक्रियामादाय सामर्थ्यं वाच्यम्, `दध्योदनः' `गुडधानाः' इत्यादिवत्~।
अन्यथा तत्रापि "अन्नेन व्यञ्जनम्"\footnote{२-१-३४} "भक्ष्येण मिश्रीकरणम्"\footnote{२-१-३५} इति समासो न स्यादिति वाच्यम्, तत्र विध्यानर्थक्यादगत्या तथा स्वीकारेऽपि `हरिकृतम्' इत्यादौ साक्षाद्धात्वर्थान्वयेनोपपद्यमानस्य, "कर्तृकरणे" इत्यस्याक्षेपेण परम्परासम्बन्धे प्रवृत्त्ययोगात्~।
 नचैकक्रियान्वयित्वमेव सामर्थ्यमिति शङ्कयम्, `असूर्यम्पश्याः' इत्यादेरसमर्थसमासत्वानापत्तेः~।
इष्टापत्तौ कृतः सर्वो मृत्तिकयेत्यत्र 'कृतसर्वमृत्तिकः' इत्यापत्तेः~।
नचाऽत्र समासविधायकाभावः, "सह सुपा"\footnote{२-१-४} इत्यस्य सत्त्वात्~।
अन्यथा असमर्थसमासोऽपि विधायकाभावान्न स्यादिति~।
 किञ्च, भावनायास्तिङर्थत्वे `भावयति घटम्' इतिवद् `भवति घटम्' इत्यपि स्यात्, धात्वर्थफलाश्रयत्वरूपकर्मत्वसत्त्वात्~।
नचाख्यातार्थव्यापाराश्रयत्वेन कर्तृत्वात्तत्संज्ञया कर्मसंज्ञाया बाधान्न द्वितियेति वाच्यम्, आख्यातार्थव्यापाराश्रयस्य कर्तृत्वे `पाचयति देवदत्तो विष्णुमित्रेण' इत्यत्र विष्णुमित्रस्याकर्तृतापत्तौ तृतीयानापत्तेः, `ग्रामं गमयति देवदत्तो विष्णुमित्रम्' इत्यत्र विष्णुमित्रस्याकर्तृतापत्तौ ग्रामस्य गमिकर्मत्वानापत्तेश्च~।
तथाच ग्रामाय गमयति देवदत्तो विष्णुमित्रम्, इत्यपि न स्यात्, "गत्यर्थकर्माणि द्वितीयाचतुर्थ्यौ चेष्टायामनध्वनि"\footnote{२-३-१२} इति गत्यर्थकर्मण्येव चतुर्थोविधानात्~।
एतेन णिजन्ते आख्यातार्थ उभयम्,तदाश्रयत्वाद्देवदत्त-यज्ञदत्तयोः कर्तृतेत्यपास्तम्~।
 किञ्च, `तस्मिन् प्रयोगे य आख्यातार्थः' इत्यस्यावश्यकत्वेनाऽऽख्यातशुन्ये-`देवदत्तः पक्ता' इत्यादौ देवदत्तस्याऽकर्तृतापत्तेरिति दिक्~।
 सूत्रानुपपत्तिमपि मानत्वेन प्रदर्शयन्नुक्तार्थस्य स्वोत्प्रेक्षितत्वं निरस्यति- किञ्चेति~।
धातुसंज्ञाविधायकम्-"भूवादयो धातवः"\footnote{१-३-१} इति सूत्रम्~।
तत्र भूश्च वाश्चेति द्वन्द्वः~।
आदिशब्दयोर्व्यवस्थाप्रकारकवाचिनोरेकशेषः~।
आदिश्च आदिश्च आदी, ततो भूवौ आदी येषां ते भूवादयः~।
तथाच भूप्रभृतयो वासदृशा धातव इत्यर्थः~।तच्च क्रियावाचकत्वेन~।
तथाच `क्रियावाचकत्वेन~।
तथाच 'क्रियावाचकत्वे सति भ्वादिगणपठितत्वं धातुत्वम्' पर्य्यवसन्नम्~।
अत्र क्रियावाचित्वमत्रोक्तौ वर्ज्जनादिरूपक्रियावाचके `हिरुक्' `नाना' इत्यादावतिव्याप्तिरिति `भ्वादिगणपठितत्वम्' उक्तम्॥९॥\par
तावन्मात्रोक्तौ चाऽह-
\begin{center}{\bfseries सर्वनामाव्ययादीनां यावादीनां प्रसङ्गतः~।\\
 नहि तत्पाठमात्रेण युक्तमित्याकरे स्पुटम्॥१०॥}
\end{center} 
गणपठितत्वमात्रोक्तौ सर्वनामाव्ययानामपि धातुत्वं स्यात्~।
तथाच- `याः पश्यसि' इत्यादौ " आतो धातोः"\footnote{६-४-२४०} इत्याकारलोपापत्तिः~।
ननु लक्षण-प्रतिपदोक्तयोः प्रतिपदोक्तस्यैव ग्रहणान्न सर्वनाम्रो ग्रहणम्, तस्य लाक्षणिकत्वादित्यत आह-वेत्यादि~।
 अव्यये=`वा' इत्यादावतिप्रसङ्गः तादृशस्यैव गणे पाठेन निर्णयासम्भवात्~।
तथाय विकल्पार्थकः `वाति'इति प्रयोगः स्यादिति भावः~।
नच गतिगन्धनाद्यर्थोनर्द्देशो नियामकः, तस्य "अर्थानादेशनात्" इति भाष्यपर्य्यालोचनया आधुनिकत्वलाभात्॥१०॥\par
नन्वस्तु `क्रियावाचकत्वे सति गणपठितत्वं धातुत्वम्', क्रिया च धात्वर्थ एव, न व्यापार इत्याशङ्कां समाधत्ते-
\begin{center}{\bfseries धात्वर्थत्वं क्रियात्वञ्चेद्धातुत्वं च क्रियार्थता~।\\
 अन्योऽन्यसंश्रयः स्पष्टस्तस्मादस्तु यथाऽऽकरम्॥११॥}
\end{center}
यदि क्रियात्वं धात्वर्थत्वमेव तर्हि धातुत्वग्रहे तदर्थत्वरूपक्रियात्वग्रहः, क्रियात्वग्रहे च तदवच्छिन्नवाचकत्वघटितधातुत्वग्रह इत्यन्योन्याश्रय इति ग्रहपदं पूरयित्वा व्याख्येयम्~।
यथाश्रुते चान्योन्याश्रयस्योत्पत्तौ ज्ञप्तौ वा प्रतिबन्धकत्वाभ्युपगमेनासङ्गत्यापत्तेः~।
 नच `अन्यतमत्वं धातुत्वम्', "भूवादयः" इत्यस्य वैयर्थ्यापत्तेरित्यभिप्रेत्याह - अस्त्विति~।
व्यापारसन्तानः क्रिया, तद्वाचकत्वे सति गणपठितत्वमित्यर्थः~।
ननु सत्तादीन् फलांशानन्यतमत्वेनादाय तद्वाचकत्वे सति सणपठितत्वं लक्षणमुच्यताम्~।
धात्वर्थत्वात् तेषां क्रियाशब्देन व्यवहारो भाष्यादौ कृतोऽप्युपपत्स्यत इति चेन्न, अन्यतममध्ये विकल्पस्यापि `विकल्पयति'इति प्रयोगानुसारात् प्रवेशावश्यकत्वेन तदर्थके `वा
' इत्यव्यये उक्तरीत्या गणपठितत्वसत्त्वेनाऽतिव्याप्तेरिति ॥११॥\par
 नन्वस्यैव धातुत्वे `आस्ति' इत्यादौ क्रियाप्रतीत्यभावादस्त्यादीनां तदवाचकानामधातुत्वप्रसङ्ग इत्यत आह-
\begin{center}{\bfseries अस्त्यावपि धर्म्यंशे भाव्येऽस्त्येव हि भावना~।\\
 अन्यत्राशेषभावात्तु सा तथा न प्रकाशते॥१२॥}
\end{center} 
 अस्त्यादौ = `अस् भुवि' इत्यादौ, धर्म्यंशे = धर्म्मिभागे, भाव्ये = भाव्यत्वेन विवक्षिते, अस्त्येव = प्रतीयत एव~।
अयमर्थः- `स ततो गतो न वा' इति प्रश्ने, महता यत्नेन`अस्ति' इति प्रयोगे सत्तारूपफलानुकूला भावना प्रतीयत एव~।
 उत्पत्त्यादिबोधने तु सुतराम्~।
 \begin{center}रोहितो लोहितादासीद् धुन्धुस्तस्य सुतोऽभवत्~।
\footnote{रामायणम् }
\end{center}इत्यादिदर्शनात्~।
 किञ्च, अत्र भावनाविरहे लडादिव्यवस्था न स्यात्, तस्या एव वर्त्तमानत्वादिविवक्षायां तद्विधानात्~।
\begin{center} "क्रियाभेदाय कालस्तु सङ्खया सर्वस्य भेदिका"\end{center}
इति वाक्यपदीयादिति~।
 नन्वेवम् अस्ति इत्यत्र स्पष्टं कुतो न बुद्धयत इत्यत आह- अन्यत्रेति~।
 अशेषभावात् = भावनायाः फलसमानाधिकरणत्वात्~।
तथाच भावनायाः फलसामानाधिकरण्यं तत्स्पष्टत्वे दोष इति भावः~।
 नन्वेवम् `किं करोति' इति प्रश्रे, `पचति' इत्युक्तरस्येव `अस्ति' इत्युत्तरमपि स्यादिति चेत्, इष्टापत्तेः, आसन्नविनाशं कञ्चिदुद्दिश्य `किं करोति' इति प्रश्रे, पचतीत्युत्तरस्येव, `अस्ति' इत्युत्तरस्य सर्वसम्मतत्वात्~।
इतरत्र तु सुस्थतया निश्चिते `किं करोति' इति प्रश्नः, पाकादिविशेषगोचर एवेत्यवधारणात् `अस्ति' इति नोत्तरमिति॥१२॥\par
 नन्वेवं भावनायाः फलनियतत्वात्, पलाश्रयस्य च कर्मत्वात्, सर्व्वेषां क्रियावाचकत्वे सकर्मकतापत्तिरित्यत आह-
\begin{center}{\bfseries फलव्यापारयोरेकनिष्ठतायामकर्मकः~।\\
 धातुस्तयोर्धर्म्मिभेदे सकर्मक उदाहृतः॥१३॥}
\end{center}
एकनिष्ठतायाम् = एकमात्रनिष्ठतायां भिन्नाधिकरणावृत्तितायामिति यावत्~।
तेन गम्यादौ फलस्य कर्तृनिष्ठत्वेऽपि नातिव्याप्तिः~।
अकर्मको यथा-भ्वादिः~।
तयोः = फलव्यापारयोः आश्रयभेदे सकर्मक इत्यर्थः~।
उक्तञ्च वाक्यपदीये-
\begin{center} आत्मानमात्मना बिभ्रदस्तीति व्यापदिश्यते~।\\
 अन्तर्भावाच्च तेनासौ कर्मणा न सकर्मकः॥इति~।
\end{center}
 बिभ्रदिति-तेन स्वधारणानुकूलो व्यापारोऽत्रापि गम्यत इति भावः~।
तेन=कर्मणा सकर्मकत्वन्तु न, अन्तर्भावात्=फलांशेन सामानाधिकरण्यसत्त्वादित्यर्थः~।
`आत्मानं जानाति' `इच्छति' इत्यादौ च द्वावात्मनौ=शरीरात्मा, अन्तरात्मा च~।
तत्रान्तरात्मा तत्कर्म करोति येन शरीरात्मा सुखदुःखे अनुभवतीति "कर्मवत्कर्मणा"\footnote{३-१-८७} इति सूत्रीयभाष्योक्तरीत्या भिन्नादिकरणनिष्ठतामादाय सकर्मकत्वमित्यवधेयम्॥१३॥\\
 नन्वसत्त्वभूतक्रियाया धात्वर्थत्वे `पाकः' इत्यत्रापि तत्प्रत्ययापत्तिः~।
नचेष्टापत्तिः, " कृदभिहितो भावो द्रव्यवत् प्रकाशते" इति भाष्यविरोधादित्यत आह-
\begin{center}{\bfseries आख्यातशब्दे भागाभ्यां साध्यसादनरूपता~।\\
 प्रकल्पिता यथा शास्त्रे सा घञादिष्वपि क्रमः॥१४॥}
\end{center} 
 आख्यातशब्दे = `पश्य मृगो धावति' इत्यादौ, भागाभ्यम् = तिङन्ताभ्याम् , प्रकृतीप्रत्ययभागाभ्यामिति विवरणकारोक्तमपव्याख्यानम् , पचतीत्यत्राऽपि भागद्वयसत्त्वात्~।
साध्यसाधनरूपता यथाक्रमं ग्राह्या~।
साध्यत्वम् - क्रियान्तराकाङ्क्षानुत्थापकतावच्छेदकरूपवत्त्वम्~।
साधनत्त्वम् - कारकत्वेनान्वयित्वम्~।
स घञादिष्वपीति - प्रकृत्या साध्यावस्था, प्रत्ययेन साधनावस्था~।
इयान् परं विशेषः - घञाद्युपस्थाप्या लिङ्गसङ्खयान्वयिनी, कारकान्वयिनी च~।
आख्यातान्तोपात्ता तु नैवम्~।
तथापि कारकत्वेनान्वयित्वमात्रेण दृष्टान्तदार्ष्टन्तिकतेत्यवधेयम्~।
 नच घञन्ते धातुना तथाभिधाने मानाभावः, ओदनस्य पाकः' इति कर्मषष्ठया मानत्वात्~।
नचाध्याहृततिङन्तक्रियान्वयात् षष्ठी, "कर्तृकर्मणोः कृति"\footnote{२-३-६४} इति कृदन्तेन योग एव तद्विधानात्, "न लोकाव्ययनिष्ठाखलर्थतृनाम्"\footnote{२-३-६९} इति लादेशयोगे षष्ठीनिषेधाच्च~।
एवंरीत्या 'काष्ठैः पाकः' इत्याद्यपीष्टमेव~।
एवं फलांशोऽपि धातुना असत्त्वावस्थापन्न एवोच्यते~।
 अत एव `स्तोकं पचति' इतिवत् `स्तोकं पाकः' इत्युपपद्यते इति ॥१४॥\par
एतदेव स्पष्टयति-
\begin{center}{\bfseries साध्यत्वेन क्रिया तत्र धातुरूपनिबन्धना~।\\
 सिद्धभावस्तु यस्तस्याः स घञादिनिबन्धनः॥१५॥}
\end{center}
 न च घञादिभिः सिद्धत्वेनाभिधाने मानाभावः, `पाकः' इत्युक्ते, भवति नष्टो वेत्याद्याकाङ्क्षोत्थापनस्यैव मानत्वात्,धातूपस्थाप्यायां तदसम्भवस्योक्तत्वात्, 'स्तोकः पाकः' इत्यनापत्तेश्च~।
तस्माद्धात्वर्थान्वये स्तोकादिशब्देभ्यो द्वितीया~।
घञर्थान्वये प्रथमा, पुँल्लिङ्गता चेति - तत्सिद्धये घञादेः शक्तिरुपेया~।
एतेन घञादीनां प्रयोगसाधुतामात्रमिति नैयायिकनव्योक्तमपास्तम्~।
 न च घञन्तशक्त्युपस्थाप्यान्वये `स्तोकः पाकः' इति भवतीति वाच्यम्, घञन्तानुपूर्व्याः शक्ततावच्छेदकत्वे गौरवादनुशासनाच्च घञादेरेव तथा शक्तिकल्पनादिति दिक्~।
 एवञ्च घञ्सक्तयभिप्रायेण `कृदभिहितः' इति भाष्यम्, अतो न तद्विरोध इति भावः ॥१५॥\par
 ननु कारकाणां भावनान्वयनियम एव `पाकः' इत्यत्रापि कर्मषष्ठयनुसारेण भावनाया वाच्यत्वं सिद्ध्येत्~।
स एव कुत इत्याशङ्कां समाधत्ते-
\begin{center}{\bfseries सम्बोधनान्तं कृत्वोऽर्थाः कारकं प्रथमो वतिः~।\\
 धातुसम्बन्धाधिकारनिष्पन्नमसमस्तनञ्॥१६॥}
\end{center}
सम्बोधनान्तस्य क्रियायामन्वयः, `त्वं ब्रूहि देवदत्त' इत्यादौ निघातानुरोधात्, "समानवाक्ये निघातयुष्मदस्मदादेशाः" इत्यनेन समानवाक्ये एव तन्नियमात्~।
उक्तञ्च वाक्यपदीये-
\begin{center} सम्बोधनपदं यच्च तत् क्रियायां विशेषणम्~।\\
 व्रहानि देवदत्तेति निघातोऽत्र तथा सति॥इति॥
\end{center} `पचति भवति देवदत्तः' इत्यादौ तु सूत्रभाष्यादिरीत्यैकवाक्यतासत्त्वात् स्यादेव निघातः, "तिङ्ङतिडः\footnote{७-१-८८} इति सूत्रयता तिङ्न्तानामप्येकवाक्यतास्वीकारात्~।
`एकतिङ् वाक्यम्' इति वदतां
 वार्त्तिककाराणां मते परं न~।
वस्तुतस्तु `एकतिङ्-विशेष्यकं वाक्यम्' इति तदभिप्रायस्य हेलाराजीयादौ, वैयाकरणभूषणेऽस्माभिश्च प्रतिपादितत्वत्तन्मतेऽपि भवत्येवेत्यवधेयम्॥कृत्वोऽर्थाः,
 "संख्यायाः क्रियाभ्यावृत्तिगणने कृत्वसुच्"\footnote{५-४-१७} इति क्रियायोगे तत्साधुत्वोक्तेः, क्रियाया अभ्यावृत्तिः=पुनः पुनर्जन्म, तस्मिन् द्योत्ये इति तदर्थात्~।
 कारकम्, `कारके' इत्यधिकृत्य तेषां व्युत्पादनात्~।
कारकशब्दो हि क्रियापरः, करोति कर्तृकर्मादिव्यपदेशानिति व्युत्पत्तेः~।
तथाचाग्रिमेष्वपादानादिसंज्ञाविधिषु क्रियऽर्थककारकशब्दानुवृत्त्या क्रियान्वयिनामेव संज्ञेति भाष्ये स्पष्टम्~।
 प्रथमो वतिः, "तेन लुल्यं क्रिया चेद्वतिः"\footnote{५-१-११४} इति विहितः, तत्र यत्तुल्यं सा क्रिया चेदित्युक्तत्वात्~।
धातुसम्बन्धाधिकारे, "धातुसम्बन्धे प्रत्ययाः"\footnote{३-४-१} इत्यधिकृत्य तेषां विधानात्~।
 असमस्तनञ् - समासायोग्यः प्रसज्यप्रतिषेधीयो नञित्यर्थः, उत्तरपदार्थान्वयेऽपि समासविकल्पेन पक्षेऽसमस्तत्वाद् यथाश्रुतग्रहणायोगात्~।
 नचासमस्तनञः क्रियान्वये मानाभावः, न त्वं पचसि, न युवां पचयः, चैत्रो न पचति, घटो न जायते इत्यादौ क्रियाया एव निषेधप्रतीतेः~।
अत एव विद्यमानेऽपि घटे तादृशप्रयोगः~।
 तथाच `घटो नास्ति' इत्यत्राप्यस्तित्वाभाव एव बोध्यते~।
नहि घटो न जायते, नास्तीत्यनयोर्धात्वर्थभेदान्तरेणास्ति विशेषः~।
तथाच `भूतले न घटः' इत्यत्राप्यस्तीत्यध्यध्याहार्य्यम्, प्रकारतासम्बनाधेन नञर्थविशेष्यकबोधे धातुजन्यभावनोपस्थितेर्हतुत्वस्य क्लृप्तत्वात्~।
शेषं नञर्थनिर्णये (निरूपणे) वक्ष्यते ॥१६॥\par
\begin{center}{\bfseries तथा यस्य च भावेन षष्ठी चेत्युदितं द्वयम्~।\\
 साधुत्वमष्टकस्यास्य क्रिययैवावधार्य्यताम्॥१७॥}\end{center}
 "यस्य च भावेन भावलक्षणम्"\footnote{२-३-३७} इत्यत्र भावनार्थकभावशब्देन तद्योगे साधुत्वाख्यानलाभात्~।
"पष्ठी चानादरे"\footnote{२-३-३८} इति तदग्रिमसूत्रे~।ऽपि चकाराद्यस्य च भावेनेत्यातीत्यर्थः~।
 साधुत्वमिति - तत्स्वरूपं तु वक्ष्यते~।
 क्रिययैवेति- अयं भावः- भूवादिसूत्रादिषु प्रायशः क्रियाशब्देन भावनाव्यपदेशात्, तत्र तस्य साङ्केतिकी शक्तिः~।
फलांशे क्वाचित्कः-क्रियत इति यौगिकः प्रयोगः~।
तथाच संज्ञाशब्दस्यानपेक्ष्य प्रवृत्तत्वेन बलवत्त्वाद्भावनान्वय एव साधुता लभ्यते~।
अत एव संज्ञाशब्दप्राबल्यात् `रथन्तरमुत्तराग्रन्थपठितऋक्ष्वेव गेयम्, नतु वेदे तदुत्तरपठयमानऋक्षु' इति नवमे निर्णीतम्~।
 किञ्च फलांशोऽपि भावनायां विशेषणम्, कारकाण्यपि क्वचित्तथाभूतानि, इति "गुणानाञ्च परार्थत्वादसम्बन्धः समत्वात् स्यात्"\footnote{जै.सू.३-१-२२} इति न्यायेन सर्वे सेवका राजानमिव भावनायामेव परस्परनिरपेक्षाण्यन्वियन्ति~।
'नहि भिक्षुको भिक्षुकान्तरं याचितुमर्हति सत्यन्यस्मिन्नभिक्षुके' इति न्यायेनापि फलं त्यक्त्वा भावनायामेवान्वियन्तीति मीमांसका अपि मन्वते~।
एवञ्च `विशेष्यतया कारकादिप्रकारबोधं प्रति धातुजन्यभावनोपस्थितिर्हेतुः' इति कार्य्यकारणभावस्य क्लृप्तत्वात्~।
यत्रापि `पक्ता' `पाचकः' इत्यादौ भावना गुणभूता, तत्रापि क्लृप्तकार्य्यकारणभावानुरोधात् तस्यामेवान्वय इत्यवसीयते, इत्यादि भूषणे प्रपञ्चितम्~।
केचितु भूतले घटः देवदत्तो घटमित्यादावन्वयबोधाकाङ्क्षनिवृत्त्योरदर्शनान्न तव्द्यतिरेकेण साधुत्वलाभ इत्याहुः॥१७॥\par
\begin{center}{\bfseries स्वयमुपपत्तिमाह-यदि पक्षेऽपि वत्यर्थः कारकञ्च नञादिषु~।\\
अन्वेति त्यज्यतां तर्हि चतुर्थ्याः स्पृहिकल्पना॥१८॥}
\end{center}
 `पर्वतो वह्निमान्, धूमात्, महानसवत्', भूतले न घटः `भूतले घटः' इत्यादिपादात्~।
एवमादिष्वनुशासनविरोधेऽपि यदि साधुत्वमन्वयबोधश्चाभ्युपेयते तर्हि चतुर्थ्याः स्पृहिकल्पनाऽपि त्यज्यतामित्यर्थः~।
अनुशासनानुरोधतौल्ये अर्द्धजरतीयमयुक्तमिति भावः॥१८॥\par
 एवं कर्त्रादौ विहितानामिन्यादीनां क्रिययैवान्वय इत्याह-
\begin{center}{\bfseries अविग्रहा गतादिस्था यथा ग्रामादिकर्मभिः~।\\
 क्रिया सम्बन्ध्यते तद्वत् कृतपूर्व्यादिषु स्थिता॥१९॥}
\end{center}
 न विविच्य ग्रहः = ग्रहणं यस्याः सा अविग्रहा = गुणीभूतेति यावत्~।
यथाच `ग्रामं गतः' इत्यत्र क्तप्रत्ययार्थगुणीभूतापि क्रिया ग्रामादिकर्मभिः सम्बध्यते, तथा `कृतपूर्वो कटम्' इत्यत्रापि गुणभूता इन्यादिभिरित्यर्थः~।
 नच वृत्तिमात्रे समुदायशक्तेर्वक्ष्यमाणत्वात् तत्रान्तर्गता भावना पदार्थैकदेश इति कथं तत्रान्वय इति वाच्यम्, नित्यसापेक्षेष्वेकदेशेऽपि `देवदत्तस्य गुरुकुलम्', `चैत्रस्य नप्ता' इत्यादाविवान्वयाभ्युपगमात्~।
एवं भोक्तुं पाकः, भुक्तवा पाक इत्यत्रापि द्रष्टव्यम् ॥१९॥\par
 अतिप्रसङ्गमाशङ्कय समाधत्ते-
\begin{center}
{\bfseries कृत्वोऽर्थाः क्त्वातुमुन्वत्स्युरिति चेत् सन्ति हि क्वचित्~।\\ अतिप्रसङ्गो नोद्भाव्योऽभिधानस्य समाश्रयात्॥२०॥}
\end{center}
 `भोक्तुं पाकः' भुक्तवा पाक इत्यादौ, "तुमुन्ण्वुलौ क्रियायां क्रियार्थायाम्"\footnote{३-३-१०}, "समानकर्तृकयोः पूर्वकाले"\footnote{३-४-२१} इति क्रियावाचकोपपदे क्रिययोः पूर्वोत्तरकाले विधीयमाना अपि तुमुन्नादयो गुणभूतां तामादाय यथा जायन्ते, तथा कृत्वोऽर्था अपि स्युः~।
`एकः पाकः' इत्यत्र, "एकस्य सकृच्च"\footnote{५-४-१९} द्वौ पाकौ, त्रयः, चत्वार इत्यत्र "द्वित्रिचतुर्भ्यः सुच्"\footnote{५-४-१८}~।
`पञ्च' इत्यत्र कृत्वसुच् स्यात्~।
तथाच `सकृत् पाकः' द्विस्त्रिश्चतुः पाका इत्याद्यापत्तिरिति चेदिष्टापत्तिः, "द्विर्वचनम्" इत्यादिदर्शनात्~।
 केचित्तु-"सङ्ख्यायाः क्रियाभ्यावृत्तिगणने" इत्यत्र क्रियाग्रहणं व्यर्थम्, तस्या एवाभ्यावृत्तिसम्भवेन सामर्थ्यात्तल्लाभात्~।
तथाच साध्यमात्रस्वभावक्रियालाभाय तदिति वाच्यम्, नच `पाकः', इत्यादौ तादृशीति नातिप्रसङ्गः, द्विर्वचनमिति च `द्विःप्रयोगो द्विर्वचनम्' इति व्युत्पत्त्या, "द्विर्वचनेऽचि"\footnote{१-१-५९} इति ज्ञापकं वा आश्रित्योपपादनीयमित्याहुः॥२०॥\par
 ननु सिद्धान्ते बोधकतारूपा शक्तिराख्यातशक्तिग्रहवतां बोधादावश्यकी, इति धातोरेव भावना वाच्या, नाऽऽख्यातस्येति कथं निर्णय इत्याशङ्कां समाधत्ते-
\begin{center}{\bfseries भेद्यभेदकसम्बन्धोपाधिभेदनिबन्धनम्~।\\
 साधुत्वं तदभावेऽपि बोधो नेह निवार्य्यते॥२१॥}
\end{center} 
 भेद्यम् = विशेष्यम्, भेदकम् = विशेषणं तयोर्यः सम्बन्धस्तस्य यो भेदस्तन्निबन्धनं साधुत्वम्~।
अयमर्थः- व्याकरणस्मृतिः शब्दसाधुत्वपरा तत्रैवावच्छेदकतया कल्प्यमानधर्म्मस्य शक्तित्वं वदतां मीमांसकानां पुनः शक्तत्वं साधुत्वमीत्येकमेवेति तद्रीत्या विचारे साधुत्वनिर्णये एव शक्तिनिर्णय उच्यते~।
 अतिरिक्तशक्तिवादेऽप्याख्यातानामसाधुता भावनायां स्यादेव~।
एवञ्च चतुर्थ्यर्थे तृतीयाप्रयोगवद्धात्वर्थभावनायामाख्यातप्रयोगे, याज्ञे कर्मण्यसाधुशब्दप्रयोगात् `नानुतं वदेत्' इति निषेधोल्लङ्घनप्रयुक्तं प्रायश्चित्तं दर्शनान्तरीयव्युत्पत्तिमतां स्यादेवेति~।
 ननु त्वन्मते `नानृतम्' इति निषेधः क्रत्वर्थ एव न सिद्धयेत् , आख्यातेन कर्तुरुक्तत्वाच्छ्रुत्या पुरुषार्थतैव स्यात्, प्रकारणाद्धि क्रत्वर्थता, तच्च श्रुतीविरोधे बाध्यते, इति चेन्न, `तिङर्थस्तु विशेषणम्' इत्यनेन परिहृतत्वात्~।
नहि गुणभूतः कर्तनिषेधं स्वाङ्गत्वेन ग्रहीतुमलम्~।
भावना तु प्रधानं तं ग्रहीतुं समर्थेति प्रकारणात् क्रत्वर्थतैव~।
अस्तु वा क्रतुयुक्तपुरुषधर्मः, अनुष्ठाने विशेषाभावात्~।
"जञ्जभ्यमानोऽनुब्रूयान्मयि दक्षक्रतू" इति वाक्योक्तमन्त्रविधिदित्यादि भूषणे प्रपञ्चितम्~।
 नन्वाख्यातस्य भावनायामसाधुत्वे ततस्तब्दोधो न स्यात्, साधुत्वज्ञानस्य शाब्दबोधहेतुत्वात्, इत्यत आह-बोध इति-असाधुत्वेऽपि साधुत्वभ्रमाद् बोधोऽस्तु नाम, अपभ्रंशवत्~।
असाधुत्वन्तु स्यादेवेति भावः~।
 वस्तुतः साधुत्वज्ञानं हेतुः, तद्व्यतिरेकनिर्णयोऽपि न प्रतिबन्धक इति `असाधुरनुमानेन' इत्यत्र वक्ष्यामः ॥२१॥
\begin{center} रङ्गोजिभट्टपुत्रेण कौण्डभट्टेन निर्म्मिते~।\\ पूर्णो भूषणसारेऽस्मिन् धात्वाख्यातार्थनिर्णयः ॥१॥\\
 इति श्रीकौण्डभट्टविरचिते वैयाकरणभूषणसारे धात्वर्थाऽऽख्यातसामान्यार्थयोर्निरूपणम् ॥१॥ \end{center}

\section*{\begin{center}अथ लकरार्थनिर्णयः\end{center}}
\addcontentsline{toc}{section}{लकरार्थनिर्णयः}
 \fancyhead[LE,RO]{लकरार्थनिर्णयः}
 प्रत्येकं दशलकाराणामर्थं निरूपयति-
\begin{center}{\bfseries वर्त्तमाने परोऽक्षे श्वो भाविन्यर्थे भविष्यति~।\\
 विध्यादौ प्रार्थनादौ च क्रमाज्ज्ञेया लडादयः॥२२॥}
\end{center}
 लडादयष्टितः षट् क्रमेणैष्वर्थेषु द्रष्टव्याः~।
तथाहि- वर्त्तमानेऽर्थे लट् "वर्त्तमाने लट्"\footnote{३-१-१८३} इति सूत्रात्~।
प्रारब्धापरिसमाप्तत्वम्, भूतभविष्यद्भिन्नत्वं वा वर्त्तमानत्वम्~।
`पचति' इत्यादावधिश्रयणाद्यधःश्रयणान्ते मध्ये तदस्तीति भवति लट्-प्रयोगः~।
 आत्माऽस्ति पर्वताः सन्तीयादौ तत्तत्कालिकानां राज्ञां क्रियाया अनित्यत्वात्तद्विशिष्टस्योत्पत्त्यादिकमादाय वर्त्तमानत्वमूह्यम्~।
उक्तं हि भाष्ये- "इह भूतभविष्यद्वर्त्तमानानां राज्ञां क्रियास्तिष्ठतेरधिकारणम्" इति,
\begin{center} परतो भिद्यते सर्वमात्मा तु न विकम्पते~।\\
 पर्वतादिस्थितिस्तस्मात् पररूपेण भिद्यते॥
 \end{center}
 इति वाक्यपदीये च~।
एवम् , "तम आसीत्" "तुच्छेनाभ्यपिहितं यदासीत्" "अहमेकः प्रभममासम्, वर्त्तमि च, भविष्यामि च" इत्यादिश्रुतयोऽपि योज्याः~।
 तच्च वर्त्तमानत्वादि लडादिभिर्द्योत्यते, क्रियासामान्यवाचकस्य तद्विशिष्टे लक्षणायां लडादेस्तात्पर्य्यग्राहकत्वेनोपयोगात्~।
अन्वयव्यतिरेकाभ्यां तद्रूपं लडादिवाच्यमेव~।
अन्यथा प्रत्ययानां वाचकत्वविलोपापत्तिरित्यपि पक्षान्तरम्~।
 लिडर्थमाह- परोक्षे इति- "परोक्षे लिट्"\footnote{३-२-११५} इति सूत्रात्~।
कालस्तावदद्यतनानद्यतनभेदेन द्विविधः~।
द्विविधोऽपि भूतभविष्यद्रूपः~।
तत्रानद्यतने भूते परोक्षे लिडित्यर्थः~।
तेनाद्यतने भूते, अनद्यतने भविष्यति, भूतेऽऽयपरोक्षे च न लिट्-प्रयोगः~।
परोक्षत्वञ्च 'साक्षात्करोमि' इत्येतादृशविषयताशालिज्ञानाविषयत्वम्~।
 नच "क्रिया नामेयमत्यन्तापर{रि)दृष्टा पूर्वापरीभूतावयवा न शक्या पिण्डीभूता निदर्शयितुम्" इति भाष्यात् तस्या अतीन्द्रियत्वेन `परोक्षे' इत्यव्यावर्त्तकमिति शङ्कयम्, पिण्डीभूताया निदर्शयितुमशक्यत्वेऽप्यवयवशः `साक्षात्करोमि' इति प्रतीतिविषयत्वसम्भवात्~।
अन्यथा `पश्य मृगो धावति' इत्यत्र तस्या दर्शनकर्मता न स्यादिति प्रतिभाति~।
 व्यापाराविष्टानां क्रियानुकूलसाधनानामेवात्र पारोक्ष्यं विवक्षितमतो नोक्तदोषः~।
`अयं पपाच' इत्याद्यनुरोधाद् व्यापाराविष्टानामित्यपि वदन्ति~।
 कथं तर्हि "व्यातेने किरणावलीमुदयनः" इति, स्वक्रियायाः स्वप्रत्यक्षत्वादिति चेत्, असङ्गतमेव, व्यासङ्गादिना स्वव्यापारस्य परोक्ष्त्वोपपादनेऽपि (बहुतरमनःप्रणिधानसाध्यशास्त्रार्थनिर्णयजनकशब्दरचनात्मके ग्रन्थे) अनद्यतनत्वातीतत्वयोर्विस्तारक्रियायामसत्त्वेन अनद्यतनातीतत्वयोरभावेन तदर्थकलिडसम्भवात्~।
 लुडर्थमाह- श्वो भाविनीति~।
अनद्यतने भाविनीत्यर्थः, "अनद्यतने लुट्"\footnote{३-३-१५} इति सूत्रात्~।
यथा "श्वो भविता" इत्यादौ~।
 लृडर्थमाह- भविष्यतीति~।
भविष्यत्सामान्ये इत्यर्थः, "लृट् शेषे च"\footnote{३-३-१३} इति सूत्रात्~।
यथा "घटो भविष्यति" इत्यादौ~।
तत्त्वञ्च-वर्त्तमानप्रागभावप्रतियोगिसमयोत्पत्तिमत्त्वम्~।
 लेडर्थमाह-विध्यादाविति, "लिङर्थे लेट्"\footnote{३-४-७} इति सूत्रात्~।
लिङर्थश्च विध्यादिरिति वक्ष्यते~।
 लोडर्थमाह-प्रार्थनेति~।
आदिना विध्याद्याशिषो गृह्यन्ते, "आशिषि लिङ्लोटौ"\footnote{३-३-१७३}, "लोट् च"\footnote{३-३-१६२}इति सूत्राभ्यां तथाऽवगमात्~।
यथा-`भवतु ते शिवप्रसादः' इत्यादौ~।
एतयोर्थो लिङर्थ एव, त्रयाणां समानार्थत्वादिति तन्निर्णयेनैव निर्णयः॥२२॥\par
लङादिक्रमेण ङितामर्थमाह-
\begin{center}{\bfseries ह्यो भूते प्रेरणादौ च भृतमात्रे लङादयः~।\\
 सत्यां क्रियातिपत्तौ च भूते भाविनि लृङ् स्मृतः॥२३॥}
\end{center}
 लङर्थमाह- ह्यो भूत इति~।
अनद्यतने भूत इत्यर्थः, "अनद्यतने लङ्"\footnote{३-१-१११} इति सूत्रात्~।
यथा- 'अस्य पुत्रोऽभवत्' इत्यादि~।
 लिङर्थमाह- प्रेरणादाविति, "विधिनिमन्त्रणामन्त्रणाधीष्टसम्प्रश्नप्रार्थनेषु लिङ्"\footnote{३-३-१६१} इति सूत्रात्~।
तत्र विधिः = प्रेरणम्, भृत्यादेर्निकृष्टस्य प्रवर्त्तनम्~।
निमन्त्रणम् = नियोगकरणम्,आवश्यके (श्राद्धभोजनादौ दौहित्रादेः प्रवर्तनम्) प्रेरणेत्यर्थः~।
आमन्त्रणम् = कामचारानुज्ञा~।
अधिष्टम् = सत्कारपूर्वको व्यापारः~।
सम्प्रश्रः = सम्प्रधारणम्~।
 एव च्चतुष्टयानुगतप्रवर्त्तनात्वेन वाच्यता, लाघवात्~।
उक्तञ्च-
\begin{center} अस्ति प्रवर्त्तनारूपमनुस्यूतं चतुर्ष्वपि~।\\
 तत्रैव लिङ् विधातव्यः किं भेदस्य विवक्षया॥\\[10pt]
 न्यायव्युत्पादनार्थं वा प्रपञ्चार्थमथापि वा~।\\
 विद्धयादीनामुपादानं चतुर्णामादितः कृतम्॥इति~।\\
 \end{center}
प्रवर्त्तनात्वञ्च- प्रवृत्तिजनकज्ञानविषयताऽवच्छेदकत्वम्~।
तच्चेष्टसाधनत्वस्यास्तीति तदेव विध्यर्थः~।
यद्यप्येतत् कृतिसाध्यत्वस्यापि अस्ति, तज्ज्ञानस्यापि प्रवर्त्तकत्वात्, तथाऽपि यागादौ सर्वत्र तल्लोकत एव लभ्यत इत्यन्यलभ्यत्वान्न तच्छक्यम्~।
बलवदनिष्टाननुबन्धित्वज्ञानञ्च न हेतुः, द्वेषाभावेनान्यथासिद्धत्वात्, आस्तिककामुकस्य नरकसाधनताज्ञानदशायामप्युत्कटेत्छाया द्वेषाभावदशायां प्रवृत्तेर्व्यभिचाराच्च~।
तस्मादिष्टसाधनत्वमेव प्रवर्त्तना~।
उक्तञ्च मण्डनमिश्रैः-
\begin{center} पुंसां नेष्टाभ्युपायत्वात् क्रियास्वन्यः प्रवर्त्तकः~।\\
 प्रवृत्तिहेतुं धर्म्मञ्च प्रवदन्ति प्रवर्त्तनाम् ॥ इति~।
\end{center}
प्रपञ्चितं चैतद् वैयाकरणभूषणे~।

 आदिना "हेतुहेतुमतोर्लिङ्"\footnote{३-३-१५६} "आशिषि लिङ्लोटौ"\footnote{३-३-१७३} इति सूत्रोक्ता हेतुहेतुमद्भावादयो गृह्यन्ते~।
"यो ब्राह्मणायावगुरेत्तं शतेन यातयेत्" इति यथा~।
 लुङर्थमाह-भूतमात्र इति-भूतसामान्ये इत्यर्थः, भूते लुङ् {भूते इत्यधिकृत्य "लुङ्"\footnote{३-२-११०} इति सूत्रात्) अत्र `विद्यमानध्वंसप्रतियोगित्वं भूतत्वम्'~।
तच्च क्रियायां निर्बाधमिति विद्यमानेऽपि घटे घटोऽभूदिति प्रयोगः~।
विद्यमानध्वंसप्रतियोगी घटाबिन्नाश्रयक उत्पत्त्याद्यनुकूलो व्यापार इति बोधः~।
 अयमत्र संग्रहः- कालो द्विविधः, अद्यतनोऽनद्यतनश्च~।
आद्यस्त्रिविधः, भूतभविष्यद्वर्त्तमानभेदात्~।
अन्त्यो द्विविधः- भूतो भविष्य(ष्यँ)श्च~।
तत्र वर्त्तमानत्वे लट्~।
भूतत्वमात्रे लुङ्~।
भविष्यत्तामात्रे लृट्~।
हेतुहेतुमद्भावाद्यधिकार्थविवक्षायामानयोर्लृङ~।
अनद्यतने भूतत्वेन विवक्षिते लङ्~।
तत्रैव परोक्षत्वविवक्षायां लिट्~।
ततदृशे भविष्यति लुट्, इति द्रष्टव्यः~। 
 लृङर्थमाह- सत्यामिति- क्रियाया अतिपत्ति = अनिष्पत्तिस्तस्यां गम्यमानायाम् , भूते भाविनि हेतुहेतुमद्भावे सति लृङित्यर्थः~।
"लिङ्-निमित्ते लृङ् क्रियाऽतिपत्तौ"\footnote{३-३-१३८} इति सूत्रात्~।
लिङो निमित्तं हेतुहेतुमद्भावादि~।
यथा-`सुवृष्टश्चेदभविष्यत् सुभिक्षमभविष्यत्'~।
`वह्निश्चेत् प्राज्वलिष्यद् ओदनमपक्ष्यत्' इत्यादौ~।
अत्र वह्न्यभिन्नाऽऽश्रयकप्रज्वलनानुकूलव्यापाराभावप्रयोज्य- ओदनाभिन्नाश्रयकविक्लित्त्यनुकूलव्यापाराभाव इति शाब्दबोधः~।
 एवंरीत्या द्रष्टव्यम्~।
अयञ्च अर्थनिर्देश उपलक्षणम्, अर्थन्तरेऽपि बहुशो विधानदर्शनात्~।
प्रसिद्धत्वादेष्वेवार्थेषु शक्तिरन्यत्र लक्षणेति मतान्तररीत्या वोक्तम्~।
एतेषां क्रमनियामकश्चानुबन्धक्रम एव~।
अत एव पञ्चमो लकार इत्यनेन मीमांसकैर्लेट् व्यवह्रियत इति दिक् ॥२३॥\par
\begin{center} इति वैयाकरणभूषणसारे लकारविशेषार्थनिरूपणं समाप्तम् ॥२॥ \end{center}





\section*{\begin{center}अथ सुबर्थनिर्णयः\end{center}}
\addcontentsline{toc}{section}{सुबर्थनिर्णयः}
\fancyhead[LE,RO]{सुबर्थनिर्णयः}
सुबर्थमाह-
\begin{center}{\bfseries आश्रयोऽवधिरुद्देश्यः सम्बन्धः शक्तिरेव वा~।\\
 यथायथं विभक्त्यर्थाः सुपां कर्मेति भाष्यतः॥२४॥}
\end{center}
 द्वितीया-तृतीया-सप्तमीनामाश्रयोऽर्थ~।
तथाहि-"कर्मणि द्वितीया"\footnote{२-३-२}~।
तच्च कर्तुरीप्सिततमम्-क्रियाजन्यफलाश्रय इत्यर्थः, क्रियाजन्यफलवत्त्वेन कर्मण एव कर्तुरीप्सिततमत्वात्~।
 "तथायुक्तञ्चानीप्सितम्"\footnote{१-४-५०} इत्यादिसंग्रहाच्चैवमेव युक्तम्~।ईप्सितानीप्सितत्वयोः शाब्दबोधे भानाभावेन संज्ञायामेव तदुपयोगः, नतु वाच्यकोटौ तत्प्रवेशः~।
 तथाच क्रियायाः फलस्य च धातुनैव लाभादनन्यलभ्य आश्रय एवार्थः~।
तत्त्वञ्चखण्डशक्तिरूपमवच्छेदकम्~।
`ओदनं पचति' इत्यत्र विक्लित्त्याश्रयत्वात् कर्मता~।
 `घटं करोती' इत्यत्र उत्पत्त्याश्रयत्वात्, उत्पत्तेर्धात्वर्थत्वात्~।
`जानाति' इत्यत्र आवरणभङ्गरूपधात्वर्थफलाऽऽश्रयत्वात्~।
अतीतानागतादिपरोक्षस्थलेऽपि ज्ञानजन्यस्य तस्याऽऽवश्यकत्वात्~।
अन्यथा `यथापूर्वं न जानामि' इत्यापत्तेः~।
 अतीतादेराश्रयता च विषयतया ज्ञानाश्रयताया नैयायिकानामिव सत्कार्य्यवादसिद्धान्ताद्वोपपद्यते इति~।
उक्तञ्च-
\begin{center}
 तिरोभावाभ्युपगमे भावानां सैव नास्तिता~।\\
 लब्धक्रमे तिरोभावे नश्यतीति प्रतीयते॥इति~।
\end{center} ननु `चौत्रो ग्रामं गच्छति' इत्यत्र ग्रामस्येव चैत्रस्यापि क्रियाजन्यग्रामसंयोगरूपफलाश्रयत्वात् कर्मताऽऽपत्तौ `चैत्रश्चैत्रं गच्छति' इत्यापत्तिः, प्रयोगतः काशीं गच्छति चैत्रे `प्रयागं गच्छति' इत्यापत्तिश्च, क्रियाजन्यसंयोगस्य काश्यामिव विभागस्य प्रयागेऽपि सत्त्वात्, इति चेन्न, ग्रामस्येव चैत्रस्यापि फलाऽऽश्रयत्वेऽपि तदीयकर्तृसंज्ञया कर्मसंज्ञाया बाधेन `चैत्रश्चैत्रम्' इति प्रयोगासम्भवात्, द्वीतीयोत्पत्तौ संज्ञाया एव नियामकत्वात्~।
अन्यथा `गमयति कृष्णं गोकुलम्' इत्यत्रेव `पाचयति कृष्णेन' इत्यत्रापि कृष्णापदाद् द्वितीयाऽऽपत्तेः~।
 शाब्दबोधः `चैत्रश्चैत्रम्' इत्यत्र स्यादिति चेन्न, तथाव्युत्पन्नामिष्टाऽऽपत्तेः~।
उच्यतां वा `प्रकारतासम्बन्धेन धात्वर्थफलविशेष्यकबोधं प्रति धात्वर्थव्यापारानधिकरणाऽऽश्रयोपस्थितिर्हेतुः' इति कार्य्यकारणभावान्तरम्~।
प्रकृते चैत्रस्य व्यापारानधिकरणत्वाभावान्न दोषः~।
प्रयागस्य कर्मत्वन्तुसम्भावितमपि न, समभिव्याहृतधात्वर्थफलशालित्वस्यैव क्रियाजन्येत्यनेन विवक्षणस्य उक्तप्रायत्वात्~।
 नैयायिकास्त्वाद्यदोषवारणाय परसमवेतत्वम्, द्वितीयदोषवारणाय धात्वर्थताऽवच्छेदकत्वं फले विशेषणं द्वीतीयावाच्यमित्युपाददते~।
परसमवेतत्वं धात्वर्थक्रियायामन्वेति, तथैव कार्यकारणभावान्तरकल्पनात्~।
परत्वञ्च द्वितीयया स्वप्रकृत्यर्थापेक्षया बोध्यते~। 
 तथाच `चैत्रस्तण्डुलं पचति' इत्यादौ तण्डुलान्यसमवेतव्यापारजन्यधात्वर्थताऽवच्छेदकविक्लित्तिशालित्वात् तण्डुलस्य कर्मता~।
शाब्दबोधस्तु तण्डुलसमवेतधात्वर्थतावच्छेदकविक्लित्त्यनुकूलतण्डुलान्यसमवेतक्रियाजनक\-कृतिमाँश्चैत्र इत्याहुः,
 तन्न रोचयामहे, परसमवेतत्वादेर्गौरवेणावाच्यत्वात्~।
अतिप्रसङ्गः किम् द्वितीयायाः, शाब्दबोधस्य वा? नाद्यः, तावद्वाच्यकथनेऽपि तत्तादवस्थ्यात्, `गमयति कृष्णं गोकुलम्' इतिवत् `पाचयति कृष्णं गोपः' इति द्वितीयाऽऽपत्तेः,`तण्डुलं पचति चैत्रः' इतिवत् `तण्डुलं पच्यते स्वयमेव' इत्यापत्तेश्च, विक्लित्त्यनुकूलतण्डुलान्यसमवेताग्निसंयोगरूपधात्वर्थाऽऽश्रयत्वात्~।
शाब्दबोधातिप्रसङ्गोऽप्युक्तरीत्यैव निरस्तः, परसमवेतत्वस्य शक्यत्वेऽपि परत्वस्य, परसमवेतत्वस्य च इष्टान्वयलाभायानेकशः कार्य्यकारणभावाभ्युपगमे गौरवान्तरत्वादिति स्पष्टं भूषणे~।

एतच्च सप्तविधम्-
\begin{center} निर्वर्त्त्यञ्च विकार्य्यञ्च प्राप्यञ्चेति त्रिधा मतम्~।\\
 तच्चेप्सिततमं कर्म चतुर्द्धाऽन्यत्तु कल्पितम्॥\\[10pt]
 औदासीन्येन यत् प्राप्यं यच्च कर्तुरनीप्सितम्~।\\
 संज्ञान्तरैरनाख्यातं यद्यच्चाप्यन्यपूर्वकम्॥\\
\end{center}
इति वाक्यपदीयात्~।
\begin{center} यदसज्जायते सद्वा जन्मना यत्प्रकाशते~।\\
 तन्निर्वर्त्यं विकार्यं तु द्वेधा कर्म व्यवस्थितम्॥\\[10pt]
 प्रकृत्युच्छेदसम्भूतं किञ्चित् काष्ठादिभस्मवत्~।\\
 किञ्चिद्गुणान्तरोत्पत्त्या सुवर्णादिविकारवत्॥\\[10pt]
 क्रियाकृतविशेषाणां सिद्धिर्यत्र न गम्यते~।\\
 दर्शनादनुमानाद्वा तत्प्राप्यमिति कथ्यते॥
\end{center}
इति च तत्रैवोक्तम्~।
 `घटं करोति' इत्याद्यम्~।
`काष्ठं भस्म करोति' इति, `सुवर्णं कुण्डलं करोति' इति च द्वितीयम्~।
`घटं पश्यति' इति तृतीयम्~।
`तृणं स्पृशति' इत्युदासीनम्~।
`विषं भुङ्क्ते' इति द्वेष्यम्~।
`गां दोग्धि' इति संज्ञान्तरैनाख्यातम्~।
 `क्रूरमभिक्रुद्धयति' इत्यन्यपुर्वकम्~।

 कर्तृतृतीयाया आश्रयोऽर्थः~।
तथाहि- "स्वतन्त्रः कर्ता"\footnote{१-४-५४} स्वातन्त्र्यञ्चधात्वर्थव्यापाराऽऽश्रयत्वम्~।
\begin{center} धातुनोक्तक्रिये नित्यं कारके कर्तृतेष्यते~।
\end{center} इति वाक्यपदीयात्~।
अत एव `यदा यदीयो व्यापारो धातुनाऽभिधीयते तदा स कर्ता' इति `स्थाली पचति' `अग्निः पचति' `एधांसि पचन्ति' `तण्डुलः पच्यते स्वयमेव' इत्यादि सङ्गच्छते~।
 नन्वेवम् "कर्मकर्तृव्यपदेशाच्च" {ब्र.सू.१-२-४} इति सूत्रे "मनोमयः प्राणशरीरः" इति वाक्यश्थमनोमयस्य जीवत्वे वाक्यशेषे तस्य "एवमितः प्रेत्याभिसम्भविताऽस्मि" इति प्राप्तिकर्मत्वकर्तृत्वव्यपदेशो विरुद्ध इति भगवता व्यासेन निर्णोतं कथं सङ्गच्छताम्? उच्यते-जीवस्यैव ज्ञेयत्वे प्राप्तिकर्मत्वमपि वाच्यम्, कर्तृत्वञ्च तस्य आख्यातेनोक्तम्~।
नचैकस्यैकदा संज्ञाद्वयं युक्तम्, कर्तृसंज्ञाया कर्मसंज्ञाया बाधात्~।
तथाच `एतम्' इति द्वितीया न स्यात्~।
कर्मकर्तृतायाञ्च यगाद्यापत्तिरिति शब्दविरोधद्वारा भवति स भेदहेतुः~।
 एवञ्च व्यापारंशस्य धातुलभ्यत्वादाश्रयमात्रं तृतीयार्थः~।
कारकचक्रप्रयोक्तृत्वम्, कृत्याश्रयत्वं `दण्डः करोति' इत्यत्राव्याप्तम्~।
 अयञ्च त्रिवधः-सुद्धः, प्रयोजको हेतुः, कर्मकर्ता च~।
`मया हरिः सेव्यते' `कार्य्यते हरिणा' `गमयति कृष्णं गोकुलम्' मदभिन्नश्रयको हरिकर्मकसेवनानुकूलो व्यापारः, हर्य्यभिन्नश्रयक उत्पादनानुकूलो व्यापारः, गोकुलकर्मकगमनानुकूलकृष्णाश्रयकतादृशव्यापारानुकूलो व्यापार इति शाब्दबोधः~।
करणतृतीयायास्त्वाश्रयव्यापारौ वाच्यौ~।
तथाहि-"साधकतमं कारणम्"\footnote{१-४-४९}~।
तमबर्थः प्रकर्षः~।
स चाव्यवधानेन फलजनकव्यापारवत्ता~।
 तादृशव्यापारवत्कारणञ्च कारणम्~।
उक्तञ्च वाक्यपदीये-
\begin{center} क्रियायाः परिनिष्पत्तिर्यद्व्यापारादनन्तरम्~।\\
 विवक्ष्यते यदा यत्र करणं त्तत्तदा स्मृतम्~।\\[10pt]
 वस्तुतस्तदनिर्देशयं न हि वस्तु व्यवस्थितम्~।\\
 स्थाल्या पच्यत इत्येषा विवक्षा दृश्यते यतः॥इति॥
\end{center} `विवक्षयते' इत्यनेन सकृदनेकेषां तदाभावाद् द्वितीयासप्तम्यादेरवकाशं सूचयति~।
 नचैवम् "कर्ता शास्त्रार्थवत्त्वात्" {ब्र.सू.२-३-३३} इत्युत्तरमीमांसाधिकरणे "शक्तिविपर्य्ययात्" इति सूत्रेणान्तःकरणस्य कर्तृत्वे करणशक्तिविपर्य्ययापत्तिरुक्ता न युज्येतेति वाच्यम्, "तदेवैतेषां प्राणानां विज्ञानेन विज्ञानमादाय" इति श्रुत्यन्तरे करणतया क्लृप्तस्य कर्तृतां प्रकल्प्य शक्तिविपर्य्ययापत्तिर्निष्प्रमाणा कल्प्येतेत्यभिप्रायात्~।
 वस्तुतस्तु अभ्युच्चयमात्रमेतदिति "यथा च तक्षोभयथा" {ब्र.सू.२-३-४०} इत्यधिकरणे भाष्य एव स्पष्टमित्यादि प्रपञ्चितं भूषणे ॥
 सप्तम्या अप्याश्रयोऽर्थः~।
"सप्तम्यधिकरणे च"\footnote{२-३-३६} इत्यधिकरणे सप्तमी~।
तच्च "आधारोऽधिकरणम्"\footnote{१-४-४५} इति सूत्रादाधारः~।
तत्त्वञ्चाश्रयत्वम्~।
तत्राश्रयांशः शक्यः, तत्त्वमवच्छेदकम्~।
नचाश्रयत्वमात्रेण कर्मकर्तृकरणानामाधारसंज्ञास्यात्, स्यादेव, यदि ताभिरस्या न बाधः स्यात्~।
"कारके"\footnote{१-४-२३} इत्यधिकृत्य विहितसप्तम्याः क्रियाश्रय इत्येव यद्यपि तात्पर्य्यम्, तथाप्यत्र चकर्तृकर्मद्वारा तदाश्रयत्वमस्त्येव~।
{नचैवमपि साङ्कर्यमेव स्यादिति वाच्यम्, निरूपकभेदेन भेदात्~।
फलनिरूपिताधारो द्वितीयार्थः, व्यापारनिरूपितस्तुल तृतीयार्थः, कर्तृकर्मव्यापारफलनिरूपितः सप्तम्यर्थः}~।
स्थाल्यादेर्भूतलकटादेश्चेति, `स्थाल्यां पचति' `भूतले वसति' ` कटे शेते' इत्याद्युपपद्यते~।
उक्तञ्च वाक्यपदीये-
\begin{center} कर्तृकर्मव्यवहितामसाक्षाद्धारयत् क्रियाम्~।\\
 उपकुर्वत् क्रियासिद्धौ शास्त्रेऽधिकरणं स्मृतम्॥इति~।
\end{center}
एतच्च त्रिविधम् औपश्लेषिकम्, वैषयिकम्, अभिव्यापकञ्च~।
`कटे शेते' ` गुरौ वसति' `मोक्षे इच्छास्ति' `तिलेषु तैलम्' इति~।
एतच्च "संहितायाम्"\footnote{६-१-७२} इति सूत्रे भाष्ये स्पष्टम्~।
 अवधिः पञ्चम्यर्थः, "अपादाने पञ्चमी"\footnote{२-३-२९} {इति}~।
सूत्रात्~।
तच्च "ध्रुवमपायेऽपादानम्"\footnote{१-४-२४} इति सूत्रात्~।
अपायः=विश्लेषस्तज्जनकक्रिया, तत्रावधिभूतमपादानमित्यर्थकादवधिभूतमिति भावः~।
उक्तञ्च वाक्यपदीये-
\begin{center} अपाये यदुदासीनं चलं वा यदि वाऽचलम्~।\\
 ध्रुवमेवातदावेशात्तदपादानमुच्यते॥\\[10pt]
 पततो ध्रुव एवाश्वो यस्मादश्वात् पतत्यसौ~।\\
 तस्याप्यश्वस्य पतने कुडयादि ध्रुवमुच्यते॥\\[10pt]
 उभावप्यध्रुवौ मेषौ यद्यप्युभयकर्मजे (के)~।\\
 विभागे प्रविभक्ते तु क्रिये तत्र विवक्षिते (व्यवस्थिते)॥\\[10pt]
 मेषान्तरक्रियापेक्षमवधित्वं पृथक् पृथक्~।\\
 मेषयोः स्वक्रियापेक्षं कर्तृत्वञ्च पृथक् पृथक्॥इति~।
\end{center}
 अस्यार्थः-अपाये=विश्लेषहेतुक्रियायाम्, उदासीनाम्=अनाश्रयः, अतदावेशात्=तत्क्रियानाश्रयत्वात्~।
एवञ्च ` निश्लेषहेतुक्रियानाश्रयत्वे सति विश्लेषाश्रयत्वम्' फलितम्~।
`वृक्षात् पर्णं पतति' इत्यत्र पर्णस्य तद्वारणाय सत्यन्तम्~।
 `धावतोऽश्वात्, पतति' इत्यत्राश्वस्य क्रियाश्रयत्वाद्विश्लेषहेत्विति~।
` कुडयात् पततोऽश्वात् पतति' इत्यत्राश्वस्य विश्लेषजनकक्रियाश्रयत्वेऽपि तन्न विरुद्धमित्याह-यस्मादश्वादिति~।
तद्वीश्लेषहेतुक्रियानाश्रयत्वे सतीति विशेषणीयमिति भावः~।
एवमश्वनिष्ठक्रियानाश्रयत्वात् कुडयादेरपि ध्रुवत्वमित्याह-तस्यापीति~।
 उभयकर्मजविभागस्थले विभागस्यैक्यात् तद्विश्लेषजनकक्रियानाश्रयत्वाभावात् ` परस्परान्मेषावपसरतः' इति न स्यादित्याशङ्कय समाधत्ते-उभावपीति~।
मेषान्तरेति-यथा निश्चलमेषादपसरद्द्वितीयमेषस्थले निश्चलमेषस्यापसरन्मेषक्रियामादाय ध्रुवत्वम्, तथात्रापि विभागैक्येऽपि क्रियाभेदादेकक्रियामादाय परस्य ध्रुवत्वमिति~।
तथाच ` विश्लेषाश्रयत्वे सति तज्जनकतत्क्रियानाश्रयत्वं तत्क्रियायामपादानत्वम्' वाच्यम्~।
क्रिया चात्र धात्वर्थः नतु स्पन्दः~।
तेन वृक्षकर्मजविभागवति वस्त्रे, `वृक्षाद् वस्त्रं पतति' इति सङ्गच्छते~।
 वस्तुतो नैतावत् पञ्चम्या वाच्यम्,किन्तु अवधेर्लक्षणमात्रम्, द्वितीयार्थोक्तरीत्या प्रयोगातिप्रसङ्गस्यासम्भवेन वाच्यकोटौ प्रवेशस्य गौरवेणासम्भवाद् इति तु प्रतीभाति~।
नचैवमपि `वृक्षात् स्यान्दते' इति स्यादिति शङ्कयम्, `आसनाच्चलितः' `राज्याच्चलितः' इतिवद् इष्टत्वात्~।
एतेन पञ्चमीजन्यापादानत्वबोधे सकर्मकधातुजन्योपस्थितेर्हेतुत्वमिति समाधानाभासोऽप्यपास्तः~।
 नचैवमपि `वृक्षात् त्यजति' इति दुर्वारम्, कर्मसंज्ञया अपादानसंज्ञायाः बाधेन पञ्चम्यसम्भवात्~।
भ्रमात् कृते तथा प्रयोगे यदि बोधाभावोऽनुभवसिद्धस्तर्हि पञ्चमीजन्यापादानत्वबोधे त्यजादिभिन्नधातुजन्यबुद्धेर्हेतुत्वं वाच्यम्~।
`बलाहकाद् विद्योतते' इत्यादौ निःसृत्येत्यध्याहार्य्यम्~।
`रूपं रसात् पृथक्' इत्यत्र तु 
 बुद्धिपरिकल्पितमपादानत्वं द्रष्टव्यम्, पृथग्विनेति पञ्चमी वा~।
इदञ्च-
\begin{center} निर्दिष्टविषयं किञ्चिदुपात्तविषयं तथा~।\\
 अपेक्षितक्रियञ्चेति त्रिधाऽपादानमुच्यते॥
\end{center}
इति वाक्यपदीयात् त्रिविधम्~।
`यत्र साक्षाद्धातुना गतिर्निर्दिश्यते तन्निर्दिष्टविषयम्'~।
यथा `अश्वात् पतति'~।
`यत्र धात्वन्तरार्थाङ्गं स्वार्थं धातुराह तदुपात्तविषयम्'~।
यथा-`बलाहकाद् विद्योतते'~।
निःसरणाङ्गे विद्योतने द्युतिर्वर्त्तते~।
अपेक्षिता क्रिया यत्र तदन्त्यम्~।
यथा-`कुतो भवान्' `पाटलिपुत्रात्'~।
अत्रगमनमर्थमध्याहृत्यान्वयः कार्य्यः~।
 उद्देश्यश्चतुर्थ्यर्थः~।
तथाहि - `सम्प्रदाने चतुर्थो'~।
तच्च "कर्मणा यमभिप्रैति स सम्प्रदानम्"\footnote{१-४-३२} इति सूत्रात् कर्मणा=करणभूतेन यमभिप्रैति=ईप्सति तत् कारकं सम्प्रदानमित्यर्थकादुद्देश्यम्~।
 इदमेव शेषित्वम्~।
तदुद्देश्यकेच्छाविषयत्वं च शेषत्वमित्येव पूर्वतन्त्रे निरूपितम्~।
अत एव "प्रासनवन्मैत्रावरुणाय दण्डप्रदानम्\footnote{अ.४पा.२अ.६} इत्यधिकरणे क्रीते सोमे मैत्रावरुणाय `दण्डं प्रयच्छति' इति विहितं दण्डदानं न प्रतिपत्तिः, किन्तु चतुर्थोश्रुत्यार्थकर्मेति तत्र निर्णोतम्~।
`रजकाय वस्त्रं ददाति' इत्यपि `खण्डिकोपाध्यायः शिष्याय चपेटां ददाति' इति भाष्योदाहरणादिष्टमेव~।
 वृत्तिकारास्तु-सम्यक् प्रतीयते यस्मै तत् सम्प्रदानमित्यन्वर्थसंज्ञया स्वस्वत्वनिवृत्तिपर्यन्तमर्थं वर्णयन्तः `रजकस्य वस्त्रम्' इत्येवाहुः~।
इदञ्च-
\begin{center}अनिराकरणात् कर्तुस्त्यागङ्गं कर्मणेप्सितम्~।\\
 प्रेरणानुमतिभ्यां च लभते सम्प्रदानताम्॥
\end{center}
 इति वाक्यपदीयात् त्रिविधम्~।
"सूर्यायार्ध्यं ददाति" इत्याद्यम्~।
नात्र सूर्यः प्रार्थयते, नानुमन्यते, न निराकरोति~।
प्रेरकम्-"विप्राय गां ददाति"~।
अनुमन्तृ-"उपाध्यायाय गां ददाति"~।

 अत्र सर्वत्र प्रकृतिप्रत्ययार्थयोरभेदः संसर्गः, विभक्तीनां धर्मिवाचकत्वात्, धर्ममात्रवाचकत्वे "कर्मणि द्वितीया"\footnote{२-३-२} इति सूत्रस्वारसभङ्गापत्तेः, कर्मार्थककृत्तद्धितादौ तथादर्शनाच्च, द्वितीयाद्यर्थकबहुव्रीहौ `प्राप्तोदकः' इत्यादौ धर्मिवाचकत्वलाभाच्च~।
\begin{center}{\bfseries सुपां कर्मादयोऽप्यर्थाः सङ्खया चैव तथा तिङाम्~।}\end{center}
इति भाष्याच्चेति दिक्~।
 आश्रयस्यापि प्रकृत्यैव लाभान्न विभक्तिवाच्यता, किन्त्वाश्रयत्वमात्रं वाच्यम्~।
तदेव च तादात्म्येनावच्छेदकम्~।
करणतृतीयायाश्च व्यापारोऽपि, पञ्चम्या विभागमात्रम्, चतुर्थ्या उद्देश्यत्वमात्रम्~।
 अत एवाकृत्यधिकरणमपि न विरुद्धयत इत्यभिप्रेत्याह- शक्तिरेव वेति~।
षण्णामपीति शेषः~।
"शेषे पष्ठी"\footnote{२-३-५०} इति सूत्रात् तस्याः सम्बन्धमात्रं वाच्यम्~।
कारकषष्ठयास्तु शक्तिरेवेत्यलम्~।
 "सप्तमीपञ्चम्यौ कारकमध्ये"\footnote{२-३-७} इति सूत्रे `शक्तिः कारकम्' इति पक्षस्य भाष्ये दर्शनात्~।
एवञ्च `देवदत्तस्य गौर्ब्राह्नणाय गेहाद् गङ्गायां हस्तेन मया दीयते' इत्यत्र देवदत्तसम्बन्धिनी या गौस्तदभिन्नाश्रयकत्यागानुकूलो ब्राह्नणोद्देस्यको गेहनिष्ठविभागजनको गङ्गाधिकरणको हस्तकरणको मन्निष्ठो व्यापार इति बोधः~।यथायथम् = उक्तप्रकारेण~।

 अत्र मानमुपदर्शयन् `घटं जानाति' इत्यादौ द्वितीयाया विषयतायां लक्षणेति बह्वाकुलं वदतो नैयायिकादीन् प्रत्याह-सूपां कर्मेति~।
अयं भावः-
\begin{center} सुपां कर्मादयोऽप्यर्थाः संख्या चैव तथा तिङाम्~।\\
 प्रसिद्धो नियमस्तत्र नियमः प्रकृतेषु वा ॥
\end{center} वार्त्तिकतद्भाष्याभ्यां कर्मादेर्वाच्यतायास्तन्नियमस्य च लाभः~।
 तथाहि-"स्वौजसमौट्"\footnote{४-१-२} "कर्मणि द्वितीया"\footnote{२-३-२} इति "द्वयोकयोर्द्विवचनैकवचने"\footnote{१-४-२२} इत्यादेः; {लः कर्मणिच} "लस्य"\footnote{३-४-७७} "तिप्तस्झि"\footnote{३-४-७९} "तान्येकवचनद्विवचन"\footnote{१-४-१०१} इत्यादेरेकवाक्यतया कर्मादेस्तत्सङ्खयायाश्च वाच्यता लभ्यते~।
तथा तन्नियमश्च द्विविधो लभ्यते-द्वितीया कर्मण्येव, तृतीया करण एवेत्येवमादिः शब्दनियमः~।
कर्मणि द्वितीयैव, करणे तृतीयैवेत्येवमर्थनियमश्च~।
उभयथापि सिद्धनियमविरुद्धं लक्षणादिकमसाधुत्वप्रयोजकमिति याज्ञे कर्मण "नानृतं वदेत्" इति निषेधविषयो भवत्येवेति स्वेच्छया लक्षणाऽपि विभक्तावप्रयोजिकैव~।
अत एव विभक्तौ न लक्षणा' इत्यादिर्नैयायिकवृद्धानां व्यवहार इति दिक्॥२४॥
\begin{center} इति श्रीवैयाकरणभूषणसारे सुबर्थनिर्णयः॥३॥\end{center}

\section*{\begin{center}अथ नामार्थनिर्णयः\end{center}}\addcontentsline{toc}{section}{नामार्थनिर्णयः}
\fancyhead[LE,RO]{नामार्थनिर्णयः}
नामार्थमाह-
\begin{center}{\bfseries एकं द्विकं त्रिकं चाथ चतुष्कं पञ्चकं तथा~।\\
 नामार्थ इति सर्वेऽमी पक्षाः शास्त्रे निरूपिताः॥२५॥}\end{center}
 एकम्=जातिः, लाघवेन तस्या एव वाच्यत्वौचित्यात्, अनेकव्यक्तीनां वाच्यत्वे गौरवात्~।
 नच व्याक्तीनामपि प्रत्येकमेकत्वाद्विनिगमनाविरहः, एवं हि एकस्यामेव व्यक्तौ शक्त्यभ्युपगमे व्यक्त्यन्तरे लक्षणायां स्वसमवेताश्रयत्वं संसर्ग इति गौरवम्,जात्या तु सहाश्रयत्वमेव संसर्ग इति लाघवम्~।
 किञ्चैवं विशिष्टवाच्यत्वमपेक्ष्य नागृहीतविशेषणन्यायाज्जातिरेव वाच्येति युक्तम्, व्यक्तिबोधस्तु लक्षणया~।
एवञ्च तत्र विभक्त्यर्थान्वयोऽप्युपपद्यते इति दिक्~।
 यद्वा, केवलव्यक्तिरेव एकशब्दार्थः, केवलव्यक्तिपक्षे एवाण्ग्रहणस्य, एकशेषस्य चारम्भेण तस्यापि शास्त्रसिद्धत्वात्~।
युक्तत्र्चैतत्-व्यवहारेण व्यक्तावेव तद्ग्रहणात्~।
सम्बन्धितावच्छेदकस्य (सम्बन्धितावच्छेदिकायाः) जातेरैक्याच्छक्तिरप्येकैवेति न गौरवमपि~।
 नचैवं घटत्वमपि वाच्यं स्यात्, शक्यतावच्छेदकत्वात्, तथाच `नागृहीतविशेषण'न्यायात् तदेव वाच्यमस्त्विति शङ्कयम्, अकारणत्वेऽपि कारणतावच्छेदकत्ववत्, अलक्ष्येऽपि लक्ष्यतावच्छेदकत्ववत्तथात्रापि सम्भवात्~।
उक्तञ्च-
\begin{center}
 आनन्त्येऽपि हि भावानामेकं कृत्वोपलक्षणम्~।\\
 शब्दः सुकरसम्बन्धो नच व्यभिचरिष्यति॥इति~।
\end{center}
वस्तुतस्तु "नह्याकृतिपदार्थतस्य द्रव्यं न पदार्थः" इति~।
 भाष्याद् विशिष्टं वाच्यम्~।
`एकम्' इत्यस्य चायमभिप्रायः-शक्तिज्ञाने च विषयतयाऽवच्छेदिका जातिरेकैव~।
तथाच घटत्वविशिष्टबोधे घटत्वांशेऽन्याप्रकारकघटत्वशक्तिज्ञानत्वेन हेतुतेति कार्यकारणभाव इत्यादि प्रपञ्चितं भूषणे~।
 तदेतदभिप्रेत्याह- द्विकमिति~। जातिव्यक्ती इत्यर्थः~।
पूर्वपक्षाद् विरोधपरिहारः पूर्ववत्~।
 त्रिकमिति- जातिव्यक्तिलिङ्गानीत्यर्थः~।
सत्त्वरजस्तमोगुणानां साम्यावस्था नपुंसकत्वम्~।
आधिक्यं पुंस्त्वम्~।
अपचयः स्त्रीत्वम्~।
तत्तच्छब्दनिष्ठं तत्तच्छब्दवाच्यञ्च, तमेव विरुद्धधर्ममादाय तटादिशब्दा भिद्यन्ते~।
 केषाञ्चिदनेकलिङ्गत्वव्यावहारस्तु समानानुपुर्वोकत्वेन शब्दानामभेदारोपात्~।
 एवञ्च पदार्थपदे पुँस्त्वमेव~।
व्यक्तिपदे स्त्रीत्वमेव~।
वस्तुपदे नपुंसकत्वमेव इति सर्वत्रैव `अयं पदार्थः' `इयं व्यक्तिः' `इदं वस्तु' इति व्यवहारः,`तटः' `तटी' 'तटम्' इति चोपपद्यते~।
 तच्च लिङ्गमर्थपरिच्छेदकत्वेन अन्वेति, इति पश्वादिशब्दोक्तम्-`पशुः स्त्रियां नास्ति' इति, `पशुना' इत्यादिविधिर्न छाग्यादीनङ्गत्वेन प्रयोजयतीति विभावनीयम्~।
नच व्यक्त्यादिशब्दोक्तलिङ्गस्येव पश्वादिशब्दोक्तस्यापि साधारण्यं शङ्कयम्, व्यक्तिशब्दस्य नित्यस्त्रीलिङ्गत्वेन तथा सम्भवेऽपि पशुशब्दस्य नित्यपुंल्लिङ्गत्वे प्रमाणाभावात्~।
 `पश्वा न तायुं गुहा चतन्तम्' (ऋवे.म.१अनु.१२ सू.६५)~।
 "पश्वे नृभ्यो यथा गवे" (ऋ.वे.अ.१ अ.३ व.२६) इत्यादिवेदे दर्शनाच्च, प्तीमांसायां चतुर्थे "पशुना यजेत' इत्यत्रैकत्व-पुँस्त्वयोर्विवक्षितत्वान्नानेकपशुभिः, पशुस्त्रिया वा याग इति प्रतिपादितत्वाच्च~।
 वस्तुतस्तु विशेषविध्यभावे उप्रत्ययान्तानां पुँस्त्वस्य व्याकरणेन निर्णितत्वाद् वेदभाष्येऽपि जसादिषु छन्दसि वा वचनम्' इति नाभावाभाव इत्युक्तेः पशुशब्दस्य नित्यपुँस्त्वनिर्णयात्~।
प्रकृते "छागो वा मन्त्रवर्णात्" इति न्यायेनैव निर्णयः~।
मन्त्रवर्णे हि `छागस्य वपाया मेदसः' इति श्रूयते~।
तत्र छागस्येति छाग्यामसम्भावितम्, इति भवति ततः पुंस्त्वनिर्णय इति विस्तरेण प्रपञ्चितं भूषणे~।
 चतुष्कम्-सङ्खयासहितं त्रिकमित्यर्थः~।
 पञ्चकम्-कारकसहितं चतुष्कमित्यर्थः~।
 नन्वन्वय-व्यतिरेकाभ्यां प्रत्ययस्यैव तद्वाच्यम्, तत एव लिङ्गादीनामुपस्थितौ प्रकृतिवाच्यत्वे मानाभावाच्चेति चेत्, सत्यम्, प्रत्ययवर्जिते `दधि पश्य' इत्यादौ प्रत्ययमजानतोऽपि बोधात् प्रकृतेरेव वाचकत्वं कल्प्यते, लिङ्गानुशासनस्य प्रकृतेरेव दर्शनाच्च~।
 अत एवैषु पक्षेषु न निर्बन्धः (प्रत्ययस्यौव वाचकताया युक्तत्वात्)~।
 `द्योतिका वाचिका वा स्युर्द्धित्वादीनां विभक्तयः'~। 
 इति वाक्यपदीयेऽपि पक्षद्वयस्य व्युत्पादनात्~।
 शास्त्र इति-बहुषु स्थलेषु व्युत्पादनं व्यञ्जयितुम्, प्राधान्येन तु सरूपसूत्रादौ व्यक्तम्॥२५॥
 शब्दस्तावच्छाब्दबोधे भासते,
\begin{center}न सोऽस्ति प्रत्ययो लोके यः शब्दानुगमादृते~।\\
 अनुविद्धमिव ज्ञानं सर्वं शब्देन भासते॥\end{center}
इत्याद्यभियुक्तोक्तेः~।
 `विष्णुमुच्चारय' इत्यादावर्थोच्चारणासम्भवेन, विना शब्दविषयं शाब्दबोधासङ्गतिश्चेति सोऽपि प्रातिपदिकार्थः~।
 नच लक्षणया निर्वाहः, निरूढलक्षणायाः शक्त्यनतिरेकात्, `जबगडदशमुच्चारय' इत्यादौ शक्याग्रहेण शक्यसम्बन्धरूपलक्षणाया अग्रहाच्च~।
 अज्ञातायाश्च वृत्तेरनुपयोगात्, ` गावमुच्चारय' इति भाषाशब्दानामनुकरणे साधुतासम्प्रतिपत्तेः, तेषां शक्त्यभावेन परनये लक्षणाया असम्भवाच्चेत्यभिप्रेत्य षोढाऽपि क्वचित् प्रातिपदिकार्थ इत्याह-
 \begin{center}{\bfseries शब्दोऽपि यदि भेदेन विवक्षा स्यात् तदा तथा~।\\
 नोचेच्छ्रोत्रादिभिः सिद्धोऽप्यसावर्थो व भासते ॥२६॥}\end{center}
 यद्यनुकार्यानुकरणयोर्भेदविवक्षा तदा शब्दोऽपि प्रातिपदिकार्थः~।
यदि न भेदविवक्षा, तदा श्रोत्रादिभिरुपस्थितोऽप्यर्थवद् भासते~।
अपिर्हेतौ, उपस्थितत्वाद्भासते इत्यर्थः~।
अयं भावः-अनुकार्यानुकरणयोर्भेदेऽनुकार्यस्य पदानुपस्थितत्वात्, तत्सिद्धये शक्तिरुपेया, अभेदे प्रत्यक्षे विषयस्य हेतुत्वात् स्वप्रत्यक्षरूपां पदजन्योपस्थितिमादाय शाब्दबोधविषयतोपपत्तिरिति~।
 यद्यप्यतिप्रसङ्गवारणाय वृत्तिजन्यपदोपस्थितिरेव हेतुः, तथाप्यत्राश्रयतया वृत्तिमत्त्वस्य सत्त्वान्नानुपपत्तिः~।
निरूपकताऽऽश्रयताऽन्यतरसम्बन्धेन वृत्तिमत एव शाब्दबोधविषयत्वं कल्प्यते इत्यनवद्यम्, सम्बन्धस्योभयनिरूप्यत्वात्, पदादर्थस्येव तद्बोधकत्वेन स्वस्यापि ज्ञानसम्भवाच्चेति~।
उक्तञ्च वाक्यपदीये--
\begin{center} ग्राह्यत्वं ग्राहकत्वं च द्वे शक्ती तेजसो यथा~।\\
 तथैव सर्वशब्दानामेते पृथगवस्थिते ॥ इति,\\[10pt]
 विषयत्वमनादृत्य शब्दैर्नार्थः प्रकाश्यते~।\\ \end{center}
इति चेति~।

प्रसङ्गादनुकार्य्यानुकरणयोरभेदपक्षे साधकमाह-
\begin{center}{\bfseries अत एव गावित्याह भू सत्तायामितीदृशम्~।\\
 न प्रातिपदिकं नापि पदं साधु तु तत् स्मृतम्॥२७॥}\end{center}



 `गवित्ययमाह' `भू सत्तायाम्' इत्येवमादयो यतोऽनुकरणशब्दा अनुकार्य्यान्न भिद्यन्ते~।
अतस्तेषामर्थवत्त्वाभावात् ` अर्थवदधातुः'\footnote{१-४-४२} इत्याद्यप्रवृत्तौ न प्रातिपदिकत्वम्, नापि पदत्वम्, अथ च साधुत्वमीत्युपपद्यते~।
अन्यथा `प्रत्ययः'\footnote{३-१-१} `परश्च'\footnote{३-१-२} `अपदं न प्रयुञ्जीत्' इति निषेधादिलङ्घनादसाधुतापत्तिरित्यर्थः॥२७॥
 \begin{center}इति श्रीमद्रङ्गोजिभट्टात्मजकौण्डभट्टकृते वैयाकरणभूषणसारे नामार्थनिर्णयः ॥४॥\end{center}
 
 
\section*{\begin{center}अथ समासशक्तिनिर्णयः\end{center}}\addcontentsline{toc}{section}{समासशक्तिनिर्णयः}
\fancyhead[LE,RO]{समासशक्तिनिर्णयः}
सामासान् विभजते--
\begin{center}{\bfseries सुपां सुपा तिङा नाम्ना धातुनाऽथ तिङां तिङा~।\\
 सुबन्तेनेति विज्ञायः समासः षड्विधो बुधैः॥२८॥}\end{center}
 
 सुपां सुपा-पदद्वयमपि सुबन्तम् `राजपुरुषः' इत्यादिः~।
सुपां तिङा-पूर्वपदं सुबन्तं उत्तरपदं तिङन्तम् `पर्य्यभूषयत्' `अनुव्यचलत्'~।
 `गतिमतोदात्तवता तिङाऽपि समासः' इति वार्त्तिकात् समासः~।
सुपां नाम्ना `कुम्भकारः' इत्यादिः~।
`उपपदमतिङ्'\footnote{२-२-१९} इति समासः~।
स च "गतिकारकोपपदानां कृद्भिः सह समासवचनं प्राक् सुबुत्पत्तेः' इति परिभाषया भवति सुबुत्पत्तेः पाक्, अत्रोत्तरपदे सुबुत्पत्तेः प्रागित्यर्थात्~।
अन्यथा `चर्म्मक्रीती' इत्यादौ नलोपानापत्तेः~।
 सुपां धातुना-उत्तरपदं धातुमात्रम्, न तिङन्तम्, `कटप्रुः' `आयतस्तूः' "क्विब्वचिप्रच्छयायतस्तुकटप्रुजुश्रीणां दीर्घश्च" इति वार्त्तिकात्~।
तिङां तिङा-`पिबतखादता' `पचतभृज्जता' इत्यादिः, "आख्यातमाख्यातेन क्रियासातत्ये" इति मयूरव्यंसकाद्यन्तर्गणासूत्रात्~।
तिङां सुबन्तेन-पूर्वपदं तिङन्तमुत्तरं सुबन्तम्-`जहिस्तम्बः', "जहि कर्मणा बहुलमाभीक्षण्ये कर्तारञ्चाभिदधाति' इति मयूरव्यंसकाद्यन्तगणसूत्रात्~।
 अयं षड्विधोऽपि समासः "सह सुपा"\footnote{२-१-४} इत्यत्र योगविभागेन भा,्ये व्युत्पादितः, स्पष्टः शब्दकौस्तुभादौ॥२८॥
 स्वयं भाष्यादिसिद्धं तद्भेदं व्युत्पाद्य प्राचीनवैयाकरणोक्तविभागस्याव्याप्त्यतिव्याप्त्यादिभिस्तल्लक्षणास्य प्रायिकत्वं दर्शयति--
\begin{center}{\bfseries समासस्तु चतुर्द्धेति प्रायोवादस्तथा परः~।\\
 योऽयं पूर्वपदार्थादिप्राधान्यविषयः स च॥२९॥\\[10pt]
 भौतपूर्व्यात् सोऽपि रेखागवयादिवदास्थितः॥२९ १/२॥}\end{center}
 चतुर्धा, अव्ययीभाव-तत्पुरुष-द्वन्द्व-बहुव्रीहिभेदात्~।
अयं प्रायोवादः, `भूतपूर्वः', `दृन्भूः', {`काराभूः'} `आयतस्तूः', `वागर्थाविव' इत्याद्यसङ्गहात्~।
 तथा पूर्वपदार्थप्रधानोऽव्ययीभावः~।
उत्तरपदार्थप्रधानस्तत्पुरुषः~।
उभयपदार्थप्रधानो द्वन्द्वः~।
अन्यपदार्थप्रधानो बहुव्रीहिरित्यादिलक्षणमपि प्रायिकम्, `उन्मत्तगङ्गम् `सूपप्रति' `अर्द्धपिप्पली' द्वित्राः' `शशकुशपलाशम्' इत्यादौ परस्परव्यभिचारात्~।
तथाहि-`उन्मत्तगङ्गम्' इत्यव्ययीभावे पूर्वपदार्थप्राधान्याभावादव्याप्तिः, अन्यपदार्थप्राधान्याद् बहुव्रीहिलक्षणातिव्याप्तिश्च, "अन्यपदार्थे च संज्ञायाम्"\footnote{२-१-२१} इति समासात्~।
`सूपप्रति' इत्यव्ययीभावे उत्तरपदार्थप्राधान्यात्तत्पुरुषलक्षणातिव्याप्तिः, अव्ययीभावाव्याप्तिश्च~।
`सुप् प्रतिना मात्रार्थे'\footnote{२-१-९} इति समासात्~।
`अर्द्धपिप्पली' इति तत्पुरुषे पूर्वपदार्थप्राधान्यसत्त्वाद् अव्ययीमावातिव्याप्तिः, तत्पुरुषाव्याप्तिश्च, " अर्द्धं नपुंसकम्"\footnote{२-२-२} इति समासात्~।
 एवम् `पूर्वकायः' इत्यादौ द्रष्टव्यम्~।
`द्वित्राः' इति बहुव्रीहावुभयपदार्थप्राधान्याद्, द्वन्द्वातिव्याप्तिः, बहुव्रीह्यव्याप्तिश्च~।
` शशकुशपलाशम्' इत्यादिद्वन्द्वे समाहारान्यपदार्थप्राधान्याद् बहुव्रीह्यतिव्याप्तिः, द्वन्द्वातिव्याप्तिः, स्यादिति भावः~।
 सिद्धान्ते तु `अव्ययीमावादिकारपठितत्वमव्ययीभावत्वम्' इत्यादि द्रष्टव्यम्~।
असम्भवश्चैषामित्याह-भौतपूर्व्यादित्यादि-रेखागवयादिनिष्ठलाङ्गूलादेर्वास्तवपश्वलक्षणत्ववदेतेषामपि न समासलक्षणत्वम्~।
बोधकता तु तद्वदेव स्यादिति भावः॥२९ १/२॥
 ननु पूर्वपदार्थप्राधान्यादि समासे सुवचम्~।
तथाहि-`समर्थः पदविधि' इति सूत्रे भाष्यकारैरनेकधोक्तेष्वपि पक्षेषु जहत्स्वार्थाऽजहत्स्वार्थपक्षयोरवैकार्थोभाव-व्यपेक्षारूपयोः पर्य्यवसानं लभ्यते~।
तत्राजहत्स्वार्थापक्षे उक्तव्यवस्था नासम्भविनीत्याशङ्कां मनसिकृत्य आह--
\begin{center}{\bfseries जहत्स्वार्थाजहत्स्वार्थे द्वे वृत्ती ते पुनस्त्रिधा॥३०॥\\
 भेदः संसर्ग उभयं चेति वाच्यब्यवस्थितेः॥३० १/२॥}\end{center}

 जहति पदानि स्वार्थं यस्यां सा जहत्स्वार्था~।
पदे वर्णवद् वृत्तौ पदानामानर्थक्यमित्यर्थः~।
अयं भावः- समासशक्त्यैव राजविशिष्टपुरुषभान {बोध} सम्भवे न राजपुरुषपदयोरपि पुनस्तद्बोधकत्वं कल्प्यम्, वृषभयावकादिपदेषु वृषादिपदानामिव~।
 अन्यथा राजपदेन विग्रहवाक्य इव, राज्ञः स्वातन्त्र्येणोस्थितिसत्त्वात्, `ऋद्धस्य राज्ञः पुरुष इत्यत्रेव `ऋद्धस्य राजपुरुषः इत्यस्याप्यापत्तेरिति~।
 अजहदिति-न जहति पदानि स्वार्थं यस्यां सा अजहत्स्वार्था~।
 अयमभिप्रायः-राजपुरुषादिसमासादौ नातिरिक्ता शक्तिः, कल्पकाभावात्, क्लृप्तराजादिपदादेवार्थोपस्थितिसम्भवे लत्कल्पनस्य गौरवपराहतत्त्वाच्च~।
कलॄप्तशक्तित्यागोऽप्यप्रामाणिकः कल्प्येत~।
तथाचाऽऽकाङ्क्षादिवशात् कलॄप्तशक्त्यैव विशिष्टार्थबोधः~।
 अयमेव व्यपेक्षापक्षो मतान्तरत्वेन भाष्यकारैरुक्तः~।
 नचात्र मते समासे `ऋद्धस्य' इति विशेषणान्वयापत्तिः, "सविशेषणानां वृत्तिर्न, वृत्तस्य वा विशेषणयोगो न" इति वार्त्तिकात्~।
तथाचैतन्मतवादिनां पूर्वोत्तरपदार्थसत्त्वाद् पूर्वपदार्थप्रधान इत्यादिव्यवस्था सूपपदेति भावः~।
 प्रसङ्गाद् वृत्तिभेदमपि निरूपयति-ते पुनरिति-द्वे अपि वृत्ति त्रिविधे, वाच्यत्रैविध्यात्~।
वाच्यमेवाह-भेद इत्यादि~।
भेदः=अन्योन्याभावः~।
तथाच `राजपुरुषः' इत्यादावराजकीयभिन्न इति बोधः~।
अस्यावाच्यत्वे च `राजपुरुषः सुन्दरः' इतिवत्, राजपुरुषो देवदत्तस्य' चेत्यपि स्यात्~।
वाच्यत्वे तद्विरोधान्नैवं प्रयोग इति भावः~।
 `राजसम्बन्धवान्' इत्येव शाब्दं भानम्, भेदस्तूत्तरकालमुपतिष्ठते इत्याशयेनाह-संसर्ग इति~।
विनिगमनाविरहम्, अस्वामिकेऽपि `राजपुरुषः' इत्यादिप्रयोगापत्तिञ्च मनसि कृत्वोभयं वाच्यमित्याह-अभयं वेति~।
तथाच `अराजकीयभिन्नो राजसम्बन्धवाँश्चायम्' इति बोधः॥३० १/२॥
 व्यपेक्षावादस्यैवं युक्तिभाष्यविरुद्धत्वात् तन्मूलकः `पूर्वपदार्थप्रधानः' इत्याद्युत्सर्गोऽप्युक्तः~।
किन्तु `रेखागवय' न्यायेनोऽत्सर्गोऽपि परम्परयैव बोधक इत्याशयेन समाधत्ते--
\begin{center}{\bfseries समासे खलु भिन्नैव शक्तिः पङ्कजशब्दवात्॥३१॥\\[10pt]
 बहूनां वृत्तिधर्म्माणां वचनैरेव साधने~।\\
 स्यान्महद् गौरवं तस्मादेकार्थाभाव आश्रितः॥३२॥}\end{center}
 समास इति-वृत्तिमात्रोपलक्षणम्, "समर्थः" पदविधिः"\footnote{२-१-१} इत्यत्र पदमुद्दिश्य यो विधीयते-समासादिः स समर्थः-विग्रहवाक्यार्थाभिधाने शक्तः सन् साधुरिति सूत्रार्थस्य भाष्याल्लाभात्~।
 पदोद्देश्यकविधित्वञ्च-कृत्तद्धितसमासैकशेष सनाद्यन्तधातुरूपासु पञ्चस्वपि वृत्तिष्वस्त्येव~।
विशिष्टशक्तयस्वीकर्तॄणां व्यपेक्षावादिनां मते दूषणं शक्तिसाधकमेवेत्याशयेनाह-पङ्कजशब्दवदिति-पङ्कजनिकर्तुरपि योगादेवोपस्थितौ तत्रापि समुदायशक्तिर्न सिद्धयेत्~। 
 नच पद्मत्वरूपेणोपस्थितये सा कल्प्यत इति वात्यम्, चित्रग्वादिपदेऽपि स्वामित्वेनोपस्थितये तत्कल्पनाऽऽवश्यकत्वात्~।
लक्षणयैव तथोपस्थितिरिति चेत्, पङ्कजपदेऽपि सा सुवचा~।
एवं रथकारपदेऽपि~।
तथाच "वर्षासु रथकारोऽग्निमादधीत" इत्यत्रापि विना लक्षणां क्लृप्तयोगेन ब्राह्मणादिविषयतयैवोपपत्तौ तत्कल्पनां कृत्वा जातिविशेषस्याधिकारित्वं प्रकल्प्यापूर्वाविद्याकल्पनमयुक्तं स्यादिति भावः~।
 साधकान्तरमाह-बहूनामिति-वृत्तोर्धर्म्माः विशेषणलिङ्गसङ्खयाद्ययोगादयस्तेषां वचनैः {एव} साधने गौरवमित्यर्थः~।
 अयं भावः विशिष्टशक्तयस्वीकारे `राज्ञः पुरुषः' इत्यत्रेव `राजपुरुषः' इत्यत्रापि स्याद् विशेषणाद्यन्वयः, राजपदेन स्वतन्त्रोपस्थितिसत्त्वात्~।
विभाषावचनञ्च सामासनियमवारणाय कार्य्यामिति~।
 ननु "सविशेषणानाम्" इति वचनान्न विशेषणाद्यन्वयः, विभाषावचनञ्च कृतमेवेत्याशङ्कां समाधत्ते-वचनैरेवेति~।
न्यायसिद्धमेव सूत्रम्~।
व्यापेक्षाविवक्षायां {व्यासस्य} वाक्यस्य, एकार्थाभावे समासस्येति स्वभावादेव प्रयोगनियमसम्भवात्~।
सविशेषणोत्यपि विशिष्टशक्तौ राज्ञः पदार्थैकदेशतयाऽन्वयासम्भवान्नयायसिद्धिमिति भावः~।
 अत एव व्यपेक्षापक्षरमुद्भाव्य "अथैतस्मिन् व्यपेक्षायां सामर्थ्ये योऽसावेकार्थोभावकृतो विशेषः स वक्तव्यः" इति भाष्यकारेण दूषणमप्युक्तम्॥३२॥
 तथा `धवखदिरौ' ` निष्कौशाम्बिः' `गोरथः' `घृतघटः' `गुडधानाः' `केशचूडः' `सुवर्णालङ्कारः' `द्विदशाः' `सप्तपर्णाः' इत्यादावितरेतरयोग-निष्क्रान्त-युक्त-पूर्णा-मिश्र-सङ्घात-विकार-सुच्प्रत्ययलोप-वीप्साद्यधर्थो वाचनिको वाच्य इत्यतिगौरवं स्यादिति दूषणान्तरमाह--
\begin{center}{\bfseries चकारादिनिषेधोऽथ बहुव्युत्त्पत्तिभञ्जनम्~।\\
कर्तव्यं ते न्यायसिद्धं त्वस्माकं तदिति स्थितिः॥३३॥ }\end{center}
 आदिना घनश्यामः' `हंसगमनः' इत्यादाविवादीनां पुर्वोक्तानाञ्च संग्रहः~।
दूषणान्तरमाह-बहुयुत्त्पत्तिभञ्जनमिति~।
अयमाशयः- `चित्रगुः' इत्यत्र स्वाम्यादिप्रतीतिरनुभवसिद्धा~।
 नच तत्र लक्षणा, `प्राप्तोदको ग्रामः' इत्यादौ तदसम्भवात्, `प्राप्तकर्त्रभिन्नमुदकम्' इत्यादिबोधोत्तरं तत्सम्बन्धिग्रामलक्षणायामपि `उदककर्तृकप्राप्तिकर्म ग्रामः' इत्यर्थालाभात्~।
 प्राप्तेति क्तप्रत्ययस्यैव कर्त्रर्थकस्य कर्मणि लक्षणेति चेत्तहि समानाधिकरणप्रातिपदिकार्थयोरभेदान्वयव्युत्त्पत्तेरुदकाभिन्नप्राप्तिकर्मेति स्यात्, अन्यथा समानाधिकरणप्रातिपदिकार्थयोरभेदान्वयव्युत्पत्तिभङ्गापत्तेः, प्राप्तेर्धात्वर्थया कर्तृतासम्बन्धेन भेदेनोदकस्य तत्रान्वयासम्भवाच्च~।
अन्यथा `देवदत्तः पच्यते' इत्यत्र कर्तृतासम्बन्धेन देवदत्तस्यान्वयसम्भवेनानन्वयानापत्तेः~।
 अथोदकाभिन्नकर्तृका प्राप्तिरिति बोधोत्तरं तत्सम्बन्धिग्रामो लक्षयते इति चेन्न, प्राप्तेर्धात्वर्थया क्तार्थकर्तारं प्रति विशेष्यताया असम्भवात्, `प्रकृतिप्रत्ययार्थयोः प्रत्ययार्थस्यैव प्राधान्यम्' इति व्युत्पत्तेः प्राप्तपदे प्राप्तेर्विशेष्यत्वे तस्या एव नामार्थत्वेनोदकेन सममभेदान्वयापत्तेश्च~।
 एवम् `ऊढरथः' `उपदहृतपशुः' ` उद्धृतौदना' `बहुपाचिका' इत्यादावपि द्रष्टव्यम्~।
अत्र हि रथकर्मकवहनकर्त, पशुकर्मकोपहरणोद्देशयः, ओदनकर्मकोद्धरणावधिः, बहुपाककर्त्रधिकरणमिति बोधाभ्युपगमात्~।
अतिरिक्तवृत्तिपक्षे च घटत्वविशिष्टे घटपदस्येवोदककर्तृकप्राप्तिकर्मत्वविशिष्टे प्राप्तोदक इत्यादिसमुदायसक्तयैव निर्वाह इति भावः॥३३॥

साधकान्तरमाह-
\begin{center}{\bfseries अषष्ठयर्थबहुव्रीहौ व्युत्त्पत्त्यन्तरकल्पना~।\\
 क्लृप्तत्यागश्चास्ति तव तत् किं शक्तिं न कल्पयेः॥३४॥}\end{center}
 
 अयं भावः~।
चित्रगुरित्यादिषु चित्रगवीणां स्वाम्यादिप्रतीतिर्न विना शक्तिमुपपद्यते~।
नच तत्र लक्षणा~।
सा हि न चित्रपदे,'चित्रस्वामी गौः' इति बोधापत्तेः~।
नापि गोपदे, गोस्वामी चित्रः इत्यन्वयबोधापत्तेः~।
चित्रादिमात्रस्य लक्षयैकदेशत्वेन तत्र गवादेरन्वयायोगात्~।
 नच `चित्राभिन्ना गौः' इति शक्तयुपस्थाप्ययोरन्वयबोधोत्तरं तादृशगोस्वामी गोपदेन लक्षयते इति वाच्यम्, गोपदस्य, चित्रपदस्य वा विनिगमनाविरहेण लक्षकत्वासम्भवात्~।
 नच गोपदे साक्षात् सम्बन्ध एव विनिगमक इति वाच्यम्, एवमपि `प्राप्तोदकः' `कृतविश्वः' इत्याद्यषष्ठयर्थबहुव्रीहौ विनिगमकाप्राप्तेः~।
यौगिकानां कर्त्रद्यर्थकतया साक्षात्सम्बन्धाविशेषत्~।
 नच `पदद्वये लक्षणा' इति नौयायिकोक्तं युक्तम्, बोधावृत्तिप्रसङ्गात्~।
नच परस्परं तात्पर्य्यग्राहकत्वादेकस्यौवैकदा लक्षणा' न द्वयोरिति न बोधाकृत्तिरिति वाच्यम्, एवमपि विनिगमनाविरहतादवस्थ्येन लक्षणाया असम्भवात्~।
 नच चरमपदे एव सा, प्रत्ययार्थान्वयानुरोधात्, प्रत्ययानां सन्निहितपदार्थ गतस्वार्थबोधकत्वव्युत्पत्तेरिति वाच्यम्, एवं हि वहुव्रीह्यसम्भवापत्तेः~।
"अनेकमन्यपदार्थे"\footnote{२-२-४}इत्यनेकसुबन्तानामन्यपदार्थप्रतिपादकत्वेन तद्विधानात्~।
 किञ्चैवं सति घटाऽऽदिपदेष्वपि चरमवर्ण एव वाचकताकल्पना स्यात्, पूर्वपूर्ववर्णानां तात्पर्य्यग्राहकत्वेनोपयोगसम्भवात्~।
एवं सति चरमवर्णमात्रश्रवणेऽर्थबोधापत्तिरिति चेद् अत्राप्युदकपदमात्रश्रवणादर्थप्रत्ययापत्तिस्तुल्येत्यन्यत्र विस्तरः~।
 एवञ्च, अषष्ठयर्थबहुव्रीहौ व्युत्पत्त्यन्तरकल्पना, उक्तयुक्ततेः~।
अगत्या शक्तयन्तरकल्पनेत्यर्थः~।
क्लृप्तत्याग इत्यस्य क्लृप्तशक्तयोपपत्तिरिति व्युत्पत्तित्यागश्च तवास्ति~।
तत् किं सर्वत्र समासे शक्तिं न कल्पयेरिति वाक्यार्थः~।
यत्तु व्यपेक्षावादिनो नैयायिक-मीमांसकादयः~।
न समासे शक्तिः, 'राजपुरुषः' इत्यादौ राजपदादेः सम्बन्धिलणयैव `राजसम्बन्ध्यभिन्नः पुरुषः' इति बोधोपपत्तेः~।
 अत एव राज्ञः पदार्थैकदेशतया न तत्र `शोभनस्य' इत्यादिविशेषणान्वयः~।
नवा घनश्यामः, `निष्कौशाम्बिः' `गोरथः' इत्यादौ इवादिप्रयोगापत्तिः, उक्तार्थतयैव क्रान्तादिपदप्रयोगासम्भवात्~।
न वा "विभाषा"\footnote{२-७-११} इति सूत्रावश्यकत्वम्, लक्षणया `राजसम्बन्ध्यभिन्नः पुरुषः' इति बुबोधयिषायां समासस्य, राजसम्बन्दवानिति बुबोधयिषायां विग्रहस्येत्यादिप्रयोगनियमसम्भवात्~।
नापि पङ्कजपदप्रतिबन्दी शक्तिसाधिका, तत्रावयवशक्तिमजानतोऽपि बोधात्~।
 नच शक्तयग्रहे लक्षणया तेभ्यो विशिष्टार्थप्रत्ययः सम्भवति~।
अत एव राजादिपदशक्तयग्रहे `राजपुरुषः' `चित्रगुः' इत्यादौ न बोधः|नापि {नच} चित्रगुरित्यादौ लक्षणासंभवेऽप्यषष्ठयर्थबहुव्रीहौ लक्षणाया असम्भवः बहुव्युत्पत्तिभञ्जनापत्तेरिति वाच्यम्, `प्राप्तोदकः' इत्यादावुदकपदे एव लक्षणास्वीकारात्~।
पूर्वपदस्य यौगिकत्वेन तल्लक्षणाया धातुप्रत्ययतदर्थज्ञानसाध्यतया विलम्बितत्वात्, प्रत्ययानां सन्निहितपदार्थगतस्वार्थबोधकत्वव्युत्पत्त्यनुरोधाच्च~।
 घटादिपदे चातिरिक्ता शक्तिः कल्प्यमाना विशिष्टे कल्प्यते, विशिष्टस्यैव सङ्केतितत्वात्~।
बोधकत्वस्यापि प्रत्येकं वर्णाष्वसत्त्वात्~।
प्रकृते चात्यन्तसन्निधानेन प्रत्ययार्थान्वयसौलभ्यायोत्तरपदे एव सा कल्प्यत इति विशेषः~।
 स्वीकृतञ्च घटादिपदेष्वपि चरमवर्णास्यैव वाचकत्वं मीमांसकम्मन्यैरित्याहुः~।
 अत्रोच्यते-समासे शक्त्यस्वीकारे तस्य प्रातिपदिकसंज्ञादिकं न स्यात्, अर्थवत्त्वाभावात्, "अर्थवदधातुरप्रत्ययः प्रातिपदिकम्"\footnote{१-२-४५} इत्यस्याप्रवृत्तेः~।
नच "कृत्तद्धितसमासाश्च"\footnote{१-२-४६} इत्यत्र समासग्रहणात् सा, तस्य नियमार्थताया भाष्यसिद्धाया वैयाकरणभूषणे स्पष्टं प्रतिपादीतत्वात्~।
 समासवाक्ये शक्त्यभावेन शक्यसम्बन्धरूपलक्षणाया अप्यसम्भवेन लाक्षणिकार्थवत्त्वस्याप्यसम्भवात्~।
अथ "तिप्तस्झि' इत्यारभ्य "ङयोस्सुप्" इति तिप्प्रत्याहारो भाष्यसिद्धस्तमादाय "अतिप्प्रातिपदिकम्" इत्येव सूत्र्यताम्, कृतसर्थवदादिसूत्रद्वयेन~।
समासग्रहणञ्च नियमार्थमस्तु~।
तथाच अतिप्=सुप्तिङन्तभिन्नं प्रातिपदिकमित्यर्थात् समासस्यापि सा स्यादिति चेत्तथापि प्रेत्येकं वर्णेषु संज्ञावारणायार्थवत्त्वाऽऽवश्यकत्वेन समासाव्याप्तितादवस्थ्यमेव~।
तथाच प्रातिपदिकसंज्ञारूपं कार्यमेवार्थवत्त्बमनुमापयति-धूम इव वह्निम्~।
 किञ्चैवं `चित्रगुमानय' इत्यादौ कर्मत्वाद्यनन्बयापत्तिः, प्रत्ययानां प्रकृत्यर्थान्बितस्वार्थबोधजनकत्वव्युत्पत्तेः विशिष्टोत्तरमेव प्रत्ययोत्पत्तेर्विशिष्टस्यैव प्रकृतित्वात्~।
यत्तु सन्निहितपदार्थगतस्वार्तबोधकत्वव्युत्पत्तिरेव कल्प्यत इति, तन्न, `उपकुम्भम्' `अर्द्धपिप्पली' इत्यादौ पूर्वपदार्थो विभक्त्यर्थान्वयेन व्यभिचारात्~।
 नच तत्रापि सन्निधानमेव, आनुशासनिकसन्निधेर्विवक्षितत्वात्~।
तथाच यत्पदोत्तरं याऽनुशिष्टा सा तदर्थगतं स्वार्थं बोधयति~।
समासे च समस्यमानपदोत्तरमेवानुशासनमिति वाच्यम्, अर्थवत्सूत्रेण विशिष्टस्यैव प्रातिपदिकत्वेन विशिष्टोत्तरं विभक्त्यनुशासनात्~।
 अथ `प्रकृतित्वाश्रये विभक्त्यर्थान्वयः' इत्येव कल्प्यत इति चेत्तहिं, `पङ्कजमानय' `दण्हिनं पश्य' `शूलिनं पूजय' इत्यादौ पङ्क-दण्ड-शूलेष्वानयन-दर्शन-पूजनादेरन्वयप्रसङ्गात्, `अघटमानय' इत्यत्र घटेऽप्यानयनान्वयापत्तेश्च~।
नच दण्डादीनां विशेषणतया न तत्रानयनाद्यन्वयः, `पाकान्नीलः' `धर्मात् सुखी' इत्यादौ पाक-धर्मादिहेतुताया रूप-सुखादावनन्वयप्रसङ्गात्~।
यच्च-प्रकृत्यर्थत्वं तज्जन्यज्ञानविषयत्वमात्रम्, तच्चात्राविरुद्धमिति, तन्न, `घटं पश्य' इत्यत्र घटपदात् समवायेनोपस्थिताकाशवारणाय वृत्त्या प्रकृत्यर्थत्वस्यावश्यकत्वात्~।
 अथ `प्रत्ययप्राग्वर्त्तिपदजन्योपस्थितिविशेष्यत्वं प्रकृत्यर्थत्वम्' इति चेन्न, `गामानयति कृष्णो दण्डेन' इत्यत्र कृष्णो तृतीयार्थान्वयप्रसङ्गात्~।
 अथ समस्यमानपदार्थगतस्वार्थबोधकत्वं समासोत्तरविभक्तेः कल्प्यत इति चेन्न, अक्लृप्तकल्पनां क्लृप्तव्युत्पत्तित्यागञ्चापेक्षय समुदायशक्तिकल्पनस्यैव युक्तत्वादिति दिक्~।
 अपिच समासे विशिष्टशक्त्यस्वीकारे `राजपुरुषः' `चित्रगुः' `नीलोत्पलम्' इत्यादौ सर्वत्रानन्वयप्रसङगः, राजपदादेः सम्बन्धिनि लक्षणायामपि, `तण्डुलः पचति' इत्यादौ कर्मत्वादिसंसर्गेण तण्डुलादेः पाकादावन्वयवारणाय प्रातिपदिकार्थप्रकारकबोधं प्रति विभक्तिजन्योपस्थितेर्हेतुताया आवश्यकत्वात्, पुरुषादेस्तथात्वाभावात्~।
`तण्डुलः शुभ्रः' इत्यादौ च प्रातिपदिकार्थकप्रथमार्थे तण्डुलादेः, तस्य च शुक्ले अभेदेनैवान्वयः~।
 `शुभ्रेण तण्डुलेन' इत्यादौ च विशेषणविभक्तिरभेदार्थिका, पार्ष्ठिको वाऽन्वय इति नातिप्रसङ्गः~।
तथाच समासे परस्परमन्वयासम्भवादावश्यिकैव समुदायस्य तादृशे विशिष्टार्थे शक्तिः~।
 किञ्च `राजपुरुषः' इत्यादौ सम्बन्धिनि, सम्बन्धे वा लक्षणा~।
नाद्यः, `राज्ञः पुरुषः' इति विवरणविरोधात्, समाससमानार्थकवाक्यस्यैव विग्रहत्वात्~।
अन्यथा तस्माच्छक्तिनिर्णयो न स्यात्~।
नान्त्यः `राजसम्बन्धरुपः पुरुषः' इति बोधप्रसङ्गात्~।
विरुद्धविभक्तिरहितप्रातिपदिकार्थयोरभेदान्वयव्युत्पत्तेरित्यादि प्रपञ्चितं वैयाकरणभूषणे~।
 अत एव "वषट्कर्तुः प्रथमभक्षः" इत्यत्र न भक्षमुद्दिश्य प्राथम्यविधानं युक्तम् एकप्रसरताभङ्गापत्तेरिति तृतीये, "त्र्यङ्गैः स्विष्टकृतं यजति" इत्यत्राङ्गानुवादेन त्रित्वविधानं न युक्तम्, एकप्रसरताभङ्गापत्तेरिति दशमे च निरूपितं सङगच्छते, सङ्गच्छते चारुणाधिकरणारम्भः~।
 अन्यथा "अरुणया एकहायन्या पिङ्गाक्षया सोमं क्रीणाति" इत्यत्रारुण्यपदवदितरयोरपि एकाब्दत्वादिगुणमात्रवाचकतया अमूर्त्तत्वात् क्रीणातौ करणत्वासम्भवस्य तुल्यत्वादारुण्यस्यैव वाक्याद्भेदशङ्काया असम्भवादिति प्रपञ्चितं भूषणे~।
तस्मात् समासशक्तिपक्षो जैमिनीयैरवश्याभ्युपेय इत्यास्तां विस्तरः॥३५॥
 `राजपुरुषः' इत्यादौ राजा चासौ पुरुषश्चेत्येव विग्रहः~।
चित्रगुरित्यादौ चित्राणां गवामयमित्येव, समानार्थत्वानुरोधात् यद्यपि प्रथमान्तानामेव बहुव्रीहिरिति "शोषो बहुव्रीहिः" {पा.सृ.२-२-२३} इति सूत्राल्लभ्यते इति प्रथमान्तम्, पक्षे वाक्यम्-चित्रा गावो यस्येत्येवं सम्भवत्येव~।
"षष्ठी"\footnote{२-२-८} इति समासविधानात् `राज्ञः पुरुषः' इति च पक्षे वाक्यम्~।
तथापि तस्य न विग्रहत्वम्, भिन्नार्थत्वात्, किन्तूक्तस्यैवेति मीमांसकास्तान्प्रसङ्गान्निरस्यति-
\begin{center}{\bfseries आख्यातं तद्धितकृतोर्यत्किञ्चिदुपदर्शकम्~।\\
 गुणप्रधानभावादौ तत्र दृष्टो विपर्ययः॥३६॥}\end{center}
 तद्धितकृतोर्यात्किञ्चिदर्थबोधकं विवरणमाख्यातम्, तत्र विपर्य्ययो दृष्टः~।
तथाहि-`आक्षिकः" "कुम्भकारः" इत्यत्राक्षकरमकव्यापाराश्रयः, कुम्भोत्पत्त्यनुकूलव्यापाराश्रय इति बोधः~।
`अक्षैर्दिव्यति' `कुम्भं करोति' इत्यत्राक्षकरणिका देवनानुकूला भावना, कुम्भोपत्तयनुकूला भावनेति बोधः~।
कृत्प्रत्यये कारकाणाम्, आख्याते च भावनायाः प्राधान्यं वदतो मीमांसकस्यापि गुणप्रधानभावांशव्यत्यासो न विवरणत्वबाधक इति नात्र पाक्षिकस्य, `चित्रा गावो यस्य' इत्यादेर्विग्रहत्वे बाधकमस्तीति भावः॥३६॥
 नन्वस्तूक्तरीत्या सर्वत्र समासे शक्तिः, `अस्तु च तथाविग्रहस्तथापि षष्ठीतत्पुरुष-कर्मधारययोः शक्तितमत्त्वाविशेषान्निषादस्थपत्यधिकरणसिद्धान्तसिद्धिर्न स्यादित्यत आह--
\begin{center}{\bfseries पर्यवस्यच्छाब्दबोधाविदूरप्राक्क्षणस्थितेः~।\\
 शक्तिग्रहेऽन्तरङ्गत्वबहिरङ्गत्वचिन्तनम्॥३७॥}\end{center}

 पर्य्यवस्यंश्चासौ शाब्दबोधश्च तस्मादविदूरश्चासौ प्राक्क्षणश्च तदानीन्तनलाधवमादायाधिकरणाविरोध इत्यर्थः~।
अयं भावः-निषादस्थपतिपदस्य समासशक्तितपक्षे, निषादरूपे, निषादानाञ्च स्थपतौ, निषादस्वमिके पुरुषान्तरे चेत्येवं सर्वत्र शक्तत्वान्नानार्थत्वम्~।
तथाच "नानार्थे तात्पर्याद् विशेषावगतिः" इति न्यायेन तत्कल्पनायां पदद्वयेन पूर्वोपस्थितार्थे एवोपस्थित्यादिलाघवात् तत् कल्प्यत इति~।
परेषामपि सति तात्पर्य्ये `यष्टीः प्रवेशय' इतिवल्लक्षणाया दुर्वारत्वात्, तात्पर्य्यमेव कल्प्यकोटाववशिष्यत इति दिक्॥३६॥
\begin{center}इति वैयाकरणभूषणसारे समासशक्तिनिरूपणम्॥५॥\end{center}

\section*{\begin{center}अथ शक्तिनिर्णयः\end{center}}\addcontentsline{toc}{section}{शक्तिनिर्णयः}
\fancyhead[LE,RO]{शक्तिनिर्णयः}
शक्तिप्रसङ्गात् तस्याः स्वरूपमाह--
\begin{center}{\bfseries इन्द्रियाणां स्वविषयेष्वनादिर्योग्यता यथा~।\\
 अनादिरर्थैः शब्दानां सम्बन्धो योग्यता तथा॥३७॥}\end{center}

 इन्द्रियाणाम्=चक्षुरादीनां स्वविषयेषु=चाक्षुषेषु घटादिषु यथाऽनादिर्योग्यता=तदीयचाक्षुषादिकारणता, तथा शब्दानामपि अर्थैः सह तद्बोधकारणतैव योग्यता, सैव शक्तिरित्यर्थः~।
 ननु न बोधकारणत्वमनादिभूतं शक्तिः, आधुनिकदेवदत्तादिपदे तदभावात्~।
 अन्यथा पित्रादीनां सङ्केताज्ञानेऽप्यन्वयबोधप्रसङ्गः, लाक्षणिकातिव्याप्तिश्चेति सङ्केतज्ञानमपि हेतुर्वाच्यम्~।
तथाचावश्यकत्वात् स एव शक्तिरस्तु~।
 स चाधुनिके पित्रादेः, गवादौ चेश्वरस्य चेति चेत्, अत्रोच्यते-सङ्केतो न स्वरूपेण हेतुः, अगृहीतशक्तिकादर्थबोधप्रसङ्गात्~।
 नापि सामान्यतो ज्ञातः, प्रमेयत्वादिना तज्ज्ञानेऽपि बोधप्रसङ्गात्~।
नाऽपि सङ्केतत्वेन तज्ज्ञानं हेतुः, गवादिपदेष्वपीश्वरादेः सङ्केतत्वेन तज्ज्ञानशून्यानां लौकिकमीमांसकादीनां तत्तदर्थबोधजनकत्वग्रहवतामेव बोधोदयेन व्यभिचारात्~।
नचार्थधीजनकतावच्छेदकत्वेन तज्ज्ञानं तथा, ततोऽपि लाघवेनार्थधीजनकत्वेनैव हेतुतायामस्मत्पत्रसिद्धेः~।
नचाधुनिकदेवदत्तादौ सङ्केतज्ञानादेव बोधेनास्य व्यभिचारः, तत्रापि "इदम्पदम् एनमर्थं बोधयतु" इतिच्छाग्रहेपदे तदर्थबोधकत्वस्यावगाहनेन व्यभिचाराभावात्~।
 नच स्वातन्त्र्येणार्थबोधकताज्ञानं कारणं बाच्यम्~।
अन्यथा `नेदं तद्धीजनकम्' इति ज्ञानवतः `अस्माच्छब्दादयमर्थो बुद्धोऽनेन' इति जानतस्तद्ग्रहापत्तेरिति वाच्यम्, नेदं तद्धीजनकमिति ग्रहवते बाधेन पदे परग्रहं जानतोऽपि तद्ग्रहासम्भवात्~।
अन्यथा भ्रान्तिज्ञस्यापि भ्रान्तत्वापत्तेरिति~।

 इदञ्चार्थधीजनकत्वं पित्रादिसङ्गेतज्ञानादेव गृह्यते, अतस्तज्ज्ञानात् पूर्वं न बोधः~।
नापि लाक्षणिकोच्छेदापत्तिः, इष्टत्वात्, शक्तिग्राहकव्यवहारस्य मुख्य-लक्षयसाधारण्यात्~।
 किञ्च प्रत्यक्षादिजन्योपस्थितेः शाब्दबोधानङ्गत्वाच्छाब्दबोधं प्रति शक्तितजन्योपस्थितेः, लक्षणाजन्योपस्थितेश्च कारणत्वं वाच्यम्, तथाच कार्य्यकारणभावद्वयकल्पने गौरवं स्यत्~।
अस्माकं पुनः शक्तितजन्योपस्थितित्वेनैव हेतुतेति लाघवम्~।
 अपिच लक्षणावृत्तिस्वीकारे कार्यकारणभावस्य प्रत्येकं व्यभिचारः, शक्तितजन्योपस्थितिं विनाऽपि लक्षणाजन्योपस्थितितः शाब्दबोधात्~।
नचाव्यवहितोत्तरत्वसम्बन्धेन तत्तदुपस्थितिमत्त्वं कार्य्यतावच्छेदकम्, तत्तदुपस्थितित्वञ्च कारणतावच्छेदकम्, अनन्तकार्य्यकारणभावप्रसङ्गात्~।
किञ्च पदार्थोपस्थितिं प्रत्यपि शक्तिज्ञानत्वेन, लक्षणाज्ञानत्वेन च हेतुतेति व्यभिचारः, गौरवञ्च प्राग्वदेव द्रष्टव्यम्~।
नच `इदं पदमेतदर्थबोधकम्' इति शक्तिज्ञानेन कार्य्यकारणभावकल्पनेऽपि तत्तदर्थभेदेनानेककार्य्यकारणभावकल्पने गौरवं तवापि समानम्, परस्परव्यभिचारवारणायाव्यवहितोत्तरत्वघटितत्वे च सुतरामिति वाच्यम्, शक्तिभ्रमानुरोधेन तत्तत्पदतत्तदर्थभेदेन कार्य्यकारणभावानन्त्यस्य तवापि साम्यात्~।
लक्षणाकार्य्यकारणभावकल्पनागौरवं परं तवातिरिच्यते~।
अथ वृत्तिजन्योपस्थितित्वेनैव शाब्दबोधहेतुता, वृत्तिज्ञानत्वेन च पदार्थोपस्थितिकारणतेत्येवं मया वाच्यमिति चेन्न, शक्ति-लक्षणान्यतरत्वस्य, शाब्दबोधहेतुपदार्थोपस्थित्यनुकूलपदपदार्थसम्बन्धत्वस्य वा वृत्तित्वस्य कारणतावच्छेदकत्वाच्छक्तित्वमपेक्षय गुरुत्वात्, शाब्दबोधहेतुताऽवच्छेदकपदार्थोपस्थितिहेतुवृत्तेरज्ञाने तद्घटितकार्य्यकारणभावग्रहस्याप्यसम्भवात्~।
 अथ ममापि शक्तिज्ञानत्वेनैव हेतुता, शक्यसम्बन्धज्ञानरूपलक्षणायां शक्तेरपि प्रवेशात्, इति चेन्न, शक्तिज्ञान-पदार्थोपस्थित्योः कार्य्यकारणभावे समानविषयत्वस्यावश्यकत्वात्~।
अन्यथा गङ्गा-तीरयोः सम्बन्धाग्रहवतो गङ्गापदशक्तिं जानतोऽपि `गङ्गायां घोषः' इति वाक्यात्तीरबोधप्रसङ्गः, शक्तिज्ञानस्य हेतोः सत्त्वात्~।
अपिच `घटमानय' इति वाक्यम्, हस्तिनञ्च स्मरतः, घटपदादिभ्यो घटादेः, गजाद्धस्तिपकस्य च समूहालम्बनस्मरणवतो घटानयनवद्धस्तिपकस्यापि शाब्दबोधापत्तिः, समूहालम्बनरूपायां पदार्थोपस्थितौ वृत्ति{पद} जन्यत्वसत्त्वात्~।
 तथाच `विषयतया शाब्दबोधं प्रति तदंशविषयकवृत्तिजन्योपस्थितिर्हेतुः' इति वाच्यम्, एवञ्च लक्षणाया अपि शक्तितज्ञानत्वेन हेतुत्वमसम्भवदुक्तिकमिति~।
 एतेन शक्तितप्रयोज्यैवोपस्थितिर्हेतुरिति न लक्षणाज्ञाने कार्य्यकारणभावान्तरं ममापीति परास्तम्, प्रयोज्यत्वस्यानतिप्रसक्तस्य दुर्वचत्वाच्चेत्यादि विस्तरेण प्रपञ्चितं भूषणे॥३७॥
 नन्वेवं भाषादितो बोधदर्शनाद् बोधकतरूपा शक्तिस्तत्रापि स्यात्~।
तथाच साधुताऽपि स्यात्, शक्तत्वस्यैव साधुताया व्याकरणाधिकरणे प्रतिपादनादित्याशङ्कां द्विधा समाधत्ते-
\begin{center}{\bfseries असाधुरनुमानेन वाचकः कैश्चिदिष्यते~।\\
 वाचकत्वाविशेषे वा नियमः पुण्यपापयोः॥३८॥}\end{center}
 असाधुः=गाव्यादिः, अनुमानेन=साधुशब्दमनुमाय, वाचकः=बोधकः कैश्चिदिष्यते~।
तथाच लिपिवत्तेषां साधुस्मरणे एवोपयोगः, नतु साक्षात्तद्वाचकत्वम्, अतो न सादुत्वमिति भावः~।
उक्तं हि वाक्यपदीये-
\begin{center} ते साधुष्वनुमानेन प्रत्ययोत्पत्तिहेतवः~।\\
 तादात्म्यमुपगम्येव शब्दार्थस्य प्रकाशकाः॥\\[10pt]
 न शिष्टैरनुगम्यन्ते पर्य्याया इव साधवः~।\\
 ते (न) यतः स्मृतिशास्त्रेण तस्मात् साक्षादवाचकाः॥\\[10pt]
 बम्बम्बेति यदा बालः शिक्षयणः प्रभाषते~।\\
 अव्यक्तं तद्विदां तेन व्यक्ते भवति निर्णयः॥\\[10pt]
 एवं साधौ प्रयोक्तव्ये योऽपभ्रंशः प्रयुज्यते~।\\
 तेन साधुव्यवहितः कश्चिदर्थोऽभिधीयते॥इति॥\end{center}

 नन्वपभ्रंशानां साक्षादवाचकत्वे किं मानम्,? शक्तिकल्पकव्यवहारादेस्तुल्यत्वात्, इति चेत् सत्यम्, तत्तद्देशभेदेन भिन्नेषु तेषु तेषु शक्तिकल्पने गौरवात्~।
नच पर्य्ययतुल्यता शङ्कया, तेषां सर्वदेशेष्वेकत्वात्, विनिगमनाविरहेण सर्वत्र शक्तिकल्पना, नही अपभ्रंशे तथा, अन्यथा माषाणां पर्यायतया गणनापत्तेश्च~।
एवञ्च `शक्तत्वमेवास्तु साधुत्बम्' इति नैयायिकमीमांसकादीनां मतं तन्मतेनैव द्रष्टव्यम्~।
 इदानीं स्वमतमाह- वाचकत्वाविशेषे वेति~।
अयं भावः- अपभ्रंशानामशक्तत्वे ततो बोध एव न स्यात्~।
नच साधुस्मरणात् ततो बोधः, तानविदुषां पामराणामपि बोधात्, तेषां साधोरबोधाच्च नच शक्तिभ्रमात् तेभ्यो बोधः, बोधकत्वस्याबाधेन तद्ग्रहस्याभ्रमत्वात्~।
`ईश्वरेच्छा शक्तिः' इति मतेऽपि सन्मात्रविषयिण्यास्तस्या बाधाभावात्, शक्तेः पदपदार्थविशेषघटिताया भ्रमासम्भवाच्चेति~।

उक्तञ्च वाक्यपदीये-
\begin{center} पारम्पर्यादपभ्रंशा विगुणेष्वभिधातृषु~।\\
 प्रसिद्धिमागता येषु तेषां साधुरवाचकः॥\\[10pt]
 दैवी वाग् व्यवकीर्णेयमशक्तैरभिधातृभिः~।\\
 अनित्यदर्शिनां त्वस्मिन् वादे बुद्धिविपर्य्ययः॥इति॥\end{center}

 अवाचकः=अबोधकः, बुद्धिविपर्य्ययः=एते एव वाचका नान्ये इति विपर्य्यय इत्यर्थः~।
किञ्च विनिगमनाविरहाद् भाषायामपि शक्तिः~।
नच तासां नानात्वं दोषः, संस्कृतवन्महाराष्ट्रादिभाषायाः सर्वत्रैकत्वेन प्रत्येकं विनिगमनाविरहतदवस्थ्यात्~।
 किञ्चानुपूर्वो पदेऽवच्छेदिका~।
सा च पर्य्यायेष्विव भाषायामप्यन्यान्यैवेति कस्तयोर्विशेष इति विभाव्यं सूरिभिः~।
तथाच संस्कृतवद् भाषायाः सर्वत्रैकत्वेन प्रत्येकं शब्दाः शक्ता एव~।
नच पर्य्यायतया भाषाणां गणनाऽऽपत्तिः, साधूनामेव कोशादौ विभागाभिधानात्~।
 नन्वेवं साधुता तेषां स्यादित्यत आह-नियम इति-पुण्यजननबोधनाय साधूनाम्, "साधुभिर्भाषितव्यम्" इति विधिः~।
पापजननबोधनाय "नासाधुभिः" इति निषेधः~।
तथाच पुण्यजननयोग्यत्वं साधुत्वम्~।
पापजननयोग्यत्वमसाधुत्वम् तत्र जनकताऽवच्छेदिका च जातिः~।
तज्ज्ञापकञ्च कोसादि, व्याकरणादि च~।
 एवमेव च राजसूयादेर्ब्राह्मणे फलाजनकत्ववद् गवादिशब्दानां नाश्वादै साधुत्वमिति सङ्गच्छते~।
आधुनिकदेवदत्तादिनाम्नामपि "द्वयक्षरम्" इत्यादिभाष्येण व्युत्पादितत्वात् साधुत्वम्~।
एवञ्चयः शब्दो यत्रार्थे व्याकरणे व्युत्पादितः स तत्र साधुरिति पर्यवसितम्~।
 गौणानां गुणे व्युत्पादनात् तत्पुरस्कारेण प्रवृत्तौ साधुत्वमेव~।
आधुनिकलाक्षणिकानां त्वसाधुत्वमिष्टमेव~।
अत एव `ब्राह्मणाय देहि' इत्यर्थे `ब्राह्मणं देहि' इत्यादिकं लक्षणयाऽपि न साधुरित्यादि विस्तरेण प्रपञ्चितं भूषणे॥३८॥
अतिरिक्तशक्तिग्रहोपायमाह-
\begin{center}{\bfseries सम्बन्धिशब्दे सम्बन्धो योग्यतां प्रति योग्यता॥\\
 समयाद् योग्यतासंविन्मातापुत्रादियोगवत्॥३९॥}\end{center}

 सम्बन्धो विषयः~।
योग्यतां प्रति योग्यता-सम्बन्धिशब्दं प्रति योग्यता विषय इति~।
समयात्-व्यवहारात्, योग्यतासंवित्-शक्तिग्रहः~।
घटपदमत्र योग्यमेतत्सम्बन्धीति व्यवहारात्, सा ग्राह्योत्यर्थः॥३९॥
\begin{center}इति वैयाकरणभूषणसारे शक्तिनिर्णयः॥६॥\end{center}

\section*{\begin{center}अथ नञर्थनिर्णयः\end{center}}\addcontentsline{toc}{section}{नञर्थनिर्णयः}
\fancyhead[LE,RO]{नञर्थनिर्णयः}
नञर्थमाह-
\begin{center}{\bfseries नञ्समासे चापरस्य प्राधान्यात् सर्वनामता॥\\
 आरोपितत्वं नञ्द्योत्यं न ह्यसोऽप्यतिसर्ववत्॥४०॥}\end{center}

 नञ्समासे अपरस्य=उत्तरपदार्थस्य प्राधान्यात् सर्वनामता सिद्धयतीति शेषः~।
अत एव आरोपितत्वमेव, नञ्द्योत्यमिति अभ्युपेयमिति शेषः~।
 अयं भावः-`असर्वः' इत्यादौ `आरोपितः सर्वः' इत्यर्थे सर्वशब्दस्य प्राधान्याबाधात् सर्वनामता सिद्धयति~।
अन्यथा `अतिसर्वः' इत्यत्रेव सा न स्यात्, `घटो नास्ति' इत्यादाव भावविषयकबोधे तस्य विशेष्यताया एव दर्शनात्~।
अस्मद्रीत्या च स आर्थो बोधो मानसः~।
 तथाच `असर्वस्मै इत्याद्यसिद्धिप्रसङ्गो नेति~।
अत्र चारोपितत्वम्-आरोपविषयत्वम्~।
आरोपमात्रमर्थो विषयत्वं संसर्ग इति निष्कर्षः~।
द्योत्यत्वोक्तिर्निपातानां द्योतकत्वमभिप्रेत्य॥४०॥
 `घटो नास्ति' `अब्राह्मणः इत्यादावारोपबोधस्य सर्वानुभवविरुद्धत्वात् पक्षान्तरमाह-
\begin{center}{\bfseries अभावो वा तदर्थोऽस्तु भाष्यस्य हि तदाशयात्~।\\
 विशेषणं विशेष्यो वा न्यायतस्त्ववधार्य्यताम्॥४१॥}\end{center}

 तदर्थः=नञर्थः अर्थपदं द्योत्यत्व-वाच्यत्वपक्षयोः साधारण्येन कीर्त्तनाय~।
भाष्यस्येति-तथाच नञ्सूत्रे महाभाष्यम्-"निवृत्तपदार्थकः" इति, निवृत्तं पदार्थो यस्य "नपुंसके भावे क्तः"\footnote{३-१-११४} इति क्तः~।
अभावार्थक इत्यर्थः~।
 यत्तु-निवृत्तः पदार्थो यस्मिन्नित्यर्थः~।
सादृश्यादिनाऽध्यारोपितब्राह्मण्याः क्षत्रियादयोऽर्था यस्येत्यर्थ इति कैयटः, तन्न, अध्यारोपितब्राह्मण्यस्य क्षत्रियादेर्नञवाच्यत्वात्~।

अन्यथा सादृश्यादेरपि वाच्यतापत्तेः~।
 यत्तु-
 \begin{center}
 तत्सादृश्यमभावश्च तदन्यत्वं तदल्पता~।\\
 अप्राशस्त्यं विरोधश्च नञर्थाः षट् प्रकीर्त्तिताः॥
\end{center}
 इति पठीत्वा अब्राह्मणः, अपापम्, अनश्वः, अनुदरा कन्या, अपशवो वा अन्ये गोऽश्वेभ्यः, अधर्म इत्युदाहरन्ति~।
तत्त्वार्थिकार्थमभिप्रेत्येति स्पष्टमन्यत्र~।
 विशेषणमिति-प्रतियोगिनीति शेषः~।
तथाचासर्वापदे सर्वनामता~।
"अनेकमन्यपदार्थे" "सेव्यतेऽनेकया सन्नतापाङ्गया" इत्यादावेकशब्दार्थप्राधान्यादेकवचननियमः~।
`अब्राह्मणः इत्यादावुत्तरपदार्थप्राधान्यात् तत्पुरुषत्वम्~।
`अत्वं भवसि' `अनहं भवामि' इत्यादौ पुरुषवचनादिव्यवस्था चोपपद्यते~।
 अन्यथा त्वदभावो मदभाव इतिवदभावांशे युष्मदस्मदोरन्वयेन युष्मत्सामानाधिकरण्यस्य तिङ्क्ष्वसत्त्वात् पुरुषव्यवस्था न स्यात्~।
 अस्मन्मते च भेदप्रतियोगित्वभिन्नाश्रयिका भवनक्रियेत्यन्वयात्, सामानाधिकरण्यं नानुपपन्नमिति भावः~।
 विशेष्यो वेति-प्रतियोगिनीति शेषः~।
अयं भावः-गौणत्वेऽपि नञ्समासे "एतत्तदोः सुलोपोऽकोरनञ्समासे हलि"\footnote{३-१-१३२} इति ज्ञापकात् सर्वनामसंज्ञानानुपपन्न~।
`असः शिवः' इत्यत्र सुलोपवारणाय `अनञ्सम्से' इति हि विशेषणम्~।
न च तत्र तच्छब्दस्य सर्वनामताऽस्ति, गौणत्वात्~।
`अकोः' इत्यकच्सहितव्यावृत्त्या सर्वनाम्नोरेव तत्र ग्रहणलाभात्~।
तथाच `अनञ्समासे' इति ज्ञापकं सुवचम्~।
 "अनेकमन्यपदार्थे" इत्यादावेकवचनम्, विशेष्यानुरोधात्~।
"सुबामन्त्रिते पराङ्गवत् स्वरे"\footnote{२-१-२} इत्यतोऽनुवर्तमानं सुब्ग्रहणं विशेष्यमेकवचनान्तमेव~।
किञ्चानेकशब्दाद् द्विवचनोपादाने बहूनां, बहुवचनोपादाने द्वयोर्बहुव्रीहिर्न सिद्धयोदित्युभयसंग्रहायैकवचनम्, जात्यभिप्रायम्, औत्सर्गिकं वा~।
 "सेव्यतेऽनेकया" इत्यत्रापि `योषया' इति विशेष्यानुरोधात्, प्रत्येकं सेवनान्वयबोदनाय चैकवचनम्, नतूत्तरपदार्थप्राधान्यप्रयुक्तम्~।
अत एव "पतन्त्यनेके जलधेरिवोर्मयः" इत्यादिकमपि सूपपादम्~।
 `अत्बं भबसि' इत्यादौ युष्मदस्मदोस्तद्भिन्ने लक्षणा~।
नञ् द्योतकः~।
तथाच भिन्नेन युष्मदर्थेन तिङः सामानाधिकरण्यात् पुरुषव्यवस्था~।
`त्वद्भिन्नाश्रयिका भवनक्रिया इति शाब्दबोधः~।
एवम् ` न त्वं पचसि' इत्यत्र त्वदभिन्नाश्रयकपाकानुकूलभावनाऽभावः~।
`घटो नास्ति' इत्यत्र घटाभिन्नाश्रयकास्तित्वाभाव इतिरीत्या बोधः~।
असमस्तनञः क्रियायामेवान्वयात्~।
 स चाभावोऽत्यन्ताभावत्वान्योन्याभावत्वादिरूपेण शक्यः, तत्तद्रूपेण बोधादित्याद्यन्यत्र विस्तरः॥४१॥
\begin{center} इति वैयाकरणभूषणसारे नञर्थनिर्णयः॥७॥\end{center}
\section*{\begin{center}अथ निपातार्थनिर्णयः\end{center}}\addcontentsline{toc}{section}{निपातार्थनिर्णयः}
\fancyhead[LE,RO]{निपातार्थनिर्णयः}
 प्रादयो द्योतकाः, चादयो वाचकाः, इति नैयायिकमतमयुक्तम्, बैषम्ये बीजाभावादिति ध्वनयन्निपातानां द्योतकत्बं समर्थयते-
 \begin{center}{\bfseries द्योतकाः प्रादयो येन निपाताश्चादयस्तथा~।\\
 `उपास्येते हरिहरौ' लकारो दृश्यते यथा॥४२॥}\end{center}

 येन हेतुना प्रादयो द्योतकास्तेनैव हेतुना चादयः=निपातास्तथा=द्योतका इत्यर्थः~।
अयं भावः-`ईश्वरमनुभवति' इत्यादावनुभवादिः प्रतीयमानो न धात्वर्थः, `भवति' इत्यत्राप्यापत्तेः~।
नोपसर्गार्थः, तथा सत्यप्रकृत्यर्थतया तत्राख्यातार्थानन्वयापत्तेः, प्रत्ययानां प्रकृत्यर्थान्विस्वार्थबोधकत्वव्युत्पत्तेः, `अनुगच्छति' इत्यादौ अनुभवादिप्रत्ययापत्तेश्च~।
न विशिष्टार्थः, गौरवात्~।
तथाच धातोरेव विद्यमानत्वादिवाचकस्यास्तु तक्षणा, उपसर्गस्तात्पर्य्यग्राहक इत्यस्तु~।
तथाच तात्पर्य्यग्राहकत्वमेव द्योतकत्वमिति~।
 तच्च चादिष्वपि तुल्यम्~।
`चैत्रमिव पश्यति' इत्यादौ सादृश्यविशिष्टं चैत्रपदलक्षयम्, इवशब्दस्तात्पर्य्यग्राहक इत्यस्य सुवचत्वादिति~।
तत्र स्वयं युक्तयन्तरमाह-`उपास्येते हरिहरौ' इति-अत्र ह्युपासना किमुपसर्गार्थः, विशिष्टस्य, धातुमात्रस्य वा~।
नाद्यः,
तथा सति स्वार्थफलव्यधिकरणव्यापारवाचकत्वरूपसकर्मत्वस्यास्धातोरुपासनारूपफलवाचकत्वाभावा\-दनापत्तेस्ततः कर्मणि लकारो न स्यात्~।
न द्वितीयः, गौरवात्~।
तृतीये त्वागतं द्योतकत्वम्, तात्पर्य्यग्राहकत्वलाभादिति भावः~।
`दृश्यते' इत्यत्र `कर्मणि' इति शेषः॥४॥

तच्चादिष्वपि तुल्यमित्याह-
 \begin{center}{\bfseries तथान्यत्र निपातेऽपि लकारः कर्मवाचकः~।\\
 विशेषणद्ययोगोऽपि प्रादिवच्चादिके समः॥४३॥}\end{center}

 अन्यत्र=`साक्षात्क्रियते,' `अलङ्क्रियते,' ऊरीक्रियते शिवः,' इत्यादौ~।
अत्रापि धातोस्तत्तदर्थे कर्मणि लकारसिद्धयर्थं तत्तदर्थवाचकत्वं वाच्यमित्युपसर्गवत्,द्योतकत्वममीषामपीत्यर्थः~।
यद्यपि कृधातोः सकर्मकत्वमस्त्येव, तथाप्येष्वर्थेषु सकर्मकता न स्यात्~।
अन्यथा `वायुर्विकुरुते' सैन्धवा विकुर्वते, इत्यत्रापि स्यादिति भावः~।
 अथोपासना-साक्षात्कारादिर्निपातार्थोऽस्तु, "साक्षात्प्रत्यक्षतुल्ययोः" इति कोशस्वरसात्~।
तदनुकूलो व्यापार एव धात्वर्थोऽस्तु~।
स्व-स्वयुक्तनिपातान्यतरार्थफलव्यधिरणव्यापारवत्त्वं सकर्मकत्वमपि सुवचमिति दृष्टान्तदार्ष्टान्तिकावयुक्ताविति नेदं साधकमिति चेन्न, नामार्थधात्वर्थयोर्भेदेन साक्षादन्वयासम्भवेन निपातधात्वर्थयोरन्वयासम्भवात्~।
अन्यथा `तण्डुलः पचति' इत्यत्रापि कर्मतया तण्डुलानां धात्वर्थेऽन्वयापत्तेरिति~।
किञ्च प्रादीनां वाचकत्वे `भूयान् प्रकर्षः' `कीदृशो निञ्चयः' इतिवत् `भूयान् प्र' `कीदृशो निः' इत्यपि स्यात्~।
अस्मन्मते प्रादेरनर्थकत्वान्न तदन्वय इत्यतो द्योतकता तेषां स्यादिति~।
साधकान्तरमभिप्रेत्याह-विशिषणेति~।
`शोभनः समुच्चयो द्रष्टव्यः' इति `शोभनश्च द्रष्टव्यः' इत्यस्यापत्तेस्तुल्यसमाधेयत्वादिति भावः~।
 अपिच निपातानां वाचकत्वे प्रातिपदिकार्थयोर्विना षष्ठयादिकं भेदेनान्वयासम्भवः~।
अन्यथा `राजा पुरुषः' इत्यस्य `राजसम्बन्धी पुरुषः' इत्यर्थापत्तेरित्यभिप्रेत्याह-आदीति-`धवखदिरयोः समुच्चयः' इतिवत्`धवस्य च खदिरस्य च' इत्येव स्यादिति भावः॥४३॥
 ननु प्रातिपदिकार्थयोर्भेदान्वयबोधे विरुद्धविभक्तिजन्योपस्थितिर्हेतुरिति कार्य्यकारणभावो निपातातिरिक्तविषय एवेति नोक्तदोष इत्याशङ्कयाह-
 \begin{center}{\bfseries पदार्थः सदृशान्वेति विभागेन कदापि न~।\\
 निपातेतरसङ्कोचे प्रमाणं किं विभावय॥४४॥}\end{center}

 सदृशा=सदृशेन, समानाधिकरणेनेति यावत्~।
अन्वेति, अभेदेनेति शेषः~।
विभागेन=असदृशेन, असमानाधिकरणेनेति यावत्~।
 अयमर्थः-समानाधिकरणप्रातिपदिकार्थयोरभेदान्वयव्युत्पत्तिर्निपातातिरिक्तविषयेति कल्पने मानाभावः, गौरवञ्च~।
अस्माकं निपातानां द्योतकत्वादन्वय एव नास्तीति नायं दोषः~।
अत एव `घटो नास्ति' इत्यादौ घटपदं तत्प्रतियोगिके लाक्षणिकमिति नैयायिकाः॥४४॥
 अपिच निपातानां वाचकत्वे काव्यादावन्वयो न स्यादिति साधकान्तरमाह-

 \begin{center}{\bfseries शरैरुस्त्रैरिवोदीच्यानुद्धरिष्यन् रसानिव~।\\
 इत्यादावन्वयो न स्यात् सुपाञ्च श्रवणं ततः॥४५॥}\end{center}

 अत्रोस्त्रसदृशैः शरै रससदृशानुदीच्यानुद्धरिष्यन्नित्यर्थः~।
अयञ्चोस्त्रादिशब्दानां तत्सदृशपरत्वे इवशब्दस्य द्योतकत्वे सत्येव सङ्गच्छते~।
अन्यथा प्रत्ययानां प्रकृत्यर्थान्वितस्वार्थबोधकत्वव्युत्पत्तिविरोधः~।
 तथाह-उस्त्रैरिति करणे तृतीया~।
नचोस्त्रोऽत्र करणम्~।
इवार्थसदृशस्य करणत्वेऽपि तस्य करणत्वं नानेन बोधयितं शक्यम्~।
अप्रकृत्यर्थत्वात्~।
इवशब्दस्य चासत्त्वार्थकतया तदुत्तरतृतीयाया असम्भवात्~।
सम्भवे वा श्रवणप्रसङ्गात्~।
उस्त्रपदोत्तरतृतीयान्वयप्रसङ्गाच्चेत्याह-सुपाञ्चेति~।
सुपां श्रवणञ्चेत्यर्थः~।
चकारादुस्त्रपदोत्तरतृतीयाऽनन्वयः समुच्चीयते~।
 इत्यादावित्यादिपदात्~।
"वागर्थाविव सम्पृत्तौ" "पार्वतीपरमेश्वरौ वन्दे" इत्यत्र वागर्थयोर्वदिकर्मात्वाभावात्तदुत्तरद्वितीयाया अनन्वयः~।
इवार्थस्य कर्मत्वान्वयबोधासम्भवश्च संगृह्यते~।
 यदि च विशेषणविभक्तिरभेदार्था, साधुत्वमात्रार्था वा तदाऽपि इवशब्दस्य वाचकत्वेऽनन्य एव~।
उस्त्रसदृशशराणां समानाधिकरणपदोपस्थाप्यतया भेदेनान्वयायोगात्~।
बाधादभेदेनापि न सः~।
नह्युस्त्राभिन्नसदृशाभिन्नः शर इत्यर्थो द्रष्टव्यः॥४५॥
 ननु त्वन्मते `अब्राह्मणः' इत्यादौ तत्पुरुषलक्षणाव्याप्त्यापत्तिः, पूर्वपदस्यानर्थकत्वेनोत्तरपदार्थप्राधान्याभावात्~।
उपसर्गस्यार्थवत्त्वाभावेन प्रातिपदीकत्वाभावाद्विभक्तिश्च ततो न स्यादित्यत आह-
 \begin{center}{\bfseries नञ्समासे चापरस्य द्योत्यं प्रत्येव मुख्यता~।\\
 द्योत्यमेवार्थमादाय जायन्ते नामतः सुपः॥४६॥}\end{center}

 नञ्समासादौ योत्तरंपदार्थप्रधानता सा द्योत्यमर्थमादायैव~।
तमेवार्थमादायार्थवत्त्वात् प्रातिपदिकत्वमित्यर्थः~।
 वस्तुतः "अव्ययादाप्सुपः"\footnote{२-४-८२} इति ज्ञापकात् सुबुत्पत्तिः~।
"निपातस्यानर्थकस्य" इति बार्त्तिकाद्वा प्रातिपदिकत्वम्~।
"कृत्तद्धितसमासाश्च"\footnote{१-२-४६} इत्यनुक्तसमुच्चयार्थकचकारेण निपातानां संग्रह इति वा बोध्यम्~।
तस्माद् युक्तं निपातानां द्योतकत्वम्~।
उक्तञ्चाकृत्यधिकरणवार्त्तिके-
\begin{center} चतुर्विधे पदे चात्र द्विविधस्यार्थनीर्णयः~।\\
 क्रियते संशयोत्पत्तेर्नोपसर्गानिपातयोः॥\\[10pt]
 तयोरर्थाभिधाने हि व्यापारो नैव विद्यते~।\\
 यदर्थद्योतकौ तौ तु वाचकः स विचार्य्यते॥इति॥\\[10pt]
 उपसर्गेण धात्वर्थो बलादन्यः प्रतीयते~।\\
 प्रहाराहारसंहारविहारपरिहारवत्॥इति॥\end{center}

अत्रोपसर्गपदं निपातोपलक्षणम्~।
धातुपदं पदान्तररयेति बोध्यम्॥४६॥
 नन्वन्वयव्यतिरेकाभ्यां निपातानां तत्तदर्थवाचकत्वमेव युक्तम्, बोधकतारूपशक्तेरबाधात्~।
किञ्चोक्तरीत्या `पचति' इत्यादौ धातोरेव कर्तृविशिष्टभावनायां लक्षणाऽस्तु तात्पर्य्यग्राहकत्वमात्रं तिङादेः स्यादित्यरुचेः पक्षान्तरमाह-
\begin{center}{\bfseries निपातानां वाचकत्वमन्वयव्यतिरेकयोः~।\\
 युक्तं वा नतु तद्युक्तं परेषां मतमेव नः॥४७॥}\end{center}

 एवञ्च `धात्वर्थप्रातिपदिकार्थयोर्भेदेनान्वयबोधो न व्युक्पन्नः' इति निपातातिरिक्तविषयः~।
समानाधिकरणप्रातिपदिकार्थयोरभेदान्वय इत्यपि तथेत्यगत्या कल्पनीयमिति भावः~।
नत्वितिनैयायिकोक्तं प्रादि-चाद्योर्वैषम्यमित्यर्थः~।
 यत्तु-सर्वेषां निपातानां वाचकत्वेऽर्थवत्सूत्रेणैव तेषां प्रातिपदिकत्वसम्भवात् "निपातस्यानर्थकस्य" इति विधिवैयर्थ्यम्~।
 सर्वेषां द्योतकत्वे "चानर्थकस्य" इति व्यर्थम्~।
तथाच केचिद् द्योतकाः केचिद्वाचका इत्यभ्युपेयमिति, तन्न, एवं हि `चादयो द्योतका प्रादयो वाचकाः' इति वैपरीत्यावारणात्~।
सर्वथानर्थकानां पादपूरणमात्रार्थमुपात्तानां संग्रहाय वार्त्तिकारम्भस्य कैयटादौ स्पष्टत्वात्, तस्य प्रत्याख्यातत्वाच्च~।

 `परेषाम्' इति बहुवचनं मीमांसकसंग्रहाय~।
केवलवृक्षशब्दात् समुच्चयाबोधाच्चकारश्रवणे तद्बोधाच्चकार एव तद्वाचकः, न दयोतकः~।
किञ्च द्योतकत्वे पदान्तराणां तत्र शक्तिः कल्प्या, चकारादेद्र्योतकत्वशक्तिश्च कल्प्येति गौरवं स्यादिति हि समुच्चयाधिकरणे स्थितम्, तदपि न युक्तमिति भावः~।
 तथाहि-अन्वयव्यतिरेकौ तात्पर्य्यग्राहकत्वेनाप्युपयुक्तौ~।
घटादिपदानामेव समुच्चिते लक्षणा, तात्पर्य्यग्राहकः प्रकारणादिवच्चादिरिति स्वीकारान्न शक्तिद्वयकल्पनाऽपि~।
अस्माकं लक्षणाग्रहदशायां बोधात्तत्तत्कार्य्यकारणभाव आवश्यकः~।
एवं शक्तिग्रहस्यापीति पक्षद्वयेऽपि कल्प्यान्तराभावेन गौरवाभावादुभयमपि युक्तमित्यभिमतम्~।
 अत एव~।
"स वाचको विशेषाणां सम्भवाद् द्योतकोऽपि वा"~।
 इति वाक्यपदीयं सङ्गच्छते~।
दर्शनान्तररीत्या वाचकत्वमेव द्योतकत्वमेवेति नियमस्तु न युक्त इति ध्वनयन्नाह-`मतमेव नः' इति॥४७॥

पर्य्यवसितमुपसंहरन्नाह-
\begin{center}{\bfseries निपातत्वं परेषां यत्तदस्माकमिति स्थितिः~।\\
 व्यापकत्वाच्छक्ततायास्त्ववच्छेदकमिष्यते॥४८॥}\end{center}

 परेषां यन्निपातत्वम्-असत्त्वार्थकत्वे सति चाऽदिगणपठितत्वम्, शक्तितसम्बन्धेन निपातपदवत्त्वञ्च~।
जातिर्वा, उपाधिर्वा तदेवास्माकमषि~।
परन्तु सामान्य धर्म्मेप्रमाणानां पक्षपाताच्छक्तता, द्योतकता वा तदवच्छेदेनैव कल्प्येति नैयायिकोक्त प्रादि-चाद्योर्वैषम्यमयुक्तमित्यर्थः~।
व्यापकत्वात्=सामान्यत्वात्~।
शक्तताया इत्युपलक्षणम्-द्योतकताया वेत्यपि द्रष्टव्यम्॥४८॥
 \begin{center}इति वैयाकरणभूषणसारे निपातानां द्योतकत्वादिनिर्णयः॥८॥\end{center}
\section*{\begin{center}अथ त्वादिभावप्रत्ययार्थनिर्णयः\end{center}}\addcontentsline{toc}{section}{त्वादिभावप्रत्ययार्थनिर्णयः}
\fancyhead[LE,RO]{त्वादिभावप्रत्ययार्थनिर्णयः}
भावप्रत्ययार्थमाह-
\begin{center}{\bfseries कृत्तद्धितसमासेभ्यो मतभेदनिबन्धन्म्~।\\
 त्वतलोरर्थकथनं टीकायां हरिणा कृतम्॥४९॥}\end{center}

 "कृत्तद्धितसमासेभ्यः सम्बन्धाभिधानं भावप्रत्ययेनाऽन्यत्र रूढयभिन्नरूपाव्यभिचरितसम्बन्धेभ्यः" इति वार्त्तिकवचनमिति मीमांसकादीनां भ्रममपाकुर्वन्नाह-टीकायामिति~।
भर्तृहरिणा महाभाष्यटीकायामित्यर्थः~।
त्वतलोरिति-भावप्रत्ययमात्रोपलक्षणम्~।
 अयमर्थः-समासादौ शक्तिः कल्प्यमाना राजादिसम्बन्धविशिष्टे कल्प्यते इत्युक्तम्~।
तथाच तदुत्तरभावप्रत्ययः सम्बन्धं वदतीत्यर्थः~।
एतदपि `भेदः' संसर्गः, उभयं वेत्युक्तेषु भेदपक्षे न सम्भवतीत्यत आह-मतभेदेति-पक्षभेदेत्यर्थः~।
 एवञ्च राजपुरुषत्वम्, औपगवत्वम्, पक्तृत्वमित्यादौ स्वस्वामिभावसम्बन्धः, उपग्वपत्यसम्बन्धः, {अपत्यापत्यवत्सम्बन्धः} क्रियाकारकभावसम्बन्धः, इत्यन्वयबोधः~।
औपगवादावव्यभिचरितसम्बन्धे तु अर्थन्तरवृत्तिस्तद्धित उदाहार्य्यः~।
 `दामोदरत्वम्' कृष्णसर्पत्वम्' इत्यादौ जातिविशेषबोधादाह-अन्यत्रेति-रूढेरभिन्नरूपादव्यभिचरितसम्बन्धेभ्यश्चान्यत्रेत्यर्थः~।
रूढिरुक्ता~।
द्वितीयं यथा शुक्लत्वम्~।
अत्र "तदस्यास्त्यस्मिन्"\footnote{५-२-१४} इति मतुपः "गुणवचनेभ्यो मतुपो लुगिष्टः" {पा.वाo} इति लुप्तत्वात् तद्धितान्तत्वेऽपि `घटः सुक्लः' इत्यभेदप्रत्ययाद् गुणस्यैव प्रकारत्वेन भानं जायते~।
तृतीयम्-सतो भावः सत्तेति~।
अत्र जातावेव प्रत्यय इति दिक्॥४९॥
 दण्डीत्यादौ प्रकृत्यर्थविशिष्टद्रव्यमात्रवाचकता तद्धितस्येति वदन्तं मीमांसकम्मन्यं प्रत्याह-
\begin{center}{\bfseries अत्रार्द्धजरतीयं स्याद् दर्शनान्तरगामिनाम्~।\\
 सिद्धान्ते तु स्थितं पक्षद्वयं त्वादिषु तच्छृणु॥५०॥ }
 \end{center}

 अत्र=भावप्रत्ययविषये~।
तथाहि-दामोदरत्वम्, घटत्वमित्यादौ भावप्रत्ययस्य सम्बन्धानभिधायकत्वेन मीमांसकानां दण्डित्वमित्यादिष्वपि तदभिधानं न स्यात्~।
प्रकृतिजन्यबोधे प्रकारः प्रकृत्यर्थसमवेतो हि तदुत्तरभावप्रत्ययेनाभिधियते~।
अन्यथा घटत्वमित्यत्र द्रव्यत्वादेः, दण्डित्वमित्यादौ दण्डादेश्च तदुत्तरभावप्रत्ययवाच्यताऽऽपत्तेः~।
नच तन्मते दण्डित्यादिबोधे सम्बन्धः प्रकारः~।
 यत्तु-
 \begin{center}यदा स्वसमवेतोऽत्र वाच्यो नास्ति गुणोऽपरः~।\\
 तदा गत्यन्तराभावात् सम्बन्धो वाच्य आश्रितः॥इति॥\end{center}
 तन्न, इनादेः सम्बन्धिवाचकत्वेनोपपत्तौ गत्यभावाभावात्~।
प्रपञ्चितञ्चैतदादावेव वैयाकरणभूषणे~।
 ननु तवापीदं वैषम्यं कथमित्यत आह- सिद्धान्ते त्विति~।
`जायन्ते' इति वक्षयमाणविशेषणेऽन्वितम्~।
सिद्धान्ते प्रकृतीजन्यबोधे प्रकारे त्वादयो जायन्त इत्यर्थः~।
 प्रकृतिजन्यबोधे प्रकार इत्यत्र पक्षद्वयं स्थितमिति योजना॥५०॥
तौ पक्षावाह-
\begin{center}{\bfseries प्रयोगोपाधिमाश्रित्य प्रकृत्यर्थप्रकारताम्~।\\
 धर्म्ममात्रं वाच्यमिति यद्वा शब्दपरा अमी॥५१॥\\[10pt]
 जायन्ते तज्जन्यबोधप्रकारे भावसंज्ञिते~।}\end{center}
 प्रयोगे उपाधिर्निमित्तं प्रकृत्यर्थप्रकारतम्=प्रकारतया भासमानं धर्म्मं वाच्यतया आश्रित्य त्वादयो जायन्ते~।
प्रकृतिजन्यबोधे प्रकारस्त्वाद्यर्थ इति यावात्~।
 ननु `घटत्वम्' इत्यत्र प्रकारत्वात् तदुत्तरभावप्रत्ययेन घटत्वत्वस्यापि वाच्यता स्यादित्यत्रेष्टापत्तिमाह-धर्ममात्रमिति, नत्वत्र लघुगुरुविचार इत्यभिप्रायः~।
तत्तद्व्यक्तिविशिष्टब्रह्मसत्ताया एव घटत्व-घटत्वत्वादिरूपत्वात्~।
\begin{center} सम्बन्धिभेदात् सत्तैव भिद्यमाना गवादिषु~।\\
 जातिरित्युच्यते तस्यां सर्वे शब्दा व्यावस्थिताः॥\\[10pt]
 तां प्रातिपदिकार्थञ्च धात्वर्थञ्च प्रचक्षते~।\\
 सा नित्या सा महानात्मा तामाहुस्त्वतलादयः॥\end{center}
 इति वाक्यपदीयात्~।
उक्तञ्च "तस्य भावस्त्वतलौ"\footnote{५-१-११९} इति सूत्रे वार्त्तिककारैः-"यस्य गुणस्य भावाद् द्रव्ये शब्दनिवेशस्तदभिधाने त्वतलौ" इति~।
 यस्य गुणस्य=विशेषणतया भासमानस्य, भावात्=आश्रयत्वात्, द्रव्ये=विशेष्ये, शब्दनिवेशः=शब्दप्रवृत्तिः, तस्मिन् वाच्ये त्वतलावित्यर्थः~।
तथाच रूपादिशब्देभ्यो जातौ, शुक्लाणुदीर्घमहदादिभ्यो गुणे, पाचकादिशब्देभ्यः क्रियायाम्, घटादिशब्देभ्योजातौ प्रत्ययः, रूपादिशब्दानां जातिप्रकारकबोधजनकत्वात्~।
पाचकादिशब्दानां क्रियाप्रकारकबोधजनकत्वे तस्यां प्रत्ययः~।
 `संसर्गप्रकारकबोधजनकत्वम्' इति मते च संसर्ग इति व्यवस्था सूपपादेति भावः~।
तत्र जातिवाचकानां व्यक्तय एव शक्यताऽवच्छेदिकाः~।
तथाच `घटत्वम्' इत्यत्र `घटवृत्तिरसाधारणो धर्म्मः' इति बोध इत्यादि द्रष्टव्यम्~।
 पक्षान्तरमाह-यद्वेति-"यद्वा सर्वे भावाः स्वेनार्थेन भवन्ति, स तेषां भावः" इति वार्त्तिकोक्तेः~।
यद्वाशब्दस्तत्सूचनप्रयोजनकोऽपि~।
भवन्ति=वाचकत्वेन प्रवर्त्तन्ते इति भावाः=शब्दाः~।
स्वेन=रूपेण=अर्थेन भवन्ति=प्रवर्त्तन्ते~।
अतः स तेषां भावः=प्रवृत्तिनिमित्तमित्यर्थः~।
अयं भावः-अर्थवच्छब्दोऽपि द्रव्ये प्रकारः~।
हरिहरनलेक्षवाकुयुधिष्ठिरवसिष्ठादिशब्देभ्यस्तत्तद्वाच्यः कश्चिदासीदिति शब्दप्रकारकबोधस्य सर्वसिद्धत्वात्, अन्यथा वनौषधिवर्गादेर्नागरिकान् प्रत्यबोधकत्वापत्तेश्च~।
एवमेवाप्रसिद्धार्थकपदेष्वनुभवः सर्वसिद्धः, नतु घटादिपदेष्विव तत्तज्जात्यादिभेदेन {रूपेण}~।
तथाचोभयमवच्छेदकम्~।
यस्य तथाशक्तिग्रहस्तस्य जात्यादिरूपेणैवोपस्थितिः~।
 पदाप्रकारकः शक्तिग्रहस्तु विशिष्य नापेक्षितः~।
किन्तु `इदं पदं क्वचिच्छक्तम्, साधुपदत्वात् इत्यादिरूप एवापेक्षयते इति विशिष्यागृहीतशक्तिकेभ्यस्तथैव बोधः~।
तथाच `शब्दोऽपि त्वप्रत्ययार्थः' इति प्रपञ्चितं भूषणे॥५१॥

\begin{center}इति वैयाकरणभूषणसारे त्वादिभावप्रत्ययार्थनिर्णयः॥९॥\end{center}

\section*{\begin{center}अथ देवताप्रत्ययार्थनिर्णयः\end{center}}\addcontentsline{toc}{section}{देवताप्रत्ययार्थनिर्णयः}
\fancyhead[LE,RO]{देवताप्रत्ययार्थनिर्णयः}
 "सास्य देवता"\footnote{४-२-४} इत्यत्र देवताविशिष्टं देयं प्रत्ययार्थः~।
`ऐन्द्री' `वैश्वदेवी' इत्यादौ इन्द्रादेर्देवतात्वोपस्थापकान्तराभावात्, तेन रूपेणोपस्थितये शक्तिकल्पनाऽऽवश्यकत्वात्~।
अत एव--
\begin{center}आमिक्षां देवतायुक्तां वदत्येवैष तद्धितः~।\\
 आमिक्षापदसान्निध्यात्तस्यैव विषयार्पणम्॥इति॥\\[10pt]

 केवलाद्देवतावाची तद्धितोऽग्नेः समुच्चरन्~।\\
 नान्ययुक्तोऽग्निदैवत्यं प्रतिपादयितं क्षमः॥इति च\end{center} 
मीमांसकैरप्युक्तमित्याशयेनाह-
\begin{center}{\bfseries प्रत्ययार्थस्यैकदेशे प्रकृत्यर्थो विशेषणम्॥५२॥\\[10pt]
 अभेदश्चात्र संसर्ग आग्नेयादावियं स्थितिः~।\\
 देवतायां प्रदेये च खण्डशः शक्तिरस्तु वा॥५३॥}\end{center}

 एकदेशे=देवतारूपे~।
तच्च विशेषणमभेदेनेत्याह-अभेदश्चेति~।
ननु देवतायाः प्रत्ययार्थोकदेशत्वान्न प्रकृत्यर्थस्य तत्राभेदेनाप्यन्वय इत्याशयेनाह-देवतायामिति-तथाच पदार्थैकदेशतैव नास्तीति भावः॥५३॥
 नन्वग्न्यादिदेवस्य प्रकृत्यैव लाभान्न तत्र शक्तिः कल्प्या~।
नच देवतात्वेन रूपेणोपस्थितये सा कल्प्यते~।
प्रकृतेर्लक्षणयैव तथोपस्थितिसम्भवात्~।
उपसर्गाणां द्योतकत्वनये `प्रजयति' इत्यत्र प्रकृष्टजयप्रत्ययवदित्यभिप्रेत्याह-
\begin{center}{\bfseries प्रदेय एव वा शक्तिः प्रकृतेर्वास्तु लक्षणा~।\\
 देवतायां निरूढेति सर्वे पक्षा अमी स्थिताः॥५४॥}\end{center}

 नच `ऐन्द्रं दधि' इत्यादौ द्रव्यस्य पदान्तराल्लाभात् कुतः पुनः प्रत्ययस्य तत्र शक्तिः कल्प्यत इति वाच्यम्, पदान्तराश्रवणोऽपि तत्प्रतीतेः, `ऐन्द्रं दधि' इति सामानाधिकारण्याच्च~।
 अन्यथाऽऽख्यातस्यापि कर्तृकर्मवाचित्वं न स्यात्~।
मीमांसकानां पुनः प्रत्ययस्य देवतात्वमेवार्थोऽस्तु~।
द्रव्यं पदान्तराल्लभ्यत एवेति आख्यातस्य कर्तृवद्वाच्यत्वं माऽस्त्विति कुतो न शक्यते वक्तुमिति दिक्~।
 देवतायां देवतात्वेन रूपेण निरूढेति-अनुपपत्तिज्ञानापूर्वकत्वम्, अनादिप्रयोगावच्छिन्नत्वं वा तत्त्वमिति भावः॥५४॥

अनयैव रीत्याऽन्यत्राप्यवधेयमित्याह-
\begin{center}{\bfseries क्रीडायां णस्तदस्यास्तीत्यादावेषैव दिक् समृता~।\\
 वस्तुतो वृत्तिरेवेति नात्रातीव प्रयत्यते॥५५॥}\end{center}

 "तदस्यां प्रहरणमिति क्रीडायां णः"\footnote{४-२-५७} इत्यत्र प्रहरणविशिष्टा क्रीडा, प्रहरणक्रीडे, क्रीडामात्रं बाऽर्थः~।
आदिना "सोऽस्य निवासः"\footnote{४-३-८९} "साऽस्मिन् पौर्णमासीति" {४-२-२१} "तदास्यास्त्यरिमन्निति मतुप्"\footnote{५-२-१४} इत्यादिकं संगृह्यते~।
 वृत्तिमात्रेऽतिरिक्तशक्तेः, "समर्थः पदविधिः"\footnote{२-१-१} इति सूत्राल्लाभादुक्तो विचारः शास्त्रान्तरीयैः सह तद्रीत्यैवोक्तः~।
आरोपितप्रकृतिप्रत्ययार्थमादाय वा~।
वस्तुतो विशष्टशक्तयैवार्थोपस्थितिरित्याह-वस्तुत इति ॥५५॥

\begin{center}इति वैयाकरणभूषणसारे देवताप्रत्ययार्थनिर्णयः ॥१०॥\end{center}

\section*{\begin{center}अथाभेदैकत्वसङ्ख्यानिर्णयः\end{center}}\addcontentsline{toc}{section}{अभेदैकत्वसङ्ख्यानिर्णयः}
\fancyhead[LE,RO]{अभेदैकत्वसङ्ख्यानिर्णयः}
 वृत्तिप्रसङ्गात् तत्राभेदैकत्वसङ्खया प्रतीयते इति सिद्धान्तं दृष्टान्तेनोपपादयति--
\begin{center}{\bfseries अभेदैकत्वसङ्खयाया वृत्तौ भानमिति स्थितिः~।\\
 कपिञ्जलालम्भवाक्ये त्रित्वं न्यायाद् यथोच्यते ॥५६॥}\end{center}

 सङ्खयाविशेषाणामविभागेन सत्त्वम्=अभेदैकत्वसङ्खया~।
उक्तञ्च वाक्यपदीये-
\begin{center} यथैपधिरसाः सर्वे मधुन्याहितशक्तयः~।\\
 अविभागेन वर्त्तन्ते सङ्खयां तां तादृशीं विदुः॥इति॥\end{center}

 परित्यक्तविशेषं वा सङ्खयासामान्यं तत्~।
उक्तञ्च--
\begin{center}
 भेदानां वा परित्यागात् सङ्खयात्मा स तथाविधः~।\\
 व्यापाराज्जतिभागस्य भेदापोहे(हो) न वर्त्तते॥\\[10pt]
 अगृहीतविशेषेण यथा रूपेण रूपवान्~।\\
 प्रख्यायते न शुक्लादिर्भेदापोहस्तु गम्यते॥इति॥\end{center}

 अस्या वृत्तौ=समासादौ भानं न्यायसिद्धमिति शेषः~।
इति मतस्थितिर्वैयाकरणानाम्~।
 अयं भावः-`राजपुरुषः' इत्यादौ राज्ञः, राज्ञोः, राज्ञां वाऽयं पुरुष इति जिज्ञासा जायते~।
विशेषजिज्ञासा च सामान्यज्ञानपूर्वकेति सामान्यरूपेण तत्प्रतीतिः शब्दादावश्यकी~।
अतस्तस्यां शक्तिरिति~।
तस्या एकत्वत्वेन प्रतीतौ न्यायमाह-कपिञ्जलेति-बहुत्वगणनायां त्रित्वस्यैव प्रथमोपस्थितत्वात् तद्रूपेणैव भानवद् एकत्वस्य सर्वतः प्रथमोपस्थितत्वमस्तीति भावः~।
 वस्तुतस्तु जिज्ञासैव नानुभवसिद्धा~।
तथात्वे वा ज्ञानेच्छयोः समानप्रकारकत्वेनैव हेतुहेतुमद्भावात्तद्रूपेणैव वाच्यता स्यादिति ध्येयम्॥५६॥

\begin{center}इति वैयाकरणभूषणसारे अभेदैकत्वसङ्खयानिरूपणम् ॥११॥\end{center}

\section*{\begin{center}अथ सङ्ख्याविवक्षानिर्णयः\end{center}}\addcontentsline{toc}{section}{सङ्ख्याविवक्षानिर्णयः}
\fancyhead[LE,RO]{सङ्ख्याविवक्षानिर्णयः}
सङ्खयाप्रसङ्गादुद्देश्यविधेययोः सङ्खयाविवक्षाऽविवक्षे निरूपयति--

\begin{center}{\bfseries लक्षयानुरोधात् सङ्खयायास्तन्त्रातन्त्रे मते यतः~।\\
 पश्वेकत्वादिहेतूनामाश्रयणमनाकरम्॥५७॥}\end{center}

 "ग्रहं समार्ष्टि" इत्यत्रोद्देशयग्रहगतमेकत्वमविवक्षितमितिवन्नास्माकमुद्देश्य- विशेषणविवक्षानियमः, `धातोः' इत्येकत्वस्य विवक्षितत्वात्,
\begin{center} उत्पद्येत समस्तेभ्यो धातुभ्यः प्रत्ययो यदि~।\\
 तदा सर्वैर्विशिष्येत द्वन्द्वोत्पन्नसुबर्थवत्॥इति॥\end{center}

 शब्दान्तराधिकरणे भट्टपादैरभिधानाच्च, "आर्द्धधातुकस्येड् वलादेः"\footnote{७-२-३५} इत्यत्रानुवाद्यार्द्धधातुकविशेषणस्य वलादित्वस्य विवक्षितत्वाच्च~।
 एवम्-`पशुना यजेत' इतिवद् विधेयविशेषणं विवक्षितमित्यपि नियमो न, "रदाभ्यांनिष्ठातो नः पूर्वस्य च दः"\footnote{८-२-४२} इत्यत्र नकारद्वयविधानानापत्तेः~।
तथाच `भिन्नः' इत्यत्र नकारद्वयलाभो न स्यात्~।
``आद् गुणः" {६-१-८७} इत्यादावेकत्वविवक्षयैवोपपत्तौ ``एकः पूर्वपरयोः" {६-१-८४} इत्येकग्रहणवैयर्थ्यापत्तेश्चेति भावः~।
शब्दार्थस्तु-सङ्खयाया लक्षयानुरोधात्तन्त्राऽतन्त्रे यतो मते, अतः पश्वेकत्वाधिकरणोक्तहेतूनामाश्रयणं नास्मत्सिद्धान्तसिद्धमिति~।
आदिना ग्रहैकत्वसंग्रहः॥५७॥
 ननु विधेयविशेषणविवक्षा आवश्यकी, अन्यथा `सुद्धयुपास्यः' इत्यादावनन्तयकाराद्यापत्तेः~।
`भिन्नः' इत्यत्र नकारद्वयवदन्येषामप्यापत्तेः~।
`एकः पूर्वपरयोः' इत्यत्रैकग्रहणञ्च स्थानिभेदादादेशभेदवारणायेत्यभिप्रेत्याह--
\begin{center}{\bfseries विधेये भेदकं तन्त्रमन्यतो नियमो नहि~।\\
 ग्रहैकत्वादिहेतूनामाश्रयणमनाकरम्॥५८॥}\end{center}

 भेदकम्=विशेषणम्~।
तन्त्रम्=बिवक्षितम्~।
विधेयविशेषणं विवक्षितमित्यस्तु, तथाप्यन्यतः- अनुवाद्यस्य नियमो नहि~।
क्वचित् तन्त्रम्, क्वचिन्नेत्यर्थः~।
ग्रहैकत्वादौ यो हेतुर्वायभेदादिस्तस्यात्राश्रयणमनाकरम्~।
एकत्वविशिष्टं धातुम्, वलादित्वविशिष्टमार्द्धधातुकञ्चोद्दिस्य प्रत्ययेडागमादेर्विधिसम्भवादिति भावः॥५८॥

 नन्वेवम् `भिन्नः' इत्यत्र नकारद्वयलाभो न स्यादित्यत आह-

\begin{center}{\bfseries रदाभ्यां वाक्यभेदेन नकारद्वयलाभतः~।\\
 क्षतिर्नैवास्ति तन्त्रत्वे विधेये भेदकस्य तु॥५९॥}\end{center}

 चकारसूचितम्-निष्ठातस्य नः, पूर्वस्य दकारस्य च न इति वाक्यभेदमादाय नकारद्वयलाभ इत्यर्थः॥५९॥

\begin{center} इति वैयाकरणभूषणसारे सङ्खयाविवक्षानिर्णयः॥१२॥\end{center}
 
 
\section*{\begin{center}अथ क्त्वाद्यर्थनिर्णयः\end{center}}\addcontentsline{toc}{section}{क्त्वाद्यर्थनिर्णयः}
\fancyhead[LE,RO]{क्त्वाद्यर्थनिर्णयः}
क्त्वाप्रत्ययादेरर्थं निरूपयति-
\begin{center}{\bfseries अव्ययकृत इत्युक्तेः प्रकृत्यर्थे तुमादयः~।\\
 समानकर्तृकत्वादि द्योत्यमेषामिति स्थितिः॥६०॥}\end{center}

 तुमादयः - तुमुन्नादयः~।
प्रकृत्यर्थे - भावे~।

 आदिना क्त्वादेः संग्रहः~।
भावे इत्यत्र मानमाह - अव्ययकृत इति~।
'अव्ययकृतो भावे' इति वार्त्तिकादित्यर्थः~।
ननु `समानकर्तृकयोः पूर्वकाले' इत्यादिसूत्राणां का गतिस्तत्राह - समानकर्तृकत्वादीति~।
अयं भावः-`भोक्तुं पचति' `भुक्त्वा व्रजति' इत्यादावेकवाक्यता सर्वसिद्धा भोजनपाकक्रिययोर्विशेष्यर्विशेषणभावमन्तरेणानुपपन्ना~।
अन्यथा `भुङ्क्ते' `व्रजति' इत्यादावप्येकवाक्यताऽऽपत्तेः~।
तथाच तयोर्विशेष्यविशेषणभावनिरूपकः संसर्गः - जन्यत्वम्~, सामानाधिकरण्यम्~, पूर्वोत्तरभावः, व्याप्यत्वञ्चेत्यादिरनेकविधः~।
तथाच - `भोक्तुं पचति', `भुक्त्वा तृप्तः' इत्यादौ भोजनजनिका पाकक्रिया, भोजनजन्या तृप्तिरिति बोधः~।
अत एव जलपानानन्तर्यस्य तृप्तौ सत्त्वेऽपि `पीत्वा तृप्तः' इति न प्रयोगः~।
सामानाधिकरण्यस्यापि संसर्गत्वेनार्थात्समानकर्तृकत्वमपि लब्धम्~।
`भुक्त्वा व्रजति' इत्यादौ पूर्वोत्तरभावः, सामानाधिकरण्यञ्च संसर्ग इति भोजनसमानाधिकरणा तदुत्तरकालिकी व्रजनक्रियेति बोधः~।
`अधीत्य तिष्ठति', 'मुखं व्यादाय स्वपिति' इत्यादावध्ययनव्यादानयोरभावकालेऽप्रयोगात् यदा यदाऽस्य स्थितिः, स्वापश्च, तदा तदाऽध्ययनं मुखव्यादानञ्चेति कालविशेषावच्छिन्नव्याप्यत्वबोधात्~।
व्याप्यत्वं सामानाधिकरण्यञ्च संसर्गः~।
 एवञ्चान्यलभ्यत्वान्न सूत्रात्तेषां वाच्यत्वलाभः इति युक्तम्-`अव्ययकृतो भावे' इति~।
एवञ्च प्रकृत्यर्थक्रिययोः संसर्गे तात्पर्य्यग्राहकत्वरूपं द्योतकत्वं क्त्वादीनाम्~।
अत एव 'समानकर्तृकयोः' इति सूत्रे `स्वशब्देनोपात्तत्वान्न' इतिभाष्यप्रतीकमादाय `--पौर्वापर्य्यकाले द्योत्ये क्त्वादिर्विधीयते, न तु विषये इति भाव इति' कैयटः~।
 यत्तु-`समानकर्तृकयोः' इति सूत्रात्समानकर्तृकत्वं क्त्वावाच्यम्~, अन्यथौदनं पक्त्वाऽहं भोक्ष्ये - इत्यत्र मयेति तृतीयाप्रसङ्गश्च~।
न चाख्यातेन कर्तुरभिधानान्न सेति वाच्यम्~, भोजनक्रियाकर्तुरभिधानेऽपि पाकक्रियाकर्तुस्तदभावात्~।
 अनभिहिते भवति इति पर्युदासाश्रयणात्~।
अत एव `प्रासादे आस्ते' इत्यत्र प्रसादनक्रियाधिकरणस्याभिधानेऽप्यास्तिक्रियाधिकरणस्यानभिधानात्सप्तमीति भाष्ये स्पष्टम्~।
तस्मात् क्त्वाप्रत्ययस्य कर्तृवाचित्वमावश्यकमिति, तन्न~।
सूत्रात्तस्य वाच्यत्वालाभात्~, `समानकर्तृकयोः क्रिययोः पूर्वकाले क्त्वा' इत्येव तदर्थात्~।
अन्यथा `समानकर्तरि' इत्येव सूत्रन्यासः स्यात्~।
तृतीयापादनन्तु आख्यातार्थक्रियायाः प्रधानभूतायाः कर्तुरभिधानात्प्रधानानुरोधेन गुणे कार्यप्रवृत्तेर्न सम्भवति~।
उक्तञ्च वाक्यपदीये--
\begin{center} प्रधानेतरयोर्यत्र द्रव्यस्य क्रिययोः पृथक्~।\\
 शक्तिर्गुणाश्रया तत्र प्रधानमनुरुद्धयते ॥\\[10pt]
 प्रधानविषया शक्तिः प्रत्ययेनाभिधीयते~।\\
यदा गुणे तदा तद्वदनुक्ताऽपि प्रतीयते ॥इति~।\end{center}
 किञ्चान्यथा कर्मणोऽपि क्त्वार्थताऽऽपत्तिः, `पक्त्वैदनो मया भुज्यते' इत्यत्र द्वितीयायाः प्रकारान्तरेणावारणादित्यास्तां विस्तरः॥६०॥
 इति वैयाकरणभूषणसारे क्त्वाद्यर्थनिर्णयः॥१३॥

\section*{\begin{center}अथ स्फोटनिरूपणम्\end{center}}\addcontentsline{toc}{section}{स्फोटनिरूपणम्}
\fancyhead[LE,RO]{स्फोटनिरूपणम्}
सिद्धान्तनिष्कर्षमाह-
\begin{center}{\bfseries वाक्यस्फोटोऽतिनिष्कर्षे तिष्ठतीति मतस्थितिः~।\\
 साधुशब्देऽन्तर्गता हि बोधका नतु तत्स्मृताः ॥६१॥}\end{center}

 यद्यपि वर्णस्फोटः, पदस्फोटः, वाक्यस्फोटः, अखण्डपदवाक्यस्फोटौ, वर्णपदवाक्यभेदेन त्रयो जातिस्फोटा इत्यष्टौ पक्षाः सिद्धान्तसिद्धा इति वाक्यग्रहणमनर्थकम् , दुरर्थकञ्च, तथाऽपि वाक्यस्फोटातिरिक्तानामन्येषामप्यवास्तवत्वबोधनाय तदुपादानम्~।
एतदेव ध्वनयति - अतिनिष्कर्ष इति~। इति मतस्थितिर्वैयाकरणानाम् = महाभाष्यकारादीनाम्~।
तत्र क्रमेण सर्वांस्तान् निरूपयन् वर्णस्फोटं प्रथममाह - साधुशब्द इति~। साधुशब्दान्तर्गता वाचका नवेति विप्रतिपत्तिः~।
 विधिकोटिरन्येषाम् , नेति वैयाकरणानाम्~।
साधुशब्दे `पचति', `रामः' इत्यादिप्रयुज्यमानेऽन्तर्गतास्तिब्विसर्गादय एव बोधकाः = वाचकाः, तस्यैव शक्तत्वस्य प्राग्व्यवस्थापितत्वात्~।
न तु तैः स्मृताः लादयः, स्वादयश्चेत्यर्थः ॥६१॥
ये तु प्रयोगान्तर्गतास्तिब्विसर्गादयो न वाचकाः, तेषां बहुत्वेन शक्त्यानन्त्यापत्तेः, `एधाञ्चक्रे', `ब्रह्म' इत्यादावादेशलुगादेरभावरूपस्य बोधकत्वासम्भवाच्च~।
किन्तु तैः स्मृता लकाराः स्वादयश्च वाचकाः, लत्वस्य जातिरूपतया शक्तताऽवच्छेदकत्वौचित्यात् , अव्यभिचाराच्च~।
आदेशानां भिन्नतया परस्परव्यभिचरितत्वात्~।
लः कर्मणीत्याद्यनुशासनानुगुण्याच्च~।
न ह्यादेशेष्वर्थवत्ताबोधकमनुशासनमुपलभामहे इत्याहुः~।
तान् स्वसाधकयुक्तिभिर्निराचष्टे-
\begin{center}{\bfseries व्यवस्थितेर्व्यवहृतेस्तद्धेतुन्यायतस्तथा~।\\
 किञ्चाऽख्यातेन शत्राद्यैर्लडेव स्मार्यते यदि ॥६२॥\\[10pt]
 कथं कर्तुरवाच्यत्ववाच्यत्वे तद्विभावय~।}\end{center}
 व्यास्थानुरोधात् प्रयोगान्तर्गता एव वाचकाः, न तु तैः स्मृता इत्यर्थः~।
तथाहि - पचतीत्यादौ लकारमविदुषो बोधान्न तस्य वाचकत्वम्~।
न च तेषां तिङ्क्षु शक्तिभ्रमाद् बोधः, तस्य भ्रमत्वे मानाभावात्, आदेशिनामपि तत्तद्वैयाकरणैः स्वेच्छया भिन्नानामभ्युपगमात्, कः शक्तः को नेति व्यवस्थानापत्तेश्च~।
सर्वेषां शक्तत्वे गौरवम्, व्यभिचारश्चास्त्येव~।
आदेशानां प्रयोगान्तर्गततया नियतत्वाद् युक्तं तेषां शक्तत्वम्~।
तथाचादेशिस्मरणकल्पना नेति लाघवम्~।
साधकान्तरमाह- व्यवहृतेरिति~। व्यवहारस्तावच्छक्तिग्राहकेषु मुख्यः~।
 स च श्रूयमाणतिङादिष्वेवेति ते एव वाचका इत्यर्थः~।
किञ्च तद्धेतुन्यायत इति--
 लकारस्य बोधकत्वे `भू ल्' इत्यतोऽपि बोधापत्तिः स्यात्~।
तादृशबोधे भवतीति समभिव्याहारोऽपि कारणमिति चेत्, तर्ह्यावशयकत्वादस्तु तादृशसमभिव्याहारस्यैव वाचकत्वशक्तिः~।
 अन्यथा लकारस्य वाचकत्वम्, समभिव्याहारस्य कारणत्वञ्चेत्युभयं कल्प्यमिति गौरवं स्यात्~।
तथाच तादृशसमभिव्याहृता वर्णा वेत्यत्र विनिगमनाविरहात् प्रयोगान्तर्गता वर्णा वाचका इति सिद्ध्यतीति भावः~।
 अपिच लकारस्यैव वाचकत्वे कृत्तिङोः कर्तृभावनावाचकत्वव्यवस्था त्वत्सिद्धान्तसिद्धा न स्यादित्याह- किञ्चेति~। 
आदेशानां वाचकत्वे च तिङ्त्वेन भावनायाम्, शानचादिना कर्तरि शक्तिरित्युपपद्यते विभाग इति भावः~।
न च शानजादौ कृतिर्लकारार्थः, आश्रयः शानजर्थ इत्यस्तु, "कर्तरि कृत्" इत्यनुशासनादिति शङ्क्यम्~।
स्थान्यर्थेन निराकाङ्क्षतया शानजादौ कर्तरीत्यस्याप्रवृत्तेः~।
अन्यथा घञादावपि प्रवर्त्तेत~॥६२॥
'देवदत्तः पचमानः' इत्यादिसामानाधिकरण्यानुरोधाच्छानचः कर्ता वाच्यः स्यादित्याशङ्क्याह-
\begin{center}{\bfseries तरबाद्यन्ततिङ्क्षवस्ति नामता कृत्स्विव स्फुटा॥६३॥\\
 नामार्थयोरभेदोऽपि तस्मात्तुल्योऽवधार्य्यताम्~।}\end{center}

'पचतितरां मैत्रः' 'पचतिकल्पं मैत्रः' इत्यादिषु नामार्थत्वाभेदान्वययोः सम्भव एवेति कर्तृवाचकता स्यादिति भावः~।
नच पचतिकल्पमित्यत्र सामानाधिकरण्यानुरोधात् कर्तरि लक्षणा, 'पचमानः' इत्यत्राप्यापत्तेरिति~।
 लः कर्मणीत्यनुशासनञ्च लाघवात् कल्पिते लकारे कर्त्रादिर्वाचित्वं कल्पितमादायेत्युक्तम्॥६३॥
 
 \begin{center}॥इति वर्णस्फोटनिरूपणम्॥\end{center}
\begin{center}{\bfseries अथादेशा वाचकाश्चेत् पदस्फोटस्ततः स्फुटः॥६४॥}\end{center}
 एवमादेशानां वाचकत्वे सिद्धे पदस्फोटोऽपि सिद्ध एवेत्याह-अथेत्यादि~।
आदेशाः=तिब्विसर्गादयः~।
अयं भावः-समभिव्याहृतवर्णानां वाचकत्वे सिद्धे तादृशवर्णसमभिव्याहाररूपपदस्य वाचकता सिद्धयति, प्रतिवर्णमर्थस्मरणस्यानुभवविरुद्धत्वात्~, प्रत्येकं वर्णानामर्थवत्त्वे प्रातिपदिकत्वापत्तौ "न लोपः प्रातिपदिकान्तस्य" इत्यादिभिः `धनम्' `वनम्' इत्यादौ नलोपाद्यापत्तेश्च~।
 एतच्च चरमवर्णे एव वाचकत्वशक्तिः, शक्तेर्व्यासज्यवृत्तित्वे मानाभावात्~।
पूर्वपूर्ववर्णानुभवजन्यसंस्काराश्चरमेणार्थधीजनने सहकारिण इति न तन्मात्रोच्चारणादर्थधीरिति वर्णास्फोटवादिनां मतान्तरस्य दूषणायोक्तम्~।
रामोऽस्तीति वक्तव्ये राम् इत्यनन्तरं घटिकोत्तरमोकारोच्चारणेऽर्थबोधापत्त्या तादृशानुपूर्व्या एव शक्तताऽवच्छेदककत्वैचित्यादिति दिक्॥६४॥
 `सुप्तिङन्तं पदम्' इति पारिभाषिकपदस्य वाचकत्वस्वीकर्तॄणां मतमाह-
\begin{center}{\bfseries घटेनेत्यादिषु नहि प्रकृत्यादिभिदा स्थिता~।\\
 वस्नसादाविवेहापि सम्प्रमोहो हि दृश्यते॥६५॥}\end{center}

 घटेनेत्यादौ `घटे' इति प्रकृतिः, `न' इति प्रत्ययः, `घट्' इति प्रकृतिः, `एन' प्रत्यय इति विभागस्य, `सर्वे सर्वपदादेशाः' इति स्वीकारे विशिष्य प्रकृतिप्रत्यययोर्ज्ञानासम्भवान्न वाचकत्वमित्यर्थः~।
वैयाकरणैर्विभागः सुज्ञेय इत्यतो दृष्टान्तव्याजेनाह-वस्नसादाविति~।
`बहुवचनस्य वस्नसौ' इति समुदायस्याऽऽदेशविधानान्नात्र तद्विभागः सम्भवतीत्यर्थः~॥६५॥
सुप्तिङन्तचयरूपवाक्यस्यापि तदाह-

\begin{center}{\bfseries हरेऽवेत्यादि दृष्ट्वा च वाक्यस्फोटं विनिश्चिनु~।\\[10pt]
अर्थे विशिष्यसम्बन्धाग्रहणं चेत् समं पदे॥६६॥\\
 लक्षणादधुना चेत्तत्पदेऽर्थेऽप्यस्तु तत् तथा~।}\end{center}

 `हरेऽव' 'विष्णोऽव' इत्यादौ पदयोः "एङः पदान्तादति" इत्येकादेशे सति न तद्विभागः सुज्ञानः~।
तथाच प्रत्येकं पदाज्ञानेऽपि समुदायशक्तिज्ञानाच्छाब्दबोधात् समुदायेऽप्यावशियकी शक्तिः~।
एवञ्च प्रकृतिप्रत्ययेषु विशिष्याज्ञायमानेष्वपि समुदायव्युत्पत्त्या बोधात् तत्राप्यावश्यिकैव शक्तिरिति भावः~।
वस्तुतः पदैः पदार्थबोधवद्वाक्येन वाक्यार्थबोध इति पदार्थशक्तिः पदेष्विव वाक्यार्थशक्तिर्वाक्येऽभ्युपेयेति पदस्फोटवाक्यस्फोटौ व्यवस्थितौ~।
अन्यथा `घटः कर्मत्वमानयनं कृतिः' इत्यादौ तादृशव्युत्पत्तिरहितस्यापि बोधपसङ्गः~।
`घटमानय' इत्यत्रेव पदार्थानामुपस्थितौ सत्यपि तात्पर्य्यज्ञाने बोधाभावाच्च~।
तत्रैव घटकर्मकमानयनमिति बोधे घटार्थकप्रातिपदिकोत्तरं कर्मत्ववाचकविभक्तेस्ततो धातोस्तत आख्यातस्य समभिव्याहारः कारणमिति कार्यकारणभावज्ञानवतो बोधात्तज्ज्ञानमपि हेतुरिति चेत्तर्हि सिद्धो वाक्यस्फोटः, घटादिपदार्थबोधे बोधकतारूपपदशक्तिज्ञानकार्यकारणभावस्येव विशिष्टवाक्यार्थबोधे पदसमभिव्याहाररूपवाक्यनिष्ठबोधकतारूपवाक्यशक्तिज्ञानस्यापि हेतुत्वकल्पनात्~, अर्थोपस्थापकज्ञानविषयशब्दवृत्तिज्ञानकारणत्वस्यैव शक्तित्वात्~।
 युक्तञ्चैतत्- विषयतासम्बन्धेन शाब्दबोधमात्रे वृत्तिज्ञानस्य लाघवेन हेतुत्वसिद्धेः~।
विवेचितञ्चैतद् भूषणे~।
 ननु वाक्यार्थस्यापूर्वत्वात् कथं तत्र शक्तिग्रह इत्याशङ्क्याह-अर्थ इति~।
वाक्यस्येति शेषः~।
वाक्यस्य वाक्यार्थे विशिष्य शक्त्यग्रहणञ्चेत्तर्हि पदेऽपि समम्~।
पदे एवान्वयांशे शक्तिरिति पक्षेऽपि तद्ग्रहासम्भवस्तुल्य इत्यर्थः~।
यदि च पदशक्तिः पदार्थंशे ज्ञाता, अन्वयांशे चाज्ञातोपयुज्यत इति कुब्जशक्तिवादस्तदा ममापि वाक्यस्य शक्तिरज्ञातैवोपयुज्यत इति वादाभ्युपगमस्तुल्य इति भावः~।
ननु वृद्धव्यवहारं पश्यतो मनसा पदार्थवद्वाक्यार्थेऽपि तद्ग्रह इति चेत्~, तुल्यमित्याह-लक्षणादिति~।
लक्ष्यते-तर्क्यतेऽनेनेति लक्षणम् - मनस्तस्मात्~।
अपिपदं पदपदोत्तरं बोध्यम्~।
पदेऽपि लक्षणात्तदग्रहश्चेत्तर्ह्यस्तु वाक्येऽपीति शेषः~।
वस्तुतस्तु समुदितार्थे विशिष्टवाक्यस्यैव प्रथमं तद्ग्रहः~।
आवापोद्वापाभ्यां परं प्रत्येकं तद्ग्रह इति बोध्यम्॥६६॥

 इयमेव मीमांसकानां वेदान्तैकदेशिनाञ्च गतिरित्याह-
\begin{center}{\bfseries सर्वत्रैव हि वाक्यार्थो लक्ष्य एवेति ये विदुः~।\\
 भाट्टास्तेऽपीत्थमेवाहुर्लक्षणाया ग्रहे गतिम्॥६७॥}\end{center}

 भाट्टा इति~।
 तदनुयायिनां वाचस्पति-कल्पतरुप्रभृतीनामुपलक्षणम्~।
ननूक्तपक्षद्वयमनुपपन्नम्, उत्पत्तेरभिव्यक्तेर्वैकदाऽसम्भवेन उत्पन्नानामभिव्यक्तानां वर्णसमूहरूपपदज्ञानासम्भवात्~।
तथाच सुतरां तत्समूहरूपवाक्यज्ञानासम्भव इति चेन्न, उत्तरवर्णप्रत्यक्षसमये तस्मिन्नव्यवहितोत्तरत्वसम्बन्धेनोपस्थितपूर्ववर्णावत्त्वम्, तथा तदुत्तरप्रत्यक्षकाले उपस्थितविशिष्टतद्वर्णवत्त्वं तस्मिन् सुग्रहमिति तादृशानुपूर्वोघटितपदत्वस्येव वाक्यत्वस्यापि सुग्रहत्वात्॥६७॥

इदानीमखण्डपक्षमाह-
\begin{center}{\bfseries पदे न वर्णा विद्यन्ते वर्णोष्ववयवा न च~।\\
 वाक्यात् पदानामत्यन्तं प्रविवेको न कश्चन॥६८॥}\end{center}

 पदे - पचतीत्यादौ, न वर्णाः, नातो वर्णसमूहः पदमिति शेषः~।
दृष्टान्तव्याजेनाह - वर्णोष्विति~।
एकारौकारॠकारऌकारादिवर्णेऽष्ववयवाः प्रतीयमाना अपि यथा नेत्यर्थः~।
क्वचिदिवेत्येव पाठः~।
एवं वाक्येऽप्याह- वाक्यादिति~।
पदानामपि वाक्याद्विवेकः - भेदो नास्तीत्यर्थः~।
अयं भावः- वाक्यं पदञ्चाखण्डमेव, नतु वर्णसमूहः, अनन्तवर्णकल्पने मानाभावात्~।
तत्तद्वर्णोत्पादकत्वेनाभिमतवायुसंयोगनिष्ठं तत्तद्वर्णजनकतायाः, व्याञ्जकताया वाऽवच्छेदकं वैजात्यमादायैव ककारो गकार इत्यादिप्रतीतिवैलक्षण्यसम्भवात्~।
स्पष्टं हि भामत्याम्-"तारत्वादि वायुनिष्ठं वर्णेष्वारोप्यते" इत्युक्तं देवताधिकरणे~।
न चैवं वायुसंयोग एव वाचकोऽपि किं न स्यादिति वाच्यम्, प्रत्यक्षोपलभ्यमानककारादेरेव वाचकत्वस्यानुभवसिद्धत्वात्~।
तथाच वाचकत्वान्यथानुपपत्त्या 'तदेवेदं पदम्', `तदेवेदं वाक्यम्' 'सोऽयं गकारः', इति प्रतीत्या च स्फोटोऽखण्डः सिद्धयति~।
एतेन गौरित्यादौ गकारौकारविसर्गादिव्यतिरेकेण स्फोटाननुभवाच्छ्रूयमाणवर्णानामेव वाचकत्वमस्तीत्यापास्तम्~।
तेषां स्फोटातिरिक्तत्वाभावात्~।
यत्तु वर्णानां प्रत्येकं वाचकत्वे प्रत्येकादर्थबोधापत्तिः~।
समुदायस्य तु क्रमवतामाशुतरोत्पन्नानां तथैवाभिव्यक्तानां वा ज्ञानमसम्भाव्यमेव~।
पूर्वपूर्ववर्णानुभवसंस्कारसहकारेणैकदा समूहालम्बनरूपसकलज्ञानसम्भवस्तु 'सरो - रसः', 'राज-जरा', 'नदी-दीना'दिसाधारण इत्यतिप्रसङ्ग इति स्फोट एवाखण्डो नादाभिव्यङ्ग्यो वाचक इति कैयटः, तत्तुच्छम्~, पदज्ञानसम्भवस्योपपादितत्वात्~।
वर्णानां प्रत्येकं व्यञ्जकत्वं समुदितानां वेत्यादिविकल्पग्रासाच्च~।
ननु त्वन्मतेऽप्येष दोषः~।
तत्तद्वर्णोत्पादकत्वेनाभिमतवायुसंयोगानां प्रत्येकं व्यञ्जकत्वं समुदितानां वेति विकल्पस्य सद्भावादिति चेत्~, उच्यते~।
प्रत्येकमेव संयोगा अभिव्यञ्जकाः, परन्तु केचिद् गत्वेन, केचिदौत्वेन केचिद्विसर्गत्वेनेत्यनेकैः प्रकारैः~।
अत एव वर्णानां तदतिरेकास्वीकारोऽप्युपपद्यते~।
एवञ्चाव्यवहितोत्तरत्वसम्बन्धेन घवत्त्वं टकारे गृह्यते~।
एतादृशपदज्ञानकारणताया अविवादात्~।
परन्त्वव्यहितोत्तरत्वं स्वज्ञानाधिकरणक्षणोत्पत्तिकज्ञानविषयत्वं वाच्यम्~।
अत एव घज्ञानानन्तरटज्ञानविषयत्वरूपानुपूर्वीत्यादिर्नैयायिकवृद्धानां व्यवहारः~।
एवञ्च न कश्चिद्दोषः~।
एतेन पर्य्यायस्थलेष्वेक एव स्फोटो नाना वा? नाद्यः, घटपदे एव गृहीतशक्तिकस्य कलशादेर्बोधप्रसङ्गात्~। नच तत्पर्यायाभिव्यक्ते शक्तिग्रहस्तत्पर्य्यायश्रवणेऽर्थधीहेतुरिति वाच्यम्, एवं सति प्रतिपर्य्यायं शक्तिग्रहावश्यम्भावेन तत्तत्पर्य्यायगतशक्तिग्रहहेतुताया उचितत्वात्~।
तथा सति शक्तिग्रहत्वेनैव हेतुत्वे लाघवात्~।
अन्यथा तत्पर्य्यायाभिव्यक्तगतशक्तिग्रहत्वेन तत्त्वे गौरवात्~।
न द्वितीयः, अनन्तपदानां तेषां शक्तिञ्चापेक्ष्य क्लृप्तवर्णेष्वेव शक्तिकल्पनस्य लघुत्वादिति परिमलोक्तमपास्तम्~।
पर्यायेष्वनेकशक्तिस्वीकारस्य सर्वसिद्धत्वात्, तदवच्छेदकानुपूर्व्याः प्रागुपपादनादिति दिक्~।
शब्दकौस्तुभे तु वर्णमालायां पदमिति प्रतीतेर्वर्णातिरिक्त एव स्फोटः~।
अन्यथा कपालातिरिक्तघटाद्यसिद्धिप्रसङ्गश्चेति प्रतिपादितम्॥६८॥

नन्वेवं शास्त्रस्याप्रामाण्यप्रसङ्गः, पदस्याखण्डत्वात्, शास्त्रस्य च प्रकृतिप्रत्ययाभ्यां पदव्युत्पादनमात्रार्थत्वादित्याशङ्कां समाधत्ते-
\begin{center}{\bfseries पञ्चकोशादिवत्तस्मात् कल्पनैषा समाश्रिता~।\\
उपेयप्रतिपत्त्यर्था उपाया अव्यवस्थिताः॥६९॥}\end{center}
उपेयप्रतिपत्त्यर्था इत्यन्तेनान्वयः~।
अयं भावः-यथा भृगुवल्याम् "भृगुर्वै वारुणि र्वरुणं ब्रह्म पृष्टवान्~।
स उवाच `अन्नम्' इति, तस्योत्पत्त्यादिकं बुद्ध्वा पृष्टे-प्राणमनोविज्ञानाऽनन्दात्मकपञ्चकोशोत्तरं " ब्रह्मपुच्छं प्रतिष्ठा" इति ज्ञेयं ब्रह्म प्रतिपादितम्~।
तत्र कोशपञ्चकव्युत्पादनं शुद्धब्रह्मबोधनाय~।
यथा वा आनन्दवल्लीस्थपञ्चकोशव्युत्पादनं वास्तवशुद्धब्रह्मबोधनाय~।
एवं प्रकृतिप्रत्ययादिव्युत्पादनं वास्तवस्फोटव्युत्पादनायैवेति~।
ननु प्रत्यक्षस्य स्फोटस्य श्रवणादितोऽपि बोधसम्भवात् न शास्त्रं तदुपाय इत्यत आह-उपाया इति~।
उपायस्योपायान्तरादूषकत्वात्~।
तथाच व्याकरणाभ्यासजन्यज्ञाने वैजात्यं कल्प्यते~।
मन्त्रजन्यमिवार्थस्मरणे~।
वेदान्तजन्यमिव ब्रह्मज्ञाने~।
तस्य च ज्ञानस्य यज्ञादीनामन्तःकरणशुद्धाविव शरीरादिशुद्धावुपयोगः, साक्षात्परम्परया वा स्वर्गमोक्षादिहेतुत्वञ्च~।
तदुक्तं वाक्यपदीये-
\begin{center}तद्द्वारमपवर्गस्य वाङ्-मलानां चिकित्सितम्~।\\
पवित्रं सर्वविद्यानामधिविद्यं प्रकाशते~।\\[10pt]
इदमाद्यं पदस्थानं सिद्धिसोपानपर्वणाम्~।\\
इयं सा मोक्षमाणानामजिह्मा राजपद्धतिः~।\\[10pt]
अत्रातीतविपर्यासः केवलामनुपश्यति॥ इति~।\end{center}
न चालीकया प्रकृतिप्रत्ययकल्पनया कथं वास्तवस्फोटबोधः, तस्या अलीकत्वासिद्धेर्वक्षयमाणत्वात्~।
एवं 'रेखागवयन्यायः' आदिपदेन गृह्यते॥६९॥
 ननु स्फोटस्य वर्णजातीनाञ्च नित्यतया ककार उत्पन्न इति न स्यात्~।
वायुसंयोगनिष्ठजातेः स्फोटे भाने कादिप्रतीतीनां भ्रमत्वापत्तिश्चेत्यत आह-
\begin{center}{\bfseries कल्पितानामुपाधित्वं स्वीकृतं हि परैरपि~।\\
स्वरदैर्घ्याद्यपि ह्यन्ये वर्णेभ्योऽन्यस्य मन्वते॥७०॥}\end{center}
 स्वीकारस्थलमाह- स्वरदैर्घ्याद्यपीति~।
आदिनोत्पत्तिविनाशसङ्ग्रहः~।
उदात्तत्वादि न वर्णनिष्ठम्, तस्यैकत्वात्, नित्यत्वाच्च~।
तच्च, स एवायमिति प्रत्यभिज्ञानात्~।
न च गत्वावच्छिन्नप्रतियोगिताकभेदाभावस्तद्विषयः~।
व्यक्त्यंशाभेदस्यापि भासमानस्य विना बाधकं त्यागायोगत्~।
न चोत्पत्तिप्रतीतिर्बाधिका, प्रागसत्त्वे सति सत्त्वरूपाया उत्पत्तेर्वर्णेष्वनुभवविरुद्धत्वात्~।
अत एव वर्णमुच्चारयतीति प्रत्ययः, नतूत्पादयतीति प्रत्ययो व्यवहारश्च~।
उच्चरितत्वञ्च-ताल्वोष्ठसंयोगादिजन्याभिव्यक्तिविशिष्टत्वम्~।
किञ्च व्यञ्जकध्वनिनिष्ठोत्पत्त्यादेः परम्परया वर्णनिष्ठत्वविषयत्वेनाऽप्युपपत्तेर्न साऽतिरिक्तवर्णसाधिका~।
परम्परया वर्णनिष्ठत्वाभ्युपगमाच्च न भ्रमत्वम्~।
साक्षात्सम्बन्धांशे भ्रम इत्यवशिष्यते~।
तदपि सोऽयमित्यत्र व्यक्त्यभेदांशे तव भ्रमत्ववत्तुल्यम्~।
परन्तु ममातिरिक्तवर्णतत्प्रागभावध्वंसकल्पना नेति लाघवमतिरिच्यते~।
नच वर्णस्थले ध्वनिसत्त्वे मानाभावः, तदुत्पादकशङ्खाद्यभावेन तदसम्भवश्चेति वाच्यम्~, ककाराद्युच्चारणस्थले तत्तत्स्थानस्य जिह्वाया ईषदन्तरपाते वर्णानुत्पत्तेर्ध्वन्युत्पत्तेश्च दर्शनात्~, जिह्वाभिघातजवायुकण्ठसंयोगादेर्ध्वनिजनकत्वकल्पनात्~।
तस्य च वर्णोत्पत्तिस्थलेऽपि सत्त्वात्तवैव प्रतिबन्ध्यप्रतिबन्धकभावकल्पना निष्प्रामाणिकी स्यादिति विपरीतगौरवम्~।
एवं परस्परविरोधादुदात्तत्वानुदात्तत्वह्रस्वत्वदीर्घत्वादिकमपि न वर्णनिष्ठं युक्तमिति तेषामभिप्रायः~।
एवञ्चोत्पत्त्यादिप्रतीतीनां तत्प्रमात्वस्य च निर्वाहः परेषामपि समान इति प्रतिबन्द्यैवोत्तरमिति भावः॥७०॥
 इत्थञ्च पञ्चधा व्यक्तिस्फोटाः~।
जातिस्फोटमाह-
\begin{center}{\bfseries शक्यत्व इव शक्तत्वे जातेर्लाघवमीक्ष्यताम्~।\\
औपाधिको वा भेदोऽस्तु वर्णानां तारमन्दवत्॥७१॥}\end{center}
 अयं भावः- वर्णास्तावदावश्यकाः~।
उक्तरीत्या च `सोऽयं गकारः' इतिवत्~, योऽयं गकारः श्रुतः सोऽयं हकार इत्यपि स्यात्~, स्फोटस्यैकत्वात्~, गकारोऽयं न हकार इत्यनापत्तेश्च~।
किञ्च स्फोटे गत्वाद्यभ्युपेयं न वा ? आद्ये तदेव गकारोऽस्तु~।
वर्णनित्यतावादिभिरतिरिक्तगत्वानङ्गीकारात्~।
तथा चातिरिक्तस्फोटकल्पने एव गौरवम्~।
अन्त्ये- गकारादिप्रतीतिविरोधः~।
वायुसंयोगवृत्ति, ध्वनिवृत्ति वा वैजात्यमारोप्य तथा प्रत्यय इति चेन्न, प्रतीतेर्विना बाधकं भ्रमत्वायोगात्~।
अस्तु वा वायुसंयोग एव गकारोऽपि~।
तस्यातीन्द्रियत्वं दोष इति चेद्धर्मवदुपपत्तेरिति कृतं स्फोटेन~।
तस्मात् सन्त्येव वर्णाः, परन्तु न वाचकाः, गौरवात्~।
आकृत्यधिकरणन्यायेन जातेरेव वाच्यत्ववद्वाचकस्यापि युक्तत्वाच्च~।
इदं हरिपदमित्यनुगतप्रतीत्या हर्युपस्थितित्वावच्छेदेन हरिपदज्ञानत्वेन हेतुत्वात्तदवच्छेदकतया च जातिविशेषस्यावश्यकल्प्यत्वात्~।
नच वर्णानुपूर्व्यैव प्रतीत्यवच्छेदकत्वयोर्निर्वाहः, घटघटत्वादेरपि संयोगविशेषविशिष्टमृदाकारादिभिश्चान्यथासिद्ध्यापत्तेः~।
तस्मात् सा जातिरेव वाचिका, तादात्म्येनावच्छेदिका चेति~।
ननु सरो - रस इत्यादौ तयोर्जात्योः सत्त्वार्थबोधभेदो न स्यादित्यत आह- औपाधिको वेति~।
वा त्वर्थे उपाधिरानुपूर्वी, सैव जातिविशेषाभिव्यञ्जिकेति भेदः करणीभूतज्ञानस्येति नातिप्रसङ्ग इति भावः~।
उपाधिप्रयुक्तज्ञानवैलक्षण्ये दृष्टान्तमाह- वर्णानामिति॥७१॥
ननु जातेः प्रत्येकं वर्णोष्वपि सत्त्वात् प्रत्येकादर्थबोधापत्तिः स्यादित्यत आह-
\begin{center}{\bfseries अनेकव्यक्त्यभिव्यङ्गया जातिः स्फोट इति स्मृता~।\\
कैश्चत् व्यक्तय एवास्या ध्वनित्वेन प्रकल्पिताः॥७२॥}\end{center}
 अनेकाभिर्वर्णव्यक्तिभिरभिव्यक्तैव जातिः स्फोट इति स्मृता~।
योगार्थतया बोधिकेति यावत्~।
एतेन स्फोटस्य नित्यत्वात्सर्वदार्थबोधापत्तिरित्यपास्तम्~।
अयं भावः यद्यपि वर्णस्फोटपक्षे उक्तदोषोऽस्ति~।
तथापि पदवाक्यपक्षयोर्न, तत्र तस्या व्यासज्यवृत्तित्वस्य धर्मिग्राहकमानसिद्धत्वादिति~।
कैश्चित् व्यक्तयो ध्वनय एव, ध्वनिवर्णयोर्भेदाभावादित्यभ्युपेयन्ते इति शेषार्थः~।
उक्तं हि काव्यप्रकाशे-"बुधैर्वैयाकरणैः प्रधानीभूतस्फोटरूपव्यञ्जकस्य शब्दस्य ध्वनिरिति व्यवहारः कृतः" इति॥७२॥
 ननु का सा जातिस्तत्राह-
 \begin{center}{\bfseries सत्यासत्यौ तु यौ भागौ प्रतिभावं व्यवस्थितौ~।\\
सत्यं यत्तत्र सा जातिरसत्या व्यक्तयो मताः॥७३॥}\end{center}
प्रतिभावम् - प्रतिपदार्थम्~।
सत्यांशो जातिः, असत्या व्यक्तयः~।
तत्तद्व्यक्तिविशिष्टे ब्रह्मैव जातिरिति भावः~।
उक्तञ्च कैयटेन "असत्योपाध्यवच्छिन्नं ब्रह्मतत्वं द्रव्यशब्दवाच्यमित्यर्थः" इति~।
"ब्रह्मतत्त्वमेव शब्दस्वरूपतया भाति" इति च~।
कथं तर्हि, ब्रह्मदर्शने च गोत्वादिजातेरप्यसत्वादनित्यत्वम्, "आत्मैवेदं सर्वम्' इति श्रुतिवचनादिति कैयटः सङ्गच्छताम्~।
अविद्या - आविद्यको धर्मविशेषो वेति पक्षान्तरमादायेति द्रष्टव्यम्॥७३॥
तमेव सत्यांशे स्पष्टयति-
\begin{center}{\bfseries इत्थं निष्कृष्यमाणं यच्छब्दतत्त्वं निरञ्जनम्~।\\
ब्रह्मैवेत्यक्षरं प्राहुस्तस्मै पूर्णात्मने नमः॥७४॥}\end{center}
 अयमर्थः-"नामरूपे व्याकरवाणि" इति श्रुतिसिद्धा द्वयी सृष्टिः~।
तत्र रूपस्येव नाम्नोऽपि तदेव तत्त्वम्~।
प्रक्रियांशस्त्वविद्याविजृम्भणमात्रम्~।
उक्तञ्च वाक्यपदीये-
\begin{center}
शास्त्रेषु प्रक्रियाभेदैरविद्यैवोपवर्ण्यते~।\\
समारम्भस्तु भावानामनादि ब्रह्म शाश्वतम्॥इति॥\end{center}
ब्रह्मैवेत्यनेन "अत्रायं पुरुषः स्वयञ्ज्योतिः" "तमेव भान्तमनुभाति सर्वम्" " तस्य भासा सर्वमिदं विभाति" इति श्रुतिसिद्धस्वपरप्रकाशत्वं सूचयन् स्फुटत्यर्थोऽस्मादिति स्फोट इति यौगिकस्फोटशब्दाभिधेयत्वं सूचयति~।
निर्विघ्नप्रचारायान्ते मङ्गलं स्तुतिनतिरूपमाह - पूर्णात्मने इत्यादिना॥७४॥
\begin{center} अशेषफलदातारमपि सर्वेश्वरं गुरुम्~।\\
श्रीमद्भूषणसारेण भूषये शेषभूषणम्॥१॥\\[10pt]
 भट्टोजीदीक्षितैः श्रेष्ठैर्निर्मिताः कारिकाः शुभाः~।\\
कौण्डभट्टेन व्याख्याताः कारिकास्ताः सुविस्तरम् ॥२॥\end{center}
इति श्रीमत्पदवाक्यप्रमाणपारावारीणधुरीणरङ्गोजिभट्टाऽत्मजकौण्डभट्टकृते वैयाकरणभूषणसारे स्फोटवादः~।
समाप्तोऽयं ग्रन्थश्च॥
\section*{\begin{center} वैयाकरणभूषणसारकारिकाः\end{center}}\addcontentsline{toc}{section}{कारिकासङ्ग्रहः}
\fancyhead[LE,RO]{कारिकासङ्ग्रहः} \begin{center} फणिभाषितभाष्याब्धेः शब्दकौस्तुभ उद्धृतः~।\\
तत्र निर्णीत एवाऽर्थः सङ्क्षेपेणेह कथ्यते ॥१॥\\[10pt]
 फलव्यापारयोर्धातुराश्रये तु तिङः स्मृताः~।\\
 फले प्रधानं व्यापारस्तिङर्थस्तु विशेषणम्॥२॥\\[10pt]
 फलव्यापारयोस्तत्र फले तङ्यक्चिणादयः~।\\
 व्यापारे शप्श्नमाद्यास्तु द्योतयन्त्याश्रयाऽन्वयम्॥३॥\\[10pt]
 उत्सर्गोऽयं कर्मकर्तृविषयादौ विपर्य्ययात्~।\\
 तस्माद् यथोचितं ज्ञेयं द्योतकत्वं यथागमम्॥४॥\\[10pt]
 व्यापारो भावना सैवोत्पादना सैव च क्रिया~।\\
 कृञोऽकर्मकतापत्तेर्न हि यत्नोऽर्थ इष्यते॥५॥\\[10pt]
 किन्तूत्पादनमेवातः कर्मवत् स्याद् यगाद्यपि~।\\
 कर्मकर्तर्य्यन्यथा तु न भवेत् तद् दृशेरिव॥६॥\\[10pt]
 निर्वर्त्त्ये च विकार्य्ये च कर्मवद्भाव इष्यते~।\\
 न तु प्राप्ये कर्मणीति सिद्धान्तोऽत्र व्यवस्थितः॥७॥\\[10pt]
 तस्मात् करोतिर्धातोः स्याद् व्याख्यानं नत्वसौ तिङाम्~।\\
 पक्वान् कृतवान् पाकं किं कृतं पक्वमित्यपि॥८॥\\[10pt]
किं कार्य्यं पचनीयं चेत्यादि दृष्टं हि कृत्स्वपि~।\\
किञ्च क्रियावाचकतां विना धातुत्वमेव न॥९॥\\[10pt]
 सर्वनामाव्ययादीनां यावादीनां प्रसङ्गतः~।\\
 नहि तत्पाठमात्रेण युक्तमित्याकरे स्पुटम्॥१०॥\\[10pt]
 धात्वर्थत्वं क्रियात्वञ्चेद्धातुत्वं च क्रियार्थता~।\\
 अन्योऽन्यसंश्रयः स्पष्टस्तस्मादस्तु यथाऽऽकरम्॥११॥\\[10pt]
 अस्त्यावपि धर्म्यंशे भाव्येऽस्त्येव हि भावना~।\\
 अन्यत्राशेषभावात्तु सा तथा न प्रकाशते॥१२॥\\[10pt]
 फलव्यापारयोरेकनिष्ठतायामकर्मकः~।\\
 धातुस्तयोर्धर्म्मिभेदे सकर्मक उदाहृतः॥१३॥\\[10pt]
 आख्यातशब्दे भागाभ्यां साध्यसादनरूपता~।\\
 प्रकल्पिता यथा शास्त्रे सा घञादिष्वपि क्रमः॥१४॥\\[10pt]
 साध्यत्वेन क्रिया तत्र धातुरूपनिबन्धना~।\\
 सिद्धभावस्तु यस्तस्याः स घञादिनिबन्धनः॥१५॥\\[10pt]
 सम्बोधनान्तं कृत्वोऽर्थाः कारकं प्रथमो वतिः~।\\
 धातुसम्बन्धाधिकारनिष्पन्नमसमस्तनञ्॥१६॥\\[10pt]
 तथा यस्य च भावेन षष्ठी चेत्युदितं द्वयम्~।\\
 साधुत्वमष्टकस्यास्य क्रिययैवावधार्य्यताम्॥१७॥\\[10pt]
यदि पक्षेऽपि वत्यर्थः कारकञ्च नञादिषु~।\\
अन्वेति त्यज्यतां तर्हि चतुर्थ्याः स्पृहिकल्पना॥१८॥\\[10pt]
 अविग्रहा गतादिस्था यथा ग्रामादिकर्मभिः~।\\
 क्रिया सम्बन्ध्यते तद्वत् कृतपूर्व्यादिषु स्थिता॥१९॥\\[10pt]
 कृत्वोऽर्थाः क्तवातुमुन्वत्स्युरिति चेत् सन्ति हि क्वचित्~।\\
 अतिप्रसङ्गो नोद्भाव्योऽभिधानस्य समाश्रयात्॥२०॥\\[10pt]
 भेद्यभेदकसम्बन्धोपाधिभेदनिबन्धनम्~।\\
 साधुत्वं तदभावेऽपि बोधो नेह निवार्य्यते॥२१॥\\[10pt]
 वर्त्तमाने परोऽक्षे श्वो भाविन्यर्थे भविष्यति~।\\
 विध्यादौ प्रार्थनादौ च क्रमाज्ज्ञेया लडादयः॥२२॥\\[10pt]
 ह्यो भूते प्रेरणादौ च भृतमात्रे लङादयः~।\\
 सत्यां क्रियातिपत्तौ च भूते भाविनि लृङ् स्मृतः॥२३॥\\[10pt]
 आश्रयोऽवधिरुद्देश्यः सम्बन्धः शक्तिरेव वा~।\\
 यथायथं विभक्त्यर्थाः सुपां कर्मेति भाष्यतः॥२४॥\\[10pt]
 एकं द्विकं त्रिकं चाथ चतुष्कं पञ्चकं तथा~।\\
 नामार्थ इति सर्वेऽमी पक्षाः शास्त्रे निरूपिताः॥२५॥\\[10pt]
 शब्दोऽपि यदि भेदेन विवक्षा स्यात् तदा तथा~।\\
 नोचेच्छ्रोत्रादिभिः सिद्धोऽप्यसावर्थो व भासते ॥२६॥\\[10pt]
 अत एव गावित्याह भू सत्तायामितीदृशम्~।\\
 न प्रातिपदिकं नापि पदं साधु तु तत् स्मृतम्॥२७॥\\[10pt]
 सुपां सुपा तिङा नाम्ना धातुनाऽथ तिङां तिङा~।\\
 सुबन्तेनेति विज्ञायः समासः षड्विधो बुधैः॥२८॥\\[10pt]
 समासस्तु चतुर्द्धेति प्रायोवादस्तथा परः~।\\
 योऽयं पूर्वपदार्थादिप्राधान्यविषयः स च॥२९॥\\[10pt]
भौतपूर्व्यात् सोऽपि रेखागवयादिवदास्थितः~।\\
जहत्स्वार्थाजहत्स्वार्थे द्वे वृत्ती ते पुनस्त्रिधा॥३०॥\\[10pt]
 भेदः संसर्ग उभयं चेति वाच्यब्यवस्थितेः~।\\
समासे खलु भिन्नैव शक्तिः पङ्कजशब्दवात्॥३१॥\\[10pt]
 बहूनां वृत्तिधर्म्माणां वचनैरेव साधने~।\\
 स्यान्महद् गौरवं तस्मादेकार्थाभाव आश्रितः॥३२॥\\[10pt]
चकारादिनिषेधोऽथ बहुव्युत्त्पत्तिभञ्जनम्~।\\
कर्तव्यं ते न्यायसिद्धं त्वस्माकं तदिति स्थितिः॥३३॥\\
 अषष्ठयर्थबहुव्रीहौ व्युत्त्पत्त्यन्तरकल्पना~।\\
 क्लृप्तत्यागश्चास्ति तव तत् किं शक्तिं न कल्पयेः॥३४॥\\[10pt]
 आख्यातं तद्धितकृतोर्यत्किञ्चिदुपदर्शकम्~।\\
 गुणप्रधानभावादौ तत्र दृष्टो विपर्ययः॥३५॥\\[10pt]
 पर्यवस्यच्छाब्दबोधाविदूरप्राक्-क्षणस्थितेः~।\\
 शक्तिग्रहेऽन्तरङ्गत्वबहिरङ्गत्वचिन्तनम्॥३६॥\\[10pt]
 इन्द्रियाणां स्वविषयेष्वनादिर्योग्यता यथा~।\\
 अनादिरर्थैः शब्दानां सम्बन्धो योग्यता तथा॥३७॥\\[10pt]
 असाधुरनुमानेन वाचकः कैश्चिदिष्यते~।\\
 वाचकत्वाविशेषे वा नियमः पुण्यपापयोः॥३८॥\\[10pt]
 सम्बन्धिशब्दे सम्बन्धो योग्यतां प्रति योग्यता॥\\[10pt]
 समयाद् योग्यतासंविन्मातापुत्रादियोगवत्॥३९॥\\[10pt]
 नञ्समासे चापरस्य प्राधान्यात् सर्वनामता॥\\[10pt]
 आरोपितत्वं नञ्द्योत्यं न ह्यसोऽप्यतिसर्ववत्॥४०॥\\[10pt]
 अभावो वा तदर्थोऽस्तु भाष्यस्य हि तदाशयात्~।\\
 विशेषणं विशेष्यो वा न्यायतस्त्ववधार्य्यताम्॥४१॥\\[10pt]
 द्योतकाः प्रादयो येन निपाताश्चादयस्तथा~।\\
 `उपास्येते हरिहरौ' लकारो दृश्यते यथा॥४२॥\\[10pt]
 तथान्यत्र निपातेऽपि लकारः कर्मवाचकः~।\\
 विशेषणद्ययोगोऽपि प्रादिवच्चादिके समः॥४३॥\\[10pt]
 पदार्थः सदृशान्वेति विभागेन कदापि न~।\\
 निपातेतरसङ्कोचे प्रमाणं किं विभावय॥४४॥\\[10pt]
 शरैरुस्त्रैरिवोदीच्यानुद्धरिष्यन् रसानिव~।\\
 इत्यादावन्वयो न स्यात् सुपाञ्च श्रवणं ततः॥४५॥\\[10pt]
 नञ्समासे चापरस्य द्योत्यं प्रत्येव मुख्यता~।\\
 द्योत्यमेवार्थमादाय जायन्ते नामतः सुपः॥४६॥\\[10pt]
 निपातानां वाचकत्वमन्वयव्यतिरेकयोः~।\\
 युक्तं वा नतु तद्युक्तं परेषां मतमेव नः॥४७॥\\[10pt]
 निपातत्वं परेषां यत्तदस्माकमिति स्थितिः~।\\
 व्यापकत्वाच्छक्ततायास्त्ववच्छेदकमिष्यते॥४८॥\\[10pt]
 कृत्तद्धितसमासेभ्यो मतभेदनिबन्धन्म्~।\\
 त्वतलोरर्थकथनं टीकायां हरिणा कृतम्॥४९॥\\[10pt]
 अत्रार्द्धजरतीयं स्याद् दर्शनान्तरगामिनाम्~।\\
 सिद्धान्ते तु स्थितं पक्षद्वयं त्वादिषु तच्छृणु॥५०॥\\[10pt]
 प्रयोगोपाधिमाश्रित्य प्रकृत्यर्थप्रकारताम्~।\\
 धर्म्ममात्रं वाच्यमिति यद्वा शब्दपरा अमी॥५१॥\\[10pt]
 जायन्ते तज्जन्यबोधप्रकारे भावसंज्ञिते~।\\
 प्रत्ययार्थस्यैकदेशे प्रकृत्यर्थो विशेषणम्॥५२॥\\[10pt]
 अभेदश्चात्र संसर्ग आग्नेयादावियं स्थितिः~।\\
 देवतायां प्रदेये च खण्डशः शक्तिरस्तु वा॥५३॥\\[10pt]
 प्रदेय एव वा शक्तिः प्रकृतेर्वास्तु लक्षणा~।\\
 देवतायां निरूढेति सर्वे पक्षा अमी स्थिताः॥५४॥\\[10pt]
 क्रीडायां णस्तदस्यास्तीत्यादावेषैव दिक् समृता~।\\
 वस्तुतो वृत्तिरेवेति नात्रातीव प्रयत्यते॥५५॥\\[10pt]
 अभेदैकत्वसङ्खयाया वृत्तौ भानमिति स्थितिः~।\\
 कपिञ्जलालम्भवाक्ये त्रित्वं न्यायाद् यथोच्यते॥५६॥\\[10pt]
 लक्षयानुरोधात् सङ्खयायास्तन्त्रातन्त्रे मते यतः~।\\
 पश्वेकत्वादिहेतूनामाश्रयणमनाकरम्॥५७॥\\[10pt]
 विधेये भेदकं तन्त्रमन्यतो नियमो नहि~।\\
 ग्रहैकत्वादिहेतूनामाश्रयणमनाकरम्॥५८॥\\[10pt]
 रदाभ्यां वाक्यभेदेन नकारद्वयलाभतः~।\\
 क्षतिर्नैवास्ति तन्त्रत्वे विधेये भेदकस्य तु॥५९॥\\[10pt]
 अव्ययकृत इत्युक्तेः प्रकृत्यर्थे तुमादयः~।\\
 समानकर्तृकत्बादि द्योत्यमेषामिति स्थितिः॥६०॥\\[10pt]
 वाक्यस्फोटोऽतिनिष्कर्षे तिष्ठतीति मतस्थितिः~।\\
 साधुशब्देऽन्तर्गता हि बोधका नतु तत्स्मृताः॥६१॥\\[10pt]
 व्यवस्थितेर्व्यवहृतेस्तद्धेतुन्यायतस्तथा~।\\
 किञ्चाऽख्यातेन शत्राद्यैर्लडेव स्मार्यते यदि॥६२॥\\[10pt]
 कथं कर्तुरवाच्यत्ववाच्यत्वे तद्विभावय~।\\
 तरबाद्यन्ततिङ्क्ष्वस्ति नामता कृत्स्विव स्फुटा॥६३॥\\[10pt]
 नामार्थयोरभेदोऽपि तस्मात्तुल्योऽवधार्य्यताम्~।\\
 अथादेशा वाचकाश्चेत् पदस्फोटस्ततः स्फुटः॥६४॥\\[10pt]
 घटेनेत्यादिषु नहि प्रकृत्यादिभिदा स्थिता~।\\
 वस्नसादाविवेहापि सम्प्रमोहो हि दृश्यते॥६५॥\\[10pt]
 हरेऽवेत्यादि दृष्ट्वा च वाक्यस्फोटं विनिश्चिनु~।\\
 अर्थे विशिष्यसम्बन्धाग्रहणं चेत् समं पदे॥६६॥\\[10pt]
 लक्षणादधुना चेत्तत्पदेऽर्थेऽप्यस्तु तत् तथा~।\\
 सर्वत्रैव हिच वाक्यार्थो लक्षय एवेति ये विदुः ॥६७॥\\[10pt]
 भाट्टास्तेऽपीत्थमेवाहुर्लक्षणाया ग्रहे गतिम्~।\\
 पदे न वर्णा विद्यन्ते वर्णोष्ववयवा नच ॥६८॥\\[10pt]
 वाक्यात् पदानामत्यन्तं प्रविवेको न कश्चन~।\\
 पञ्चकोशादिवत्तस्मात् कल्पनैषा समाश्रिता ॥६९॥\\[10pt]
 उपेयप्रतिपत्त्यर्था उपाया अव्यवस्थिताः~।\\
 कल्पितानामुपाधित्वं स्वीकृतं हि परैरपि ॥७०॥\\[10pt]
 स्वरदैर्घ्यायाद्यपि ह्यन्ये वर्णोभ्योऽन्यस्य मन्वते~।\\
 शक्यत्व३ इव शक्तत्वे जातेर्लाघवमीक्षयताम् ॥७१॥\\[10pt]
 औपाधिको वा भेदोऽस्तु वर्णानां तारमन्दवात्~।\\
अनेकव्यक्तयभिव्यङ्गया जातिः स्फोट इति स्मृता ॥७२॥\\[10pt]
कैश्चत् व्यक्तय एवास्या ध्वनित्वेन प्रकल्पिताः~।\\
सत्यासत्यौ तु यौ भागौ प्रतिभावं व्यवस्थितौ ॥७३॥\\[10pt]
सत्यं यत्तत्र सा जातिरसत्या व्यक्तयो मताः~।\\
इत्थं निष्कृष्यमाणं यच्छब्दतत्त्वं निचज्जनम्~।\\
ब्रह्मैवेत्यक्षरं प्राहुस्तस्मै पूर्णात्मने नमः ॥७४॥\\[10pt]
इति भूषणसारस्थमूलकारिकासङ्कलनम्~।
\end{center}
