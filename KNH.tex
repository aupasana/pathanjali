\documentclass[twoside,top=1.7cm, bottom=1.7cm, outer=1cm,landscape, inner=1.5cm,a5paper,]{book}
\usepackage[top=1.7cm, bottom=1.7cm, outer=1cm, inner=1.5cm]{geometry}
\usepackage[T1]{fontenc}
\usepackage[compact]{titlesec}
\usepackage{polyglossia}
\setdefaultlanguage{sanskrit}
\setotherlanguage{english}
\pagestyle{empty}
\setmainfont[Script=Devanagari,AutoFakeBold=3.5,WordSpace=1.5]{Sanskrit2003}
\newfontfamily\englishfont{Times New Roman}
\usepackage{setspace}
\tolerance=1
\emergencystretch=\maxdimen
\hyphenpenalty=100
\hbadness=100
\hyphenchar\font=-1
\sloppy
\linespread{1.5}
\begin{document}
\LARGE
\frontmatter
\title{\Huge\bfseries ॥श्रीलक्ष्मीनारायणहृदयस्तोत्रम्॥}
\date{ }
	\begin{titlepage}
	\vfill
	\vfill
		\centering
		\maketitle
	\end{titlepage}

\begin{center}{\bfseries\LARGE ॥लक्ष्मीहृदयन्यासाः॥}\end{center}
{\bfseries देवीमातृकान्यासः~।}\\
अस्य श्रीकलामातृकान्यासस्य प्रजापतिर्ऋषिः~। गायत्री छन्दः~। शारदा देवता~। हलो बीजानि~। स्वराः शक्तयः~। अमुकमन्त्राङ्गतया न्यासे विनियोगः~।\\
{\bfseries ऋष्यादिन्यासः ~।}\\
ॐ प्रजापति-ऋषये नमः शिरसि~। ॐ गायत्रीच्छन्दसे नमः मुखे~। ॐ शारदादेवतायै नमः हृदि~। ॐ हल्बीजेभ्यो नमः गुह्ये~। ॐ स्वरशक्तिभ्यो नमः पादयोः~। ॐ विनियोगाय नमः सर्वाङ्गे~।
अं अां इं ईं -- ळं क्षं~।(अञ्जलिना सर्वाङ्गे विन्यसेत्~।)\\
{\bfseries करन्यासः षडङ्गन्यासश्च~।}\\ 
ॐ अं कं खं गं घं ङं आं  अङ्गुष्ठाभ्यां नमः। हृदयाय नमः~।\\
ॐ इं चं छं जं झं ञं ईँ     तर्जनीभ्यां नमः। शिरसे स्वाहा~।\\
ॐ उं टं ठं डं ढं णं ऊं    	मध्यमाभ्यां नमः। शिखायै वषट्‌~।\\
ॐ एं तं थं दं धं नं ऐं   	अनामिकाभ्यां नमः। कवचाय हुं~।\\
ॐ ओं पं फं बं भं मं औं 	कनिष्ठिकाभ्यां नमः। नेत्रत्रयाय वौषट्‌~।\\
ॐ अं यं रं लं वं शं षं सं हं ळं क्षं अः  करतलकरपृष्ठाभ्यां नमः। अस्त्राय फट्‌~।\\
{\bfseries ध्यानम् ~।}\\
शङ्खचक्राब्जपरशुकपालान्यक्षमालिकां~।\\
पुस्तकामृतकुम्भौ च त्रिशूलं दधतीं करैः ॥\\
सितपीतासितश्वेतरक्तवर्णैस्त्रिलोचनैः~।\\
पञ्चास्यैः संयुतां चन्द्रसकान्तिं शारदां भजे ॥\\[10pt]
{\bfseries बहिर्मातृकान्यासः~।}\\
ॐ अं निवृत्यै नमः शिरसि~।
ॐ आं प्रतिष्ठायै नमः मुखवृत्ते~।\\
ॐ इं विद्यायै नमः दक्षनेत्रे~।
ॐ ईं शान्त्यै नमः वामनेत्रे~।\\
ॐ उं इन्धिकायै नमः दक्षकर्णे~।
ॐ ऊं दीपिकायै नमः वामकर्णे~।\\
ॐ ऋं रेचिकायै नमः दक्षनासापुटे~।
ॐ ॠं मोचिकायै नमः वामनासापुटे~।\\
ॐ ऌं परायै नमः दक्षगण्डे~।
ॐ ऌृं सूक्ष्मायै नमः वामगण्डे~।\\
ॐ एं सूक्ष्मामृतायै नमः ओष्ठे~।
ॐ ऐं ज्ञानामृतायै नमः अधरे~।\\
ॐ ओं आप्यायन्यै नमः ऊर्ध्वदन्तपङ्क्तौ~।
ॐ औं व्यापिन्यै नमः अधोदन्तपङ्क्तौ~।\\
ॐ अं व्योमरुपायै नमः जिह्वाग्रे~।
ॐ अः अनन्तायै नमः कण्ठदेशे~।\\
ॐ कं सृष्ट्यै नमः दक्षबाहुमूले~।
ॐ खं ऋद्ध्यै नमः दक्षकूर्परे~।\\
ॐ गं स्मृत्यै नमः दक्षमणिबन्धे~।
ॐ घं मेधायै नमः दक्षहस्ताङ्गुलिमूले~।\\
ॐ ङं कान्त्यै नमः दक्षहस्ताङ्गुल्यग्रे~। 
ॐ चं लक्ष्म्यै नमः वामबाहुमूले~।\\
ॐ छं द्युत्यै नमः वामकूर्परे~।
ॐ जं स्थिरायै नमः वाममणिबन्धे~।\\
ॐ झं स्थित्यै नमः वामहस्ताङ्गुलिमूले~।
ॐ ञं सिद्ध्यै नमः वामहस्ताङ्गुल्यग्रे~।\\
ॐ टं जरायै नमः दक्षोरुमूले~।
ॐ ठं पालिन्यै नमः दक्षजानौ~।\\
ॐ डं क्षान्त्यै नमः दक्षपाङ्गुलिमूले~।
ॐ ढं ईश्वरिकायै नमः दक्षगुल्फे~।\\
ॐ णं रत्यै नमः दक्षपाङ्गुल्यग्रे~।
ॐ तं कामिकायै नमः  वामोरुमूले~।\\
ॐ थं वरदायै नमः वामजानुनि~।
ॐ दं आह्लादिन्यै नमः वामगुल्फे~।\\
ॐ धं प्रीत्यै नमः वामपादाङ्गुलिमूले~।
ॐ नं दीर्घायै नमः वामपादाङ्गुल्यग्रे~।\\
ॐ पं तीक्ष्णायै नमः दक्षपार्श्वे~।
ॐ फं रौद्र्यै नमः  वामपार्श्वे~।\\
ॐ बं भयायै नमः  पृष्ठे~।
ॐ भं निद्रायै नमः नाभौ~।\\
ॐ मं तन्द्रिकायै नमः उदरे ।
ॐ यं क्षुधायै नमः हृदि~।\\
ॐ रं क्रोधिन्यै नमः दक्षांसे~।  
ॐ लं क्रियायै नमः ककुदि~।\\
ॐ वं उत्कार्यै नमःवामांसे~।
ॐ शं समृत्युकायै नमः  हृदयादिदक्षहस्तान्तम्~।\\
ॐ षं पीतायै नमः हृदयादिवामहस्तान्तम्~।
ॐ सं श्वेतायै नमः हृदयादिदक्षपादन्तम्~।\\
ॐ हं अरुणायै नमः हृदयादिवामपादान्तम्।
ॐ ळं असितायै नमःहृदयादिपादान्तम् ~।\\
ॐ क्षं अनन्तायै नमः हृदयादिमस्तकान्तम् ~।\\[10pt]
{\bfseries अन्तर्मातृकान्यासः~।}\\
आधारे लिङ्गनाभिप्रकटितहृदये तालुमूले ललाटे~।\\
     द्वे पत्रे षोडशारे द्विदशदशदलद्वादशार्धे चतुष्के~।\\
वासान्ते बालमध्ये डफकठसहिते कण्ठदेशे स्वराणां\\
     हं क्षं तत्त्वार्थयुक्तं सकलदलगतं वर्णरूपं नमामि ॥\\[10pt]
(कण्ठस्थाने)ॐ श्रींह्रींऐं ऐंह्रींश्रीं अं नमः+अां नमः+---+अं नमः+अः नमः\\
(हृदयस्थाने)ॐ श्रींह्रींऐं ऐंह्रींश्रीं  कं नमः+खं नमः +---+टं नमः+ठं नमः  \\
(नाभिस्थाने)ॐ श्रींह्रींऐं ऐंह्रींश्रीं  डं नमः+ढं नमः -----+पं नमः+फं नमः \\
(स्वाधिष्ठाने)ॐ श्रींह्रींऐं ऐंह्रींश्रीं  बं नमः+भं नमः----+रं नमः+लं नमः \\
(मूलाधारे)ॐ श्रींह्रींऐं ऐंह्रींश्रीं  वं नमः+शं नमः+षं नमः+सं नमः \\
(भ्रूमध्ये)ॐ श्रींह्रींऐं ऐंह्रींश्रीं  हं नमः+ क्षं नमः\\
(सहस्रारे)ॐ श्रींह्रींऐं ऐंह्रींश्रीं अं नमः+अां नमः+---+ हं नमः+क्षं नमः~।\\
(प्राग्वदुत्तरन्यासं कुर्यात् ।)\\
{\bfseries करशुद्धिन्यासः~।}\\
ॐ अं नमः(दक्षकरतले) ॐ अां नमः(दक्षकरपृष्ठे) ॐ श्रीं नमः (दक्षकरतले)\\
ॐ अं नमः(वामकरतले) ॐ अां नमः(वामकरपृष्ठे) ॐ श्रीं नमः (वामकरतले)\\
ॐ अं नमः मध्यमाभ्यां नमः। ॐ अां नमः अनामिकाभ्यां नमः। ॐ श्रीं नमः कनिष्ठिकाभ्यां नमः~।
ॐ अं नमः अङ्गुष्ठाभ्यां नमः~। ॐ अां नमः तर्जनीभ्यां नमः~। ॐ श्रीं नमः करतलकरपृष्ठाभ्यां नमः।\\[10pt]
{\bfseries पादादिबीजन्यासः~।}\\
ॐ श्रीं ह्रीं ऐं महालक्ष्म्यै नमः अां ईं यं पं कं लं हं  पादयोः~।\\
ॐ श्रीं ह्रीं ऐं महालक्ष्म्यै नमः ह्रां ह्रीं अां व्यां यं भां सां मुखे~।\\
ॐ श्रीं ह्रीं ऐं महालक्ष्म्यै नमः घ्रां घ्रीं घ्रूं घ्रैं घ्रौं घ्रः नेत्रयोः~।\\
ॐ श्रीं ह्रीं ऐं महालक्ष्म्यै नमः ह्रां ह्रीं ह्रूं ह्रैं ह्रौं ह्रः जिह्वाग्रे~।\\
ॐ श्रीं ह्रीं ऐं महालक्ष्म्यै नमः अां ईं एं ऐं कुक्षौ~।\\
ॐ श्रीं ह्रीं ऐं महालक्ष्म्यै नमः अां क्रौं हुं फट् कुरु कुरु स्वाहा हृदये ~।\\
ॐ श्रीं ह्रीं ऐं महालक्ष्म्यै नमः श्रां श्रीं श्रूं श्रैं श्रौं श्रः कण्ठे~।\\
ॐ श्रीं ह्रीं ऐं महालक्ष्म्यै नमः यं हं कं लं  पं श्रीं शिरसि~।\\[10pt]
{\bfseries मन्त्रन्यासः ~।}\\
अस्य श्रीमहालक्ष्मीमहामन्त्रस्य भार्गव ऋषिः। अनुष्टुबादिनानाच्छन्दांसि। श्रीं बीजम्। ह्रीं शक्तिः। ऐं किलकम्। जपे विनियोगः।\\
ॐ श्रीं नमः शिरसि।ॐ ह्रीं नमः मूलाधारे।ॐ ऐं नमः हृदये।ॐ श्रीं नमः नेत्रयोः।ॐ ह्रीं नमः नासिके।ॐ ऐं नमः कर्णयोः।ॐ श्रीं नमः ओष्ठौ।ॐ ह्रीं नमः दन्तपङ्क्तौ।ॐ ऐं नमः जिह्वाग्रे।ॐ श्रीं नमः कण्ठे।ॐ ह्रीं नमः भुजयोः।ॐ ऐं नमः स्तनयोः।ॐ श्रीं नमः बाह्वोः।ॐ ह्रीं नमः ऊर्वोः।ॐ ऐं नमः कटौ।ॐ श्रीं नमः जानुनोः।ॐ ह्रीं नमः गुल्फयोः।ॐ ऐं नमः पादयोः।ॐ श्रीं नमः पृष्ठे।ॐ ह्रीं नमः दक्षिणपार्श्वे।ॐ ऐं नमः वामपपार्श्वे।ॐ श्रीं नमः मणिबन्धयोः।ॐ ह्रीं नमः हस्ताङ्गुलीषु।ॐ ऐं नमः पादाङ्गुलीषु।ॐ श्रीं नमः नाभौ।ॐ ह्रीं नमः हृदये।ॐ ऐं नमः शिरसि।
\newpage
\begin{center}{\bfseries\LARGE ॥लक्ष्मीनारायणहृदयन्यासाः॥}\end{center}
{\bfseries केशवादिमातृकान्यासः~।}\\
केशवादिमातृकान्यासस्य साध्यो नारायण ऋषिः~। अमृतगायत्रीच्छन्दः~। श्रीलक्ष्मीनारायणौ देवते~। हलो बीजानि~। स्वराः शक्तयः~। विष्णोः अमुकमन्त्रजपाङ्गतया न्यासे विनियोगः~।\\
साध्यनारायणऋषये नमः (शिरसि)। गायत्रीच्छन्दसे नमः (मुखे)~। लक्ष्मीनारायणदेवताभ्यां नमः (हृदये)। हल्बीजेभ्यो नमः (गुह्ये)। स्वरशक्तिभ्यो नमः (पादयोः)।
अं अां  इं ईं ---ळं क्षं~। (अञ्जलिना सर्वाङ्गे विन्यसेत्~।)\\
ॐ ह्रीं अङ्गुष्ठाभ्यां नमः-हृदयाय नमः~। 
ॐ श्रीं तर्जनीभ्यां नमः-शिरसे स्वाहा~।
ॐ क्लीं मध्यमाभ्यां नमः-शिखायै वषट्~।
ॐ ह्रीं  अनामिकाभ्यां नमः-कवचाय हुं~।
ॐ श्रीं कनिष्ठिकाभ्यां नमः-नेत्रत्रयाय वौषट्~।
ॐ क्लीं करतलकरपृष्ठाभ्यां नमः-अस्त्राय फट्~।\\[10pt]
{\bfseries ध्यानम्।}\\
शङ्खचक्रगदापद्मकुम्भादर्शाब्जपुस्तकम्~।\\
बिभ्रन्तं मेघचपलावर्णं लक्ष्मीहरिं भजे ॥(पञ्चोपचारपूजा~।)\\[10pt]
{\bfseries बहिर्मातृकान्यासः~।}\\
ॐ ह्रीं श्रीं क्लीं अं क्लीं श्रीं ह्रीं केशवकीर्तिभ्यां नमः शिरसि~।\\
ॐ ह्रीं श्रीं क्लीं आं क्लीं श्रीं ह्रीं नारायणकान्तिभ्यां नमः, मुखवृत्ते~।\\
ॐ ह्रीं श्रीं क्लीं इं क्लीं श्रीं ह्रीं माधवतुष्टिभ्यां नमः, दक्षनेत्रे~।\\
ॐ ह्रीं श्रीं क्लीं ईं क्लीं श्रीं ह्रीं गोविन्दपुष्टिभ्यां नमः, वामनेत्रे~।\\
ॐ ह्रीं श्रीं क्लीं उं क्लीं श्रीं ह्रीं विष्णुधृतिभ्यां नमः, दक्षकर्णे~।\\
ॐ ह्रीं श्रीं क्लीं ऊं क्लीं श्रीं ह्रीं मधुसूदनशान्तिभ्यां नमः, वामकर्णे~।\\
ॐ ह्रीं श्रीं क्लीं ऋ क्लीं श्रीं ह्रीं त्रिविक्रमक्रियाभ्यां नमः, दक्षनासापुटे~।\\
ॐ ह्रीं श्रीं क्लीं ॠं क्लीं श्रीं ह्रीं वामनदयाभ्यां नमः, वामनासापुटे~।\\
ॐ ह्रीं श्रीं क्लीं ऌं क्लीं श्रीं ह्रीं श्रीधरमेधाभ्यां नमः, दक्षगण्डे\\
ॐ ह्रीं श्रीं क्लीं ऌृं क्लीं श्रीं ह्रीं हृषीकेशहर्षाभ्यां नमः, वामगण्डे~।\\
ॐ ह्रीं श्रीं क्लीं एं क्लीं श्रीं ह्रीं पद्मनाभश्रद्धाभ्यां नमः, ऊर्ध्वोष्ठे~।\\
ॐ ह्रीं श्रीं क्लीं ऐं क्लीं श्रीं ह्रीं दामोदरलज्जाभ्यां नमः, अधरोष्ठे~।\\
ॐ ह्रीं श्रीं क्लीं ओं क्लीं श्रीं ह्रीं वासुदेवलक्ष्मीभ्यां नमः, ऊर्ध्वदन्तपङ्क्तौ~।\\
ॐ ह्रीं श्रीं क्लीं औं क्लीं श्रीं ह्रीं सङ्कर्षणसरस्वतीभ्यां नमः, अधोदन्तपङ्क्तौ~।\\
ॐ ह्रीं श्रीं क्लीं अं क्लीं श्रीं ह्रीं प्रद्युम्नप्रीतिभ्यां नमः, जिह्वाग्रे~।\\
ॐ ह्रीं श्रीं क्लीं अः क्लीं श्रीं ह्रीं अनिरुद्धरतिभ्यां नमः, कण्ठे~।\\
ॐ ह्रीं श्रीं क्लीं कं क्ली श्रीं ह्रीं चक्रिजयाभ्यां नमः, दक्षबाहुमूले~।\\
ॐ ह्रीं श्रीं क्लीं खं क्लीं श्रीं ह्रीं गदिदुर्गाभ्यां नमः, दक्षकूर्परे~।\\
ॐ ह्रीं श्रीं क्लीं गं क्ली श्रीं ह्रीं शार्ङ्गिप्रभाभ्यां नमः, दक्षमणिबन्धे~।\\
ॐ ह्रीं श्रीं क्लीं घं क्लीं श्रीं ह्रीं खड्गिसत्याभ्यां नमः, दक्षाङ्गुलीमूले\\
ॐ ह्रीं श्रीं क्लीं ङं क्लीं श्रीं ह्रीं शङ्खिचण्डाभ्यां नमः, दक्षाङ्गुल्यग्रे~।\\
ॐ ह्रीं श्रीं क्लीं चं क्लीं श्रीं ह्रीं हलिवाणीभ्यां नमः, वामबाहुमूले~।\\
ॐ ह्रीं श्रीं क्लीं छं क्लीं श्रीं ह्रीं मुसलिविलासिनीभ्यां नमः, वामकूर्परे~।\\
ॐ ह्रीं श्रीं क्लीं जं क्लीं श्रीं ह्रीं शूलिविजयाभ्यां नमः, वाममणिबन्धे~।\\
ॐ ह्रीं श्रीं क्लीं झं क्लीं श्रीं ह्रीं पाशिविरजाभ्यां नमः, वामाङ्गुलीमूले~।\\
ॐ ह्रीं श्रीं क्लीं ञं क्लीं श्रीं ह्रीं अङ्कुशिविश्वाभ्यां नमः, वामाङ्गुल्यग्रे~।\\
ॐ ह्रीं श्रीं क्लीं टं क्लीं श्रीं ह्रीं मुकुन्दविनदाभ्यां नमः, दक्षोरुमूले~।\\
ॐ ह्रीं श्रीं क्लीं ठं क्लीं श्रीं ह्रीं नन्दजसुनदाभ्यां नमः, दक्षजानुनि~।\\
ॐ ह्रीं श्रीं क्लीं डं क्लीं श्रीं ह्रीं नन्दिसत्याभ्यां नमः, दक्षगुल्फे~।\\
ॐ ह्रीं श्रीं क्लीं ढं क्लीं श्रीं ह्रीं नरर्द्धिभ्यां नमः, दक्षपादाङ्गुलीमूले~।\\
ॐ ह्रीं श्रीं क्लीं णं क्लीं श्रीं ह्रीं नरकजित्समृद्धिभ्यां नमः दक्षपादाङ्गुल्यग्रे~।\\
ॐ ह्रीं श्रीं क्लीं तं क्लीं श्रीं ह्रीं हरिशुद्धिभ्यां नमः वामोरुमूले~।\\
ॐ ह्रीं श्रीं क्लीं थं क्लीं श्रीं ह्रीं कृष्णबुद्धिभ्यां नमः, वामजानुनि~।\\
ॐ ह्रीं श्रीं क्लीं दं क्लीं श्रीं ह्रीं सत्यभुक्तिभ्यां नमः, वामगुल्फे~।\\
ॐ ह्रीं श्रीं क्लीं धं क्लीं श्रीं ह्रीं सात्वतमतिभ्यां नमः, वामपादाङ्गुलीमूले~।\\
ॐ ह्रीं श्रीं क्लीं नं क्लीं श्रीं ह्रीं सौरिक्षमाभ्यां नमः, वामपादाङ्गुल्यग्रे~।\\
ॐ ह्रीं श्रीं क्लीं पं क्लीं श्रीं ह्रीं शूररमाभ्यां नमः, दक्षपार्श्वे~।\\
ॐ ह्रीं श्रीं क्लीं फं क्लीं श्रीं ह्रीं जनार्दनोमाभ्यां नमः, वामपार्श्वे~।\\
ॐ ह्रीं श्रीं क्लीं बं क्लीं श्रीं ह्रीं भूधरक्लेदिनीभ्यां नमः, पृष्ठे~।\\
ॐ ह्रीं श्रीं क्लीं भं क्लीं श्रीं ह्रीं विश्वमूर्तिक्लिन्नाभ्यां नमः, नाभौ~।\\
ॐ ह्रीं श्रीं क्लीं मं क्लीं श्रीं ह्रीं वैकुण्ठवसुधाभ्यां नमः, जठरे~।\\
ॐ ह्रीं श्रीं क्लीं यं क्लीं श्रीं ह्रीं त्वगात्मभ्यां पुरुषोत्तमवसुदाभ्यां नमः, हृदि~।\\
ॐ ह्रीं श्रीं क्लीं रं क्लीं श्रीं ह्रीं असृगात्मभ्यां बलिपराभ्यां नमः, दक्षांसे~।\\
ॐ ह्रीं श्रीं क्लीं लं क्लीं श्रीं ह्रीं मांसात्मभ्यां बलानुजपरायणाभ्यां नमः, कुकुदि~।\\
ॐ ह्रीं श्रीं क्लीं वं क्लीं श्रीं ह्रीं मेदअात्मभ्यां बालसूक्ष्माभ्यां नमः, वामांसे~।\\
ॐ ह्रीं श्रीं क्लीं शं क्लीं श्रीं ह्रीं अस्थ्यात्मभ्यां वृषघ्नसन्ध्याभ्यां नमः, हृदयादिदक्षकरान्तम्~।\\
ॐ ह्रीं श्रीं क्लीं षं क्लीं श्रीं ह्रीं मज्जात्मभ्यां वृषप्रज्ञाभ्यां नमः, हृदयादिवामकरान्तम्~।\\
ॐ ह्रीं श्रीं क्लीं सं क्लीं श्रीं ह्रीं शुक्रात्मभ्यां हंसप्रभाभ्यां नमः, हृदयादिदक्षपादान्तम्~।\\
ॐ ह्रीं श्रीं क्लीं हं क्लीं श्रीं ह्रीं प्राणात्मभ्यां वराहनिशाभ्यां नमः, हृदयादिवामपादान्तम्~।\\
ॐ ह्रीं श्रीं क्लीं ळं क्लीं श्रीं ह्रीं शक्त्यात्मभ्यां विमलमेघाभ्यां नमः, हृदयादिउदरान्तम्~।\\
ॐ ह्रीं श्रीं क्लीं क्षं क्लीं श्रीं ह्रीं क्रोधात्मभ्यां नृसिंहविद्युद्भ्यां नमः, हृदयादिमूर्धपर्यन्तम् ~।\\[10pt]
{\bfseries अन्तर्मातृकान्यासः~।}\\
आधारे लिङ्गनाभौ प्रकटितहृदये तालुमूले ललाटे\\
     द्वे पत्रे षोडशारे द्विदशदशदले द्वादशार्धे चतुष्के~।\\
वासान्ते बालमध्ये डफकठसहिते कण्ठदेशे स्वराणां\\
     हं क्षं तत्त्वार्थयुक्तं सकलदलगतं वर्णरूपं नमामि ॥\\
(कण्ठस्थाने)ॐ ह्रींश्रींक्लीं क्लींश्रींह्रीं अं नमः+अां नमः+--+अं नमः+अः नमः।\\
(हृदयस्थाने)ॐ ह्रींश्रींक्लीं क्लींश्रींह्रीं कं नमः+खं नमः +----+टं नमः+ठं नमः।\\
(नाभिस्थाने)ॐ ह्रींश्रींक्लीं क्लींश्रींह्रीं डं नमः+ढं नमः ------+पं नमः+फं नमः। \\
(स्वाधिष्ठाने)ॐ ह्रींश्रींक्लीं क्लींश्रींह्रीं बं नमः+भं नमः--------+ रं नमः+लं नमः। \\
(मूलाधारे)ॐह्रींश्रींक्लीं क्लींश्रींह्रीं वं नमः+शं नमः+षं नमः+सं नमः।\\
(भ्रूमध्ये)ॐह्रींश्रींक्लीं क्लींश्रींह्रीं हं नमः+क्षं नमः।\\
(सहस्रारे)ॐह्रींश्रींक्लीं क्लींश्रींह्रीं अं नमः+--+क्षं नमः॥(प्राग्वदुत्तरन्यासं कुर्यात् ।)\\
{\bfseries अष्टाक्षरमन्त्रन्यासः~।}\\
अस्य श्री अष्टाक्षरीमहामन्त्रस्य साध्यनारायण ऋषिः~। गायत्रीच्छन्दः~। महाविष्णुर्देवता~। ॐ इति बीजम्~। नम इति शक्तिः। नरायणायेति कीलकम्~।\\
{\bfseries ऋष्यादिन्यासः~।}\\
साध्यनारायणऋषये नमः (शिरसि)। गायत्रीच्छन्दसे नमः (मुखे)~। महाविष्णुदेवतायै नमः (हृदये)। ॐबीजाय नमः (गुह्ये)। नमश्शक्तये नमः(पादयोः)।
नरायणायकीलकाय नमः॥\\
{\bfseries अङ्गन्यासकरन्यासौ।}\\
ॐ क्रुद्धोल्काय स्वाहा अङ्गुष्ठाभ्यां नमः-हृदयाय नमः~। 
ॐ महोल्काय स्वाहा तर्जनीभ्यां नमः-शिरसे स्वाहा~।
ॐ वीरोल्काय स्वाहा मध्यमाभ्यां नमः-शिखायै वषट्~।
ॐ द्व्युल्काय स्वाहा अनामिकाभ्यां नमः-कवचाय हुं~।
ॐ ज्ञानोल्काय स्वाहा कनिष्ठिकाभ्यां नमः-नेत्रत्रयाय वौषट्~।
ॐ सहस्रोल्काय स्वाहा करतलकरपृष्ठाभ्यां नमः-अस्त्राय फट्~।\\[10pt]
{\bfseries मन्त्रवर्णैः षडङ्गन्यासः।}\\
ॐ ॐ नमः हृदयाय नमः~। ॐ नं नमः शिरसे स्वाहा~। ॐ मों नमः शिखायै वषट्~। ॐ नां नमः कवचाय हुं~। ॐ रां नमः नेत्रत्रयाय वौषट्~। ॐ यं नमः अस्त्राय फट्~। ॐ णां नमः कुक्ष्योः~। ॐ यं नमः पृष्ठे ॥\\
ॐ नमः सुदर्शनायास्त्राय फट्~। इति दिग्बन्धः ॥\\[10pt]
{\bfseries मन्त्रवर्णैः अष्टाङ्गन्यासः।}\\
ॐ ॐ नमः अाधारे~। ॐ नं नमः हृदि~। ॐ मों नमः वक्त्रे~। ॐ नां नमः दक्षिणभुजे~। ॐ रां नमः वामभुजे~। ॐ यं नमः दक्षिणपादे~। ॐ णां नमः वामपादे~। ॐ यं नमः नाभौ ॥१॥\\
ॐ ॐ नमः कण्ठे~। ॐ नं नमः नाभौ~। ॐ मों नमः हृदि~। ॐ नां नमः दक्षिणस्तने~। ॐ रां नमः वामस्तने~। ॐ यं नमः दक्षिणपार्श्वे~। ॐ णां नमः वामपार्श्वेे~। ॐ यं नमः पृष्ठे ॥२॥\\[10pt]
ॐ ॐ नमः मूर्ध्नि~। ॐ नं नमः मुखे~। ॐ मों नमः दक्षनेत्रे~। ॐ नां नमः वामनेत्रे~। ॐ रां नमः दक्षिणकर्णे~। ॐ यं नमः वामकर्णे~। ॐ णां नमः दक्षिणनासापुटे~। ॐ यं नमः वामनासापुटे ॥३॥\\[10pt]
ॐ ॐ नमः दक्षबाहुमूले~। ॐ नं नमः दक्षकूर्परे~। ॐ मों नमः दक्षमणिबन्धे~। ॐ नां नमः दक्षहस्ताङ्गुलीमूले~। ॐ रां नमः दक्षहस्ताङ्गुल्यग्रे~। ॐ यं नमः वामबाहुमूले~। ॐ णां नमः वामकूर्परेे~। ॐ यं नमः वाममणिबन्धे ॥४॥\\[10pt]
ॐ ॐ नमः वामहस्ताङ्गुलीमूले~। ॐ नं नमः वामहस्ताङ्गुल्यग्रे~। ॐ मों नमः दक्षोरुमूले~। ॐ नां नमः दक्षिणजानुनि~। ॐ रां नमः दक्षिणगुल्फे~। ॐ यं नमः दक्षिणपादाङ्गुलीमूले~। ॐ णां नमः दक्षिणपादाङ्गुल्यग्रेे~। ॐ यं नमः वामोरुमूले ॥५॥\\[10pt]
ॐ ॐ नमः वामजानुनि~। ॐ नं नमः वामगुल्फे~। ॐ मों नमः वामपादाङ्गुलीमूले~। ॐ नां नमः वामपादाङ्गुल्यग्रे~।(हृदये करं दत्वा) ॐ रां नमः त्वचि~। ॐ यं नमः रक्ते~। ॐ णां नमः मांसे~। ॐ यं नमः  मेदसि ॥६॥\\[10pt]
ॐ ॐ नमः अस्थ्नि~। ॐ नं नमः मज्जायाम्~। ॐ मों नमः शुक्रे~। ॐ नां नमः प्राणे~। ॐ रां नमः हृदि~। ॐ यं नमः दक्षिणगले~। ॐ णां नमः वामगले~। ॐ यं नमः  हृदि ॥७॥\\[10pt]
ॐ ॐ नमः मूर्ध्नि~। ॐ नं नमः नेत्रयोः। ॐ मों नमः मुखे~। ॐ नां नमः हृदि~। ॐ रां नमः कुक्षौ~। ॐ यं नमः ऊर्वोः~। ॐ णां नमः जङ्घयोः~। ॐ यं नमः पादयोः ॥८॥(इति मन्त्रवर्णाष्टन्यासाः ॥)\\[10pt]
{\bfseries अायुधन्यासः~।}\\ॐ चक्राय नमः दक्षिणगण्डे~। ॐ शङ्खाय नमः वामगण्डे~। ॐ गदायै नमः दक्षिणांसे~। ॐ पद्माय नमः वामांसे ॥ \\[10pt]
{\bfseries मूर्तिपञ्जरन्यासः ।}\\
ॐ ॐ अं धातृसहित-केशवाय नमः (ललाटे)\\
ॐ नं  अां अर्यमसहित-नारायणाय नमः (कुक्षौ)\\
ॐ मों इं मित्रसहित-माधवाय नमः (हृदि)\\
ॐ भं ईं वरुणसहित-गोविन्दाय नमः (कण्ठे)\\
ॐ गं उं अंशुसहित-विष्णवे नमः (दक्षपार्श्वे)\\
ॐ वं ऊं भगसहित-मधुसूदनाय नमः (दक्षांसे)\\
ॐ तें एं विवस्वत्सहित-त्रिविक्रमाय नमः (गलदक्षिणभागे)\\
ॐ वां ऐं इन्द्रसहित- वामनाय नमः (वामपार्श्वे)\\
ॐ सुं ओं पूषसहित-श्रीधराय नमः (वामांसे )\\
ॐ दें औं पर्जन्यसहित- हृषीकेशाय नमः (गलवामभागे)\\
ॐ वां अं त्वष्टृसहित-पद्मनाभाय नमः (पृष्ठे)\\
ॐ यं अः विष्णुसहित- दामोदराय नमः (ककुदि )\\
ॐ नमो भगवते वासुदेवाय  (मूर्ध्नि) (इति द्वादशमूर्तिपञ्जरन्यासः)\\
{\bfseries किरीटमन्त्रन्यासः॥}\\
ॐ किरीटकेयूरहारमकरकुण्डलधरशङ्खचक्रगदाम्भोजपीताम्बरधरश्रीवत्साङ्कितवक्षःस्थल-\\श्रीभूमीसहितात्मज्योतिर्द्वयदीप्तकराय सहस्रादित्यतेजसे नमः ॥ (इति मन्त्रेण शिरअादिपादान्तं व्यापकं विन्यसेत् ॥)\\
उद्यत्कोटिदिवाकराभमनिशं शङ्खं गदां पङ्कजं\\
 चक्रं बिभ्रतमिन्दिरावसुमतीसंशोभिपार्श्वद्वयम्~।\\
कोटीराङ्गदहारकुण्डलधरं पीताम्बरं कौस्तुभो-\\
द्दीप्तं विश्वधरं स्ववक्षसि लसच्छ्रीवत्सचिह्नं भजे ॥\\[10pt]
{\bfseries द्वादशाक्षरविष्णुमन्त्रन्यासः॥}\\
अस्य श्रीद्वादशाक्षरविष्णुमन्त्रस्य प्रजापतिर्ऋषिः~। गायत्री छन्दः~। वासुदेवपरमात्मा देवता~। जपे विनियोगः ॥\\
{\bfseries ऋष्यादिन्यासः~।}\\
प्रजापति-ऋषये नमः (शिरसि)। गायत्रीच्छन्दसे नमः (मुखे)। वासुदेवपरमात्मदेवतायै नमः(हृदये)। विनियोगाय नमः (सर्वाङ्गे)। \\[10pt]
{\bfseries करन्यास-पञ्चाङ्गन्यासौ ॥}\\
ॐ ॐ अङ्गुष्ठाभ्यां नमः~। हृदयाय नमः~।\\
ॐ नमः	तर्जनीभ्यां नमः~। शिरसे स्वाहा~।\\
ॐ भगवते मध्यमाभ्यां नमः~। शिखायै वषट्~।\\
ॐ वासुदेवाय अनामिकाभ्यां नमः~। कवचाय हुं~।\\
ॐ ॐ नमो भगवते वासुदेवाय कनिष्ठिकाभ्यां नमः~। अस्त्राय फट्~।\\
{\bfseries मन्त्रवर्णन्यासः॥}\\
ॐ ॐ नमः मूर्ध्नि~। ॐ नं  नमः भाले~। ॐ मों नमः नेत्रयोः~। 
ॐ भं नमः मुखे~। ॐ गं नमः गले~। ॐ वं बाह्वोः~। 
ॐ तें नमः हृदये~। ॐ वां नमः कुक्षौ~। ॐ सुं नमः नाभौ~। ॐ दें नमः लिङ्गे~। ॐ वां नमः जान्वोः~। ॐ यं नमः पादयोः ॥
 इति विन्यस्य ध्यायेत् -\\
क्षीरोदन्वत्प्रदेशे शुचिमणिविलसत्सैकते मौक्तिकानां\\
     मालाकॢप्तासनस्थः स्फटिकमणिनिभैर्मौक्तिकैर्मण्डिताङ्गः~।\\
शुभ्रैरभ्रैरदभ्रैरुपरिविरचितैर्मुक्तपीयूषवर्षैः\\
     आनन्दी नः पुनीयादरिनळिनगदाशङ्खपाणिर्मुकुन्दः ॥\\[10pt]
भूः पादौ यस्य नाभिर्वियदसुरनिलश्चन्द्रसूर्यौ च नेत्रे\\
     कर्णावाशाश्शिरो द्यौर्मुखमपि दहनो यस्य वास्तेयमब्धिः~।\\[10pt]
अन्तस्थं यस्य विश्वं सुरनरखगगोभोगिगन्धर्वदैत्यैः\\
     चित्रं रंरम्यते तं त्रिभुवनवपुषं विष्णुमीशं नमामि ॥\\
शान्ताकारं भुजगशयनं पद्मनाभं सुरेशं\\
     विश्वाधारं गगनसदृशं मेघवर्णं शुभाङ्गम्~।\\
लक्ष्मीकान्तं कमलनयनं योगिहृद्ध्यानगम्यं\\
     वन्दे विष्णुं भवभयहरं सर्वलोकैकनाथम् ॥\\
मेघश्यामं पीतकौशेयवासं श्रीवत्साङ्कं कौस्तुभोद्भासिताङ्गम्~।\\
पुण्योपेतं पुण्डरीकायताक्षं विष्णुं वन्दे सर्वलोकैकनाथम् ॥\\
सशङ्खचक्रं सकिरीटकुण्डलं सपीतवस्त्रं सरसीरुहेक्षणम्~।\\
सहारवक्षःस्थलशोभिकौस्तुभं नमामि विष्णुं शिरसा चतुर्भुजम् ॥\\[10pt] 
विष्णुं शारदचन्द्रकोटिसदृशं शङ्खं रथाङ्गं गदा-\\
मम्भोजं दधतं सिताब्जनिलयं कान्त्या जगन्मोहनम्~।\\
अाबद्धाङ्गदहारकुण्डलमहामौलिं स्फुरत्कङ्कणं\\
श्रीवत्साङ्कमुदारकौस्तुभधरं वन्दे मुनीन्द्रैः स्तुतम् ॥\\[10pt]
{\bfseries अात्मरक्षान्यासः ॥}\\
ॐ ऐं ह्रीं श्रीं लक्ष्मीनारायण मम अात्मानं रक्ष रक्ष ॥(अञ्जलिं हृदि समर्पयेत्~।)\\
{\bfseries अथ सूक्तन्यासः ॥}\\
सहस्रशीर्षा पुरुषः----------दशाङ्गुलम्॥(शिरसि)\\
पुरुष एवेदम् ------------ना तिरोहति॥(नेत्रयोः)\\
एतावानस्य -------- त्रिपादस्यामृतं दिवि॥(कर्णयोः)\\
त्रिपादूर्द्ध्व------------------शने अभि॥(नासिकयोः)\\
तस्मात् विराळ---------पश्चात् भूमिमथो पुरः॥(मुखे)\\
यत् पुरुषेण-------------- इद्ध्मः शरद्धविः॥(गले)\\
तं यज्ञं ----------------- ऋषयश्च ये॥(बाह्वोः)\\
तस्माद्यज्ञा-----------------ग्राम्याश्च ये॥(हृदि)\\
तस्माद्यज्ञा---------------तस्मादजायत॥(नाभौ)\\
तस्मादश्वा--------------जाता अजावयः॥(गुह्ये)\\
यत्पुरुषं व्यदधुः----------- पादा उच्येते॥(गुदे)\\
ब्राह्मणोऽस्य------------- शूद्रो अजायत॥(ऊर्वोः)\\
चन्द्रमा मनसो--------- प्राणाद्वायुरजायत॥(जानुनोः)\\
नाभ्या ----------------लोकाँ अकल्पयन्॥(जङ्घयोः)\\
सप्तास्यासन्------------- अबध्नन्पुरुषं पशुं॥(पादयोः)\\
यज्ञेन यज्ञमय-------पूर्वे साध्याः सन्ति देवा:॥(सर्वाङ्गे)
\newpage
\begin{center}{\bfseries\Huge ॥ श्रीमहालक्ष्मीहृदयस्तोत्रम् ॥}\end{center}
अस्य श्रीमहालक्ष्मीहृदयस्तोत्रमहामन्त्रस्य भार्गव ऋषिः~। अनुष्टुबादिनानाछन्दांसि~। आद्यादिश्रीमहालक्ष्मीर्देवता~। श्रीं बीजम्~। ह्रीं शक्तिः~। ऐं कीलकम्~। श्रीमहालक्ष्मीप्रसादसिद्ध्यर्थे जपे विनियोगः ॥\\
भार्गव ऋषये नमः~।अनुष्टुबादिनानाछन्दोभ्यो नमः~। आद्यादिश्रीमहालक्ष्मीदेवतायै नमः~। श्रींबीजाय नमः। ह्रींशक्तये नमः~। ऐंकीलकाय नमः~।\\
ॐ श्रां अङ्गुष्ठाभ्यां नमः~। हृदयाय नमः~।\\
ॐ श्रीं तर्जनीभ्यां नमः~। शिरसे स्वाहा~।\\
ॐ श्रूं मध्यमाभ्यां नमः~। शिखायै वषट्~।\\
ॐ श्रैं अनामिकाभ्यां नमः~। कवचाय हुं~।\\
ॐ श्रौं कनिष्ठिकाभ्यां नमः~। नेत्रत्रयाय वौषट्~।\\
ॐ श्रः करतलकरपृष्ठाभ्यां नमः~। अस्त्राय फट्~।\\
ध्यानफलम् ॥
\begin{center}पीतवस्त्रां सुवर्णाङ्गीं पद्महस्तां गदान्विताम्~।\\
लक्ष्मीं ध्यायेत मन्त्रेण स भवेत् पृथिवीपतिः ॥\\[10pt]
॥ध्यानम्॥\\
मातुलुङ्गं गदां खेटं पानपात्रञ्च बिभ्रतीम्~।\\
नागं लिङ्गञ्च योनिञ्च बिभ्रतीं चैव मूर्धनि ॥\\[10pt]
विष्णुस्तुतिपरां लक्ष्मीं  स्वर्णवर्णां स्तुतिप्रियाम्~।\\
वराभयप्रदां देवीं वन्दे तां कमलेक्षणाम् ॥\\[10pt]
कौशेयपीतवसनां अरविन्दनेत्रां पद्मद्वयाभयवरोद्यतपद्महस्ताम्~।\\
उद्यच्छतार्कसदृशीं परमाङ्कसंस्थां ध्यायेद्विधीशनतपादयुगां जनित्रीम् ॥\\[10pt]
या सा पद्मासनस्था विपुलकटितटी पद्मपत्रायताक्षी\\
गम्भीरावर्तनाभिः स्तनभरनमिता शुभ्रवस्त्रोत्तरीया~।\\[10pt]
लक्ष्मीर्दिव्यैर्गजेन्द्रैर्मणिगणखचितैः स्नापिता हेमकुम्भैः\\
नित्यं सा पद्महस्ता मम वसतु गृहे सर्वमांगल्ययुक्ता ॥\\
\newpage
हस्तद्वयेन कमले धारयन्तीं स्वलीलया~।\\
हारनूपुरसंयुक्तां लक्ष्मीं देवीं विचिन्तये ॥(इति पञ्चोपचारैः सम्पूज्य~।)\\[10pt]
शङ्कचक्रगदाहस्ते शुभ्रवर्णे शुभानने~।\\
मम देहि वरं लक्ष्मीः सर्वसिद्धिप्रदायिनी ॥(इति सम्प्रार्थ्य ॥)\\[10pt]
{\bfseries ॐ श्रीं ह्रीं ऐं महालक्ष्म्यै कमलधारिण्यै सिंहवाहिन्यै स्वाहा ॥}(१०८)

वन्दे लक्ष्मीं प्रहसितमुखीं शुद्धजाम्बूनदाभां\\
तेजोरूपां कनकवसनां सर्वभूषोज्ज्वलाङ्गीम्~।\\
बीजापूरं कनककलशं हेमपद्मे दधानाम्\\
अाद्यां शक्तिं सकलजननीं विष्णुमाङ्कसंस्थाम् ॥१॥\\[10pt]
श्रीमत्सौभाग्यजननीं स्तौमि लक्ष्मीं सनातनीम्~।\\
सर्वकामफलावाप्ति-साधनैकसुखावहाम् ॥२॥\\[10pt]
ॐ श्रीं ह्रीं एें लक्ष्म्यै नमः ॥\\[10pt]
स्मरामि नित्यं देवेशि त्वया प्रेरितमानसः~।\\
त्वदाज्ञां शिरसा धृत्वा भजामि परमेश्वरीम् ॥३॥\\[10pt]
समस्तसम्पत्सुखदां महाश्रियं समस्तसौभाग्यकरीं महाश्रियम्~।\\
समस्तकल्याणकरीं महाश्रियं भजाम्यहं ज्ञानकरीं महाश्रियम् ॥४॥\\[10pt]
विज्ञानसम्पत्सुखदां सनातनीं विचित्रवाग्भूतिकरीं मनोहराम्~।\\
अनन्तसम्मोदसुखप्रदायिनीं नमाम्यहं भूतिकरीं हरिप्रियाम् ॥५॥\\[10pt]
समस्तभूतान्तरसंस्थिता त्वं समस्तभोक्त्रीश्श्वरि विश्वरूपे~।\\
तन्नास्ति यत्त्वद्व्यतिरिक्तवस्तु त्वत्पादपद्मं प्रणमाम्यहं श्रीः ॥६॥\\[10pt]
दारिद्र्यदुःखौघतमोपहन्त्रि त्वत्पादपद्मं मयि सन्निधत्स्व~।\\
दीनार्तिविच्छेदनहेतुभूतैः कृपाकटाक्षैरभिषिञ्च मां श्रीः ॥७॥\\[10pt]
अम्ब प्रसीद करुणासुधयार्द्रदृष्ट्या मां त्वत्कृपाद्रविणगेहमिमं कुरुष्व~।\\
आलोकय प्रणतहृद्गतशोकहन्त्रि त्वत्पादपद्मयुगलं प्रणमाम्यहं श्रीः ॥८॥\\[10pt]
शान्त्यै नमोऽस्तु शरणागतरक्षणायै कान्त्यै नमोऽस्तु कमनीयगुणाश्रयायै~।\\
क्षान्त्यै नमोऽस्तु दुरितक्षयकारणायै धात्र्यै नमोऽस्तु धनधान्यसमृद्धिदायै ॥९॥\\[10pt]
\newpage
शक्त्यै नमोऽस्तु शशिशेखरसंस्तुतायै रत्यै नमोऽस्तु रजनीकरसोदरायै~।\\
भक्त्यै नमोऽस्तु भवसागरतारकायै मत्यै नमोऽस्तु मधुसूदनवल्लभायै ॥१०॥\\[10pt]
लक्ष्म्यै नमोऽस्तु शुभलक्षणलक्षितायै सिद्ध्यै नमोऽस्तु शिवसिद्धसुपूजितायै~।\\
धृत्यै नमोऽस्त्वमितदुर्गतिभञ्जनायै गत्यै नमोऽस्तु वरसद्गतिदायिकायै ॥११॥\\[10pt]
देव्यै नमोऽस्तु दिवि देवगणार्चितायै भूत्यै नमोऽस्तु भुवनार्तिविनाशनायै~।\\
धात्र्यै नमोऽस्तु धरणीधरवल्लभायै पुष्ट्यै नमोऽस्तु पुरुषोत्तमवल्लभायै ॥१२॥\\[10pt]
सुतीव्रदारिद्र्यविदुःखहन्त्र्यै नमोऽस्तु ते सर्वभयापहन्त्र्यै~।\\
श्रीविष्णुवक्षःस्थलसंस्थितायै नमो नमः सर्वविभूतिदायै ॥१३॥\\
\newpage
जयतु जयतु लक्ष्मीर्लक्षणालङ्कृताङ्गी जयतु जयतु पद्मा पद्मसद्माभिवन्द्या~।\\
जयतु जयतु विद्या विष्णुवामाङ्कसंस्था जयतु जयतु सम्यक्सर्वसम्पत्करी श्रीः ॥१४॥\\[10pt]
जयतु जयतु देवी  देवसङ्घाभिपूज्या जयतु जयतु भद्रा भार्गवी भाग्यरूपा~।\\
जयतु जयतु नित्या निर्मलज्ञानवेद्या जयतु जयतु सत्या सर्वभूतान्तरस्था ॥१५॥\\[10pt]
जयतु जयतु रम्या रत्नगर्भान्तरस्था जयतु जयतु शुद्धा शुद्धजाम्बूनदाभा~।\\
जयतु जयतु कान्ता कान्तिमद्भासिताङ्गी जयतु जयतु शान्ता शीघ्रमागच्छ सौम्ये ॥१६॥\\[10pt]
यस्याः कलायाः कमलोद्भवाद्या रुद्राश्च शक्रप्रमुखाश्च देवाः~।\\
जीवन्ति सर्वेऽपि सशक्तयस्ते प्रभुत्वमापुः परमायुषस्ते ॥१७॥\\[10pt]
{\bfseries पादबीजम्~। ॐ आं ईं यं पं कं लं हं ॥}\\[10pt]

लिलेख निटिले विधिर्मम लिपिं विसृज्यान्तरं
त्वया विलिखितव्यमेतदिति तत्फलप्राप्तये~।\\
तदन्तरतले स्फुटं कमलवासिनि श्रीरिमां
समर्पय समुद्रिकां सकलभाग्यसंसूचिकाम् ॥१८॥\\[10pt]
तदिदं तिमिरं फाले स्फुटं कमलवासिनि~।\\
श्रियं समुद्रिकां देहि सर्वभाग्यस्य सूचिकाम् ॥१९॥\\[10pt]

{\bfseries मुखबीजम्~। ॐ ह्रां ह्रीं आं व्यां यं भां सां  ॥}\\[10pt]

कलया ते यथा देवि जीवन्ति सचराचराः~।\\
तथा सम्पत्करे लक्ष्मि सर्वदा सम्प्रसीद मे ॥२०॥\\[10pt]

यथा विष्णुर्ध्रुवे नित्यं स्वकलां सन्न्यवेशयत्~।\\
तथैव स्वकलां लक्ष्मि मयि सम्यक्समर्पय ॥२१॥\\[10pt]

सर्वसौख्यप्रदे देवि भक्तानामभयप्रदे~।\\
अचलां कुरु यत्नेन कलां मयि निवेशिताम् ॥२२॥\\[10pt]
मुदास्तां मत्फाले परमपदलक्ष्मीः स्फुटकला\\
सदा वैकुण्ठश्रीर्निवसतु कला मे नयनयोः~।\\
वसेत्सत्ये लोके मम वचसि लक्ष्मीर्वरकला\\
श्रियः श्वेतद्वीपे निवसतु कला मेऽस्तु करयोः ॥२३॥\\[10pt]

{\bfseries नेत्रबीजम्~। ॐ घ्रां घ्रीं घ्रूं घ्रैं घ्रौं घ्रः ॥}\\[10pt]

तावन्नित्यं ममाङ्गेषु क्षीराब्धौ श्रीकला वसेत्~।\\
सूर्याचन्द्रमसौ यावद्यावल्लक्ष्मीपतिः श्रिया ॥२४॥\\[10pt]

सर्वमङ्गलसम्पूर्णा सर्वैश्वर्यसमन्विता~।\\
आद्यादिश्रीर्महालक्ष्मीस्त्वत्कला मयि तिष्ठतु ॥२५॥\\[10pt]

अज्ञानतिमिरं हन्तुं शुद्धज्ञानप्रकाशिका~।\\
सर्वैश्वर्यप्रदा मेऽस्तु त्वत्कला मयि संस्थिता ॥२६॥\\[10pt]

अलक्ष्मीं हरतु क्षिप्रं तमः सूर्यप्रभा यथा~।\\
वितनोतु मम श्रेयस्त्वत्कला मयि संस्थिता ॥२७॥\\[10pt]

ऐश्वर्यमङ्गलोत्पत्तिः त्वत्कलायां निधीयते~।\\
मयि तस्मात्कृतार्थोऽस्मि पात्रमस्मि स्थितेस्तव ॥२८॥\\[10pt]
\newpage
भवदावेशभाग्यार्हो भाग्यवानस्मि भार्गवि~।\\
त्वत्प्रसादात्पवित्रोऽहं लोकमातर्नमोऽस्तु ते ॥२९॥\\[10pt]
{\bfseries जिह्वाबीजम्~। ॐ ह्रां ह्रीं ह्रूं ह्रैं ह्रौं ह्रः ॥}\\[10pt]
पुनासि मां त्वं कलयैव यस्मादतः समागच्छ ममाग्रतस्त्वम्~।\\
परं पदं श्रीर्भव सुप्रसन्ना मय्यच्युतेन प्रविशाऽऽदिलक्ष्मीः ॥३०॥\\[10pt]
श्रीवैकुण्ठस्थिते लक्ष्मि समागच्छ ममाग्रतः~।\\
नारायणेन सह मां कृपादृष्ट्याऽवलोकय ॥३१॥\\[10pt]
सत्यलोकस्थिते लक्ष्मि त्वं ममागच्छ सन्निधिम्~।\\
वासुदेवेन सहिता प्रसीद वरदा भव ॥३२॥\\[10pt]
श्वेतद्वीपस्थिते लक्ष्मि शीघ्रमागच्छ सुव्रते~।\\
विष्णुना सहिता देवि जगन्मातः प्रसीद मे ॥३३॥\\[10pt]
क्षीराम्बुधिस्थिते लक्ष्मि समागच्छ समाधवे~।\\
त्वत्कृपादृष्टिसुधया सततं मां विलोकय ॥३४॥\\[10pt]
रत्नगर्भस्थिते लक्ष्मि परिपूर्णहिरण्मयि~।\\
समागच्छ समागच्छ स्थित्वाशु पुरतो मम ॥३५॥\\[10pt]
स्थिरा भव महालक्ष्मि निश्चला भव निर्मले~।\\
प्रसन्ने  कमले देवि प्रसन्नहृदया भव ॥३६॥\\[10pt]
\newpage
श्रीधरे श्रीमहाभूमे त्वदन्तःस्थं महानिधिम्~।\\
शीघ्रमुद्धृत्य पुरतः प्रदर्शय समर्पय ॥३७॥\\[10pt]
वसुन्धरे श्रीवसुधे वसुदोग्ध्रि कृपामयि~।\\
त्वत्कुक्षिगतसर्वस्वं शीघ्रं मे सम्प्रदर्शय ॥३८॥\\[10pt]
विष्णुप्रिये रत्नगर्भे समस्तफलदे शिवे~।\\
त्वद्गर्भगतहेमादीन् सम्प्रदर्शय दर्शय ॥३९॥\\[10pt]
रसातलगते लक्ष्मि शीघ्रमागच्छ मे पुरः~।\\
न जाने परमं रूपं मातर्मे सम्प्रदर्शय ॥४०॥\\[10pt]
\newpage
आविर्भव मनोवेगात् शीघ्रमागच्छ मे पुरः~।\\
मा वत्स भैरिहेत्युक्त्वा कामं गौरिव रक्ष माम् ॥४१॥\\[10pt]
देवि शीघ्रं ममागच्छ धरणीगर्भसंस्थिते~।\\
मातस्त्वद्भृत्यभृत्योऽहं मृगये त्वां कुतूहलात् ॥४२॥\\[10pt]
उत्तिष्ठ जागृहि त्वं मे समुत्तिष्ठ सुजागृहि~।\\
अक्षयान् हेमकलशान् सुवर्णेन सुपूरितान् ॥४३॥\\[10pt]
निक्षेपान्मे समाकृष्य समुद्धृत्य ममाग्रतः~।\\
समुन्नतानना भूत्वा समाधेहि धरान्तरात् ॥४४॥\\[10pt]
\newpage
मत्सन्निधिं समागच्छ मदाहितकृपारसात् ~।\\
प्रसीद श्रेयसां दोग्ध्रि लक्ष्मि मे नयनाग्रतः ॥४५॥\\[10pt]
अत्रोपविश्य लक्ष्मि त्वं स्थिरा भव हिरण्मयि~।\\
सुस्थिरा भव सम्प्रीत्या प्रसीद वरदा भव ॥४६॥\\[10pt]
 आनीय त्वं  तथा देवि निधीन्मे सम्प्रदर्शय~।\\
अद्य क्षणेन सहसा दत्त्वा संरक्ष मां सदा ॥४७॥\\[10pt]
 मयि तिष्ठ तथा नित्यं यथेन्द्रादिषु तिष्ठसि~।\\
अभयं कुरु मे देवि महालक्ष्मि नमोऽस्तु ते ॥४८॥\\[10pt]
\newpage
समागच्छ महालक्ष्मि शुद्धजाम्बूनदप्रभे~।\\
प्रसीद पुरतः स्थित्वा प्रणतं मां विलोकय ॥४९॥\\[10pt]
लक्ष्मीर्भुवं गता भासि यत्र यत्र हिरण्मयि~।\\
तत्र तत्र स्थिता त्वं मे तव रूपं प्रदर्शय ॥५०॥\\[10pt]
 क्रीडसे बहुधा भूमौ परिपूर्णा हिरण्मयी~।\\
मम मूर्धनि ते हस्तमविलम्बितमर्पय ॥५१॥\\[10pt]
फलद्भाग्योदये लक्ष्मि समस्तपुरवासिनि~।\\
प्रसीद मे महालक्ष्मि परिपूर्णमनोरथे ॥५२॥\\[10pt]
\newpage
अयोध्यादिषु सर्वेषु नगरेषु समास्थिते~।\\
वैभवैर्विविधैर्युक्ता समागच्छ बलान्विते ॥५३॥\\[10pt]
समागच्छ समागच्छ ममाग्रे भव सुस्थिरा~।\\
करुणारसनिष्यन्दनेत्रद्वयविलासिनि ॥५४॥\\[10pt]
सन्निधत्स्व महालक्ष्मि त्वत्पाणिं मम मस्तके~।\\
करुणासुधया मां त्वमभिषिञ्च स्थिरीकुरु ॥५५॥\\[10pt]
सर्वराजगृहे लक्ष्मि समागच्छ बलान्विते~।\\
स्थित्वाऽऽशु पुरतो मेऽद्य प्रसादेनाभयं कुरु ॥५६॥\\[10pt]
\newpage
सादरं मस्तके हस्तं मम त्वं कृपयाऽर्पय~।\\
सर्वराजगृहे लक्ष्मि त्वत्कला मयि तिष्ठतु ॥५७॥\\[10pt]
आद्यादि श्रीर्महालक्ष्मि विष्णुवामाङ्कसंस्थिते~।\\
प्रत्यक्षं कुरु मे रूपं रक्ष मां शरणागतम् ॥५८॥\\[10pt]
प्रसीद मे महालक्ष्मि सुप्रसीद महाशिवे~।\\
अचला भव सम्प्रीत्या सुस्थिरा भव मद्गृहे ॥५९॥\\[10pt]
यावत्तिष्ठन्ति वेदाश्च यावत्त्वन्नाम तिष्ठति~।\\
यावद्विष्णुश्च यावत्त्वं तावत्कुरु कृपां मयि ॥६०॥\\[10pt]
\newpage
चान्द्री कला यथा शुक्ले वर्धते सा दिने दिने~।\\
तथा दया ते मय्येव वर्धतामभिवर्धताम् ॥६१॥\\[10pt]
यथा वैकुण्ठनगरे यथा वै क्षीरसागरे~।\\
तथा मद्भवने तिष्ठ स्थिरं श्रीविष्णुना सह ॥६२॥\\[10pt]
योगिनां हृदये नित्यं यथा तिष्ठसि विष्णुना~।\\
तथा मद्भवने तिष्ठ स्थिरं श्रीविष्णुना सह ॥६३॥\\[10pt]
नारायणस्य हृदये भवती यथाऽऽस्ते नारायणोऽपि तव हृत्कमले यथाऽऽस्ते~।\\
नारायणस्त्वमपि नित्यमुभौ तथैव तौ तिष्ठतां हृदि ममापि दयावति श्रीः ॥६४॥\\[10pt]
\newpage
विज्ञानवृद्धिं हृदये कुरु श्रीः सौभाग्यवृद्धिं कुरु मे गृहे श्रीः~।\\
दयासुवृद्धिं कुरुतां मयि श्रीः सुवर्णवृद्धिं कुरु मे गृहे श्रीः ॥६५॥\\[10pt]
न मां त्यजेथाः श्रितकल्पवल्लि सद्भक्तचिन्तामणिकामधेनो~।\\
न मां त्यजेथा भव सुप्रसन्ने गृहे कलत्रेषु च पुत्रवर्गे ॥६६॥\\[10pt]
{\bfseries कुक्षिबीजम्~। ॐ आं ईं एं ऐं ॥}\\[10pt]
आद्यादिमाये त्वमजाण्डबीजं त्वमेव साकारनिराकृतिस्त्वम्~।\\
त्वया धृताश्चाब्जभवाण्डसङ्घाश्चित्रं चरित्रं तव देवि विष्णोः ॥६७॥\\[10pt]
ब्रह्मरुद्रादयो देवा वेदाश्चापि न शक्नुयुः~।\\
महिमानं तव स्तोतुं मन्दोऽहं शक्नुयां कथम् ॥६८॥\\[10pt]
अम्ब त्वद्वत्सवाक्यानि सूक्तासूक्तानि यानि च~।\\
तानि स्वीकुरु सर्वज्ञे दयालुत्वेन सादरम् ॥६९॥\\[10pt]
भवतीं शरणं गत्वा कृतार्थाः स्युः पुरातनाः~।\\
इति सञ्चिन्त्य मनसा त्वामहं शरणं व्रजे ॥७०॥\\[10pt]
अनन्ता नित्यसुखिनः त्वद्भक्तास्त्वत्परायणाः~।\\
इति वेदप्रमाणाद्धि देवि त्वां शरणं व्रजे ॥७१॥\\[10pt]
तव प्रतिज्ञा मद्भक्ता न नश्यन्तीत्यपि क्वचित्~।\\
इति सञ्चिन्त्य सञ्चिन्त्य प्राणान् सन्धारयाम्यहम् ॥७२॥\\[10pt]
\newpage
त्वदधीनस्त्वहं मातः त्वत्कृपा मयि विद्यते~।\\
यावत्सम्पूर्णकामः स्यां तावद्देहि दयानिधे ॥७३॥\\[10pt]
क्षणमात्रं न शक्नोमि जीवितुं त्वत्कृपां विना~।\\
न जीवन्तीह जलजा जलं त्यक्त्वा जलग्रहाः ॥७४॥\\[10pt]
यथा हि पुत्रवात्सल्यात् जननी प्रस्नुतस्तनी~।\\
वत्सं त्वरितमागत्य सम्प्रीणयति वत्सला ॥७५॥\\[10pt]
यदि स्यां तव पुत्रोऽहं माता त्वं यदि मामकी~।\\
दयापयोधरस्तन्य-सुधाभिरभिषिञ्च माम् ॥७६॥\\[10pt]
\newpage
मृग्यो न गुणलेशोऽपि मयि दोषैकमन्दिरे~।\\
पांसूनां वृष्टिबिन्दूनां दोषाणां च न मे मतिः ॥७७॥\\[10pt]
पापिनामहमेकाग्र्यो दयालूनां त्वमग्रणीः~।\\
दयनीयो मदन्योऽस्ति तव कोऽत्र जगत्त्रये ॥७८॥\\[10pt]
विधिनाहं न सृष्टश्चेत् न स्यात्तव दयालुता~।\\
आमयो वा न सृष्टश्चेदौषधस्य वृथोदयः ॥७९॥\\[10pt]
कृपा मदग्रजा किं ते त्वहं किं वा तदग्रजः~।\\
विचार्य देहि मे वित्तं तव देवि दयानिधे ॥८०॥\\[10pt]
\newpage
माता पिता त्वं गुरुसद्गती श्रीस्त्वमेव सञ्जीवनहेतुभूता~।\\
अन्यं न मन्ये जगदेकनाथे त्वमेव सर्वं मम देवि सत्ये ॥८१॥\\[10pt]
{\bfseries हृदय बीजम् ॥ॐ अां क्रौं हुं फट् कुरु कुरु स्वाहा ॥}\\[10pt]
आद्यादिलक्ष्मीर्भव सुप्रसन्ना विशुद्धविज्ञानसुखैकदोग्ध्री~।\\
अज्ञानहन्त्री त्रिगुणातिरिक्ता प्रज्ञाननेत्री भव सुप्रसन्ना ॥८२॥\\[10pt]
अशेषवाग्जाड्यमलापहन्त्री नवं नवं स्पष्टसुवाक्प्रदायिनी~।\\
ममेह जिह्वाङ्गणरङ्गनर्तकी भव प्रसन्ना वदने च मे श्रीः ॥८३॥\\[10pt]
समस्तसम्पत्सु विराजमाना समस्ततेजश्चयभासमाना~।\\
विष्णुप्रिये त्वं भव दीप्यमाना वाग्देवता मे नयने प्रसन्ना ॥८४॥\\[10pt]
भक्त्या नतानां सकलार्थसिद्धिप्रदे सुलावण्यदयाप्रदोग्ध्रि~।\\
सुवर्णदे त्वं सुमुखी भव श्रीर्हिरण्मयी मे नयने प्रसन्ना ॥८५॥\\[10pt]
सर्वार्थदा सर्वजगत्प्रसूतिः सर्वेश्वरी सर्वभयापहन्त्री~।\\
सर्वोन्नता त्वं सुमुखी भव श्रीर्हिरण्मयी मे नयने प्रसन्ना ॥८६॥\\[10pt]
समस्तविघ्नौघविनाशकारिणी समस्तभक्तोद्धरणे विचक्षणा~।\\
अनन्तसौभाग्यसुखप्रदायिनी हिरण्मयी मे नयने प्रसन्ना ॥८७॥\\[10pt]
देवि  प्रसीद दयनीयतमाय मह्यं देवाधिनाथभवदेवगणाभिवन्द्ये~।\\
मातस्तथैव भव सन्निहिता दृशोर्मे पत्या समं मम मुखे भव सुप्रसन्ना ॥८८॥\\
\newpage
मा वत्स भैरभयदानकरोऽर्पितस्ते मौलौ ममेति मयि दीनदयानुकम्पे~।\\
मातः समर्पय मुदा करुणाकटाक्षं माङ्गल्यबीजमिह नः सृज जन्म मातः ॥८९॥\\[10pt]
{\bfseries कण्ठबीजम्~। ॐ श्रां श्रीं श्रूं श्रैं श्रौं श्रः ॥}\\[10pt]
कटाक्ष इह कामधुक् तव मनस्तु चिन्तामणिः करः सुरतरुः सदा नवनिधिस्त्वमेवेन्दिरे~।\\
भवेत्तव दयारसो मम रसायनं चान्वहं मुखं तव कलानिधिर्विविध-वाञ्छितार्थप्रदम् ॥९०॥\\[10pt]
यथा रसस्पर्शनतोऽयसोऽपि सुवर्णता स्यात्कमले तथा ते~।\\
कटाक्षसंस्पर्शनतो जनानाममङ्गलानामपि मङ्गलत्वम् ॥९१॥\\[10pt]
देहीति नास्तीति वचः प्रवेशाद्भीतो रमे त्वां शरणं प्रपद्ये~।\\
अतः सदास्मिन्नभयप्रदा त्वं सहैव पत्या मयि सन्निधेहि ॥९२॥\\[10pt]
कल्पद्रुमेण मणिना सहिता सुरभ्या श्रीस्ते कला मयि रसेन रसायनेन~।\\
आस्तां यतो मम च दृक्शिरपाणिपादस्पृष्टाः सुवर्णवपुषः स्थिरजङ्गमाः स्युः ॥९३॥\\[10pt]
आद्यादिविष्णोः स्थिरधर्मपत्नी त्वमेव पत्या मयि सन्निधेहि~।\\
आद्यादिलक्ष्मि त्वदनुग्रहेण पदे पदे मे निधिदर्शनं स्यात् ॥९४॥\\[10pt]
आद्यादिलक्ष्मीहृदयं पठेद्यः स राज्यलक्ष्मीमचलां तनोति~।\\
महादरिद्रोऽपि भवेद्धनाढ्यः तदन्वये श्रीः स्थिरतां प्रयाति ॥९५॥\\[10pt]
यस्य स्मरणमात्रेण तुष्टा स्याद्विष्णुवल्लभा~।\\
तस्याभीष्टं ददात्याशु तं पालयति पुत्रवत् ॥९६॥\\[10pt]
\newpage
इदं रहस्यं हृदयं सर्वकामफलप्रदम्~।\\
जपः पञ्चसहस्रं तु पुरश्चरणमुच्यते ॥९७॥\\[10pt]
त्रिकालं एककालं वा नरो भक्तिसमन्वितः~।\\
यः पठेत् शृणुयाद्वापि स याति परमां श्रियम् ॥९८॥\\[10pt]
महालक्ष्मीं समुद्दिश्य निशि भार्गववासरे~।\\
इदं श्रीहृदयं जप्त्वा पञ्चवारं धनी भवेत् ॥९९॥\\[10pt]
अनेन हृदयेनान्नं गर्भिण्यै चाभिमन्त्रितम्~।\\
ददाति तत्कुले पुत्रो जायते श्रीपतिः स्वयम् ॥१००॥\\[10pt]
\newpage
नरेण वाथवा नार्या लक्ष्मीहृदयमन्त्रिते~।\\
जले पीते च तद्वंशे मन्दभाग्यो न जायते ॥१०१॥\\[10pt]
य आश्विने मासि च शुक्लपक्षे रमोत्सवे सन्निहितैकभक्त्या~।\\
पठेत्तथैकोत्तरवारवृद्ध्या लभेत्स सौवर्णमयीं सुवृष्टिम् ॥१०२॥\\[10pt]
य एकभक्तोऽन्वहमेकवर्षं विशुद्धधीः सप्ततिवारजापी~।\\
स मन्दभाग्योऽपि रमाकटाक्षात् भवेत्सहस्राक्षशताधिकश्रीः ॥१०३॥\\[10pt]
श्रीशाङ्घ्रिभक्तिं हरिदासदास्यं प्रसन्नमन्त्रार्थदृढैकनिष्ठाम्~।\\
गुरोः स्मृतिं निर्मलबोधबुद्धिं प्रदेहि मातः परमं पदं श्रीः ॥१०४॥\\[10pt]
\newpage
पृथ्वीपतित्वं पुरुषोत्तमत्वं विभूतिवासं विविधार्थसिद्धिम्~।\\
सम्पूर्णकीर्तिं बहुवर्षभोगं प्रदेहि मे देवि पुनःपुनस्त्वम् ॥१०५॥\\[10pt]
वादार्थसिद्धिं बहुलोकवश्यं वयःस्थिरत्वं ललनासु भोगम्~।\\
पौत्रादिलब्धिं सकलार्थसिद्धिं प्रदेहि मे भार्गवि जन्मजन्मनि ॥१०६॥\\[10pt]
सुवर्णवृद्धिं कुरु मे गृहे श्रीः सुधान्यवृद्धिं कुरु मे गृहे श्रीः~।\\
कल्याणवृद्धिं कुरु मे गृहे श्रीः विभूतिवृद्धिं कुरु मे गृहे श्रीः ॥१०७॥\\[10pt]
{\bfseries शिरो बीजम्~। ॐ यं हं कं लं पं श्रीं ॥}\\[10pt]
ध्यायेल्लक्ष्मीं प्रहसितमुखीं कोटिबालार्कभासां
विद्युत्वर्णाम्बरवरधरां भूषणाढ्यां सुशोभाम्~।\\
बीजापूरं सरसिजयुगं बिभ्रतीं स्वर्णपात्रं
भर्त्रा युक्तां मुहुरभयदां मह्यमप्यच्युतश्रीः ॥१०८॥\\[10pt]
गुह्यातिगुह्यगोप्त्री त्वं गृहाणास्मत्कृपं जपम्~।\\
सिद्धिर्भवतु मे देवि त्वत्प्रसादान्मयि स्थिरा ॥१०९॥\\[10pt]
॥ इत्याथर्वणरहस्ये श्रीलक्ष्मीहृदयस्तोत्रं सम्पूर्णम् ॥\\[10pt]
(मूलेन दशवारं तर्पयेत् ।)
\end{center}
\newpage
\begin{center}{\bfseries\Huge ॥श्रीनारायणहृदयम्॥}\end{center}
अस्य श्रीनारायणहृदयस्तोत्रमन्त्रस्य भार्गव-ऋषिः।अनुष्टुप्छन्दः।श्रीलक्ष्मीनारायणो देवता।\\ॐ बीजम्।नमः शक्तिः।नारायणायेति कीलकम्।श्रीलक्ष्मीनारायणप्रीत्यर्थे जपे विनियोगः॥\\
ॐ नारायणः परञ्ज्योतिः अङ्गुष्ठाभ्यां नमः~। हृदयाय नमः~।\\
ॐ नारायणः परम्ब्रह्म 	तर्जनीभ्यां नमः~। शिरसे स्वाहा~।\\
ॐ नारायणः परो देवः मध्यमाभ्यां नमः~। शिखायै वषट्~।\\
ॐ नारायणः परो ध्याता अनामिकाभ्यां नमः~। कवचाय हुं~।\\
ॐ नारायणः परं धाम कनिष्ठिकाभ्यां नमः~। नेत्रत्रयाय वौषट्~।\\
ॐ नारायणः परो धर्मः करतलकरपृष्ठाभ्यां नमः~। अस्त्राय फट्~।\\
ॐ नमः सुदर्शनाय सहस्रार हुं फट् ऐन्द्र्यादिदशदिशं बध्नामि नमश्चक्राय स्वाहा~।
\begin{center}
॥ध्यानम्॥\\
उद्यद्भास्वत्समाभासश्चिदानन्दैकदेहवान्~।\\
चक्रशङ्खगदापद्मधरो ध्येयोऽहमीश्वरः ॥१॥\\[10pt]
लक्ष्मीधराभ्यामाश्लिष्टः स्वमूर्तिगणमध्यगः~।\\
ब्रह्मवायुशिवाहीशविपैः शक्रादिकैरपि ॥२॥\\[10pt]
सेव्यमानोऽधिकं भक्त्या नित्यनिश्शेषभक्तिमान्~।\\
मूर्तयोऽष्टावपि ध्येयाश्चक्रशङ्खवराभयैः ॥३॥\\[10pt]
\newpage
उद्यादादित्यसङ्काशं पीतवाससमच्युतम्~।\\
शङ्खचक्रगदापाणिं ध्यायेल्लक्ष्मीपतिं हरिम् ॥४॥\\[10pt]
{\bfseries ॐ ऐं ह्रीं श्रीं श्रीलक्ष्मीनारायणाय स्वाहा ॥}(१०८)\\
॥मूलाष्टकम्॥\\
नारायणः परं ज्योतिरात्मा नारायणः परः~।\\
नारायणः परं ब्रह्म नारायण नमोऽस्तु ते ॥१॥\\[10pt]
नारायणः परो देवो धाता नारायणः परः~।\\
नारायणः परो ध्याता नारायण नमोऽस्तु ते ॥२॥\\[10pt]
\newpage
नारायणः परं धाम ध्यानं नारायणः परः~।\\
नारायणः परो धर्मो नारायण नमोऽस्तु ते ॥३॥\\[10pt]
नारायणः परो देवो विद्या नारायणः परः~।\\
विश्वं नारायणः साक्षान्नारायण नमोऽस्तु ते ॥४॥\\[10pt]
नारायणाद्विधिर्जातो जातो नारायणाद्धरः~।\\
जातो नारायणादिन्द्रो नारायण नमोऽस्तु ते ॥५॥\\[10pt]
रविर्नारायणस्तेजः चन्द्रो नारायणो परः~।\\
वह्निर्नारायणः साक्षान्नारायण नमोऽस्तु ते ॥६॥\\[10pt]
\newpage
नारायण उपास्यः स्याद्गुरुर्नारायणः परः~।\\
नारायणः परो बोधो नारायण नमोऽस्तु ते ॥७‍॥\\[10pt]
नारायणः फलं मुख्यं सिद्धिर्नारायणः सुखम्~।\\
सेव्यो नारायणः शुद्धो नारायण नमोऽस्तु ते ॥८॥\\
॥इति मूलाष्टकम् ॥\\
(अष्टवारं मूलं जप्त्वा सकृत्  तर्पयेत् ।)\\[10pt]
नारायणस्त्वमेवासि दहराख्ये हृदि स्थितः~।\\
प्रेरकः प्रेर्यमाणानां त्वया प्रेरित मानसः ॥१॥\\[10pt]
त्वदज्ञां शिरसा धृत्वा जपामि जनपावनम्~।\\
नानोपासनमार्गाणां भवहृद्भावबोधकः ॥२॥\\[10pt]
भावार्थकृद् भावभूतो भवसौख्यप्रदो भव~।\\
त्वन्मायामोहितं विश्वं त्वयैव परिकल्पितम् ॥३॥\\[10pt]
त्वदधिष्ठानमात्रेण सैव सर्वार्थकारिणी~।\\
त्वमेव तां पुरस्कृत्य मम कामान् समर्थय ॥४॥\\[10pt]
न मे त्वदन्यस्त्रातास्ति त्वदन्यन्न हि दैवतम्~।\\
त्वदन्यं न हि जानामि पालकं पुण्यरूपकम् ॥५॥\\[10pt]
यावत्सांसारिको भावो मनःस्थो भावनात्मकः~।\\
तावत्सिद्धिर्भवेत् सद्यः सर्वथा सर्वदा विभो ॥६॥\\[10pt]
\newpage
पापिनामहमेवाग्र्यो दयालूनां त्वमग्रणीः~।\\
दयनीयो मदन्योऽस्ति तव कोऽत्र जगत्त्रये ॥७॥\\[10pt]
त्वयाप्यहं न सृष्टश्चेत् न स्यात्तव दयालुता~।\\
आमयो नैव सृष्टश्चेदौषधस्य वृथोदयः ॥८॥\\[10pt]
पापसङ्घपरिक्रान्तः पापात्मा पापरूपधृक्~।\\
त्वदन्यः कोऽत्र पापेभ्यस्त्राता मे जगतीतले ॥९॥\\[10pt]
त्वमेव माता च पिता त्वमेव त्वमेव बन्धुश्च सखा त्वमेव~।\\
त्वमेव विद्या द्रविणं त्वमेव त्वमेव सर्वं मम देव देव ॥१०॥\\[10pt]
\newpage
प्रार्थनादशकं चैव मूलष्टकमितिद्वयम्~।\\
यः पठेच्छृणुयान्नित्यं तस्य लक्ष्मीः स्थिरा भवेत् ॥११॥\\[10pt]
नारायणस्य हृदयं सर्वाभीष्टफलप्रदम्~।\\
लक्ष्मीहृदयकं स्तोत्रं यदि चैतद्विनाकृतम् ॥१२॥\\[10pt]
तत्सर्वं निष्फलं प्रोक्तं लक्ष्मीः क्रुध्यति सर्वदा~।\\
एतत्सम्पुटितं स्तोत्रं सर्वकर्मफलप्रदम् ॥१३॥\\[10pt]
लक्ष्मीहृदयकं चैव तथा नारायणात्मकम्~।\\
जपेद्यः सङ्कलीकृत्य सर्वाभीष्टमवाप्नुयात् ॥१४॥\\[10pt]
\newpage
नारायणस्य हृदयं आदौ जप्त्वा ततःपरम्~।\\
लक्ष्मीहृदयकं स्तोत्रं जपेन्नारायणं पुनः ॥१५॥\\[10pt]
पुनर्नारायणं जप्त्वा पुनर्लक्ष्मीस्तवं जपेत्~।\\
पुनर्नारायणं जाप्यं सङ्कलीकरणं भवेत् ॥१६॥\\[10pt]
एवं मध्ये द्विवारेण जपेत् सङ्कलितं तु तत्~।\\
लक्ष्मीहृदयकं स्तोत्रं सर्वकामप्रकाशितम् ॥१७॥\\[10pt]
तद्वज्जपादिकं कुर्यादेतत्सङ्कलितं शुभम्~।\\
सर्वान्कामानवाप्नोति आधिव्याधिभयं हरेत् ॥१८॥\\[10pt]
\newpage
गोप्यमेतत् सदा कुर्यात् न सर्वत्र प्रकाशयेत्~।\\
इति गुह्यतमं शास्त्रं प्रोक्तं ब्रह्मादिकैः पुरा ॥१९॥\\[10pt]
तस्मात्सर्वप्रयत्नेन गोपयेत्साधयेद्  सुधीः~।
यत्रैतपुस्तकं तिष्ठेल्लक्ष्मीनारायणात्मकम् ॥२०॥\\[10pt]
भूतपैशाचवेताळभयं तत्र न जायते~।
लक्ष्मीहृदयकं प्रोक्तं विधिना साधयेत् सुधीः ॥२१॥\\[10pt]
भृगुवारे तथा रात्रौ पूजयेत् पुस्तकद्वयं~।\\
सर्वथा सर्वदा सत्यं गोपयेत् साधयेत् सुधीः~।\\
गोपनात् साधनाल्लोके सर्वां सिद्धिं लभेन्नरः ॥२२॥\\[10pt]
इति नारायणहृदयस्तोत्रं सम्पूर्णम् ॥\\[10pt]

\end{center}
\end{document}
