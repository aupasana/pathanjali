\titleformat*{\section}{\large\bfseries }
\renewcommand*\contentsname{\Large विषयानुक्रमणिका}
\begin{document}
\title{॥वेदान्तपरिभाषा॥}\par
\author{धर्मराजाध्वरीन्द्रः \\ \\
सम्पादकः - श्रीनिवास कारन्तः}
\date{}
\maketitle
\thispagestyle{empty}
\frontmatter
\tableofcontents
\newpage\thispagestyle{empty}
\mainmatter
\renewcommand{\thepage}{\devanagarinumeral{page}}
\fancyhead[RE,LO]{वेदान्तपरिभाषा}
\fancyhead[LE,RO]{मङ्गलम्}
\section{मङ्गलम्}
\begin{figure}[h]
\centering
\includegraphics[width=8cm,hight=5cm]{Lord-Shiva-And-Parvathi-HD-Wallpaper}
\end{figure}
\begin{center}
यदविद्याविलासेन भूतभौतिकसृष्टयः~।\\
तं नौमि परमात्मानं सच्चिदानन्दविग्रहम् ॥१॥\\[10pt]
यदन्तेवासिपञ्चास्यैर्निरस्ता भेदिवारणाः~।\\ 
तं प्रणौमि नृसिंहाख्यं यतीन्द्रं परमं गुरुम् ॥२॥\\[10pt]
श्रीमद्वेङ्कटनाथाख्यान् वेलाङ्गुडिनिवासिनः~।\\ 
जगद्गुरूनहं वन्दे सर्वतन्त्रप्रवर्तकान् ॥३॥\\[10pt]
येन चिन्तामणौ टीका दशटीकाविभञ्जिनी~।\\ 
तर्कचूडामणिर्नाम कृता विद्वन्मनोरमा ॥४॥\\[10pt]
तेन बोधाय मन्दानां वेदान्तार्थावलम्बिनी~।\\ 
धर्मराजाध्वरीन्द्रेण परिभाषा वितन्यते ॥५॥\\[10pt]
\end{center}\par
\newpage
\fancyhead[LE,RO]{प्रत्यक्षप्रमाणम्}
\section{प्रत्यक्षप्रमाणम्}
	इह खलु धर्मार्थकाममोक्षाख्येषु चतुर्विधपुरुषार्थेषु मोक्ष एव परमपुरुषार्थः~। {\bfseries न स पुनरावर्तते}\footnote{शरभोपनिषत्} इत्यादिश्रुत्या तस्यैव नित्यत्वावगमात्~। इतरेषां त्रयाणां प्रत्यक्षेण, {\bfseries तद्यथेह कर्मचितो लोकः क्षीयते, एवमेवामुत्र पुण्यचितो लोकः क्षीयते}\footnote{छा.उ. ८.१.६} इत्यादिश्रुत्या च अनित्यत्वावगमाच्च~। स च ब्रह्मज्ञानादिति, ब्रह्म, तज्ज्ञानं, तत्प्रमाणञ्च सप्रपञ्चं निरूप्यते~। \par
	तत्र प्रमाकरणं प्रमाणम्~। तत्र स्मृतिव्यावृत्तं प्रमात्वमनधिगताबाधितार्थविषयकज्ञानत्वम्~। स्मृतिसाधारणन्तु अबाधितार्थविषयकज्ञानत्वम्~। नीरूपस्यापि कालस्येन्द्रियवेद्यत्वाभ्युपगमेन, धारावाहिकबुद्धेरपि पूर्वपूर्वज्ञानाविषयतत्तत्क्षणविशेषविषयकत्वेन न तत्राव्याप्तिः~। किञ्च सिद्धान्ते धारावाहिकबुद्धिस्थले न ज्ञानभेदः~। किन्तु यावद्घटस्फुरणं तावद्घटाकारान्तःकरणवृत्तिरेकैव, न तु नाना~। वृत्तेः स्वविरोधिवृत्युत्पत्तिपर्यन्तं स्थायित्वाभ्युपगमात्~। तथा च तत्प्रतिफलितचैतन्यरूपं घटादिज्ञानमपि तत्र तावत्कालीनमेकमेव इति नाव्याप्तिशङ्कापि~।\par
	ननु सिद्धान्ते घटादेर्मिथ्यात्वेन बाधितत्वात् कथं तज्ज्ञानं प्रमाणम् ? उच्यते~। ब्रह्मसाक्षात्कारानन्तरं हि घटादीनां बाधः, {\bfseries यत्र त्वस्य सर्वमात्मैवाभूत् तत् केन कं पश्येत्}\footnote{बृ.उ. ४.५.२५} इति श्रुतेः~। न तु संसारदशायां बाधः, {\bfseries यत्र हि द्वैतमिव भवति तदितर इतरं पश्यति}\footnote{बृ.उ. २.४.१४} इति श्रुतेः~। तथा च अबाधितपदेन संसारदशायामबाधितत्वं विवक्षितम्~। इति न घटादिप्रमायमव्याप्तिः~। तदुक्तम् - {\bfseries देहात्मप्रत्ययो यद्वत् प्रमाणत्वेन कल्पितः~। लौकिकं तद्वदेवेदं प्रमाणन्त्वाऽऽत्मनिश्चयात् ॥} इति~। 'आ आत्मनिश्चयात्'- ब्रह्मसाक्षात्कारपर्यन्तमित्यर्थः~। 'लौकिकम्' इति घटादिज्ञानमित्यर्थः~।\par
	तानि च प्रमाणानि षट् प्रत्यक्षानुमानोपमानागमार्थापत्त्यनुपलब्धिभेदात्~। तत्र प्रत्यक्षप्रमायाः करणं प्रत्यक्षप्रमाणम्~। प्रत्यक्षप्रमा चात्र चैतन्यमेव, {\bfseries यत् साक्षादपरोक्षाद् ब्रह्म}\footnote{बृ.उ. ३.४.१} इति श्रुतेः~। अपरोक्षादित्यस्य अपरोक्षमित्यर्थः~। \par
	ननु चैतन्यमनादि~। तत्कथं चक्षुरादेस्तत्करणत्वेन प्रमाणत्वमिति~। उच्यते~। चैतन्यस्यानादित्वेऽपि तदभिव्यञ्जकान्तःकरणवृत्तिरिन्द्रियसन्निकर्षादिना जायते इति वृत्तिविशिष्टं चैतन्यमादिमदित्युच्यते~। ज्ञानावच्छेदकत्वाच्च वृत्तौ ज्ञानत्वोपचारः~। तदुक्तं विवरणे {\bfseries अन्तःकरणवृत्तौ ज्ञानत्वोपचारात्}~। \par 
	ननु निरवयवस्यान्तःकरणस्य परिणामात्मिका वृत्तिः कथम्? इत्थम्~। न तावदन्तःकरणं निरवयवम्~। सादिद्रव्यत्वेन सावयवत्वात्~। सादित्वञ्च {\bfseries तन्मनोऽसृजत} इत्यादिश्रुतेः~। वृत्तिरूपज्ञानस्य मनोधर्मत्वे च {\bfseries कामः सङ्कल्पो विचिकित्सा श्रद्धाऽश्रद्धा धृतिरधृतिर्ह्रीर्धीर्भीरित्येतत् सर्वं मन एव}\footnote{बृ.उ. १.५.३} इति श्रुतिर्मानम्~, धीशब्देन वृत्तिरूपज्ञानाभिधानात्~। अत एव कामादेरपि मनोधर्मत्वम् ॥ \par
	ननु कामादेरन्तःकरणधर्मत्वे  "अहमिच्छामि, अहं जानामि, अहं बिभेमि" इत्याद्यनुभव आत्मधर्मत्वमवगाहमानः कथमुपपद्यते ? उच्यते~। अयःपिण्डस्य दग्धृत्वाभावेऽपि दग्धृत्वाश्रयवह्नितादात्म्याध्यासाद् यथा "अयो दहति" इति व्यवहारः, तथा सुखाद्याकारपरिणाम्यन्तःकरणैक्याध्यासात् "अहं सुखी, अहं दुःखी" इत्यादिव्यवहारः~। ननु अन्तःकरणस्येन्द्रियतयाऽतीन्द्रियत्वात् कथं प्रत्यक्षविषयतेति~। उच्यते~। न तावदन्तःकरणमिन्द्रियमित्यत्र मानमस्ति~। {\bfseries मनःषष्ठानीन्द्रियाणि}\footnote{भ.गी. १५.७} इति भगवद्गीतावचनं प्रमाणमिति चेत् न~। अनिन्द्रियेणापि मनसा षट्त्वसंख्यापूरणाविरोधात्~। नहीन्द्रियगतसंख्यापूरणमिन्द्रियेणैवेति नियमः~। {\bfseries यजमानपञ्चमा इडां भक्षयन्ति} इत्यत्र ऋत्विग्गतपञ्चत्वसंख्याया अनृत्विजाऽपि यजमानेन, {\bfseries वेदानध्यापयामास महाभारतपञ्चमान्} इत्यादौ च वेदगतपञ्चत्वसंख्याया अवेदेनापि भारतेन पूरणदर्शनात्~, {\bfseries इन्द्रियेभ्यः परा ह्यर्था अर्थेभ्यश्च परं मनः}\footnote{क.उ. १.३.१०} इत्यादिश्रुत्या मनसोऽनिन्द्रियत्वावगमाच्च~। \par
	न चैवं मनसोऽनिन्द्रियत्वे सुखादिप्रत्यक्षस्य साक्षात्त्वं न स्यात्~, इन्द्रियाजन्यत्वादिति वाच्यम्~। न हीन्द्रियजन्यत्वेन ज्ञानस्य साक्षात्त्वम् अनुमित्यादेरपि मनोजन्यतया साक्षात्त्वापत्तेः ईश्वरज्ञानस्यानिन्द्रियजन्यस्य साक्षात्त्वानापत्तेश्च~। \par
	सिद्धान्ते प्रत्यक्षत्वप्रयोजकं किमिति चेत्~, किं ज्ञानगतस्य प्रत्यक्षत्वस्य प्रयोजकं पृच्छसि ? किंवा विषयगतस्य ? आद्ये प्रमाणचैतन्यस्य विषयावच्छिन्नचैतन्याभेद इति ब्रूमः~। तथाहि त्रिविधं चैतन्यम् - विषयचैतन्यं, प्रमाणचैतन्यं, प्रमातृचैतन्यं चेति~। तत्र घटाद्यवच्छिन्नं चैतन्यं विषयचैतन्यम्~, अन्तःकरणवृत्त्यवच्छिन्नं चैतन्यं प्रमाणचैतन्यम्~, अन्तःकरणावच्छिन्नं चैतन्यं प्रमातृचैतन्यम्~।\par
	तत्र यथा तडागोदकं छिद्रान्निर्गत्य कुल्यात्मना केदारान् प्रविश्य तद्वदेव चतुष्कोणाद्याकारं भवति, तथा तैजसमन्तःकरणमपि चक्षुरादिद्वारा निर्गत्य घटादिविषयदेशं गत्वा घटादिविषयाकारेण परिणमते~। स एव परिणामो वृत्तिरित्युच्यते~। अनुमित्यादिस्थले तु अन्तःकरणस्य न वह्न्यादिदेशगमनम्~, वह्न्यादेश्चक्षुराद्यसन्निकर्षात्~। तथा च "अयं घटः" इत्यादिप्रत्यक्षस्थले घटादेस्तदाकारवृत्तेश्च बहिरेकत्र देशे समवधानात् तदुभयावच्छिन्नं चैतन्यमेकमेव, विभाजकयोरप्यन्तःकरणवृत्तिघटादिविषययोः एकदेशस्थत्वेन भेदाजनकत्वात्~। अत एव मठान्तर्वर्तिघटावच्छिन्नाकाशो न मठावच्छिन्नाकाशाद्भिद्यते~। तथा च "अयं घटः" इति घटप्रत्यक्षस्थले घटाकारवृत्तेर्घटसंयोगितया घटावच्छिन्नचैतन्यस्य तद्वृत्त्यवच्छिन्नचैतन्यस्य चाभिन्नतया तत्र घटज्ञानस्य घटांंशे प्रत्यक्षत्वम्~। सुखाद्यवच्छिन्नचैतन्यस्य तद्वृत्त्यवच्छिन्नचैतन्यस्य च नियमेनैकदेशस्थितोपाधिद्वयावच्छिन्नत्वात् नियमेन "अहं सुखी" इत्यादिज्ञानस्य प्रत्यक्षत्वम्~।\par
	नन्वेवं स्ववृत्तिसुखादिस्मरणस्यापि सुखाद्यंशे प्रत्यक्षत्वापत्तिरिति चेत् न~। तत्र स्मर्यमाणसुखस्यातीतत्वेन स्मृतिरूपान्तःकरणवृत्तेर्वर्तमानत्वेन तत्रोपाध्योर्भिन्नकालीनतया तत्तदवच्छिन्नचैतन्ययोर्भेदात्~। उपाध्योरेकदेशस्थत्वे सति एककालीनत्वस्यैवोपधेयाभेदप्रयोजकत्वात्~। यदि चैकदेशस्थत्वमात्रमुपधेयाभेदप्रयोजकम्~, तदा "अहं पूर्वं सुखी" इत्यादिस्मृतावतिव्याप्तिवारणाय वर्तमानत्वं विषयविशेषणं देयम्~।\par
	नन्वेवमपि स्वकीयधर्माधर्मौ वर्तमानौ यदा शब्दादिना ज्ञायेते तदा तादृशशाब्दज्ञानादावतिव्याप्तिः, तत्र धर्माद्यवच्छिन्नतद्वृत्त्यवच्छिन्नचैतन्ययोरेकत्वादिति चेत् न~। योग्यत्वस्यापि विषयविशेषणत्वात्~। अन्तःकरणधर्मत्वाविशेषेऽपि किञ्चिद्योग्यं किञ्चिदयोग्यमित्यत्र फलबलकल्प्यः स्वभाव एव शरणम्~। अन्यथा न्यायमतेऽप्यात्मधर्मत्वाविशेषात् सुखादिवत् धर्मादेरपि प्रत्यक्षत्वापत्तिर्दुर्वारा~।\par
	न चैवमपि सुखस्य वर्तमानतादशायां "त्वं सुखी" इत्यादिवाक्यजन्यज्ञानस्य प्रत्यक्षता स्यादिति वाच्यम् इष्टत्वात्~। "दशमस्त्वमसि" इत्यादौ सन्निकृष्टविषये शब्दादप्यपरोक्षज्ञानाभ्युपगमात्~। अत एव "पर्वतो वह्निमान्" इत्यादिज्ञानमपि वह्न्यंशे परोक्षम्~, पर्वतांशेऽपरोक्षम्~, पर्वताद्यवच्छिन्नचैतन्यस्य बहिर्निःसृतान्तःकरणवृत्त्यवच्छिन्नचैतन्याभेदात्~। वह्न्यंशे तु अन्तःकरणवृत्तिनिर्गमनाभावेन वह्न्यवच्छिन्नचैतन्यस्य प्रमाणचैतन्यस्य च परस्परं भेदात्~। तथा चानुभवः "पर्वतं पश्यामि, वह्निमनुमिनोमि" इति~। न्यायमते तु "पर्वतमनुमिनोमि" इत्यनुव्यवसायापत्तिः~।\par असन्निकृष्टपक्षकानुमितौ तु सर्वांशेऽपि ज्ञानं परोक्षम्~। "सुरभि चन्दनम्" इत्यादिज्ञानमपि चन्दनखण्डांशेऽपरोक्षम् सौरभांशे च परोक्षम् सौरभस्य चक्षुरिन्द्रियायोग्यतया योग्यत्वघटितस्य निरुक्तलक्षणस्याभावात्~।\par
	न चैवमेकत्र ज्ञाने परोक्षत्वापरोक्षत्वयोरभ्युपगमे तयोर्जातित्वं न स्यादिति वाच्यम्~, इष्टत्वात्~। जातित्वोपाधित्वपरिभाषायाः सकलप्रमाणागोचरतयाऽप्रामाणिकत्वात्~। "घटोऽयम्" इत्यादिप्रत्यक्षं हि घटत्वादिसद्भावे मानम्~, न तु तस्य जातित्वेऽपि~। जातित्वरूपसाध्याप्रसिद्धौ तत्साधकानुमानस्याप्यनवकाशात्~। समवायासिद्ध्या ब्रह्मभिन्ननिखिलप्रपञ्चस्यानित्यतया च नित्यत्वसमवेतत्वघटितजातित्वस्य घटत्वादावसिद्धेश्च~। एवमेवोपाधित्वं निरसनीयम्~। "पर्वतो वह्निमान्" इत्यादौ च पर्वतांशे वह्न्यंशे चान्तःकरणवृत्तिभेदाङ्गीकारेण तत्तद्वृत्त्यवच्छेदकभेदेन परोक्षत्वापरोक्षत्वयोरेकत्र चैतन्ये वृत्तौ न कश्चित् विरोधः~। तथा च {\bfseries तत्तदिन्द्रिययोग्यवर्तमानविषयावच्छिन्नचैतन्याभिन्नत्वं तत्तदाकारवृत्त्यवच्छिन्नज्ञानस्य तत्तदंशे प्रत्यक्षत्वम्}~।\par
	घटादेर्विषयस्य प्रत्यक्षत्वन्तु प्रमात्रभिन्नत्वम्~। ननु कथं घटादेरन्तःकरणावच्छिन्नचैतन्याभेदः "अहमिदं पश्यामि" इति भेदानुभवविरोधादिति चेत् उच्यते~। प्रमात्रभेदो नाम न तावदैक्यम् किन्तु प्रमातृसत्तातिरिक्तसत्ताकत्वाभावः~। तथा च घटादेः स्वावच्छिन्नचैतन्येऽध्यस्ततया विषयचैतन्यसत्तैव घटादिसत्ता, अधिष्ठानसत्तातिरिक्ताया आरोपितसत्ताया अनङ्गीकारात्~। विषयचैतन्यं च पूर्वोक्तप्रकारेण प्रमातृचैतन्यमेवेति प्रमातृचैतन्यस्यैव घटाद्यधिष्ठानतया प्रमातृसत्तैव घटादिसत्ता नान्येति सिद्धं घटादेरपरोक्षत्वम्~। अनुमित्यादिस्थले त्वन्तःकरणस्य वह्न्यादिदेशनिर्गमनाभावेन वह्न्यवच्छिन्नचैतन्यस्य प्रमातृचैतन्यानात्मकतया वह्न्यादिसत्ता प्रमातृसत्तातो भिन्ना इति नातिव्याप्तिः~।\par
	नन्वेवमपि धर्माधर्मादिगोचरानुमित्यादिस्थले धर्माधर्मयोः प्रत्यक्षत्वापत्तिः~। धर्माद्यवच्छिन्नचैतन्यस्य प्रमातृचैतन्याभिन्नतया, धर्मादिसत्तायाः प्रमातृसत्तानतिरेकादिति चेत् न, योग्यत्वस्यापि विषयविशेषणत्वात्~। \par
	नन्वेवमपि "रूपी घटः" इति प्रत्यक्षस्थले घटगतपरिमाणादेः प्रत्यक्षत्वापत्तिः, रूपावच्छिन्नचैतन्यस्य परिमाणाद्यवच्छिन्नचैतन्यस्य चैकतया, रूपावच्छिन्नचैतन्यस्य प्रमातृचैतन्याभेदे परिमाणाद्यवच्छिन्नचैतन्यस्यापि प्रमात्रभिन्नतया परिमाणादिसत्तायाः प्रमातृसत्तातिरिक्त्तत्वाभावात् इति चेत् न~। तत्तदाकारवृत्त्युपहितत्वस्यापि प्रमातृविशेषणत्वात्~। रूपाकारवृत्तिदशायां परिमाणाद्याकारवृत्त्यभावेन अतिव्याप्त्यभावात्~। \par
	नन्वेवं वृत्तावव्याप्तिः~। अनवस्थाभिया वृत्तिगोचरवृत्त्यनङ्गीकारेण, तत्र स्वाकारवृत्त्युपहितत्वघटितोक्तलक्षणाभावात् इति चेत् न~। अनवस्थाभिया वृत्तेर्वृत्त्यन्तराविषयत्वेऽपि स्वविषयत्वाभ्युपगमेन स्वविषयवृत्त्युपहितप्रमातृचैतन्याभिन्नसत्ताकत्वस्य तत्रापि भावात्~।\par
	एवाञ्चान्तःकरणतद्धर्मादीनां केवलसाक्षिविषयत्वेऽपि तत्तदाकारवृत्त्यभ्युपगमेन उक्तलक्षणस्य तत्रापि सत्त्वान्नाव्याप्तिः~। न चान्तःकरणतद्धर्मादीनां वृत्तिविषयत्वाभ्युपगमे केवलसाक्षिवेद्यत्वाभ्युपगमविरोध इति वाच्यम्~। न हि वृत्तिं विना साक्षिविषयत्वं केवलसाक्षिवेद्यत्वम्~, किन्त्विन्द्रियानुमानादिप्रमाणव्यापारमन्तरेण साक्षिविषयत्वम्~। अत एवाहङ्कारटीकायामाचार्यैरहमाकारान्तःकरणवृत्तिरङ्गीकृता~। अत एव च प्रातिभासिकरजतस्थले रजताकाराविद्यावृत्तिः साम्प्रदायिकैरङ्गीकृता~। तथा चान्तःकरणतद्धर्मादिषु केवलसाक्षिवेद्येषु वृत्त्युपहितत्वघटितलक्षणस्य सत्त्वान्नाव्याप्तिः~। तदयं निर्गलितार्थः {\bfseries स्वाकारवृत्त्युपहितप्रमातृचैतन्यसत्तातिरिक्तसत्ताकत्वशून्यत्वे सति योग्यत्वं विषयस्य प्रत्यक्षत्वम्~।} तत्र संयोगसंयुक्ततादात्म्यादीनां सन्निकार्षाणां चैतन्याभिव्यञ्जकवृत्तिजनने विनियोगः~। \par
	सा च वृत्तिश्चतुर्विधा संशयो, निश्चयो, गर्वः, स्मरणमिति~। एवंविधवृत्तिभेदेन एकमप्यन्तःकरणं मन इति, बुद्धिरिति, अहङ्कार इति, चित्तमिति चाख्यायते~। तदुक्तम् - {\bfseries मनोबुद्धिरहङ्कारश्चित्तं करणमान्तरम्~। संशयो निश्चयो गर्वः स्मरणं विषया इमे ॥} \par
	तच्च प्रत्यक्षं द्विविधम् सविकल्पकनिर्विकल्पकभेदात्~। तत्र सविकल्पकं वैशिष्ट्यावगाहि ज्ञानम्~। यथा "घटमहं जानामि" इत्यादिज्ञानम्~। निर्विकल्पकन्तु संसर्गानवगाहि ज्ञानम्~। यथा "सोऽयं देवदत्तः", {\bfseries तत्त्वमसि} इत्यादिवाक्यजन्यं ज्ञानम्~। \par
	ननु शाब्दमिदं ज्ञानम् न प्रत्यक्षम् इन्द्रियाजन्यत्वात् इति चेत् न~। नहि इन्द्रियजन्यत्वं प्रत्यक्षत्वे तन्त्रम्~, दूषितत्वात्~। किन्तु योग्यवर्तमानविषयकत्वे सति प्रमाणचैतन्यस्य विषयचैतन्याभिन्नत्वमित्युक्तम्~। तथाच "सोऽयं देवदत्तः" इति वाक्यजन्यज्ञानस्य सन्निकृष्टविषयतया बहिर्निःसृतान्तःकरणवृत्त्यभ्युपगमेन देवदत्तावच्छिन्नचैतन्यस्य वृत्त्यवच्छिन्नचैतन्याभिन्नतया "सोऽयं देवदत्तः" इति वाक्यजन्यज्ञानस्य प्रत्यक्षत्वम्~। एवं {\bfseries तत्त्वमसि} इत्यादिवाक्यजन्यज्ञानस्यापि~। तत्र प्रमातुरेव विषयतया तदुभयाभेदस्य सत्त्वात्~।\par
	ननु वाक्यजन्यज्ञानस्य पदार्थसंसर्गावगाहितया कथं निर्विकल्पकत्वम् ? उच्यते~। वाक्यजन्यज्ञानविषयत्वे हि न पदार्थसंसर्गत्वं तन्त्रम्~, अनभिमतसंसर्गस्यापि वाक्यजन्यज्ञानविषयत्वापत्तेः~। किन्तु तात्पर्यविषयत्वम्~। प्रकृते च {\bfseries सदेव सोम्येदमग्र आसीत्}\footnote{छा.उ. ६.२.१} इत्युपक्रम्य {\bfseries तत् सत्यम्~, स आत्मा, तत्त्वमसि श्वेतकेतो}\footnote{छा.उ. ६.४.१४} इत्युपसंहारेण विशुद्धे ब्रह्मणि वेदान्तानां तात्पर्यमवसितम् इति कथं तात्पर्याविषयं संसर्गमवबोधयेत्~? इदमेव {\bfseries तत्त्वमसि}\footnote{छा.उ. ६.८.७} इत्यादिवाक्यानामखण्डार्थत्वं यत् संसर्गानवगाहियथार्थज्ञानजनकत्वमिति~। तदुक्तम्- {\bfseries संसर्गासङ्गिसम्यग्धीहेतुता या गिरामियम्~। उक्ताखण्डार्थता यद्वा तत्प्रातिपदिकार्थता ॥} प्रातिपदिकार्थमात्रपरत्वं वाऽखण्डार्थत्वम् इति चतुर्थपादार्थः~।\par
	तच्च प्रत्यक्षं पुनर्द्विविधम्~, जीवसाक्षि ईश्वरसाक्षि चेति~। तत्र जीवो नाम अन्तःकरणावच्छिन्नं चैतन्यम् तत्साक्षी तु अन्तःकरणोपहितं चैतन्यम्~। अन्तःकरणस्य विशेषणत्वोपाधित्वाभ्यामनयोर्भेदः~। विशेषणञ्च कार्यान्वयि वर्तमानं व्यावर्तकम्~। उपाधिश्च कार्यानन्वयी व्यावर्तको वर्तमानश्च~। "रूपविशिष्टो घटोऽनित्यः" इत्यत्र रूपं विशेषणम्~, "कर्णशष्कुल्यवच्छिन्नं नभः श्रोत्रम्" इत्यत्र कर्णशष्कुल्युपाधिः~। अयमेवोपाधिर्नैयायिकैः परिचायक इत्युच्यते~। प्रकृते चान्तःकरणस्य जडतया विषयभासकत्वायोगेन विषयभासकचैतन्योपाधित्वम्~। अयञ्च जीवसाक्षी प्रत्यात्मं नाना एकत्वे चैत्रावगते मैत्रस्याप्यनुसन्धानप्रसङ्गः~। \par
	ईश्वरसाक्षी तु मायोपहितं चैतन्यम्~। तच्चैकम् तदुपाधिभूतमायाया एकत्वात्~। {\bfseries इन्द्रो मायाभिः पुरुरूप ईयते}\footnote{बृ.उ. २.५.१९} इत्यादिश्रुतौ मायाभिरिति बहुवचनस्य मायागतशक्तिविशेषाभिप्रायतया, मायागतसत्त्वरजस्तमोरूपगुणाभिप्रायतया वोपपत्तिः~। {\bfseries मायान्तु प्रकृतिं विद्यान्मायिनं तु महेश्वरम्~। तरत्यविद्यां विततां हृदि यस्मिन्निवेशिते~। योगी मायाममेयाय तस्मै विद्यात्मने नमः ॥ अजामेकां लोहितशुक्लकृष्णां बह्वीः प्रजाः सृजमानां सरूपाः~। अजो ह्येको जुषमाणोऽनुशेते जहात्येनां भुक्तभोगामजोऽन्यः ॥}\footnote{श्वे.उ. ४.५.} इत्यादि श्रुतिस्मृतिषु एकवचनेन लाघवानुगृहीतेन मायाया एकत्वं निश्चीयते~। ततश्च तदुपहितं चैतन्यम् ईश्वरसाक्षी~। तच्चानादि, तदुपाधेर्मायाया अनादित्वात्~। मायावच्छिन्नं चैतन्यं च परमेश्वरः~। मायाया विशेषणत्वे ईश्वरत्वम्~, उपाधित्वे साक्षित्वं इतीश्वरत्वसाक्षित्वयोर्भेदः, न तु धर्मिणोरीश्वरतत्साक्षिणोः~। स च परमेश्वर एकोऽपि स्वोपाधिभूतमायानिष्ठसत्त्वरजस्तमोगुणभेदेन ब्रह्मविष्णुमहेश्वरादिशब्दवाच्यतां भजते~। \par
	ननु ईश्वरसाक्षिणोऽनादित्वे {\bfseries तदैक्षत, बहुस्यां प्रजायेय}\footnote{छा.उ. ६.२.३} इत्यादौ सृष्टिपूर्वसमये परमेश्वरस्यागन्तुकमीक्षणमुच्यमानं कथमुपपद्यते ? उच्यते~। यथा विषयेन्द्रियसन्निकर्षादिकारणवशेन जीवोपाध्यन्तःकरणस्य वृत्तिभेदा जायन्ते, तथा सृज्यमानप्राणिकर्मवशेन परमेश्वरोपाधिभूतमायाया वृत्तिविशेषाः "इदमिदानीं स्रष्टव्यम्", "इदमिदानीं पालयितव्यम्"~, "इदमिदानीं संहर्तव्यम्" इत्याद्याकारा जायन्ते~। तासां च वृत्तीनां सादित्वात्तत्प्रतिबिम्बितचैतन्यमपि सादीत्युच्यते~। एवं साक्षिद्वैविध्येन प्रत्यक्षज्ञानद्वैविध्यम्~। \par
	प्रत्यक्षत्वं च ज्ञेयगतं, ज्ञप्तिगतं च निरूपितम्~। तत्र ज्ञप्तिगतप्रत्यक्षत्वस्य सामान्यलक्षणं चित्त्वमेव~। "पर्वतो वह्निमान्" इत्यादावपि वह्न्याद्याकारवृत्त्युपहितचैतन्यस्य स्वात्मांशे स्वप्रकाशतया प्रत्यक्षत्वात्~। तत्तद्विषयांशे प्रत्यक्षत्वन्तु पूर्वोक्तमेव~। तस्य च भ्रान्तिरूपप्रत्यक्षे नातिव्याप्तिः, भ्रमप्रमासाधारणप्रत्यक्षत्वसामान्यनिर्वचनेन तस्यापि लक्ष्यत्वात्~। यदा तु प्रत्यक्षप्रमाया एव लक्षणं वक्तव्यं तदा पूर्वोक्तलक्षणेऽबाधितत्वं विषयविशेषणं देयम्~। शुक्तिरूप्यादिभ्रमस्य संसारकालीनबाधविषयप्रातिभासिकरजतादिविषयकत्वेनोक्तलक्षणाभावात् नातिव्याप्तिः~।\par
	ननु विसंवादिप्रवृत्त्या भ्रान्तिज्ञानसिद्धावपि तस्य प्रातिभासिकतत्कालोत्पन्नरजतादिविषयकत्वे न प्रमाणम्~, देशान्तरीयरजतस्य क्लृप्तस्यैव तद्विषयत्वसम्भवादिति चेन्न~। तस्यासन्निकृष्टतया प्रत्यक्षविषयत्वायोगात्~। न च ज्ञानं तत्र प्रत्यासत्तिः, ज्ञानस्य प्रत्यासत्तित्वे, तत एव वह्न्यादेः प्रत्यक्षत्वापत्तौ अनुमानाद्युच्छेदापत्तेः~।\par
	ननु रजतोत्पादकानां रजतावयवादीनामभावे शुक्तौ कथं तवापि रजतमुत्पद्यते इति चेत् उच्यते~। न हि लोकसिद्धसामग्री प्रातिभासिकरजतोत्पादिका किन्तु विलक्षणैव~। तथा हि- काचादिदोषदूषितलोचनस्य पुरोवर्तिद्रव्यसंयोगादिदमाकारा, चाकचिक्याकारा च काचिदन्तःकरणवृत्तिरुदेति~। तस्यां च वृत्तौ इदमवच्छिन्नचैतन्यं प्रतिबिम्बते~। तत्र पूर्वोक्तरीत्या वृत्तेर्निर्गमनेन इदमवच्छिन्नचैतन्यं, वृत्त्यवच्छिन्नचैतन्यं, प्रमातृचैतन्यं चाभिन्नं भवति~। ततश्च प्रमातृचैतन्याभिन्नविषयचैतन्यनिष्ठा शुक्तित्वप्रकारिकाऽविद्या, चाकचिक्यादिसादृश्यसन्दर्शनसमुद्बोधितरजतसंस्कारसध्रीचीना, काचादिदोषसमवहिता रजतरूपार्थाकारेण रजतज्ञानाभासाकारेण च परिणमते~।\par
	परिणामो नाम उपादानसमसत्ताककार्यापत्तिः~। विवर्तो नाम उपादानविषमसत्ताककार्यापत्तिः~। प्रातिभासिकरजतञ्चाविद्यापेक्षया परिणाम इति चैतन्यापेक्षया विवर्त इति चोच्यते~। अविद्यापरिणामरूपञ्च तद्रजतमविद्याधिष्ठाने इदमवच्छिन्नचैतन्ये वर्तते~। अस्मन्मते सर्वस्यापि कार्यस्य स्वोपादानाविद्याधिष्ठानाश्रितत्वनियमात्~।\par
	ननु चैतन्यनिष्ठरजतस्य कथमिदं रजतमितिपुरोवर्तिना तादात्म्यम्~। उच्यते यथा न्यायमते आत्मनिष्ठस्य सुखादेः शरीरनिष्ठत्वेनोपलम्भः, शरीरस्य सुखाद्यधिकरणतावच्छेदकत्वात्~, तथा चैतन्यमात्रस्य रजतं प्रत्यनधिष्ठानतया इदमवच्छिन्नचैतन्यस्य तदधिष्ठानत्वेन इदमोऽवच्छेदकतया रजतस्य पुरोवर्तिसंसर्गप्रत्यय उपपद्यते~। तस्य च विषयचैतन्यस्य तदन्तःकरणोपहितचैतन्याभिन्नतया विषयचैतन्येऽध्यस्तमपि रजतं साक्षिण्यध्यस्तं केवलसाक्षिवेद्यं सुखादिवदनन्यवेद्यमिति चोच्यते~। ननु साक्षिण्यध्यस्तत्वे "अहं रजतम्" इति "तद्वान्" इति वा प्रत्ययः स्यात्~, "अहं सुखी" इतिवत् इति चेत् उच्यते~। न हि सुखादीनामन्तःकरणावच्छिन्नचैतन्यनिष्ठाविद्याकार्यत्वप्रयुक्तम् "अहं सुखी" इति ज्ञानम्~, सुखादीनां घटादिवच्छुद्धचैतन्य एवाध्यासात्~। किन्तु यस्य यदाकारानुभवाहितसंस्कारसहकृताविद्याकार्यत्वं तस्य तदाकारानुभवविषयत्वम् इत्येवानुगतं नियामकम्~। तथा च इदमाकारानुभवाहितसंस्कारसहिताविद्याकार्यत्वाद् घटादेरिदमाकारानुभवविषयत्वम्~, अहमाकारानुभवाहितसंस्कारसहकृताविद्याकार्यत्वादन्तःकरणादेरहमाकारानुभवविषयत्वम्~, शरीरेन्द्रियादेरुभयविधानुभवसंस्कारसहिताविद्याकार्यत्वादुभयविधानुभवविषयत्वम्~। तथा चोभयविधानुभवः इदं शरीरम्~, अहं मनुष्यः, अहं ब्राह्मणः, इदं चक्षुः, अहं काणः, इदं श्रोत्रम्, अहं बधिर, इति~। प्रकृते प्रातिभासिकरजतस्य प्रमातृचैतन्याभिन्नेदमवच्छिन्नचैतन्यनिष्ठाविद्याकार्यत्वेऽपि इदं रजतमिति सत्यस्थलीयेदमाकारानुभवाहितसंस्कारजन्यत्वादिदमाकारानुभवविषयता न तु "अहं रजतम्" इत्यहमाकारानुभवविषयता इत्यनुसन्धेयम्~। \par
	नन्वेवमपि मिथ्यारजतस्य साक्षात् साक्षिसम्बन्धितया भानसम्भवे रजतगोचरज्ञानाभासरूपाया अविद्यावृत्तेरभ्युपगमः किमर्थमिति चेत् उच्यते~। स्वगोचरवृत्त्युपहितप्रमातृचैतन्यभिन्नसत्ताकत्वाभावस्य विषयापरोक्षत्वरूपतया रजतस्यापरोक्षत्वसिद्धये तदभ्युपगमात्~। नन्विदंवृत्ते रजताकारवृत्तेश्च प्रत्येकमेकैकविषयत्वे गुरुमतवद्विशिष्टज्ञानानभ्युपगमे कुतो भ्रमज्ञानसिद्धिरिति चेत् न~। वृत्तिद्वयप्रतिबिम्बितचैतन्यस्यैकस्य सत्यमिथ्यावस्तुतादात्म्यावगाहित्वेन भ्रमत्वस्वीकारात्~। अत एव साक्षिज्ञानस्य सत्यासत्यविषयतया प्रामाण्यानियमादप्रामाण्योक्तिः साम्प्रदायिकानाम्~।\par
	ननु सिद्धान्ते देशान्तरीयरजतमप्यविद्याकार्यमध्यस्तञ्चेति कथं शुक्तिरूप्यस्य ततो वैलक्षण्यम् इति चेत् न~। त्वन्मते सत्यत्वाविशेषेऽपि केषाञ्चित् क्षणिकत्वं केषाञ्चित् स्थायित्वम् इत्यत्र यदेव नियामकं तदेव स्वभावविशेषादिकं ममापि~। यद्वा घटाद्यध्यासे अविद्यैव दोषत्वेन हेतुः~। शुक्तिरूप्याद्यध्यासे तु काचादयोऽपि दोषाः~। तथा चागन्तुकदोषजन्यत्वं प्रातिभासिकत्वे प्रयोजकम्~। अत एव स्वप्नोपलब्धरथादीनामागन्तुकनिद्रादोषजन्यत्वात् प्रातिभासिकत्वम्~।\par
	ननु स्वप्नस्थले पूर्वानुभूतरथादेः स्मरणमात्रेणैव व्यवहारोपपत्तौ न रथादिसृष्टिकल्पनम्~, गौरवात् इति चेत् न~। रथादेः स्मृतिमात्राभ्युपगमे "रथं पश्यामि", "स्वप्ने रथमद्राक्षम्" इत्याद्यनुभवविरोधापत्तेः~। {\bfseries अथ रथान् रथयोगान् पथः सृजते}\footnote{बृ.उ. ४.३.१०} इति रथादिसृष्टिप्रतिपादकश्रुतिविरोधापत्तेश्च~। तस्मात् शुक्तिरूप्यवत् स्वप्नोपलब्धरथादयोऽपि प्रातिभासिका यावत्प्रतिभासमवतिष्ठन्ते~। ननु स्वप्ने रथाद्यधिष्ठानतयोपलभ्यमानदेशविशेषस्यापि तदाऽसन्निकृष्टतया अनिर्वचनीयप्रातिभासिकदेशोऽभ्युपगन्यव्यः~। तथा च रथाद्यध्यासः कुत्र इति चेत् न~। चैतन्यस्य स्वयंप्रकाशस्य रथाद्यधिष्ठानत्वात्~। प्रतीयमानं रथाद्यस्तीत्येव प्रतीयते इति सद्रूपेण प्रकाशमानं चैतन्यमेवाधिष्ठानम्~। देशविशेषोऽपि चिदध्यस्तः प्रातिभासिकः, रथादाविन्द्रियग्राह्यत्वमपि प्रातिभासिकम्~, तदा सर्वेन्द्रियाणामुपरमात्~। "अहं रथः" इत्यादिप्रतीत्यापादनन्तु पूर्ववन्निरसनीयम्~। स्वप्नरथादयः साक्षान्मायापरिणामा इति केचित्~। अन्तःकरणद्वारा तत्परिणामा इत्यन्ये~। \par
	ननु रथादेः शुद्धचैतन्याध्यस्तत्वे इदानीं तत्साक्षात्काराभावेन जागरणेऽपि स्वप्नोपलब्धरथादयोऽनुवर्तेरन्~। उच्यते~। कार्यविनाशो हि द्विविधः~। कश्चिदुपादानेन सह, कश्चित्तु विद्यमान एवोपादाने~। आद्यो बाधः, द्वितीयस्तु निवृत्तिः~। आद्यस्य कारणमधिष्ठानतत्त्वसाक्षात्कारः~। तेन विनोपादानभूताया अविद्याया अनिवृत्तेः~। द्वितीयस्य कारणं विरोधिवृत्त्युत्पत्तिर्दोषनिवृत्तिर्वा~। तदिह ब्रह्मसाक्षात्काराभावात् स्वप्नप्रपञ्चो मा बाधिष्ट, मुसलप्रहारेण घटादेरिव विरोधिप्रत्ययान्तरोदयेन स्वजनकीभूतनिद्रादिदोषनाशेन वा रथादिनिवृत्तौ को विरोधः ?\par
	एवं च शुक्तिरूप्यस्य शुक्त्यवच्छिन्नचैतन्यनिष्ठतूलाविद्याकार्यत्वपक्षे शुक्तिरितिज्ञानेन तदज्ञानेन सह रजतस्य बाधः~। मूलाविद्याकार्यत्वपक्षे तु मूलाविद्याया ब्रह्मसाक्षात्कारमात्रनिवर्त्यतया शुक्तितत्त्वज्ञानेनानिवर्त्यतया तत्र शुक्तिज्ञानान्निवृत्तिमात्रम् मुसलप्रहारेण घटस्येव~। \par
	ननु शुक्तौ रजतस्य प्रतिभाससमये सत्ताभ्युपगमे "नेदं रजतम्" इति त्रैकालिकनिषेधज्ञानं न स्यात्~। किन्तु "इदानीमिदं न रजतम्" इति स्यात्~, "इदानीं घटः श्यामो न" इतिवत् इति चेत् न~। नहि तत्र रजतत्वावच्छिन्नप्रतियोगिताकाभावो निषेधधीविषयः~। किन्तु लौकिकपारमार्थिकत्वावच्छिन्नप्रातिभासिकरजतप्रतियोगिताकः व्यधिकरणधर्मावच्छिन्नप्रतियोगिताकाभावाभ्युपगमात्~। \par
	ननु प्रातिभासिके रजते पारमार्थिकत्वमवगतं न वा~। अनवगमे प्रतियोगितावच्छेदकावच्छिन्नरजतसत्त्वज्ञानाभावादभावप्रत्यक्षानुपपत्तिः~। अवगमेऽपरोक्षावभासस्य तत्कालीनविषयसत्तानियतत्वाद् रजते पारमार्थिकत्वमप्यनिर्वचनीयं रजतवदेवोत्पन्नमिति तदवच्छिन्नरजतसत्त्वे तदवच्छिन्नाभावस्तत्र कथं वर्तत इति चेत् न~। पारमार्थिकत्वस्याधिष्ठाननिष्ठस्य रजते प्रतिभाससम्भवेन रजतनिष्ठपारमार्थिकत्वोत्पत्त्यनभ्युपगमात्~। यत्रारोप्यमसन्निकृष्टं तत्रैव प्रातिभासिकवस्तूत्पत्तेरङ्गीकारात्~। अत एव इन्द्रियसन्निकृष्टतया जपाकुसुमगतलौहित्यस्य स्फटिके भानसम्भवात् न स्फटिकेऽनिर्वचनीयलौहित्योत्पत्तिः~। नन्वेवं यत्र जपाकुसुमं द्रव्यान्तरव्यवधानादसन्निकृष्टं तत्र लौहित्यप्रतीत्या प्रातिभासिकं लौहित्यं स्वीक्रियतामिति चेत् न, इष्टत्वात्~। एवं प्रत्यक्षभ्रमान्तरेष्वपि प्रत्यक्षसामान्यलक्षणानुगमो यथार्थप्रत्यक्षलक्षणासद्भावश्च दर्शनीयः~।\par
	उक्तं प्रत्यक्षं प्रकारान्तरेण द्विविधम् इन्द्रियजन्यं तदजन्यं चेति~। तत्रेन्द्रियाजन्यं सुखादिप्रत्यक्षम्~, मनस इन्द्रियत्वनिराकरणात्~। इन्द्रियाणि पञ्च घ्राणरसनचक्षुःश्रोत्रत्वगात्मकानि~। सर्वाणि चेन्द्रियाणि स्वस्वविषयसंयुक्तान्येव प्रत्यक्षज्ञानं जनयन्ति~। तत्र घ्राणरसनत्वगिन्द्रियाणि स्वस्थानस्थितान्येव गन्धरसस्पर्शोपलम्भान् जनयन्ति~। चक्षुःश्रोत्रे तु स्वत एव विषयदेशं गत्वा स्वस्वविषयं गृह्णीतः, श्रोत्रस्यापि चक्षुरादिवत् परिच्छिन्नतया भेर्यादिदेशगमनसम्भवात्~। अत एवानुभवो "भेरीशब्दो मया श्रुतः" इति~। वीचितरङ्गादिन्यायेन कर्णशष्कुलीप्रदेशेऽनन्तशब्दोत्पत्तिकल्पनायां गौरवम्~। "भेरीशब्दो मया श्रुतः" इति प्रत्यक्षस्य भ्रमत्वकल्पनायां गौरवं च स्यात्~। तदेवं व्याख्यातं प्रत्यक्षम्~।\\
	\begin{center} इति वेदान्तपरिभषायां प्रत्यक्षपरिच्छेदः~।\end{center} 
\fancyhead[LE,RO]{अनुमानम्}
\section{अनुमानम् } \par
	अथानुमानं निरूप्यते~। अनुमितिकरणमनुमानम्~। अनुमितिश्च व्याप्तिज्ञानत्वेन व्याप्तिज्ञानजन्या~। व्याप्तिज्ञानानुव्यवसायादेस्तत्त्वेन तज्जन्यत्वाभावान्नानुमितित्वम्~। \par
	अनुमितिकरणं च व्याप्तिज्ञानम्~। तत्संस्कारोऽवान्तरव्यापारः~। न तु तृतीयलिङ्गपरामर्शोऽनुमितौ करणम्~, तस्यानुमितिहेतुत्वासिद्ध्या तत्करणत्वस्य दूरनिरस्तत्वात्~। न च संस्कारजन्यत्वेनानुमितेः स्मृतित्वापत्तिः~। स्मृतिप्रागभावजन्यत्वस्य संस्कारमात्रजन्यत्वस्य वा स्मृतित्वप्रयोजकतया संस्कारध्वंससाधारणसंस्कारजन्यत्वस्य तदप्रयोजकत्वात्~।\par
	न च यत्र व्याप्तिस्मरणादनुमितिस्तत्र कथं संस्कारो हेतुरिति वाच्यम्~। व्याप्तिस्मृतिस्थलेऽपि तत्संस्कारस्यैवानुमितिहेतुत्वात्~। नहि स्मृतेः संस्कारनाशकत्वनियमः स्मृतिधारादर्शनात्~। न चानुद्बुद्धसंस्कारादप्यनुमित्यापत्तिः, तदुद्बोधस्यापि सहकारित्वात्~। एवं च "अयं धूमवान्" इति पक्षधर्मताज्ञानेन "धूमो वह्निव्याप्यः" इत्यनुभवाहितसंस्कारोद्बोधे च सति "वह्निमान्" इत्यनुमितिर्भवति~। न तु मध्ये व्याप्तिस्मरणं, तज्जन्यं "वह्निव्याप्यधूमवानयम्" इत्यादिविशिष्टज्ञानं वा हेतुत्वेन कल्पनीयम् गौरवान्मानाभावाच्च~। तच्च व्याप्तिज्ञानं वह्निविषयकज्ञानांश एव करणम्~, न तु पर्वतविषयकज्ञानांश इति "पर्वतो वह्निमान्" इति ज्ञानस्य वह्न्यंश एव अनुमितित्वम् न पर्वतांशे, तदंशे प्रत्यक्षत्वस्योपपादितत्वात्~। \par
	व्याप्तिश्चाशेषसाधनाश्रयाश्रितसाध्यसामानाधिकरण्यरूपा~। सा च व्यभिचारादर्शने सति सहचारदर्शनेन गृह्यते~। तच्च सहचारदर्शनं भूयोदर्शनं सकृद्दर्शनं वेति विशेषो नादरणीयः, सहचारदर्शनस्यैव प्रयोजकत्वात्~।\par
	तच्चानुमानमन्वयिरूपमेकमेव~। न तु केवलान्वयि, सर्वस्यापि धर्मस्यास्मन्मते ब्रह्मनिष्ठात्यन्ताभावप्रतियोगित्वेन अत्यन्ताभावाप्रतियोगिसाध्यकत्वरूपकेवलान्वयित्वस्यासिद्धेः~। नाप्यनुमानस्य व्यतिरेकिरूपत्वम् साध्याभावे साधनाभावनिरूपितव्याप्तिज्ञानस्य साधनेन साध्यानुमितावनुपयोगात्~। कथं तर्हि धूमादावन्वयव्याप्तिमविदुषोऽपि व्यतिरेकव्याप्तिज्ञानादनुमितिः~। अर्थापत्तिप्रमाणादिति वक्ष्यामः~। अत एवानुमानस्य नान्वयव्यतिरेकिरूपत्वम्~, व्यतिरेकव्याप्तिज्ञानस्य अनुमित्यहेतुत्वात्~।\par
	तच्चानुमानं स्वार्थपरार्थभेदेन द्विविधम्~। तत्र स्वार्थन्तूक्तमेव~। परार्थन्तु न्यायसाध्यम्~। न्यायो नामावयवसमुदायः~। अवयवाश्च त्रय एव प्रतिज्ञाहेतूदाहरणरूपाः उदाहरणोपनयनिगमनरूपा वा~। न तु पञ्च अवयवत्रयेणैव व्याप्तिपक्षधर्मतयोरुपदर्शनसम्भवेनाधिकावयवद्वयस्य व्यर्थत्वात्~।\par
	एवमनुमाने निरूपिते तस्माद् ब्रह्मभिन्ननिखिलप्रपञ्चस्य मिथ्यात्वसिद्धिः~। तथाहि - ब्रह्मभिन्नं सर्वं मिथ्या, ब्रह्मभिन्नत्वात्~, यदेवं तदेवम्~, यथा शुक्तिरूप्यम्~। न च दृष्टान्तासिद्धिः तस्य साधितत्वात्~। न चाप्रयोजकत्वम् शुक्तिरूप्यरज्जुसर्पादीनां मिथ्यात्वे ब्रह्मभिन्नत्वस्यैव लाघवेन प्रयोजकत्वात्~। मिथ्यात्वं च स्वाश्रयत्वेनाभिमतयावन्निष्ठात्यन्ताभावप्रतियोगित्वम्~। 'अभिमत'पदं वस्तुतः स्वाश्रयाप्रसिद्ध्याऽसम्भववारणाय, 'यावत्'-पदमर्थान्तरवारणाय~। तदुक्तम्- {\bfseries सर्वेषामपि भावानां स्वाश्रयत्वेन सम्मते~। प्रतियोगित्वमत्यन्ताभावं प्रति मृषात्मता ॥} इति~। \par
	यद्वा अयं पट एतत्तन्तुनिष्ठात्यन्ताभावप्रतियोगी पटत्वात्~, पटान्तरवत् इत्याद्यनुमानं मिथ्यात्वे प्रमाणम्~। तदुक्तम्- {\bfseries अंशिनः स्वांशगात्यन्ताभावस्य प्रतियोगिनः~। अंशित्वादितरांशीव दिगेषैव गुणादिषु ॥} इति~।  \par
	न च घटादेर्मिथ्यात्वे "सन् घटः" इति प्रत्यक्षेण बाधः, अधिष्ठानब्रह्मसत्तायास्तत्र विषयतया, घटादेः सत्यत्वासिद्धेः~।\par
	न च नीरूपस्य ब्रह्मणः कथं चाक्षुषादिज्ञानविषयतेति वाच्यम्~, नीरूपस्यापि रूपादेः प्रत्यक्षविषयत्वात्~। न च नीरूपस्य द्रव्यस्य चक्षुराद्ययोग्यत्वमिति नियमः मन्मते ब्रह्मणो द्रव्यत्वासिद्धेः~। गुणाश्रयत्वं समवायिकारणत्वं वा द्रव्यत्वम् इति तेऽभिमतम्~। न हि निर्गुणस्य ब्रह्मणो गुणाश्रयता नापि समवायिकारणता, समवायासिद्धेः~। अस्तु वा द्रव्यत्वं ब्रह्मणः, तथापि नीरूपस्य कालस्येव चाक्षुषादिज्ञानविषयत्वे न विरोधः~।\par
	यद्वा त्रिविधं सत्त्वम् पारमार्थिकं व्यावहारिकं प्रातिभासिकं चेति~। पारमार्थिकं सत्त्वं ब्रह्मणः, व्यावहारिकं सत्त्वमाकाशादेः, प्रातिभासिकं सत्त्वं शुक्तिरजतादेः~। तथा च "घटः सन्" इति प्रत्यक्षस्य व्यावहारिकसत्त्वविषयत्वेन प्रामाण्यम्~। अस्मिन् पक्षे घटादेर्ब्रह्मणि निषेधो न स्वरूपेण, किन्तु पारमार्थिकत्वेनैवेति न विरोधः~। अस्मिन् पक्षे च मिथ्यात्वलक्षणे पारमार्थिकत्वावच्छिन्नप्रतियोगिताकत्वमत्यन्ताभावविशेषणं द्रष्टव्यम्~। तस्मादुपपन्नं मिथ्यात्वानुमानमिति~।
	\begin{center} इति वेदान्तपरिभाषायामनुमानपरिच्छेदः~।\end{center} 
\fancyhead[LE,RO]{उपमानम्}
\section{उपमानम्}\par
	अथोपमानं निरूप्यते~। तत्र सादृश्यप्रमाकरणमुपमानम्~। तथा हि नगरेषु दृष्टगोपिण्डस्य पुरुषस्य वनं गतस्य गवयेन्द्रियसन्निकर्षे सति भवति प्रतीतिः "अयं पिण्डो गोसदृशः" इति~। तदनन्तरं च भवति निश्चयः "अनेन सदृशी मदीया गौः" इति~। तत्रान्वयव्यतिरेकाभ्यां गवयनिष्ठगोसादृश्यज्ञानं करणम्~, गोनिष्ठगवयसादृश्यज्ञानं फलम्~।\par
	न चेदं प्रत्यक्षेण सम्भवति गोपिण्डस्य तदेन्द्रियासन्निकर्षात्~। नाप्यनुमानेन गवयनिष्ठगोसादृश्यस्य अतल्लिङ्गत्वात्~। नापि मदीया गौरेतद्गवयसदृशी, एतन्निष्ठसादृश्यप्रतियोगित्वात्~, यो यद्गतसादृश्यप्रतियोगी स तत्सदृशः, यथा मैत्रनिष्ठसादृश्यप्रतियोगी चैत्रो मैत्रसदृशः इत्यनुमानात् तत्सम्भव इति वाच्यम्~। एवंविधानुमानानवतारेऽपि "अनेन सदृशी मदीया गौः" इति प्रतीतेरनुभवसिद्धत्वात् "उपमिनोमि" इत्यनुव्यवसायाच्च~। तस्मादुपमानं मानान्तरम्~।\\
	\begin{center} इति वेदान्तपरिभाषायामुपमानपरिच्छेदः~।\end{center} 
\fancyhead[LE,RO]{आगमः}
\section{आगमः}\par
	अथागमो निरूप्यते~। यस्य वाक्यस्य तात्पर्यविषयीभूतसंसर्गो मानान्तरेण न बाध्यते तद्वाक्यं प्रमाणम्~। वाक्यजन्यज्ञाने च आकांक्षायोग्यताऽऽसत्तयस्तात्पर्यज्ञानं च इति चत्वारि कारणानि~।\par
	तत्र पदार्थानां परस्परजिज्ञासाविषयत्वयोग्यत्वमाकांक्षा~। क्रियाश्रवणे कारकस्य, कारकश्रवणे क्रियायाः, करणश्रवणे इतिकर्तव्यतायाश्च जिज्ञासाविषयत्वात्~। अजिज्ञासोरपि वाक्यार्थबोधात् योग्यत्वमुपात्तम्~। तदवच्छेदकं च क्रियात्वकारकत्वादिकम् (इति नातिव्याप्तिः गौरश्व इत्यादौ)~। अभेदान्वये च समानविभक्तिकपदप्रतिपाद्यत्वं तदवच्छेदकमिति तत्त्वमस्यादिवाक्येषु नाव्याप्तिः~।\par
	एतादृशाकांक्षाभिप्रायेणैव बलाबलाधिकरणे {\bfseries सा वैश्वदेव्यामिक्षा वाजिभ्यो वाजिनम्} इत्यत्र वैश्वदेवयागस्यामिक्षान्वितत्वेन न वाजिनाकांक्षा इत्यादिव्यवहारः~। ननु तत्रापि वाजिनस्य जिज्ञासाऽविषयत्वेऽपि तद्योग्यत्वमस्त्येव, प्रदेयद्रव्यत्वस्य यागनिरूपितजिज्ञासाविषयतायोग्यतावच्छेदकत्वादिति चेत् न~। स्वसमानजातीयपदार्थान्वयबोधविरहसहकृतप्रदेयद्रव्यत्वस्यैव तदवच्छेदकत्वेन, वाजिनद्रव्यस्य स्वसमानजातीयामिक्षाद्रव्यान्वयबोधसहकृतत्वेन, तादृशावच्छेदकवत्वाभावात्~। आमिक्षायान्तु नैवम् वाजिनान्वयस्य तदानुपस्थितत्वात्~। उदाहरणान्तरेष्वपि दुर्बलत्वप्रयोजक आकांक्षाविरह एव द्रष्टव्यः~।\par
	योग्यता तात्पर्यविषयसंसर्गाबाधः~। "वह्निना सिञ्चति" इत्यादौ तादृशसंसर्गबाधान्न योग्यता~। {\bfseries स प्रजापतिरात्मनो वपामुदखिदत्}\footnote{तै.सं. २.१.१.४} इत्यादावपि तात्पर्यविषयीभूतपशुप्राशस्त्याबाधात् योग्यता~। तत्त्वमस्यादिवाक्येष्वपि वाच्याभेदबाधेऽपि लक्ष्यस्वरूपाभेदे बाधाभावात् योग्यता~।\par
	आसत्तिश्चाव्यवधानेन पदजन्यपदार्थोपस्थितिः~। मानान्तरोपस्थापितपदार्थस्यान्वयबोधाभावात् पदजन्येति~। अत एवाश्रुतस्थले तत्तत्पदाध्याहारः, 'द्वारम्' इत्यादौ 'पिधेहि' इति~। अत एव 'इषे त्वा' इत्यादिमन्त्रे 'छिनद्मि' इति पदाध्याहारः~। अत एव विकृतिषु 'सूर्याय जुष्टं निर्वपामि' इति पदप्रयोगः~।\par
	पदार्थश्च द्विविधः शक्यो लक्ष्यश्चेति~। तत्र शक्तिर्नाम पदानामर्थेषु मुख्या वृत्तिः~। यथा घटपदस्य पृथुबुध्नोदराद्याकृतिविशिष्टे वस्तुविशेषे वृत्तिः~। सा च शक्तिः पदार्थान्तरम्~, सिद्धान्ते कारणेषु कार्यानुकूलशक्तिमात्रस्य पदार्थान्तरत्वात्~। सा च तत्तत्पदजन्यपदार्थज्ञानरूपकार्यानुमेया~। तादृशशक्तिविषयत्वं शक्यत्वम्~।\par
	तच्च जातेरेव, न व्यक्तेः व्यक्तीनामानन्त्येन गुरुत्वात्~। कथं तर्हि गवादिपदाद्व्यक्तिभानमिति चेत्~, जातेर्व्यक्तिसमानसंवित्संवेद्यत्वादिति ब्रूमः~। यद्वा गवादिपदानां व्यक्तौ शक्तिः स्वरूपसती, न तु ज्ञाता, हेतुः~। जातौ तु सा ज्ञाता हेतुः~। न च व्यक्त्यंशे शक्तिज्ञानमपि कारणं गौरवात्~। जातिशक्तिमत्त्वज्ञाने सति, व्यक्तिशक्तिमत्त्वज्ञानं विना व्यक्तिधीविलम्बाभावाच्च~। अत एव न्यायमतेऽप्यन्वये शक्तिः स्वरूपसतीति सिद्धान्तः~।\par
	ज्ञायमानशक्तिविषयत्वमेव वाच्यत्वमिति जातिरेव वाच्या~। अथवा व्यक्तेर्लक्षणयाऽवगमः~। यथा "नीलो घटः" इत्यत्र नीलशब्दस्य नीलगुणविशिष्टे लक्षणा, तथा जातिवाचकस्य तद्विशिष्टे लक्षणा~। तदुक्तम् {\bfseries अनन्यलभ्यः शब्दार्थः} इति~। एवं शक्यो निरूपितः~।\par
	अथ लक्ष्यपदार्थो निरूप्यते~। तत्र लक्षणाविषयो लक्ष्यः~। लक्षणा च द्विविधा केवललक्षणा लक्षितलक्षणा चेति~। तत्र शक्यसाक्षात्सम्बन्धः केवललक्षणा~। यथा "गङ्गायां घोषः" इत्यत्र प्रवाहसाक्षात्सम्बन्धिनि तीरे गङ्गापदस्य केवललक्षणा~। यत्र शक्यपरम्परासम्बन्धेनार्थान्तरप्रतीतिस्तत्र लक्षितलक्षणा~। यथा द्विरेफपदस्य रेफद्वये शक्तस्य भ्रमरपदघटितपरम्परासम्बन्धेन मधुकरे वृत्तिः~। गौण्यपि लक्षितलक्षणैव~। यथा "सिंहो माणवकः" इत्यत्र सिंहशब्दवाच्यसम्बन्धिक्रौर्यादिसम्बन्धेन माणवकस्य प्रतीतिः~।\par
	प्रकारान्तरेण लक्षणा त्रिविधा जहल्लक्षणा, अजहल्लक्षणा, जहदजहल्लक्षणा~। तत्र शक्यमनन्तर्भाव्य यत्रार्थान्तरप्रतीतिस्तत्र जहल्लक्षणा~। यथा "विषं भुंक्ष्व" इत्यत्र स्वार्थं विहाय शत्रुगृहे भोजननिवृत्तिर्लक्ष्यते~। यत्र शक्यार्थमन्तर्भाव्यैवार्थान्तरप्रतीतिः तत्राजहल्लक्षणा यथा "शुक्लो घटः" इति~। अत्र हि शुक्लशब्दः स्वार्थं शुक्लगुणमन्तर्भाव्यैव तद्वति द्रव्ये लक्षणया वर्तते~। यत्र हि विशिष्टवाचकः शब्द एकदेशं विहाय एकदेशे वर्तते तत्र जहदजहल्लक्षणा~। यथा "सोऽयम् देवदत्तः" इति~। अत्र हि पदद्वयवाच्ययोर्विशिष्टयोरैक्यानुपपत्या पदद्वयस्य विशेष्यमात्रपरत्वम्~। यथा वा {\bfseries तत्त्वमसि} इत्यादौ तत्पद्वाच्यस्य सर्वज्ञत्वादिविशिष्टस्य त्वम्पदवाच्येनान्तःकरणविशिष्टेनैक्यायोगात् ऐक्यसिद्ध्यर्थं स्वरूपे लक्षणेति साम्प्रदायिकाः~।\par
	वयन्तु ब्रूमः "सोऽयं देवदत्तः, तत्त्वमसि" इत्यादौ विशिष्टवाचकपदानामेकदेशपरत्वेऽपि न लक्षणा, शक्त्युपस्थितविशिष्टयोः अभेदान्वयानुपपत्तौ विशेष्ययोः शक्त्युपस्थितयोरेव अभेदान्वयाविरोधात्~। यथा "घटोऽनित्यः" इत्यत्र घटपदवाच्यैकदेशघटत्वस्य अयोग्यत्वेऽपि योग्यघटव्यक्त्या सहानित्यत्वान्वयः~। यत्र पदार्थैकदेशस्य विशेषणतयोपस्थितिः तत्रैव स्वातन्त्र्येणोपस्थितये लक्षणाभ्युपगमः~। यथा "नित्यो घटः" इत्यत्र घटपदात् घटत्वस्य शक्त्या स्वातन्त्र्येणानुपस्थित्या तादृशोपस्थित्यर्थं घटपदस्य घटत्वे लक्षणा~। एवमेव {\bfseries तत्त्वमसि} इत्यादिवाक्येऽपि न लक्षणा, शक्त्या स्वातन्त्र्येणोपस्थितयोः तत्त्वम्पदार्थयोरभेदान्वये बाधकाभावात्~। अन्यथा "गेहे घटः, घटे रूपम्, घटमानय" इत्यादौ घटत्वगेहत्वादेरभिमतान्वयबोधायोग्यतया तत्रापि घटादिपदानां विशेष्यमात्रपरत्वं लक्षणयैव स्यात्~। तस्मात् {\bfseries तत्त्वमसि} इत्यादिवाक्येषु आचार्याणां लक्षणोक्तिरभ्युपगमवादेन बोध्या~।\par
	जहदजहल्लक्षणोदाहरणन्तु "काकेभ्यो दधि रक्ष्यताम्" इत्याद्येव तत्र शक्यकाकत्वपरित्यागेन अशक्यदध्युपघातकत्वपुरस्कारेणाकाकेऽपि काकशब्दप्रवृत्तेः~।\par
	लक्षणाबीजन्तु तात्पर्यानुपपत्तिरेव न तु अन्वयानुपपत्तिः~। "काकेभ्यो दधि रक्ष्यताम्" इत्यत्र अन्वयानुपपत्त्यभावात्~। "गङ्गायां घोषः" इत्यादौ तात्पर्यानुपपत्तेरपि सम्भवात्~।\par
	लक्षणा च न पदमात्रवृत्तिः, किन्तु वाक्यवृत्तिरपि~। यथा "गम्भीरायां नद्यां घोषः" इत्यत्र "गम्भीरायां नद्याम्" इति पदद्वयसमुदायस्य तीरे लक्षणा~। ननु वाक्यस्याशक्ततया कथं शक्यसम्बन्धरूपा लक्षणा ? उच्यते~। शक्त्या यत् पदेन ज्ञाप्यते तत्सम्बन्धो लक्षणा~। शक्तिज्ञाप्यश्च यथा पदार्थस्तथा वाक्यार्थोऽपीति न काचिदनुपपत्तिः~।\par
	एवमर्थवादवाक्यानां प्रशंसारूपाणां प्राशस्त्ये लक्षणा {\bfseries सोऽरोदीत्} इत्यादिनिन्दार्थवादवाक्यानां निन्दितत्वे लक्षणा~। अर्थवादगतपदानां प्राशस्त्यादिलक्षणाभ्युपगमे एकेन पदेन लक्षणया तदुपस्थितिसम्भवे पदान्तरवैयर्थ्यं स्यात्~। एवं च विध्यपेक्षितप्राशस्त्यरूपपदार्थप्रत्यायकतया अर्थवादपदसमुदायस्य पदस्थानीयतया, पदं सत् अर्थवादवाक्यं विधिवाक्येनैकवाक्यं भवति, इत्यर्थवादवाक्यानां पदैकवाक्यता~। क्व तर्हि वाक्यैकवाक्यता~? यत्र प्रत्येकं भिन्नभिन्नसंसर्गप्रतिपादकयोर्वाक्ययोराकांक्षावशेन महावाक्यार्थबोधकत्वं, तत्र वाक्यैकवाक्यता~। यथा {\bfseries दर्शपूर्णमासाभ्यां स्वर्गकामो यजेत} इत्यादिवाक्यानां {\bfseries समिधो यजति} इत्यादिवाक्यानां च परस्परापेक्षिताङ्गाङ्गिभावबोधकतया एकवाक्यता~। तदुक्तं भट्टपादैः- {\bfseries स्वार्थबोधे समाप्तानामङ्गाङ्गित्वाद्यपेक्षया~। वाक्यानामेकवाक्यत्वं पुनः संहत्य जायते ॥} इति~।\par 
	एवं द्विविधोऽपि पदार्थो निरूपितः~। तदुपस्थितिश्चासत्तिः~। सा च शाब्दबोधे हेतुः तथैवान्वयव्यतिरेकदर्शनात्~। एवं महावाक्यार्थबोधेऽवान्तरवाक्यार्थबोधो हेतुः तथैवान्वयाद्यवधारणात्~।\par
	क्रमप्राप्तं तात्पर्यं निरूप्यते~। तत्र तत्प्रतीतीच्छयोच्चरितत्वं न तात्पर्यम्~, अर्थज्ञानशून्येन पुरुषेणोच्चरिताद्वेदादर्थप्रत्ययाभावप्रसङ्गात्~। "अयमध्यापकोऽव्युत्पन्नः" इति विशेषदर्शनेन तात्पर्यभ्रमस्याप्यभावात्~। न चेश्वरीयतात्पर्यज्ञानात् तत्र शाब्दबोध इति वाच्यम्~, ईश्वरानङ्गीकर्तुरपि तद्वाक्यार्थप्रतिपत्तिदर्शनात्~। उच्यते~। तत्प्रतीतिजननयोग्यत्वं तात्पर्यम्~। "गेहे घटः" इति वाक्यं गेहे घटसंसर्गप्रतीतिजननयोग्यम्~, न तु पटसंसर्गप्रतीतिजननयोग्यमिति तद्वाक्यं घटसंसर्गपरम्~, न तु पटसंसर्गपरमित्युच्यते~।\par
	ननु "सैन्धवमानय" इत्यादिवाक्यं यदा लवणानयनप्रतीतीच्छया प्रयुक्तं, तदापि अश्वसंसर्गप्रतीतिजनने स्वरूपयोग्यतासत्त्वात् लवणपरत्वज्ञानदशायामश्वादिसंसर्गज्ञानापत्तिरिति चेत् न~। तदितरप्रतीतीच्छयानुच्चरितत्वस्यापि तात्पर्यं प्रति विशेषणत्वात्~। तथा च यद्वाक्यं यत्प्रतीतिजननस्वरूपयोग्यत्वे सति, यदन्यप्रतीतीच्छया नोच्चरितं तद्वाक्यं तत्संसर्गपरमित्युच्यते~। शुकादिवाक्ये, अव्युत्पन्नोच्चरितवेदवाक्यादौ च प्रतीतीच्छाया एवाभावेन तदन्यप्रतीतीच्छयोच्चरितत्वाभावेन लक्षणसत्त्वान्नाव्याप्तिः~। न चोभयप्रतीतीच्छयोच्चरितेऽव्याप्तिः तदन्यमात्रप्रतीतीच्छयाऽनुच्चरितत्वस्य विवक्षितत्वात्~।\par
	उक्तप्रतीतिमात्रजननयोग्यतायाश्चावच्छेदिका शक्तिः~। अस्माकं मते सर्वत्र कारणतायाः शक्तेरेवावच्छेदकत्वान्न कोऽपि दोषः~।\par
	एवं तात्पर्यस्य तत्प्रतीतिजनकत्वरूपस्य शाब्दज्ञानजनकत्वे सिद्धे, चतुर्थवर्णके तात्पर्यस्य शाब्दज्ञानहेतुत्वनिराकरणवाक्यं तत्प्रतीतीच्छयोच्चरितत्वरूपतात्पर्यनिराकरणपरम्~, अन्यथा तात्पर्यनिश्चयफलकवेदान्तविचारवैयर्थ्यप्रसङ्गात्~। केचित्तु शाब्दज्ञानत्वावच्छेदेन न तात्पर्यज्ञानं हेतुरित्येवंपरं चतुर्थवर्णकवाक्यम्~। तात्पर्यसंशयविपर्ययोत्तरशाब्दज्ञानविशेषे च तात्पर्यज्ञानं हेतुरेव~। इदं वाक्यमेतत्परमुतान्यपरमिति संशये तद्विपर्यये च, तदुत्तरवाक्यार्थविशेषनिश्चयस्य तात्पर्यनिश्चयं विनाऽनुपपत्तेरित्याहुः~।\par
	तच्च तात्पर्यं वेदे मीमांसापरिशोधितन्यायादेवावधर्यते, लोके तु प्रकरणादिना~। तत्र लौकिकवाक्यानां मानान्तरावगतार्थतयाऽनुवादकत्वम्~। वेदे तु वाक्यार्थस्यापूर्वतया नानुवादकत्वम्~। तत्र लोके वेदे च कार्यपराणामिव सिद्धर्थानामपि प्रामाण्यम्~, "पुत्रस्ते जातः" इत्यादिषु सिद्धार्थेऽपि पदानां सामर्थ्यावधारणात्~। अत एव वेदान्तवाक्यानां ब्रह्मणि प्रामाण्यम्~। यथा चैतत् तथा विषयपरिच्छेदे वक्ष्यते~।\par
	तत्र वेदानां नित्यसर्वज्ञपरमेश्वरप्रणीतत्वेन प्रामाण्यमिति नैयायिकाः~। वेदानां नित्यत्वेन निरस्तसमस्तपुंदूषणतया प्रामाण्यमित्यध्वरमीमांसकाः~। अस्माकं तु मते वेदो न नित्यः, उत्पत्तिमत्त्वात्~। उत्पत्तिमत्त्वं च {\bfseries अस्य महतो भूतस्य निःश्वसितमेतद्यदृग्वेदो यजुर्वेदः सामवेदोऽथर्ववेदः} इत्यादिश्रुतेः~।\par
	नापि वेदानां द्विक्षणावस्थायित्वम्~, "य एव वेदो देवदत्तेनाधीतः, स एव मयापि" इत्यादिप्रत्यभिज्ञाविरोधात्~। अत एव गकारादिवर्णानामपि न क्षणिकत्वम्~, "सोऽयं गकारः" इत्यादिप्रत्यभिज्ञाविरोधात्~। तथा च वर्णपदवाक्यसमुदायस्य वेदस्य वियदादिवत् सृष्टिकालीनोत्पत्तिकत्वं, प्रलयकालीनध्वंसप्रतियोगित्वं च~। न तु मध्ये वर्णानामुत्पत्तिविनाशौ, अनन्तगकारादिकल्पनायां गौरवात्~। अनुच्चारणदशायां वर्णानामनभिव्यक्तिस्तदुच्चारणरूपव्यञ्जकाभावात् न विरुध्यते, अन्धकारस्थघटानुपलम्भवत्~। "उत्पन्नो गकारः" इत्यादिप्रत्ययस्तु "सोऽयं गकारः" इत्यादिप्रत्यभिज्ञाविरोधादप्रमाणम्~। वर्णाभिव्यञ्जकध्वनिगतोत्पत्तिनिरूपितपरम्परासम्बन्धविषयत्वेन प्रमाणं वा~। तस्मान्न वेदानां क्षणिकत्वम्~।\par
	ननु क्षणिकत्वाभावेऽपि वियदादिप्रपञ्चवदुत्पत्तिमत्वेन परमेश्वरकर्तृकतया पौरुषेयत्वादपौरुषेयत्वं वेदानामिति तव सिद्धान्तो भज्येत इति चेत् न~। न हि तावत् पुरुषेण उच्चार्यमाणत्वं पौरुषेयत्वम्~, गुरुमतेऽप्यध्यापकपरम्परया पौरुषेयत्वापत्तेः~। नापि पुरुषाधीनोत्पत्तिकत्वं पौरुषेयत्वम्~, नैयायिकाभिमतपौरुषेयत्वानुमानेऽस्मदादिना सिद्धसाधनापत्तेः~। किन्तु सजातीयोच्चारणानपेक्षोच्चारणविषयत्वम्~। तथा च सर्गाद्यकाले परमेश्वरः पूर्वसर्गसिद्धवेदानुपूर्वीसमानानुपूर्वीकं वेदं विरचितवान्~, न तु तद्विजातीयं वेदमिति न सजातीयोच्चारणानपेक्षोच्चारणविषयत्वं पौरुषेयत्वं वेदस्य~। भारतादीनान्तु सजातीयोच्चारणमनपेक्ष्यैवोच्चारणमिति तेषां पौरुषेयत्वम्~। एवं पौरुषेयापौरुषेयभेदेन द्विविध आगमो निरूपितः~।\\
	\begin{center} इति वेदान्तपरिभाषायामागमपरिच्छेदः~।\end{center} 
\fancyhead[LE,RO]{अर्थापत्तिः}
\section{अर्थापत्तिः}
	इदानीमर्थापत्तिर्निरूप्यते~। तत्रोपपाद्यज्ञानेनोपपादककल्पनमर्थापत्तिः~। तत्रोपपाद्यज्ञानं करणम्~, उपपादकज्ञानं फलम्~। येन विना यदनुपपन्नं तत् तत्रोपपाद्यम्~। यस्याभावे यस्यानुपपत्तिः तत् तत्रोपपादकम्~। यथा रात्रिभोजनेन विना दिवाऽभुञ्जानस्य पीनत्वमनुपपन्नम् इति तादृशं पीनत्वमुपपाद्यम्~। यथा वा रात्रिभोजनस्याभावे तादृशपीनत्वस्यानुपपत्तिः इति रात्रिभोजनमुपपादकम्~।\par
	रात्रिभोजनकल्पनारूपायां प्रमितौ "अर्थस्य आपत्तिः - कल्पना" इति षष्ठीसमासेन अर्थापत्तिशब्दो वर्तते~। कल्पनाकरणे पीनत्वादिज्ञाने तु "अर्थस्य आपत्तिः - कल्पना यस्मात्" इति बहुव्रीहिसमासेन वर्तते, इति फलकरणयोरुभयोस्तत्पदप्रयोगः~।\par
	सा चार्थापत्तिर्द्विविधा, दृष्टार्थापत्तिः श्रुतार्थापत्तिश्चेति~। तत्र दृष्टार्थापत्तिर्यथा "इदं रजतम्" इति पुरोवर्तिनि प्रतिपन्नस्य रजतस्य "नेदं रजतम्" इति तत्रैव निषिध्यमानत्वं सत्यत्वेऽनुपपन्नम् इति रजतस्य सद्भिन्नत्वं सत्यत्वात्यन्ताभाववत्त्वं वा मिथ्यात्वं कल्पयति~। श्रुतार्थापत्तिर्यथा यत्र श्रूयमाणवाक्यस्य स्वार्थानुपपत्तिमुखेन अर्थान्तरकल्पनम्~। यथा {\bfseries तरति शोकमात्मवित्}\footnote{छा.उ. ७.१.३} इत्यत्र श्रुतस्य शोकशब्दवाच्यबन्धजातस्य ज्ञाननिवर्त्यत्वस्यान्यथानुपपत्या बन्धस्य मिथ्यात्वं कल्प्यते~। यथा वा "जीवी देवदत्तो गृहे न" इति वाक्यश्रवणानन्तरं जीविनो गृहासत्त्वं बहिःसत्त्वं कल्पयति~।\par
	श्रुतार्थापत्तिश्च द्विविधा, अभिधानानुपपत्तिः अभिहितानुपपत्तिश्च~। तत्र, यत्र वाक्यैकदेशश्रवणेऽन्वयाभिधानानुपपत्त्या अन्वयाभिधानोपयोगि पदान्तरं कल्प्यते तत्राभिधानानुपपत्तिः~। यथा 'द्वारम्' इत्यत्र 'पिधेहि' इति पदाध्याहारः यथा वा {\bfseries विश्वजिता यजेत} इत्यत्र स्वर्गकामपदाध्याहारः~।
	ननु 'द्वारम्' इत्यादावन्वयाभिधानात्पूर्वम् इदमन्वयाभिधानं पिधानोपस्थापकपदं विनाऽनुपपन्नमिति कथं ज्ञानमिति चेत् न~। अभिधानपदेन करणव्युत्पत्या तात्पर्यस्य विवक्षितत्वात्~। तथा च द्वारकर्मकपिधानक्रियासंसर्गपरत्वं पिधानोपस्थापकपदं विनाऽनुपपन्नमिति ज्ञानं तत्रापि सम्भाव्यते~।\par
	अभिहितानुपपत्तिस्तु यत्र वाक्यावगतोऽर्थोऽनुपपन्नत्वेन ज्ञातः सन्नर्थान्तरं कल्पयति तत्र द्रष्टव्या~। तथा {\bfseries स्वर्गकामो ज्योतिष्टोमेन यजेत} इत्यत्र स्वर्गसाधनत्वस्य क्षणिकयागगततयाऽवगतस्यानुपपत्त्या मध्यवर्त्यपूर्वं कल्प्यते~।
	न चेयमर्थापत्तिरनुमानेऽन्तर्भवितुमर्हति अन्वयव्याप्त्यज्ञानेनान्वयिन्यनन्तर्भावात्~। व्यतिरेकिणश्चानुमानत्वं प्रागेव निरस्तम्~। अत एवार्थापत्तिस्थले 'अनुमिनोमि' इति नानुव्यवसायः~। किन्तु "अनेन इदं कल्पयामि" इति~।\par
	ननु अर्थापत्तिस्थले "इदमनेन विनाऽनुपपन्नमिति ज्ञानं करणम्" इत्युक्तम्~। तत्र किमिदं "तेन विनाऽनुपपन्नत्वम्"? तदभावव्यपकाभावप्रतियोगित्वमिति ब्रूमः~। एवमर्थापत्तेर्मानान्तरत्वसिद्धौ व्यतिरेकि नानुमानान्तरम्~, "पृथिवीतरेभ्यो भिद्यते" इत्यादौ गन्धवत्त्वमितरभेदं विनाऽनुपपन्नमित्यादिज्ञानस्य करणत्वात्~। अत एवानुव्यवसायः "पृथिव्यामितरभेदं कल्पयामि" इति~।\\
	\begin{center} इति वेदान्तपरिभाषायामर्थापत्तिपरिच्छेदः~।\end{center}
\fancyhead[LE,RO]{अनुपलब्धिः}
\section{अनुपलब्धिः}
	इदानीं षष्ठं प्रमाणं निरूप्यते~। ज्ञानकरणाजन्य-अभावानुभव-असाधारणकारणम् अनुपलब्धिरूपं प्रमाणम्~। अनुमानादिजन्य-अतीन्द्रियाभावानुभवहेतौ अनुमानादावतिव्याप्तिवारणाय अजन्यान्तं पदम्~। अदृष्टादौ साधारणकारणेऽतिव्याप्तिवारणायासाधारणेति~। अभावस्मृत्यसाधारणहेतुसंस्कारेऽतिव्याप्तिवारणायानुभवेति विशेषणम्~। न चातीन्द्रियाभावानुमितिस्थलेऽप्यनुपलब्ध्यैवाभावो गृह्यताम्~, विशेषाभावादिति वाच्यम्~। धर्माधर्माद्यनुपलब्धिसत्त्वेऽपि तदभावानिश्चयेन योग्यानुपलब्धेरेवाभावग्राहकत्वात्~।\par
	ननु केयं योग्यानुपलब्धिः ? किं योग्यस्य प्रतियोगिनोऽनुपलब्धिः ? उत योग्येऽधिकरणे प्रतियोगिनोऽनुपलब्धिः ? नाद्यः, स्तम्भे पिशाचादिभेदस्याप्रत्यक्षत्वापत्तेः~। नान्त्यः, आत्मनि धर्माद्यभावस्यापि प्रत्यक्षत्वापत्तेरिति चेत् न~। "योग्या चासावनुपलब्धिश्चेति" कर्मधारयाश्रयणात्~। अनुपलब्धेर्योग्यता च तर्कितप्रतियोगिसत्त्वप्रसञ्जितप्रतियोगिकत्वम्~। यस्याभावो गृह्यते, तस्य यः प्रतियोगी, तस्य सत्त्वेन - अधिकरणे तर्कितेन, प्रसञ्जनयोग्यम् - अापादनयोग्यं प्रतियोग्युपलब्धिस्वरूपं यस्य अनुपलम्भस्य तत्त्वं - तदनुपलब्धेर्योग्यत्वमित्यर्थः~। तथाहि - स्फीतालोकवति भूतले यदि घटः स्यात्~, तदा घटोपलम्भः स्यादित्यापादनसम्भवात्~, तादृशभूतले घटाभावोऽनुपलब्धिगम्यः~। अन्धकारे तु तादृशापादनासम्भवान्नानुपलब्धिगम्यता~। अत एव स्तम्भे तादात्म्येन पिशाचसत्त्वे स्तम्भवत् प्रत्यक्षत्वापत्त्या तदभावोऽनुपलब्धिगम्यः~। आत्मनि धर्मादिसत्त्वेऽप्यस्यातीन्द्रियतया निरुक्तोपलम्भापादनासम्भवात् न धर्माद्यभावस्यानुपलब्धिगम्यत्वम्~।\par
	ननूक्तरीत्याऽधिकरणेन्द्रियसन्निकर्षस्थले अभावस्यानुपलब्धिगम्यत्वं त्वदनुमतम्~। तत्र क्लृप्तेन्द्रियमेवाभावाकारवृत्तावपि करणम्~, इन्द्रियान्वयव्यतिरेकानुविधानादिति चेत् न~। तत्प्रतियोग्यनुपलब्धेरप्यभावग्रहे हेतुत्वेन क्लृप्तत्वेन करणत्वमात्रस्य कल्पनात्~। इन्द्रियस्य चाभावेन समं सन्निकर्षाभावेनाभावग्रहाहेतुत्वात् इन्द्रियान्वयव्यतिरेकयोः अधिकरणज्ञानाद्युपक्षीणत्वेन अन्यथासिद्धेः~।\par
	ननु "भूतले घटो न" इत्याद्यभावानुभवस्थले भूतलांशे प्रत्यक्षत्वमुभयसिद्धमिति तत्र वृत्तिनिर्गमनस्यावश्यकत्वेन भूतलावच्छिन्नचैतन्यवत् तन्निष्ठघटाभावावच्छिन्नचैतन्यस्यापि प्रमात्रभिन्नतया घटाभावस्य प्रत्यक्षतैव सिद्धान्तेऽपि इति चेत्~, सत्यम्~। अभावप्रतीतेः प्रत्यक्षत्वेऽपि तत्कारणस्यानुपलब्धेर्मानान्तरत्वात्~। न हि फलीभूतज्ञानस्य प्रत्यक्षत्वे तत्करणस्य प्रत्यक्षप्रमाणतानियतत्वमस्ति~। {\bfseries दशमस्त्वमसि} इत्यादिवाक्यजन्यज्ञानस्य प्रत्यक्षत्वेऽपि तत्करणस्य वाक्यस्य प्रत्यक्षप्रमाणभिन्नप्रमाणत्वाभ्युपगमात्~।\par
	फलवैजात्यं विना कथं प्रमाणभेद इति चेत् न~। वृत्तिवैजात्यमात्रेण प्रमाणवैजात्योपपत्तेः~। तथा च घटाद्यभावाकारवृत्तिर्नेन्द्रियजन्या, इन्द्रियस्य विषयेणासन्निकर्षात्~, किन्तु घटाद्यनुपलब्धिरूपमानान्तरजन्येति भवत्यनुपलब्धेर्मानान्तरत्वम्~।\par
	ननु अनुपलब्धिरूपमानान्तरपक्षेऽभावप्रतीतेः प्रत्यक्षत्वे घटवति घटाभावभ्रमस्यापि प्रत्यक्षत्वापत्तौ तत्राप्यनिर्वचनीयघटाभावोऽभ्युपगम्येत~। न चेष्टापत्तिः, तस्य मायोपादानकत्वेऽभावत्वानुपपत्तेः~। मायोपादानकत्वाभावे मायायाः सफलकार्योपादानत्वानुपपत्तिरिति चेत् न~। घटवति घटाभावभ्रमो न तत्कालोत्पन्नघटाभावविषयकः, किन्तु भूतलरूपादौ विद्यमानो लौकिको घटाभावो भूतले आरोप्यते इत्यन्यथाख्यातिरेव, आरोप्यसन्निकर्षस्थले सर्वत्रान्यथाख्यातेरेव व्यवस्थापनात्~।\par
	अस्तु वा प्रतियोगिमति तदभावभ्रमस्थले तदभावस्यानिर्वचनीयत्वम्~, तथापि तदुपादानं मायैव~। न ह्युपादानोपादेययोरत्यन्तसाजात्यम्~, तन्तुपटयोरपि तन्तुत्वपटत्वादिना वैजात्यात्~। यत्किञ्चित्साजात्यस्य मायाया अनिर्वचनीयघटाभावस्य च मिथ्यात्वधर्मस्य विद्यमानत्वात्~। अन्यथा व्यावहारिकं घटाभावं प्रति कथं मायोपादानमिति कुतो नाशङ्केथाः~। न च विजातीययोरप्युपादानोपादेयभावे ब्रह्मैव जगदुपादानं स्यादिति वाच्यम् प्रपञ्चविभ्रमाधिष्ठानत्वरूपस्य तस्येष्टत्वात् परिणामित्वरूपस्योपादानत्वस्य निरवयवे ब्रह्मण्यनुपपत्तेः~। तथा च प्रपञ्चस्य परिणाम्युपादानं माया, न ब्रह्म इति सिद्धान्त इत्यलमतिप्रसङ्गेन~।\par
	स चाभावश्चतुर्विधः प्रागभावः प्रध्वंसाभावोऽत्यन्ताभावोऽन्योन्याभावश्चेति~। तत्र मृत्पिण्डादौ कारणे कार्यस्य घटादेरुत्पत्तेः पूर्वं योऽभावः स प्रागभावः~। स च भविष्यतीति प्रतीतिविषयः~। तत्रैव घटस्य मुद्गरपातानन्तरं योऽभावः स प्रध्वंसाभावः~। ध्वंसस्यापि स्वाधिकरणकपालनाशे नाश एव~। न च घटोन्मज्जनापत्तिः, घटध्वंसध्वंसस्यापि घटप्रतियोगिकध्वंसत्वात्~। अन्यथा प्रागभावध्वंसात्मकघटस्य विनाशे प्रागभावोन्मज्जनापत्तिः~। न चैवमपि यत्र ध्वंसाधिकरणं नित्यं तत्र कथं ध्वंसनाश इति वाच्यम्~। तादृशमधिकरणं यदि चैतन्यव्यतिरिक्तं तदा तस्य नित्यत्वमसिद्धम्~, ब्रह्मव्यतिरिक्तस्य सर्वस्य ब्रह्मज्ञाननिवर्त्यताया वक्ष्यमाणत्वात्~। यदि च ध्वंसाधिकरणं चैतन्यं, तदाऽसिद्धिः आरोपितप्रतियोगिकध्वंसस्याधिष्ठाने प्रतीयमानस्याधिष्ठानमात्रत्वात्~। तदुक्तम् {\bfseries अधिष्ठानावशेषो हि नाशः कल्पितवस्तुनः} इति~। एवं शुक्तिरूप्यविनाशोऽपीदमवच्छिन्नचैतन्यमेव~।\par
	यत्राधिकरणे यस्य कालत्रयेऽप्यभावः सोऽत्यन्ताभावः~। यथा वायौ रूपात्यन्ताभावः~। सोऽपि वियदादिवत् ध्वंसप्रतियोग्येव~। 'इदमिदं न' इति प्रतीतिविषयोऽन्योन्याभावः~। अयमेव विभागो, भेदः, पृथक्त्वं चेति व्यवह्रियते, भेदातिरिक्तपृथक्त्वादौ प्रमाणाभावात्~। अयं चान्योन्याभावोऽधिकरणस्य सादित्वे सादिः, यथा घटे पटभेदः~। अधिकरणस्यानादित्वेऽनादिरेव, यथा जीवे ब्रह्मभेदः, ब्रह्मणि वा जीवभेदः~। द्विविधोऽपि भेदो ध्वंसप्रतियोग्येव अविद्यानिवृत्तौ तत्परतन्त्राणां निवृत्त्यवश्यम्भावात्~।\par
	पुनरपि भेदो द्विविधः सोपाधिको निरुपाधिकश्चेति~। तत्रोपाधिसत्ताव्याप्यसत्ताकत्वं सोपाधिकत्वम्~। तच्छून्यत्वं निरुपाधिकत्वम्~। तत्राद्यो यथा एकस्यैवाकाशस्य घटाद्युपाधिभेदेन भेदः~। यथा वा एकस्य सूर्यस्य जलभाजनभेदेन भेदः~। तथा च ब्रह्मणोऽन्तःकरणभेदाद्भेदः~। निरुपाधिकभेदो यथा घटे पटभेदः~। न च ब्रह्मण्यपि प्रपञ्चभेदाभ्युपगमेऽद्वैतविरोधः, तात्त्विकभेदानभ्युपगमेन वियदादिवदद्वैताव्याघातकत्वात्~। प्रपञ्चस्याद्वैते ब्रह्मणि कल्पितत्वाङ्गीकारात्~। तदुक्तं सुरेश्वराचार्यैः- {\bfseries अक्षमा भवतः केयं साधकत्वप्रकल्पने~। किं न पश्यसि संसारं तत्रैवाज्ञानकल्पितम् ॥} इति~। अत एव विवरणेऽविद्यानुमाने प्रागभावव्यतिरिक्तत्वविशेषणम्~, तत्त्वप्रदीपिकायामविद्यालक्षणे भावत्वविशेषणं च सङ्गच्छते~।
	एवं चतुर्विधानामभावानां योग्यानुपलब्ध्या प्रतीतिः~। तत्रानुपलब्धिर्मानान्तरम्~।\par
	एवमुक्तानां प्रमाणानां प्रामाण्यं स्वत एवोत्पद्यते ज्ञायते च~। तथा हि स्मृत्यनुभवसाधारणं संवादिप्रवृत्त्यनुकूलं तद्वति तत्प्रकारकज्ञानत्वं प्रामाण्यम्~। तच्च ज्ञानसामान्यसामग्रीप्रयोज्यं, न त्वधिकं गुणमपेक्षते प्रमामात्रेऽनुगतगुणाभावात्~। नापि प्रत्यक्षप्रमायां भूयोऽवयवेन्द्रियसन्निकर्षः, रूपादिप्रत्यक्षे आत्मप्रत्यक्षे च तदभावात्~। सत्यपि तस्मिन् "पीतः शङ्खः" इति प्रत्यक्षस्य भ्रमत्वाच्च~। अत एव न सल्लिङ्गपरामर्शादिकमप्यनुमित्यादिप्रमायां गुणः, असल्लिङ्गपरामर्शादिस्थलेऽपि विषयाबाधेनानुमित्यादेः प्रमात्वात्~। न चैवमप्रमापि प्रमा स्यात्~, ज्ञानसामान्यसामग्र्या अविशेषादिति वाच्यम्~। दोषाभावस्यापि हेतुत्वाङ्गीकारात्~। न चैवं परतस्त्वम्~, आगन्तुकभावकारणापेक्षायामेव परतस्त्वात्~।\par
	ज्ञायते च प्रामाण्यं स्वतः~। स्वतो ग्राह्यत्वं च दोषाभावे सति यावत्स्वाश्रयग्राहकसामग्रीग्राह्यत्वम्~। स्वाश्रयो वृत्तिज्ञानम्~, तद्ग्राहकं साक्षिज्ञानम्~। तेनापि वृत्तिज्ञाने गृह्यमाणे तद्गतं प्रामाण्यमपि गृह्यते~। न चैवं प्रामाण्यसंशयानुपपत्तिः तत्र संशयानुरोधेन दोषस्यापि सत्त्वेन दोषाभावघटितस्वाश्रयग्राहकाभावेन तत्र प्रामाण्यस्यैवाग्रहात्~। यद्वा, यावत्स्वाश्रयग्राहकग्राह्यत्वयोग्यत्वं स्वतस्त्वम्~। संशयस्थले प्रामाण्यस्योक्तयोग्यतासत्त्वेऽपि दोषवशेनाग्रहात् न संशयानुपपत्तिः~।
	अप्रामाण्यन्तु न ज्ञानसामान्यसामग्रीप्रयोज्यम् प्रमायामप्यप्रामाण्यापत्तेः, किन्तु दोषप्रयोज्यम्~। नाप्यप्रामाण्यं यावत्स्वाश्रयग्राहकग्राह्यम्~, अप्रामाण्यघटकतदभाववत्त्वादेर्वृत्तिज्ञानानुपनीतत्वेन साक्षिणा ग्रहीतुमशक्यत्वात्~। किन्तु विसंवादिप्रवृत्त्यादिलिङ्गकानुमित्यादिविषय इति परत एवाप्रामाण्यमुत्पद्यते ज्ञायते च~।\\
	\begin{center} इति वेदान्तपरिभाषायामनुपलब्धिपरिच्छेदः~।\end{center} 
\fancyhead[LE,RO]{विषयः}
\section{विषयः}
	एवं निरूपितानां प्रमाणानां प्रामाण्यं द्विविधम् - व्यावहारिकतत्त्वावेदकत्वं पारमार्थिकतत्त्वावेदकत्वं चेति~। तत्र ब्रह्मस्वरूपावगाहिप्रमाणव्यतिरिक्तानां सर्वप्रमाणानामाद्यं प्रामाण्यम्~, तद्विषयाणां व्यवहारदशायां बाधाभावात्~। द्वितीयन्तु जीवब्रह्मैक्यपराणां {\bfseries सदेव सोम्येदमग्र आसीत्}\footnote{छा.उ. ६.२.१.} इत्यादीनां {\bfseries तत्त्वमसि}\footnote{छा.उ. ६.८.७.} इत्यन्तानाम्~, तद्विषयस्य जीवपरैक्यस्य कालत्रयाबाध्यत्वात्~। तच्चैक्यं तत्त्वम्पदार्थज्ञानाधीनज्ञानमिति प्रथमं तत्-पदार्थो लक्षणप्रमाणाभ्यां निरूप्यते~।\par
	तत्र लक्षणं द्विविधम्~, स्वरूपलक्षणं तटस्थलक्षणं चेति~। तत्र स्वरूपमेव लक्षणं स्वरूपलक्षणम्~। यथा सत्यादिकं ब्रह्मस्वरूपलक्षणम्~, {\bfseries सत्यं ज्ञानमनन्तं ब्रह्म\footnote{तै.उ. २.१.१.}, आनन्दो ब्रह्मेति व्यजानात्}\footnote{तै.उ. ३.६.१.} इत्यादिश्रुतेः~। ननु स्वस्य स्ववृत्तित्वाभावे कथं लक्षणत्वमिति चेत् न~। स्वस्यैव स्वापेक्षया धर्मिधर्मभावकल्पनया लक्ष्यलक्षणत्वसम्भवात्~। तदुक्तम्- {\bfseries आनन्दो विषयानुभवो नित्यत्वं चेति सन्ति धर्माः, अपृथक्त्वेऽपि चैतन्यात् पृथगिवावभासन्ते} इति~।\par
	तटस्थलक्षणं नाम यावल्लक्ष्यकालमनवस्थितत्वे सति यद्व्यावर्तकं तदेव~। यथा गन्धवत्वं पृथिवीलक्षणम्~। महाप्रलये परमाणुषु, उत्पत्तिकाले घटादिषु च गन्धाभावात्~। प्रकृते च जगज्जन्मादिकारणत्वम्~। अत्र जगत्पदेन कार्यजातं विवक्षितम्~। कारणत्वं च कर्तृत्वम्~, अतोऽविद्यादौ नातिव्याप्तिः~। कर्तृत्वं च तत्तदुपादानगोचरापरोक्षज्ञानचिकिर्षाकृतिमत्वम्~। ईश्वरस्य तावदुपादानगोचरापरोक्षज्ञानसद्भावे- {\bfseries यः सर्वज्ञः सर्ववित् यस्य ज्ञानमयं तपः~। तस्मादेतद्ब्रह्म नाम रूपमन्नं च जायते ॥}\footnote{मु.उ. १.१.९.} इत्यादिश्रुतिर्मानम्~। तादृशचिकीर्षासद्भावे च {\bfseries सोऽकामयत, बहु स्यां प्रजायेय}\footnote{तै.उ. २.६.१.} इत्यादिश्रुतिर्मानम्~। तादृशकृतौ च {\bfseries तन्मनोऽकुरुत}\footnote{बृ.उ. १.२.१.} इत्यादिवाक्यम्~।\par
	ज्ञानेच्छाद्यन्यतमगर्भं लक्षणत्रितयं विवक्षितम्~, अन्यथा व्यर्थविशेषणापत्तेः~। अत एव जन्मस्थितिध्वंसानामन्यतमस्यैव लक्षणे प्रवेशः~। एवं च लक्षणानि नव सम्पद्यन्ते~। ब्रह्मणो जगज्जन्मादिकरणत्वे च {\bfseries यतो वा इमानि भूतानि जायन्ते, येन जातानि जीवन्ति, यत् प्रयन्त्यभिसंविशन्ति}\footnote{तै.उ. ३.१.१.} इत्यादिश्रुतिर्मानम्~।\par
	यद्वा निखिलजगदुपादानत्वं ब्रह्मणो लक्षणम्~। उपादानत्वं च जगदध्यासाधिष्ठानत्वम्~, जगदाकारेण परिणममानमायाधिष्ठानत्वं वा~। एतादृशमेवोपादानत्वमभिप्रेत्य {\bfseries इदं सर्वं यदयमात्मा\footnote{बृ.उ. २.४.६.}, सच्च त्यच्चाभवत्\footnote{तै.उ. २.६.१.}, बहु स्यां प्रजायेय\footnote{तै.उ. २.६.१.}} इत्यादिश्रुतिषु ब्रह्मप्रपञ्चयोस्तादात्म्यव्यपदेशः~। "घटः सन्" , "घटो भाति", "घट इष्टः" इत्यादिलौकिकव्यपदेशोऽपि सच्चिदानन्दरूपब्रह्मैक्याध्यासात्~।\par
	नन्वानन्दात्मकचिदध्यासाद्घटादेरिष्टत्वव्यवहारे, दुःखस्यापि तत्राध्यासात् तस्यापीष्टत्वव्यवहारापत्तिरिति चेत् न~। "आरोपे सति निमित्तानुसरणम्~, न तु निमित्तमस्तीत्यारोपः" इत्यभ्युपगमेन, दुःखादौ सच्चिदंशाध्यासेऽपि आनन्दांशाध्यासाभावात्~। जगति नामरूपांशद्वयव्यवहारस्तु अविद्यापरिणामात्मकनामरूपसम्बन्धात्~। तदुक्तम्- {\bfseries अस्ति भाति प्रियं रूपं नाम चेत्यंशपञ्चकम्~। आद्यं त्रयं ब्रह्मरूपं जगद्रूपं ततो द्वयम् ॥} इति ॥\par
	अथ जगतो जन्मक्रमो निरूप्यते~। तत्र सर्गाद्यकाले परमेश्वरः सृज्यमानप्रपञ्चवैचित्र्यहेतुप्राणिकर्मसहकृतोऽपरिमितानिरूपितशक्तिविशेषविशिष्टमायासहितः सन् नामरूपात्मकनिखिलप्रपञ्चं प्रथमं बुद्धावाकलय्य, "इदं करिष्यामि" इति सङ्कल्पयति, {\bfseries तदैक्षत बहु स्यां प्रजायेय}\footnote{छा.उ. ६.२.३.}, {\bfseries सोऽकामयत बहु स्यां प्रजायेय}\footnote{तै.उ. २.६.१.} इत्यादिश्रुतेः~। तत आकाशादीनि पञ्चभूतान्यपञ्चीकृतानि तन्मात्रपदप्रतिपाद्यान्युत्पद्यन्ते~। तत्राकाशस्य शब्दो गुणः, वायोस्तु शब्दस्पर्शौ, तेजसस्तु शब्दस्पर्शरूपाणि, अपां तु शब्दस्पर्शरूपरसाः, पृथिव्यास्तु शब्दस्पर्शरूपरसगन्धाः~। न तु शब्दस्याकाशमात्रगुणत्वम्~, वाय्वादावपि तदुपलम्भात्~। न चासौ भ्रमः बाधकाभावात्~।\par
	इमानि भूतानि त्रिगुणमायाकार्याणि त्रिगुणानि~। गुणाः सत्त्वरजस्तमांसि~। एतैश्च सत्त्वगुणोपेतैः पञ्चभूतैर्व्यस्तैर्यथाक्रमं श्रोत्रत्वक्चक्षूरसनघ्राणानि पञ्च ज्ञानेन्द्रियाणि जायन्ते~। एतैरेव सत्त्वगुणोपेतैः पञ्चभूतैर्मिलितैर्मनोबुद्ध्यहङ्कारचित्तनि जायन्ते~। श्रोत्रादीनां पञ्चानां क्रमेण दिग्वातार्कवरुणाश्विनोऽधिष्ठातृदेवताः~। मन आदीनां चतुर्णां क्रमेण चन्द्रचतुर्मुखशङ्कराच्युता अधिष्ठातृदेवताः~।\par
	एतैरेव रजोगुणोपेतैः पञ्चभूतैर्यथाक्रमं वाक्पाणिपादपायूपस्थाख्यानि कर्मेन्द्रियाणि जायन्ते~। तेषां च क्रमेण वह्नीन्द्रोपेन्द्रमृत्युप्रजापतयोऽधिष्ठातृदेवताः~। रजोगुणोपेतैः पञ्चभूतैरेव मिलितैः पञ्च वायवः प्राणापनव्यानोदानसमानाख्या जायन्ते~। तत्र प्राग्गमनवान् वायुः प्राणो नासादिस्थानवर्ती~। अर्वाग्गमनवानपानः पाय्वादिस्थानवर्ती~। विष्वगमनवान् व्यानः अखिलशरीरवर्ती~। ऊर्ध्वगमनवानुत्क्रमणवायुरुदानः कण्ठस्थानवर्ती~। अशितपीतान्नादिसमीकरणकरः समानः नाभिस्थानवर्ती~।\par
	तैरेव तमोगुणोपेतैरपञ्चीकृतभूतैः पञ्चीकृतभूतानि जायन्ते~। {\bfseries तासां त्रिवृतं त्रिवृतमेकैकामकरोत्}\footnote{छा.उ. ६.३.३.} इति श्रुतेः पञ्चीकरणोपलक्षणार्थत्वात्~।\par
	पञ्चीकरणप्रकारश्चेत्थम् आकाशमादौ द्विधा विभज्य तयोरेकं भागं पुनश्चतुर्धा विभज्य तेषां चतुर्णामंशानां वाय्वादिषु चतुर्षु भूतेषु संयोजनम्~। एवं वायुं द्विधा विभज्य तयोरेकं भागं पुनश्चतुर्धा विभज्य तेषां चतुर्णामांशानामाकाशादिषु संयोजनम्~। एवं तेजआदीनामपि~। तदेवमेकैकभूतस्यार्धं स्वांशात्मकम् अर्धान्तरं चतुर्विधभूतमयमिति पृथिव्यादिषु स्वांशाधिक्यात् पृथिव्यादिव्यवहारः~। तदुक्तम्- {\bfseries वैशेष्यात्तु तद्वादस्तद्वादः}\footnote{ब्र.सू. २.४.२२.} इति~।\par
	पूर्वोक्तैरपञ्चीकृतभूतैर्लिङ्गशरीरं परलोकयात्रानिर्वाहकं मोक्षपर्यन्तस्थायि मनोबुद्धिभ्यामुपेतं ज्ञानेन्द्रियपञ्चक-कर्मेन्द्रियपञ्चक-प्राणदिपञ्चकसंयुक्तं जायते~। तदुक्तम्- {\bfseries पञ्चप्राणमनोबुद्धिदशेन्द्रियसमन्वितम्~। अपञ्चीकृतभूतोत्थं सूक्ष्माङ्गं भोगसाधनम् ॥} इति~। तच्च द्विविधम् परमपरं च~। परं हिरण्यगर्भलिङ्गशरीरम्~,अपरमस्मदादिलिङ्गशरीरम्~। तत्र हिरण्यगर्भलिङ्गशरीरम् महत्तत्त्वम्~,अस्मदादिलिङ्गशरीरं चाहङ्कार इत्याख्यायते~।\par
	एवं तमोगुणयुक्तेभ्यः पञ्चीकृतभूतेभ्यो भूम्यन्तरिक्षस्वर्महर्जनतपःसत्यात्मकस्योर्ध्वलोकसप्तकस्य अतलवितलसुतलतलातलरसातलमहातलपातालाख्यस्य अधोलोकसप्तकस्य ब्रह्माण्डस्य जरायुजाण्डजस्वेदजोद्भिज्जाख्यचतुर्विधस्थूलशरीराणां चोत्पत्तिः~। तत्र जरायुजानि जरायुभ्यो जातानि मनुष्यपश्वादिशरीराणि~। अण्डजान्यण्डेभ्यो जातानि पक्षिपन्नगादिशरीराणि~। स्वेदजानि स्वेदाज्जातानि यूकमशकादिशरीराणि~। उद्भिज्जानि भूमिमुद्भिद्य जातानि वृक्षादीनि~। वृक्षादीनामपि पापफलभोगायतनत्वेन शरीरत्वम्~।\par
	तत्र परमेश्वरस्य पञ्चतन्मात्राद्युत्पत्तौ सप्तदशावयवोपेतलिङ्गशरीरोत्पत्तौ हिरण्यगर्भस्थूलशरीरोत्पत्तौ च साक्षात् कर्तृत्वम् इतरनिखिलप्रपञ्चोत्पत्तौ च हिरण्यगर्भादिद्वारा, {\bfseries हन्ताहमिमास्तिस्रो देवता अनेन जीवेनात्मनानुप्रविश्य नामरूपे व्याकरवाणि}\footnote{छा.उ. ६.३.२.} इति श्रुतेः~।\par
	हिरण्यगर्भो नाम मूर्तित्रयादन्यः प्रथमो जीवः~। {\bfseries स वै शरीरी प्रथमः स वै पुरुष उच्यते~। आदिकर्ता स भूतानां ब्रह्माग्रे समवर्तत ॥}, {\bfseries हिरण्यगर्भः समवर्तताग्रे}\footnote{ऋ.सं. १०.१२१.१.} इत्यादिश्रुतेः~। एवं भूतभौतिकसृष्टिर्निरूपिता~।\par
	इदानीं प्रलयो निरूप्यते~। प्रलयो नाम त्रैलोक्यविनाशः~। स च चतुर्विधः नित्यः प्राकृतो नैमित्तिक आत्यन्तिकश्चेति~। तत्र नित्यः प्रलयः सुषुप्तिः, तस्याः सकलकार्यप्रलयरूपत्वात्~। धर्माधर्मपूर्वसंस्काराणां च तदा कारणात्मनाऽवस्थानम्~। तेन सुप्तोत्थितस्य न सुखदुःखाद्यनुपपत्तिः, न वा स्मरणानुपपत्तिः~। न च सुषुप्तौ अन्तःकरणस्य विनाशे तदधीनप्राणादिक्रियानुपपत्तिः, वस्तुतः श्वासाद्यभावेऽपि तदुपलब्धेः पुरुषान्तरविभ्रममात्रत्वात्~, सुप्तशरीरोपलम्भवत्~। न च एवं सुप्तस्य परेतादविशेषः, सुप्तस्य हि लिङ्गशरीरं संस्कारात्मनाऽत्रैव वर्तते परेतस्य तु लोकान्तरे इति वैलक्षण्यात्~।\par
	यद्वा अन्तःकरणस्य द्वे शक्ती ज्ञानशक्तिः क्रियाशक्तिश्चेति~। तत्र ज्ञानशक्तिविशिष्टान्तःकरणस्य सुषुप्तौ विनाशः, न तु क्रियाशक्तिविशिष्टस्य इति प्राणद्यवस्थानमविरुद्धम्~। {\bfseries यदा सुप्तः स्वप्नं न कञ्चन पश्यति अथास्मिन् प्राण एवैकधा भवति अथैनं वाक् सर्वैर्नामभिः सहाप्येति}\footnote{कौ.उ. ३.३.}, {\bfseries सता सोम्य तदा सम्पन्नो भवति स्वमपीतो भवति}\footnote{छा.उ. ६.८.१.} इत्यादिश्रुतिरुक्तसुषुप्तौ मानम्~।\par
	प्राकृतप्रलयस्तु कार्यब्रह्मविनाशनिमित्तकः सकलकार्यविनाशः~। यदा तु प्रागेवोत्पन्नब्रह्मसाक्षात्कारस्य कार्यब्रह्मणो ब्रह्माण्डाधिकारलक्षणप्रारब्धकर्मसमाप्तौ विदेहकैवल्यात्मिका परा मुक्तिः, तदा तल्लोकवासिनामप्युत्पन्नब्रह्मसाक्षात्काराणां ब्रह्मणा सह विदेहकैवल्यम्~। {\bfseries ब्रह्मणा सह ते सर्वे सम्प्राप्ते प्रतिसञ्चरे~। परस्यान्ते कृतात्मानः प्रविशन्ति परं पदम् ॥} इति स्मृतेः~। एवं तल्लोकवासिभिः सह कार्यब्रह्मणि मुच्यमाने, तदधिष्ठितब्रह्माण्ड -तदन्तर्वर्तिनिखिललोक-तदन्तर्वर्तिस्थावरादीनां भौतिकानां भूतानां च प्रकृतौ मायायां लयः, न तु ब्रह्मणि, बाधरूपविनाशस्यैव ब्रह्मनिष्ठत्वात्~। अतः प्राकृत इत्युच्यते~।\par
	कार्यब्रह्मणो दिवसावसाननिमित्तकः त्रैलोक्यमात्रप्रलयः नैमित्तिकप्रलयः~। ब्रह्मदिवसश्चतुर्युगसहस्रपरिमितकालः, {\bfseries चतुर्युगसहस्राणि ब्रह्मणो दिनमुच्यते} इत्यादिवचनात्~। प्रलयकालोऽपि दिवसकालपरिमितः, रात्रिकालस्य दिवसकालतुल्यत्वात्~।\par
	प्राकृतप्रलये नैमित्तिकप्रलये च पुराणवचनानि प्रमाणानि~। {\bfseries द्विपरार्धे त्वतिक्रान्ते ब्रह्मणः परमेष्ठिनः~। तदा प्रकृतयः सप्त कल्प्यन्ते प्रलयाय हि ॥ एष प्राकृतिको राजन् प्रलयो यत्र लीयते~।} इति वचनं प्राकृतप्रलये मानम्~। {\bfseries एष नैमित्तिकः प्रोक्तः प्रलयो यत्र विश्वसृक्~। शेतेऽनन्तासने नित्यमात्मसात्कृत्य चाखिलम्~।} इति वचनं नैमित्तिके प्रलये मानम्~।\par
	तुरीयप्रलयस्तु ब्रह्मसाक्षात्कारनिमित्तकः सर्वमोक्षः~। स चैकजीववादे युगपदेव, नानाजीववादे तु क्रमेण~। {\bfseries सर्व एकीभवन्ति}\footnote{मु.उ. ३.२.७.} इत्यादिश्रुतेः~। तत्राद्यास्त्रयोऽपि लयाः कर्मोपरमनिमित्ताः तुरीयस्तु ज्ञानोदयनिमित्तो लयोऽज्ञानेन सहैवेति विशेषः~।\par
	एवं चतुर्विधप्रलयो निरूपितः~। तस्येदानीं क्रमो निरूप्यते~। भूतानां भौतिकानां च न कारणलयक्रमेण लयः कारणलयसमये कार्याणामाश्रयमन्तरेणावस्थानानुपपत्तेः, किन्तु सृष्टिक्रमविपरीतक्रमेण~। तत्तत्कार्यनाशे तत्तज्जनकादृष्टनाशस्यैव प्रयोजकतया उपादाननाशस्याप्रयोजकत्वात्~। अन्यथा न्यायमतेऽपि महाप्रलये पृथिवीपरमाणुगतरूपरसादेरविनाशापत्तेः~। तथा च पृथिव्या अप्सु, अपां तेजसि, तेजसो वायौ, वायोराकाशे, आकाशस्य जीवाहङ्कारे, तस्य हिरण्यगर्भाहङ्कारे, तस्य चाविद्यायाम् इत्येवंरूप एव प्रलयः~। तदुक्तं विष्णुपुराणे {\bfseries जगत्प्रतिष्ठा देवर्षे पृथिव्यप्सु प्रलीयते~। तेजस्यापः प्रलीयन्ते तेजो वायौ प्रलीयते ॥ वायुश्च लीयते व्योम्नि तच्चाव्यक्ते प्रलीयते~। अव्यक्तं पुरुषे ब्रह्मन्निष्कले सम्प्रलीयते} इति~। एवंविधप्रलयकारणत्वं तत्-पदार्थस्य ब्रह्मणस्तटस्थलक्षणम्~।\par
	ननु वेदान्तैर्ब्रह्मणि जगत्कारणत्वेन प्रतिपाद्यमाने सति सप्रपञ्चं स्यात्~, अन्यथा सृष्टिवाक्यानामप्रामाण्यापत्तिरिति चेत् न~। न हि सृष्टिवाक्यानां सृष्टौ तात्पर्यम्~। किन्तु अद्वये ब्रह्मण्येव~। तत्प्रतिपत्तौ कथं सृष्टेरुपयोगः ? इत्थम् - यदि सृष्टिमनुपन्यस्य प्रपञ्चस्य निषेधो ब्रह्मणि प्रतिपाद्येत, तदा ब्रह्मणि प्रतिषिद्धस्य प्रपञ्चस्य वायौ प्रतिषिद्धस्य रूपस्येव ब्रह्मणोऽन्यत्रावस्थानशङ्कायां न निर्विचिकित्समद्वितीयत्वं प्रतिपादितं स्यात्~। ततः सृष्टिवाक्याद्ब्रह्मोपादेयत्वज्ञाने सति, उपादानं विना कार्यस्यान्यत्र सद्भावशङ्कायां निरस्तायां, {\bfseries नेति नेति}\footnote{बृ.उ. २.३.६.} इत्यादिना ब्रह्मण्यपि तस्यासत्त्वोपपादनेन प्रपञ्चस्य तुच्छत्वावगमे, निरस्तनिखिलद्वैतविभ्रममखण्डं सच्चिदानन्दैकरसं ब्रह्म सिद्ध्यतीति परम्परया सृष्टिवाक्यानामपि अद्वितीये ब्रह्मण्येव तात्पर्यम्~। उपासनाप्रकरणपठितसगुणब्रह्मवाक्यानां च उपासनाविध्यपेक्षितगुणारोपमात्रपरत्वम्~, न गुणपरत्वम्~। निर्गुणप्रकरणपठितानां सगुणवाक्यानान्तु निषेधवाक्यापेक्षितनिषेध्यसमर्पकत्वेन विनियोग इति न किञ्चिदपि वाक्यमद्वितीयब्रह्मप्रतिपादनेन विरुध्यते~।\par
	तदेवं स्वरूपतटस्थलक्षणलक्षितं~। तत्-पदवाच्यमीश्वरचैतन्यं मायाप्रतिबिम्बरूपमिति केचित्~। तेषामयमाशयः- जीवपरमेश्वरसाधारणं चैतन्यमात्रं बिम्बम्~, तस्यैव बिम्बस्याविद्यात्मिकायां मायायां प्रतिबिम्बमीश्वरचैतन्यम्~, अन्तःकरणेषु प्रतिबिम्बं जीवचैतन्यम् {\bfseries कार्योपाधिरयं जीवः कारणोपधिरीश्वरः} इति श्रुतेः~। एतन्मते जलाशयगतशरावजलगतसूर्यप्रतिबिम्बयोरिव जीवपरमेश्वरयोर्भेदः~। अविद्यात्मकोपाधेर्व्यापकतया तदुपाधिकेश्वरस्यापि व्यापकत्वम्~। अन्तःकरणस्य परिच्छिन्नतया तदुपाधिकजीवस्यापि परिच्छिन्नत्वम्~।\par
	एतन्मतेऽविद्याकृतदोषा जीव इव परमेश्वरेऽपि स्युः, उपाधेः प्रतिबिम्बपक्षपातित्वात्~, इत्यस्वरसात् बिम्बात्मकमीश्वरचैतन्यमित्यपरे~। तेषामयमाशयः- एकमेव चैतन्यं बिम्बत्वाक्रान्तमीश्वरचैतन्यम्~। प्रतिबिम्बत्वाक्रान्तं जीवचैतन्यम्~। बिम्बप्रतिबिम्बकल्पनोपाधिश्चैकजीववादे अविद्या, अनेकजीववादे तु अन्तःकरणान्येव~। अविद्यान्तःकरणरूपोपाधिप्रयुक्तो जीवपरभेदः~। उपाधिकृतदोषाश्च प्रतिबिम्बे जीव एव वर्तन्ते, न तु बिम्बे परमेश्वरे, उपाधेः प्रतिबिम्बपक्षपातित्वात्~। एतन्मते च गगनसूर्यस्य जलादौ भासमानप्रतिबिम्बसूर्यस्येव जीवपरयोर्भेदः~।\par
	ननु ग्रीवास्थमुखस्य दर्पणप्रदेश इव बिम्बचैतन्यस्य परमेश्वरस्य जीवप्रदेशेऽभावात् तस्य सर्वान्तर्यामित्वं न स्यादिति चेत् न~। साभ्रनक्षत्रस्याकाशस्य जलादौ प्रतिबिम्बितत्वे बिम्बभूतमहाकाशस्यापि जलादिप्रदेशसम्बन्धदर्शनेन परिच्छिन्नबिम्बस्य प्रतिबिम्बदेशासम्बन्धित्वेऽप्यपरिच्छिन्नब्रह्मबिम्बस्य प्रतिबिम्बदेशसम्बन्धाविरोधात्~। \par
	न च रूपहीनस्य ब्रह्मणो न प्रतिबिम्बसम्भवः, रूपवत एव तथात्वदर्शनात् इति वाच्यम्~। नीरूपस्यापि रूपस्य प्रतिबिम्बदर्शनात्~। न च नीरूपस्य द्रव्यस्य प्रतिबिम्बाभावनियमः~। आत्मनो द्रव्यत्वाभावस्य उक्तत्वात्~। {\bfseries एकधा बहुधा चैव दृश्यते जलचन्द्रवत्~।}, {\bfseries यथा ह्ययं ज्योतिरात्मा विवस्वानपो भिन्ना बहुधैकोऽनुगच्छन्~।} इत्यादिवाक्येन ब्रह्मप्रतिबिम्बाभावानुमानस्य बाधितत्वाच्च~। तदेवं तत्-पदार्थो निरूपितः~।\par
	इदानीं त्वम्-पदार्थो निरूप्यते~। एकजीववादेऽविद्याप्रतिबिम्बो जीवः~। अनेकजीववादे तु अन्तःकरणप्रतिबिम्बः~। स च जाग्रत्स्वप्नसुषुप्तिरूपावस्थात्रयवान्~। तत्र जाग्रद्दशा नाम इन्द्रियजन्यज्ञानावस्था~। अवस्थान्तरे इन्द्रियाभावात् नातिव्याप्तिः~। इन्द्रियजन्यज्ञानं च अन्तःकरणवृत्तिः, स्वरूपज्ञानस्यानादित्वात्~।\par
	सा चान्तःकरणवृत्तिरावरणाभिभवार्था इत्येकं मतम्~। तथा हि- अविद्योपहितचैतन्यस्य जीवत्वपक्षे घटाद्यधिष्ठानचैतन्यस्य जीवरूपतया, जीवस्य सर्वदा घटादिभानप्रसक्तौ घटाद्यवच्छिन्नचैतन्यावरकमज्ञानं मूलाविद्यापरतन्त्रमवस्थापदवाच्यमभ्युपगन्तव्यम्~। एवं सति न सर्वदा घटादेर्भानप्रसङ्गः~। अनावृतचैतन्यसम्बन्धस्यैव भानप्रयोजकत्वात्~। तस्य चावरणस्य सदातनत्वे कदाचिदपि घटभानं न स्यादिति तद्भङ्गे वक्तव्ये, तद्भङ्गजनकं न चैतन्यमात्रम्~। तद्भासकस्य तदनिवर्तकत्वात्~। नापि वृत्त्युपहितं चैतन्यम्~। परोक्षस्थलेऽपि तन्निवृत्त्यापत्तेरिति परोक्षव्यावृत्तवृत्तिविशेषस्य, तदुपहितचैतन्यस्य वा आवरणभञ्जकत्वम् - इति आवरणाभिभवार्था वृत्तिरुच्यते~।\par
	सम्बन्धार्था वृत्तिरित्यपरं मतम्~। तत्राविद्योपाधिकोऽपरिच्छिन्नो जीवः~। स च घटादिप्रदेशे विद्यमानोऽपि घटाद्याकारापरोक्षवृत्तिविरहदशायां न घटादिकमवभासयति, घटादिना समं सम्बन्धाभावात्~। तत्तदाकारवृत्तिदशायां तु भासयति, तदा सम्बन्धसत्त्वात्~।\par
	ननु अविद्योपाधिकस्यापरिच्छिन्नस्य जीवस्य स्वत एव समस्तवस्तुसम्बद्धस्य वृत्तिविरहदशायां सम्बन्धाभावाभिधानमसङ्गतम्~। असङ्गत्वदृष्ट्या सम्बन्धाभावाभिधाने च वृत्त्यनन्तरमपि सम्बन्धो न स्यात्~, इति चेत्~। उच्यते - न हि वृत्तिविरहदशायां जीवस्य घटादिना सह सम्बन्धसामान्यं निषेधामः~। किन्तर्हि ? घटादिभानप्रयोजकं सम्बन्धविशेषम्~। स च सम्बन्धविशेषो विषयस्य जीवचैतन्यस्य च व्यङ्ग्य-व्यञ्जकतालक्षणः कादाचित्कस्तत्तदाकारवृत्तिनिबन्धनः~। तथा हि तैजसमन्तःकरणं स्वच्छद्रव्यत्वात् स्वत एव जीवचैतन्याभिव्यञ्जनसमर्थम्~। घटादिकस्तु न तथा, अस्वच्छद्रव्यत्वात्~। स्वाकारवृत्तिसंयोगदशायान्तु वृत्त्यभिभूतजाड्यधर्मकतया वृत्त्युत्पादितचैतन्याभिव्यञ्जनयोग्यताश्रयतया च वृत्त्युदयानन्तरं चैतन्यमभिव्यनक्ति~। तदुक्तं विवरणे {\bfseries अन्तःकरणं हि स्वस्मिन्निव स्वसंसर्गिण्यपि घटादौ चैतन्याभिव्यक्तियोग्यतामापादयति} इति~। दृष्टं चास्वच्छद्रव्यस्यापि स्वच्छद्रव्यसम्बन्धदशायां प्रतिबिम्बग्राहित्वम्~। तथा कुड्यादेर्जलादिसंयोगदशायां मुखादिप्रतिबिम्बग्राहिता~। घटादेरभिव्यञ्जकत्वं च तत्प्रतिबिम्बग्राहित्वम्~, चैतन्यस्याभिव्यक्तत्वं च तत्र प्रतिबिम्बितत्वम्~।\par
	एवंविधाभिव्यञ्जकत्वसिद्ध्यर्थमेव वृत्तेरपरोक्षस्थले बहिर्निर्गमनाङ्गीकारः~। परोक्षस्थले तु वह्न्यादेर्वृत्तिसंसर्गाभावेन चैतन्यानभिव्यञ्जकतया नापरोक्षत्वम्~। एतन्मते च विषयाणामपरोक्षत्वं चैतन्याभिव्यञ्जकत्वमिति द्रष्टव्यम्~। एवं जीवस्यापरिच्छिन्नत्वेऽपि वृत्तेः सम्बन्धार्थत्वं निरूपितम्~।\par
	इदानीं परिच्छिन्नत्वपक्षे सम्बधार्थत्वं निरूप्यते~। तथा हि अन्तःकरणोपाधिको जीवः~। तस्य न घटाद्युपादानता, घटादिदेशासम्बन्धात्~। किन्तु ब्रह्मैव घटाद्युपादानम्~, तस्य मायोपहितस्य सकलघटाद्यन्वयित्वात्~। अत एव ब्रह्मणः सर्वज्ञता~। तथा च जीवस्य घटाद्यधिष्ठानब्रह्मचैतन्याभेदमन्तरेण घटाद्यवभासासम्भवे प्राप्ते, तदवभासाय घटाद्यधिष्ठानब्रह्मचैतन्याभेदसिद्ध्यर्थं घटाद्याकारवृत्तिरिष्यते~।\par
	ननु वृत्त्यापि कथं प्रमातृचैतन्यविषयचैतन्ययोरभेदः सम्पाद्यते, घटान्तःकरणरूपोपाधिभेदेन तदवच्छिन्नचैतन्ययोरभेदासम्भवात् इति चेत् न~। वृत्तेर्बहिर्देशनिर्गमनाङ्गीकारेण वृत्त्यन्तःकरणविषयाणामेकदेशस्थत्वेन तदुपधेयभेदाभावस्य उक्तत्वात्~। एवमपरोक्षस्थले वृत्तेर्मतभेदेन विनियोग उपपादितः~।\par
	इन्द्रियाजन्यविषयगोचरापरोक्षान्तःकरणवृत्त्यवस्था स्वप्नावस्था~। जाग्रदवस्थाव्यावृत्त्यर्थम् इन्द्रियाजन्येति~। अविद्यावृत्तिमत्यां सुषुप्तावतिव्याप्तिवारणाय अन्तःकरणेति~। सुषुप्तिर्नाम अविद्यागोचराविद्यावृत्त्यवस्था~। जाग्रत्स्वप्नयोरविद्याकारवृत्तेरन्तःकरणवृत्तित्वान्न तत्रातिव्याप्तिः~। अत्र केचिन्मरणमूर्च्छयोरवस्थान्तरत्वमाहुः~। अपरे तु सुषुप्तावेव तयोरन्तर्भावमाहुः~। तत्र तयोरवस्थात्रयान्तर्भावबहिर्भावयोः त्वं-पदार्थनिरूपणे उपयोगाभावात् न तत्र प्रयत्यते~।\par
	तस्य मायोपाध्यपेक्षया एकत्वम्~, अन्तःकरणोपाध्यपेक्षया च नानात्वं व्यवह्रियते~। एतेन जीवस्याणुत्वं प्रत्युक्तम्~। {\bfseries बुद्धेर्गुणेनात्मगुणेन चैवं ह्याराग्रमात्रो ह्यवरोऽपि दृष्टः~।} इत्यादौ जीवस्य बुद्धिशब्दवाच्यान्तःकरणपरिमाणोपाधिकस्य परिमाणत्वश्रवणात्~।\par
	स च जीवः स्वयंप्रकाशः~। स्वप्नावस्थामधिकृत्य {\bfseries अत्रायं पुरुषः स्वयं ज्योतिः} इति श्रुतेः~। अनुभवरूपश्च, {\bfseries प्रज्ञानघन एव} इत्यादिश्रुतेः~। "अनुभवामि" इति व्यवहारस्तु वृत्तिप्रतिबिम्बितचैतन्यमादाय उपपद्यते~। एवं त्वं-पदार्थो निरूपितः~।\par
	अधुना तत्-त्वम्-पदार्थयोरैक्यं महावाक्यप्रतिपाद्यमभिधीयते~। ननु "नाहमीश्वरः" इत्यादिप्रत्यक्षेण, किञ्चिज्ज्ञत्वसर्वज्ञत्वादिविरुद्धधर्माश्रयत्वादिलिङ्गेन, {\bfseries द्वा सुपर्णा} इत्यादिश्रुत्या, {\bfseries द्वाविमौ पुरुषौ लोके क्षरश्चाक्षर एव च~। क्षरः सर्वाणि भूतानि कूटस्थोऽक्षर उच्यते~॥} इत्यादिस्मृत्या च जीवपरभेदस्यावगतत्वेन तत्त्वमस्यादिवाक्यम् {\bfseries आदित्यो यूपः~।}, {\bfseries यजमानः प्रस्तरः} इत्यादिवाक्यवदुपचरितार्थमेव, इति चेत् न~। भेदप्रत्यक्षस्य सम्भावितकरणदोषस्यासम्भावितदोषवेदजन्यज्ञानेन बाध्यमानत्वात्~। अन्यथा चन्द्रगताधिकपरिमाणग्राहि-ज्योतिःशास्त्रस्य चन्द्रप्रादेशग्राहिप्रत्यक्षेण बाधापत्तेः~। पाकरक्ते घटे "रक्तोऽयम्~। न श्यामः" इतिवत् {\bfseries सविशेषणे हि} इति न्यायेन जीवपरभेदग्राहिप्रत्यक्षस्य विशेषणीभूतधर्मभेदविषयत्वाच्च~। अत एव नानुमानमपि प्रमाणम्~, आगमविरोधात्~, मेरुपाषाणमयत्वानुमानवत्~।\par
	नाप्यागमान्तरविरोधः~। तत्परातत्परवाक्ययोस्तत्परवाक्यस्य बलवत्वेन लोकसिद्धभेदानुवादि - {\bfseries द्वा सुपर्णा} - इत्यादिवाक्यापेक्षया उपक्रमोपसंहाराद्यवगताद्वैततात्पर्यविशिष्टस्य तत्त्वमस्यादिवाक्यस्य प्रबलत्वात्~। न च जीवपरैक्ये विरुद्धधर्माश्रयत्वानुपपत्तिः~। शीतस्यैव जलस्यौपाधिकौष्ण्याश्रयत्ववत् स्वभावतो निर्गुणस्यैव जीवस्यान्तःकरणाद्युपाधिककर्तृत्वाद्याश्रयत्वप्रतिभासोपपत्तेः~। यदि च जलादावौष्ण्यमारोपितम्~, तदा प्रकृतेऽपि तुल्यम्~। न च सिद्धान्ते कर्तृत्वस्य क्वचिदप्यभावादारोप्यप्रमाहितसंस्काराभावे कथमारोपः इति वाच्यम्~। लाघवेनारोप्यविषयकसंस्कारत्वेनैव तस्य हेतुत्वात्~। न च प्राथमिकारोपे का गतिः, कर्तृत्वाद्यध्यासप्रवाहस्यानादित्वात्~।\par
	तत्र तत्त्वम्-पदवाच्ययोर्विशिष्टयोरैक्यायोगेऽपि लक्ष्यस्वरूपयोरैक्यमुपपादितमेव~। अत एव तत्प्रतिपादक-तत्त्वमस्यादिवाक्यानामखण्डार्थत्वम्~, "सोऽयम्" इत्यादिवाक्यवत्~। न च कार्यपराणामेव प्रामाण्यम्~। "चैत्र, पुत्रस्ते जातः" इत्यादौ सिद्धेऽपि सङ्गतिग्रहात्~। एवं सर्वप्रमाणाविरुद्धं श्रुतिस्मृतीतिहासपुराणप्रतिपाद्यं जीवपरैक्यं वेदान्तशास्त्रस्य विषय इति सिद्धम्~।\par
	\begin{center} इति वेदान्तपरिभाषायां विषयपरिच्छेदः~।\end{center} 
\fancyhead[LE,RO]{प्रयोजनम्}
\section{प्रयोजनम्}
	इदानीं प्रयोजनं निरूप्यते~। यदवगतं सत् स्ववृत्तितया इष्यते तत्प्रयोजनम्~। तच्च द्विविधम् मुख्यं गौणं चेति~। तत्र सुखदुःखाभावौ मुख्ये प्रयोजने~। तदन्यतरसाधनं गौणं प्रयोजनम्~। सुखं च द्विविधम् सातिशयं निरतिशयं च~। तत्र सातिशयं सुखं विषयानुषङ्गजनितान्तःकरणवृत्तितारतम्यकृतानन्दलेशाविर्भावविशेषः, {\bfseries एतस्यैवानन्दस्यान्यानि भूतानि मात्रामुपजीवन्ति} इत्यादिश्रुतेः~। निरतिशयं सुखं च ब्रह्मैव, {\bfseries आनन्दो ब्रह्मेति व्यजानात्~।}, {\bfseries विज्ञानमानन्दं ब्रह्म} इत्यादिश्रुतेः~।\par
	आनन्दात्मकब्रह्मावाप्तिश्च मोक्षः, शोकनिवृत्तिश्च, {\bfseries ब्रह्म वेद ब्रह्मैव भवति~।}, {\bfseries तरति शोकमात्मवित्} इत्यादिश्रुतेः~। न तु लोकान्तरावाप्तिः, तज्जन्यवैषयिकानन्दो वा मोक्षः~। तस्य कृतकत्वेनानित्यत्वे मुक्तस्य पुनरावृत्त्यापत्तेः~। \par
	ननु त्वन्मतेऽप्यानन्दावाप्तेरनर्थनिवृत्तेश्च सादित्वे तुल्यो दोषः, अनादित्वे मोक्षमुद्दिश्य श्रवणादौ प्रवृत्त्यनुपपत्तिरिति चेत् न~। सिद्धस्यैव ब्रह्मस्वरूपस्य मोक्षस्यासिद्धत्वभ्रमेण तत्साधने प्रवृत्त्युपपत्तेः~। अनर्थनिवृत्तिरप्यधिष्ठानभूतब्रह्मस्वरूपतया सिद्धैव~। लोकेऽपि प्राप्तप्राप्ति-परिहृतपरिहारयोः प्रयोजनत्वं दृष्टमेव~। यथा हस्तगतविस्मृतसुवर्णादौ "तव हस्ते सुवर्णम्" इत्याप्तोपदेशादप्राप्तमिव प्राप्नोति~। यथा वा वलयितचरणायां स्रजि सर्पत्वभ्रमवतः "नायं सर्पः" इत्याप्तवाक्यात् परिहृतस्यैव सर्पस्य परिहारः~। एवं प्राप्तस्याप्यानन्दस्य प्राप्तिः, परिहृतस्याप्यनर्थस्य निवृत्तिर्मोक्षः प्रयोजनं च~।\par
	स च ज्ञानैकसाध्यः, {\bfseries तमेव विदित्वाऽतिमृत्युमेति~। नान्यः पन्था विद्यतेऽयनाय} इतिश्रुतेः, अज्ञाननिवृत्तेः ज्ञानैकसाध्यत्वनियमाच्च~। तच्च ज्ञानं ब्रह्मात्मैक्यगोचरम्~, {\bfseries अभयं वै जनक प्राप्तोऽसि~।}, {\bfseries तदात्मानमेवावेत् अहं ब्रह्मास्मि} इत्यादिश्रुतेः~। {\bfseries तत्त्वमस्यादिवाक्योत्थं ज्ञानं मोक्षस्य साधनम्} इति नारदीयवचनाच्च~।\par
	तच्च ज्ञानमपरोक्षरूपम्~। परोक्षत्वेऽपरोक्षभ्रमनिवर्तकत्वानुपपत्तेः~। तच्चापरोक्षज्ञानं तत्त्वमस्यादिवाक्यादिति केचित्~। मनननिदिध्यासनसंस्कृतान्तःकरणादेवेत्यपरे~।\par
	तत्र पूर्वाचार्याणामाशयः संविदापरोक्ष्यं न करणविशेषोत्पत्तिनिबन्धनम्~। किन्तु प्रमेयविशेषनिबन्धनम् इत्युपपादितम्~। तथा च ब्रह्मणः प्रमातृजीवाभिन्नतया तद्गोचरं शब्दजन्यमपि ज्ञानमपरोक्षम्~। अत एव प्रतर्दनाधिकरणे प्रतर्दनं प्रति {\bfseries प्राणोऽस्मि प्रज्ञात्मा~। तं मामायुरमृतमुपास्व} इति इन्द्रप्रोक्तवाक्ये प्राणशब्दस्य ब्रह्मपरत्वे निश्चिते सति {\bfseries मामुपास्व} इत्यस्मच्छब्दानुपपत्तिमाशङ्क्य तदुत्तरत्वेन प्रवृत्ते {\bfseries शास्त्रदृष्ट्या तूपदेशो वामदेववत्} इत्यत्र सूत्रे- शास्त्रात् दृष्टिः शास्त्रदृष्टिः - तत्त्वमस्यादिवाक्यजन्यम् {\bfseries अहं ब्रह्म} इति ज्ञानं 'शास्त्रदृष्टि'-शब्देनोक्तमिति~।\par
	अन्येषां त्वेवमाशयः करणविशेषनिबन्धनमेव ज्ञानानां प्रत्यक्षत्वम्~, न विषयविशेषनिबन्धनम्~। एकस्मिन्नेव सूक्ष्मवस्तुनि पटुकरणापटुकरणयोः प्रत्यक्षत्वाप्रत्यक्षत्वव्यवहारदर्शनात्~। तथा च संवित्साक्षात्वे इन्द्रियजन्यत्वस्यैव प्रयोजकतया न शब्दजन्यज्ञानस्यापरोक्षत्वम्~। ब्रह्मसाक्षात्कारेऽपि मनननिदिध्यासनसंस्कृतं मन एव करणम्~, {\bfseries मनसैवानुद्रष्टव्यम्} इत्यादिश्रुतेः~। मनोऽगम्यत्वश्रुतिश्चासंस्कृतमनोविषया~। न चैवं ब्रह्मण औपनिषदत्वानुपपत्तिः~। अस्मदुक्तमनसो वेदजन्यज्ञानानन्तरमेव प्रवृत्ततया वेदोपजीवित्वात् ; वेदानुपजीविमानान्तरगम्यत्वस्यैव विरोधित्वात्~। 'शास्त्रदृष्टि'-सूत्रमपि ब्रह्मविषयकमानसप्रत्यक्षस्य शास्त्रप्रयोज्यत्वादुपपद्यते~। तदुक्तम् {\bfseries 'अपि संराधने' सूत्रात् शास्त्रार्थध्यानजा प्रमा~। शास्त्रदृष्टिर्मता, तान्तु वेत्ति वाचस्पतिः परम् ॥}\par
	तच्च ज्ञानं पापक्षयात्~। स च कर्मानुष्ठानादिति परम्परया कर्मणां विनियोगः~। अत एव {\bfseries तमेतं वेदानुवचनेन ब्राह्मणा विविदिषन्ति यज्ञेन दानेन तपसाऽनाशकेन} इत्यादि श्रुतिः, {\bfseries कषाये कर्मभिः पक्वे ततो ज्ञानं प्रवर्तते} इत्यादिस्मृतिश्च सङ्गच्छते~।\par
	एवं श्रवणमनननिदिध्यासनान्यपि ज्ञानसाधनानि, मैत्रेयीब्राह्मणे {\bfseries आत्मा वा अरे द्रष्टव्यः} इति दर्शनमनूद्य तत्साधनत्वेन {\bfseries श्रोतव्यो मन्तव्यो निदिध्यासितव्यः} इति श्रवणमनननिदिध्यासनानां विधानात्~। श्रवणं नाम वेदान्तानामद्वितीये ब्रह्मणि तात्पर्यावधारणानुकूला मानसी क्रिया~। मननं नाम शब्दावधारितेऽर्थे मानान्तरविरोधशङ्कायां, तन्निराकरणानुकूलतर्कात्मकज्ञानजनको मानसो व्यापारः~। निदिध्यासनं नाम अनादिदुर्वासनया विषयेष्वाकृष्यमाणस्य चित्तस्य विषयेभ्योऽपकृष्य आत्मविषयकस्थैर्यानुकूलो मानसो व्यापारः~।\par
	तत्र निदिध्यासनं ब्रह्मसाक्षात्कारे साक्षात् कारणम्~, {\bfseries ते ध्यानयोगानुगता अपश्यन्~, देवात्मशक्तिं स्वगुणैर्निगूढाम्} इत्यादिश्रुतेः~। निदिध्यासने च मननं हेतुः~। अकृतमननस्यार्थदार्ढ्याभावेन तद्विषयकनिदिध्यासनायोगात्~। मनने च श्रवणं हेतुः, श्रवणाभावे तात्पर्यानिश्चयेन शाब्दज्ञानाभावेन श्रुतार्थविषयकयुक्तत्वनिश्चयानुकूलमननायोगात्~। एतानि त्रीण्यपि ज्ञानोत्पत्तौ कारणानीति केचिदाचार्या ऊचिरे~।\par
	अपरे तु श्रवणं प्रधानम्~। मनननिदिध्यासनयोस्तु श्रवणात् पराचीनयोरपि श्रवणफलब्रह्मदर्शननिर्वर्तकतया आरादुपकारकतयाऽङ्गत्वमित्याहुः~। तदप्यङ्गत्वं न तार्तीयशेषत्वरूपम्~, तस्य श्रुत्याद्यन्यतमप्रमाणगम्यस्य प्रकृते श्रुत्याद्यभावेऽसम्भवात्~। तथा हि {\bfseries व्रीहिभिर्यजेत~।}, {\bfseries दध्ना जुहोति} इत्यादाविव मनननिदिध्यासनयोरङ्गत्वे न काचित् तृतीयाश्रुतिरस्ति~। नापि {\bfseries बर्हिर्देवसदनं दामि} इत्यादि मन्त्राणां बर्हिःखण्डन-प्रकाशनसामर्थ्यवत् किञ्चिल्लिङ्गमस्ति~। नापि प्रदेशान्तरपठितस्य प्रवर्गस्य {\bfseries अग्निष्टोमे प्रवृणक्ति} इति वाक्यवत् श्रवणानुवादेन मनननिदिध्यासनविनियोजकं किञ्चिद्वाक्यमस्ति~। नापि {\bfseries दर्शपूर्णमासाभ्यां स्वर्गकामो यजेत} इति वाक्यावगतफलसाधनताक-दर्शपूर्णमासप्रकरणे प्रयाजादीनामिव फलसाधनत्वेनावगतस्य श्रवणस्य प्रकरणे मनननिदिध्यासनयोराम्नानम्~।\par
	ननु 'द्रष्टव्यः' इति दर्शनानुवादेन श्रवणे विहिते सति फलवत्तया श्रवणप्रकरणे तत्सन्निधावाम्नातयोर्मनननिदिध्यासनयोः प्रयाजन्यायेन प्रकरणादेवाङ्गतेति चेत् न~। {\bfseries ते ध्यानयोगानुगता अपश्यन्} इत्यादिश्रुत्यन्ते ध्यानस्य दर्शनसाधनत्वेनावगतस्य अङ्गाकाङ्क्षायां प्रयाजन्ययेन श्रवणमननयोरेवाङ्गत्वापत्तेः~। क्रमसमाख्ये च दूरनिरस्ते~।\par
	किञ्च प्रयाजादिष्वङ्गत्वविचारः सप्रयोजनः~। पूर्वपक्षे विकृतिषु न प्रयाजाद्यनुष्ठानम् ; सिद्धान्ते तु तत्रापि तदनुष्ठानमिति~। प्रकृते तु श्रवणं न कस्यचित् प्रकृतिः, येन मनननिदिध्यासनयोस्तत्राप्यनुष्ठानमङ्गत्वविचारफलं भवेत्~। तस्मान्न तार्तीयशेषत्वं मनननिदिध्यासनयोः~। किन्तु यथा घटादिकार्ये मृत्पिण्डादीनां प्रधानकारणता, चक्रादीनां सहकारिकारणतेति प्राधान्याप्राधान्यव्यपदेशः, तथा श्रवणमनननिदिध्यासनानामपीति मन्तव्यम्~।\par
	सूचितं चैतद्विवरणाचार्यैः- {\bfseries शक्तितात्पर्यविशिष्टशब्दावधारणं प्रमेयावगमं प्रत्यव्यवधानेन कारणं भवति, प्रमाणस्य प्रमेयावगमं प्रत्यव्यवधानात्~। मनननिदिध्यासने तु चित्तस्य प्रत्यगात्मप्रवणतासंस्कारपरिनिष्पन्न-तदेकाग्रवृत्तिकार्यद्वारेण ब्रह्मानुभवहेतुतां प्रतिपद्येते इति फलं प्रत्यव्यवहितकारणस्य शक्तितात्पर्यविशिष्टशब्दावधारणस्य व्यवहिते मनननिदिध्यासने तदङ्गे अङ्गीक्रियेतेेेे} इति~।\par
	श्रवणादिषु च मुमुक्षूणामधिकारः~। काम्ये कर्मणि फलकामस्याधिकारित्वात्~। मुमुक्षायां च नित्यानित्यवस्तुविवेकस्येहामुत्रार्थफलभोगविरागस्य शमदमोपरतितितिक्षासमाधानश्रद्धानां च विनियोगः~। अन्तरिन्द्रियनिग्रहः शमः~। बहिरिन्द्रियनिग्रहो दमः~। विक्षेपाभाव उपरतिः~। शीतोष्णादिद्वन्द्वसहनं तितिक्षा~। चित्तैकाग्र्यं समाधानम्~। गुरुवेदान्तवाक्येषु विश्वासः श्रद्धा~।\par
	अत्र उपरमशब्देन संन्यासोऽभिधीयते~। तथा च संन्यासिनामेव श्रवणादावधिकारः इति केचित्~। अपरे तु, उपरमशब्दस्य संन्यासवाचकत्वाभावात्~, विक्षेपाभावमात्रस्य गृहस्थेष्वपि सम्भवात्~, जनकादेरपि ब्रह्मविचारस्य श्रूयमाणत्वात्~, सर्वाश्रमसाधारणं श्रवणादिविधानमित्याहुः~।\par
	सगुणोपासनमपि चित्तैकाग्र्यद्वारा निर्विशेषब्रह्मसाक्षात्कारहेतुः~। तदुक्तम् {\bfseries निर्विशेषं परं ब्रह्म साक्षात्कर्तुमनीश्वराः~। ये मन्दास्तेऽनुकम्प्यन्ते सविशेषनिरूपणैः ॥ वशीकृते मनस्येषां सगुणब्रह्मशीलनात्~। तदेवाविर्भवेत् साक्षादपेतोपाधिकल्पनम् ॥} इति~। सगुणोपासकानां च अर्चिरादिमार्गेण ब्रह्मलोकगतानां तत्रैव श्रवणाद्युत्पन्नतत्त्वसाक्षात्काराणां ब्रह्मणा सह मोक्षः~। कर्मिणान्तु धूमादिमार्गेण पितृलोकं गतानामुपभोगेन कर्मक्षये सति पूर्वकृतसुकृतदुष्कृतानुसारेण ब्रह्मादिस्थावरान्तेषु पुनरुत्पत्तिः~। तथा च श्रुतिः- {\bfseries रमणीयचरणा रमणीयां योनिमापद्यन्ते~। कपूयचरणाः कपूयां योनिमापद्यन्ते} इति~। प्रतिषिद्धानुष्ठायिनां तु रौरवादिनरकविशेषेषु तत्तत्पापोचित-तीव्रदुःखमनुभूय श्वशूकरादितिर्यग्योनिषु स्थावरादिषु चोत्पत्तिः इत्यलं प्रसङ्गागतप्रपञ्चेन~।\par
	निर्गुणब्रह्मसाक्षात्कारवतस्तु न लोकान्तरगमनम्~। {\bfseries न तस्य प्राणा उत्क्रामन्ति} इतिश्रुतेः~। किन्तु यावत्प्रारब्धकर्मक्षयं सुखदुःखे अनुभूय पश्चादपवृज्यते~।\par
	ननु {\bfseries क्षीयन्ते चास्य कर्माणि तस्मिन् दृष्टे परावरे} इत्यादिश्रुत्या, {\bfseries ज्ञानाग्निः सर्वकर्माणि भस्मसात् कुरुते तथा} इत्यादिस्मृत्या च ज्ञानस्य सकलकर्मक्षयहेतुत्वनिश्चये सति प्रारब्धकर्मावस्थानमनुपपन्नमिति चेत् न~। {\bfseries तस्य तावदेव चिरं यावन्न विमोक्ष्ये~। अथ सम्पत्स्ये} इत्यादिश्रुत्या, {\bfseries नाभुक्तं क्षीयते कर्म} इत्यादिस्मृत्या चोत्पादितकार्यकर्मव्यतिरिक्तानां सञ्चितकर्मणामेव ज्ञानविनाश्यत्वावगमात्~।\par
	सञ्चितं द्विविधम् सुकृतं दुष्कृतं च~। तथा च श्रुतिः {\bfseries तस्य पुत्रा दायमुपयन्ति~। सुहृदः साधुकृत्याम्~। द्विषन्तः पापकृत्याम्} इति~। ननु ब्रह्मज्ञानान्मूलाज्ञाननिवृत्तौ तत्कार्यप्रारब्धकर्मणोऽपि निवृत्तेः कथं ज्ञानिनां देहधारणमुपपद्यते ? इति चेत् न~। अप्रतिबद्धज्ञानस्यैवाज्ञाननिवर्तकतया प्रारब्धकर्मरूपप्रतिबन्धकदशायामज्ञाननिवृत्तेरनङ्गीकारात्~।\par
	नन्वेवमपि तत्त्वज्ञानादेकस्य मुक्तौ सर्वमुक्तिः स्यात्~। अविद्याया एकत्वेन तन्निवृत्तौ क्वचिदपि संसारायोगादिति चेत् न~। इष्टापत्तेरित्येके~। अपरे तु एतद्दोषपरिहारायैव {\bfseries इन्द्रो मायाभिः} इति बहुवचनश्रुत्यनुगृहीतमविद्यानानात्वमङ्गीकर्तव्यमित्याहुः~। अन्ये तु एकैवाविद्या~। तस्याश्चाविद्याया जीवभेदेन ब्रह्मस्वरूपावरणशक्तयो नाना~। तथा च यस्य ब्रह्मज्ञानं तस्य ब्रह्मस्वरूपावरणशक्तिविशिष्टाविद्यानाशः, न त्वन्यं प्रति, इत्युपगमात् नैकमुक्तौ सर्वमुक्तिः~। अत एव {\bfseries यावदधिकारमवस्थितिराधिकारिकाणाम्} इत्यस्मिन्नधिकरणे अाधिकारिकपुरुषाणामुत्पन्नतत्त्वज्ञानानामिन्द्रादीनां देहधारणानुपपत्तिमाशङ्क्य, अधिकारापादकप्रारब्धकर्मसमाप्त्यनन्तरं विदेहकैवल्यमिति सिद्धान्तितम्~। तदुक्तमाचार्यवाचस्पतिमिश्रैः- {\bfseries उपसनादिसंसिद्धितोषितेश्वरचोदिताः~
	। अधिकारं समाप्यैते प्रविशन्ति परं पदम् ॥} इति~। एतच्चैकमुक्तौ सर्वमुक्तिरिति पक्षे नोपपद्यते~। तस्मादेकाविद्यापक्षेऽपि प्रतिजीवमावरणभेदोपगमेन व्यवस्थोपपादनीया~।\par
	तदेवं ब्रह्मज्ञानान्मोक्षः~। स चानर्थनिवृत्तिर्निरतिशयब्रह्मानन्दावाप्तिश्चेति सिद्धं प्रयोजनम्~।\par
	\begin{center} इति वेदान्तपरिभाषायां प्रयोजनपरिच्छेदः~। \\ इति धर्मराजाध्वरीन्द्रविरचिता वेदान्तपरिभाषा समाप्ता ॥\end{center} 
