\titleformat*{\section}{\large\\bfseries }
\renewcommand*\contentsname{\Large विषयानुक्रमणिका}
\begin{document}

\begin{center}{\bfseries नारायणः परोऽव्यक्तादण्डमव्यक्तसम्भवम्~।\\ अण्डस्यान्तस्त्विमे लोकाः सप्तद्वीपा च मेदिनी~॥\\[10pt]\end{center}
\indentस भगवान् सृष्ट्वेदं जगत्~, तस्य च स्थितिं चिकीर्षुः, मरीच्यादीनग्रे सृष्ट्वा प्रजापतीन्~, प्रवृत्तिलक्षणं धर्मं ग्राहयामास वेदोक्तम्~। ततोऽन्यांश्च सनकसनन्दनादीनुत्पाद्य, निवृत्तिलक्षणं धर्मं ज्ञानवैराग्यलक्षणं ग्राहयामास~। द्विविधो हि वेदोक्तो धर्मः, प्रवृत्तिलक्षणो निवृत्तिलक्षणश्च, जगतः स्थितिकारणम्~। प्राणिनां साक्षादभ्युदयनिःश्रेयसहेतुर्यः स धर्मो ब्राह्मणाद्यैर्वर्णिभिराश्रमिभिश्च श्रेयोर्थिभिः अनुष्ठीयमानो दीर्घेण कालेन~। अनुष्ठातॄणां कामोद्भवात् हीयमानविवेकविज्ञानहेतुकेन अधर्मेण अभिभूयमाने धर्मे, प्रवर्धमाने च अधर्मे, जगतः स्थितिं परिपिपालयिषुः स आदिकर्ता नारायणाख्यो विष्णुः भौमस्य ब्रह्मणो ब्राह्मणत्वस्य रक्षणार्थं देवक्यां वसुदेवादंशेन कृष्णः किल सम्बभूव~। ब्राह्मणत्वस्य हि रक्षणे रक्षितः स्याद्वैदिको धर्मः, तदधीनत्वाद्वर्णाश्रमभेदानाम्~॥~} 
स च भगवान् ज्ञानैश्वर्यशक्तिबलवीर्यतेजोभिः सदा सम्पन्नः त्रिगुणात्मिकां स्वां मायां मूलप्रकृतिं वशीकृत्य, अजोऽव्ययो भूतानामीश्वरो नित्यशुद्धबुद्धमुक्तस्वभावोऽपि सन्~, स्वमायया देहवानिव जात इव च लोकानुग्रहं कुर्वन् लक्ष्यते~। स्वप्रयोजनाभावेऽपि भूतानुजिघृक्षया वैदिकं धर्मद्वयम् अर्जुनाय शोकमोहमहोदधौ निमग्नाय उपदिदेश, गुणाधिकैर्हि गृहीतोऽनुष्ठीयमानश्च धर्मः प्रचयं गमिष्यतीति~। तं धर्मं भगवता यथोपदिष्टं वेदव्यासः सर्वज्ञो भगवान् गीताख्यैः सप्तभिः श्लोकशतैरुपनिबबन्ध~॥~} 
तदिदं गीताशास्त्रं समस्तवेदार्थसारसङ्ग्रहभूतं दुर्विज्ञेयार्थम्~, तदर्थाविष्करणायानेकैर्विवृतपदपदार्थवाक्यार्थन्यायमपि अत्यन्तविरुद्धानेकार्थवत्वेन लौकिकैर्गृह्यमाणमुपलभ्य अहं विवेकतोऽर्थनिर्धारणार्थं सङ्क्षेपतो विवरणं करिष्यामि~॥~} 
तस्य अस्य गीताशास्त्रस्य सङ्क्षेपतः प्रयोजनं परं निःश्रेयसं सहेतुकस्य संसारस्य अत्यन्तोपरमलक्षणम्~। तच्च सर्वकर्मसंन्यासपूर्वकादात्मज्ञाननिष्ठारूपात् धर्मात् भवति~। तथा इममेव गीतार्थं धर्ममुद्दिश्य भगवतैवोक्तम् — ‘स हि धर्मः सुपर्याप्तो ब्रह्मणः पदवेदने’\footnote{अश्व. १६~। १२} इति अनुगीतासु~। तत्रैव चोक्तम् — ‘नैव धर्मी न चाधर्मी न चैव हि शुभाशुभी~। ’\footnote{अश्व. १९~। ७} ‘यः स्यादेकासने लीनस्तूष्णीं किञ्चिदचिन्तयन्’\footnote{अश्व. १९~। १}~॥~इति ‘ज्ञानं संन्यासलक्षणम्’\footnote{अश्व. ४३~। २६} इति च~। इहापि च अन्ते उक्तमर्जुनाय — ‘सर्वधर्मान् परित्यज्य मामेकं शरणं व्रज’\footnote{भ. गी. १८~। ६६} इति~। अभ्युदयार्थोऽपि यः प्रवृत्तिलक्षणो धर्मो वर्णानाश्रमांश्चोद्दिश्य विहितः, स देवादिस्थानप्राप्तिहेतुरपि सन्~, ईश्वरार्पणबुद्ध्या अनुष्ठीयमानः सत्त्वशुद्धये भवति फलाभिसन्धिवर्जितः~। शुद्धसत्त्वस्य च ज्ञाननिष्ठायोग्यताप्राप्तिद्वारेण ज्ञानोत्पत्तिहेतुत्वेन च निःश्रेयसहेतुत्वमपि प्रतिपद्यते~। तथा चेममर्थमभिसन्धाय वक्ष्यति — ‘ब्रह्मण्याधाय कर्माणि’\footnote{भ. गी. ५~। १०} ‘योगिनः कर्म कुर्वन्ति सङ्गं त्यक्त्वात्मशुद्धये’\footnote{भ. गी. ५~। ११} इति~॥~} 
इमं द्विप्रकारं धर्मं निःश्रेयसप्रयोजनम्~, परमार्थतत्त्वं च वासुदेवाख्यं परं ब्रह्माभिधेयभूतं विशेषतः अभिव्यञ्जयत् विशिष्टप्रयोजनसम्बन्धाभिधेयवद्गीताशास्त्रम्~। यतः तदर्थविज्ञाने समस्तपुरुषार्थसिद्धिः, अतः तद्विवरणे यत्नः क्रियते मया~॥\par

\begin{center}{\bfseries धृतराष्ट्र उवाच —\\ धर्मक्षेत्रे कुरुक्षेत्रे समवेता युयुत्सवः~।\\मामकाः पाण्डवाश्चैव किमकुर्वत सञ्जय~॥~१~॥}\\[10pt]
{\bfseries सञ्जय उवाच —}\\ 
{\bfseries दृष्ट्वा तु पाण्डवानीकं व्यूढं दुर्योधनस्तदा~।\\आचार्यमुपसङ्गम्य राजा वचनमब्रवीत्~॥~२~॥}\\[10pt]
{\bfseries पश्यैतां पाण्डुपुत्राणामाचार्य महतीं चमूम्~।\\व्यूढां द्रुपदपुत्रेण तव शिष्येण धीमता~॥~३~॥}\\[10pt]
{\bfseries अत्र शूरा महेष्वासा भीमार्जुनसमा युधि~।\\युयुधानो विराटश्च द्रुपदश्च महारथः~॥~४~॥}\\[10pt]
{\bfseries धृष्टकेतुश्चेकितानः काशीराजश्च वीर्यवान्~।\\पुरुजित्कुन्तिभोजश्च शैब्यश्च नरपुङ्गवः~॥~५~॥}\\[10pt]
{\bfseries युधामन्युश्च विक्रान्त उत्तमौजाश्च वीर्यवान्~।\\सौभद्रो द्रौपदेयाश्च सर्व एव महारथाः~॥~६~॥}\\[10pt]
{\bfseries अस्माकं तु विशिष्टा ये तान्निबोध द्विजोत्तम~।\\नायका मम सैन्यस्य संज्ञार्थं तान्ब्रवीमि ते~॥~७~॥}\\[10pt]
{\bfseries भवान्भीष्मश्च कर्णश्च कृपश्च समितिञ्जयः~।\\अश्वत्थामा विकर्णश्च सौमदत्तिर्जयद्रथः~॥~८~॥}\\[10pt]
{\bfseries अन्ये च बहवः शूरा मदर्थे त्यक्तजीविताः~।\\नानाशस्त्रप्रहरणाः सर्वे युद्धविशारदाः~॥~९~॥}\\[10pt]
{\bfseries अपर्याप्तं तदस्माकं बलं भीष्माभिरक्षितम्~।\\पर्याप्तं त्विदमेतेषां बलं भीमाभिरक्षितम्~॥~१०~॥}\\[10pt]
{\bfseries अयनेषु च सर्वेषु यथाभागमवस्थिताः~।\\भीष्ममेवाभिरक्षन्तु भवन्तः सर्व एव हि~॥~११~॥}\\[10pt]
{\bfseries तस्य सञ्जनयन्हर्षं कुरुवृद्धः पितामहः~।\\सिंहनादं विनद्योच्चैः शङ्खं दध्मौ प्रतापवान्~॥~१२~॥}\\[10pt]
{\bfseries ततः शङ्खाश्च भेर्यश्च पणवानकगोमुखाः~।\\सहसैवाभ्यहन्यन्त स शब्दस्तुमुलोऽभवत्~॥~१३~॥}\\[10pt]
{\bfseries ततः श्वेतैर्हयैर्युक्ते महति स्यन्दने स्थितौ~।\\माधवः पाण्डवश्चैव दिव्यौ शङ्खौ प्रदध्मतुः~॥~१४~॥}\\[10pt]
{\bfseries पाञ्चजन्यं हृषीकेशो देवदत्तं धनञ्जयः~।\\पौण्ड्रं दध्मौ महाशङ्खं भीमकर्मा वृकोदरः~॥~१५~॥}\\[10pt]
{\bfseries अनन्तविजयं राजा कुन्तीपुत्रो युधिष्ठिरः~।\\नकुलः सहदेवश्च सुघोषमणिपुष्पकौ~॥~१६~॥}\\[10pt]
{\bfseries काश्यश्च परमेष्वासः शिखण्डी च महारथः~।\\धृष्टद्युम्नो विराटश्च सात्यकिश्चापराजितः~॥~१७~॥}\\[10pt]
{\bfseries द्रुपदो द्रौपदेयाश्च सर्वशः पृथिवीपते~।\\सौभद्रश्च महाबाहुः शङ्खान्दध्मुः पृथक्पृथक्~॥~१८~॥}\\[10pt]
{\bfseries स घोषो धार्तराष्ट्राणां हृदयानि व्यदारयत्~।\\नभश्च पृथिवीं चैव तुमुलो व्यनुनादयन्~॥~१९~॥}\\[10pt]
{\bfseries अथ व्यवस्थितान्दृष्ट्वा धार्तराष्ट्रान्कपिध्वजः~।\\प्रवृत्ते शस्त्रसम्पाते धनुरुद्यम्य पाण्डवः~॥~२०~॥}\\[10pt]
{\bfseries हृषीकेशं तदा वाक्यमिदमाह महीपते~।\\अर्जुन उवाच —\\सेनयोरुभयोर्मध्ये रथं स्थापय मेऽच्युत~॥~२१~॥}\\[10pt]
{\bfseries यावदेतान्निरीक्षेऽहं योद्धुकामानवस्थितान्~।\\कैर्मया सह योद्धव्यमस्मिन्रणसमुद्यमे~॥~२२~॥}\\[10pt]
{\bfseries योत्स्यमानानवेक्षेऽहं य एतेऽत्र समागताः~।\\धार्तराष्ट्रस्य दुर्बुद्धेर्युद्धे प्रियचिकीर्षवः~॥~२३~॥}\\[10pt]
{\bfseries सञ्जय उवाच —}\\ 
{\bfseries एवमुक्तो हृषीकेशो गुडाकेशेन भारत~।\\सेनयोरुभयोर्मध्ये स्थापयित्वा रथोत्तमम्~॥~२४~॥}\\[10pt]
{\bfseries भीष्मद्रोणप्रमुखतः सर्वेषां च महीक्षिताम्~।\\उवाच पार्थ पश्यैतान्समवेतान्कुरूनिति~॥~२५~॥}\\[10pt]
{\bfseries तत्रापश्यत्स्थितान्पार्थः पितॄनथ पितामहान्~।\\आचार्यान्मातुलान्भ्रातॄन्पुत्रान्पौत्रान्सखींस्तथा~॥~२६~॥}\\[10pt]
{\bfseries श्वशुरान्सुहृदश्चैवसेनयोरुभयोरपि~।\\तान्समीक्ष्य स कौन्तेयः सर्वान्बन्धूनवस्थितान्~॥~२७~॥}\\[10pt]
{\bfseries कृपया परयाविष्टो विषीदन्निदमब्रवीत्~।\\अर्जुन उवाच —\\दृष्ट्वेमान्स्वजनान्कृष्ण युयुत्सून्समुपस्थितान्~॥~२८~॥}\\[10pt]
{\bfseries सीदन्ति मम गात्राणि मुखं च परिशुष्यति~।\\वेपथुश्च शरीरे मे रोमहर्षश्च जायते~॥~२९~॥}\\[10pt]
{\bfseries गाण्डीवं स्रंसते हस्तात्त्वक्चैव परिदह्यते~।\\न च शक्नोम्यवस्थातुं भ्रमतीव च मे मनः~॥~३०~॥}\\[10pt]
{\bfseries निमित्तानि च पश्यामि विपरीतानि केशव~।\\न च श्रेयोऽनुपश्यामि हत्वा स्वजनमाहवे~॥~३१~॥}\\[10pt]
{\bfseries न काङ्क्षे विजयं कृष्ण न च राज्यं सुखानि च~।\\किं नो राज्येन गोविन्द किं भोगैर्जीवितेन वा~॥~३२~॥}\\[10pt]
{\bfseries येषामर्थे काङ्क्षितं नो राज्यं भोगाः सुखानि च~।\\त इमेऽवस्थिता युद्धे प्राणांस्त्यक्त्वा धनानि च~॥~३३~॥}\\[10pt]
{\bfseries आचार्याः पितरः पुत्रास्तथैव च पितामहाः~।\\मातुलाः श्वशुराः पौत्राः स्यालाः सम्बन्धिनस्तथा~॥~३४~॥}\\[10pt]
{\bfseries एतान्न हन्तुमिच्छामि घ्नतोऽपि मधुसूदन~।\\अपि त्रैलोक्यराज्यस्य हेतोः किं नु महीकृते~॥~३५~॥}\\[10pt]
{\bfseries निहत्य धार्तराष्ट्रान्नः का प्रीतिः स्याज्जनार्दन~।\\पापमेवाश्रयेदस्मान्हत्वैतानाततायिनः~॥~३६~॥}\\[10pt]
{\bfseries तस्मान्नार्हा वयं हन्तुं धार्तराष्ट्रान्सबान्धवान्~।\\स्वजनं हि कथं हत्वा सुखिनः स्याम माधव~॥~३७~॥}\\[10pt]
{\bfseries यद्यप्येते न पश्यन्ति लोभोपहतचेतसः~।\\कुलक्षयकृतं दोषं मित्रद्रोहे च पातकम्~॥~३८~॥}\\[10pt]
{\bfseries कथं न ज्ञेयमस्माभिः पापादस्मान्निवर्तितुम्~।\\कुलक्षयकृतं दोषं प्रपश्यद्भिर्जनार्दन~॥~३९~॥}\\[10pt]
{\bfseries कुलक्षये प्रणश्यन्ति कुलधर्माः सनातनाः~।\\धर्मे नष्टे कुलं कृत्स्नमधर्मोऽभिभवत्युत~॥~४०~॥}\\[10pt]
{\bfseries अधर्माभिभवात्कृष्ण प्रदुष्यन्ति कुलस्त्रियः~।\\स्त्रीषु दुष्टासु वार्ष्णेय जायते वर्णसङ्करः~॥~४१~॥}\\[10pt]
{\bfseries सङ्करो नरकायैव कुलघ्नानां कुलस्य च~।\\पतन्ति पितरो ह्येषां लुप्तपिण्डोदकक्रियाः~॥~४२~॥}\\[10pt]
{\bfseries दोषैरेतैः कुलघ्नानां वर्णसङ्करकारकैः~।\\उत्साद्यन्ते जातिधर्माः कुलधर्माश्च शाश्वताः~॥~४३~॥}\\[10pt]
{\bfseries उत्सन्नकुलधर्माणां मनुष्याणां जनार्दन~।\\नरके नियतं वासो भवतीत्यनुशुश्रुम~॥~४४~॥}\\[10pt]
{\bfseries अहो बत महत्पापं कर्तुं व्यवसिता वयम्~।\\यद्राज्यसुखलोभेन हन्तुं स्वजनमुद्यताः~॥~४५~॥}\\[10pt]
{\bfseries यदि मामप्रतीकारमशस्त्रं शस्त्रपाणयः~।\\धार्तराष्ट्रा रणे हन्युस्तन्मे क्षेमतरं भवेत्~॥~४६~॥}\\[10pt]
{\bfseries सञ्जय उवाच —}\\ 
{\bfseries एवमुक्त्वार्जुनः सं‍ख्ये रथोपस्थ उपाविशत्~।\\विसृज्य सशरं चापं शोकसंविग्नमानसः~॥~४७~॥}\end{center}
 
इति श्रीमहाभारते शतसाहस्र्यां संहितायां वैयासिक्यां भीष्मपर्वणि श्रीमद्भगवद्गीतासूपनिषत्सु ब्रह्मविद्यायां योगशास्त्रे श्रीकृष्णार्जुनसंवादे अर्जुनविषादयोगो नाम प्रथमोऽध्यायः~॥\par
 
\begin{center}{\bfseries सञ्जय उवाच —\\तं तथा कृपयाविष्टमश्रुपूर्णाकुलेक्षणम्~।\\विषीदन्तमिदं वाक्यमुवाच मधुसूदनः~॥~१~॥}\\[10pt]
{\bfseries श्रीभगवानुवाच —}\\ 
{\bfseries कुतस्त्वा कश्मलमिदं विषमे समुपस्थितम्~।\\अनार्यजुष्टमस्वर्ग्यमकीर्तिकरमर्जुन~॥~२~॥}\\[10pt]
{\bfseries क्लैब्यं मा स्म गमः पार्थ नैतत्त्वय्युपपद्यते~।\\क्षुद्रं हृदयदौर्बल्यं त्यक्त्वोत्तिष्ठ परन्तप~॥~३~॥}\\[10pt]
{\bfseries अर्जुन उवाच —}\\ 
{\bfseries कथं भीष्ममहं सं‍ख्ये द्रोणं च मधुसूदन~।\\इषुभिः प्रतियोत्स्यामि पूजार्हावरिसूदन~॥~४~॥}\\[10pt]
{\bfseries गुरूनहत्वा हि महानुभावान् श्रेयो भोक्तुं भैक्षमपीह लोके~।\\हत्वार्थकामांस्तु गुरूनिहैव भुञ्जीय भोगान्रुधिरप्रदिग्धान्~॥~५~॥}\\[10pt]
{\bfseries न चैतद्विद्मः कतरन्नो गरीयो यद्वा जयेम यदि वा नो जयेयुः~।\\यानेव हत्वा न जिजीविषामस्तेऽवस्थिताः प्रमुखे धार्तराष्ट्राः~॥~६~॥}\\[10pt]
{\bfseries कार्पण्यदोषोपहतस्वभावः पृच्छामि त्वां धर्मसंमूढचेताः~।\\यच्छ्रेयः स्यान्निश्चितं ब्रूहि तन्मे शिष्यस्तेऽहं शाधि मां त्वां प्रपन्नम्~॥~७~॥}\\[10pt]
{\bfseries न हि प्रपश्यामि ममापनुद्याद्यच्छोकमुच्छोषणमिन्द्रियाणाम्~।\\अवाप्य भूमावसपत्नमृद्धं राज्यं सुराणामपि चाधिपत्यम्~॥~८~॥}\\[10pt]
{\bfseries सञ्जय उवाच —}\\ 
{\bfseries एवमुक्त्वा हृषीकेशं गुडाकेशः परन्तपः~।\\न योत्स्य इति गोविन्दमुक्त्वा तूष्णीं बभूव ह~॥~९~॥}\\[10pt]
{\bfseries तमुवाच हृषीकेशः प्रहसन्निव भारत~।\\सेनयोरुभयोर्मध्ये विषीदन्तमिदं वचः~॥~१०~॥}\end{center}
 
अत्र ‘दृष्ट्वा तु पाण्डवानीकम्’\footnote{भ. गी. १~। २} इत्यारभ्य यावत् ‘न योत्स्य इति गोविन्दमुक्त्वा तूष्णीं बभूव ह’\footnote{भ. गी. २~। ९} इत्येतदन्तः प्राणिनां शोकमोहादिसंसारबीजभूतदोषोद्भवकारणप्रदर्शनार्थत्वेन व्याख्येयो ग्रन्थः~। तथाहि — अर्जुनेन राज्यगुरुपुत्रमित्रसुहृत्स्वजनसम्बन्धिबान्धवेषु ‘अहमेतेषाम्’ ‘ममैते’ इत्येवंप्रत्ययनिमित्तस्नेहविच्छेदादिनिमित्तौ आत्मनः शोकमोहौ प्रदर्शितौ ‘कथं भीष्ममहं सङ्‍ख्ये’\footnote{भ. गी. २~। ४} इत्यादिना~। शोकमोहाभ्यां ह्यभिभूतविवेकविज्ञानः स्वत एव क्षत्रधर्मे युद्धे प्रवृत्तोऽपि तस्माद्युद्धादुपरराम~; परधर्मं च भिक्षाजीवनादिकं कर्तुं प्रववृते~। तथा च सर्वप्राणिनां शोकमोहादिदोषाविष्टचेतसां स्वभावत एव स्वधर्मपरित्यागः प्रतिषिद्धसेवा च स्यात्~। स्वधर्मे प्रवृत्तानामपि तेषां वाङ्मनःकायादीनां प्रवृत्तिः फलाभिसन्धिपूर्विकैव साहङ्कारा च भवति~। तत्रैवं सति धर्माधर्मोपचयात् इष्टानिष्टजन्मसुखदुःख़ादिप्राप्तिलक्षणः संसारः अनुपरतो भवति~। इत्यतः संसारबीजभूतौ शोकमोहौ तयोश्च सर्वकर्मसंन्यासपूर्वकादात्मज्ञानात् नान्यतो निवृत्तिरिति तदुपदिदिक्षुः सर्वलोकानुग्रहार्थम् अर्जुनं निमित्तीकृत्य आह भगवान्वासुदेवः — ‘अशोच्यान्’\footnote{भ. गी. २~। ११} इत्यादि~॥~} 
अत्र केचिदाहुः — सर्वकर्मसंन्यासपूर्वकादात्मज्ञाननिष्ठामात्रादेव केवलात् कैवल्यं न प्राप्यत एव~। किं तर्हि~? अग्निहोत्रादिश्रौतस्मार्तकर्मसहितात् ज्ञानात् कैवल्यप्राप्तिरिति सर्वासु गीतासु निश्चितोऽर्थ इति~। ज्ञापकं च आहुरस्यार्थस्य — ‘अथ चेत्त्वमिमं धर्म्यं सङ्ग्रामं न करिष्यसि’\footnote{भ. गी. २~। ३३} ‘कर्मण्येवाधिकारस्ते’\footnote{भ. गी. २~। ४७} ‘कुरु कर्मैव तस्मात्त्वम्’\footnote{भ. गी. ४~। १५} इत्यादि~। हिंसादियुक्तत्वात् वैदिकं कर्म अधर्माय इतीयमप्याशङ्का न कार्या~। कथम्~? क्षात्रं कर्म युद्धलक्षणं गुरुभ्रातृपुत्रादिहिंसालक्षणमत्यन्तं क्रूरमपि स्वधर्म इति कृत्वा न अधर्माय~; तदकरणे च ‘ततः स्वधर्मं कीर्तिं च हित्वा पापमवाप्स्यसि’\footnote{भ. गी. २~। ३३} इति ब्रुवता यावज्जीवादिश्रुतिचोदितानां पश्वादिहिंसालक्षणानां च कर्मणां प्रागेव नाधर्मत्वमिति सुनिश्चितमुक्तं भवति — इति~॥~} 
तदसत्~; ज्ञानकर्मनिष्ठयोर्विभागवचनाद्बुद्धिद्वयाश्रययोः~। ‘अशोच्यान्’\footnote{भ. गी. २~। ११} इत्यादिना भगवता यावत् ‘स्वधर्ममपि चावेक्ष्य’\footnote{भ. गी. २~। ३१} इत्येतदन्तेन ग्रन्थेन यत्परमार्थात्मतत्त्वनिरूपणं कृतम्~, तत्साङ्ख्यम्~। तद्विषया बुद्धिः आत्मनो जन्मादिषड्विक्रियाभावादकर्ता आत्मेति प्रकरणार्थनिरूपणात् या जायते, सा साङ्ख्याबुद्धिः~। सा येषां ज्ञानिनामुचिता भवति, ते साङ्ख्याः~। एतस्या बुद्धेः जन्मनः प्राक् आत्मनो देहादिव्यतिरिक्तत्वकर्तृत्वभोक्तृत्वाद्यपेक्षो धर्माधर्मविवेकपूर्वको मोक्षसाधनानुष्ठानलक्षणो योगः~। तद्विषया बुद्धिः योगबुद्धिः~। सा येषां कर्मिणामुचिता भवति ते योगिनः~। तथा च भगवता विभक्ते द्वे बुद्धी निर्दिष्टे ‘एषा तेऽभिहिता साङ्‍ख्ये बुद्धिर्योगे त्विमां शृणु’\footnote{भ. गी. २~। ३९} इति~। तयोश्च साङ्‍ख्यबुद्ध्याश्रयां ज्ञानयोगेन निष्ठां साङ्‍ख्यानां विभक्तां वक्ष्यति ‘पुरा वेदात्मना मया प्रोक्ता’\footnote{भ. गी. ३~। ३} इति~। तथा च योगबुद्ध्याश्रयां कर्मयोगेन निष्ठां विभक्तां वक्ष्यति — ‘कर्मयोगेन योगिनाम्’ इति~। एवं साङ्‍ख्यबुद्धिं योगबुद्धिं च आश्रित्य द्वे निष्ठे विभक्ते भगवतैव उक्ते ज्ञानकर्मणोः कर्तृत्वाकर्तृत्वैकत्वानेकत्वबुद्ध्याश्रययोः युगपदेकपुरुषाश्रयत्वासम्भवं पश्यता~। यथा एतद्विभागवचनम्~, तथैव दर्शितं शातपथीये ब्राह्मणे — ‘एतमेव प्रव्राजिनो लोकमिच्छन्तो ब्राह्मणाः प्रव्रजन्ति’ इति सर्वकर्मसंन्यासं विधाय तच्छेषेण ‘किं प्रजया करिष्यामो येषां नोऽयमात्मायं लोकः’\footnote{बृ. उ. ४~। ४~। २२} इति~। तत्र च प्राक् दारपरिग्रहात् पुरुषः आत्मा प्राकृतो धर्मजिज्ञासोत्तरकालं लोकत्रयसाधनम् — पुत्रम्~, द्विप्रकारं च वित्तं मानुषं दैवं च~; तत्र मानुषं कर्मरूपं पितृलोकप्राप्तिसाधनं विद्यां च दैवं वित्तं देवलोकप्राप्तिसाधनम् — ‘सोऽकामयत’\footnote{बृ. उ. १~। ४~। १७} इति अविद्याकामवत एव सर्वाणि कर्माणि श्रौतादीनि दर्शितानि~। तेभ्यः ‘व्युत्थाय, प्रव्रजन्ति’ इति व्युत्थानमात्मानमेव लोकमिच्छतोऽकामस्य विहितम्~। तदेतद्विभागवचनमनुपपन्नं स्याद्यदि श्रौतकर्मज्ञानयोः समुच्छयोऽभिप्रेतः स्याद्भगवतः~॥~} 
न च अर्जुनस्य प्रश्न उपपन्नो भवति ‘ज्यायसी चेत्कर्मणस्ते’\footnote{भ. गी. ३~। १} इत्यादिः~। एकपुरुषानुष्ठेयत्वासम्भवं बुद्धिकर्मणोः भगवता पूर्वमनुक्तं कथमर्जुनः अश्रुतं बुद्धेश्च कर्मणो ज्यायस्त्वं भगवत्यध्यारोपयेन्मृषैव ‘ज्यायसी चेत्कर्मणस्ते मता बुद्धिः’\footnote{भ. गी. ३~। १} इति~॥~} 
किञ्च — यदि बुद्धिकर्मणोः सर्वेषां समुच्छय उक्तः स्यात् अर्जुनस्यापि स उक्त एवेति, ‘यच्छ्रेय एतयोरेकं तन्मे ब्रूहि सुनिश्चितम्’\footnote{भ. गी. ५~। १} इति कथमुभयोरुपदेशे सति अन्यतरविषय एव प्रश्नः स्यात्~? न हि पित्तप्रशमनार्थिनः वैद्येन मधुरं शीतलं च भोक्तव्यम् इत्युपदिष्टे तयोरन्यतरत्पित्तप्रशमनकारणं ब्रूहि इति प्रश्नः सम्भवति~॥~} 
अथ अर्जुनस्य भगवदुक्तवचनार्थविवेकानवधारणनिमित्तः प्रश्नः कल्प्येत, तथापि भगवता प्रश्नानुरूपं प्रतिवचनं देयम् — मया बुद्धिकर्मणोः समुच्छय उक्तः, किमर्थमित्थं त्वं भ्रान्तोऽसि — इति~। न तु पुनः प्रतिवचनमननुरूपं पृष्टादन्यदेव ‘द्वे निष्ठे मया पुरा प्रोक्ते’\footnote{भ. गी. ३~। ३} इति वक्तुं युक्तम्~॥~} 
नापि स्मार्तेनैव कर्मणा बुद्धेः समुच्चये अभिप्रेते विभागवचनादि सर्वमुपपन्नम्~। किञ्च — क्षत्रियस्य युद्धं स्मार्तं कर्म स्वधर्म इति जानतः ‘तत्किं कर्मणि घोरे मां नियोजयसि’\footnote{भ. गी. ३~। १} इति उपालम्भोऽनुपपन्नः~॥~} 
तस्माद्गीताशास्त्रे ईषन्मात्रेणापि श्रौतेन स्मार्तेन वा कर्मणा आत्मज्ञानस्य समुच्चयो न केनचिद्दर्शयितुं शक्यः~। यस्य तु अज्ञानात् रागादिदोषतो वा कर्मणि प्रवृत्तस्य यज्ञेन दानेन तपसा वा विशुद्धसत्त्वस्य ज्ञानमुत्पन्नम्परमार्थतत्त्वविषयम् ‘एकमेवेदं सर्वं ब्रह्म अकर्तृ च’ इति, तस्य कर्मणि कर्मप्रयोजने च निवृत्तेऽपि लोकसङ्ग्रहार्थं यत्नपूर्वं यथा प्रवृत्तिः, तथैव प्रवृत्तस्य यत्प्रवृत्तिरूपं दृश्यते न तत्कर्म येन बुद्धेः समुच्चयः स्यात्~; यथा भगवतो वासुदेवस्य क्षत्रधर्मचेष्टितं न ज्ञानेन समुच्चीयते पुरुषार्थसिद्धये, तद्वत् तत्फलाभिसन्ध्यहङ्काराभावस्य तुल्यत्वाद्विदुषः~। तत्त्वविन्नाहं करोमीति मन्यते, न च तत्फलमभिसन्धत्ते~। यथा च स्वर्गादिकामार्थिनः अग्निहोत्रादिकर्मलक्षणधर्मानुष्ठानाय आहिताग्नेः काम्ये एव अग्निहोत्रादौ प्रवृत्तस्य सामि कृते विनष्टेऽपि कामे तदेव अग्निहोत्राद्यनुतिष्ठतोऽपि न तत्काम्यमग्निहोत्रादि भवति~। तथा च दर्शयति भगवान् — ‘कुर्वन्नपि न लिप्यते’\footnote{भ. गी. ५~। ७} ‘न करोति न लिप्यते’\footnote{भ. गी. १३~। ३१} इति तत्र तत्र~॥~} 
यच्च ‘पूर्वैः पूर्वतरं कृतम्’\footnote{भ. गी. ४~। १५} ‘कर्मणैव हि संसिद्धिमास्थिता जनकादयः’\footnote{भ. गी. ३~। २०} इति, तत्तु प्रविभज्य विज्ञेयम्~। तत्कथम्~? यदि तावत् पूर्वे जनकादयः तत्त्वविदोऽपि प्रवृत्तकर्माणः स्युः, ते लोकसङ्ग्रहार्थम् ‘गुणा गुणेषु वर्तन्ते’\footnote{भ. गी. ३~। २८} इति ज्ञानेनैव संसिद्धिमास्थिताः, कर्मसंन्यासे प्राप्तेऽपि कर्मणा सहैव संसिद्धिमास्थिताः, न कर्मसंन्यासं कृतवन्त इत्यर्थः~। अथ न ते तत्त्वविदः~; ईश्वरसमर्पितेन कर्मणा साधनभूतेन संसिद्धिं सत्त्वशुद्धिम्~, ज्ञानोत्पत्तिलक्षणां वा संसिद्धिम्~, आस्थिता जनकादय इति व्याख्येयम्~। एवमेवार्थं वक्ष्यति भगवान् ‘सत्त्वशुद्धये कर्म कुर्वन्ति’\footnote{भ. गी. ५~। ११} इति~। ‘स्वकर्मणा तमभ्यर्च्य सिद्धिं विन्दति मानवः’\footnote{भ. गी. १८~। ४६} इत्युक्त्वा सिद्धिं प्राप्तस्य पुनर्ज्ञाननिष्ठां वक्ष्यति — ‘सिद्धिं प्राप्तो यथा ब्रह्म’\footnote{भ. गी. १८~। ५०} इत्यादिना~॥~} 
तस्माद्गीताशास्त्रे केवलादेव तत्त्वज्ञानान्मोक्षप्राप्तिः न कर्मसमुच्चितात्~, इति निश्चितोऽर्थः~। यथा चायमर्थः, तथा प्रकरणशो विभज्य तत्र तत्र दर्शयिष्यामः~॥~} तत्रैवं धर्मसंमूढचेतसो मिथ्याज्ञानवतो महति शोकसागरे निमग्नस्य अर्जुनस्य अन्यत्रात्मज्ञानादुद्धरणमपश्यन् भगवान्वासुदेवः ततः कृपया अर्जुनमुद्दिधारयिषुः आत्मज्ञानायावतारयन्नाह —  
\begin{center}{\bfseries श्रीभगवानुवाच —\\ अशोच्यानन्वशोचस्त्वं प्रज्ञावादांश्च भाषसे~।\\गतासूनगतासूंश्च नानुशोचन्ति पण्डिताः~॥~११~॥}\end{center} 
अशोच्यान् इत्यादि~। न शोच्या अशोच्याः भीष्मद्रोणादयः, सद्वृत्तत्वात् परमार्थस्वरूपेण च नित्यत्वात्~, तान् अशोच्यान् अन्वशोचः अनुशोचितवानसि ‘ते म्रियन्ते मन्निमित्तम्~, अहं तैर्विनाभूतः किं करिष्यामि राज्यसुखादिना’ इति~। त्वं प्रज्ञावादान् प्रज्ञावतां बुद्धिमतां वादांश्च वचनानि च भाषसे | तदेतत् मौढ्यं पाण्डित्यं च विरुद्धम् आत्मनि दर्शयसि उन्मत्त इव इत्यभिप्रायः~। यस्मात् गतासून् गतप्राणान् मृतान्~, अगतासून् अगतप्राणान् जीवतश्च न अनुशोचन्ति पण्डिताः आत्मज्ञाः~। पण्डा आत्मविषया बुद्धिः येषां ते हि पण्डिताः, ‘पाण्डित्यं निर्विद्य’\footnote{बृ. उ. ३~। ५~। १} इति श्रुतेः~। परमार्थतस्तु तान् नित्यान् अशोच्यान् अनुशोचसि, अतो मूढोऽसि इत्यभिप्रायः~॥~११~॥\par
 कुतस्ते अशोच्याः, यतो नित्याः~। कथम्~? — } 
\begin{center}{\bfseries न त्वेवाहं जातु नासं न त्वं नेमे जनाधिपाः~।\\न चैव न भविष्यामः सर्वे वयमतः परम्~॥~१२~॥}\end{center} 
न तु एव जातु कदाचित् अहं नासम्~, किं तु आसमेव~। अतीतेषु देहोत्पत्तिविनाशेषु घटादिषु वियदिव नित्य एव अहमासमित्यभिप्रायः~। तथा न त्वं न आसीः, किं तु आसीरेव~। तथा न इमे जनाधिपाः न आसन्~, किं तु आसन्नेव~। तथा न च एव न भविष्यामः, किं तु भविष्याम एव, सर्वे वयम् अतः अस्मात् देहविनाशात् परम् उत्तरकाले अपि~। त्रिष्वपि कालेषु नित्या आत्मस्वरूपेण इत्यर्थः~। देहभेदानुवृत्त्या बहुवचनम्~, नात्मभेदाभिप्रायेण~॥~१२~॥\par
 तत्र कथमिव नित्य आत्मेति दृष्टान्तमाह — } 
\begin{center}{\bfseries देहिनोऽस्मिन्यथा देहे कौमारं यौवनं जरा~।\\तथा देहान्तरप्राप्तिर्धीरस्तत्र न मुह्यति~॥~१३~॥}\end{center} 
देहः अस्य अस्तीति देही, तस्य देहिनो देहवतः आत्मनः अस्मिन् वर्तमाने देहे यथा येन प्रकारेण कौमारं कुमारभावो बाल्यावस्था, यौवनं यूनो भावो मध्यमावस्था, जरा वयोहानिः जीर्णावस्था, इत्येताः तिस्रः अवस्थाः अन्योन्यविलक्षणाः~। तासां प्रथमावस्थानाशे न नाशः, द्वितीयावस्थोपजने न उपजन आत्मनः~। किं तर्हि~? अविक्रियस्यैव द्वितीयतृतीयावस्थाप्राप्तिः आत्मनो दृष्टा~। तथा तद्वदेव देहात् अन्यो देहो देहान्तरम्~, तस्य प्राप्तिः देहान्तरप्राप्तिः अविक्रियस्यैव आत्मनः इत्यर्थः~। धीरो धीमान्~, तत्र एवं सति न मुह्यति न मोहमापद्यते~॥~१३~॥\par
 यद्यपि आत्मविनाशनिमित्तो मोहो न सम्भवति नित्य आत्मा इति विजानतः, तथापि शीतोष्णसुखदुःखप्राप्तिनिमित्तो मोहो लौकिको दृश्यते, सुखवियोगनिमित्तो मोहः दुःखसंयोगनिमित्तश्च शोकः~। इत्येतदर्जुनस्य वचनमाशङ्क्य भगवानाह —} 
\begin{center}{\bfseries मात्रास्पर्शास्तु कौन्तेय शीतोष्णसुखदुःखदाः~।\\आगमापायिनोऽनित्यास्तांस्तितिक्षस्व भारत~॥~१४~॥}\end{center} 
मात्राः आभिः मीयन्ते शब्दादय इति श्रोत्रादीनि इन्द्रियाणि~। मात्राणां स्पर्शाः शब्दादिभिः संयोगाः~। ते शीतोष्णसुखदुःखदाः शीतम् उष्णं सुखं दुःखं च प्रयच्छन्तीति~। अथवा स्पृश्यन्त इति स्पर्शाः विषयाः शब्दादयः~। मात्राश्च स्पर्शाश्च शीतोष्णसुखदुःखदाः~। शीतं कदाचित् सुखं कदाचित् दुःखम्~। तथा उष्णमपि अनियतस्वरूपम्~। सुखदुःखे पुनः नियतरूपे यतो न व्यभिचरतः~। अतः ताभ्यां पृथक् शीतोष्णयोः ग्रहणम्~। यस्मात् ते मात्रास्पर्शादयः आगमापायिनः आगमापायशीलाः तस्मात् अनित्याः~। अतः तान् शीतोष्णादीन् तितिक्षस्व प्रसहस्व~। तेषु हर्षं विषादं वा मा कार्षीः इत्यर्थः~॥~१४~॥\par
 शीतोष्णादीन् सहतः किं स्यादिति शृणु — } 
\begin{center}{\bfseries यं हि न व्यथयन्त्येते पुरुषं पुरुषर्षभ~।\\ समदुःखसुखं धीरं सोऽमृतत्वाय कल्पते~॥~१५~॥}\end{center} 
यं हि पुरुषं समे दुःखसुखे यस्य तं समदुःखसुखं सुखदुःखप्राप्तौ हर्षविषादरहितं धीरं धीमन्तं न व्यथयन्ति न चालयन्ति नित्यात्मदर्शनात् एते यथोक्ताः शीतोष्णादयः, सः नित्यात्मस्वरूपदर्शननिष्ठो द्वन्द्वसहिष्णुः अमृतत्वाय अमृतभावाय मोक्षायेत्यर्थः, कल्पते समर्थो भवति~॥~१५~॥\par
 इतश्च शोकमोहौ अकृत्वा सीतोष्णादिसहनं युक्तम्~, यस्मात् — } 
\begin{center}{\bfseries नासतो विद्यते भावो नाभावो विद्यते सतः~।\\उभयोरपि दृष्टोऽन्तस्त्वनयोस्तत्त्वदर्शिभिः~॥~१६~॥}\end{center} 
न असतः अविद्यमानस्य शीतोष्णादेः सकारणस्य न विद्यते नास्ति भावो भवनम् अस्तिता~॥~} 
न हि शीतोष्णादि सकारणं प्रमाणैर्निरूप्यमाणं वस्तुसद्भवति~। विकारो हि सः, विकारश्च व्यभिचरति~। यथा घटादिसंस्थानं चक्षुषा निरूप्यमाणं मृद्व्यतिरेकेणानुपलब्धेरसत्~, तथा सर्वो विकारः कारणव्यतिरेकेणानुपलब्धेरसन्~। जन्मप्रध्वंसाभ्यां प्रागूर्ध्वं च अनुपलब्धेः कार्यस्य घटादेः मृदादिकारणस्य च तत्कारणव्यतिरेकेणानुपलब्धेरसत्त्वम्~॥~} 
तदसत्त्वे सर्वाभावप्रसङ्ग इति चेत्~, न~; सर्वत्र बुद्धिद्वयोपलब्धेः, सद्बुद्धिरसद्बुद्धिरिति~। यद्विषया बुद्धिर्न व्यभिचरति, तत् सत्~; यद्विषया व्यभिचरति, तदसत्~; इति सदसद्विभागे बुद्धितन्त्रे स्थिते, सर्वत्र द्वे बुद्धी सर्वैरुपलभ्येते समानाधिकरणे न नीलोत्पलवत्~, सन् घटः, सन् पटः, सन् हस्ती इति~। एवं सर्वत्र तयोर्बुद्ध्योः घटादिबुद्धिः व्यभिचरति~। तथा च दर्शितम्~। न तु सद्बुद्धिः~। तस्मात् घटादिबुद्धिविषयः असन्~, व्यभिचारात्~; न तु सद्बुद्धिविषयः, अव्यभिचारात्~॥~} 
घटे विनष्टे घटबुद्दौ व्यभिचरन्त्यां सद्बुद्धिरपि व्यभिचरतीति चेत्~, न~; पटादावपि सद्बुद्धिदर्शनात्~। विशेषणविषयैव सा सद्बुद्धिः~॥~} 
सद्बुद्धिवत् घटबुद्धिरपि घटान्तरे दृश्यत इति चेत्~, न~; पटादौ अदर्शनात्~॥~} 
सद्बुद्धिरपि नष्टे घटे न दृश्यत इति चेत्~, न~; विशेष्याभावात् सद्बुद्धिः विशेषणविषया सती विशेष्याभावे विशेषणानुपपत्तौ किंविषया स्यात्~? न तु पुनः सद्बुद्धेः विषयाभावात्~॥~} 
एकाधिकरणत्वं घटादिविशेष्याभावे न युक्तमिति चेत्~, न~; ‘इदमुदकम्’ इति मरीच्यादौ अन्यतराभावेऽपि सामानाधिकरण्यदर्शनात्~॥~} 
तस्माद्देहादेः द्वन्द्वस्य च सकारणस्य असतो न विद्यते भाव इति~। तथा सतश्च आत्मनः अभावः अविद्यमानता न विद्यते, सर्वत्र अव्यभिचारात् इति अवोचाम~॥~} 
एवम् आत्मानात्मनोः सदसतोः उभयोरपि दृष्टः उपलब्धः अन्तो निर्णयः सत् सदेव असत् असदेवेति, तु अनयोः यथोक्तयोः तत्त्वदर्शिभिः~। तदिति सर्वनाम, सर्वं च ब्रह्म, तस्य नाम तदिति, तद्भावः तत्त्वम्~, ब्रह्मणो याथात्म्यम्~। तत् द्रष्टुं शीलं येषां ते तत्त्वदर्शिनः, तैः तत्त्वदर्शिभिः~। त्वमपि तत्त्वदर्शिनां दृष्टिमाश्रित्य शोकं मोहं च हित्वा शीतोष्णादीनि नियतानियतरूपाणि द्वन्द्वानि ‘विकारोऽयमसन्नेव मरीचिजलवन्मिथ्यावभासते’ इति मनसि निश्चित्य तितिक्षस्व इत्यभिप्रायः~॥~१६~॥\par
 किं पुनस्तत्~, यत् सदेव सर्वदा इति~; उच्यते —} 
\begin{center}{\bfseries अविनाशि तु तद्विद्धि येन सर्वमिदं ततम्~।\\विनाशमव्ययस्यास्य न कश्चित्कर्तुमर्हति~॥~१७~॥}\end{center} 
अविनाशि न विनष्टुं शीलं यस्येति~। तुशब्दः असतो विशेषणार्थः~। तत् विद्धि विजानीहि~। किम्~? येन सर्वम् इदं जगत् ततं व्याप्तं सदाख्येन ब्रह्मणा साकाशम्~, आकाशेनेव घटादयः~। विनाशम् अदर्शनम् अभावम्~। अव्ययस्य न व्येति उपचयापचयौ न याति इति अव्ययं तस्य अव्ययस्य~। नैतत् सदाख्यं ब्रह्म स्वेन रूपेण व्येति व्यभिचरति, निरवयवत्वात्~, देहादिवत्~। नाप्यात्मीयेन, आत्मीयाभावात्~। यथा देवदत्तो धनहान्या व्येति, न तु एवं ब्रह्म व्येति~। अतः अव्ययस्य अस्य ब्रह्मणः विनाशं न कश्चित् कर्तुमर्हति, न कश्चित् आत्मानं विनाशयितुं शक्नोति ईश्वरोऽपि~। आत्मा हि ब्रह्म, स्वात्मनि च क्रियाविरोधात्~॥~१७~॥\par
 किं पुनस्तदसत्~, यत्स्वात्मसत्तां व्यभिचरतीति, उच्यते — } 
\begin{center}{\bfseries अन्तवन्त इमे देहा नित्यस्योक्ताः शरीरिणः~।\\अनाशिनोऽप्रमेयस्य तस्माद्युध्यस्व भारत~॥~१८~॥}\end{center} 
अन्तः विनाशः विद्यते येषां ते अन्तवन्तः~। यथा मृगतृष्णिकादौ सद्बुद्धिः अनुवृत्ता प्रमाणनिरूपणान्ते विच्छिद्यते, स तस्य अन्तः~; तथा इमे देहाः स्वप्नमायादेहादिवच्च अन्तवन्तः नित्यस्य शरीरिणः शरीरवतः अनाशिनः अप्रमेयस्य आत्मनः अन्तवन्त इति उक्ताः विवेकिभिरित्यर्थः~। ‘नित्यस्य’ ‘अनाशिनः’ इति न पुनरुक्तम्~; नित्यत्वस्य द्विविधत्वात् लोके, नाशस्य च~। यथा देहो भस्मीभूतः अदर्शनं गतो नष्ट उच्यते~। विद्यमानोऽपि यथा अन्यथा परिणतो व्याध्यादियुक्तो जातो नष्ट उच्यते~। तत्र ‘नित्यस्य’ ‘अनाशिनः’ इति द्विविधेनापि नाशेन असम्बन्धः अस्येत्यर्थः~। अन्यथा पृथिव्यादिवदपि नित्यत्वं स्यात् आत्मनः~; तत् मा भूदिति ‘नित्यस्य’ ‘अनाशिनः’ इत्याह~। अप्रमेयस्य न प्रमेयस्य प्रत्यक्षादिप्रमाणैः अपरिच्छेद्यस्येत्यर्थः~॥~} 
ननु आगमेन आत्मा परिच्छिद्यते, प्रत्यक्षादिना च पूर्वम्~। न~; आत्मनः स्वतःसिद्धत्वात्~। सिद्धे हि आत्मनि प्रमातरि प्रमित्सोः प्रमाणान्वेषणा भवति~। न हि पूर्वम् ‘इत्थमहम्’ इति आत्मानमप्रमाय पश्चात् प्रमेयपरिच्छेदाय प्रवर्तते~। न हि आत्मा नाम कस्यचित् अप्रसिद्धो भवति~। शास्त्रं तु अन्त्यं प्रमाणम् अतद्धर्माध्यारोपणमात्रनिवर्तकत्वेन प्रमाणत्वम् आत्मनः प्रतिपद्यते, न तु अज्ञातार्थज्ञापकत्वेन~। तथा च श्रुतिः — ‘यत्साक्षादपरोक्षाद्ब्रह्म य आत्मा सर्वान्तरः’\footnote{बृ. उ. ३~। ५~। १} इति~॥~} 
यस्मादेवं नित्यः अविक्रियश्च आत्मा तस्मात् युध्यस्व, युद्धात् उपरमं मा कार्षीः इत्यर्थः~॥~} 
न हि अत्र युद्धकर्तव्यता विधीयते, युद्धे प्रवृत्त एव हि असौ शोकमोहप्रतिबद्धः तूष्णीमास्ते~। अतः तस्य प्रतिबन्धापनयनमात्रं भगवता क्रियते~। तस्मात् ‘युध्यस्व’ इति अनुवादमात्रम्~, न विधिः~॥~१८~॥\par
 शोकमोहादिसंसारकारणनिवृत्त्यर्थः गीताशास्त्रम्~, न प्रवर्तकम् इत्येतस्यार्थस्य साक्षिभूते ऋचौ आनीनाय भगवान्~। यत्तु मन्यसे ‘युद्धे भीष्मादयो मया हन्यन्ते’ ‘अहमेव तेषां हन्ता’ इति, एषा बुद्धिः मृषैव ते~। कथम्~? —} 
\begin{center}{\bfseries य एनं वेत्ति हन्तारं यश्चैनं मन्यते हतम्~।\\उभौ तौ न विजानीतो नायं हन्ति न हन्यते~॥~१९~॥}\end{center} 
य एनं प्रकृतं देहिनं वेत्ति विजानाति हन्तारं हननक्रियायाः कर्तारं यश्च एनम् अन्यो मन्यते हतं देहहननेन ‘हतः अहम्’ इति हननक्रियायाः कर्मभूतम्~, तौ उभौ न विजानीतः न ज्ञातवन्तौ अविवेकेन आत्मानम्~। ‘हन्ता अहम्’ ‘हतः अस्ति अहम्’ इति देहहननेन आत्मानमहं प्रत्ययविषयं यौ विजानीतः तौ आत्मस्वरूपानभिज्ञौ इत्यर्थः~। यस्मात् न अयम् आत्मा हन्ति न हननक्रियायाः कर्ता भवति, न च हन्यते न च कर्म भवतीत्यर्थः, अविक्रियत्वात्~॥~१९~॥\par
 कथमविक्रय आत्मेति द्वितीयो मन्त्रः —} 
\begin{center}{\bfseries न जायते म्रियते वा कदाचिन्नायं भूत्वाभविता वा न भूयः~।\\अजो नित्यः शाश्वतोऽयं पुराणो न हन्यते हन्यमाने शरीरे~॥~२०~॥}\end{center} 
न जायते न उत्पद्यते, जनिलक्षणा वस्तुविक्रिया न आत्मनो विद्यते इत्यर्थः~। तथा न म्रियते वा~। वाशब्दः चार्थे~। न म्रियते च इति अन्त्या विनाशलक्षणा विक्रिया प्रतिषिध्यते~। कदाचिच्छब्दः सर्वविक्रियाप्रतिषेधैः सम्बध्यते — न कदाचित् जायते, न कदाचित् म्रियते, इत्येवम्~। यस्मात् अयम् आत्मा भूत्वा भवनक्रियामनुभूय पश्चात् अभविता अभावं गन्ता न भूयः पुनः, तस्मात् न म्रियते~। योहि भूत्वा न भविता स म्रियत इत्युच्यते लोके~। वाशब्दात् नशब्दाच्च अयमात्मा अभूत्वा वा भविता देहवत् न भूयः~। तस्मात् न जायते~। यो हि अभूत्वा भविता स जायत इत्युच्यते~। नैवमात्मा~। अतो न जायते~। यस्मादेवं तस्मात् अजः, यस्मात् न म्रियते तस्मात् नित्यश्च~। यद्यपि आद्यन्तयोर्विक्रिययोः प्रतिषेधे सर्वा विक्रियाः प्रतिषिद्धा भवन्ति, तथापि मध्यभाविनीनां विक्रियाणां स्वशब्दैरेव प्रतिषेधः कर्तव्यः अनुक्तानामपि यौवनादिसमस्तविक्रियाणां प्रतिषेधो यथा स्यात् इत्याह — शाश्वत इत्यादिना~। शाश्वत इति अपक्षयलक्षणा विक्रिया प्रतिषिध्यते~। शश्वद्भवः शाश्वतः~। न अपक्षीयते स्वरूपेण, निरवयवत्वात्~। नापि गुणक्षयेण अपक्षयः, निर्गुणत्वात्~। अपक्षयविपरीतापि वृद्धिलक्षणा विक्रिया प्रतिषिध्यते — पुराण इति~। यो हि अवयवागमेन उपचीयते स वर्धते अभिनव इति च उच्यते~। अयं तु आत्मा निरवयवत्वात् पुरापि नव एवेति पुराणः~; न वर्धते इत्यर्थः~। तथा न हन्यते~। हन्ति~; अत्र विपरिणामार्थे द्रष्टव्यः अपुनरुक्ततायै~। न विपरिणम्यते इत्यर्थः~। हन्यमाने विपरिणम्यमानेऽपि शरीरे~। अस्मिन् मन्त्रे षड् भावविकारा लौकिकवस्तुविक्रिया आत्मनि प्रतिषिध्यन्ते~। सर्वप्रकारविक्रियारहित आत्मा इति वाक्यार्थः~। यस्मादेवं तस्मात् ‘उभौ तौ न विजानीतः’ इति पूर्वेण मन्त्रेण अस्य सम्बन्धः~॥~२०~॥\par
 ‘य एनं वेत्ति हन्तारम्’\footnote{भ. गी. २~। १९} इत्यनेन मन्त्रेण हननक्रियायाः कर्ता कर्म च न भवति इति प्रतिज्ञाय, ‘न जायते’ इत्यनेन अविक्रियत्वं हेतुमुक्त्वा प्रतिज्ञातार्थमुपसंहरति —} 
\begin{center}{\bfseries वेदाविनाशिनं नित्यं य एनमजमव्ययम्~।\\कथं स पुरुषः पार्थ कं घातयति हन्ति कम्~॥~२१~॥}\end{center} 
वेद विजानाति अविनाशिनम् अन्त्यभावविकाररहितं नित्यं विपरिणामरहितं यो वेद इति सम्बन्धः~। एनं पूर्वेण मन्त्रेणोक्तलक्षणम् अजं जन्मरहितम् अव्ययम् अपक्षयरहितं कथं केन प्रकारेण सः विद्वान् पुरुषः अधिकृतः हन्ति हननक्रियां करोति, कथं वा घातयति हन्तारं प्रयोजयति~। न कथञ्चित् कञ्चित् हन्ति, न कथञ्चित् कञ्चित् घातयति इति उभयत्र आक्षेप एवार्थः, प्रश्नार्थासम्भवात्~। हेत्वर्थस्य च अविक्रियत्वस्य तुल्यत्वात् विदुषः सर्वकर्मप्रतिषेध एव प्रकारणार्थः अभिप्रेतो भगवता~। हन्तेस्तु आक्षेपः उदाहरणार्थत्वेन कथितः~॥~} 
विदुषः कं कर्मासम्भवहेतुविशेषं पश्यन् कर्माण्याक्षिपति भगवान् ‘कथं स पुरुषः’ इति~। ननु उक्त एवात्मनः अविक्रियत्वं सर्वकर्मासम्भवकारणविशेषः~। सत्यमुक्तः~। न तु सः कारणविशेषः, अन्यत्वात् विदुषः अविक्रियादात्मनः~। न हि अविक्रियं स्थाणुं विदितवतः कर्म न सम्भवति इति चेत्~, न~; विदुष— आत्मत्वात्~। न देहादिसङ्घातस्य विद्वत्ता~। अतः पारिशेष्यात् असंहतः आत्मा विद्वान् अविक्रियः इति तस्य विदुषः कर्मासम्भवात् आक्षेपो युक्तः ‘कथं स पुरुषः’ इति~। यथा बुद्ध्याद्याहृतस्य शब्दाद्यर्थस्य अविक्रिय एव सन् बुद्धिवृत्त्यविवेकविज्ञानेन अविद्यया उपलब्धा आत्मा कल्प्यते, एवमेव आत्मानात्मविवेकज्ञानेन बुद्धिवृत्त्या विद्यया असत्यरूपयैव परमार्थतः अविक्रिय एव आत्मा विद्वानुच्यते~। विदुष— कर्मासम्भववचनात् यानि कर्माणि शास्त्रेण विधीयन्ते तानि अविदुषो विहितानि इति भगवतो निश्चयोऽवगम्यते~॥~} 
ननु विद्यापि अविदुष एव विधीयते, विदितविद्यस्य पिष्टपेषणवत् विद्याविधानानर्थक्यात्~। तत्र अविदुषः कर्माणि विधीयन्ते न विदुषः इति विशेषो नोपपद्यते इति चेत्~, न~; अनुष्ठेयस्य भावाभावविशेषोपपत्तेः~। अग्निहोत्रादिविध्यर्थज्ञानोत्तरकालम् अग्निहोत्रादिकर्म अनेकसाधनोपसंहारपूर्वकमनुष्ठेयम् ‘कर्ता अहम्~, मम कर्तव्यम्’ इत्येवंप्रकारविज्ञानवतः अविदुषः यथा अनुष्ठेयं भवति, न तु तथा ‘न जायते’ इत्याद्यात्मस्वरूपविध्यर्थज्ञानोत्तरकालभावि किञ्चिदनुष्ठेयं भवति~; किं तु ‘नाहं कर्ता, नाहं भोक्ता’ इत्याद्यात्मैकत्वाकर्तृत्वादिविषयज्ञानात् नान्यदुत्पद्यते इति एष विशेष उपपद्यते~। यः पुनः ‘कर्ता अहम्’ इति वेत्ति आत्मानम्~, तस्य ‘मम इदं कर्तव्यम्’ इति अवश्यंभाविनी बुद्धिः स्यात्~; तदपेक्षया सः अधिक्रियते इति तं प्रति कर्माणि सम्भवन्ति~। स च अविद्वान्~, ‘उभौ तौ न विजानीतः’\footnote{भ. गी. २~। १९} इति वचनात्~, विशेषितस्य च विदुषः कर्माक्षेपवचनाच्च ‘कथं स पुरुषः’ इति~। तस्मात् विशेषितस्य अविक्रियात्मदर्शिनः विदुषः मुमुक्षोश्च सर्वकर्मसंन्यासे एव अधिकारः~। अत एव भगवान् नारायणः साङ्ख्यान् विदुषः अविदुषश्च कर्मिणः प्रविभज्य द्वे निष्ठे ग्राहयति — ‘ज्ञानयोगेन साङ्‍ख्यानां कर्मयोगेन योगिनाम्’\footnote{भ. गी. ३~। ३} इति~। तथा च पुत्राय आह भगवान् व्यासः — ‘द्वाविमावथ पन्थानौ’\footnote{शां. २४१~। ६} इत्यादि~। तथा च क्रियापथश्चैव पुरस्तात् पश्चात्संन्यासश्चेति~। एतमेव विभागं पुनः पुनर्दर्शयिष्यति भगवान् — अतत्त्ववित् ‘अहङ्कारविमूढात्मा कर्ताहमिति मन्यते’\footnote{भ. गी. ३~। २७}, तत्त्ववित्तु नाहं करोमि इति~। तथा च ‘सर्वकर्माणि मनसा संन्यस्यास्ते’\footnote{भ. गी. ५~। १३} इत्यादि~॥~} 
तत्र केचित्पण्डितंमन्या वदन्ति — ‘जन्मादिषड्भावविक्रियारहितः अविक्रियः अकर्ता एकः अहमात्मा’ इति न कस्यचित् ज्ञानम् उत्पद्यते, यस्मिन् सति सर्वकर्मसंन्यासः उपदिश्यते इति~। तन्न~; ‘न जायते’\footnote{भ. गी. २~। २०} इत्यादिशास्त्रोपदेशानर्थक्यप्रसङ्गात्~। यथा च शास्त्रोपदेशसामर्थ्यात् धर्माधर्मास्तित्वविज्ञानं कर्तुश्च देहान्तरसम्बन्धविज्ञानमुत्पद्यते, तथा शास्त्रात् तस्यैव आत्मनः अविक्रियत्वाकर्तृत्वैकत्वादिविज्ञानं कस्मात् नोत्पद्यते इति प्रष्टव्याः ते~। करणागोचरत्वात् इति चेत्~, न~; ‘मनसैवानुद्रष्टव्यम्’\footnote{बृ. उ. ४~। ४~। १९} इति श्रुतेः~। शास्त्राचार्योपदेशशमदमादिसंस्कृतं मनः आत्मदर्शने करणम्~। तथा च तदधिगमाय अनुमाने आगमे च सति ज्ञानं नोत्पद्यत इति साहसमात्रमेतत्~। ज्ञानं च उत्पद्यमानं तद्विपरीतमज्ञानम् अवश्यं बाधते इत्यभ्युपगन्तव्यम्~। तच्च अज्ञानं दर्शितम् ‘हन्ता अहम्~, हतः अस्मि’ इति उभौ तौ न विजानीतः’ इति~। अत्र च आत्मनः हननक्रियायाः कर्तृत्वं कर्मत्वं हेतुकर्तृत्वं च अज्ञानकृतं दर्शितम्~। तच्च सर्वक्रियास्वपि समानं कर्तृत्वादेः अविद्याकृतत्वम्~, अविक्रियत्वात् आत्मनः~। विक्रियावान् हि कर्ता आत्मनः कर्मभूतमन्यं प्रयोजयति ‘कुरु’ इति~। तदेतत् अविशेषेण विदुषः सर्वक्रियासु कर्तृत्वं हेतुकर्तृत्वं च प्रतिषेधति भगवान्वासुदेवः विदुषः कर्माधिकाराभावप्रदर्शनार्थम् ‘वेदाविनाशिनं . . . कथं स पुरुषः’ इत्यादिना~। क्व पुनः विदुषः अधिकार इति एतदुक्तं पूर्वमेव ‘ज्ञानयोगेन साङ्ख्यानाम्’\footnote{भ. गी. ३~। ३} इति~। तथा च सर्वकर्मसंन्यासं वक्ष्यति ‘सर्वकर्माणि मनसा’\footnote{भ. गी. ५~। १३} इत्यादिना~॥~} 
ननु मनसा इति वचनात् न वाचिकानां कायिकानां च संन्यासः इति चेत्~, न~; सर्वकर्माणि इति विशेषितत्वात्~। मानसानामेव सर्वकर्मणामिति चेत्~, न~; मनोव्यापारपूर्वकत्वाद्वाक्कायव्यापाराणां मनोव्यापाराभावे तदनुपपत्तेः~। शास्त्रीयाणां वाक्कायकर्मणां कारणानि मानसानि कर्माणि वर्जयित्वा अन्यानि सर्वकर्माणि मनसा संन्यस्येदिति चेत्~, न~; ‘नैव कुर्वन्न कारयन्’\footnote{भ. गी. ५~। १३} इति विशेषणात्~। सर्वकर्मसंन्यासः अयं भगवता उक्तः मरिष्यतः न जीवतः इति चेत्~, न~; ‘नवद्वारे पुरे देही आस्ते’\footnote{भ. गी. ५~। १३} इति विशेषणानुपपत्तेः~। न हि सर्वकर्मसंन्यासेन मृतस्य तद्देहे आसनं सम्भवति~। अकुर्वतः अकारयतश्च देहे संन्यस्य इति सम्बन्धः न देहे आस्ते इति चेत्~, न~; सर्वत्र आत्मनः अविक्रियत्वावधारणात्~, आसनक्रियायाश्च अधिकरणापेक्षत्वात्~, तदनपेक्षत्वाच्च संन्यासस्य~। सम्पूर्वस्तु न्यासशब्दः अत्र त्यागार्थः, न निक्षेपार्थः~। तस्मात् गीताशास्त्रे आत्मज्ञानवतः संन्यासे एव अधिकारः, न कर्मणि इति तत्र तत्र उपरिष्टात् आत्मज्ञानप्रकरणे दर्शयिष्यामः~॥~२१~॥\par
 प्रकृतं तु वक्ष्यामः~। तत्र आत्मनः अविनाशित्वं प्रतिज्ञातम्~। तत् किमिवेति, उच्यते —} 
\begin{center}{\bfseries वासांसि जीर्णानि यथा विहाय नवानि गृह्णाति नरोऽपराणि~।\\तथा शरीराणि विहाय जीर्णान्यन्यानि संयाति नवानि देही~॥~२२~॥}\end{center} 
वासांसि वस्त्राणि जीर्णानि दुर्बलतां गतानि यथा लोके विहाय परित्यज्य नवानि अभिनवानि गृह्णाति उपादत्ते नरः पुरुषः अपराणि अन्यानि, तथा तद्वदेव शरीराणि विहाय जीर्णानि अन्यानि संयाति सङ्गच्छति नवानि देही आत्मा पुरुषवत् अविक्रिय एवेत्यर्थः~॥~२२~॥\par
 कस्मात् अविक्रिय एवेति, आह —} 
\begin{center}{\bfseries नैनं छिन्दन्ति शस्त्राणि नैनं दहति पावकः~।\\न चैनं क्लेदयन्त्यापो न शोषयति मारुतः~॥~२३~॥}\end{center} 
एनं प्रकृतं देहिनं न च्छिन्दन्ति शस्त्राणि, निरवयवत्वात् न अवयवविभागं कुर्वन्ति~। शस्त्राणि अस्यादीनि~। तथा न एनं दहति पावकः, अग्निरपि न भस्मीकरोति~। तथा न च एनं क्लेदयन्ति आपः~। अपां हि सावयवस्य वस्तुनः आर्द्रीभावकरणेन अवयवविश्लेषापादने सामर्थ्यम्~। तत् न निरवयवे आत्मनि सम्भवति~। तथा स्नेहवत् द्रव्यं स्नेहशोषणेन नाशयति वायुः~। एनं तु आत्मानं न शोषयति मारुतोऽपि~॥~२३~॥\par
 यतः एवं तस्मात् — } 
\begin{center}{\bfseries अच्छेद्योऽयमदाह्योऽयमक्लेद्योऽशोष्य एव च~।\\नित्यः सर्वगतः स्थाणुरचलोऽयं सनातनः~॥~२४~॥}\end{center} 
यस्मात् अन्योन्यनाशहेतुभूतानि एनमात्मानं नाशयितुं नोत्सहन्ते अस्यादीनि तस्मात् नित्यः~। नित्यत्वात् सर्वगतः~। सर्वगतत्वात् स्थाणुः इव, स्थिर इत्येतत्~। स्थिरत्वात् अचलः अयम् आत्मा~। अतः सनातनः चिरन्तनः, न कारणात्कुतश्चित् निष्पन्नः, अभिनव इत्यर्थः~॥~} 
नैतेषां श्लोकानां पौनरुक्त्यं चोदनीयम्~, यतः एकेनैव श्लोकेन आत्मनः नित्यत्वमविक्रियत्वं चोक्तम् ‘न जायते म्रियते वा’\footnote{भ. गी. २~। २०} इत्यादिना~। तत्र यदेव आत्मविषयं किञ्चिदुच्यते, तत् एतस्मात् श्लोकार्थात् न अतिरिच्यते~; किञ्चिच्छब्दतः पुनरुक्तम्~, किञ्चिदर्थतः इति~। दुर्बोधत्वात् आत्मवस्तुनः पुनः पुनः प्रसङ्गमापाद्य शब्दान्तरेण तदेव वस्तु निरूपयति भगवान् वासुदेवः कथं नु नाम संसारिणामसंसारित्वबुद्धिगोचरतामापन्नं सत् अव्यक्तं तत्त्वं संसारनिवृत्तये स्यात् इति~॥~२४~॥\par
 किं च —} 
\begin{center}{\bfseries अव्यक्तोऽयमचिन्त्योऽयमविकार्योऽयमुच्यते~।\\तस्मादेवं विदित्वैनं नानुशोचितुमर्हसि~॥~२५~॥}\end{center} 
सर्वकरणाविषयत्वात् न व्यज्यत इति अव्यक्तः अयम् आत्मा~। अत एव अचिन्त्यः अयम्~। यद्धि इन्द्रियगोचरः तत् चिन्ताविषयत्वमापद्यते~। अयं त्वात्मा अनिन्द्रियगोचरत्वात् अचिन्त्यः~। अत एव अविकार्यः, यथा क्षीरं दध्यातञ्चनादिना विकारि न तथा अयमात्मा~। निरवयवत्वाच्च अविक्रियः~। न हि निरवयवं किञ्चित् विक्रियात्मकं दृष्टम्~। अविक्रियत्वात् अविकार्यः अयम् आत्मा उच्यते~। तस्मात् एवं यथोक्तप्रकारेण एनम् आत्मानं विदित्वा } त्वं न अनुशोचितुमर्हसि हन्ताहमेषाम्~, मयैते हन्यन्त इति~॥~२५~॥\par
 आत्मनः अनित्यत्वमभ्युपगम्य इदमुच्यते —} 
\begin{center}{\bfseries अथ चैनं नित्यजातं नित्यं वा मन्यसे मृतम्~।\\तथापि त्वं महाबाहो नैवं शोचितुमर्हसि~॥~२६~॥}\end{center} 
अथ च इति अभ्युपगमार्थः~। एनं प्रकृतमात्मानं नित्यजातं लोकप्रसिद्ध्या प्रत्यनेकशरीरोत्पत्ति जातो जात इति मन्यसे तथा प्रतितत्तद्विनाशं नित्यं वा मन्यसे मृतं मृतो मृत इति~; तथापि तथाभावेऽपि आत्मनि त्वं महाबाहो, न एवं शोचितुमर्हसि, जन्मवतो जन्म नाशवतो नाशश्चेत्येताववश्यम्भाविनाविति~॥~२६~॥\par
 तथा च सति —} 
\begin{center}{\bfseries जातस्य हि ध्रुवो मृत्युर्ध्रुवं जन्म मृतस्य च~।\\तस्मादपरिहार्येऽर्थे न त्वं शोचितुमर्हसि~॥~२७~॥}\end{center} 
जातस्य हि लब्धजन्मनः ध्रुवः अव्यभिचारी मृत्युः मरणं ध्रुवं जन्म मृतस्य च~। तस्मादपरिहार्योऽयं जन्ममरणलक्षणोऽर्थः~। तस्मिन्नपरिहार्येऽर्थे न त्वं शोचितुमर्हसि~॥~२७~॥\par
 कार्यकरणसङ्घातात्मकान्यपि भूतान्युद्दिश्य शोको न युक्तः कर्तुम्~, यतः —} 
\begin{center}{\bfseries अव्यक्तादीनि भूतानि व्यक्तमध्यानि भारत~।\\अव्यक्तनिधनान्येव तत्र का परिदेवना~॥~२८~॥}\end{center} 
अव्यक्तादीनि अव्यक्तम् अदर्शनम् अनुपलब्धिः आदिः येषां भूतानां पुत्रमित्रादिकार्यकरणसङ्घातात्मकानां तानि अव्यक्तादीनि भूतानि प्रागुत्पत्तेः, उत्पन्नानि च प्राङ्मरणात् व्यक्तमध्यानि~। अव्यक्तनिधनान्येव पुनः अव्यक्तम् अदर्शनं निधनं मरणं येषां तानि अव्यक्तनिधनानि~। मरणादूर्ध्वमप्यव्यक्ततामेव प्रतिपद्यन्ते इत्यर्थः~। तथा चोक्तम् — ‘अदर्शनादापतितः पुनश्चादर्शनं गतः~। नासौ तव न तस्य त्वं वृथा का परिदेवना’\footnote{मो. ध. १७४~। १७} इति~। तत्र का परिदेवना को वा प्रलापः अदृष्टदृष्टप्रनष्टभ्रान्तिभूतेषु भूतेष्वित्यर्थः~॥~२८~॥\par
 दुर्विज्ञेयोऽयं प्रकृत आत्मा~; किं त्वामेवैकमुपालभे साधारणे भ्रान्तिनिमित्ते~। कथं दुर्विज्ञेयोऽयमात्मा इत्यत आह —} 
\begin{center}{\bfseries आश्चर्यवत्पश्यति कश्चिदेनमाश्चर्यवद्वदति तथैव चान्यः~।\\आश्चर्यवच्चैनमन्यः शृणोति श्रुत्वाप्येनं वेद न चैव कश्चित्~॥~२९~॥}\end{center} 
आश्चर्यवत् आश्चर्यम् अदृष्टपूर्वम् अद्भुतम् अकस्माद्दृश्यमानं तेन तुल्यं आश्चर्यवत् आश्चर्यमिति एनम् आत्मानं पश्यति कश्चित्~। आश्चर्यवत् एनं वदति तथैव च अन्यः~। आश्चर्यवच्च एनमन्यः शृणोति~। श्रुत्वा दृष्ट्वा उक्त्वापि एनमात्मानं वेद न चैव कश्चित्~। अथवा योऽयमात्मानं पश्यति स आश्चर्यतुल्यः, यो वदति यश्च शृणोति सः अनेकसहस्रेषु कश्चिदेव भवति~। अतो दुर्बोध आत्मा इत्यभिप्रायः~॥~२९~॥\par
 अथेदानीं प्रकरणार्थमुपसंहरन्ब्रूते —} 
\begin{center}{\bfseries देही नित्यमवध्योऽयं देहे सर्वस्य भारत~।\\तस्मात्सर्वाणि भूतानि न त्वं शोचितुमर्हसि~॥~३०~॥}\end{center} 
देही शरीरी नित्यं सर्वदा सर्वावस्थासु अवध्यः निरवयवत्वान्नित्यत्वाच्च तत्र अवध्योऽयं देहे शरीरे सर्वस्य सर्वगतत्वात्स्थावरादिषु स्थितोऽपि सर्वस्य प्राणिजातस्य देहे वध्यमानेऽपि अयं देही न वध्यः यस्मात्~, तस्मात् भीष्मादीनि सर्वाणि भूतानि उद्दिश्य न त्वं शोचितुमर्हसि~॥~३०~॥\par
 इह परमार्थतत्त्वापेक्षायां शोको मोहो वा न सम्भवतीत्युक्तम्~। न केवलं परमार्थतत्त्वापेक्षायामेव~। किं तु —} 
\begin{center}{\bfseries स्वधर्ममपि चावेक्ष्य न विकम्पितुमर्हसि~।\\धर्म्याद्धि युद्धाच्छ्रेयोऽन्यत्क्षत्त्रियस्य न विद्यते~॥~३१~॥}\end{center} 
स्वधर्ममपि स्वो धर्मः क्षत्रियस्य युद्धं तमपि अवेक्ष्य त्वं न विकम्पितुं प्रचलितुम् नार्हसि क्षत्रियस्य स्वाभाविकाद्धर्मात् आत्मस्वाभाव्यादित्यभिप्रायः~। तच्च युद्धं पृथिवीजयद्वारेण धर्मार्थं प्रजारक्षणार्थं चेति धर्मादनपेतं परं धर्म्यम्~। तस्मात् धर्म्यात् युद्धात् श्रेयः अन्यत् क्षत्रियस्य न विद्यते हि यस्मात्~॥~३१~॥\par
 कुतश्च तत् युद्धं कर्तव्यमिति, उच्यते —} 
\begin{center}{\bfseries यदृच्छया चोपपन्नं स्वर्गद्वारमपावृतम्~।\\सुखिनः क्षत्रियाः पार्थ लभन्ते युद्धमीदृशम्~॥~३२~॥}\end{center} 
यदृच्छया च अप्रार्थिततया उपपन्नम् आगतं स्वर्गद्वारम् अपावृतम् उद्धाटितं ये एतत् ईदृशं युद्धं लभन्ते क्षत्रियाः हे पार्थ, किं न सुखिनः ते~?~॥~३२~॥\par
 एवं कर्तव्यताप्राप्तमपि —} 
\begin{center}{\bfseries अथ चेत्त्वमिमं धर्म्यं सङ्ग्रामं न करिष्यसि~।\\ततः स्वधर्मं कीर्तिं च हित्वा पापमवाप्स्यसि~॥~३३~॥}\end{center} 
अथ चेत् त्वम् इमं धर्म्यं धर्मादनपेतं विहितं सङ्ग्रामं युद्धं न करिष्यसि चेत्~, ततः तदकरणात् स्वधर्मं कीर्तिं च महादेवादिसमागमनिमित्तां हित्वा केवलं पापम् अवाप्स्यसि~॥~३३~॥\par
 न केवलं स्वधर्मकीर्तिपरित्यागः —} 
\begin{center}{\bfseries अकीर्तिं चापि भूतानि कथयिष्यन्ति तेऽव्ययाम्~।\\सम्भावितस्य चाकीर्तिर्मरणादतिरिच्यते~॥~३४~॥}\end{center} 
अकीर्तिं चापि युद्धे भूतानि कथयिष्यन्ति ते तव अव्ययां दीर्घकालाम्~। धर्मात्मा शूर इत्येवमादिभिः गुणैः सम्भावितस्य च अकीर्तिः मरणात् अतिरिच्यते, सम्भावितस्य च } 
अकीर्तेः वरं मरणमित्यर्थः~॥~३४~॥\par
 
किञ्च —} 
\begin{center}{\bfseries भयाद्रणादुपरतं मंस्यन्ते त्वां महारथाः~।\\येषां च त्वं बहुमतो भूत्वा यास्यसि लाघवम्~॥~३५~॥}\end{center} 
भयात् कर्णादिभ्यः रणात् युद्धात् उपरतं निवृत्तं मंस्यन्ते चिन्तयिष्यन्ति न कृपयेति त्वां महारथाः दुर्योधनप्रभृतयः~। येषां च त्वं दुर्योधनादीनां बहुमतो बहुभिः गुणैः युक्तः इत्येवं मतः बहुमतः भूत्वा पुनः यास्यसि लाघवं लघुभावम्~॥~३५~॥\par
 किञ्च—} 
\begin{center}{\bfseries अवाच्यवादांश्च बहून्वदिष्यन्ति तवाहिताः~।\\निन्दन्तस्तव सामर्थ्यं ततो दुःखतरं नु किम्~॥~३६~॥}\end{center} 
अवाच्यवादान् अवक्तव्यवादांश्च बहून् अनेकप्रकारान् वदिष्यन्ति तव अहिताः शत्रवः निन्दन्तः कुत्सयन्तः तव त्वदीयं सामर्थ्यं निवातकवचादियुद्धनिमित्तम्~। ततः तस्मात् निन्दाप्राप्तेर्दुःखात् दुःखतरं नु किम्~, ततः कष्टतरं दुःखं नास्तीत्यर्थः~॥~३६~॥\par
 युद्धे पुनः क्रियमाणे कर्णादिभिः —} 
\begin{center}{\bfseries हतो वा प्राप्स्यसि स्वर्गं जित्वा वा भोक्ष्यसे महीम्~।\\तस्मादुत्तिष्ठ कौन्तेय युद्धाय कृतनिश्चयः~॥~३७~॥}\end{center} 
हतो वा प्राप्स्यसि स्वर्गम्~, हतः सन् स्वर्गं प्राप्स्यसि~। जित्वा वा कर्णादीन् शूरान् भोक्ष्यसे महीम्~। उभयथापि तव लाभ एवेत्यभिप्रायः~। यत एवं तस्मात् उत्तिष्ठ कौन्तेय युद्धाय कृतनिश्चयः ‘जेष्यामि शत्रून्~, मरिष्यामि वा’ इति निश्चयं कृत्वेत्यर्थः~॥~३७~॥\par
 तत्र युद्धं स्वधर्मं इत्येवं युध्यमानस्योपदेशमिमं शृणु —} 
\begin{center}{\bfseries सुखदुःखे समे कृत्वा लाभालाभौ जयाजयौ~।\\ततो युद्धाय युज्यस्व नैवं पापमवाप्स्यसि~॥~३८~॥}\end{center} 
सुखदुःखे समे तुल्ये कृत्वा, रागद्वेषावप्यकृत्वेत्येतत्~। तथा लाभालाभौ जयाजयौ च समौ कृत्वा ततो युद्धाय युज्यस्व घटस्व~। न एवं युद्धं कुर्वन् पापम् अवाप्स्यसि~। इत्येष उपदेशः प्रासङ्गिकः~॥~३८~॥\par
 शोकमोहापनयनाय लौकिको न्यायः ‘स्वधर्ममपि चावेक्ष्य’\footnote{भ. गी. २~। ३१} इत्याद्यैः श्लोकैरुक्तः, न तु तात्पर्येण~। परमार्थदर्शनमिह प्रकृतम्~। तच्चोक्तमुपसंह्रियते — ‘एषा तेऽभिहिता’\footnote{भ. गी. २~। ३९} इति शास्त्रविषयविभागप्रदर्शनाय~। इह हि प्रदर्शिते पुनः शास्त्रविषयविभागे उपरिष्टात् ‘ज्ञानयोगेन साङ्‍ख्यानां कर्मयोगेन योगिनाम्’\footnote{भ. गी. ३~। ३} इति निष्ठाद्वयविषयं शास्त्रं सुखं प्रवर्तिष्यते, श्रोतारश्च विषयविभागेन सुखं ग्रहीष्यन्ति इत्यत आह —} 
\begin{center}{\bfseries एषा तेऽभिहिता साङ्‍ख्ये बुद्धिर्योगे त्विमां शृणु~।\\बुद्ध्या युक्तो यया पार्थ कर्मबन्धं प्रहास्यसि~॥~३९~॥}\end{center} 
एषा ते तुभ्यम् अभिहिता उक्ता साङ्‍ख्ये परमार्थवस्तुविवेकविषये बुद्धिः ज्ञानं साक्षात् शोकमोहादिसंसारहेतुदोषनिवृत्तिकारणम्~। योगे तु तत्प्राप्त्युपाये निःसङ्गतया द्वन्द्वप्रहाणपूर्वकम् ईश्वराराधनार्थे कर्मयोगे कर्मानुष्ठाने समाधियोगे च इमाम् अनन्तरमेवोच्यमानां बुद्धिं शृणु~। तां च बुद्धिं स्तौति प्ररोचनार्थम् — बुद्ध्या यया योगविषयया युक्तः हे पार्थ, कर्मबन्धं कर्मैव धर्माधर्माख्यो बन्धः कर्मबन्धः तं प्रहास्यसि ईश्वरप्रसादनिमित्तज्ञानप्राप्त्यैव इत्यभिप्रायः~॥~३९~॥\par
 किञ्च अन्यत् —} 
\begin{center}{\bfseries नेहाभिक्रमनाशोऽस्ति प्रत्यवायो न विद्यते~।\\स्वल्पमप्यस्य धर्मस्य त्रायते महतो भयात्~॥~४०~॥}\end{center} 
न इह मोक्षमार्गे कर्मयोगे अभिक्रमनाशः अभिक्रमणमभिक्रमः प्रारम्भः तस्य नाशः नास्ति यथा कृष्यादेः~। योगविषये प्रारम्भस्य न अनैकान्तिकफलत्वमित्यर्थः~। किञ्च — नापि चिकित्सावत् प्रत्यवायः विद्यते भवति~। किं तु स्वल्पमपि अस्य धर्मस्य योगधर्मस्य अनुष्ठितं त्रायते रक्षति महतः भयात् संसारभयात् जन्ममरणादिलक्षणात्~॥~४०~॥\par
 येयं साङ्‍ख्ये बुद्धिरुक्ता योगे च, वक्ष्यमाणलक्षणा सा —} 
\begin{center}{\bfseries व्यवसायात्मिका बुद्धिरेकेह कुरुनन्दन~।\\बहुशाखा ह्यनन्ताश्च बुद्धयोऽव्यवसायिनाम्~॥~४१~॥}\end{center} 
व्यवसायात्मिका निश्चयस्वभावा एका एव बुद्धिः इतरविपरीतबुद्धिशाखाभेदस्य बाधिका, सम्यक्प्रमाणजनितत्वात्~, इह श्रेयोमार्गे हे कुरुनन्दन~। याः पुनः इतरा विपरीतबुद्धयः, यासां शाखाभेदप्रचारवशात् अनन्तः अपारः अनुपरतः संसारो नित्यप्रततो विस्तीर्णो भवति, प्रमाणजनितविवेकबुद्धिनिमित्तवशाच्च उपरतास्वनन्तभेदबुद्धिषु संसारोऽप्युपरमते ता बुद्धयः बहुशाखाः बह्वयः शाखाः यासां ताः बहुशाखाः, बहुभेदा इत्येतत्~। प्रतिशाखाभेदेन हि अनन्ताश्च बुद्धयः~। केषाम्~? अव्यवसायिनां प्रमाणजनितविवेकबुद्धिरहितानामित्यर्थः~॥~४१~॥\par
 येषां व्यवसायात्मिका बुद्धिर्नास्ति ते —} 
\begin{center}{\bfseries यामिमां पुष्पितां वाचं प्रवदन्त्यविपश्चितः~।\\वेदवादरताः पार्थ नान्यदस्तीति वादिनः~॥~४२~॥}\end{center} 
याम् इमां वक्ष्यमाणां पुष्पितां पुष्पित इव वृक्षः शोभमानां श्रूयमाणरमणीयां वाचं वाक्यलक्षणां प्रवदन्ति~। के~? अविपश्चितः अमेधसः अविवेकिन इत्यर्थः~। वेदवादरताः बह्वर्थवादफलसाधनप्रकाशकेषु वेदवाक्येषु रताः हे पार्थ, न अन्यत् स्वर्गपश्वादिफलसाधनेभ्यः कर्मभ्यः अस्ति इति एवं वादिनः वदनशीलाः~॥~४२~॥\par
 ते च —} 
\begin{center}{\bfseries कामात्मानः स्वर्गपरा जन्मकर्मफलप्रदाम्~।\\क्रियाविशेषबहुलां भोगैश्वर्यगतिं प्रति~॥~४३~॥}\end{center} 
कामात्मानः कामस्वभावाः, कामपरा इत्यर्थः~। स्वर्गपराः स्वर्गः परः पुरुषार्थः येषां ते स्वर्गपराः स्वर्गप्रधानाः~। जन्मकर्मफलप्रदां कर्मणः फलं कर्मफलं जन्मैव कर्मफलं जन्मकर्मफलं तत् प्रददातीति जन्मकर्मफलप्रदा, तां वाचम्~। प्रवदन्ति इत्यनुषज्यते~। क्रियाविशेषबहुलां क्रियाणां विशेषाः क्रियाविशेषाः ते बहुला यस्यां वाचि तां स्वर्गपशुपुत्राद्यर्थाः यया वाचा बाहुल्येन प्रकाश्यन्ते~। भोगैश्वर्यगतिं प्रति भोगश्च ऐश्वर्यं च भोगैश्वर्ये, तयोर्गतिः प्राप्तिः भोगैश्वर्यगतिः, तां प्रति साधनभूताः ये क्रियाविशेषाः तद्बहुलां तां वाचं प्रवदन्तः मूढाः संसारे परिवर्तन्ते इत्यभिप्रायः~॥~४३~॥\par
 तेषां च —} 
\begin{center}{\bfseries भोगैश्वर्यप्रसक्तानां तयापहृतचेतसाम्~।\\व्यवसायात्मिका बुद्धिः समाधौ न विधीयते~॥~४४~॥}\end{center} 
भोगैश्वर्यप्रसक्तानां भोगः कर्तव्यः ऐश्वर्यं च इति भोगैश्वर्ययोरेव प्रणयवतां तदात्मभूतानाम्~। तया क्रियाविशेषबहुलया वाचा अपहृतचेतसाम् आच्छादितविवेकप्रज्ञानां व्यवसायात्मिका साङ्‍ख्ये योगे वा बुद्धिः समाधौ समाधीयते अस्मिन् पुरुषोपभोगाय सर्वमिति समाधिः अन्तःकरणं बुद्धिः तस्मिन् समाधौ, न विधीयते न भवति इत्यर्थः~॥~४४~॥\par
 ये एवं विवेकबुद्धिरहिताः तेषां कामात्मनां यत् फलं तदाह —} 
\begin{center}{\bfseries त्रैगुण्यविषया वेदा निस्त्रैगुण्यो भवार्जुन~।\\निर्द्वन्द्वो नित्यसत्त्वस्थो निर्योगक्षेम आत्मवान्~॥~४५~॥}\end{center} 
त्रैगुण्यविषयाः त्रैगुण्यं संसारो विषयः प्रकाशयितव्यः येषां ते वेदाः त्रैगुण्यविषयाः~। त्वं तु निस्त्रैगुण्यो भव अर्जुन, निष्कामो भव इत्यर्थः~। निर्द्वन्द्वः सुखदुःखहेतू सप्रतिपक्षौ पदार्थौ द्वन्द्वशब्दवाच्यौ, ततः निर्गतः निर्द्वन्द्वो भव~। नित्यसत्त्वस्थः सदा सत्त्वगुणाश्रितो भव~। तथा निर्योगक्षेमः अनुपात्तस्य उपादानं योगः, उपात्तस्य रक्षणं क्षेमः, योगक्षेमप्रधानस्य श्रेयसि प्रवृत्तिर्दुष्करा इत्यतः निर्योगक्षेमो भव~। आत्मवान् अप्रमत्तश्च भव~। एष तव उपदेशः स्वधर्ममनुतिष्ठतः~॥~४५~॥\par
 सर्वेषु वेदोक्तेषु कर्मसु यान्युक्तान्यनन्तानि फलानि तानि नापेक्ष्यन्ते चेत्~, किमर्थं तानि ईश्वरायेत्यनुष्ठीयन्ते इत्युच्यते~; शृणु —} 
\begin{center}{\bfseries यावानर्थ उदपाने सर्वतःसम्प्लुतोदके~।\\तावान् सर्वेषु वेदेषु ब्राह्मणस्य विजानतः~॥~४६~॥}\end{center} 
यथा लोके कूपतडागाद्यनेकस्मिन् उदपाने परिच्छिन्नोदके यावान् यावत्परिमाणः स्नानपानादिः अर्थः फलं प्रयोजनं स सर्वः अर्थः सर्वतः सम्प्लुतोदकेऽपि यः अर्थः तावानेव सम्पद्यते, तत्र अन्तर्भवतीत्यर्थः~। एवं तावान् तावत्परिमाण एव सम्पद्यते सर्वेषु वेदेषु वेदोक्तेषु कर्मसु यः अर्थः यत्कर्मफलं सः अर्थः ब्राह्मणस्य संन्यासिनः परमार्थतत्त्वं विजानतः यः अर्थः यत् विज्ञानफलं सर्वतःसम्प्लुतोदकस्थानीयं तस्मिन् तावानेव सम्पद्यते तत्रैवान्तर्भवतीत्यर्थः~। ‘यथा कृताय विजितायाधरेयाः संयन्त्येवमेनं सर्वं तदभिसमेति यत् किञ्चित् प्रजाः साधु कुर्वन्ति यस्तद्वेद यत्स वेद’\footnote{छा. उ. ४~। १~। ४} इति श्रुतेः~। ‘सर्वं कर्माखिलम्’\footnote{भ. गी. ४~। ३३} इति च वक्ष्यति~। तस्मात् प्राक् ज्ञाननिष्ठाधिकारप्राप्तेः कर्मण्यधिकृतेन कूपतडागाद्यर्थस्थानीयमपि कर्म कर्तव्यम्~॥~४६~॥\par
 तव च —} 
\begin{center}{\bfseries कर्मण्येवाधिकारस्ते मा फलेषु कदाचन~।\\मा कर्मफलहेतुर्भूर्मा ते सङ्गोऽस्त्वकर्मणि~॥~४७~॥}\end{center} 
कर्मण्येव अधिकारः न ज्ञाननिष्ठायां ते तव~। तत्र च कर्म कुर्वतः मा फलेषु अधिकारः अस्तु, कर्मफलतृष्णा मा भूत् कदाचन कस्याञ्चिदप्यवस्थायामित्यर्थः~। यदा कर्मफले तृष्णा ते स्यात् तदा कर्मफलप्राप्तेः हेतुः स्याः, एवं मा कर्मफलहेतुः भूः~। यदा हि कर्मफलतृष्णाप्रयुक्तः कर्मणि प्रवर्तते तदा कर्मफलस्यैव जन्मनो हेतुर्भवेत्~। यदि कर्मफलं नेष्यते, किं कर्मणा दुःखरूपेण~? इति मा ते तव सङ्गः अस्तु अकर्मणि अकरणे प्रीतिर्मा भूत्~॥~४७~॥\par
 यदि कर्मफलप्रयुक्तेन न कर्तव्यं कर्म, कथं तर्हि कर्तव्यमिति~; उच्यते —} 
\begin{center}{\bfseries योगस्थः कुरु कर्माणि सङ्गं त्यक्त्वा धनञ्जय~।\\सिद्ध्यसिद्ध्योः समो भूत्वा समत्वं योग उच्यते~॥~४८~॥}\end{center} 
योगस्थः सन् कुरु कर्माणि केवलमीश्वरार्थम्~; तत्रापि ‘ईश्वरो मे तुष्यतु’ इति सङ्गं त्यक्त्वा धनञ्जय~। फलतृष्णाशून्येन क्रियमाणे कर्मणि सत्त्वशुद्धिजा ज्ञानप्राप्तिलक्षणासिद्धिः, तद्विपर्ययजा असिद्धिः, तयोः सिद्ध्यसिद्ध्योः अपि समः तुल्यः भूत्वा कुरु कर्माणि~। कोऽसौ योगः यत्रस्थः कुरु इति उक्तम्~? इदमेव तत् — सिद्ध्यसिद्ध्योः समत्वं योगः उच्यते~॥~४८~॥\par
 यत्पुनः समत्वबुद्धियुक्तमीश्वराराधनार्थं कर्मोक्तम्~, एतस्मात्कर्मणः —} 
\begin{center}{\bfseries दूरेण ह्यवरं कर्म बुद्धियोगाद्धनञ्जय~।\\बुद्धौ शरणमन्विच्छ कृपणाः फलहेतवः~॥~४९~॥}\end{center} 
दूरेण अतिविप्रकर्षेण अत्यन्तमेव हि अवरम् अधमं निकृष्टं कर्म फलार्थिना क्रियमाणं बुद्धियोगात् समत्वबुद्धियुक्तात् कर्मणः, जन्ममरणादिहेतुत्वात्~। हे धनञ्जय, यत एवं ततः योगविषयायां बुद्धौ तत्परिपाकजायां वा साङ्‍ख्यबुद्धौ शरणम् आश्रयमभयप्राप्तिकारणम् अन्विच्छ प्रार्थयस्व, परमार्थज्ञानशरणो भवेत्यर्थः~। यतः अवरं कर्म कुर्वाणाः कृपणाः दीनाः फलहेतवः फलतृष्णाप्रयुक्ताः सन्तः, ‘यो वा एतदक्षरं गार्ग्यविदित्वास्माल्लोकात्प्रैति स कृपणः’\footnote{बृ. उ. ३~। ८~। १०} इति श्रुतेः~॥~४९~॥\par
 समत्वबुद्धियुक्तः सन् स्वधर्ममनुतिष्ठन् यत्फलं प्राप्नोति तच्छृणु —} 
\begin{center}{\bfseries बुद्धियुक्तो जहातीह उभे सुकृतदुष्कृते~।\\तस्माद्योगाय युज्यस्व योगः कर्मसु कौशलम्~॥~५०~॥}\end{center} 
बुद्धियुक्तः कर्मसमत्वविषयया बुद्ध्या युक्तः बुद्धियुक्तः सः जहाति परित्यजति इह अस्मिन् लोके उभे सुकृतदुष्कृते पुण्यपापे सत्त्वशुद्धिज्ञानप्राप्तिद्वारेण यतः, तस्मात् समत्वबुद्धियोगाय युज्यस्व घटस्व~। योगो हि कर्मसु कौशलम्~, स्वधर्माख्येषु कर्मसु वर्तमानस्य या सिद्ध्यासिद्ध्योः समत्वबुद्धिः ईश्वरार्पितचेतस्तया तत् कौशलं कुशलभावः~। तद्धि कौशलं यत् बन्धनस्वभावान्यपि कर्माणि समत्वबुद्ध्या स्वभावात् निवर्तन्ते~। तस्मात्समत्वबुद्धियुक्तो भव त्वम्~॥~५०~॥\par
 यस्मात् —} 
\begin{center}{\bfseries कर्मजं बुद्धियुक्ता हि फलं त्यक्त्वा मनीषिणः~।\\जन्मबन्धविनिर्मुक्ताः पदं गच्छन्त्यनामयम्~॥~५१~॥}\end{center} 
कर्मजं फलं त्यक्त्वा इति व्यवहितेन सम्बन्धः~। इष्टानिष्टदेहप्राप्तिः कर्मजं फलं कर्मभ्यो जातं बुद्धियुक्ताः समत्वबुद्धियुक्ताः सन्तः हि यस्मात् फलं त्यक्त्वा परित्यज्य मनीषिणः ज्ञानिनो भूत्वा, जन्मबन्धविनिर्मुक्ताः जन्मैव बन्धः जन्मबन्धः तेन विनिर्मुक्ताः जीवन्त एव जन्मबन्धात् विनिर्मुक्ताः सन्तः, पदं परमं विष्णोः मोक्षाख्यं गच्छन्ति अनामयं सर्वोपद्रवरहितमित्यर्थः~। अथवा ‘बुद्धियोगाद्धनञ्जय’\footnote{भ. गी. २~। ४९} इत्यारभ्य परमार्थदर्शनलक्षणैव सर्वतःसम्प्लुतोदकस्थानीया कर्मयोगजसत्त्वशुद्धिजनिता बुद्धिर्दर्शिता, साक्षात्सुकृतदुष्कृतप्रहाणादिहेतुत्वश्रवणात्~॥~५१~॥\par
 योगानुष्ठानजनितसत्त्वशुद्धिजा बुद्धिः कदा प्राप्स्यते इत्युच्यते —} 
\begin{center}{\bfseries यदा ते मोहकलिलं बुद्धिर्व्यतितरिष्यति~।\\तदा गन्तासि निर्वेदं श्रोतव्यस्य श्रुतस्य च~॥~५२~॥}\end{center} 
यदा यस्मिन्काले ते तव मोहकलिलं मोहात्मकमविवेकरूपं कालुष्यं येन आत्मानात्मविवेकबोधं कलुषीकृत्य विषयं प्रत्यन्तःकरणं प्रवर्तते, तत् तव बुद्धिः व्यतितरिष्यति व्यतिक्रमिष्यति, अतिशुद्धभावमापत्स्यते इत्यर्थः~। तदा तस्मिन् काले गन्तासि प्राप्स्यसि निर्वेदं वैराग्यं श्रोतव्यस्य श्रुतस्य च, तदा श्रोतव्यं श्रुतं च ते निष्फलं प्रतिभातीत्यभिप्रायः~॥~५२~॥\par
 मोहकलिलात्ययद्वारेण लब्धात्मविवेकजप्रज्ञः कदा कर्मयोगजं फलं परमार्थयोगमवाप्स्यामीति चेत्~, तत् शृणु — } 
\begin{center}{\bfseries श्रुतिविप्रतिपन्ना ते यदा स्थास्यति निश्चला~।\\समाधावचला बुद्धिस्तदा योगमवाप्स्यसि~॥~५३~॥}\end{center} 
श्रुतिविप्रतिपन्ना अनेकसाध्यसाधनसम्बन्धप्रकाशनश्रुतिभिः श्रवणैः प्रवृत्तिनिवृत्तिलक्षणैः विप्रतिपन्ना नानाप्रतिपन्ना विक्षिप्ता सती ते तव बुद्धिः यदि यस्मिन् काले स्थास्यति स्थिरीभूता भविष्यति निश्चला विक्षेपचलनवर्जिता सती समाधौ, समाधीयते चित्तमस्मिन्निति समाधिः आत्मा, तस्मिन् आत्मनि इत्येतत्~। अचला तत्रापि विकल्पवर्जिता इत्येतत्~। बुद्धिः अन्तःकरणम्~। तदा तस्मिन्काले योगम् अवाप्स्यसि विवेकप्रज्ञां समाधिं प्राप्स्यसि~॥~५३~॥\par
 प्रश्नबीजं प्रतिलभ्य अर्जुन उवाच लब्धसमाधिप्रज्ञस्य लक्षणबुभुत्सया —} 
\begin{center}{\bfseries अर्जुन उवाच —\\ स्थितप्रज्ञस्य का भाषा समाधिस्थस्य केशव~।\\स्थितधीः किं प्रभाषेत किमासीत व्रजेत किम्~॥~५४~॥}\end{center} 
स्थिता प्रतिष्ठिता ‘अहमस्मि परं ब्रह्म’ इति प्रज्ञा यस्य सः स्थितप्रज्ञः तस्य स्थितप्रज्ञस्य का भाषा किं भाषणं वचनं कथमसौ परैर्भाष्यते समाधिस्थस्य समाधौ स्थितस्य हे केशव~। स्थितधीः स्थितप्रज्ञः स्वयं वा किं प्रभाषेत~। किम् आसीत् व्रजेत किम् आसनं व्रजनं वा तस्य कथमित्यर्थः~। स्थितप्रज्ञस्य लक्षणमनेन श्लोकेन पृच्छ्यते~॥~५४~॥\par
 यो ह्यादित एव संन्यस्य कर्माणि ज्ञानयोगनिष्ठायां प्रवृत्तः, यश्च कर्मयोगेन, तयोः ‘प्रजहाति’ इत्यारभ्य आ अध्यायपरिसमाप्तेः स्थितप्रज्ञलक्षणं साधनं चोपदिश्यते~। सर्वत्रैव हि अध्यात्मशास्त्रे कृतार्थलक्षणानि यानि तान्येव साधनानि उपदिश्यन्ते, यत्नसाध्यत्वात्~। यानि यत्नसाध्यानि साधनानि लक्षणानि च भवन्ति तानि श्रीभगवानुवाच —}\\ 
\begin{center}{\bfseries श्रीभगवानुवाच —\\ प्रजहाति यदा कामान्सर्वान्पार्थ मनोगतान्~।\\आत्मन्येवात्मना तुष्टः स्थितप्रज्ञस्तदोच्यते~॥~५५~॥}\end{center} 
प्रजहाति प्रकर्षेण जहाति परित्यजति यदा यस्मिन्काले सर्वान् समस्तान् कामान् इच्छाभेदान् हे पार्थ, मनोगतान् मनसि प्रविष्टान् हृदि प्रविष्टान्~। सर्वकामपरित्यागे तुष्टिकारणाभावात् शरीरधारणनिमित्तशेषे च सति उन्मत्तप्रमत्तस्येव प्रवृत्तिः प्राप्ता, इत्यत उच्यते — आत्मन्येव प्रत्यगात्मस्वरूपे एव आत्मना स्वेनैव बाह्यलाभनिरपेक्षः तुष्टः परमार्थदर्शनामृतरसलाभेन अन्यस्मादलंप्रत्ययवान् स्थितप्रज्ञः स्थिता प्रतिष्ठिता आत्मानात्मविवेकजा प्रज्ञा यस्य सः स्थितप्रज्ञः विद्वान् तदा उच्यते~। त्यक्तपुत्रवित्तलोकैषणः संन्यासी आत्माराम आत्मक्रीडः स्थितप्रज्ञ इत्यर्थः~॥~५५~॥\par
 किञ्च —} 
\begin{center}{\bfseries दुःखेष्वनुद्विग्नमनाः सुखेषु विगतस्पृहः~।\\वीतरागभयक्रोधः स्थितधीर्मुनिरुच्यते~॥~५६~॥}\end{center} 
दुःखेषु आध्यात्मिकादिषु प्राप्तेषु न उद्विग्नं न प्रक्षुभितं दुःखप्राप्तौ मनो यस्य सोऽयम् अनुद्विग्नमनाः~। तथा सुखेषु प्राप्तेषु विगता स्पृहा तृष्णा यस्य, न अग्निरिव इन्धनाद्याधाने सुखान्यनु विवर्धते स विगतस्पृहः~। वीतरागभयक्रोधः रागश्च भयं च क्रोधश्च वीता विगता यस्मात् स वीतरागभयक्रोधः~। स्थितधीः स्थितप्रज्ञो मुनिः संन्यासी तदा उच्यते~॥~५६~॥\par
 किञ्च —} 
\begin{center}{\bfseries यः सर्वत्रानभिस्नेहस्तत्तत्प्राप्य शुभाशुभम्~।\\नाभिनन्दति न द्वेष्टि तस्य प्रज्ञा प्रतिष्ठिता~॥~५७~॥}\end{center} 
यः मुनिः सर्वत्र देहजीवितादिष्वपि अनभिस्नेहः अभिस्नेहवर्जितः तत्तत् प्राप्य शुभाशुभं तत्तत् शुभं अशुभं वा लब्ध्वा न अभिनन्दति न द्वेष्टि शुभं प्राप्य न तुष्यति न हृष्यति, अशुभं च प्राप्य न द्वेष्टि इत्यर्थः~। तस्य एवं हर्षविषादवर्जितस्य विवेकजा प्रज्ञा प्रतिष्ठिता भवति~॥~५७~॥\par
 किञ्च—} 
\begin{center}{\bfseries यदा संहरते चायं कूर्मोऽङ्गानीव सर्वशः~।\\इन्द्रियाणीन्द्रियार्थेभ्यस्तस्य प्रज्ञा प्रतिष्ठिता~॥~५८~॥}\end{center} 
यदा संहरते सम्यगुपसंहरते च अयं ज्ञाननिष्ठायां प्रवृत्तो यतिः कूर्मः अङ्गानि इव यथा कूर्मः भयात् स्वान्यङ्गानि उपसंहरति सर्वशः सर्वतः, एवं ज्ञाननिष्ठः इन्द्रियाणि इन्द्रियार्थेभ्यः सर्वविषयेभ्यः उपसंहरते~। तस्य प्रज्ञा प्रतिष्ठिता इत्युक्तार्थं वाक्यम्~॥~५८~॥\par
 तत्र विषयाननाहरतः आतुरस्यापि इन्द्रियाणि कूर्माङ्गानीव संह्रियन्ते न तु तद्विषयो रागः स कथं संह्रियते इति उच्यते —} 
\begin{center}{\bfseries विषया विनिवर्तन्ते निराहारस्य देहिनः~।\\रसवर्जं रसोऽप्यस्य परं दृष्ट्वा निवर्तते~॥~५९~॥}\end{center} 
यद्यपि विषयाः विषयोपलक्षितानि विषयशब्दवाच्यानि इन्द्रियाणि निराहारस्य अनाह्रियमाणविषयस्य कष्टे तपसि स्थितस्य मूर्खस्यापि विनिवर्तन्ते देहिनो देहवतः रसवर्जं रसो रागो विषयेषु यः तं वर्जयित्वा~। रसशब्दो रागे प्रसिद्धः, स्वरसेन प्रवृत्तः रसिकः रसज्ञः, इत्यादिदर्शनात्~। सोऽपि रसो रञ्जनारूपः सूक्ष्मः अस्य यतेः परं परमार्थतत्त्वं ब्रह्म दृष्ट्वा उपलभ्य ‘अहमेव तत्’ इति वर्तमानस्य निवर्तते निर्बीजं विषयविज्ञानं सम्पद्यते इत्यर्थः~। न असति सम्यग्दर्शने रसस्य उच्छेदः~। तस्मात् सम्यग्दर्शनात्मिकायाः प्रज्ञायाः स्थैर्यं कर्तव्यमित्यभिप्रायः~॥~५९~॥\par
 सम्यग्दर्शनलक्षणप्रज्ञास्थैर्यं चिकीर्षता आदौ इन्द्रियाणि स्ववशे स्थापयितव्यानि, यस्मात्तदनवस्थापने दोषमाह —} 
\begin{center}{\bfseries यततो ह्यपि कौन्तेय पुरुषस्य विपश्चितः~।\\इन्द्रियाणि प्रमाथीनि हरन्ति प्रसभं मनः~॥~६०~॥}\end{center} 
यततः प्रयत्नं कुर्वतः हि यस्मात् कौन्तेय पुरुषस्य विपश्चितः मेधाविनः अपि इति व्यवहितेन सम्बन्धः~। इन्द्रियाणि प्रमाथीनि प्रमथनशीलानि विषयाभिमुखं हि पुरुषं विक्षोभयन्ति आकुलीकुर्वन्ति, आकुलीकृत्य च हरन्ति प्रसभं प्रसह्य प्रकाशमेव पश्यतो विवेकविज्ञानयुक्तं मनः~॥~६०~॥\par
 यतः तस्मात् —} 
\begin{center}{\bfseries तानि सर्वाणि संयम्य युक्त आसीत मत्परः~।\\वशे हि यस्येन्द्रियाणि तस्य प्रज्ञा प्रतिष्ठिता~॥~६१~॥}\end{center} 
तानि सर्वाणि संयम्य संयमनं वशीकरणं कृत्वा युक्तः समाहितः सन् आसीत मत्परः अहं वासुदेवः सर्वप्रत्यगात्मा परो यस्य सः मत्परः, ‘न अन्योऽहं तस्मात्’ इति आसीत इत्यर्थः~। एवमासीनस्य यतेः वशे हि यस्य इन्द्रियाणि वर्तन्ते अभ्यासबलात् तस्य प्रज्ञा प्रतिष्ठिता~॥~६१~॥\par
 अथेदानीं पराभविष्यतः सर्वानर्थमूलमिदमुच्यते —} 
\begin{center}{\bfseries ध्यायतो विषयान्पुंसः सङ्गस्तेषूपजायते~।\\सङ्गात्सञ्जायते कामः कामात्क्रोधोऽभिजायते~॥~६२~॥}\end{center} 
ध्यायतः चिन्तयतः विषयान् शब्दादीन् विषयविशेषान् आलोचयतः पुंसः पुरुषस्य सङ्गः आसक्तिः प्रीतिः तेषु विषयेषु उपजायते उत्पद्यते~। सङ्गात् प्रीतेः सञ्जायते समुत्पद्यते कामः तृष्णा~। कामात् कुतश्चित् प्रतिहतात् क्रोधः अभिजायते~॥~६२~॥\par
 \begin{center}{\bfseries क्रोधाद्भवति संमोहः संमोहात्स्मृतिविभ्रमः~।\\स्मृतिभ्रंशाद्बुद्धिनाशो बुद्धिनाशात्प्रणश्यति~॥~६३~॥}\end{center} 
क्रोधात् भवति संमोहः अविवेकः कार्याकार्यविषयः~। क्रुद्धो हि संमूढः सन् गुरुमप्याक्रोशति~। संमोहात् स्मृतिविभ्रमः शास्त्राचार्योपदेशाहितसंस्कारजनितायाः स्मृतेः स्यात् विभ्रमो भ्रंशः स्मृत्युत्पत्तिनिमित्तप्राप्तौ अनुत्पत्तिः~। ततः स्मृतिभ्रंशात् बुद्धिनाशः बुद्धेर्नाशः~। कार्याकार्यविषयविवेकायोग्यता अन्तःकरणस्य बुद्धेर्नाश उच्यते~। बुद्धिनाशात् प्रणश्यति~। तावदेव हि पुरुषः यावदन्तःकरणं तदीयं कार्याकार्यविषयविवेकयोग्यम्~। तदयोग्यत्वे नष्ट एव पुरुषो भवति~। अतः तस्यान्तःकरणस्य बुद्धेर्नाशात् प्रणश्यति पुरुषार्थायोग्यो भवतीत्यर्थः~॥~६३~॥\par
 सर्वानर्थस्य मूलमुक्तं विषयाभिध्यानम्~। अथ इदानीं मोक्षकारणमिदमुच्यते —} 
\begin{center}{\bfseries रागद्वेषवियुक्तैस्तु विषयानिन्द्रियैश्चरन्~।\\आत्मवश्यैर्विधेयात्मा प्रसादमधिगच्छति~॥~६४~॥}\end{center} 
रागद्वेषवियुक्तैः रागश्च द्वेषश्च रागद्वेषौ, तत्पुरःसरा हि इन्द्रियाणां प्रवृत्तिः स्वाभाविकी, तत्र यो मुमुक्षुः भवति सः ताभ्यां वियुक्तैः श्रोत्रादिभिः इन्द्रियैः विषयान् अवर्जनीयान् चरन् उपलभमानः आत्मवश्यैः आत्मनः वश्यानि वशीभूतानि इन्द्रियाणि तैः आत्मवश्यैः विधेयात्मा इच्छातः विधेयः आत्मा अन्तःकरणं यस्य सः अयं प्रसादम् अधिगच्छति~। प्रसादः प्रसन्नता स्वास्थ्यम्~॥~६४~॥\par
 प्रसादे सति किं स्यात् इत्युच्यते —} 
\begin{center}{\bfseries प्रसादे सर्वदुःखानां हानिरस्योपजायते~।\\प्रसन्नचेतसो ह्याशु बुद्धिः पर्यवतिष्ठते~॥~६५~॥}\end{center} 
प्रसादे सर्वदुःखानाम् आध्यात्मिकादीनां हानिः विनाशः अस्य यतेः उपजायते~। किञ्च — प्रसन्नचेतसः स्वस्थान्तःकरणस्य हि यस्मात् आशु शीघ्रं बुद्धिः पर्यवतिष्ठते आकाशमिव परि समन्तात् अवतिष्ठते, आत्मस्वरूपेणैव निश्चलीभवतीत्यर्थः~॥~} 
एवं प्रसन्नचेतसः अवस्थितबुद्धेः कृतकृत्यता यतः, तस्मात् रागद्वेषवियुक्तैः इन्द्रियैः शास्त्राविरुद्धेषु अवर्जनीयेषु युक्तः समाचरेत् इति वाक्यार्थः~॥~६५~॥\par
 सेयं प्रसन्नता स्तूयते —} 
\begin{center}{\bfseries नास्ति बुद्धिरयुक्तस्य न चायुक्तस्य भावना~।\\न चाभावयतः शान्तिरशान्तस्य कुतः सुखम्~॥~६६~॥}\end{center} 
नास्ति न विद्यते न भवतीत्यर्थः, बुद्धिः आत्मस्वरूपविषया अयुक्तस्य असमाहितान्तःकरणस्य~। न च अस्ति अयुक्तस्य भावना आत्मज्ञानाभिनिवेशः~। तथा — न च अस्ति अभावयतः आत्मज्ञानाभिनिवेशमकुर्वतः शान्तिः उपशमः~। अशान्तस्य कुतः सुखम्~? इन्द्रियाणां हि विषयसेवातृष्णातः निवृत्तिर्या तत्सुखम्~, न विषयविषया तृष्णा~। दुःखमेव हि सा~। न तृष्णायां सत्यां सुखस्य गन्धमात्रमप्युपपद्यते इत्यर्थः~॥~६६~॥\par
 अयुक्तस्य कस्माद्बुद्धिर्नास्ति इत्युच्यते —} 
\begin{center}{\bfseries इन्द्रियाणां हि चरतां यन्मनोऽनुविधीयते~।\\तदस्य हरति प्रज्ञां वायुर्नावमिवाम्भसि~॥~६७~॥}\end{center} 
इन्द्रियाणां हि यस्मात् चरतां स्वस्वविषयेषु प्रवर्तमानानां यत् मनः अनुविधीयते अनुप्रवर्तते तत् इन्द्रियविषयविकल्पनेन प्रवृत्तं मनः अस्य यतेः हरति प्रज्ञाम् आत्मानात्मविवेकजां नाशयति~। कथम्~? वायुः नावमिव अम्भसि उदके जिगमिषतां मार्गादुद्धृत्य उन्मार्गे यथा वायुः नावं प्रवर्तयति, एवमात्मविषयां प्रज्ञां हृत्वा मनो विषयविषयां करोति~॥~६७~॥\par
 ‘यततो हि’\footnote{भ. गी. २~। ६०} इत्युपन्यस्तस्यार्थस्य अनेकधा उपपत्तिमुक्त्वा तं चार्थमुपपाद्य उपसंहरति —} 
\begin{center}{\bfseries तस्माद्यस्य महाबाहो निगृहीतानि सर्वशः~।\\इन्द्रियाणीन्द्रियार्थेभ्यस्तस्य प्रज्ञा प्रतिष्ठिता~॥~६८~॥}\end{center} 
इन्द्रियाणां प्रवृत्तौ दोष उपपादितो यस्मात्~, तस्मात् यस्य यतेः हे महाबाहो, निगृहीतानि सर्वशः सर्वप्रकारैः मानसादिभेदैः इन्द्रियाणि इन्द्रियार्थेभ्यः शब्दादिभ्यः तस्य प्रज्ञा प्रतिष्ठिता~॥~६८~॥\par
 योऽयं लौकिको वैदिकश्च व्यवहारः स उत्पन्नविवेकज्ञानस्य स्थितप्रज्ञस्य अविद्याकार्यत्वात् अविद्यानिवृत्तौ निवर्तते, अविद्यायाश्च विद्याविरोधात् निवृत्तिः, इत्येतमर्थं स्फुटीकुर्वन् आह —} 
\begin{center}{\bfseries या निशा सर्वभूतानां तस्यां जागर्ति संयमी~।\\यस्यां जाग्रति भूतानि सा निशा पश्यतो मुनेः~॥~६९~॥}\end{center} 
या निशा रात्रिः सर्वपदार्थानामविवेककरी तमःस्वभावत्वात् सर्वभूतानां सर्वेषां भूतानाम्~। किं तत् परमार्थतत्त्वं स्थितप्रज्ञस्य विषयः~। यथा नक्तञ्चराणाम् अहरेव सदन्येषां निशा भवति, तद्वत् नक्तञ्चरस्थानीयानामज्ञानां सर्वभूतानां निशेव निशा परमार्थतत्त्वम्~, अगोचरत्वादतद्बुद्धीनाम्~। तस्यां परमार्थतत्त्वलक्षणायामज्ञाननिद्रायाः प्रबुद्धो जागर्ति संयमी संयमवान्~, जितेन्द्रियो योगीत्यर्थः~। यस्यां ग्राह्यग्राहकभेदलक्षणायामविद्यानिशायां प्रसुप्तान्येव भूतानि जाग्रति इति उच्यन्ते, यस्यां निशायां प्रसुप्ता इव स्वप्नदृशः, सा निशा अविद्यारूपत्वात् परमार्थतत्त्वं पश्यतो मुनेः~॥~} 
अतः कर्माणि अविद्यावस्थायामेव चोद्यन्ते, न विद्यावस्थायाम्~। विद्यायां हि सत्याम् उदिते सवितरि शार्वरमिव तमः प्रणाशमुपगच्छति अविद्या~। प्राक् विद्योत्पत्तेः अविद्या प्रमाणबुद्ध्या गृह्यमाणा क्रियाकारकफलभेदरूपा सती सर्वकर्महेतुत्वं प्रतिपद्यते~। न अप्रमाणबुद्ध्या गृह्यमाणायाः कर्महेतुत्वोपपत्तिः, ‘प्रमाणभूतेन वेदेन मम चोदितं कर्तव्यं कर्म’ इति हि कर्मणि कर्ता प्रवर्तते, न ‘अविद्यामात्रमिदं सर्वं निशेव’ इति~। यस्य पुनः ‘निशेव अविद्यामात्रमिदं सर्वं भेदजातम्’ इति ज्ञानं तस्य आत्मज्ञस्य सर्वकर्मसंन्यासे एव अधिकारो न प्रवृत्तौ~। तथा च दर्शयिष्यति — ‘तद्बुद्धयस्तदात्मानः’\footnote{भ. गी. ५~। १७} इत्यादिना ज्ञाननिष्ठायामेव तस्य अधिकारम्~॥~} 
तत्रापि प्रवर्तकप्रमाणाभावे प्रवृत्त्यनुपपत्तिः इति चेत्~, न~; स्वात्मविषयत्वादात्मविज्ञानस्य~। न हि आत्मनः स्वात्मनि प्रवर्तकप्रमाणापेक्षता, आत्मत्वादेव~। तदन्तत्वाच्च सर्वप्रमाणानां प्रमाणत्वस्य~। न हि आत्मस्वरूपाधिगमे सति पुनः प्रमाणप्रमेयव्यवहारः सम्भवति~। प्रमातृत्वं हि आत्मनः निवर्तयति अन्त्यं प्रमाणम्~; निवर्तयदेव च अप्रमाणीभवति, स्वप्नकालप्रमाणमिव प्रबोधे~। लोके च वस्त्वधिगमे प्रवृत्तिहेतुत्त्वादर्शनात् प्रमाणस्य~। तस्मात् न आत्मविदः कर्मण्यधिकार इति सिद्धम्~॥~६९~॥\par
 विदुषः त्यक्तैषणस्य स्थितप्रज्ञस्य यतेरेव मोक्षप्राप्तिः, न तु असंन्यासिनः कामकामिनः इत्येतमर्थं दृष्टान्तेन प्रतिपादयिष्यन् आह —} 
\begin{center}{\bfseries आपूर्यमाणमचलप्रतिष्ठं समुद्रमापः प्रविशन्ति यद्वत्~।\\तद्वत्कामा यं प्रविशन्ति सर्वे स शान्तिमाप्नोति न कामकामी~॥~७०~॥}\end{center} 
आपूर्यमाणम् अद्भिः अचलप्रतिष्ठम् अचलतया प्रतिष्ठा अवस्थितिः यस्य तम् अचलप्रतिष्ठं समुद्रम् आपः सर्वतो गताः प्रविशन्ति स्वात्मस्थमविक्रियमेव सन्तं यद्वत्~, तद्वत् कामाः विषयसंनिधावपि सर्वतः इच्छाविशेषाः यं पुरुषम् — समुद्रमिव आपः — अविकुर्वन्तः प्रविशन्ति सर्वे आत्मन्येव प्रलीयन्ते न स्वात्मवशं कुर्वन्ति, सः शान्तिं मोक्षम् आप्नोति, न इतरः कामकामी, काम्यन्त इति कामाः विषयाः तान् कामयितुं शीलं यस्य सः कामकामी, नैव प्राप्नोति इत्यर्थः~॥~७०~॥\par
 यस्मादेवं तस्मात्—} 
\begin{center}{\bfseries विहाय कामान्यः सर्वान्पुमांश्चरति निःस्पृहः~।\\निर्ममो निरहङ्कारः स शान्तिमधिगच्छति~॥~७१~॥}\end{center} 
विहाय परित्यज्य कामान् यः संन्यासी पुमान् सर्वान् अशेषतः कार्‌त्स्न्येन चरति, जीवनमात्रचेष्टाशेषः पर्यटतीत्यर्थः~। निःस्पृहः शरीरजीवनमात्रेऽपि निर्गता स्पृहा यस्य सः निःस्पृहः सन्~, निर्ममः शरीरजीवनमात्राक्षिप्तपरिग्रहेऽपि ममेदम् इत्यपभिनिवेशवर्जितः, निरहङ्कारः विद्यावत्त्वादिनिमित्तात्मसम्भावनारहितः इत्येतत्~। सः एवंभूतः स्थितप्रज्ञः ब्रह्मवित् शान्तिं सर्वसंसारदुःखोपरमलक्षणां निर्वाणाख्याम् अधिगच्छति प्राप्नोति ब्रह्मभूतो भवति इत्यर्थः~॥~७१~॥\par
 सैषा ज्ञाननिष्ठा स्तूयते —} 
\begin{center}{\bfseries एषा ब्राह्मी स्थितिः पार्थ नैनां प्राप्य विमुह्यति~।\\स्थित्वास्यामन्तकालेऽपि ब्रह्मनिर्वाणमृच्छति~॥~७२~॥}\end{center} 
एषा यथोक्ता ब्राह्मी ब्रह्मणि भवा इयं स्थितिः सर्वं कर्म संन्यस्य ब्रह्मरूपेणैव अवस्थानम् इत्येतत्~। हे पार्थ, न एनां स्थितिं प्राप्य लब्ध्वा न विमुह्यति न मोहं प्राप्नोति~। स्थित्वा अस्यां स्थितौ ब्राह्म्यां यथोक्तायां अन्तकालेऽपि अन्त्ये वयस्यपि ब्रह्मनिर्वाणं ब्रह्मनिर्वृतिं मोक्षम् ऋच्छति गच्छति~। किमु वक्तव्यं ब्रह्मचर्यादेव संन्यस्य यावज्जीवं यो ब्रह्मण्येव अवतिष्ठते स ब्रह्मनिर्वाणमृच्छति इति~॥~७२~॥\par
 
इति श्रीमत्परमहंसपरिव्राजकाचार्यस्य श्रीगोविन्दभगवत्पूज्यपादशिष्यस्य श्रीमच्छङ्करभगवतः कृतौ श्रीमत्भगवद्गीताभाष्ये द्वितीयोऽध्यायः~॥\par
 
शास्त्रस्य प्रवृत्तिनिवृत्तिविषयभूते द्वे बुद्धी भगवता निर्दिष्टे, साङ्‍ख्ये बुद्धिः योगे बुद्धिः इति च~। तत्र ‘प्रजहाति यदा कामान्’\footnote{भ. गी. २~। ५५} इत्यारभ्य आ अध्यायपरिसमाप्तेः साङ्‍ख्यबुद्ध्याश्रितानां संन्यासं कर्तव्यमुक्त्वा तेषां तन्निष्ठतयैव च कृतार्थता उक्ता — ‘एषा ब्राह्मी स्थितिः’\footnote{भ. गी. २~। ७२} इति~। अर्जुनाय च ‘कर्मण्येवाधिकारस्ते . . . मा ते सङ्गोऽस्त्वकर्मणि’\footnote{भ. गी. २~। ४७} इति कर्मैव कर्तव्यमुक्तवान् योगबुद्धिमाश्रित्य, न तत एव श्रेयःप्राप्तिम् उक्तवान्~। तदेतदालक्ष्य पर्याकुलीकृतबुद्धिः अर्जुनः उवाच~। कथं भक्ताय श्रेयोर्थिने यत् साक्षात् श्रेयःप्राप्तिसाधनं साङ्‍ख्यबुद्धिनिष्ठां श्रावयित्वा मां कर्मणि दृष्टानेकानर्थयुक्ते पारम्पर्येणापि अनैकान्तिकश्रेयःप्राप्तिफले नियुञ्ज्यात् इति युक्तः पर्याकुलीभावः अर्जुनस्य, तदनुरूपश्च प्रश्नः ‘ज्यायसी चेत्’\footnote{भ. गी. ३~। १} इत्यादिः, प्रश्नापाकरणवाक्यं च भगवतः युक्तं यथोक्तविभागविषये शास्त्रे~॥~} 
केचित्तु — अर्जुनस्य प्रश्नार्थमन्यथा कल्पयित्वा तत्प्रतिकूलं भगवतः प्रतिवचनं वर्णयन्ति, यथा च आत्मना सम्बन्धग्रन्थे गीतार्थो निरूपितः तत्प्रतिकूलं च इह पुनः प्रश्नप्रतिवचनयोः अर्थं निरूपयन्ति~। कथम्~? तत्र सम्बन्धग्रन्थे तावत् — सर्वेषामाश्रमिणां ज्ञानकर्मणोः समुच्चयः गीताशास्त्रे निरूपितः अर्थः इत्युक्तम्~; पुनः विशेषितं च यावज्जीवश्रुतिचोदितानि कर्माणि परित्यज्य केवलादेव ज्ञानात् मोक्षः प्राप्यते इत्येतत् एकान्तेनैव प्रतिषिद्धमिति~। इह तु आश्रमविकल्पं दर्शयता यावज्जीवश्रुतिचोदितानामेव कर्मणां परित्याग उक्तः~। तत् कथम् ईदृशं विरुद्धमर्थम् अर्जुनाय ब्रूयात् भगवान्~, श्रोता वा कथं विरुद्धमर्थमवधारयेत्~॥~} 
तत्रैतत् स्यात् — गृहस्थानामेव श्रौतकर्मपरित्यागेन केवलादेव ज्ञानात् मोक्षः प्रतिषिध्यते, न तु आश्रमान्तराणामिति~। एतदपि पूर्वोत्तरविरुद्धमेव~। कथम्~? सर्वाश्रमिणां ज्ञानकर्मणोः समुच्चयो गीताशास्त्रे निश्चितः अर्थः इति प्रतिज्ञाय इह कथं तद्विरुद्धं केवलादेव ज्ञानात् मोक्षं ब्रूयात् आश्रमान्तराणाम्~॥~} 
अथ मतं श्रौतकर्मापेक्षया एतद्वचनम् ‘केवलादेव ज्ञानात् श्रौतकर्मरहितात् गृहस्थानां मोक्षः प्रतिषिध्यते’ इति~; तत्र गृहस्थानां विद्यमानमपि स्मार्तं कर्म अविद्यमानवत् उपेक्ष्य ‘ज्ञानादेव केवलात्’ इत्युच्यते इति~। एतदपि विरुद्धम्~। कथम्~? गृहस्थस्यैव स्मार्तकर्मणा समुच्चितात् ज्ञानात् मोक्षः प्रतिषिध्यते न तु आश्रमान्तराणामिति कथं विवेकिभिः शक्यमवधारयितुम्~। किञ्च — यदि मोक्षसाधनत्वेन स्मार्तानि कर्माणि ऊर्ध्वरेतसां समुच्चीयन्ते तथा गृहस्थस्यापि इष्यतां स्मार्तैरेव समुच्चयो न श्रौतैः~॥~} 
अथ श्रौतैः स्मार्तैश्च गृहस्थस्यैव समुच्चयः मोक्षाय, ऊर्ध्वरेतसां तु स्मार्तकर्ममात्रसमुच्चितात् ज्ञानात् मोक्ष इति~। तत्रैवं सति गृहस्थस्य आयासबाहुल्यात्~, श्रौतं स्मार्तं च बहुदुःखरूपं कर्म शिरसि आरोपितं स्यात्~॥~} 
अथ गृहस्थस्यैव आयासबाहुल्यकारणात् मोक्षः स्यात्~, न आश्रमान्तराणां श्रौतनित्यकर्मरहितत्वात् इति~। तदप्यसत्~, सर्वोपनिषत्सु इतिहासपुराणयोगशास्त्रेषु च ज्ञानाङ्गत्वेन मुमुक्षोः सर्वकर्मसंन्यासविधानात्~,  आश्रमविकल्पसमुच्चयविधानाच्च श्रुतिस्मृत्योः~॥~} 
सिद्धस्तर्हि सर्वाश्रमिणां ज्ञानकर्मणोः समुच्चयः — न, मुमुक्षोः सर्वकर्मसंन्यासविधानात्~। ‘पुत्रैषणाया वित्तैषणायाश्च लोकैषणायाश्च व्युत्थायाथ भिक्षाचर्यं चरन्ति’\footnote{बृ. उ. ३~। ५~। १} ‘तस्मात् न्यासमेषां तपसामतिरिक्तमाहुः’\footnote{तै. ना. ७९} ‘न्यास एवात्यरेचयत्’\footnote{तै. ना. ७८} इति, ‘न कर्मणा न प्रजया धनेन त्यागेनैके अमृतत्वमानशुः’\footnote{तै. ना. १२} इति च~। ‘ब्रह्मचर्यादेव प्रव्रजेत्’\footnote{जा. उ. ४} इत्याद्याः श्रुतयः~। ‘त्यज धर्ममधर्मं च उभे सत्यानृते त्यज~। उभे सत्यानृते त्यक्त्वा येन त्यजसि तत्त्यज~। ’\footnote{मो. ध. ३२९~। ४०} ‘संसारमेव निःसारं दृष्ट्वा सारदिदृक्षया~। प्रव्रजन्त्यकृतोद्वाहाः परं वैराग्यमाश्रिताः’\footnote{~? } इति बृहस्पतिः~। ‘कर्मणा बध्यते जन्तुर्विद्यया च विमुच्यते~। तस्मात्कर्म न कुर्वन्ति यतयः पारदर्शिनः’\footnote{मो. ध. २४१~। ७} इति शुकानुशासनम्~। इहापि च ‘सर्वकर्माणि मनसा संन्यस्य’\footnote{भ. गी. ५~। १३} इत्यादि~॥~} 
मोक्षस्य च अकार्यत्वात् मुमुक्षोः कर्मानर्थक्यम्~। नित्यानि प्रत्यवायपरिहारार्थानि इति चेत्~, न~; असंन्यासिविषयत्वात् प्रत्यवायप्राप्तेः~। न हि अग्निकार्याद्यकरणात् संन्यासिनः प्रत्यवायः कल्पयितुं शक्यः, यथा ब्रह्मचारिणामसंन्यासिनामपि कर्मिणाम्~। न तावत् नित्यानां कर्मणामभावादेव भावरूपस्य प्रत्यवायस्य उत्पत्तिः कल्पयितुं शक्या, ‘कथमसतः सज्जायेत’\footnote{छा. उ. ६~। २~। २} इति असतः सज्जन्मासम्भवश्रुतेः~। यदि विहिताकरणात् असम्भाव्यमपि प्रत्यवायं ब्रूयात् वेदः, तदा अनर्थकरः वेदः अप्रमाणमित्युक्तं स्यात्~; विहितस्य करणाकरणयोः दुःखमात्रफलत्वात्~। तथा च कारकं शास्त्रं न ज्ञापकम् इत्यनुपपन्नार्थं कल्पितं स्यात्~। न चैतदिष्टम्~। तस्मात् न संन्यासिनां कर्माणि~। अतो ज्ञानकर्मणोः समुच्चयानुपपत्तिः~; ‘ज्यायसी चेत् कर्मणस्ते मता बुद्धिः’\footnote{भ. गी. ३~। १} इति अर्जुनस्य प्रश्नानुपपत्तेश्च~॥~} 
यदि हि भगवता द्वितीयेऽध्याये ज्ञानं कर्म च समुच्चित्य त्वया अनुष्ठेयम् इत्युक्तं स्यात्~, ततः अर्जुनस्य प्रश्नः अनुपपन्नः ‘ज्यायसी चेत्कर्मणस्ते मता बुद्धिः’\footnote{भ. गी. ३~। १} इति~। अर्जुनाय चेत् बुद्धिकर्मणी त्वया अनुष्ठेये इत्युक्ते, या कर्मणो ज्यायसी बुद्धिः सापि उक्तैव इति ‘तत् किं कर्मणि घोरे मां नियोजयसि केशव’\footnote{भ. गी. ३~। १} इति उपालम्भः प्रश्नो वा न कथञ्चन उपपद्यते~। न च अर्जुनस्यैव ज्यायसी बुद्धिः न अनुष्ठेया इति भगवता उक्तं पूर्वम् इति कल्पयितुं युक्तम्~, येन ‘ज्यायसी चेत्’ इति विवेकतः प्रश्नः स्यात्~॥~} 
यदि पुनः एकस्य पुरुषस्य ज्ञानकर्मणोर्विरोधात् युगपदनुष्ठानं न सम्भवतीति भिन्नपुरुषानुष्ठेयत्वं भगवता पूर्वमुक्तं स्यात्~, ततोऽयं प्रश्न उपपन्नः ‘ज्यायसी चेत्’ इत्यादिः~। अविवेकतः प्रश्नकल्पनायामपि भिन्नपुरुषानुष्ठेयत्वेन ज्ञानकर्मनिष्ठयोः भगवतः प्रतिवचनं नोपपद्यते~। न च अज्ञाननिमित्तं भगवत्प्रतिवचनं कल्पनीयम्~। अस्माच्च भिन्नपुरुषानुष्ठेयत्वेन ज्ञानकर्मनिष्ठयोः भगवतः प्रतिवचनदर्शनात् ज्ञानकर्मणोः समुच्चयानुपपत्तिः~। तस्मात् केवलादेव ज्ञानात् मोक्ष इत्येषोऽर्थो निश्चितो गीतासु सर्वोपनिषत्सु च~॥~} 
ज्ञानकर्मणोः ‘एकं वद निश्चित्य’\footnote{भ. गी. ३~। २} इति च एकविषयैव प्रार्थना अनुपपन्ना, उभयोः समुच्चयसम्भवे~। ‘कुरु कर्मैव तस्मात्त्वम्’\footnote{भ. गी. ४~। १५} इति च ज्ञाननिष्ठासम्भवम् अर्जुनस्य अवधारणेन दर्शयिष्यति~॥~} 
\begin{center}{\bfseries अर्जुन उवाच —\\ ज्यायसी चेत्कर्मणस्ते मता बुद्धिर्जनार्दन~।\\तत्किं कर्मणि घोरे मां नियोजयसि केशव~॥~१~॥}\end{center} 
ज्यायसी श्रेयसी चेत् यदि कर्मणः सकाशात् ते तव मता अभिप्रेता बुद्धिर्ज्ञानं हे जनार्दन~। यदि बुद्धिकर्मणी समुच्चिते इष्टे तदा एकं श्रेयःसाधनमिति कर्मणो ज्यायसी बुद्धिः इति कर्मणः अतिरिक्तकरणं बुद्धेरनुपपन्नम् अर्जुनेन कृतं स्यात्~; न हि तदेव तस्मात् फलतोऽतिरिक्तं स्यात्~। तथा च, कर्मणः श्रेयस्करी भगवतोक्ता बुद्धिः, अश्रेयस्करं च कर्म कुर्विति मां प्रतिपादयति, तत् किं नु कारणमिति भगवत उपालम्भमिव कुर्वन् तत् किं कस्मात् कर्मणि घोरे क्रूरे हिंसालक्षणे मां नियोजयसि केशव इति च यदाह, तच्च नोपपद्यते~। अथ स्मार्तेनैव कर्मणा समुच्चयः सर्वेषां भगवता उक्तः अर्जुनेन च अवधारितश्चेत्~, ‘तत्किं कर्मणि घोरे मां नियोजयसि’\footnote{भ. गी. ३~। १} इत्यादि कथं युक्तं वचनम्~॥~१~॥\par
 किञ्च— } 
\begin{center}{\bfseries व्यामिश्रेणेव वाक्येन बुद्धिं मोहयसीव मे~।\\ तदेकं वद निश्चित्य येन श्रेयोऽहमाप्नुयाम्~॥~२~॥}\end{center} 
व्यामिश्रेणेव, यद्यपि विविक्ताभिधायी भगवान्~, तथापि मम मन्दबुद्धेः व्यामिश्रमिव भगवद्वाक्यं प्रतिभाति~। तेन मम बुद्धिं मोहयसि इव, मम बुद्धिव्यामोहापनयाय हि प्रवृत्तः त्वं तु कथं मोहयसि~? अतः ब्रवीमि बुद्धिं मोहयसि इव मे मम इति~। त्वं तु भिन्नकर्तृकयोः ज्ञानकर्मणोः एकपुरुषानुष्ठानासम्भवं यदि मन्यसे, तत्रैवं सति तत् तयोः एकं बुद्धिं कर्म वा इदमेव अर्जुनस्य योग्यं बुद्धिशक्त्यवस्थानुरूपमिति निश्चित्य वद ब्रूहि, येन ज्ञानेन कर्मणा वा अन्यतरेण श्रेयः अहम् आप्नुयां प्राप्नुयाम्~; इति यदुक्तं तदपि नोपपद्यते~॥~} 
यदि हि कर्मनिष्ठायां गुणभूतमपि ज्ञानं भगवता उक्तं स्यात्~, तत् कथं तयोः ‘एकं वद’ इति एकविषयैव अर्जुनस्य शुश्रूषा स्यात्~। न हि भगवता पूर्वमुक्तम् ‘अन्यतरदेव ज्ञानकर्मणोः वक्ष्यामि, नैव द्वयम्’ इति, येन उभयप्राप्त्यसम्भवम् आत्मनो मन्यमानः एकमेव प्रार्थयेत्~॥~२~॥\par
 प्रश्नानुरूपमेव प्रतिवचनं श्रीभगवानुवाच —}\\ 
\begin{center}{\bfseries श्रीभगवानुवाच —\\ लोकेऽस्मिन्द्विविधा निष्ठा पुरा प्रोक्ता मयानघ~।\\ज्ञानयोगेन साङ्‍ख्यानां कर्मयोगेन योगिनाम्~॥~३~॥}\end{center} 
लोके अस्मिन् शास्त्रार्थानुष्ठानाधिकृतानां त्रैवर्णिकानां द्विविधा द्विप्रकारा निष्ठा स्थितिः अनुष्ठेयतात्पर्यं पुरा पूर्वं सर्गादौ प्रजाः सृष्ट्वा तासाम् अभ्युदयनिःश्रेयसप्राप्तिसाधनं वेदार्थसम्प्रदायमाविष्कुर्वता प्रोक्ता मया सर्वज्ञेन ईश्वरेण हे अनघ अपाप~। तत्र का सा द्विविधा निष्ठा इत्याह — तत्र ज्ञानयोगेन ज्ञानमेव योगः तेन साङ्ख्यानाम् आत्मानात्मविषयविवेकविज्ञानवतां ब्रह्मचर्याश्रमादेव कृतसंन्यासानां वेदान्तविज्ञानसुनिश्चितार्थानां परमहंसपरिव्राजकानां ब्रह्मण्येव अवस्थितानां निष्ठा प्रोक्ता~। कर्मयोगेन कर्मैव योगः कर्मयोगः तेन कर्मयोगेन योगिनां कर्मिणां निष्ठा प्रोक्ता इत्यर्थः~। यदि च एकेन पुरुषेण एकस्मै पुरुषार्थाय ज्ञानं कर्म च समुच्चित्य अनुष्ठेयं भगवता इष्टम् उक्तं वक्ष्यमाणं वा गीतासु वेदेषु चोक्तम्~, कथमिह अर्जुनाय उपसन्नाय प्रियाय विशिष्टभिन्नपुरुषकर्तृके एव ज्ञानकर्मनिष्ठे ब्रूयात्~? यदि पुनः ‘अर्जुनः ज्ञानं कर्म च द्वयं श्रुत्वा स्वयमेवानुष्ठास्यति अन्येषां तु भिन्नपुरुषानुष्ठेयतां वक्ष्यामि इति’ मतं भगवतः कल्प्येत, तदा रागद्वेषवान् अप्रमाणभूतो भगवान् कल्पितः स्यात्~। तच्चायुक्तम्~। तस्मात् कयापि युक्त्या न समुच्चयो ज्ञानकर्मणोः~॥~} 
यत् अर्जुनेन उक्तं कर्मणो ज्यायस्त्वं बुद्धेः, तच्च स्थितम्~, अनिराकरणात्~। तस्याश्च ज्ञाननिष्ठायाः संन्यासिनामेवानुष्ठेयत्वम्~, भिन्नपुरुषानुष्ठेयत्ववचनात्~। भगवतः एवमेव अनुमतमिति गम्यते~॥~३~॥\par
 ‘मां च बन्धकारणे कर्मण्येव नियोजयसि’ इति विषण्णमनसमर्जुनम् ‘कर्म नारभे’ इत्येवं मन्वानमालक्ष्य आह भगवान् — न कर्मणामनारम्भात् इति~। अथवा — ज्ञानकर्मनिष्ठयोः परस्परविरोधात् एकेन पुरुषेण युगपत् अनुष्ठातुमशक्त्यत्वे सति इतरेतरानपेक्षयोरेव पुरुषार्थहेतुत्वे प्राप्ते कर्मनिष्ठाया ज्ञाननिष्ठाप्राप्तिहेतुत्वेन पुरुषार्थहेतुत्वम्~, न स्वातन्त्र्येण~; ज्ञाननिष्ठा तु कर्मनिष्ठोपायलब्धात्मिका सती स्वातन्त्र्येण पुरुषार्थहेतुः अन्यानपेक्षा, इत्येतमर्थं प्रदर्शयिष्यन् आह भगवान् —} 
\begin{center}{\bfseries न कर्मणामनारम्भान्नैष्कर्म्यं पुरुषोऽश्नुते~।\\न च संन्यसनादेव सिद्धिं समधिगच्छति~॥~४~॥}\end{center} 
न कर्मणां क्रियाणां यज्ञादीनाम् इह जन्मनि जन्मान्तरे वा अनुष्ठितानाम् उपात्तदुरितक्षयहेतुत्वेन सत्त्वशुद्धिकारणानां तत्कारणत्वेन च ज्ञानोत्पत्तिद्वारेण ज्ञाननिष्ठाहेतूनाम्~, ‘ज्ञानमुत्पद्यते पुंसां क्षयात्पापस्य कर्मणः~। यथादर्शतलप्रख्ये पश्यत्यात्मानमात्मनि’\footnote{मो. ध. २०४~। ८} इत्यादिस्मरणात्~, अनारम्भात् अननुष्ठानात् नैष्कर्म्यं निष्कर्मभावं कर्मशून्यतां ज्ञानयोगेन निष्ठां निष्क्रियात्मस्वरूपेणैव अवस्थानमिति यावत्~। पुरुषः न अश्नुते न प्राप्नोतीत्यर्थः~॥~} 
कर्मणामनारम्भान्नैष्कर्म्यं नाश्नुते इति वचनात् तद्विपर्ययात् तेषामारम्भात् नैष्कर्म्यमश्नुते इति गम्यते~। कस्मात् पुनः कारणात् कर्मणामनारम्भान्नैष्कर्म्यं नाश्नुते इति~? उच्यते, कर्मारम्भस्यैव नैष्कर्म्योपायत्वात्~। न ह्युपायमन्तरेण उपेयप्राप्तिरस्ति~। कर्मयोगोपायत्वं च नैष्कर्म्यलक्षणस्य ज्ञानयोगस्य, श्रुतौ इह च, प्रतिपादनात्~। श्रुतौ तावत् प्रकृतस्य आत्मलोकस्य वेद्यस्य वेदनोपायत्वेन ‘तमेतं वेदानुवचनेन ब्राह्मणा विविदिषन्ति यज्ञेन’\footnote{बृ. उ. ४~। ४~। २२} इत्यादिना कर्मयोगस्य ज्ञानयोगोपायत्वं प्रतिपादितम्~। इहापि च — ‘संन्यासस्तु महाबाहो दुःखमाप्तुमयोगतः’\footnote{भ. गी. ५~। ६} ‘योगिनः कर्म कुर्वन्ति सङ्गं त्यक्त्वात्मशुद्धये’\footnote{भ. गी. ५~। ११} ‘यज्ञो दानं तपश्चैव पावनानि मनीषिणाम्’\footnote{भ. गी. १८~। ५} इत्यादि प्रतिपादयिष्यति~॥~} 
ननु च ‘अभयं सर्वभूतेभ्यो दत्त्वा नैष्कर्म्यमाचरेत्’\footnote{अश्व. ४६~। १८} इत्यादौ कर्तव्यकर्मसंन्यासादपि नैष्कर्म्यप्राप्तिं दर्शयति~। लोके च कर्मणामनारम्भान्नैष्कर्म्यमिति प्रसिद्धतरम्~। अतश्च नैष्कर्म्यार्थिनः किं कर्मारम्भेण~? इति प्राप्तम्~। अत आह — न च संन्यसनादेवेति~। नापि संन्यसनादेव केवलात् कर्मपरित्यागमात्रादेव ज्ञानरहितात् सिद्धिं नैष्कर्म्यलक्षणां ज्ञानयोगेन निष्ठां समधिगच्छति न प्राप्नोति~॥~४~॥\par
 कस्मात् पुनः कारणात् कर्मसंन्यासमात्रादेव केवलात् ज्ञानरहितात् सिद्धिं नैष्कर्म्यलक्षणां पुरुषो नाधिगच्छति इति हेत्वाकाङ्क्षायामाह —} 
\begin{center}{\bfseries न हि कश्चित्क्षणमपि जातु तिष्ठत्यकर्मकृत्~।\\कार्यते ह्यवशः कर्म सर्वः प्रकृतिजैर्गुणैः~॥~५~॥}\end{center} 
न हि यस्मात् क्षणमपि कालं जातु कदाचित् कश्चित् तिष्ठति अकर्मकृत् सन्~। कस्मात्~? कार्यते प्रवर्त्यते हि यस्मात् अवश एव अस्वतन्त्र एव कर्म सर्वः प्राणी प्रकृतिजैः प्रकृतितो जातैः सत्त्वरजस्तमोभिः गुणैः~। अज्ञ इति वाक्यशेषः, यतो वक्ष्यति ‘गुणैर्यो न विचाल्यते’\footnote{भ. गी. १४~। २३} इति~। साङ्‍ख्यानां पृथक्करणात् अज्ञानामेव हि कर्मयोगः, न ज्ञानिनाम्~। ज्ञानिनां तु गुणैरचाल्यमानानां स्वतश्चलनाभावात् कर्मयोगो नोपपद्यते~। तथा च व्याख्यातम् ‘वेदाविनाशिनम्’\footnote{भ. गी. २~। २१} इत्यत्र~॥~५~॥\par
 यत्त्वनात्मज्ञः चोदितं कर्म नारभते इति तदसदेवेत्याह —} 
\begin{center}{\bfseries कर्मेन्द्रियाणि संयम्य य आस्ते मनसा स्मरन्~।\\इन्द्रियार्थान्विमूढात्मा मिथ्याचारः स उच्यते~॥~६~॥}\end{center} 
कर्मेन्द्रियाणि हस्तादीनि संयम्य संहृत्य यः आस्ते तिष्ठति मनसा स्मरन् चिन्तयन् इन्द्रियार्थान् विषयान् विमूढात्मा विमूढान्तःकरणः मिथ्याचारो मृषाचारः पापाचारः } 
सः उच्यते~॥~६~॥\par
 \begin{center}{\bfseries यस्त्विन्द्रियाणि मनसा नियम्यारभतेऽर्जुन~।\\कर्मेन्द्रियैः कर्मयोगमसक्तः स विशिष्यते~॥~७~॥}\end{center} 
यस्तु पुनः कर्मण्यधिकृतः अज्ञः बुद्धीन्द्रियाणि मनसा नियम्य आरभते अर्जुन कर्मेन्द्रियैः वाक्पाण्यादिभिः~। किमारभते इत्याह — कर्मयोगम् असक्तः सन् फलाभिसन्धिवर्जितः सः विशिष्यते इतरस्मात् मिथ्याचारात्~॥~७~॥\par
 यतः एवम् अतः —} 
\begin{center}{\bfseries नियतं कुरु कर्म त्वं कर्म ज्यायो ह्यकर्मणः~।\\शरीरयात्रापि च ते न प्रसिध्येदकर्मणः~॥~८~॥}\end{center} 
नियतं नित्यं शास्त्रोपदिष्टम्~, यो यस्मिन् कर्मणि अधिकृतः फलाय च अश्रुतं तत् नियतं कर्म, तत् कुरु त्वं हे अर्जुन, यतः कर्म ज्यायः अधिकतरं फलतः, हि यस्मात् अकर्मणः अकरणात् अनारम्भात्~। कथम्~? शरीरयात्रा शरीरस्थितिः अपि च ते तव न प्रसिध्येत् प्रसिद्धिं न गच्छेत् अकर्मणः अकरणात्~। अतः दृष्टः कर्माकर्मणोर्विशेषो लोके~॥~८~॥\par
 यच्च मन्यसे बन्धार्थत्वात् कर्म न कर्तव्यमिति तदप्यसत्~। कथम् —} 
\begin{center}{\bfseries यज्ञार्थात्कर्मणोऽन्यत्र लोकोऽयं कर्मबन्धनः~।\\तदर्थं कर्म कौन्तेय मुक्तसङ्गः समाचर~॥~९~॥}\end{center} 
‘यज्ञो वै विष्णुः’\footnote{तै. स. १~। ७~। ४} इति श्रुतेः यज्ञः ईश्वरः, तदर्थं यत् क्रियते तत् यज्ञार्थं कर्म~। तस्मात् कर्मणः अन्यत्र अन्येन कर्मणा लोकः अयम् अधिकृतः कर्मकृत् कर्मबन्धनः कर्म बन्धनं यस्य सोऽयं कर्मबन्धनः लोकः, न तु यज्ञार्थात्~। अतः तदर्थं यज्ञार्थं कर्म कौन्तेय, मुक्तसङ्गः कर्मफलसङ्गवर्जितः सन् समाचर निर्वर्तय~॥~९~॥\par
 इतश्च अधिकृतेन कर्म कर्तव्यम् —} 
\begin{center}{\bfseries सहयज्ञाः प्रजाः सृष्ट्वा पुरोवाच प्रजापतिः~।\\अनेन प्रसविष्यध्वमेष वोऽस्त्विष्टकामधुक्~॥~१०~॥}\end{center} 
सहयज्ञाः यज्ञसहिताः प्रजाः त्रयो वर्णाः ताः सृष्ट्वा उत्पाद्य पुरा पूर्वं सर्गादौ उवाच उक्तवान् प्रजापतिः प्रजानां स्रष्टा अनेन यज्ञेन प्रसविष्यध्वं प्रसवः वृद्धिः उत्पत्तिः तं कुरुध्वम्~। एष यज्ञः वः युष्माकम् अस्तु भवतु इष्टकामधुक् इष्टान् अभिप्रेतान् कामान् फलविशेषान् दोग्धीति इष्टकामधुक्~॥~१०~॥\par
 कथम् —} 
\begin{center}{\bfseries देवान्भावयतानेन ते देवा भावयन्तु वः~।\\परस्परं भावयन्तः श्रेयः परमवाप्स्यथ~॥~११~॥}\end{center} 
देवान् इन्द्रादीन् भावयत वर्धयत अनेन यज्ञेन~। ते देवा भावयन्तु आप्याययन्तु वृष्ट्यादिना वः युष्मान्~। एवं परस्परम् अन्योन्यं भावयन्तः श्रेयः परं मोक्षलक्षणं ज्ञानप्राप्तिक्रमेण अवाप्स्यथ~। स्वर्गं वा परं श्रेयः अवाप्स्यथ~॥~११~॥\par
 किञ्च—} 
\begin{center}{\bfseries इष्टान्भोगान्हि वो देवा दास्यन्ते यज्ञभाविताः~।\\तैर्दत्तानप्रदायैभ्यो यो भुङ्क्ते स्तेन एव सः~॥~१२~॥}\end{center} 
इष्टान् अभिप्रेतान् भोगान् हि वः युष्मभ्यं देवाः दास्यन्ते वितरिष्यन्ति स्त्रीपशुपुत्रादीन् यज्ञभाविताः यज्ञैः वर्धिताः तोषिताः इत्यर्थः~। तैः देवैः दत्तान् भोगान् अप्रदाय अदत्त्वा, आनृण्यमकृत्वा इत्यर्थः, एभ्यः देवेभ्यः, यः भुङ्क्ते स्वदेहेन्द्रियाण्येव तर्पयति स्तेन एव तस्कर एव सः देवादिस्वापहारी~॥~१२~॥\par
 ये पुनः —} 
\begin{center}{\bfseries यज्ञशिष्टाशिनः सन्तो मुच्यन्ते सर्वकिल्बिषैः~।\\भुञ्जते ते त्वघं पापा ये पचन्त्यात्मकारणात्~॥~१३~॥}\end{center} 
देवयज्ञादीन् निर्वर्त्य तच्छिष्टम् अशनम् अमृताख्यम् अशितुं शीलं येषां ते यज्ञशिष्टाशिनः सन्तः मुच्यन्ते सर्वकिल्बिषैः सर्वपापैः चुल्ल्यादिपञ्चसूनाकृतैः प्रमादकृतहिंसादिजनितैश्च अन्यैः~। ये तु आत्मम्भरयः, भुञ्जते ते तु अघं पापं स्वयमपि पापाः — ये पचन्ति पाकं निर्वर्तयन्ति आत्मकारणात् आत्महेतोः~॥~१३~॥\par
 इतश्च अधिकृतेन कर्म कर्तव्यम् जगच्चक्रप्रवृत्तिहेतुर्हि कर्म~। कथमिति उच्यते —} 
\begin{center}{\bfseries अन्नाद्भवन्ति भूतानि पर्जन्यादन्नसम्भवः~।\\यज्ञाद्भवति पर्जन्यो यज्ञः कर्मसमुद्भवः~॥~१४~॥}\end{center} 
अन्नात् भुक्तात् लोहितरेतःपरिणतात् प्रत्यक्षं भवन्ति जायन्ते भूतानि~। पर्जन्यात् वृष्टेः अन्नस्य सम्भवः अन्नसम्भवः~। यज्ञात् भवति पर्जन्यः, ‘अग्नौ प्रास्ताहुतिः सम्यगादित्यमुपतिष्ठते~। आदित्याज्जायते वृष्टिर्वृष्टेरन्नं ततः प्रजाः’\footnote{मनु. ३~। ७६} इति स्मृतेः~। यज्ञः अपूर्वम्~। स च यज्ञः कर्मसमुद्भवः ऋत्विग्यजमानयोश्च व्यापारः कर्म, तत् समुद्भवः यस्य यज्ञस्य अपूर्वस्य स यज्ञः कर्मसमुद्भवः~॥~१४~॥\par
 तच्चैवंविधं कर्म कुतो जातमित्याह —} 
\begin{center}{\bfseries कर्म ब्रह्मोद्भवं विद्धि ब्रह्माक्षरसमुद्भवम्~।\\तस्मात्सर्वगतं ब्रह्म नित्यं यज्ञे प्रतिष्ठितम्~॥~१५~॥}\end{center} 
कर्म ब्रह्मोद्भवं ब्रह्म वेदः सः उद्भवः कारणं प्रकाशको यस्य तत् कर्म ब्रह्मोद्भवं विद्धि विजानीहि~। ब्रह्म पुनः वेदाख्यम् अक्षरसमुद्भवम् अक्षरं ब्रह्म परमात्मा समुद्भवो यस्य तत् अक्षरसमुद्भवम्~। ब्रह्म वेद इत्यर्थः~। यस्मात् साक्षात् परमात्माख्यात् अक्षरात् पुरुषनिःश्वासवत् समुद्भूतं ब्रह्म तस्मात् सर्वार्थप्रकाशकत्वात् सर्वगतम्~; सर्वगतमपि सत् नित्यं सदा यज्ञविधिप्रधानत्वात् यज्ञे प्रतिष्ठितम्~॥~१५~॥\par
 \begin{center}{\bfseries एवं प्रवर्तितं चक्रं नानुवर्तयतीह यः~।\\अघायुरिन्द्रियारामो मोघं पार्थ स जीवति~॥~१६~॥}\end{center} 
एवम् इत्थम् ईश्वरेण वेदयज्ञपूर्वकं जगच्चक्रं प्रवर्तितं न अनुवर्तयति इह लोके यः कर्मणि अधिकृतः सन् अघायुः अघं पापम् आयुः जीवनं यस्य सः अघायुः, पापजीवनः इति यावत्~। इन्द्रियारामः इन्द्रियैः आरामः आरमणम् आक्रीडा विषयेषु यस्य सः इन्द्रियारामः मोघं वृथा हे पार्थ, स जीवति~॥~} 
तस्मात् अज्ञेन अधिकृतेन कर्तव्यमेव कर्मेति प्रकरणार्थः~। प्राक् आत्मज्ञाननिष्ठायोग्यताप्राप्तेः तादर्थ्येन कर्मयोगानुष्ठानम् अधिकृतेन अनात्मज्ञेन कर्तव्यमेवेत्येतत् ‘न कर्मणामनारम्भात्’\footnote{भ. गी. ३~। ४} इत्यत आरभ्य ‘शरीरयात्रापि च ते न प्रसिध्येदकर्मणः’\footnote{भ. गी. ३~। ८} इत्येवमन्तेन प्रतिपाद्य, ‘यज्ञार्थात् कर्मणोऽन्यत्र’\footnote{भ. गी. ३~। ९} इत्यादिना ‘मोघं पार्थ स जीवति’\footnote{भ. गी. ३~। १६} इत्येवमन्तेनापि ग्रन्थेन प्रासङ्गिकम् अधिकृतस्य अनात्मविदः कर्मानुष्ठाने बहु कारणमुक्तम्~। तदकरणे च दोषसङ्कीर्तनं कृतम्~॥~१६~॥\par
 एवं स्थिते किमेवं प्रवर्तितं चक्रं सर्वेणानुवर्तनीयम्~, आहोस्वित् पूर्वोक्तकर्मयोगानुष्ठानोपायप्राप्याम् अनात्मविदः ज्ञानयोगेनैव निष्ठाम् आत्मविद्भिः साङ्ख्यैः अनुष्ठेयामप्राप्तेनैव, इत्येवमर्थम् अर्जुनस्य प्रश्नमाशङ्क्य स्वयमेव वा शास्त्रार्थस्य विवेकप्रतिपत्त्यर्थम् ‘एतं वै तमात्मानं विदित्वा निवृत्तमिथ्याज्ञानाः सन्तः ब्राह्मणाः मिथ्याज्ञानवद्भिः अवश्यं कर्तव्येभ्यः पुत्रैषणादिभ्यो व्युत्थायाथ भिक्षाचर्यं शरीरस्थितिमात्रप्रयुक्तं चरन्ति न तेषामात्मज्ञाननिष्ठाव्यतिरेकेण अन्यत् कार्यमस्ति’\footnote{बृ. उ. ३~। ५~। १} इत्येवं श्रुत्यर्थमिह गीताशास्त्रे प्रतिपिपादयिषितमाविष्कुर्वन् आह भगवान् —} 
\begin{center}{\bfseries यस्त्वात्मरतिरेव स्यादात्मतृप्तश्च मानवः~।\\आत्मन्येव च सन्तुष्टस्तस्य कार्यं न विद्यते~॥~१७~॥}\end{center} 
यस्तु साङ्ख्यः आत्मज्ञाननिष्ठः आत्मरतिः आत्मन्येव रतिः न विषयेषु यस्य सः आत्मरतिरेव स्यात् भवेत् आत्मतृप्तश्च आत्मनैव तृप्तः न अन्नरसादिना सः मानवः मनुष्यः संन्यासी आत्मन्येव च सन्तुष्टः~। सन्तोषो हि बाह्यार्थलाभे सर्वस्य भवति, तमनपेक्ष्य आत्मन्येव च सन्तुष्टः सर्वतो वीततृष्ण इत्येतत्~। यः ईदृशः आत्मवित् तस्य कार्यं करणीयं न विद्यते नास्ति इत्यर्थः~॥~१७~॥\par
 किञ्च —} 
\begin{center}{\bfseries नैव तस्य कृतेनार्थो नाकृतेनेह कश्चन~।\\न चास्य सर्वभूतेषु कश्चिदर्थव्यपाश्रयः~॥~१८~॥}\end{center} 
नैव तस्य परमात्मरतेः कृतेन कर्मणा अर्थः प्रयोजनमस्ति~। अस्तु तर्हि अकृतेन अकरणेन प्रत्यवायाख्यः अनर्थः, न अकृतेन इह लोके कश्चन कश्चिदपि प्रत्यवायप्राप्तिरूपः आत्महानिलक्षणो वा नैव अस्ति~। न च अस्य सर्वभूतेषु ब्रह्मादिस्थावरान्तेषु भूतेषु कश्चित् अर्थव्यपाश्रयः प्रयोजननिमित्तक्रियासाध्यः व्यपाश्रयः व्यपाश्रयणम् आलम्बनं कञ्चित् भूतविशेषमाश्रित्य न साध्यः कश्चिदर्थः अस्ति, येन तदर्था क्रिया अनुष्ठेया स्यात्~। न त्वम् एतस्मिन् सर्वतःसम्प्लुतोदकस्थानीये सम्यग्दर्शने वर्तसे~॥~१८~॥\par
 यतः एवम् —} 
\begin{center}{\bfseries तस्मादसक्तः सततं कार्यं कर्म समाचर~।\\असक्तो ह्याचरन्कर्म परमाप्नोति पूरुषः~॥~१९~॥}\end{center} 
तस्मात् असक्तः सङ्गवर्जितः सततं सर्वदा कार्यं कर्तव्यं नित्यं कर्म समाचर निर्वर्तय~। असक्तो हि यस्मात् समाचरन् ईश्वरार्थं कर्म कुर्वन् परं मोक्षम् आप्नोति पूरुषः सत्त्वशुद्धिद्वारेण इत्यर्थः~॥~१९~॥\par
 यस्माच्च —} 
\begin{center}{\bfseries कर्मणैव हि संसिद्धिमास्थिता जनकादयः~।\\लोकसङ्ग्रहमेवापि सम्पश्यन्कर्तुमर्हसि~॥~२०~॥}\end{center} 
कर्मणैव हि यस्मात् पूर्वे क्षत्रियाः विद्वांसः संसिद्धिं मोक्षं गन्तुम् आस्थिताः प्रवृत्ताः~। के~? जनकादयः जनकाश्वपतिप्रभृतयः~। यदि ते प्राप्तसम्यग्दर्शनाः, ततः लोकसङ्ग्रहार्थं प्रारब्धकर्मत्वात् कर्मणा सहैव असंन्यस्यैव कर्म संसिद्धिमास्थिता इत्यर्थः~। अथ अप्राप्तसम्यग्दर्शनाः जनकादयः, तदा कर्मणा सत्त्वशुद्धिसाधनभूतेन क्रमेण संसिद्धिमास्थिता इति व्याख्येयः श्लोकः~। अथ मन्यसे पूर्वैरपि जनकादिभिः अजानद्भिरेव कर्तव्यं कर्म कृतम्~; तावता नावश्यमन्येन कर्तव्यं सम्यग्दर्शनवता कृतार्थेनेति~; तथापि प्रारब्धकर्मायत्तः त्वं लोकसङ्ग्रहम् एव अपि लोकस्य उन्मार्गप्रवृत्तिनिवारणं लोकसङ्ग्रहः तमेवापि प्रयोजनं सम्पश्यन् कर्तुम् अर्हसि~॥~२०~॥\par
 लोकसङ्ग्रहः किमर्थं कर्तव्य इत्युच्यते —} 
\begin{center}{\bfseries यद्यदाचरति श्रेष्ठस्तत्तदेवेतरो जनः~।\\स यत्प्रमाणं कुरुते लोकस्तदनुवर्तते~॥~२१~॥}\end{center} 
यद्यत् कर्म आचरति करोति श्रेष्ठः प्रधानः तत्तदेव कर्म आचरति इतरः अन्यः जनः तदनुगतः~। किञ्च सः श्रेष्ठः यत् प्रमाणं कुरुते लौकिकं वैदिकं वा लोकः तत् अनुवर्तते तदेव प्रमाणीकरोति इत्यर्थः~॥~२१~॥\par
 यदि अत्र ते लोकसङ्ग्रहकर्तव्यतायां विप्रतिपत्तिः तर्हि मां किं न पश्यसि~? —} 
\begin{center}{\bfseries न मे पार्थास्ति कर्तव्यं त्रिषु लोकेषु किञ्चन~।\\नानवाप्तमवाप्तव्यं वर्त एव च कर्मणि~॥~२२~॥}\end{center} 
न मे मम पार्थ न अस्ति न विद्यते कर्तव्यं त्रिषु अपि लोकेषु किञ्चन किञ्चिदपि~। कस्मात्~? न अनवाप्तम् अप्राप्तम् अवाप्तव्यं प्रापणीयम्~, तथापि वर्ते एव च कर्मणि अहम्~॥~२२~॥\par
 \begin{center}{\bfseries यदि ह्यहं न वर्तेय जातु कर्मण्यतन्द्रितः~।\\मम वर्त्मानुवर्तन्ते मनुष्याः पार्थ सर्वशः~॥~२३~॥}\end{center} 
यदि हि पुनः अहं न वर्तेय जातु कदाचित् कर्मणि अतन्द्रितः अनलसः सन् मम श्रेष्ठस्य सतः वर्त्म मार्गम् अनुवर्तन्ते मनुष्याः हे पार्थ, सर्वशः सर्वप्रकारैः~॥~२३~॥\par
 \begin{center}{\bfseries उत्सीदेयुरिमे लोका न कुर्यां कर्म चेदहम्~।\\सङ्करस्य च कर्ता स्यामुपहन्यामिमाः प्रजाः~॥~२४~॥}\end{center} 
उत्सीदेयुः विनश्येयुः इमे सर्वे लोकाः लोकस्थितिनिमित्तस्य कर्मणः अभावात् न कुर्यां कर्म चेत् अहम्~। किञ्च, सङ्करस्य च कर्ता स्याम्~। तेन कारणेन उपहन्याम् इमाः प्रजाः~। प्रजानामनुग्रहाय प्रवृत्तः उपहतिम् उपहननं कुर्याम् इत्यर्थः~। मम ईश्वरस्य अननुरूपमापद्येत~॥~२४~॥\par
 यदि पुनः अहमिव त्वं कृतार्थबुद्धिः, आत्मवित् अन्यो वा, तस्यापि आत्मनः कर्तव्याभावेऽपि परानुग्रह एव कर्तव्य इत्याह —} 
\begin{center}{\bfseries सक्ताः कर्मण्यविद्वांसो यथा कुर्वन्ति भारत~।\\कुर्याद्विद्वांस्तथासक्तश्चिकीर्षुर्लोकसङ्ग्रहम्~॥~२५~॥}\end{center} 
सक्ताः कर्मणि ‘अस्य कर्मणः फलं मम भविष्यति’ इति केचित् अविद्वांसः यथा कुर्वन्ति भारत, कुर्यात् विद्वान् आत्मवित् तथा असक्तः सन्~। तद्वत् किमर्थं करोति~? तत् शृणु — चिकीर्षुः कर्तुमिच्छुः लोकसङ्ग्रहम्~॥~२५~॥\par
 एवं लोकसङ्ग्रहं चिकीर्षोः न मम आत्मविदः कर्तव्यमस्ति अन्यस्य वा लोकसङ्ग्रहं मुक्त्वा~। ततः तस्य आत्मविदः इदमुपदिश्यते —} 
\begin{center}{\bfseries न बुद्धिभेदं जनयेदज्ञानां कर्मसङ्गिनाम्~।\\जोषयेत्सर्वकर्माणि विद्वान्युक्तः समाचरन्~॥~२६~॥}\end{center} 
बुद्धेर्भेदः बुद्धिभेदः ‘मया इदं कर्तव्यं भोक्तव्यं चास्य कर्मणः फलम्’ इति निश्चयरूपाया बुद्धेः भेदनं चालनं बुद्धिभेदः तं न जनयेत् न उत्पादयेत् अज्ञानाम् अविवेकिनां कर्मसङ्गिनां कर्मणि आसक्तानां आसङ्गवताम्~। किं नु कुर्यात्~? जोषयेत् कारयेत् सर्वकर्माणि विद्वान् स्वयं तदेव अविदुषां कर्म युक्तः अभियुक्तः समाचरन्~॥~२६~॥\par
 अविद्वानज्ञः कथं कर्मसु सज्जते इत्याह —} 
\begin{center}{\bfseries प्रकृतेः क्रियमाणानि गुणैः कर्माणि सर्वशः~।\\अहङ्कारविमूढात्मा कर्ताहमिति मन्यते~॥~२७~॥}\end{center} 
प्रकृतेः प्रकृतिः प्रधानं सत्त्वरजस्तमसां गुणानां साम्यावस्था तस्याः प्रकृतेः गुणैः विकारैः कार्यकरणरूपैः क्रियमाणानि कर्माणि लौकिकानि शास्त्रीयाणि च सर्वशः सर्वप्रकारैः अहङ्कारविमूढात्मा कार्यकरणसङ्घातात्मप्रत्ययः अहङ्कारः तेन विविधं नानाविधं मूढः आत्मा अन्तःकरणं यस्य सः अयं कार्यकरणधर्मा कार्यकरणाभिमानी अविद्यया कर्माणि आत्मनि मन्यमानः तत्तत्कर्मणाम् अहं कर्ता इति मन्यते~॥~२७~॥\par
 यः पुनर्विद्वान् —} 
\begin{center}{\bfseries तत्त्ववित्तु महाबाहो गुणकर्मविभागयोः~।\\गुणा गुणेषु वर्तन्त इति मत्वा न सज्जते~॥~२८~॥}\end{center} 
तत्त्ववित् तु महाबाहो~। कस्य तत्त्ववित्~? गुणकर्मविभागयोः गुणविभागस्य कर्मविभागस्य च तत्त्ववित् इत्यर्थः~। गुणाः करणात्मकाः गुणेषु विषयात्मकेषु वर्तन्ते न आत्मा इति मत्वा न सज्जते सक्तिं न करोति~॥~२८~॥\par
 ये पुनः —} 
\begin{center}{\bfseries प्रकृतेर्गुणसंमूढाः सज्जन्ते गुणकर्मसु~।\\तानकृत्स्नविदो मन्दान्कृत्स्नविन्न विचालयेत्~॥~२९~॥}\end{center} 
प्रकृतेः गुणैः सम्यक् मूढाः संमोहिताः सन्तः सज्जन्ते गुणानां कर्मसु गुणकर्मसु ‘वयं कर्म कुर्मः फलाय’ इति | तान् कर्मसङ्गिनः अकृत्स्नविदः कर्मफलमात्रदर्शिनः मन्दान् मन्दप्रज्ञान् कृत्स्नवित् आत्मवित् स्वयं न विचालयेत् बुद्धिभेदकरणमेव चालनं तत् न कुर्यात् इत्यर्थः~॥~२९~॥\par
 कथं पुनः कर्मण्यधिकृतेन अज्ञेन मुमुक्षुणा कर्म कर्तव्यमिति, उच्यते —} 
\begin{center}{\bfseries मयि सर्वाणि कर्माणि संन्यस्याध्यात्मचेतसा~।\\निराशीर्निर्ममो भूत्वा युध्यस्व विगतज्वरः~॥~३०~॥}\end{center} 
मयि वासुदेवे परमेश्वरे सर्वज्ञे सर्वात्मनि सर्वाणि कर्माणि संन्यस्य निक्षिप्य अध्यात्मचेतसा विवेकबुद्ध्या ‘अहं कर्ता ईश्वराय भृत्यवत् करोमि’ इत्यनया बुद्ध्या~। किञ्च, निराशीः त्यक्ताशीः निर्ममः ममभावश्च निर्गतः यस्य तव स त्वं निर्ममो भूत्वा युध्यस्व  विगतज्वरः विगतसन्तापः विगतशोकः सन्नित्यर्थः~॥~३०~॥\par
 यदेतन्मम मतं कर्म कर्तव्यम् इति सप्रमाणमुक्तं तत् तथा —} 
\begin{center}{\bfseries ये मे मतमिदं नित्यमनुतिष्ठन्ति मानवाः~।\\श्रद्धावन्तोऽनसूयन्तो मुच्यन्ते तेऽपि कर्मभिः~॥~३१~॥}\end{center} 
ये मे मदीयम् इदं मतं नित्यम् अनुतिष्ठन्ति अनुवर्तन्ते मानवाः मनुष्याः श्रद्धावन्तः श्रद्धधानाः अनसूयन्तः असूयां च मयि परमगुरौ वासुदेवे अकुर्वन्तः, मुच्यन्ते तेऽपि एवं भूताः कर्मभिः धर्माधर्माख्यैः~॥~३१~॥\par
 \begin{center}{\bfseries ये त्वेतदभ्यसूयन्तो नानुतिष्ठन्ति मे मतम्~।\\सर्वज्ञानविमूढांस्तान्विद्धि नष्टानचेतसः~॥~३२~॥}\end{center} 
ये तु तद्विपरीताः एतत् मम मतम् अभ्यसूयन्तः निन्दन्तः न अनुतिष्ठन्ति नानुवर्तन्ते मे मतम्~, सर्वेषु ज्ञानेषु विविधं मूढाः ते~। सर्वज्ञानविमूढान् तान् विद्धि जानीहि } 
नष्टान् नाशं गतान् अचेतसः अविवेकिनः~॥~३२~॥\par
 कस्मात् पुनः कारणात् त्वदीयं मतं नानुतिष्ठन्ति, परधर्मान् अनुतिष्ठन्ति, स्वधर्मं च नानुवर्तन्ते, त्वत्प्रतिकूलाः कथं न बिभ्यति त्वच्छासनातिक्रमदोषात्~? तत्राह —} 
\begin{center}{\bfseries सदृशं चेष्टते स्वस्याः प्रकृतेर्ज्ञानवानपि~।\\प्रकृतिं यान्ति भूतानि निग्रहः किं करिष्यति~॥~३३~॥}\end{center} 
सदृशम् अनुरूपं चेष्टते चेष्टां करोति | कस्य~? स्वस्याः स्वकीयायाः प्रकृतेः~। प्रकृतिर्नाम पूर्वकृतधर्माधर्मादिसंस्कारः वर्तमानजन्मादौ अभिव्यक्तः~; सा प्रकृतिः~। तस्याः सदृशमेव सर्वो जन्तुः ज्ञानवानपि चेष्टते, किं पुनर्मूर्खः~। तस्मात् प्रकृतिं यान्ति अनुगच्छन्ति भूतानि प्राणिनः~। निग्रहः निषेधरूपः किं करिष्यति मम वा अन्यस्य वा~॥~३३~॥\par
 यदि सर्वो जन्तुः आत्मनः प्रकृतिसदृशमेव चेष्टते, न च प्रकृतिशून्यः कश्चित् अस्ति, ततः पुरुषकारस्य विषयानुपपत्तेः शास्त्रानर्थक्यप्राप्तौ इदमुच्यते —} 
\begin{center}{\bfseries इन्द्रियस्येन्द्रियस्यार्थे रागद्वेषौ व्यवस्थितौ~।\\तयोर्न वशमागच्छेत्तौ ह्यस्य परिपन्थिनौ~॥~३४~॥}\end{center} 
इन्द्रियस्येन्द्रियस्य अर्थे सर्वेन्द्रियाणामर्थे शब्दादिविषये इष्टे रागः अनिष्टे द्वेषः इत्येवं प्रतीन्द्रियार्थं रागद्वेषौ अवश्यंभाविनौ तत्र अयं पुरुषकारस्य शास्त्रार्थस्य च विषय उच्यते~। शास्त्रार्थे प्रवृत्तः पूर्वमेव रागद्वेषयोर्वशं नागच्छेत्~। या हि पुरुषस्य प्रकृतिः सा रागद्वेषपुरःसरैव स्वकार्ये पुरुषं प्रवर्तयति~। तदा स्वधर्मपरित्यागः परधर्मानुष्ठानं च भवति~। यदा पुनः रागद्वेषौ तत्प्रतिपक्षेण नियमयति तदा शास्त्रदृष्टिरेव पुरुषः भवति, न प्रकृतिवशः~। तस्मात् तयोः रागद्वेषयोः वशं न आगच्छेत्~, यतः तौ हि अस्य पुरुषस्य परिपन्थिनौ श्रेयोमार्गस्य विघ्नकर्तारौ तस्करौ इव पथीत्यर्थः~॥~३४~॥\par
 तत्र रागद्वेषप्रयुक्तो मन्यते शास्त्रार्थमप्यन्यथा ‘परधर्मोऽपि धर्मत्वात् अनुष्ठेय एव’ इति, तदसत् —} 
\begin{center}{\bfseries श्रेयान्स्वधर्मो विगुणः परधर्मात्स्वनुष्ठितात्~।\\स्वधर्मे निधनं श्रेयः परधर्मो भयावहः~॥~३५~॥}\end{center} 
श्रेयान् प्रशस्यतरः स्वो धर्मः स्वधर्मः विगुणः अपि विगतगुणोऽपि अनुष्ठीयमानः परधर्मात् स्वनुष्ठितात् साद्गुण्येन सम्पादितादपि~। स्वधर्मे स्थितस्य निधनं मरणमपि श्रेयः परधर्मे स्थितस्य जीवितात्~। कस्मात्~? परधर्मः भयावहः नरकादिलक्षणं भयमावहति यतः~॥~} 
यद्यपि अनर्थमूलम् ‘ध्यायतो विषयान्पुंसः’\footnote{भ. गी. २~। ६२} इति ‘रागद्वेषौ ह्यस्य परिपन्थिनौ’\footnote{भ. गी. ३~। ३४} इति च उक्तम्~, विक्षिप्तम् अनवधारितं च तदुक्तम्~। तत् सङ्क्षिप्तं निश्चितं च इदमेवेति ज्ञातुमिच्छन् अर्जुनः उवाच ‘ज्ञाते हि तस्मिन् तदुच्छेदाय यत्नं कुर्याम्’ इति~॥~३५~॥\par
 {\bfseries अर्जुन उवाच —}\\
\begin{center}{\bfseries अथ केन प्रयुक्तोऽयं पापं चरति पूरुषः~।\\अनिच्छन्नपि वार्ष्णेय बलादिव नियोजितः~॥~३६~॥}\end{center} 
अथ केन हेतुभूतेन प्रयुक्तः सन् राज्ञेव भृत्यः अयं पापं कर्म चरति आचरति पूरुषः पुरुषः स्वयम् अनिच्छन् अपि हे वार्ष्णेय वृष्णिकुलप्रसूत, बलात् इव नियोजितः राज्ञेव इत्युक्तो दृष्टान्तः~॥~} 
शृणु त्वं तं वैरिणं सर्वानर्थकरं यं त्वं पृच्छसि इति भगवान् उवाच —~॥~३६~॥\par
 {\bfseries श्रीभगवानुवाच —}\\
\begin{center}{\bfseries काम एष क्रोध एष रजोगुणसमुद्भवः~।\\महाशनो महापाप्मा विद्ध्येनमिह वैरिणम्~॥~३७~॥}\end{center} 
‘ऐश्वर्यस्य समग्रस्य धर्मस्य यशसः श्रियः~। वैराग्यस्याथ मोक्षस्य षण्णां भग इतीङ्गना’\footnote{वि. पु. ६~। ५~। ७४} ऐश्वर्यादिषट्कं यस्मिन् वासुदेवे नित्यमप्रतिबद्धत्वेन सामस्त्येन च वर्तते, ‘उत्पत्तिं प्रलयं चैव भूतानामागतिं गतिम्~। वेत्ति विद्यामविद्यां च स वाच्यो भगवानिति’\footnote{वि. पु. ६~। ५~। ७८} उत्पत्त्यादिविषयं च विज्ञानं यस्य स वासुदेवः वाच्यः भगवान् इति~॥~} 
काम एषः सर्वलोकशत्रुः यन्निमित्ता सर्वानर्थप्राप्तिः प्राणिनाम्~। स एष कामः प्रतिहतः केनचित् क्रोधत्वेन परिणमते~। अतः क्रोधः अपि एष एव रजोगुणसमुद्भवः रजश्च तत् गुणश्च रजोगुणः सः समुद्भवः यस्य सः कामः रजोगुणसमुद्भवः, रजोगुणस्य वा समुद्भवः~। कामो हि उद्भूतः रजः प्रवर्तयन् पुरुषं प्रवर्तयति~; ‘तृष्णया हि अहं कारितः’ इति दुःखिनां रजःकार्ये सेवादौ प्रवृत्तानां प्रलापः श्रूयते~। महाशनः महत् अशनं अस्येति महाशनः~; अत एव महापाप्मा~; कामेन हि प्रेरितः जन्तुः पापं करोति~। अतः विद्धि एनं कामम् इह संसारे वैरिणम्~॥~३७~॥\par
 कथं वैरी इति दृष्टान्तैः प्रत्याययति —} 
\begin{center}{\bfseries धूमेनाव्रियते वह्निर्यथादर्शो मलेन च~।\\यथोल्बेनावृतो गर्भस्तथा तेनेदमावृतम्~॥~३८~॥}\end{center} 
धूमेन सहजेन आव्रियते वह्निः प्रकाशात्मकः अप्रकाशात्मकेन, यथा वा आदर्शो मलेन च, यथा उल्बेन च जरायुणा गर्भवेष्टनेन आवृतः आच्छादितः गर्भः तथा तेन इदम् आवृतम्~॥~३८~॥\par
 किं पुनस्तत् इदंशब्दवाच्यं यत् कामेनावृतमित्युच्यते —} 
\begin{center}{\bfseries आवृतं ज्ञानमेतेन ज्ञानिनो नित्यवैरिणा~।\\कामरूपेण कौन्तेय दुष्पूरेणानलेन च~॥~३९~॥}\end{center} 
आवृतम् एतेन ज्ञानं ज्ञानिनः नित्यवैरिणा, ज्ञानी हि जानाति ‘अनेन अहमनर्थे प्रयुक्तः’ इति पूर्वमेव~। दुःखी च भवती नित्यमेव~। अतः असौ ज्ञानिनो नित्यवैरी, न तु मूर्खस्य~। स हि कामं तृष्णाकाले मित्रमिव पश्यन् तत्कार्ये दुःखे प्राप्ते जानाति } 
‘तृष्णया अहं दुःखित्वमापादितः’ इति, न पूर्वमेव~। अतः ज्ञानिन एव नित्यवैरी~। किंरूपेण~? कामरूपेण कामः इच्छैव रूपमस्य इति कामरूपः तेन दुष्पूरेण दुःखेन पूरणमस्य इति दुष्पूरः तेन अनलेन न अस्य अलं पर्याप्तिः विद्यते इत्यनलः तेन च~॥~३९~॥\par
 किमधिष्ठानः पुनः कामः ज्ञानस्य आवरणत्वेन वैरी सर्वस्य लोकस्य~? इत्यपेक्षायामाह, ज्ञाते हि शत्रोरधिष्ठाने सुखेन निबर्हणं कर्तुं शक्यत इति —} 
\begin{center}{\bfseries इन्द्रियाणि मनो बुद्धिरस्याधिष्ठानमुच्यते~।\\एतैर्विमोहयत्येष ज्ञानमावृत्य देहिनम्~॥~४०~॥}\end{center} 
इन्द्रियाणि मनः बुद्धिश्च अस्य कामस्य अधिष्ठानम् आश्रयः उच्यते~। एतैः इन्द्रियादिभिः आश्रयैः विमोहयति विविधं मोहयति एष कामः ज्ञानम् आवृत्य आच्छाद्य देहिनं शरीरिणम्~॥~४०~॥\par
 यतः एवम् —} 
\begin{center}{\bfseries तस्मात्त्वमिन्द्रियाण्यादौ नियम्य भरतर्षभ~।\\पाप्मानं प्रजहिह्येनं ज्ञानविज्ञाननाशनम्~॥~४१~॥}\end{center} 
तस्मात् त्वम् इन्द्रियाणि आदौ पूर्वमेव नियम्य वशीकृत्य भरतर्षभ पाप्मानं पापाचारं कामं प्रजहिहि परित्यज एनं प्रकृतं वैरिणं ज्ञानविज्ञाननाशनं ज्ञानं शास्त्रतः आचार्यतश्च आत्मादीनाम् अवबोधः, विज्ञानं विशेषतः तदनुभवः, तयोः ज्ञानविज्ञानयोः श्रेयःप्राप्तिहेत्वोः नाशनं नाशकरं प्रजहिहि आत्मनः परित्यजेत्यर्थः~॥~४१~॥\par
 इन्द्रियाण्यादौ नियम्य कामं शत्रुं जहिहि इत्युक्तम्~; तत्र किमाश्रयः कामं जह्यात् इत्युच्यते —} 
\begin{center}{\bfseries इन्द्रियाणि पराण्याहुरिन्द्रियेभ्यः परं मनः~।\\मनसस्तु परा बुद्धिर्यो बुद्धेः परतस्तु सः~॥~४२~॥}\end{center} 
इन्द्रियाणि श्रोत्रादीनि पञ्च देहं स्थूलं बाह्यं परिच्छिन्नं च अपेक्ष्य सौक्ष्म्यान्तरत्वव्यापित्वाद्यपेक्षया पराणि प्रकृष्टानि आहुः पण्डिताः~। तथा इन्द्रियेभ्यः परं मनः सङ्कल्पविकल्पात्मकम्~। तथा मनसः तु परा बुद्धिः निश्चयात्मिका~। तथा यः सर्वदृश्येभ्यः बुद्ध्यन्तेभ्यः आभ्यन्तरः, यं देहिनम् इन्द्रियादिभिः आश्रयैः युक्तः कामः ज्ञानावरणद्वारेण मोहयति इत्युक्तम्~। बुद्धेः परतस्तु सः, सः बुद्धेः द्रष्टा पर आत्मा~॥~४२~॥\par
 ततः किम् —} 
\begin{center}{\bfseries एवं बुद्धेः परं बुद्ध्वा संस्तभ्यात्मानमात्मना~।\\जहि शत्रुं महाबाहो कामरूपं दुरासदम्~॥~४३~॥}\end{center} 
एवं बुद्धेः परम् आत्मानं बुद्ध्वा ज्ञात्वा संस्तभ्य सम्यक् स्तम्भनं कृत्वा आत्मानं स्वेनैव आत्मना संस्कृतेन मनसा सम्यक् समाधायेत्यर्थः~। जहि एनं शत्रुं हे महाबाहो कामरूपं दुरासदं दुःखेन आसदः आसादनं प्राप्तिः यस्य तं दुरासदं दुर्विज्ञेयानेकविशेषमिति~॥~४३~॥\par
 
इति श्रीमत्परमहंसपरिव्राजकाचार्यस्य श्रीगोविन्दभगवत्पूज्यपादशिष्यस्य श्रीमच्छङ्करभगवतः कृतौ श्रीमद्भगवद्गीताभाष्ये तृतीयोऽध्यायः~॥\par
 
योऽयं योगः अध्यायद्वयेनोक्तः ज्ञाननिष्ठालक्षणः		, ससंन्यासः कर्मयोगोपायः, यस्मिन् वेदार्थः परिसमाप्तः, प्रवृत्तिलक्षणः निवृत्तिलक्षणश्च, गीतासु च सर्वासु अयमेव योगो विवक्षितो भगवता~। अतः परिसमाप्तं वेदार्थं मन्वानः तं वंशकथनेन स्तौति श्रीभगवान् —} 
\begin{center}{\bfseries श्रीभगवानुवाच —\\ इमं विवस्वते योगं प्रोक्तवानहमव्ययम्~।\\विवस्वान्मनवे प्राह मनुरिक्ष्वाकवेऽब्रवीत्~॥~१~॥}\end{center} 
इमम् अध्यायद्वयेनोक्तं योगं विवस्वते आदित्याय सर्गादौ प्रोक्तवान् अहं जगत्परिपालयितॄणां क्षत्रियाणां बलाधानाय तेन योगबलेन युक्ताः समर्था भवन्ति ब्रह्म परिरक्षितुम्~। ब्रह्मक्षत्रे परिपालिते जगत् परिपालयितुमलम्~। अव्ययम् अव्ययफलत्वात्~। न ह्यस्य योगस्य सम्यग्दर्शननिष्ठालक्षणस्य मोक्षाख्यं फलं व्येति~। स च विवस्वान् मनवे प्राह~। मनुः इक्ष्वाकवे स्वपुत्राय आदिराजाय अब्रवीत्~॥~१~॥\par
 \begin{center}{\bfseries एवं परम्पराप्राप्तमिमं राजर्षयो विदुः~।\\स कालेनेह महता योगो नष्टः परन्तप~॥~२~॥}\end{center} 
एवं क्षत्रियपरम्पराप्राप्तम् इमं राजर्षयः राजानश्च ते ऋषयश्च राजर्षयः विदुः इमं योगम्~। स योगः कालेन इह महता दीर्घण नष्टः विच्छिन्नसम्प्रदायः संवृत्तः~। हे परन्तप, आत्मनः विपक्षभूताः परा इति उच्यन्ते, तान् शौर्यतेजोगभस्तिभिः भानुरिव तापयतीति परन्तपः शत्रुतापन इत्यर्थः~॥~२~॥\par
 दुर्बलानजितेन्द्रियान् प्राप्य नष्टं योगमिममुपलभ्य लोकं च अपुरुषार्थसम्बन्धिनम् —} 
\begin{center}{\bfseries स एवायं मया तेऽद्य योगः प्रोक्तः पुरातनः~।\\भक्तोऽसि मे सखा चेति रहस्यं ह्येतदुत्तमम्~॥~३~॥}\end{center} 
स एव अयं मया ते तुभ्यम् अद्य इदानीं योगः प्रोक्तः पुरातनः भक्तः असि मे सखा } 
च असि इति~। रहस्यं हि यस्मात् एतत् उत्तमं योगः ज्ञानम् इत्यर्थः~॥~३~॥\par
 भगवता विप्रतिषिद्धमुक्तमिति मा भूत् कस्यचित् बुद्धिः इति परिहारार्थं चोद्यमिव कुर्वन् अर्जुन उवाच — } 
\begin{center}{\bfseries अर्जुन उवाच —\\ अपरं भवतो जन्म परं जन्म विवस्वतः~।\\कथमेतद्विजानीयां त्वमादौ प्रोक्तवानिति~॥~४~॥}\end{center} 
अपरम् अर्वाक् वसुदेवगृहे भवतो जन्म~। परं पूर्वं सर्गादौ जन्म उत्पत्तिः विवस्वतः आदित्यस्य~। तत् कथम् एतत् विजानीयाम् अविरुद्धार्थतया, यः त्वमेव आदौ प्रोक्तवान् इमं योगं स एव इदानीं मह्यं प्रोक्तवानसि इति~॥~४~॥\par
 या वासुदेवे अनीश्वरासर्वज्ञाशङ्का मूर्खाणाम्~, तां परिहरन् श्रीभगवानुवाच, यदर्थो ह्यर्जुनस्य प्रश्नः —} 
\begin{center}{\bfseries श्रीभगवानुवाच —\\ बहूनि मे व्यतीतानि जन्मानि तव चार्जुन~।\\तान्यहं वेद सर्वाणि न त्वं वेत्थ परन्तप~॥~५~॥}\end{center} 
बहूनि मे मम व्यतीतानि अतिक्रान्तानि जन्मानि तव च हे अर्जुन~। तानि अहं वेद जाने सर्वाणि न त्वं वेत्थ न जानीषे, धर्माधर्मादिप्रतिबद्धज्ञानशक्तित्वात्~। अहं पुनः नित्यशुद्धबुद्धमुक्तस्वभावत्वात् अनावरणज्ञानशक्तिरिति वेद अहं हे परन्तप~॥~५~॥\par
 कथं तर्हि तव नित्येश्वरस्य धर्माधर्माभावेऽपि जन्म इति, उच्यते —} 
\begin{center}{\bfseries अजोऽपि सन्नव्ययात्मा भूतानामीश्वरोऽपि सन्~।\\प्रकृतिं स्वामधिष्ठाय सम्भवाम्यात्ममायया~॥~६~॥}\end{center} 
अजोऽपि जन्मरहितोऽपि सन्~, तथा अव्ययात्मा अक्षीणज्ञानशक्तिस्वभावोऽपि सन्~, तथा भूतानां ब्रह्मादिस्तम्बपर्यन्तानाम् ईश्वरः ईशनशीलोऽपि सन्~, प्रकृतिं स्वां मम वैष्णवीं मायां त्रिगुणात्मिकाम्~, यस्या वशे सर्वं जगत् वर्तते, यया मोहितं सत् स्वमात्मानं वासुदेवं न जानाति, तां प्रकृतिं स्वाम् अधिष्ठाय वशीकृत्य सम्भवामि देहवानिव भवामि जात इव आत्ममायया आत्मनः मायया, न परमार्थतो लोकवत्~॥~६~॥\par
 तच्च जन्म कदा किमर्थं च इत्युच्यते —} 
\begin{center}{\bfseries यदा यदा हि धर्मस्य ग्लानिर्भवति भारत~।\\अभ्युत्थानमधर्मस्य तदात्मानं सृजाम्यहम्~॥~७~॥}\end{center} 
यदा यदा हि धर्मस्य ग्लानिः हानिः वर्णाश्रमादिलक्षणस्य प्राणिनामभ्युदयनिःश्रेयससाधनस्य भवति भारत, अभ्युत्थानम् उद्भवः अधर्मस्य, तदा तदा आत्मानं सृजामि अहं मायया~॥~७~॥\par
 किमर्थम्~? —} 
\begin{center}{\bfseries परित्राणाय साधूनां विनाशाय च दुष्कृताम्~।\\धर्मसंस्थापनार्थाय सम्भवामि युगे युगे~॥~८~॥}\end{center} 
परित्राणाय परिरक्षणाय साधूनां सन्मार्गस्थानाम्~, विनाशाय च दुष्कृतां पापकारिणाम्~, किञ्च धर्मसंस्थापनार्थाय धर्मस्य सम्यक् स्थापनं तदर्थं सम्भवामि युगे युगे प्रतियुगम्~॥~८~॥\par
 तत् —} 
\begin{center}{\bfseries जन्म कर्म च मे दिव्यमेवं यो वेत्ति तत्त्वतः~।\\त्यक्त्वा देहं पुनर्जन्म नैति मामेति सोऽर्जुन~॥~९~॥}\end{center} 
जन्म मायारूपं कर्म च साधूनां परित्राणादि मे मम दिव्यम् अप्राकृतम् ऐश्वरम् एवं यथोक्तं यः वेत्ति तत्त्वतः तत्त्वेन यथावत् त्यक्त्वा देहम् इमं पुनर्जन्म पुनरुत्पत्तिं न एति न प्राप्नोति~। माम् एति आगच्छति सः मुच्यते हे अर्जुन~॥~९~॥\par
 नैष मोक्षमार्ग इदानीं प्रवृत्तः~; किं तर्हि~? पूर्वमपि —} 
\begin{center}{\bfseries वीतरागभयक्रोधा मन्मया मामुपाश्रिताः~।\\बहवो ज्ञानतपसा पूता मद्भावमागताः~॥~१०~॥}\end{center} 
वीतरागभयक्रोधाः रागश्च भयं च क्रोधश्च वीताः विगताः येभ्यः ते वीतरागभयक्रोधाः मन्मयाः ब्रह्मविदः ईश्वराभेददर्शिनः मामेव च परमेश्वरम् उपाश्रिताः केवलज्ञाननिष्ठा इत्यर्थः~। बहवः अनेके ज्ञानतपसा ज्ञानमेव च परमात्मविषयं तपः तेन ज्ञानतपसा पूताः परां शुद्धिं गताः सन्तः मद्भावम् ईश्वरभावं मोक्षम् आगताः समनुप्राप्ताः~। इतरतपोनिरपेक्षज्ञाननिष्ठा इत्यस्य लिङ्गम् ‘ज्ञानतपसा’ इति विशेषणम्~॥~१०~॥\par
 तव तर्हि रागद्वेषौ स्तः, येन केभ्यश्चिदेव आत्मभावं प्रयच्छसि न सर्वेभ्यः इत्युच्यते —} 
\begin{center}{\bfseries ये यथा मां प्रपद्यन्ते तांस्तथैव भजाम्यहम्~।\\मम वर्त्मानुवर्तन्ते मनुष्याः पार्थ सर्वशः~॥~११~॥}\end{center} 
ये यथा येन प्रकारेण येन प्रयोजनेन यत्फलार्थितया मां प्रपद्यन्ते तान् तथैव तत्फलदानेन भजामि अनुगृह्णामि अहम् इत्येतत्~। तेषां मोक्षं प्रति अनर्थित्वात्~। न हि एकस्य मुमुक्षुत्वं फलार्थित्वं च युगपत् सम्भवति~। अतः ये फलार्थिनः तान् फलप्रदानेन, ये यथोक्तकारिणस्तु अफलार्थिनः मुमुक्षवश्च तान् ज्ञानप्रदानेन, ये ज्ञानिनः संन्यासिनः मुमुक्षवश्च तान् मोक्षप्रदानेन, तथा आर्तान् आर्तिहरणेन इत्येवं यथा प्रपद्यन्ते ये तान् तथैव भजामि इत्यर्थः~। न पुनः रागद्वेषनिमित्तं मोहनिमित्तं वा कञ्चित् भजामि~। सर्वथापि सर्वावस्थस्य मम ईश्वरस्य वर्त्म मार्गम् अनुवर्तन्ते मनुष्याः — यत्फलार्थितया यस्मिन् कर्मणि अधिकृताः ये प्रयतन्ते ते मनुष्या अत्र उच्यन्ते — हे पार्थ सर्वशः सर्वप्रकारैः~॥~११~॥\par
 यदि तव ईश्वरस्य रागादिदोषाभावात् सर्वप्राणिषु अनुजिघृक्षायां तुल्यायां सर्वफलप्रदानसमर्थे च त्वयि सति ‘वासुदेवः सर्वम्’ इति ज्ञानेनैव मुमुक्षवः सन्तः कस्मात् त्वामेव सर्वे न प्रतिपद्यन्ते इति~? शृणु तत्र कारणम् —} 
\begin{center}{\bfseries काङ्क्षन्तः कर्मणां सिद्धिं यजन्त इह देवताः~।\\क्षिप्रं हि मानुषे लोके सिद्धिर्भवति कर्मजा~॥~१२~॥}\end{center} 
काङ्क्षन्तः अभीप्सन्तः कर्मणां सिद्धिं फलनिष्पत्तिं प्रार्थयन्तः यजन्ते इह अस्मिन् लोके देवताः इन्द्राग्न्याद्याः~; ‘अथ योऽन्यां देवतामुपास्ते अन्योऽसावन्योऽहमस्मीति न स वेद यथा पशुरेवं स देवानाम्’\footnote{बृ. उ. १~। ४~। १०} इति श्रुतेः~। तेषां हि भिन्नदेवतायाजिनां फलाकाङ्क्षिणां क्षिप्रं शीघ्रं हि यस्मात् मानुषे लोके, मनुष्यलोके हि शास्त्राधिकारः~। ‘क्षिप्रं हि मानुषे लोके’ इति विशेषणात् अन्येष्वपि कर्मफलसिद्धिं दर्शयति भगवान्~। मानुषे लोके वर्णाश्रमादिकर्माणि इति विशेषः, तेषां च वर्णाश्रमाद्यधिकारिकर्मणां फलसिद्धिः क्षिप्रं भवति~। कर्मजा कर्मणो जाता~॥~१२~॥\par
 मानुषे एव लोके वर्णाश्रमादिकर्माधिकारः, न अन्येषु लोकेषु इति नियमः किंनिमित्त इति~? अथवा वर्णाश्रमादिप्रविभागोपेताः मनुष्याः मम वर्त्म अनुवर्तन्ते सर्वशः इत्युक्तम्~। कस्मात्पुनः कारणात् नियमेन तवैव वर्त्म अनुवर्तन्ते न अन्यस्य इति~? उच्यते —} 
\begin{center}{\bfseries चातुर्वर्ण्यं मया सृष्टं गुणकर्मविभागशः~।\\तस्य कर्तारमपि मां विद्ध्यकर्तारमव्ययम्~॥~१३~॥}\end{center} 
चत्वार एव वर्णाः चातुर्वर्ण्यं मया ईश्वरेण सृष्टम् उत्पादितम्~, ‘ब्राह्मणोऽस्य मुखमासीत्’\footnote{ऋ. १०~। ८~। ९१} इत्यादिश्रुतेः~। गुणकर्मविभागशः गुणविभागशः कर्मविभागशश्च~। गुणाः सत्त्वरजस्तमांसि~। तत्र सात्त्विकस्य सत्त्वप्रधानस्य ब्राह्मणस्य ‘शमो दमस्तपः’\footnote{भ. गी. १८~। ४२} इत्यादीनि कर्माणि, सत्त्वोपसर्जनरजःप्रधानस्य क्षत्रियस्य शौर्यतेजःप्रभृतीनि कर्माणि, तमउपसर्जनरजःप्रधानस्य वैश्यस्य कृष्यादीनि कर्माणि, रजउपसर्जनतमःप्रधानस्य शूद्रस्य शुश्रूषैव कर्म इत्येवं गुणकर्मविभागशः चातुर्वर्ण्यं मया सृष्टम् इत्यर्थः~। तच्च इदं चातुर्वर्ण्यं न अन्येषु लोकेषु, अतः मानुषे लोके इति विशेषणम्~। हन्त तर्हि चातुर्वर्ण्यस्य सर्गादेः कर्मणः कर्तृत्वात् तत्फलेन युज्यसे, अतः न त्वं नित्यमुक्तः नित्येश्वरश्च इति~? उच्यते — यद्यपि मायासंव्यवहारेण तस्य कर्मणः कर्तारमपि सन्तं मां परमार्थतः विद्धि अकर्तारम्~। अत एव अव्ययम् असंसारिणं च मां विद्धि~॥~१३~॥\par
 येषां तु कर्मणां कर्तारं मां मन्यसे परमार्थतः तेषाम् अकर्ता एवाहम्~, यतः —} 
\begin{center}{\bfseries न मां कर्माणि लिम्पन्ति न मे कर्मफले स्पृहा~।\\इति मां योऽभिजानाति कर्मभिर्न स बध्यते~॥~१४~॥}\end{center} 
न मां तानि कर्माणि लिम्पन्ति देहाद्यारम्भकत्वेन, अहङ्काराभावात्~। न च तेषां कर्मणां फलेषु मे मम स्पृहा तृष्णा~। येषां तु संसारिणाम् ‘अहं कर्ता’ इत्यभिमानः कर्मसु, स्पृहा तत्फलेषु च, तान् कर्माणि लिम्पन्ति इति युक्तम्~, तदभावात् न मां कर्माणि लिम्पन्ति~। इति एवं यः अन्योऽपि माम् आत्मत्वेन अभिजानाति ‘नाहं कर्ता न मे कर्मफले स्पृहा’ इति सः कर्मभिः न बध्यते, तस्यापि न देहाद्यारम्भकाणि कर्माणि भवन्ति इत्यर्थः~॥~१४~॥\par
 ‘नाहं कर्ता न मे कर्मफले स्पृहा’ इति —} 
\begin{center}{\bfseries एवं ज्ञात्वा कृतं कर्म पूर्वैरपि मुमुक्षुभिः~।\\कुरु कर्मैव तस्मात्त्वं पूर्वैः पूर्वतरं कृतम्~॥~१५~॥}\end{center} 
एवं ज्ञात्वा कृतं कर्म पूर्वैः अपि अतिक्रान्तैः मुमुक्षुभिः~। कुरु तेन कर्मैव त्वम्~, न तूष्णीमासनं नापि संन्यासः कर्तव्यः, तस्मात् त्वं पूर्वैरपि अनुष्ठितत्वात्~, यदि अनात्मज्ञः त्वं तदा आत्मशुद्ध्यर्थम्~, तत्त्वविच्चेत् लोकसङ्ग्रहार्थं पूर्वैः जनकादिभिः पूर्वतरं कृतं न अधुनातनं कृतं निर्वर्तितम्~॥~१५~॥\par
 तत्र कर्म चेत् कर्तव्यं त्वद्वचनादेव करोम्यहम्~, किं विशेषितेन ‘पूर्वैः पूर्वतरं कृतम्’ इत्युच्यते~; यस्मात् महत् वैषम्यं कर्मणि~। कथम्~? —} 
\begin{center}{\bfseries किं कर्म किमकर्मेति कवयोऽप्यत्र मोहिताः~।\\तत्ते कर्म प्रवक्ष्यामि यज्ज्ञात्वा मोक्ष्यसेऽशुभात्~॥~१६~॥}\end{center} 
किं कर्म किं च अकर्म इति कवयः मेधाविनः अपि अत्र अस्मिन् कर्मादिविषये मोहिताः मोहं गताः~। तत् अतः ते तुभ्यम् अहं कर्म अकर्म च प्रवक्ष्यामि, यत् ज्ञात्वा विदित्वा कर्मादि मोक्ष्यसे अशुभात् संसारात्~॥~१६~॥\par
 न चैतत्त्वया मन्तव्यम् — कर्म नाम देहादिचेष्टा लोकप्रसिद्धम्~, अकर्म नाम तदक्रिया तूष्णीमासनम्~; किं तत्र बोद्धव्यम्~? इति~। कस्मात्~, उच्यते —} 
\begin{center}{\bfseries कर्मणो ह्यपि बोद्धव्यं बोद्धव्यं च विकर्मणः~।\\अकर्मणश्च बोद्धव्यं गहना कर्मणो गतिः~॥~१७~॥}\end{center} 
कर्मणः शास्त्रविहितस्य हि यस्मात् अपि अस्ति बोद्धव्यम्~, बोद्धव्यं च अस्त्येव विकर्मणः प्रतिषिद्धस्य, तथा अकर्मणश्च तूष्णीम्भावस्य बोद्धव्यम् अस्ति इति त्रिष्वप्यध्याहारः कर्तव्यः~। यस्मात् गहना विषमा दुर्ज्ञेया — कर्मणः इति उपलक्षणार्थं कर्मादीनाम् — कर्माकर्मविकर्मणां गतिः याथात्म्यं तत्त्वम् इत्यर्थः~॥~१७~॥\par
 किं पुनस्तत्त्वं कर्मादेः यत् बोद्धव्यं वक्ष्यामि इति प्रतिज्ञातम्~? उच्यते —} 
\begin{center}{\bfseries कर्मण्यकर्म यः पश्येदकर्मणि च कर्म यः~।\\स बुद्धिमान्मनुष्येषु स युक्तः कृत्स्नकर्मकृत्~॥~१८~॥}\end{center} 
कर्मणि, क्रियते इति कर्म व्यापारमात्रम्~, तस्मिन् कर्मणि अकर्म कर्माभावं यः पश्येत्~, अकर्मणि च कर्माभावे कर्तृतन्त्रत्वात् प्रवृत्तिनिवृत्त्योः — वस्तु अप्राप्यैव हि सर्व एव क्रियाकारकादिव्यवहारः अविद्याभूमौ एव — कर्म यः पश्येत् पश्यति, सः बुद्धिमान् मनुष्येषु, सः युक्तः योगी च, कृत्स्नकर्मकृत् समस्तकर्मकृच्च सः, इति स्तूयते कर्माकर्मणोरितरेतरदर्शी~॥~} 
ननु किमिदं विरुद्धमुच्यते ‘कर्मणि अकर्म यः पश्येत्’ इति ‘अकर्मणि च कर्म’ इति~; न हि कर्म अकर्म स्यात्~, अकर्म वा कर्म~। तत्र विरुद्धं कथं पश्येत् द्रष्टा~? — न, अकर्म एव परमार्थतः सत् कर्मवत् अवभासते मूढदृष्टेः लोकस्य, तथा कर्मैव अकर्मवत्~। तत्र यथाभूतदर्शनार्थमाह भगवान् — ‘कर्मण्यकर्म यः पश्येत्’ इत्यादि~। अतो न विरुद्धम्~। बुद्धिमत्त्वाद्युपपत्तेश्च~। ‘बोद्धव्यम्’\footnote{भ. गी. ४~। १७} इति च यथाभूतदर्शनमुच्यते~। न च विपरीतज्ञानात् अशुभात् मोक्षणं स्यात्~; ‘यत् ज्ञात्वा मोक्ष्यसेऽशुभात्’\footnote{भ. गी. ४~। १६} इति च उक्तम्~। तस्मात् कर्माकर्मणी विपर्ययेण गृहीते प्राणिभिः तद्विपर्ययग्रहणनिवृत्त्यर्थं भगवतो वचनम् ‘कर्मण्यकर्म यः’ इत्यादि~। न च अत्र कर्माधिकरणमकर्म अस्ति, कुण्डे बदराणीव~। नापि अकर्माधिकरणं कर्मास्ति, कर्माभावत्वादकर्मणः~। अतः विपरीतगृहीते एव कर्माकर्मणी लौकिकैः, यथा मृगतृष्णिकायामुदकं शुक्तिकायां वा रजतम्~। ननु कर्म कर्मैव सर्वेषां न क्वचित् व्यभिचरति — तत् न, नौस्थस्य नावि गच्छन्त्यां तटस्थेषु अगतिषु नगेषु प्रतिकूलगतिदर्शनात्~, दूरेषु चक्षुषा असंनिकृष्टेषु गच्छत्सु गत्यभावदर्शनात्~, एवम् इहापि अकर्मणि कर्मदर्शनं कर्मणि च अकर्मदर्शनं विपरीतदर्शनं येन तन्निराकरणार्थमुच्यते ‘कर्मण्यकर्म यः पश्येत्’ इत्यादि~॥~} 
तदेतत् उक्तप्रतिवचनमपि असकृत् अत्यन्तविपरीतदर्शनभाविततया मोमुह्यमानो लोकः श्रुतमपि असकृत् तत्त्वं विस्मृत्य विस्मृत्य मिथ्याप्रसङ्गम् अवतार्यावतार्य चोदयति इति पुनः पुनः उत्तरमाह भगवान्~, दुर्विज्ञेयत्वं च आलक्ष्य वस्तुनः~। ‘अव्यक्तोऽयमचिन्त्योऽयम्’\footnote{भ. गी. २~। २५} ‘न जायते म्रियते’\footnote{भ. गी. २~। २०} इत्यादिना आत्मनि कर्माभावः श्रुतिस्मृतिन्यायप्रसिद्धः उक्तः वक्ष्यमाणश्च~। तस्मिन् आत्मनि कर्माभावे अकर्मणि कर्मविपरीतदर्शनम् अत्यन्तनिरूढम्~; यतः, ‘किं कर्म किमकर्मेति कवयोऽप्यत्र मोहिताः’\footnote{भ. गी. ४~। १६}~। देहाद्याश्रयं कर्म आत्मन्यध्यारोप्य ‘अहं कर्ता, मम एतत् कर्म, मया अस्य कर्मणः फलं भोक्तव्यम्’ इति च, तथा ‘अहं तूष्णीं भवामि, येन अहं निरायासः अकर्मा सुखी स्याम्’ इति कार्यकरणाश्रयं व्यापारोपरमं तत्कृतं च सुखित्वम् आत्मनि अध्यारोप्य ‘न करोमि किञ्चित्~, तूष्णीं सुखमासे’ इति अभिमन्यते लोकः~। तत्रेदं लोकस्य विपररीतदर्शनापनयाय आह भगवान् — ‘कर्मण्यकर्म यः पश्येत्’ इत्यादि~॥~} 
अत्र च कर्म कर्मैव सत् कार्यकरणाश्रयं कर्मरहिते अविक्रिये आत्मनि सर्वैः अध्यस्तम्~, यतः पण्डितोऽपि ‘अहं करोमि’ इति मन्यते~। अतः आत्मसमवेततया सर्वलोकप्रसिद्धे कर्मणि नदीकूलस्थेष्विव वृक्षेषु गतिप्रातिलोम्येन अकर्म कर्माभावं यथाभूतं गत्यभावमिव वृक्षेषु यः पश्येत्~, अकर्मणि च कार्यकरणव्यापारोपरमे कर्मवत् आत्मनि अध्यारोपिते, ‘तूष्णीं अकुर्वन् सुखं आसे’ इत्यहङ्काराभिसन्धिहेतुत्वात्~, तस्मिन् अकर्मणि च कर्म यः पश्येत्~, यः एवं कर्माकर्मविभागज्ञः सः बुद्धिमान् पण्डितः मनुष्येषु, सः युक्तः योगी कृत्स्नकर्मकृच्च सः अशुभात् मोक्षितः कृतकृत्यो भवति इत्यर्थः~॥~} 
अयं श्लोकः अन्यथा व्याख्यातः कैश्चित्~। कथम्~? नित्यानां किल कर्मणाम् ईश्वरार्थे अनुष्ठीयमानानां तत्फलाभावात् अकर्माणि तानि उच्यन्ते गौण्या वृत्त्या~। तेषां च अकरणम् अकर्म~; तच्च प्रत्यवायफलत्वात् कर्म उच्यते गौण्यैव वृत्त्या~। तत्र नित्ये कर्मणि अकर्म यः पश्येत् फलाभावात्~; यथा धेनुरपि गौः अगौः इत्युच्यते क्षीराख्यं फलं न प्रयच्छति इति, तद्वत्~। तथा नित्याकरणे तु अकर्मणि च कर्म यः पश्येत् नरकादिप्रत्यवायफलं प्रयच्छति इति~॥~} 
नैतत् युक्तं व्याख्यानम्~। एवं ज्ञानात् अशुभात् मोक्षानुपपत्तेः ‘यज्ज्ञात्वा मोक्ष्यसेऽशुभात्’\footnote{भ. गी. ४~। १६} इति भगवता उक्तं वचनं बाध्येत~। कथम्~? नित्यानामनुष्ठानात् अशुभात् स्यात् नाम मोक्षणम्~, न तु तेषां फलाभावज्ञानात्~। न हि नित्यानां फलाभावज्ञानम् अशुभमुक्तिफलत्वेन चोदितम्~, नित्यकर्मज्ञानं वा~। न च भगवतैवेहोक्तम्~। एतेन अकर्मणि कर्मदर्शनं प्रत्युक्तम्~। न हि अकर्मणि ‘कर्म’ इति दर्शनं कर्तव्यतया इह चोद्यते, नित्यस्य तु कर्तव्यतामात्रम्~। न च ‘अकरणात् नित्यस्य प्रत्यवायो भवति’ इति विज्ञानात् किञ्चित् फलं स्यात्~। नापि नित्याकरणं ज्ञेयत्वेन चोदितम्~। नापि ‘कर्म अकर्म’ इति मिथ्यादर्शनात् अशुभात् मोक्षणं बुद्धिमत्त्वं युक्तता कृत्स्नकर्मकृत्त्वादि च फलम् उपपद्यते, स्तुतिर्वा~। मिथ्याज्ञानमेव हि साक्षात् अशुभरूपम्~। कुतः अन्यस्मादशुभात् मोक्षणम्~? न हि तमः तमसो निवर्तकं भवति~॥~} 
ननु कर्मणि यत् अकर्मदर्शनम् अकर्मणि वा कर्मदर्शनं न तत् मिथ्याज्ञानम्~; किं तर्हि~? गौणं फलभावाभावनिमित्तम् — न, कर्माकर्मविज्ञानादपि गौणात् फलस्य अश्रवणात्~। नापि श्रुतहान्यश्रुतपरिकल्पनायां कश्चित् विशेष उपलभ्यते~। स्वशब्देनापि शक्यं वक्तुम् ‘नित्यकर्मणां फलं नास्ति, अकरणाच्च तेषां नरकपातः स्यात्’ इति~; तत्र व्याजेन परव्यामोहरूपेण ‘कर्मण्यकर्म यः पस्येत्’ इत्यादिना किम्~? तत्र एवं व्याचक्षाणेन भगवतोक्तं वाक्यं लोकव्यामोहार्थमिति व्यक्तं कल्पितं स्यात्~। न च एतत् छद्मरूपेण वाक्येन रक्षणीयं वस्तु~; नापि शब्दान्तरेण पुनः पुनः उच्यमानं सुबोधं स्यात् इत्येवं वक्तुं युक्तम्~। ‘कर्मण्येवाधिकारस्ते’\footnote{भ. गी. २~। ४७} इत्यत्र हि स्फुटतर उक्तः अर्थः, न पुनर्वक्तव्यो भवति~। सर्वत्र च प्रशस्तं बोद्धव्यं च कर्तव्यमेव~। न निष्प्रयोजनं बोद्धव्यमित्युच्यते~॥~} 
न च मिथ्याज्ञानं बोद्धव्यं भवति, तत्प्रत्युपस्थापितं वा वस्त्वाभासम्~। नापि नित्यानाम् अकरणात् अभावात् प्रत्यवायभावोत्पत्तिः, ‘नासतो विद्यते भावः’\footnote{भ. गी. २~। १६} इति वचनात् ‘कथं असतः सज्जायेत’\footnote{छा. उ. ६~। २~। २} इति च दर्शितम् असतः सज्जन्मप्रतिषेधात्~। असतः सदुत्पत्तिं ब्रुवता असदेव सद्भवेत्~, सच्चापि असत् भवेत् इत्युक्तं स्यात्~। तच्च अयुक्तम्~, सर्वप्रमाणविरोधात्~। न च निष्फलं विदध्यात् कर्म शास्त्रम्~, दुःखस्वरूपत्वात्~, दुःखस्य च बुद्धिपूर्वकतया कार्यत्वानुपपत्तेः~। तदकरणे च नरकपाताभ्युपगमात् अनर्थायैव उभयथापि करणे च अकरणे च शास्त्रं निष्फलं कल्पितं स्यात्~। स्वाभ्युपगमविरोधश्च ‘नित्यं निष्फलं कर्म’ इति अभ्युपगम्य ‘मोक्षफलाय’ इति ब्रुवतः~। तस्मात् यथाश्रुत एवार्थः ‘कर्मण्यकर्म यः’ इत्यादेः~। तथा च व्याख्यातः अस्माभिः श्लोकः~॥~१८~॥\par
 तदेतत् कर्मणि अकर्मदर्शनं स्तूयते —} 
\begin{center}{\bfseries यस्य सर्वे समारम्भाः कामसङ्कल्पवर्जिताः~।\\ज्ञानाग्निदग्धकर्माणं तमाहुः पण्डितं बुधाः~॥~१९~॥}\end{center} 
यस्य यथोक्तदर्शिनः सर्वे यावन्तः समारम्भाः सर्वाणि कर्माणि, समारभ्यन्ते इति समारम्भाः, कामसङ्कल्पवर्जिताः कामैः तत्कारणैश्च सङ्कल्पैः वर्जिताः मुधैव चेष्टामात्रा अनुष्ठीयन्ते~; प्रवृत्तेन चेत् लोकसङ्ग्रहार्थम्~, निवृत्तेन चेत् जीवनमात्रार्थम्~। तं ज्ञानाग्निदग्धकर्माणं कर्मादौ अकर्मादिदर्शनं ज्ञानं तदेव अग्निः तेन ज्ञानाग्निना दग्धानि शुभाशुभलक्षणानि कर्माणि यस्य तम् आहुः परमार्थतः पण्डितं बुधाः ब्रह्मविदः~॥~१९~॥\par
 यस्तु अकर्मादिदर्शी, सः अकर्मादिदर्शनादेव निष्कर्मा संन्यासी जीवनमात्रार्थचेष्टः सन् कर्मणि न प्रवर्तते, यद्यपि प्राक् विवेकतः प्रवृत्तः~। यस्तु प्रारब्धकर्मा सन् उत्तरकालमुत्पन्नात्मसम्यग्दर्शनः स्यात्~, सः सर्वकर्मणि प्रयोजनमपश्यन् ससाधनं कर्म परित्यजत्येव~। सः कुतश्चित् निमित्तात् कर्मपरित्यागासम्भवे सति कर्मणि तत्फले च सङ्गरहिततया स्वप्रयोजनाभावात् लोकसङ्ग्रहार्थं पूर्ववत् कर्मणि प्रवृत्तोऽपि नैव किञ्चित् करोति, ज्ञानाग्निदग्धकर्मत्वात् तदीयं कर्म अकर्मैव सम्पद्यते इत्येतमर्थं दर्शयिष्यन् आह —} 
\begin{center}{\bfseries त्यक्त्वा कर्मफलासङ्गं नित्यतृप्तो निराश्रयः~।\\कर्मण्यभिप्रवृत्तोऽपि नैव किञ्चित्करोति सः~॥~२०~॥}\end{center} 
त्यक्त्वा कर्मसु अभिमानं फलासङ्गं च यथोक्तेन ज्ञानेन नित्यतृप्तः निराकाङ्क्षो विषयेषु इत्यर्थः~। निराश्रयः आश्रयरहितः, आश्रयो नाम यत् आश्रित्य पुरुषार्थं सिसाधयिषति, दृष्टादृष्टेष्टफलसाधनाश्रयरहित इत्यर्थः~। विदुषा क्रियमाणं कर्म परमार्थतोऽकर्मैव, तस्य निष्क्रियात्मदर्शनसम्पन्नत्वात्~। तेन एवंभूतेन स्वप्रयोजनाभावात् ससाधनं कर्म परित्यक्तव्यमेव इति प्राप्ते, ततः निर्गमासम्भवात् लोकसङ्ग्रहचिकीर्षया शिष्टविगर्हणापरिजिहीर्षया वा पूर्ववत् कर्मणि अभिप्रवृत्तोऽपि निष्क्रियात्मदर्शनसम्पन्नत्वात् नैव किञ्चित् करोति सः~॥~२०~॥\par
 यः पुनः पूर्वोक्तविपरीतः प्रागेव कर्मारम्भात् ब्रह्मणि सर्वान्तरे प्रत्यगात्मनि निष्क्रिये सञ्जातात्मदर्शनः स दृष्टादृष्टेष्टविषयाशीर्विवर्जिततया दृष्टादृष्टार्थे कर्मणि प्रयोजनमपश्यन् ससाधनं कर्म संन्यस्य शरीरयात्रामात्रचेष्टः यतिः ज्ञाननिष्ठो मुच्यते इत्येतमर्थं दर्शयितुमाह —} 
\begin{center}{\bfseries निराशीर्यतचित्तात्मा त्यक्तसर्वपरिग्रहः~।\\शारीरं केवलं कर्म कुर्वन्नाप्नोति किल्बिषम्~॥~२१~॥}\end{center} 
निराशीः निर्गताः आशिषः यस्मात् सः निराशीः, यतचित्तात्मा चित्तम् अन्तःकरणम् आत्मा बाह्यः कार्यकरणसङ्घातः तौ उभावपि यतौ संयतौ येन सः यतचित्तात्मा, त्यक्तसर्वपरिग्रहः त्यक्तः सर्वः परिग्रहः येन सः त्यक्तसर्वपरिग्रहः, शारीरं शरीरस्थितिमात्रप्रयोजनम्~, केवलं तत्रापि अभिमानवर्जितम्~, कर्म कुर्वन् न आप्नोति न प्राप्नोति किल्बिषम् अनिष्टरूपं पापं धर्मं च~। धर्मोऽपि मुमुक्षोः किल्बिषमेव बन्धापादकत्वात्~। तस्मात् ताभ्यां मुक्तः भवति, संसारात् मुक्तो भवति इत्यर्थः~॥~} 
‘शारीरं केवलं कर्म’ इत्यत्र किं शरीरनिर्वर्त्यं शारीरं कर्म अभिप्रेतम्~, आहोस्वित् शरीरस्थितिमात्रप्रयोजनं शारीरं कर्म इति~? किं च अतः यदि शरीरनिर्वर्त्यं शारीरं कर्म यदि वा शरीरस्थितिमात्रप्रयोजनं शारीरम् इति~? उच्यते — यदा शरीरनिर्वर्त्यं कर्म शारीरम् अभिप्रेतं स्यात्~, तदा दृष्टादृष्टप्रयोजनं कर्म प्रतिषिद्धमपि शरीरेण कुर्वन् नाप्नोति किल्बिषम् इति ब्रुवतो विरुद्धाभिधानं प्रसज्येत~। शास्त्रीयं च कर्म दृष्टादृष्टप्रयोजनं शरीरेण कुर्वन् नाप्नोति किल्बिषम् इत्यपि ब्रुवतः अप्राप्तप्रतिषेधप्रसङ्गः~। ‘शारीरं कर्म कुर्वन्’ इति विशेषणात् केवलशब्दप्रयोगाच्च वाङ्मनसनिर्वर्त्यं कर्म विधिप्रतिषेधविषयं धर्माधर्मशब्दवाच्यं कुर्वन् प्राप्नोति किल्बिषम् इत्युक्तं स्यात्~। तत्रापि वाङ्मनसाभ्यां विहितानुष्ठानपक्षे किल्बिषप्राप्तिवचनं विरुद्धम् आपद्येत~। प्रतिषिद्धसेवापक्षेऽपि भूतार्थानुवादमात्रम् अनर्थकं स्यात्~। यदा तु शरीरस्थितिमात्रप्रयोजनं शारीरं कर्म अभिप्रेतं भवेत्~, तदा दृष्टादृष्टप्रयोजनं कर्म विधिप्रतिषेधगम्यं शरीरवाङ्मनसनिर्वर्त्यम् अन्यत् अकुर्वन् तैरेव शरीरादिभिः शरीरस्थितिमात्रप्रयोजनं केवलशब्दप्रयोगात् ‘अहं करोमि’ इत्यभिमानवर्जितः शरीरादिचेष्टामात्रं लोकदृष्ट्या कुर्वन् नाप्नोति किल्बिषं~। एवंभूतस्य पापशब्दवाच्यकिल्बिषप्राप्त्यसम्भवात् किल्बिषं संसारं न आप्नोति~; ज्ञानाग्निदग्धसर्वकर्मत्वात् अप्रतिबन्धेन मुच्यत एव इति पूर्वोक्तसम्यग्दर्शनफलानुवाद एव एषः~। एवम् ‘शारीरं केवलं कर्म’ इत्यस्य अर्थस्य परिग्रहे निरवद्यं भवति~॥~२१~॥\par
 त्यक्तसर्वपरिग्रहस्य यतेः अन्नादेः शरीरस्थितिहेतोः परिग्रहस्य अभावात् याचनादिना शरीरस्थितौ कर्तव्यतायां प्राप्तायाम् ‘अयाचितमसङ्क्लृप्तमुपपन्नं यदृच्छया’\footnote{अश्व. ४६~। १९} इत्यादिना वचनेन अनुज्ञातं यतेः शरीरस्थितिहेतोः अन्नादेः प्राप्तिद्वारम् आविष्कुर्वन् आह —} 
\begin{center}{\bfseries यदृच्छालाभसन्तुष्टो द्वन्द्वातीतो विमत्सरः~।\\समः सिद्धावसिद्धौ च कृत्वापि न निबध्यते~॥~२२~॥}\end{center} 
यदृच्छालाभसन्तुष्टः अप्रार्थितोपनतो लाभो यदृच्छालाभः तेन सन्तुष्टः सञ्जातालंप्रत्ययः~। द्वन्द्वातीतः द्वन्द्वैः शीतोष्णादिभिः हन्यमानोऽपि अविषण्णचित्तः द्वन्द्वातीतः उच्यते~। विमत्सरः विगतमत्सरः निर्वैरबुद्दिः समः तुल्यः यदृच्छालाभस्य सिद्धौ असिद्धौ च~। यः एवंभूतो यतिः अन्नादेः शरीरस्थितिहेतोः लाभालाभयोः समः हर्षविषादवर्जितः कर्मादौ अकर्मादिदर्शी यथाभूतात्मदर्शननिष्ठः सन् शरीरस्थितिमात्रप्रयोजने भिक्षाटनादिकर्मणि शरीरादिनिर्वर्त्ये ‘नैव किञ्चित् करोम्यहम्’\footnote{भ. गी. ५~। ८}, ‘गुणा गुणेषु वर्तन्ते’\footnote{भ. गी. ३~। २८} इत्येवं सदा सम्परिचक्षाणः आत्मनः कर्तृत्वाभावं पश्यन्नैव किञ्चित् भिक्षाटनादिकं कर्म करोति, लोकव्यवहारसामान्यदर्शनेन तु लौकिकैः आरोपितकर्तृत्वे भिक्षाटनादौ कर्मणि कर्ता भवति~। स्वानुभवेन तु शास्त्रप्रमाणादिजनितेन अकर्तैव~। स एवं पराध्यारोपितकर्तृत्वः शरीरस्थितिमात्रप्रयोजनं भिक्षाटनादिकं कर्म कृत्वापि न निबध्यते बन्धहेतोः कर्मणः सहेतुकस्य ज्ञानाग्निना दग्धत्वात् इति उक्तानुवाद एव एषः~॥~२२~॥\par
 ‘त्यक्त्वा कर्मफलासङ्गम्’\footnote{भ. गी. ४~। २०} इत्यनेन श्लोकेन यः प्रारब्धकर्मा सन् यदा निष्क्रियब्रह्मात्मदर्शनसम्पन्नः स्यात् तदा तस्य आत्मनः कर्तृकर्मप्रयोजनाभावदर्शिनः कर्मपरित्यागे प्राप्ते कुतश्चिन्निमित्तात् तदसम्भवे सति पूर्ववत् तस्मिन् कर्मणि अभिप्रवृत्तस्य अपि ‘नैव किञ्चित् करोति सः’\footnote{भ. गी. ४~। २०} इति कर्माभावः प्रदर्शितः~। यस्य एवं कर्माभावो दर्शितः तस्यैव —} 
\begin{center}{\bfseries गतसङ्गस्य मुक्तस्य ज्ञानावस्थितचेतसः~।\\यज्ञायाचरतः कर्म समग्रं प्रविलीयते~॥~२३~॥}\end{center} 
गतसङ्गस्य सर्वतोनिवृत्तासक्तेः, मुक्तस्य निवृत्तधर्माधर्मादिबन्धनस्य, ज्ञानावस्थितचेतसः ज्ञाने एव अवस्थितं चेतः यस्य सोऽयं ज्ञानावस्थितचेताः तस्य, यज्ञाय यज्ञनिर्वृत्त्यर्थम् आचरतः निर्वर्तयतः कर्म समग्रं सह अग्रेण फलेन वर्तते इति समग्रं कर्म तत् समग्रं प्रविलीयते विनश्यति इत्यर्थः~॥~२३~॥\par
 कस्मात् पुनः कारणात् क्रियमाणं कर्म स्वकार्यारम्भम् अकुर्वत् समग्रं प्रविलीयते इत्युच्यते यतः —} 
\begin{center}{\bfseries ब्रह्मार्पणं ब्रह्म हविर्ब्रह्माग्नौ ब्रह्मणा हुतम्~।\\ब्रह्मैव तेन गन्तव्यं ब्रह्मकर्मसमाधिना~॥~२४~॥}\end{center} 
ब्रह्म अर्पणं येन करणेन ब्रह्मवित् हविः अग्नौ अर्पयति तत् ब्रह्मैव इति पश्यति, तस्य आत्मव्यतिरेकेण अभावं पश्यति, यथा शुक्तिकायां रजताभावं पश्यति~; तदुच्यते ब्रह्मैव अर्पणमिति, यथा यद्रजतं तत् शुक्तिकैवेति~। ‘ब्रह्म अर्पणम्’ इति असमस्ते पदे~। यत् अर्पणबुद्ध्या गृह्यते लोके तत् अस्य ब्रह्मविदः ब्रह्मैव इत्यर्थः~। ब्रह्म हविः तथा यत् हविर्बुद्ध्या गृह्यमाणं तत् ब्रह्मैव अस्य~। तथा ‘ब्रह्माग्नौ’ इति समस्तं पदम्~। अग्निरपि ब्रह्मैव यत्र हूयते ब्रह्मणा कर्त्रा, ब्रह्मैव कर्तेत्यर्थः~। यत् तेन हुतं हवनक्रिया तत् ब्रह्मैव~। यत् तेन गन्तव्यं फलं तदपि ब्रह्मैव ब्रह्मकर्मसमाधिना ब्रह्मैव कर्म ब्रह्मकर्म तस्मिन् समाधिः यस्य सः ब्रह्मकर्मसमाधिः तेन ब्रह्मकर्मसमाधिना ब्रह्मैव गन्तव्यम्~॥~} 
एवं लोकसङ्ग्रहं चिकीर्षुणापि क्रियमाणं कर्म परमार्थतः अकर्म, ब्रह्मबुद्ध्युपमृदितत्वात्~। एवं सति निवृत्तकर्मणोऽपि सर्वकर्मसंन्यासिनः सम्यग्दर्शनस्तुत्यर्थं यज्ञत्वसम्पादनं ज्ञानस्य सुतरामुपपद्यते~; यत् अर्पणादि अधियज्ञे प्रसिद्धं तत् अस्य अध्यात्मं ब्रह्मैव परमार्थदर्शिन इति~। अन्यथा सर्वस्य ब्रह्मत्वे अर्पणादीनामेव विशेषतो ब्रह्मत्वाभिधानम् अनर्थकं स्यात्~। तस्मात् ब्रह्मैव इदं सर्वमिति अभिजानतः विदुषः कर्माभावः~। कारकबुद्ध्यभावाच्च~। न हि कारकबुद्धिरहितं यज्ञाख्यं कर्म दृष्टम्~। सर्वमेव अग्निहोत्रादिकं कर्म शब्दसमर्पितदेवताविशेषसम्प्रदानादिकारकबुद्धिमत् कर्त्रभिमानफलाभिसन्धिमच्च दृष्टम्~; न उपमृदितक्रियाकारकफलभेदबुद्धिमत् कर्तृत्वाभिमानफलाभिसन्धिरहितं वा~। इदं तु ब्रह्मबुद्ध्युपमृदितार्पणादिकारकक्रियाफलभेदबुद्धि कर्म~। अतः अकर्मैव तत्~। तथा च दर्शितम् ‘कर्मण्यकर्म यः पश्येत्’\footnote{भ. गी. ४~। १८} ‘कर्मण्यभिप्रवृत्तोऽपि नैव किञ्चित्करोति सः’\footnote{भ. गी. ४~। २०} ‘गुणा गुणेषु वर्तन्ते’\footnote{भ. गी. ३~। २८} ‘नैव किञ्चित्करोमीति युक्तो मन्येत तत्त्ववित्’\footnote{भ. गी. ५~। ८} इत्यादिभिः~। तथा च दर्शयन् तत्र तत्र क्रियाकारकफलभेदबुद्ध्युपमर्दं करोति~। दृष्टा च काम्याग्निहोत्रादौ कामोपमर्देन काम्याग्निहोत्रादिहानिः~। तथा मतिपूर्वकामतिपूर्वकादीनां कर्मणां कार्यविशेषस्य आरम्भकत्वं दृष्टम्~। तथा इहापि ब्रह्मबुद्ध्युपमृदितार्पणादिकारकक्रियाफलभेदबुद्धेः बाह्यचेष्टामात्रेण कर्मापि विदुषः अकर्म सम्पद्यते~। अतः उक्तम् ‘समग्रं प्रविलीयते’\footnote{भ. गी. ४~। २०} इति~॥~} 
अत्र केचिदाहुः — यत् ब्रह्म तत् अर्पणादीनि~; ब्रह्मैव किल अर्पणादिना पञ्चविधेन कारकात्मना व्यवस्थितं सत् तदेव कर्म करोति~। तत्र न अर्पणादिबुद्धिः निवर्त्यते, किं तु अर्पणादिषु ब्रह्मबुद्धिः आधीयते~; यथा प्रतिमादौ विष्ण्वादिबुद्धिः, यथा वा नामादौ ब्रह्मबुद्धिरिति~॥~} 
सत्यम्~, एवमपि स्यात् यदि ज्ञानयज्ञस्तुत्यर्थं प्रकरणं न स्यात्~। अत्र तु सम्यग्दर्शनं ज्ञानयज्ञशब्दितम् अनेकान् यज्ञशब्दितान् क्रियाविशेषान् उपन्यस्य ‘श्रेयान् द्रव्यमयाद्यज्ञात् ज्ञानयज्ञः’\footnote{भ. गी. ४~। ३३} इति ज्ञानं स्तौति~। अत्र च समर्थमिदं वचनम् ‘ब्रह्मार्पणम्’ इत्यादि ज्ञानस्य यज्ञत्वसम्पादने~; अन्यथा सर्वस्य ब्रह्मत्वे अर्पणादीनामेव विशेषतो ब्रह्मत्वाभिधानमनर्थकं स्यात्~। ये तु अर्पणादिषु प्रतिमायां विष्णुदृष्टिवत् ब्रह्मदृष्टिः क्षिप्यते नामादिष्विव चेति ब्रुवते न तेषां ब्रह्मविद्या उक्ता इह विवक्षिता स्यात्~, अर्पणादिविषयत्वात् ज्ञानस्य~। न च दृष्टिसम्पादनज्ञानेन मोक्षफलं प्राप्यते~। ‘ब्रह्मैव तेन गन्तव्यम्’ इति चोच्यते~। विरुद्धं च सम्यग्दर्शनम् अन्तरेण मोक्षफलं प्राप्यते इति~। प्रकृतविरोधश्च~; सम्यग्दर्शनम् च प्रकृतम् ‘कर्मण्यकर्म यः पश्येत्’\footnote{भ. गी. ४~। १८} इत्यत्र, अन्ते च सम्यग्दर्शनम्~, तस्यैव उपसंहारात्~। ‘श्रेयान् द्रव्यमयाद्यज्ञात् ज्ञानयज्ञः’\footnote{भ. गी. ४~। ३३}, ‘ज्ञानं लब्ध्वा परां शान्तिम्’\footnote{भ. गी. ४~। ३९} इत्यादिना सम्यग्दर्शनस्तुतिमेव कुर्वन् उपक्षीणः अध्यायः~। तत्र अकस्मात् अर्पणादौ ब्रह्मदृष्टिः अप्रकरणे प्रतिमायामिव विष्णुदृष्टिः उच्यते इति अनुपपन्नम् | तस्मात् यथाव्याख्यातार्थ एव अयं श्लोकः~॥~२४~॥\par
 तत्र अधुना सम्यग्दर्शनस्य यज्ञत्वं सम्पाद्य तत्स्तुत्यर्थम् अन्येऽपि यज्ञा उपक्षिप्यन्ते —} 
\begin{center}{\bfseries दैवमेवापरे यज्ञं योगिनः पर्युपासते~।\\ब्रह्माग्नावपरे यज्ञं यज्ञेनैवोपजुह्वति~॥~२५~॥}\end{center} 
दैवमेव देवा इज्यन्ते येन यज्ञेन असौ दैवो यज्ञः तमेव अपरे यज्ञं योगिनः कर्मिणः पर्युपासते कुर्वन्तीत्यर्थः~। ब्रह्माग्नौ ‘सत्यं ज्ञानमनन्तं ब्रह्म’\footnote{तै. उ. २~। १~। १} ‘विज्ञानमानन्दं ब्रह्म’ ‘यत् साक्षादपरोक्षात् ब्रह्म य आत्मा सर्वान्तरः’\footnote{बृ. उ. ३~। ४~। १} इत्यादिवचनोक्तम् अशनायादिसर्वसंसारधर्मवर्जितम् ‘नेति नेति’\footnote{बृ. उ. ४~। ४~। २२} इति निरस्ताशेषविशेषं ब्रह्मशब्देन उच्यते~। ब्रह्म च तत् अग्निश्च सः होमाधिकरणत्वविवक्षया ब्रह्माग्निः~। तस्मिन् ब्रह्माग्नौ अपरे अन्ये ब्रह्मविदः यज्ञम् — यज्ञशब्दवाच्य आत्मा, आत्मनामसु यज्ञशब्दस्य पाठात् — तम् आत्मानं यज्ञं परमार्थतः परमेव ब्रह्म सन्तं बुद्ध्याद्युपाधिसंयुक्तम् अध्यस्तसर्वोपाधिधर्मकम् आहुतिरूपं यज्ञेनैव आत्मनैव उक्तलक्षणेन उपजुह्वति प्रक्षिपन्ति, सोपाधिकस्य आत्मनः निरुपाधिकेन परब्रह्मस्वरूपेणैव यद्दर्शनं स तस्मिन् होमः तं कुर्वन्ति ब्रह्मात्मैकत्वदर्शननिष्ठाः संन्यासिनः इत्यर्थः~॥~२५~॥\par
 सोऽयं सम्यग्दर्शनलक्षणः यज्ञः दैवयज्ञादिषु यज्ञेषु उपक्षिप्यते ‘ब्रह्मार्पणम्’ इत्यादिश्लोकैः प्रस्तुतः ‘श्रेयान् द्रव्यमयाद्यज्ञात् ज्ञानयज्ञः परन्तप’\footnote{भ. गी. ४~। ३३} इत्यादिना स्तुत्यर्थम् —} 
\begin{center}{\bfseries श्रोत्रादीनीन्द्रियाण्यन्ये संयमाग्निषु जुह्वति~।\\शब्दादीन्विषयानन्य इन्द्रियाग्निषु जुह्वति~॥~२६~॥}\end{center} 
श्रोत्रादीनि इन्द्रियाणि अन्ये योगिनः संयमाग्निषु~। प्रतीन्द्रियं संयमो भिद्यते इति बहुवचनम्~। संयमा एव अग्नयः तेषु जुह्वति इन्द्रियसंयममेव कुर्वन्ति इत्यर्थः~। शब्दादीन् विषयान् अन्ये इन्द्रियाग्निषु इन्द्रियाण्येव अग्नयः तेषु इन्द्रियाग्निषु जुह्वति श्रोत्रादिभिरविरुद्धविषयग्रहणं होमं मन्यन्ते~॥~२६~॥\par
 किञ्च —} 
\begin{center}{\bfseries सर्वाणीन्द्रियकर्माणि प्राणकर्माणि चापरे~।\\आत्मसंयमयोगाग्नौ जुह्वति ज्ञानदीपिते~॥~२७~॥}\end{center} 
सर्वाणि इन्द्रियकर्माणि इन्द्रियाणां कर्माणि इन्द्रियकर्माणि, तथा प्राणकर्माणि प्राणो वायुः आध्यात्मिकः तत्कर्माणि आकुञ्चनप्रसारणादीनि तानि च अपरे आत्मसंयमयोगाग्नौ आत्मनि संयमः आत्मसंयमः स एव योगाग्निः तस्मिन् आत्मसंयमयोगाग्नौ जुह्वति प्रक्षिपन्ति ज्ञानदीपिते स्नेहेनेव प्रदीपे विवेकविज्ञानेन उज्ज्वलभावम् आपादिते जुह्वति प्रविलापयन्ति इत्यर्थः~॥~२७~॥\par
 \begin{center}{\bfseries द्रव्ययज्ञास्तपोयज्ञा योगयज्ञास्तथापरे~।\\स्वाध्यायज्ञानयज्ञाश्च यतयः संशितव्रताः~॥~२८~॥}\end{center} 
द्रव्ययज्ञाः तीर्थेषु द्रव्यविनियोगं यज्ञबुद्ध्या कुर्वन्ति ये ते द्रव्ययज्ञाः~। तपोयज्ञाः तपः यज्ञः येषां तपस्विनां ते तपोयज्ञाः~। योगयज्ञाः प्राणायामप्रत्याहारादिलक्षणो योगो यज्ञो येषां ते योगयज्ञाः~। तथा अपरे स्वाध्यायज्ञानयज्ञाश्च स्वाध्यायः यथाविधि ऋगाद्यभ्यासः यज्ञः येषां ते स्वाध्याययज्ञाः~। ज्ञानयज्ञाः ज्ञानं शास्त्रार्थपरिज्ञानं यज्ञः येषां ते ज्ञानयज्ञाश्च यतयः यतनशीलाः संशितव्रताः सम्यक् शितानि तनूकृतानि तीक्ष्णीकृतानि व्रतानि येषां ते संशितव्रताः~॥~२८~॥\par
 किञ्च —} 
\begin{center}{\bfseries अपाने जुह्वति प्राणं प्राणेऽपानं तथापरे~।\\प्राणापानगती रुद्ध्वा प्राणायामपरायणाः~॥~२९~॥}\end{center} 
अपाने अपानवृत्तौ जुह्वति प्रक्षिपन्ति प्राणं प्राणवृत्तिम्~, पूरकाख्यं प्राणायामं कुर्वन्तीत्यर्थः~। प्राणे अपानं तथा अपरे जुह्वति, रेचकाख्यं च प्राणायामं कुर्वन्तीत्येतत्~। प्राणापानगती मुखनासिकाभ्यां वायोः निर्गमनं प्राणस्य गतिः, तद्विपर्ययेण अधोगमनम् अपानस्य गतिः, ते प्राणापानगती एते रुद्ध्वा निरुध्य प्राणायामपरायणाः प्राणायामतत्पराः~; कुम्भकाख्यं प्राणायामं कुर्वन्तीत्यर्थः~॥~२९~॥\par
 किञ्च —} 
\begin{center}{\bfseries अपरे नियताहाराः प्राणान्प्राणेषु जुह्वति~।\\सर्वेऽप्येते यज्ञविदो यज्ञक्षपितकल्मषाः~॥~३०~॥}\end{center} 
अपरे नियताहाराः नियतः परिमितः आहारः येषां ते नियताहाराः सन्तः प्राणान् वायुभेदान् प्राणेषु एव जुह्वति यस्य यस्य वायोः जयः क्रियते इतरान् वायुभेदान् तस्मिन् तस्मिन् जुह्वति, ते तत्र प्रविष्टा इव भवन्ति~। सर्वेऽपि एते यज्ञविदः यज्ञक्षपितकल्मषाः यज्ञैः यथोक्तैः क्षपितः नाशितः कल्मषो येषां ते यज्ञक्षपितकल्मषाः~॥~३०~॥\par
 एवं यथोक्तान् यज्ञान् निर्वर्त्य —} 
\begin{center}{\bfseries यज्ञशिष्टामृतभुजो यान्ति ब्रह्म सनातनम्~।\\नायं लोकोऽस्त्ययज्ञस्य कुतोऽन्यः कुरुसत्तम~॥~३१~॥}\end{center} 
यज्ञशिष्टामृतभुजः यज्ञानां शिष्टं यज्ञशिष्टं यज्ञशिष्टं च तत् अमृतं च यज्ञशिष्टामृतं तत् भुञ्जते इति यज्ञशिष्टामृतभुजः~। यथोक्तान् यज्ञान् कृत्वा तच्छिष्टेन कालेन यथाविधिचोदितम् अन्नम् अमृताख्यं भुञ्जते इति यज्ञशिष्टामृतभुजः यान्ति गच्छन्ति ब्रह्म सनातनं चिरन्तनं मुमुक्षवश्चेत्~; कालातिक्रमापेक्षया इति सामर्थ्यात् गम्यते~। न अयं लोकः सर्वप्राणिसाधारणोऽपि अस्ति यथोक्तानां यज्ञानां एकोऽपि यज्ञः यस्य नास्ति सः अयज्ञः तस्य~। कुतः अन्यो विशिष्टसाधनसाध्यः कुरुसत्तम~॥~३१~॥\par
 \begin{center}{\bfseries एवं बहुविधा यज्ञा वितता ब्रह्मणो मुखे~।\\कर्मजान्विद्धि तान्सर्वानेवं ज्ञात्वा विमोक्ष्यसे~॥~३२~॥}\end{center} 
एवं यथोक्ता बहुविधा बहुप्रकारा यज्ञाः वितताः विस्तीर्णाः ब्रह्मणो वेदस्य मुखे द्वारे वेदद्वारेण अवगम्यमानाः ब्रह्मणो मुखे वितता उच्यन्ते~; तद्यथा ‘वाचि हि प्राणं जुहुमः’\footnote{ऐ. आ. ३~। २~। ६} इत्यादयः~। कर्मजान् कायिकवाचिकमानसकर्मोद्भावान् विद्धि तान् सर्वान् अनात्मजान्~, निर्व्यापारो हि आत्मा~। अत एवं ज्ञात्वा विमोक्ष्यसे अशुभात्~। न मद्व्यापारा इमे, निर्व्यापारोऽहम् उदासीन इत्येवं ज्ञात्वा अस्मात् सम्यग्दर्शनात् मोक्ष्यसे संसारबन्धनात् इत्यर्थः~॥~३२~॥\par
 ‘ब्रह्मार्पणम्’\footnote{भ. गी. ४~। २४} इत्यादिश्लोकेन सम्यग्दर्शनस्य यज्ञत्वं सम्पादितम्~। यज्ञाश्च अनेके उपदिष्टाः~। तैः सिद्धपुरुषार्थप्रयोजनैः ज्ञानं स्तूयते~। कथम्~? —} 
\begin{center}{\bfseries श्रेयान्द्रव्यमयाद्यज्ञाज्ज्ञानयज्ञः परन्तप~।\\सर्वं कर्माखिलं पार्थ ज्ञाने परिसमाप्यते~॥~३३~॥}\end{center} 
श्रेयान् द्रव्यमयात् द्रव्यसाधनसाध्यात् यज्ञात् ज्ञानयज्ञः हे परन्तप~। द्रव्यमयो हि यज्ञः फलस्यारम्भकः, ज्ञानयज्ञः न फलारम्भकः, अतः श्रेयान् प्रशस्यतरः~। कथम्~? यतः सर्वं कर्म समस्तम् अखिलम् अप्रतिबद्धं पार्थ ज्ञाने मोक्षसाधने सर्वतःसम्प्लुतोदकस्थानीये परिसमाप्यते अन्तर्भवतीत्यर्थः ‘यथा कृताय विजितायाधरेयाः संयन्त्येवमेवं सर्वं तदभिसमेति यत् किञ्चित्प्रजाः साधु कुर्वन्ति यस्तद्वेद यत्स वेद’\footnote{छा. उ. ४~। १~। ४} इति श्रुतेः~॥~३३~॥\par
 तदेतत् विशिष्टं ज्ञानं तर्हि केन प्राप्यते इत्युच्यते —} 
\begin{center}{\bfseries तद्विद्धि प्रणिपातेन परिप्रश्नेन सेवया~।\\उपदेक्ष्यन्ति ते ज्ञानं ज्ञानिनस्तत्त्वदर्शिनः~॥~३४~॥}\end{center} 
तत् विद्धि विजानीहि येन विधिना प्राप्यते इति~। आचार्यान् अभिगम्य, प्रणिपातेन प्रकर्षेण नीचैः पतनं प्रणिपातः दीर्घनमस्कारः तेन, ‘कथं बन्धः~? कथं मोक्षः~? का विद्या~? का चाविद्या~? ’ इति परिप्रश्नेन, सेवया गुरुशुश्रूषया एवमादिना~। प्रश्रयेण आवर्जिता आचार्या उपदेक्ष्यन्ति कथयिष्यन्ति ते ज्ञानं यथोक्तविशेषणं ज्ञानिनः~। ज्ञानवन्तोऽपि केचित् यथावत् तत्त्वदर्शनशीलाः, अपरे न~; अतो विशिनष्टि तत्त्वदर्शिनः इति~। ये सम्यग्दर्शिनः तैः उपदिष्टं ज्ञानं कार्यक्षमं भवति नेतरत् इति भगवतो मतम्~॥~३४~॥\par
 तथा च सति इदमपि समर्थं वचनम् —} 
\begin{center}{\bfseries यज्ज्ञात्वा न पुनर्मोहमेवं यास्यसि पाण्डव~।\\येन भूतान्यशेषेण द्रक्ष्यस्यात्मन्यथो मयि~॥~३५~॥}\end{center} 
यत् ज्ञात्वा यत् ज्ञानं तैः उपदिष्टं अधिगम्य प्राप्य पुनः भूयः मोहम् एवं यथा इदानीं मोहं गतोऽसि पुनः एवं न यास्यसि हे पाण्डव~। किञ्च — येन ज्ञानेन भूतानि अशेषेण ब्रह्मादीनि स्तम्बपर्यन्तानि द्रक्ष्यसि साक्षात् आत्मनि प्रत्यगात्मनि ‘मत्संस्थानि इमानि भूतानि’ इति अथो अपि मयि वासुदेवे ‘परमेश्वरे च इमानि’ इति~; क्षेत्रज्ञेश्वरैकत्वं सर्वोपनिषत्प्रसिद्धं द्रक्ष्यसि इत्यर्थः~॥~३५~॥\par
 किञ्च एतस्य ज्ञानस्य माहात्म्यम् —} 
\begin{center}{\bfseries अपि चेदसि पापेभ्यः सर्वेभ्यः पापकृत्तमः~।\\सर्वं ज्ञानप्लवेनैव वृजिनं सन्तरिष्यसि~॥~३६~॥}\end{center} 
अपि चेत् असि पापेभ्यः पापकृद्भ्यः सर्वेभ्यः अतिशयेन पापकृत् पापकृत्तमः सर्वं ज्ञानप्लवेनैव ज्ञानमेव प्लवं कृत्वा वृजिनं वृजिनार्णवं पापसमुद्रं सन्तरिष्यसि~। धर्मोऽपि इह मुमुक्षोः पापम् उच्यते~॥~३६~॥\par
 ज्ञानं कथं नाशयति पापमिति दृष्टान्त उच्यते —} 
\begin{center}{\bfseries यथैधांसि समिद्धोऽग्निर्भस्मसात्कुरुतेऽर्जुन~।\\ज्ञानाग्निः सर्वकर्माणि भस्मसात्कुरुते तथा~॥~३७~॥}\end{center} 
यथा एधांसि काष्ठानि समिद्धः सम्यक् इद्धः दीप्तः अग्निः भस्मसात् भस्मीभावं कुरुते हे अर्जुन, ज्ञानमेव अग्निः ज्ञानाग्निः सर्वकर्माणि भस्मसात् कुरुते तथा निर्बीजीकरोतीत्यर्थः~। न हि साक्षादेव ज्ञानाग्निः कर्माणि इन्धनवत् भस्मीकर्तुं शक्नोति~। तस्मात् सम्यग्दर्शनं सर्वकर्मणां निर्बीजत्वे कारणम् इत्यभिप्रायः~। सामर्थ्यात् येन कर्मणा शरीरम् आरब्धं तत् प्रवृत्तफलत्वात् उपभोगेनैव क्षीयते~।  अतो यानि अप्रवृत्तफलानि ज्ञानोत्पत्तेः प्राक् कृतानि ज्ञानसहभावीनि च अतीतानेकजन्मकृतानि च तान्येव सर्वाणि भस्मसात् कुरुते~॥~३७~॥\par
 यतः एवम् अतः —} 
\begin{center}{\bfseries न हि ज्ञानेन सदृशं पवित्रमिह विद्यते~।\\तत्स्वयं योगसंसिद्धः कालेनात्मनि विन्दति~॥~३८~॥}\end{center} 
न हि ज्ञानेन सदृशं तुल्यं पवित्रं पावनं शुद्धिकरम् इह विद्यते~। तत् ज्ञानं स्वयमेव योगसंसिद्धः योगेन कर्मयोगेन समाधियोगेन च संसिद्धः संस्कृतः योग्यताम् आपन्नः सन् मुमुक्षुः कालेन महता आत्मनि विन्दति लभते इत्यर्थः~॥~३८~॥\par
 येन एकान्तेन ज्ञानप्राप्तिः भवति स उपायः उपदिश्यते —} 
\begin{center}{\bfseries श्रद्धावांल्लभते ज्ञानं तत्परः संयतेन्द्रियः~।\\ज्ञानं लब्ध्वा परां शान्तिमचिरेणाधिगच्छति~॥~३९~॥}\end{center} 
श्रद्धावान् श्रद्धालुः लभते ज्ञानम्~। श्रद्धालुत्वेऽपि भवति कश्चित् मन्दप्रस्थानः, अत आह — तत्परः, गुरूपसदनादौ अभियुक्तः ज्ञानलब्ध्युपाये श्रद्धावान्~। तत्परः अपि अजितेन्द्रियः स्यात् इत्यतः आह — संयतेन्द्रियः, संयतानि विषयेभ्यो निवर्तितानि यस्य इन्द्रियाणि स संयतेन्द्रियः~। य एवंभूतः श्रद्धावान् तत्परः संयतेन्द्रियश्च सः अवश्यं ज्ञानं लभते~। प्रणिपातादिस्तु बाह्योऽनैकान्तिकोऽपि भवति, मायावित्वादिसम्भवात्~; न तु तत् श्रद्धावत्त्वादौ इत्येकान्ततः ज्ञानलब्ध्युपायः~। किं पुनः ज्ञानलाभात् स्यात् इत्युच्यते — ज्ञानं लब्ध्वा परां मोक्षाख्यां शान्तिम् उपरतिम् अचिरेण क्षिप्रमेव अधिगच्छति~। सम्यग्दर्शनात् क्षिप्रमेव मोक्षो भवतीति सर्वशास्त्रन्यायप्रसिद्धः सुनिश्चितः अर्थः~॥~३९~॥\par
 अत्र संशयः न कर्तव्यः, पापिष्ठो हि संशयः~; कथम् इति उच्यते —} 
\begin{center}{\bfseries अज्ञश्चाश्रद्दधानश्च संशयात्मा विनश्यति~।\\नायं लोकोऽस्ति न परो न सुखं संशयात्मनः~॥~४०~॥}\end{center} 
अज्ञश्च अनात्मज्ञश्च अश्रद्दधानश्च गुरुवाक्यशास्त्रेषु अविश्वासवांश्च संशयात्मा च संशयचित्तश्च विनश्यति~। अज्ञाश्रद्दधानौ यद्यपि विनश्यतः, न तथा यथा संशयात्मा~। संशयात्मा तु पापिष्ठः सर्वेषाम्~। कथम्~? नायं साधारणोऽपि लोकोऽस्ति~। तथा न परः लोकः~। न सुखम्~, तत्रापि संशयोत्पत्तेः संशयात्मनः संशयचित्तस्य~। तस्मात् संशयो न कर्तव्यः~॥~४०~॥\par
 कस्मात्~? —} 
\begin{center}{\bfseries योगसंन्यस्तकर्माणं ज्ञानसञ्छिन्नसंशयम्~।\\आत्मवन्तं न कर्माणि निबध्नन्ति धनञ्जय~॥~४१~॥}\end{center} 
योगसंन्यस्तकर्माणं परमार्थदर्शनलक्षणेन योगेन संन्यस्तानि कर्माणि येन परमार्थदर्शिना धर्माधर्माख्यानि तं योगसंन्यस्तकर्माणम्~। कथं योगसंन्यस्तकर्मेत्याह — ज्ञानसञ्छिन्नसंशयं ज्ञानेन आत्मेश्वरैकत्वदर्शनलक्षणेन सञ्छिन्नः संशयो यस्य सः ज्ञानसञ्छिन्नसंशयः~। य एवं योगसंन्यस्तकर्मा तम् आत्मवन्तम् अप्रमत्तं गुणचेष्टारूपेण दृष्टानि कर्माणि न निबध्नन्ति अनिष्टादिरूपं फलं नारभन्ते हे धनञ्जय~॥~४१~॥\par
 यस्मात् कर्मयोगानुष्ठानात् अशुद्धिक्षयहेतुकज्ञानसञ्छिन्नसंशयः न निबध्यते कर्मभिः ज्ञानाग्निदग्धकर्मत्वादेव, यस्माच्च ज्ञानकर्मानुष्ठानविषये संशयवान् विनश्यति —} 
\begin{center}{\bfseries तस्मादज्ञानसम्भूतं हृत्स्थं ज्ञानासिनात्मनः~।\\छित्त्वैनं संशयं योगमातिष्ठोत्तिष्ठ भारत~॥~४२~॥}\end{center} 
तस्मात् पापिष्ठम् अज्ञानसम्भूतम् अज्ञानात् अविवेकात् जातं हृत्स्थं हृदि बुद्धौ स्थितं ज्ञानासिना शोकमोहादिदोषहरं सम्यग्दर्शनं ज्ञानं तदेव असिः खङ्गः तेन ज्ञानासिना आत्मनः स्वस्य, आत्मविषयत्वात् संशयस्य~। न हि परस्य संशयः परेण च्छेत्तव्यतां प्राप्तः, येन स्वस्येति विशेष्येत~। अतः आत्मविषयोऽपि स्वस्यैव भवति~। छित्त्वा एनं संशयं स्वविनाशहेतुभूतम्~, योगं सम्यग्दर्शनोपायं कर्मानुष्ठानम् आतिष्ठ कुर्वित्यर्थः~। उत्तिष्ठ च इदानीं युद्धाय भारत इति~॥~४२~॥\par
 
इति श्रीमत्परमहंसपरिव्राजकाचार्यस्य श्रीगोविन्दभगवत्पूज्यपादशिष्यस्य श्रीमच्छङ्करभगवतः कृतौ श्रीमद्भगवद्गीताभाष्ये चतुर्थोऽध्यायः~॥\par
 
‘कर्मण्यकर्म यः पश्येत्’\footnote{भ. गी. ४~। १८} इत्यारभ्य ‘स युक्तः कृत्स्नकर्मकृत्’\footnote{भ. गी. ४~। १८} ‘ज्ञानाग्निदग्धकर्माणम्’\footnote{भ. गी. ४~। १९} ‘शारीरं केवलं कर्म कुर्वन्’\footnote{भ. गी. ४~। २१} ‘यदृच्छालाभसन्तुष्टः’\footnote{भ. गी. ४~। २२} ‘ब्रह्मार्पणं ब्रह्म हविः’\footnote{भ. गी. ४~। २४} ‘कर्मजान् विद्धि तान् सर्वान्’\footnote{भ. गी. ४~। ३२} ‘सर्वं कर्माखिलं पार्थ’\footnote{भ. गी. ४~। ३३} ‘ज्ञानाग्निः सर्वकर्माणि’\footnote{भ. गी. ४~। ३७} ‘योगसंन्यस्तकर्माणम्’\footnote{भ. गी. ४~। ४१} इत्येतैः वचनैः सर्वकर्मसंन्यासम् अवोचत् भगवान्~। ‘छित्त्वैनं संशयं योगमातिष्ठ’\footnote{भ. गी. ४~। ४२} इत्यनेन वचनेन योगं च कर्मानुष्ठानलक्षणम् अनुतिष्ठ इत्युक्तवान्~। तयोरुभयोश्च कर्मानुष्ठानकर्मसंन्यासयोः स्थितिगतिवत् परस्परविरोधात् एकेन सह कर्तुमशक्यत्वात्~, कालभेदेन च अनुष्ठानविधानाभावात्~, अर्थात् एतयोः अन्यतरकर्तव्यताप्राप्तौ सत्यां यत् प्रशस्यतरम् एतयोः कर्मानुष्ठानकर्मसंन्यासयोः तत् कर्तव्यं न इतरत् इत्येवं मन्यमानः प्रशस्यतरबुभुत्सया अर्जुन उवाच — ‘संन्यासं कर्मणां कृष्ण’\footnote{भ. गी. ५~। १} इत्यादिना~॥~} 
ननु च आत्मविदः ज्ञानयोगेन निष्ठां प्रतिपिपादयिषन् पूर्वोदाहृतैः वचनैः भगवान् सर्वकर्मसंन्यासम् अवोचत्~, न तु अनात्मज्ञस्य~। अतश्च कर्मानुष्ठानकर्मसंन्यासयोः भिन्नपुरुषविषयत्वात् अन्यतरस्य प्रशस्यतरत्वबुभुत्सया अयं प्रश्नः अनुपपन्नः~। सत्यमेव त्वदभिप्रायेण प्रश्नो न उपपद्यते~; प्रष्टुः स्वाभिप्रायेण पुनः प्रश्नः युज्यत एवेति वदामः~। कथम्~? पूर्वोदाहृतैः वचनैः भगवता कर्मसंन्यासस्य कर्तव्यतया विवक्षितत्वात्~, प्राधान्यमन्तरेण च कर्तारं तस्य कर्तव्यत्वासम्भवात् अनात्मविदपि कर्ता पक्षे प्राप्तः अनूद्यत एव~; न पुनः आत्मवित्कर्तृकत्वमेव संन्यासस्य विवक्षितम्~, इत्येवं मन्वानस्य अर्जुनस्य कर्मानुष्ठानकर्मसंन्यासयोः अविद्वत्पुरुषकर्तृकत्वमपि अस्तीति पूर्वोक्तेन प्रकारेण तयोः परस्परविरोधात् अन्यतरस्य कर्तव्यत्वे प्राप्ते प्रशस्यतरं च कर्तव्यम् न इतरत् इति प्रशस्यतरविविदिषया प्रश्नः न अनुपपन्नः~॥~} 
प्रतिवचनवाक्यार्थनिरूपणेनापि प्रष्टुः अभिप्रायः एवमेवेति गम्यते~। कथम्~? ‘संन्यासकर्मयोगौ निःश्रेयसकरौ तयोस्तु कर्मयोगो विशिष्यते’\footnote{भ. गी. ५~। २} इति प्रतिवचनम्~। एतत् निरूप्यम् — किं अनेन आत्मवित्कर्तृकयोः संन्यासकर्मयोगयोः निःश्रेयसकरत्वं प्रयोजनम् उक्त्वा तयोरेव कुतश्चित् विशेषात् कर्मसंन्यासात् कर्मयोगस्य विशिष्टत्वम् उच्यते~? आहोस्वित् अनात्मवित्कर्तृकयोः संन्यासकर्मयोगयोः तदुभयम् उच्यते~? इति~। किञ्चातः — यदि आत्मवित्कर्तृकयोः कर्मसंन्यासकर्मयोगयोः निःश्रेयसकरत्वम्~, तयोस्तु कर्मसंन्यासात् कर्मयोगस्य विशिष्टत्वम् उच्यते~; यदि वा अनात्मवित्कर्तृकयोः संन्यासकर्मयोगयोः तदुभयम् उच्यते इति~। अत्र उच्यते — आत्मवित्कर्तृकयोः संन्यासकर्मयोगयोः असम्भवात् तयोः निःश्रेयसकरत्ववचनं तदीयाच्च कर्मसंन्यासात् कर्मयोगस्य विशिष्टत्वाभिधानम् इत्येतत् उभयम् अनुपपन्नम्~। यदि अनात्मविदः कर्मसंन्यासः तत्प्रतिकूलश्च कर्मानुष्ठानलक्षणः कर्मयोगः सम्भवेताम्~, तदा तयोः निःश्रेयसकरत्वोक्तिः कर्मयोगस्य च कर्मसंन्यासात् विशिष्टत्वाभिधानम् इत्येतत् उभयम् उपपद्येत~। आत्मविदस्तु संन्यासकर्मयोगयोः असम्भवात् तयोः निःश्रेयसकरत्वाभिधानं कर्मसंन्यासाच्च कर्मयोगः विशिष्यते इति च अनुपपन्नम्~॥~} 
अत्र आह — किम् आत्मविदः संन्यासकर्मयोगयोः उभयोरपि असम्भवः~? आहोस्वित् अन्यतरस्य असम्भवः~? यदा च अन्यतरस्य असम्भवः, तदा किं कर्मसंन्यासस्य, उत कर्मयोगस्य~? इति~; असम्भवे कारणं च वक्तव्यम् इति~। अत्र उच्यते — आत्मविदः निवृत्तमिथ्याज्ञानत्वात् विपर्ययज्ञानमूलस्य कर्मयोगस्य असम्भवः स्यात्~। जन्मादिसर्वविक्रियारहितत्वेन निष्क्रियम् आत्मानम् आत्मत्वेन यो वेत्ति तस्य आत्मविदः सम्यग्दर्शनेन अपास्तमिथ्याज्ञानस्य निष्क्रियात्मस्वरूपावस्थानलक्षणं सर्वकर्मसंन्यासम् उक्त्वा तद्विपरीतस्य मिथ्याज्ञानमूलकर्तृत्वाभिमानपुरःसरस्य सक्रियात्मस्वरूपावस्थानरूपस्य कर्मयोगस्य इह गीताशास्त्रे तत्र तत्र आत्मस्वरूपनिरूपणप्रदेशेषु सम्यग्ज्ञानमिथ्याज्ञानतत्कार्यविरोधात् अभावः प्रतिपाद्यते यस्मात्~, तस्मात् आत्मविदः निवृत्तमिथ्याज्ञानस्य विपर्ययज्ञानमूलः कर्मयोगो न सम्भवतीति युक्तम् उक्तं स्यात्~॥~} 
केषु केषु पुनः आत्मस्वरूपनिरूपणप्रदेशेषु आत्मविदः कर्माभावः प्रतिपाद्यते इति अत्र उच्यते — ‘अविनाशि तु तत्’\footnote{भ. गी. २~। १७} इति प्रकृत्य ‘य एनं वेत्ति हन्तारम्’\footnote{भ. गी. २~। १९} ‘वेदाविनाशिनं नित्यम्’\footnote{भ. गी. २~। २१} इत्यादौ तत्र तत्र आत्मविदः कर्माभावः उच्यते~॥~} 
ननु च कर्मयोगोऽपि आत्मस्वरूपनिरूपणप्रदेशेषु तत्र तत्र प्रतिपाद्यते एव~; तद्यथा — ‘तस्माद्युध्यस्व भारत’\footnote{भ. गी. २~। १८} ‘स्वधर्ममपि चावेक्ष्य’\footnote{भ. गी. २~। ३१} ‘कर्मण्येवाधिकारस्ते’\footnote{भ. गी. २~। ४७} इत्यादौ~। अतश्च कथम् आत्मविदः कर्मयोगस्य असम्भवः स्यादिति~? अत्र उच्यते — सम्यग्ज्ञानमिथ्याज्ञानतत्कार्यविरोधात्~, ‘ज्ञानयोगेन साङ्ख्यानाम्’\footnote{भ. गी. ३~। ३} इत्यनेन साङ्ख्यानाम् आत्मतत्त्वविदाम् अनात्मवित्कर्तृककर्मयोगनिष्ठातः निष्क्रियात्मस्वरूपावस्थानलक्षणायाः ज्ञानयोगनिष्ठायाः पृथक्करणात्~, कृतकृत्यत्वेन आत्मविदः प्रयोजनान्तराभावात्~, ‘तस्य कार्यं न विद्यते’\footnote{भ. गी. ३~। १७} इति कर्तव्यान्तराभाववचनाच्च, ‘न कर्मणामनारम्भात्’\footnote{भ. गी. ३~। ४} ‘संन्यासस्तु महाबाहो दुःखमाप्तुमयोगतः’\footnote{भ. गी. ५~। ६} इत्यादिना च आत्मज्ञानाङ्गत्वेन कर्मयोगस्य विधानात्~, ‘योगारूढस्य तस्यैव शमः कारणमुच्यते’\footnote{भ. गी. ६~। ३} इत्यनेन च उत्पन्नसम्यग्दर्शनस्य कर्मयोगाभाववचनात्~, ‘शारीरं केवलं कर्म कुर्वन्नाप्नोति किल्बिषम्’\footnote{भ. गी. ४~। २१} इति च शरीरस्थितिकारणातिरिक्तस्य कर्मणो निवारणात्~, ‘नैव किञ्चित्करोमीति युक्तो मन्येत तत्त्ववित्’\footnote{भ. गी. ५~। ८} इत्यनेन च शरीरस्थितिमात्रप्रयुक्तेष्वपि दर्शनश्रवणादिकर्मसु आत्मयाथात्म्यविदः ‘करोमि’ इति प्रत्ययस्य समाहितचेतस्तया सदा अकर्तव्यत्वोपदेशात् आत्मतत्त्वविदः सम्यग्दर्शनविरुद्धो मिथ्याज्ञानहेतुकः कर्मयोगः स्वप्नेऽपि न सम्भावयितुं शक्यते यस्मात्~, तस्मात् अनात्मवित्कर्तृकयोरेव संन्यासकर्मयोगयोः निःश्रेयसकरत्ववचनम्~, तदीयाच्च कर्मसंन्यासात् पूर्वोक्तात्मवित्कर्तृकसर्वकर्मसंन्यासविलक्षणात् सत्येव कर्तृत्वविज्ञाने कर्मैकदेशविषयात् यमनियमादिसहितत्वेन च दुरनुष्ठेयात् सुकरत्वेन च कर्मयोगस्य विशिष्टत्वाभिधानम् इत्येवं प्रतिवचनवाक्यार्थनिरूपणेनापि पूर्वोक्तः प्रष्टुरभिप्रायः निश्चीयते इति स्थितम्~॥~} 
‘ज्यायसी चेत्कर्मणस्ते’\footnote{भ. गी. ३~। १} इत्यत्र ज्ञानकर्मणोः सह असम्भवे ‘यच्छ्रेय एतयोः तद्ब्रूहि’\footnote{भ. गी. ३~। २} इत्येवं पृष्टोऽर्जुनेन भगवान् साङ्‍ख्यानां संन्यासिनां ज्ञानयोगेन निष्ठा पुनः कर्मयोगेन योगिनां निष्ठा प्रोक्तेति निर्णयं चकार~। ‘न च संन्यसनादेव केवलात् सिद्धिं समधिगच्छति’\footnote{भ. गी. ३~। ४} इति वचनात् ज्ञानसहितस्य सिद्धिसाधनत्वम् इष्टम्’ कर्मयोगस्य च, विधानात्~। ज्ञानरहितस्य संन्यासः श्रेयान्~, किं वा कर्मयोगः श्रेयान्~? ’ इति एतयोः विशेषबुभुत्सया —} 
\begin{center}{\bfseries अर्जुन उवाच —\\ संन्यासं कर्मणां कृष्ण पुनर्योगं च शंससि~।\\यच्छ्रेय एतयोरेकं तन्मे ब्रूहि सुनिश्चितम्~॥~१~॥}\end{center} 
संन्यासं परित्यागं कर्मणां शास्त्रीयाणाम् अनुष्ठेयविशेषाणां शंससि प्रशंससि कथयसि इत्येतत्~। पुनः योगं च तेषामेव अनुष्ठानम् अवश्यकर्तव्यं शंससि~। अतः मे कतरत् श्रेयः इति संशयः — किं कर्मानुष्ठानं श्रेयः, किं वा तद्धानम् इति~। प्रशस्यतरं च अनुष्ठेयम्~। अतश्च यत् श्रेयः प्रशस्यतरम् एतयोः कर्मसंन्यासकर्मयोगयोः यदनुष्ठानात् श्रेयोवाप्तिः मम स्यादिति मन्यसे, तत् एकम् अन्यतरम् सह एकपुरुषानुष्ठेयत्वासम्भवात् मे ब्रूहि सुनिश्चितम् अभिप्रेतं तवेति~॥~१~॥\par
 स्वाभिप्रायम् आचक्षाणो निर्णयाय श्रीभगवानुवाच —}\\ 
\begin{center}{\bfseries श्रीभगवानुवाच —\\ संन्यासः कर्मयोगश्च निःश्रेयसकरावुभौ~।\\तयोस्तु कर्मसंन्यासात्कर्मयोगो विशिष्यते~॥~२~॥}\end{center} 
संन्यासः कर्मणां परित्यागः कर्मयोगश्च तेषामनुष्ठानं तौ उभौ अपि निःश्रेयसकरौ मोक्षं कुर्वाते ज्ञानोत्पत्तिहेतुत्वेन~। उभौ यद्यपि निःश्रेयसकरौ, तथापि तयोस्तु निःश्रेयसहेत्वोः कर्मसंन्यासात् केवलात् कर्मयोगो विशिष्यते इति कर्मयोगं स्तौति~॥~२~॥\par
 कस्मात् इति आह — } 
\begin{center}{\bfseries ज्ञेयः स नित्यसंन्यासी यो न द्वेष्टि न काङ्क्षति~।\\निर्द्वन्द्वो हि महाबाहो सुखं बन्धात्प्रमुच्यते~॥~३~॥}\end{center} 
ज्ञेयः ज्ञातव्यः स कर्मयोगी नित्यसंन्यासी इति यो न द्वेष्टि किञ्चित् न काङ्क्षति दुःखसुखे तत्साधने च~। एवंविधो यः, कर्मणि वर्तमानोऽपि स नित्यसंन्यासी इति ज्ञातव्यः इत्यर्थः~। निर्द्वन्द्वः द्वन्द्ववर्जितः हि यस्मात् महाबाहो सुखं बन्धात् अनायासेन प्रमुच्यते~॥~३~॥\par
 संन्यासकर्मयोगयोः भिन्नपुरुषानुष्ठेययोः विरुद्धयोः फलेऽपि विरोधो युक्तः, न तु उभयोः निःश्रेयसकरत्वमेव इति प्राप्ते इदम् उच्यते —} 
\begin{center}{\bfseries साङ्‍ख्ययोगौ पृथग्बालाः प्रवदन्ति न पण्डिताः~।\\एकमप्यास्थितः सम्यगुभयोर्विन्दते फलम्~॥~४~॥}\end{center} 
साङ्‍ख्ययोगौ पृथक् विरुद्धभिन्नफलौ बालाः प्रवदन्ति न पण्डिताः~। पण्डितास्तु ज्ञानिन एकं फलम् अविरुद्धम् इच्छन्ति~। कथम्~? एकमपि साङ्ख्ययोगयोः सम्यक् आस्थितः सम्यगनुष्ठितवान् इत्यर्थः, उभयोः विन्दते फलम्~। उभयोः तदेव हि निःश्रेयसं फलम्~; अतः न फले विरोधः अस्ति~॥~} 
ननु संन्यासकर्मयोगशब्देन प्रस्तुत्य साङ्‍ख्ययोगयोः फलैकत्वं कथम् इह अप्रकृतं ब्रवीति~? नैष दोषः — यद्यपि अर्जुनेन संन्यासं कर्मयोगं च केवलम् अभिप्रेत्य प्रश्नः कृतः, भगवांस्तु तदपरित्यागेनैव स्वाभिप्रेतं च विशेषं संयोज्य शब्दान्तरवाच्यतया प्रतिवचनं ददौ ‘साङ्‍ख्ययोगौ’ इति~। तौ एव संन्यासकर्मयोगौ ज्ञानतदुपायसमबुद्धित्वादिसंयुक्तौ साङ्‍ख्ययोगशब्दवाच्यौ इति भगवतो मतम्~। अतः न अप्रकृतप्रक्रियेति~॥~४~॥\par
 एकस्यापि सम्यगनुष्ठानात् कथम् उभयोः फलं विन्दते इति उच्यते —} 
\begin{center}{\bfseries यत्साङ्‍ख्यैः प्राप्यते स्थानं तद्योगैरपि गम्यते~।\\एकं साङ्‍ख्यं च योगं च यः पश्यति स पश्यति~॥~५~॥}\end{center} 
यत् साङ्ख्यैः ज्ञाननिष्ठैः संन्यासिभिः प्राप्यते स्थानं मोक्षाख्यम्~, तत् योगैरपि ज्ञानप्राप्त्युपायत्वेन ईश्वरे समर्प्य कर्माणि आत्मनः फलम् अनभिसन्धाय अनुतिष्ठन्ति ये ते योगाः योगिनः तैरपि परमार्थज्ञानसंन्यासप्राप्तिद्वारेण गम्यते इत्यभिप्रायः~। अतः एकं साङ्‍ख्यं च योगं च यः पश्यति फलैकत्वात् स पश्यति सम्यक् पश्यतीत्यर्थः  —~॥~५~॥~} 
एवं तर्हि योगात् संन्यास एव विशिष्यते~; कथं तर्हि इदमुक्तम् ‘तयोस्तु कर्मसंन्यासात् कर्मयोगो विशिष्यते’\footnote{भ. गी. ५~। २} इति~? शृणु तत्र कारणम् — त्वया पृष्टं केवलं कर्मसंन्यासं कर्मयोगं च अभिप्रेत्य तयोः अन्यतरः कः श्रेयान् इति~। तदनुरूपं प्रतिवचनं मया उक्तं कर्मसंन्यासात् कर्मयोगः विशिष्यते इति ज्ञानम् अनपेक्ष्य~। ज्ञानापेक्षस्तु संन्यासः साङ्‍ख्यमिति मया अभिप्रेतः~। परमार्थयोगश्च स एव~। यस्तु कर्मयोगः वैदिकः स च तादर्थ्यात् योगः संन्यास इति च उपचर्यते~। कथं तादर्थ्यम् इति उच्यते\par
 \begin{center}{\bfseries संन्यासस्तु महाबाहो दुःखमाप्तुमयोगतः~।\\योगयुक्तो मुनिर्ब्रह्म नचिरेणाधिगच्छति~॥~६~॥}\end{center} 
संन्यासस्तु पारमार्थिकः हे महाबाहो दुःखम् आप्तुं प्राप्तुम् अयोगतः योगेन विना~। योगयुक्तः वैदिकेन कर्मयोगेन ईश्वरसमर्पितरूपेण फलनिरपेक्षेण युक्तः, मुनिः मननात् ईश्वरस्वरूपस्य मुनिः, ब्रह्म — परमात्मज्ञाननिष्ठालक्षणत्वात् प्रकृतः संन्यासः ब्रह्म उच्यते, ‘न्यास इति ब्रह्मा ब्रह्मा हि परः’\footnote{तै. ना. ७८} इति श्रुतेः — ब्रह्म परमार्थसंन्यासं परमार्थज्ञाननिष्ठालक्षणं नचिरेण क्षिप्रमेव अधिगच्छति प्राप्नोति~। अतः मया उक्तम् ‘कर्मयोगो विशिष्यते’\footnote{भ. गी. ५~। २} इति~॥~६~॥\par
 यदा पुनः अयं सम्यग्ज्ञानप्राप्त्युपायत्वेन —} 
\begin{center}{\bfseries योगयुक्तो विशुद्धात्मा विजितात्मा जितेन्द्रियः~।\\सर्वभूतात्मभूतात्मा कुर्वन्नपि न लिप्यते~॥~७~॥}\end{center} 
योगेन युक्तः योगयुक्तः, विशुद्धात्मा विशुद्धसत्त्वः, विजितात्मा विजितदेहः, जितेन्द्रियश्च, सर्वभूतात्मभूतात्मा सर्वेषां ब्रह्मादीनां स्तम्बपर्यन्तानां भूतानाम् आत्मभूतः आत्मा प्रत्यक्चेतनो यस्य सः सर्वभूतात्मभूतात्मा सम्यग्दर्शीत्यर्थः, स तत्रैवं वर्तमानः लोकसङ्ग्रहाय कर्म कुर्वन्नपि न लिप्यते न कर्मभिः बध्यते इत्यर्थः~॥~७~॥\par
 न च असौ परमार्थतः करोतीत्यतः —} 
\begin{center}{\bfseries नैव किञ्चित्करोमीति युक्तो मन्येत तत्त्ववित्~।\\पश्यञ्शृण्वन्स्पृशञ्जिघ्रन्नश्नन्गच्छन्स्वपञ्श्वसन्~॥~८~॥}\\[10pt]
{\bfseries प्रलपन् विसृजन्गृह्णन्नुन्मिषन्निमिषन्नपि~।\\इन्द्रियाणीन्द्रियार्थेषु वर्तन्त इति धारयन्~॥~९~॥}\end{center} 
नैव किञ्चित् करोमीति युक्तः समाहितः सन् मन्येत चिन्तयेत्~, तत्त्ववित् आत्मनो याथात्म्यं तत्त्वं वेत्तीति तत्त्ववित् परमार्थदर्शीत्यर्थः~॥~} 
कदा कथं वा तत्त्वमवधारयन् मन्येत इति, उच्यते — पश्यन्निति~। मन्येत इति पूर्वेण सम्बन्धः~। यस्य एवं तत्त्वविदः सर्वकार्यकरणचेष्टासु कर्मसु अकर्मैव, पश्यतः सम्यग्दर्शिनः तस्य सर्वकर्मसंन्यासे एव अधिकारः, कर्मणः अभावदर्शनात्~। न हि मृगतृष्णिकायाम् उदकबुद्ध्या पानाय प्रवृत्तः उदकाभावज्ञानेऽपि तत्रैव पानप्रयोजनाय प्रवर्तते~॥~९~॥\par
 यस्तु पुनः अतत्त्ववित् प्रवृत्तश्च कर्मयोगे —} 
\begin{center}{\bfseries ब्रह्मण्याधाय कर्माणि सङ्गं त्यक्त्वा करोति यः~।\\लिप्यते न स पापेन पद्मपत्रमिवाम्भसा~॥~१०~॥}\end{center} 
ब्रह्मणि ईश्वरे आधाय निक्षिप्य ‘तदर्थं कर्म करोमि’ इति भृत्य इव स्वाम्यर्थं सर्वाणि कर्माणि मोक्षेऽपि फले सङ्गं त्यक्त्वा करोति यः सर्वकर्माणि, लिप्यते न स पापेन न सम्बध्यते पद्मपत्रमिव अम्भसा उदकेन~। केवलं सत्त्वशुद्धिमात्रमेव फलं तस्य कर्मणः स्यात्~॥~१०~॥\par
 यस्मात् —} 
\begin{center}{\bfseries कायेन मनसा बुद्ध्या केवलैरिन्द्रियैरपि~।\\योगिनः कर्म कुर्वन्ति सङ्गं त्यक्त्वात्मशुद्धये~॥~११~॥}\end{center} 
कायेन देहेन मनसा बुद्ध्या च केवलैः ममत्ववर्जितैः ‘ईश्वरायैव कर्म करोमि, न मम फलाय’ इति ममत्वबुद्धिशून्यैः इन्द्रियैरपि — केवलशब्दः कायादिभिरपि प्रत्येकं सम्बध्यते — सर्वव्यापारेषु ममतावर्जनाय~। योगिनः कर्मिणः कर्म कुर्वन्ति सङ्गं त्यक्त्वा फलविषयम् आत्मशुद्धये सत्त्वशुद्धये इत्यर्थः~। तस्मात् तत्रैव तव अधिकारः इति कुरु कर्मैव~॥~११~॥\par
 यस्माच्च —} 
\begin{center}{\bfseries युक्तः कर्मफलं त्यक्त्वा शान्तिमाप्नोति नैष्ठिकीम्~।\\अयुक्तः कामकारेण फले सक्तो निबध्यते~॥~१२~॥}\end{center} 
युक्तः ‘ईश्वराय कर्माणि करोमि न मम फलाय’ इत्येवं समाहितः सन् कर्मफलं त्यक्त्वा परित्यज्य शान्तिं मोक्षाख्याम् आप्नोति नैष्ठिकीं निष्ठायां भवां सत्त्वशुद्धिज्ञानप्राप्तिसर्वकर्मसंन्यासज्ञाननिष्ठाक्रमेणेति वाक्यशेषः~। यस्तु पुनः अयुक्तः असमाहितः कामकारेण करणं कारः कामस्य कारः कामकारः तेन कामकारेण, कामप्रेरिततयेत्यर्थः, ‘मम फलाय इदं करोमि कर्म’ इत्येवं फले सक्तः निबध्यते~। अतः त्वं युक्तो भव इत्यर्थः~॥~१२~॥\par
 यस्तु परमार्थदर्शी सः —} 
\begin{center}{\bfseries सर्वकर्माणि मनसा संन्यस्यास्ते सुखं वशी~।\\नवद्वारे पुरे देही नैव कुर्वन्न कारयन्~॥~१३~॥}\end{center} 
सर्वाणि कर्माणि सर्वकर्माणि संन्यस्य परित्यज्य नित्यं नैमित्तिकं काम्यं प्रतिषिद्धं च तानि सर्वाणि कर्माणि मनसा विवेकबुद्ध्या, कर्मादौ अकर्मसन्दर्शनेन सन्त्यज्येत्यर्थः, आस्ते तिष्ठति सुखम्~। त्यक्तवाङ्मनःकायचेष्टः निरायासः प्रसन्नचित्तः आत्मनः अन्यत्र निवृत्तसर्वबाह्यप्रयोजनः इति ‘सुखम् आस्ते’ इत्युच्यते~। वशी जितेन्द्रिय इत्यर्थः~। क्व कथम् आस्ते इति, आह — नवद्वारे पुरे~। सप्त शीर्षण्यानि आत्मन उपलब्धिद्वाराणि, अवाक् द्वे मूत्रपुरीषविसर्गार्थे, तैः द्वारैः नवद्वारं पुरम् उच्यते शरीरम्~, पुरमिव पुरम्~, आत्मैकस्वामिकम्~, तदर्थप्रयोजनैश्च इन्द्रियमनोबुद्धिविषयैः अनेकफलविज्ञानस्य उत्पादकैः पौरैरिव अधिष्ठितम्~। तस्मिन् नवद्वारे पुरे देही सर्वं कर्म संन्यस्य आस्ते~; किं विशेषणेन~? सर्वो हि देही संन्यासी असंन्यासी वा देहे एव आस्ते~; तत्र अनर्थकं विशेषणमिति~। उच्यते — यस्तु अज्ञः देही देहेन्द्रियसङ्घातमात्रात्मदर्शी स सर्वोऽपि ‘गेहे भूमौ आसने वा आसे’ इति मन्यते~। न हि देहमात्रात्मदर्शिनः गेहे इव देहे आसे इति प्रत्ययः सम्भवति~। देहादिसङ्घातव्यतिरिक्तात्मदर्शिनस्तु ‘देहे आसे’ इति प्रत्ययः उपपद्यते~। परकर्मणां च परस्मिन् आत्मनि अविद्यया अध्यारोपितानां विद्यया विवेकज्ञानेन मनसा संन्यास उपपद्यते~। उत्पन्नविवेकज्ञानस्य सर्वकर्मसंन्यासिनोऽपि गेहे इव देहे एव नवद्वारे पुरे आसनम् प्रारब्धफलकर्मसंस्कारशेषानुवृत्त्या देह एव विशेषविज्ञानोत्पत्तेः~। देहे एव आस्ते इति अस्त्येव विशेषणफलम्~, विद्वदविद्वत्प्रत्ययभेदापेक्षत्वात्~॥~} 
यद्यपि कार्यकरणकर्माणि अविद्यया आत्मनि अध्यारोपितानि ‘संन्यस्यास्ते’ इत्युक्तम्~, तथापि आत्मसमवायि तु कर्तृत्वं कारयितृत्वं च स्यात् इति आशङ्क्य आह — नैव कुर्वन् स्वयम्~, न च कार्यकरणानि कारयन् क्रियासु प्रवर्तयन्~। किं यत् तत् कर्तृत्वं कारयितृत्वं च देहिनः स्वात्मसमवायि सत् संन्यासात् न सम्भवति, यथा गच्छतो गतिः गमनव्यापारपरित्यागे न स्यात् तद्वत्~? किं वा स्वत एव आत्मनः न अस्ति इति~? अत्र उच्यते — न अस्ति आत्मनः स्वतः कर्तृत्वं कारयितृत्वं च~। उक्तं हि ‘अविकार्योऽयमुच्यते’\footnote{भ. गी. २~। २५} ‘शरीरस्थोऽपि न करोति न लिप्यते’\footnote{भ. गी. १३~। ३१} इति~। ‘ध्यायतीव लेलायतीव’\footnote{बृ. उ. ४~। ३~। ७} इति श्रुतेः~॥~१३~॥\par
 किञ्च— } 
\begin{center}{\bfseries न कर्तृत्वं न कर्माणि लोकस्य सृजति प्रभुः~।\\न कर्मफलसंयोगं स्वभावस्तु प्रवर्तते~॥~१४~॥}\end{center} 
न कर्तृत्वं स्वतः कुरु इति नापि कर्माणि रथघटप्रासादादीनि ईप्सिततमानि लोकस्य सृजति उत्पादयति प्रभुः आत्मा~। नापि रथादि कृतवतः तत्फलेन संयोगं न कर्मफलसंयोगम्~। यदि किञ्चिदपि स्वतः न करोति न कारयति च देही, कः तर्हि कुर्वन् कारयंश्च प्रवर्तते इति, उच्यते — स्वभावस्तु स्वो भावः स्वभावः अविद्यालक्षणा प्रकृतिः माया प्रवर्तते ‘दैवी हि’\footnote{भ. गी. ७~। १४} इत्यादिना वक्ष्यमाणा~॥~१४~॥\par
 परमार्थतस्तु —} 
\begin{center}{\bfseries नादत्ते कस्यचित्पापं न चैव सुकृतं विभुः~।\\अज्ञानेनावृतं ज्ञानं तेन मुह्यन्ति जन्तवः~॥~१५~॥}\end{center} 
न आदत्ते न च गृह्णाति भक्तस्यापि कस्यचित् पापम्~। न चैव आदत्ते सुकृतं भक्तैः प्रयुक्तं विभुः~। किमर्थं तर्हि भक्तैः पूजादिलक्षणं यागदानहोमादिकं च सुकृतं प्रयुज्यते इत्याह — अज्ञानेन आवृतं ज्ञानं विवेकविज्ञानम्~, तेन मुह्यन्ति ‘करोमि कारयामि भोक्ष्ये भोजयामि’ इत्येवं मोहं गच्छन्ति अविवेकिनः संसारिणो जन्तवः~॥~१५~॥\par
 \begin{center}{\bfseries ज्ञानेन तु तदज्ञानं येषां नाशितमात्मनः~।\\तेषामादित्यवज्ज्ञानं प्रकाशयति तत्परम्~॥~१६~॥}\end{center} 
ज्ञानेन तु येन अज्ञानेन आवृताः मुह्यन्ति जन्तवः तत् अज्ञानं येषां जन्तूनां विवेकज्ञानेन आत्मविषयेण नाशितम् आत्मनः भवति, तेषां जन्तूनाम् आदित्यवत् यथा आदित्यः समस्तं रूपजातम् अवभासयति तद्वत् ज्ञानं ज्ञेयं वस्तु सर्वं प्रकाशयति तत् परं परमार्थतत्त्वम्~॥~१६~॥\par
 यत् परं ज्ञानं प्रकाशितम् —} 
\begin{center}{\bfseries तद्बुद्धयस्तदात्मानस्तन्निष्ठास्तत्परायणाः~।\\गच्छन्त्यपुनरावृत्तिं ज्ञाननिर्धूतकल्मषाः~॥~१७~॥}\end{center} 
तस्मिन् ब्रह्मणि गता बुद्धिः येषां ते तद्बुद्धयः, तदात्मानः तदेव परं ब्रह्म आत्मा येषां ते तदात्मानः, तन्निष्ठाः निष्ठा अभिनिवेशः तात्पर्यं सर्वाणि कर्माणि संन्यस्य तस्मिन् ब्रह्मण्येव अवस्थानं येषां ते तन्निष्ठाः, तत्परायणाश्च तदेव परम् अयनं परा गतिः येषां भवति ते तत्परायणाः केवलात्मरतय इत्यर्थः~। येषां ज्ञानेन नाशितम् आत्मनः अज्ञानं ते गच्छन्ति एवंविधाः अपुनरावृत्तिम् अपुनर्देहसम्बन्धं ज्ञाननिर्धूतकल्मषाः यथोक्तेन ज्ञानेन निर्धूतः नाशितः कल्मषः पापादिसंसारकारणदोषः येषां ते ज्ञाननिर्धूतकल्मषाः यतयः इत्यर्थः~॥~१७~॥\par
 येषां ज्ञानेन नाशितम् आत्मनः अज्ञानं ते पण्डिताः कथं तत्त्वं पश्यन्ति इत्युच्यते —} 
\begin{center}{\bfseries विद्याविनयसम्पन्ने ब्राह्मणे गवि हस्तिनि~।\\शुनि चैव श्वपाके च पण्डिताः समदर्शिनः~॥~१८~॥}\end{center} 
विद्याविनयसम्पन्ने विद्या च विनयश्च विद्याविनयौ, विनयः उपशमः, ताभ्यां विद्याविनयाभ्यां सम्पन्नः विद्याविनयसम्पन्नः विद्वान् विनीतश्च यो ब्राह्मणः तस्मिन् ब्राह्मणे गवि हस्तिनि शुनि चैव श्वपाके च पण्डिताः समदर्शिनः~। विद्याविनयसम्पन्ने उत्तमसंस्कारवति ब्राह्मणे सात्त्विके, मध्यमायां च राजस्यां गवि, संस्कारहीनायां अत्यन्तमेव केवलतामसे हस्त्यादौ च, सत्त्वादिगुणैः तज्जैश्च संस्कारैः तथा राजसैः तथा तामसैश्च संस्कारैः अत्यन्तमेव अस्पृष्टं समम् एकम् अविक्रियं तत् ब्रह्म द्रष्टुं शीलं येषां ते पण्डिताः समदर्शिनः~॥~१८~॥\par
 ननु अभोज्यान्नाः ते दोषवन्तः, ‘समासमाभ्यां विषमसमे पूजातः’\footnote{गौ. ध. २~। ८~। २०~; १७~। १८} इति स्मृतेः~। न ते दोषवन्तः~। कथम्~? —} 
\begin{center}{\bfseries इहैव तैर्जितः सर्गो येषां साम्ये स्थितं मनः~।\\निर्दोषं हि समं ब्रह्म तस्माद्ब्रह्मणि ते स्थिताः~॥~१९~॥}\end{center} 
इह एव जीवद्भिरेव तैः समदर्शिभिः पण्डितैः जितः वशीकृतः सर्गः जन्म, येषां साम्ये सर्वभूतेषु ब्रह्मणि समभावे स्थितं निश्चलीभूतं मनः अन्तःकरणम्~। निर्दोषं यद्यपि दोषवत्सु श्वपाकादिषु मूढैः तद्दोषैः दोषवत् इव विभाव्यते, तथापि तद्दोषैः अस्पृष्टम् इति निर्दोषं दोषवर्जितं हि यस्मात्~; नापि स्वगुणभेदभिन्नम्~, निर्गुणत्वात् चैतन्यस्य~। वक्ष्यति च भगवान् इच्छादीनां क्षेत्रधर्मत्वम्~, ‘अनादित्वान्निर्गुणत्वात्’\footnote{भ. गी. १३~। ३१} इति च~। नापि अन्त्या विशेषाः आत्मनो भेदकाः सन्ति, प्रतिशरीरं तेषां सत्त्वे प्रमाणानुपपत्तेः~। अतः समं ब्रह्म एकं च~। तस्मात् ब्रह्मणि एव ते स्थिताः~। तस्मात् न दोषगन्धमात्रमपि तान् स्पृशति, देहादिसङ्घातात्मदर्शनाभिमानाभावात् तेषाम्~। देहादिसङ्घातात्मदर्शनाभिमानवद्विषयं तु तत् सूत्रम् ‘समासमाभ्यां विषमसमे पूजातः’\footnote{गौ. ध. २~। ८~। २०} इति, पूजाविषयत्वेन विशेषणात्~। दृश्यते हि ब्रह्मवित् षडङ्गवित् चतुर्वेदवित् इति पूजादानादौ गुणविशेषसम्बन्धः कारणम्~। ब्रह्म तु सर्वगुणदोषसम्बन्धवर्जितमित्यतः ‘ब्रह्मणि ते स्थिताः’ इति युक्तम्~। कर्मविषयं च ‘समासमाभ्याम्’\footnote{गौ. ध. २~। ८~। २०} इत्यादि~। इदं तु सर्वकर्मसंन्यासविषयं प्रस्तुतम्~, ‘सर्वकर्माणि मनसा’\footnote{भ. गी. ५~। १३} इत्यारभ्य अध्यायपरिसमाप्तेः~॥~१९~॥\par
 यस्मात् निर्दोषं समं ब्रह्म आत्मा, तस्मात् —} 
\begin{center}{\bfseries न प्रहृष्येत्प्रियं प्राप्य नोद्विजेत्प्राप्य चाप्रियम्~।\\स्थिरबुद्धिरसंमूढो ब्रह्मविद्ब्रह्मणि स्थितः~॥~२०~॥}\end{center} 
न प्रहृष्येत् प्रहर्षं न कुर्यात् प्रियम् इष्टं प्राप्य लब्ध्वा~। न उद्विजेत् प्राप्य च अप्रियम् अनिष्टं लब्ध्वा~। देहमात्रात्मदर्शिनां हि प्रियाप्रियप्राप्ती हर्षविषादौ कुर्वाते, न केवलात्मदर्शिनः, तस्य प्रियाप्रियप्राप्त्यसम्भवात्~। किञ्च — ‘सर्वभूतेषु एकः समः निर्दोषः आत्मा’ इति स्थिरा निर्विचिकित्सा बुद्धिः यस्य सः स्थिरबुद्धिः असंमूढः संमोहवर्जितश्च स्यात् यथोक्तब्रह्मवित् ब्रह्मणि स्थितः, अकर्मकृत् सर्वकर्मसंन्यासी इत्यर्थः~॥~२०~॥\par
 किञ्च, ब्रह्मणि स्थितः —} 
\begin{center}{\bfseries बाह्यस्पर्शेष्वसक्तात्मा विन्दत्यात्मनि यत्सुखम्~।\\स ब्रह्मयोगयुक्तात्मा सुखमक्षयमश्नुते~॥~२१~॥}\end{center} 
बाह्यस्पर्शेषु बाह्याश्च ते स्पर्शाश्च बाह्यस्पर्शाः स्पृश्यन्ते इति स्पर्शाः शब्दादयो विषयाः तेषु बाह्यस्पर्शेषु, असक्तः आत्मा अन्तःकरणं यस्य सः अयम् असक्तात्मा विषयेषु प्रीतिवर्जितः सन् विन्दति लभते आत्मनि यत् सुखं तत् विन्दति इत्येतत्~। स ब्रह्मयोगयुक्तात्मा ब्रह्मणि योगः समाधिः ब्रह्मयोगः तेन ब्रह्मयोगेन युक्तः समाहितः तस्मिन् व्यापृतः आत्मा अन्तःकरणं यस्य सः ब्रह्मयोगयुक्तात्मा, सुखम् अक्षयम् अश्नुते व्याप्नोति~। तस्मात् बाह्यविषयप्रीतेः क्षणिकायाः इन्द्रियाणि निवर्तयेत् आत्मनि अक्षयसुखार्थी इत्यर्थः~॥~२१~॥\par
 इतश्च निवर्तयेत् —} 
\begin{center}{\bfseries ये हि संस्पर्शजा भोगा दुःखयोनय एव ते~।\\आद्यन्तवन्तः कौन्तेय न तेषु रमते बुधः~॥~२२~॥}\end{center} 
ये हि यस्मात् संस्पर्शजाः विषयेन्द्रियसंस्पर्शेभ्यो जाताः भोगा भुक्तयः दुःखयोनय एव ते, अविद्याकृतत्वात्~। दृश्यन्ते हि आध्यात्मिकादीनि दुःखानि तन्निमित्तान्येव~। यथा इहलोके तथा परलोकेऽपि इति गम्यते एवशब्दात्~। न संसारे सुखस्य गन्धमात्रमपि अस्ति इति बुद्ध्वा विषयमृगतृष्णिकाया इन्द्रियाणि निवर्तयेत्~। न केवलं दुःखयोनय एव, आद्यन्तवन्तश्च, आदिः विषयेन्द्रियसंयोगो भोगानाम् अन्तश्च तद्वियोग एव~; अतः आद्यन्तवन्तः अनित्याः, मध्यक्षणभावित्वात् इत्यर्थः~। कौन्तेय, न तेषु भोगेषु रमते बुधः विवेकी अवगतपरमार्थतत्त्वः~; अत्यन्तमूढानामेव हि विषयेषु रतिः दृश्यते, यथा पशुप्रभृतीनाम्~॥~२२~॥\par
 अयं च श्रेयोमार्गप्रतिपक्षी कष्टतमो दोषः सर्वानर्थप्राप्तिहेतुः दुर्निवारश्च इति तत्परिहारे यत्नाधिक्यं कर्तव्यम् इत्याह भगवान् —} 
\begin{center}{\bfseries शक्नोतीहैव यः सोढुं प्राक्छरीरविमोक्षणात्~।\\कामक्रोधोद्भवं वेगं स युक्तः स सुखी नरः~॥~२३~॥}\end{center} 
शक्नोति उत्सहते इहैव जीवन्नेव यः सोढुं प्रसहितुं प्राक् पूर्वं शरीरविमोक्षणात् आ मरणात् इत्यर्थः~। मरणसीमाकरणं जीवतोऽवश्यम्भावि हि कामक्रोधोद्भवो वेगः, अनन्तनिमित्तवान् हि सः इति यावत् मरणं तावत् न विस्रम्भणीय इत्यर्थः~। कामः इन्द्रियगोचरप्राप्ते इष्टे विषये श्रूयमाणे स्मर्यमाणे वा अनुभूते सुखहेतौ या गर्धिः तृष्णा स कामः~; क्रोधश्च आत्मनः प्रतिकूलेषु दुःखहेतुषु दृश्यमानेषु श्रूयमाणेषु स्मर्यमाणेषु वा यो द्वेषः सः क्रोधः~; तौ कामक्रोधौ उद्भवो यस्य वेगस्य सः कामक्रोधोद्भवः वेगः~। रोमाञ्चनप्रहृष्टनेत्रवदनादिलिङ्गः अन्तःकरणप्रक्षोभरूपः कामोद्भवो वेगः, गात्रप्रकम्पप्रस्वेदसन्दष्टोष्ठपुटरक्तनेत्रादिलिङ्गः क्रोधोद्भवो वेगः, तं कामक्रोधोद्भवं वेगं यः उत्सहते प्रसहते सोढुं प्रसहितुम्~, सः युक्तः योगी सुखी च इह लोके नरः~॥~२३~॥\par
 कथम्भूतश्च ब्रह्मणि स्थितः ब्रह्म प्राप्नोति इति आह भगवान् —} 
\begin{center}{\bfseries योऽन्तःसुखोऽन्तरारामस्तथान्तर्ज्योतिरेव यः~।\\स योगी ब्रह्मनिर्वाणं ब्रह्मभूतोऽधिगच्छति~॥~२४~॥}\end{center} 
यः अन्तःसुखः अन्तः आत्मनि सुखं यस्य सः अन्तःसुखः, तथा अन्तरेव आत्मनि आरामः आरमणं क्रीडा यस्य सः अन्तरारामः, तथा एव अन्तः एव आत्मन्येव ज्योतिः प्रकाशो यस्य सः अन्तर्ज्योतिरेव, यः ईदृशः सः योगी ब्रह्मनिर्वाणं ब्रह्मणि निर्वृतिं मोक्षम् इह जीवन्नेव ब्रह्मभूतः सन् अधिगच्छति प्राप्नोति~॥~२४~॥\par
 किञ्च —} 
\begin{center}{\bfseries लभन्ते ब्रह्मनिर्वाणमृषयः क्षीणकल्मषाः~।\\छिन्नद्वैधा यतात्मानः सर्वभूतहिते रताः~॥~२५~॥}\end{center} 
लभन्ते ब्रह्मनिर्वाणं मोक्षम् ऋषयः सम्यग्दर्शिनः संन्यासिनः क्षीणकल्मषाः क्षीणपापाः निर्दोषाः छिन्नद्वैधाः छिन्नसंशयाः यतात्मानः संयतेन्द्रियाः सर्वभूतहिते रताः सर्वेषां भूतानां हिते आनुकूल्ये रताः अहिंसका इत्यर्थः~॥~२५~॥\par
 किञ्च —} 
\begin{center}{\bfseries कामक्रोधवियुक्तानां यतीनां यतचेतसाम्~।\\अभितो ब्रह्मनिर्वाणं वर्तते विदितात्मनाम्~॥~२६~॥}\end{center} 
कामक्रोधवियुक्तानां कामश्च क्रोधश्च कामक्रोधौ ताभ्यां वियुक्तानां यतीनां संन्यासिनां यतचेतसां संयतान्तःकरणानाम् अभितः उभयतः जीवतां मृतानां च ब्रह्मनिर्वाणं मोक्षो वर्तते विदितात्मनां विदितः ज्ञातः आत्मा येषां ते विदितात्मानः तेषां विदितात्मनां सम्यग्दर्शिनामित्यर्थः~॥~२६~॥\par
 सम्यग्दर्शननिष्ठानां संन्यासिनां सद्यः मुक्तिः उक्ता~। कर्मयोगश्च ईश्वरार्पितसर्वभावेन ईश्वरे ब्रह्मणि आधाय क्रियमाणः सत्त्वशुद्धिज्ञानप्राप्तिसर्वकर्मसंन्यासक्रमेण मोक्षाय इति भगवान् पदे पदे अब्रवीत्~, वक्ष्यति च~। अथ इदानीं ध्यानयोगं सम्यग्दर्शनस्य अन्तरङ्गं विस्तरेण वक्ष्यामि इति तस्य सूत्रस्थानीयान् श्लोकान् उपदिशति स्म —} 
\begin{center}{\bfseries स्पर्शान्कृत्वा बहिर्बाह्यांश्चक्षुश्चैवान्तरे भ्रुवोः~।\\प्राणापानौ समौ कृत्वा नासाभ्यन्तरचारिणौ~॥~२७~॥}\\[10pt]
{\bfseries यतेन्द्रियमनोबुद्धिर्मुनिर्मोक्षपरायणः~।\\विगतेच्छाभयक्रोधो यः सदा मुक्त एव सः~॥~२८~॥}\end{center} 
स्पर्शान् शब्दादीन् कृत्वा बहिः बाह्यान् — श्रोत्रादिद्वारेण अन्तः बुद्धौ प्रवेशिताः शब्दादयः विषयाः तान् अचिन्तयतः शब्दादयो बाह्या बहिरेव कृताः भवन्ति — तान् एवं बहिः कृत्वा चक्षुश्चैव अन्तरे भ्रुवोः ‘कृत्वा’ इति अनुषज्यते~। तथा प्राणापानौ नासाभ्यन्तरचारिणौ समौ कृत्वा, यतेन्द्रियमनोबुद्धिः यतानि संयतानि इन्द्रियाणि मनः बुद्धिश्च यस्य सः यतेन्द्रियमनोबुद्धिः, मननात् मुनिः संन्यासी, मोक्षपरायणः एवं देहसंस्थानात् मोक्षपरायणः मोक्ष एव परम् अयनं परा गतिः यस्य सः अयं मोक्षपरायणो मुनिः भवेत्~। विगतेच्छाभयक्रोधः इच्छा च भयं च क्रोधश्च इच्छाभयक्रोधाः ते विगताः यस्मात् सः विगतेच्छाभयक्रोधः, यः एवं वर्तते सदा संन्यासी, मुक्त एव सः न तस्य मोक्षायान्यः कर्तव्योऽस्ति~॥~२८~॥\par
 एवं समाहितचित्तेन किं विज्ञेयम् इति, उच्यते —} 
\begin{center}{\bfseries भोक्तारं यज्ञतपसां सर्वलोकमहेश्वरम्~।\\सुहृदं सर्वभूतानां ज्ञात्वा मां शान्तिमृच्छति~॥~२९~॥}\end{center} 
भोक्तारं यज्ञतपसां यज्ञानां तपसां च कर्तृरूपेण देवतारूपेण च, सर्वलोकमहेश्वरं सर्वेषां लोकानां महान्तम् ईश्वरं सुहृदं सर्वभूतानां सर्वप्राणिनां प्रत्युपकारनिरपेक्षतया उपकारिणं सर्वभूतानां हृदयेशयं सर्वकर्मफलाध्यक्षं सर्वप्रत्ययसाक्षिणं मां नारायणं ज्ञात्वा शान्तिं सर्वसंसारोपरतिम् ऋच्छति प्राप्नोति~॥~२९~॥\par
 
इति श्रीमत्परमहंसपरिव्राजकाचार्यस्य श्रीगोविन्दभगवत्पूजयपादशिष्यस्य श्रीमच्छङ्करभगवतः कृतौ श्रीमद्भगवद्गीताभाष्ये पञ्चमोऽध्यायः~॥\par
 
अतीतानन्तराध्यायान्ते ध्यानयोगस्य सम्यग्दर्शनं प्रति अन्तरङ्गस्य सूत्रभूताः श्लोकाः ‘स्पर्शान् कृत्वा बहिः’\footnote{भ. गी. ५~। २७} इत्यादयः उपदिष्टाः~। तेषां वृत्तिस्थानीयः अयं षष्ठोऽध्यायः आरभ्यते~। तत्र ध्यानयोगस्य बहिरङ्गं कर्म इति, यावत् ध्यानयोगारोहणसमर्थः तावत् गृहस्थेन अधिकृतेन कर्तव्यं कर्म इत्यतः तत् स्तौति — अनाश्रित इति~॥~} 
ननु किमर्थं ध्यानयोगारोहणसीमाकरणम्~, यावता अनुष्ठेयमेव विहितं कर्म यावज्जीवम्~। न, ‘आरुरुक्षोर्मुनेर्योगं कर्म कारणमुच्यते’\footnote{भ. गी. ६~। ३} इति विशेषणात्~, आरूढस्य च शमेनैव सम्बन्धकरणात्~। आरुरुक्षोः आरूढस्य च शमः कर्म च उभयं कर्तव्यत्वेन अभिप्रेतं चेत्स्यात्~, तदा ‘आरुरुक्षोः’ ‘आरूढस्य च’ इति शमकर्मविषयभेदेन विशेषणं विभागकरणं च अनर्थकं स्यात्~॥~} 
तत्र आश्रमिणां कश्चित् योगमारुरुक्षुः भवति, आरूढश्च कश्चित्~, अन्ये न आरुरुक्षवः न च आरूढाः~; तानपेक्ष्य ‘आरुरुक्षोः’ ‘आरूढस्य च’ इति विशेषणं विभागकरणं च उपपद्यत एवेति चेत्~, न~; ‘तस्यैव’ इति वचनात्~, पुनः योगग्रहणाच्च ‘योगारूढस्य’ इति~; य आसीत् पूर्वं योगमारुरुक्षुः, तस्यैव आरूढस्य शम एव कर्तव्यः कारणं योगफलं प्रति उच्यते इति~। अतो न यावज्जीवं कर्तव्यत्वप्राप्तिः कस्यचिदपि कर्मणः~। योगविभ्रष्टवचनाच्च — गृहस्थस्य चेत् कर्मिणो योगो विहितः षष्ठे अध्याये, सः योगविभ्रष्टोऽपि कर्मगतिं कर्मफलं प्राप्नोति इति तस्य नाशाशङ्का अनुपपन्ना स्यात्~। अवश्यं हि कृतं कर्म काम्यं नित्यं वा — मोक्षस्य नित्यत्वात् अनारभ्यत्वे — स्वं फलं आरभत एव~। नित्यस्य च कर्मणः वेदप्रमाणावबुद्धत्वात् फलेन भवितव्यम् इति अवोचाम, अन्यथा वेदस्य आनर्थक्यप्रसङ्गात् इति~। न च कर्मणि सति उभयविभ्रष्टवचनम्~, अर्थवत् कर्मणो विभ्रंशकारणानुपपत्तेः~॥~} 
कर्म कृतम् ईश्वरे संन्यस्य इत्यतः कर्तुः कर्म फलं नारभत इति चेत्~, न~; ईश्वरे संन्यासस्य अधिकतरफलहेतुत्वोपपत्तेः~॥~} 
मोक्षायैव इति चेत्~, स्वकर्मणां कृतानां ईश्वरे संन्यासो मोक्षायैव, न फलान्तराय योगसहितः~; योगाच्च विभ्रष्टः~; इत्यतः तं प्रति नाशाशङ्का युक्तैव इति चेत्~, न~; ‘एकाकी यतचित्तात्मा निराशीरपरिग्रहः’\footnote{भ. गी. ६~। १०} ‘ब्रह्मचारिव्रते स्थितः’\footnote{भ. गी. ६~। १४} इति कर्मसंन्यासविधानात्~। न च अत्र ध्यानकाले स्त्रीसहायत्वाशङ्का, येन एकाकित्वं विधीयते~। न च गृहस्थस्य ‘निराशीरपरिग्रहः’ इत्यादिवचनम् अनुकूलम्~। उभयविभ्रष्टप्रश्नानुपपत्तेश्च~॥~} 
अनाश्रित इत्यनेन कर्मिण एव संन्यासित्वं योगित्वं च उक्तम्~, प्रतिषिद्धं च निरग्नेः अक्रियस्य च संन्यासित्वं योगित्वं चेति चेत्~, न~; ध्यानयोगं प्रति बहिरङ्गस्य यतः कर्मणः फलाकाङ्क्षासंन्यासस्तुतिपरत्वात्~। न केवलं निरग्निः अक्रियः एव संन्यासी योगी च~। किं तर्हि~? कर्म्यपि, कर्मफलासङ्गं संन्यस्य कर्मयोगम् अनुतिष्ठन् सत्त्वशुद्ध्यर्थम्~, ‘स संन्यासी च योगी च भवति’ इति स्तूयते~। न च एकेन वाक्येन कर्मफलासङ्गसंन्यासस्तुतिः चतुर्थाश्रमप्रतिषेधश्च उपपद्यते~। न च प्रसिद्धं निरग्नेः अक्रियस्य परमार्थसंन्यासिनः श्रुतिस्मृतिपुराणेतिहासयोगशास्त्रेषु विहितं संन्यासित्वं योगित्वं च प्रतिषेधति भगवान्~। स्ववचनविरोधाच्च — ‘सर्वकर्माणि मनसा संन्सस्य . . . नैव कुर्वन्न कारयन् आस्ते’\footnote{भ. गी. ५~। १३} ‘मौनी सन्तुष्टो येन केनचित् अनिकेतः स्थिरमतिः’\footnote{भ. गी. १२~। १९} ‘विहाय कामान्यः सर्वान् पुमांश्चरति निःस्पृहः’\footnote{भ. गी. २~। ७१} ‘सर्वारम्भपरित्यागी’\footnote{भ. गी. १२~। १६} इति च तत्र तत्र भगवता स्ववचनानि दर्शितानि~; तैः विरुध्येत चतुर्थाश्रमप्रतिषेधः~। तस्मात् मुनेः योगम् आरुरुक्षोः प्रतिपन्नगार्हस्थ्यस्य अग्निहोत्रादिकर्म फलनिरपेक्षम् अनुष्ठीयमानं ध्यानयोगारोहणसाधनत्वं सत्त्वशुद्धिद्वारेण प्रतिपद्यते इति ‘स संन्यासी च योगी च’ इति स्तूयते~॥~} 
\begin{center}{\bfseries श्रीभगवानुवाच —\\ अनाश्रितः कर्मफलं कार्यं कर्म करोति यः~।\\स संन्यासी च योगी च न निरग्निर्न चाक्रियः~॥~१~॥}\end{center} 
अनाश्रितः न आश्रितः अनाश्रितः~। किम्~? कर्मफलं कर्मणां फलं कर्मफलं यत् तदनाश्रितः, कर्मफलतृष्णारहित इत्यर्थः~। यो हि कर्मफले तृष्णावान् सः कर्मफलमाश्रितो भवति, अयं तु तद्विपरीतः, अतः अनाश्रितः कर्मफलम्~। एवंभूतः सन् कार्यं कर्तव्यं नित्यं काम्यविपरीतम् अग्निहोत्रादिकं कर्म करोति निर्वर्तयति यः कश्चित् ईदृशः कर्मी स कर्म्यन्तरेभ्यो विशिष्यते इत्येवमर्थमाह — ‘स संन्यासी च योगी च’ इति~। संन्यासः परित्यागः स यस्यास्ति स संन्यासी च, योगी च योगः चित्तसमाधानं स यस्यास्ति स योगी च इति एवंगुणसम्पन्नः अयं मन्तव्यः’ न केवलं निरग्निः अक्रिय एव संन्यासी योगी च इति मन्तव्यः~। निर्गताः अग्नयः कर्माङ्गभूताः यस्मात् स निरग्निः, अक्रियश्च अनग्निसाधना अपि अविद्यमानाः क्रियाः तपोदानादिकाः यस्य असौ अक्रियः~॥~१~॥\par
 ननु च निरग्नेः अक्रियस्यैव श्रुतिस्मृतियोगशास्त्रेषु संन्यासित्वं योगित्वं च प्रसिद्धम्~। कथम् इह साग्नेः सक्रियस्य च संन्यासित्वं योगित्वं च अप्रसिद्धमुच्यते इति~। नैष दोषः, कयाचित् गुणवृत्त्या उभयस्य सम्पिपादयिषितत्वात्~। तत् कथम्~? कर्मफलसङ्कल्पसंन्यासात् संन्यासित्वम्~, योगाङ्गत्वेन च कर्मानुष्ठानात् कर्मफलसङ्कल्पस्य च चित्तविक्षेपहेतोः परित्यागात् योगित्वं च इति गौणमुभयम्~; न पुनः मुख्यं संन्यासित्वं योगित्वं च अभिप्रेतमित्येतमर्थं दर्शयितुमाह —} 
\begin{center}{\bfseries यं संन्यासमिति प्राहुर्योगं तं विद्धि पाण्डव~।\\न ह्यसंन्यस्तसङ्कल्पो योगी भवति कश्चन~॥~२~॥}\end{center} 
यं सर्वकर्मतत्फलपरित्यागलक्षणं परमार्थसंन्यासं संन्यासम् इति प्राहुः श्रुतिस्मृतिविदः, योगं कर्मानुष्ठानलक्षणं तं परमार्थसंन्यासं विद्धि जानीहि हे पाण्डव~। कर्मयोगस्य प्रवृत्तिलक्षणस्य तद्विपरीतेन निवृत्तिलक्षणेन परमार्थसंन्यासेन कीदृशं सामान्यमङ्गीकृत्य तद्भाव उच्यते इत्यपेक्षायाम् इदमुच्यते — अस्ति हि परमार्थसंन्यासेन सादृश्यं कर्तृद्वारकं कर्मयोगस्य~। यो हि परमार्थसंन्यासी स त्यक्तसर्वकर्मसाधनतया सर्वकर्मतत्फलविषयं सङ्कल्पं प्रवृत्तिहेतुकामकारणं संन्यस्यति~। अयमपि कर्मयोगी कर्म कुर्वाण एव फलविषयं सङ्कल्पं संन्यस्यति इत्येतमर्थं दर्शयिष्यन् आह — न हि यस्मात् असंन्यस्तसङ्कल्पः असंन्यस्तः अपरित्यक्तः फलविषयः सङ्कल्पः अभिसन्धिः येन सः असंन्यस्तसङ्कल्पः कश्चन कश्चिदपि कर्मी योगी समाधानवान् भवति न सम्भवतीत्यर्थः, फलसङ्कल्पस्य चित्तविक्षेपहेतुत्वात्~। तस्मात् यः कश्चन कर्मी संन्यस्तफलसङ्कल्पो भवेत् स योगी समाधानवान् अविक्षिप्तचित्तो भवेत्~, चित्तविक्षेपहेतोः फलसङ्कल्पस्य संन्यस्तत्वादित्यभिप्रायः~॥~२~॥\par
 एवं परमार्थसंन्यासकर्मयोगयोः कर्तृद्वारकं संन्याससामान्यमपेक्ष्य ‘यं संन्यासमिति प्राहुर्योगं तं विद्धि पाण्डव’ इति कर्मयोगस्य स्तुत्यर्थं संन्यासत्वम् उक्तम्~। ध्यानयोगस्य फलनिरपेक्षः कर्मयोगो बहिरङ्गं साधनमिति तं संन्यासत्वेन स्तुत्वा अधुना कर्मयोगस्य ध्यानयोगसाधनत्वं दर्शयति —} 
\begin{center}{\bfseries आरुरुक्षोर्मुनेर्योगं कर्म कारणमुच्यते~।\\योगारूढस्य तस्यैव शमः कारणमुच्यते~॥~३~॥}\end{center} 
आरुरुक्षोः आरोढुमिच्छतः, अनारूढस्य, ध्यानयोगे अवस्थातुमशक्तस्यैवेत्यर्थः~। कस्य तस्य आरुरुक्षोः~? मुनेः, कर्मफलसंन्यासिन इत्यर्थः~। किमारुरुक्षोः~? योगम्~। कर्म कारणं साधनम् उच्यते~। योगारूढस्य पुनः तस्यैव शमः उपशमः सर्वकर्मभ्यो निवृत्तिः कारणं योगारूढस्य साधनम् उच्यते इत्यर्थः~। यावद्यावत् कर्मभ्यः उपरमते, तावत्तावत् निरायासस्य जितेन्द्रियस्य चित्तं समाधीयते~। तथा सति स झटिति योगारूढो भवति~। तथा चोक्तं व्यासेन — ‘नैतादृशं ब्राह्मणस्यास्ति वित्तं यथैकता समता सत्यता च~। शीलं स्थितिर्दण्डनिधानमार्जवं ततस्ततश्चोपरमः क्रियाभ्यः’\footnote{मो. ध. १७५~। ३७} इति~॥~३~॥\par
 अथेदानीं कदा योगारूढो भवति इत्युच्यते —} 
\begin{center}{\bfseries यदा हि नेन्द्रियार्थेषु न कर्मस्वनुषज्जते~।\\सर्वसङ्कल्पसंन्यासी योगारूढस्तदोच्यते~॥~४~॥}\end{center} 
यदा समाधीयमानचित्तो योगी हि इन्द्रियार्थेषु इन्द्रियाणामर्थाः शब्दादयः तेषु इन्द्रियार्थेषु कर्मसु च नित्यनैमित्तिककाम्यप्रतिषिद्धेषु प्रयोजनाभावबुद्ध्या न अनुषज्जते अनुषङ्गं कर्तव्यताबुद्धिं न करोतीत्यर्थः~। सर्वसङ्कल्पसंन्यासी सर्वान् सङ्कल्पान् इहामुत्रार्थकामहेतून् संन्यसितुं शीलम् अस्य इति सर्वसङ्कल्पसंन्यासी, योगारूढः प्राप्तयोग इत्येतत्~, तदा तस्मिन् काले उच्यते~। ‘सर्वसङ्कल्पसंन्यासी’ इति वचनात् सर्वांश्च कामान् सर्वाणि च कर्माणि संन्यस्येदित्यर्थः~। सङ्कल्पमूला हि सर्वे कामाः — ‘सङ्कल्पमूलः कामो वै यज्ञाः सङ्कल्पसम्भवाः~। ’\footnote{मनु. २~। ३} ‘काम जानामि ते मूलं सङ्कल्पात्किल जायसे~। न त्वां सङ्कल्पयिष्यामि तेन मे न भविष्यसि’\footnote{मो. ध. १७७~। २५} इत्यादिस्मृतेः~। सर्वकामपरित्यागे च सर्वकर्मसंन्यासः सिद्धो भवति, ‘स यथाकामो भवति तत्क्रतुर्भवति यत्क्रतुर्भवति तत्कर्म कुरुते’\footnote{बृ. उ. ४~। ४~। ५} इत्यादिश्रुतिभ्यः~; ‘यद्यद्धि कुरुते जन्तुः तत्तत् कामस्य चेष्टितम्’\footnote{मनु. २~। ४} इत्यादिस्मृतिभ्यश्च~; न्यायाच्च — न हि सर्वसङ्कल्पसंन्यासे कश्चित् स्पन्दितुमपि शक्तः~। तस्मात् ‘सर्वसङ्कल्पसंन्यासी’ इति वचनात् सर्वान् कामान् सर्वाणि कर्माणि च त्याजयति भगवान्~॥~४~॥\par
 यदा एवं योगारूढः, तदा तेन आत्मा उद्धृतो भवति संसारादनर्थजातात्~। अतः —} 
\begin{center}{\bfseries उद्धरेदात्मनात्मानं नात्मानमवसादयेत्~।\\आत्मैव ह्यात्मनो बन्धुरात्मैव रिपुरात्मनः~॥~५~॥}\end{center} 
उद्धरेत् संसारसागरे निमग्नम् आत्मना आत्मानं ततः उत् ऊर्ध्वं हरेत् उद्धरेत्~, योगारूढतामापादयेदित्यर्थः~। न आत्मानम् अवसादयेत् न अधः नयेत्~, न अधः गमयेत्~। आत्मैव हि यस्मात् आत्मनः बन्धुः~। न हि अन्यः कश्चित् बन्धुः, यः संसारमुक्तये भवति~। बन्धुरपि तावत् मोक्षं प्रति प्रतिकूल एव, स्नेहादिबन्धनायतनत्वात्~। तस्मात् युक्तमवधारणम् ‘आत्मैव ह्यात्मनो बन्धुः’ इति~। आत्मैव रिपुः शत्रुः~। यः अन्यः अपकारी बाह्यः शत्रुः सोऽपि आत्मप्रयुक्त एवेति युक्तमेव अवधारणम् ‘आत्मैव रिपुरात्मनः’ इति~॥~५~॥\par
 आत्मैव बन्धुः आत्मैव रिपुः आत्मनः इत्युक्तम्~। तत्र किंलक्षण आत्मा आत्मनो बन्धुः, किंलक्षणो वा आत्मा आत्मनो रिपुः इत्युच्यते —} 
\begin{center}{\bfseries बन्धुरात्मात्मनस्तस्य येनात्मैवात्मना जितः~।\\अनात्मनस्तु शत्रुत्वे वर्तेतात्मैव शत्रुवत्~॥~६~॥}\end{center} 
बन्धुः आत्मा आत्मनः तस्य, तस्य आत्मनः स आत्मा बन्धुः येन आत्मना आत्मैव जितः, आत्मा कार्यकरणसङ्घातो येन वशीकृतः, जितेन्द्रिय इत्यर्थः~। अनात्मनस्तु अजितात्मनस्तु शत्रुत्वे शत्रुभावे वर्तेत आत्मैव शत्रुवत्~, यथा अनात्मा शत्रुः आत्मनः अपकारी, तथा आत्मा आत्मन अपकारे वर्तेत इत्यर्थः~॥~६~॥\par
 \begin{center}{\bfseries जितात्मनः प्रशान्तस्य परमात्मा समाहितः~।\\शीतोष्णसुखदुःखेषु तथा मानापमानयोः~॥~७~॥}\end{center} 
जितात्मनः कार्यकरणसङ्घात आत्मा जितो येन सः जितात्मा तस्य जितात्मनः, प्रशान्तस्य प्रसन्नान्तःकरणस्य सतः संन्यासिनः परमात्मा समाहितः साक्षादात्मभावेन वर्तते इत्यर्थः~। किञ्च शीतोष्णसुखदुःखेषु तथा माने अपमाने च मानापमानयोः पूजापरिभवयोः समः स्यात्~॥~७~॥\par
 \begin{center}{\bfseries ज्ञानविज्ञानतृप्तात्मा कूटस्थो विजितेन्द्रियः~।\\युक्त इत्युच्यते योगी समलोष्टाश्मकाञ्चनः~॥~८~॥}\end{center} 
ज्ञानविज्ञानतृप्तात्मा ज्ञानं शास्त्रोक्तपदार्थानां परिज्ञानम्~, विज्ञानं तु शास्त्रतो ज्ञातानां तथैव स्वानुभवकरणम्~, ताभ्यां ज्ञानविज्ञानाभ्यां तृप्तः सञ्जातालंप्रत्ययः आत्मा अन्तःकरणं यस्य सः ज्ञानविज्ञानतृप्तात्मा, कूटस्थः अप्रकम्प्यः, भवति इत्यर्थः~; विजितेन्द्रियश्च~। य ईदृशः, युक्तः समाहितः इति स उच्यते कथ्यते~। स योगी समलोष्टाश्मकाञ्चनः लोष्टाश्मकाञ्चनानि समानि यस्य सः समलोष्टाश्मकाञ्चनः~॥~८~॥\par
 किञ्च —} 
\begin{center}{\bfseries सुहृन्मित्रार्युदासीनमध्यस्थद्वेष्यबन्धुषु~।\\साधुष्वपि च पापेषु समबुद्धिर्विशिष्यते~॥~९~॥}\end{center} 
‘सुहृत्’ इत्यादिश्लोकार्धम् एकं पदम्~। सुहृत् इति प्रत्युपकारमनपेक्ष्य उपकर्ता, मित्रं स्नेहवान्~, अरिः शत्रुः, उदासीनः न कस्यचित् पक्षं भजते, मध्यस्थः यो विरुद्धयोः उभयोः हितैषी, द्वेष्यः आत्मनः अप्रियः, बन्धुः सम्बन्धी इत्येतेषु साधुषु शास्त्रानुवर्तिषु अपि च पापेषु प्रतिषिद्धकारिषु सर्वेषु एतेषु समबुद्धिः ‘कः किङ्कर्मा’ इत्यव्यापृतबुद्धिरित्यर्थः~। विशिष्यते, ‘विमुच्यते’ इति वा पाठान्तरम्~। योगारूढानां सर्वेषाम् अयम् उत्तम इत्यर्थः~॥~९~॥\par
 अत एवमुत्तमफलप्राप्तये —} 
\begin{center}{\bfseries योगी युञ्जीत सततमात्मानं रहसि स्थितः~।\\एकाकी यतचित्तात्मा निराशीरपरिग्रहः~॥~१०~॥}\end{center} 
योगी ध्यायी युञ्जीत समादध्यात् सततं सर्वदा आत्मानम् अन्तःकरणं रहसि एकान्ते गिरिगुहादौ स्थितः सन् एकाकी असहायः~। ‘रहसि स्थितः एकाकी च’ इति विशेषणात् संन्यासं कृत्वा इत्यर्थः~। यतचित्तात्मा चित्तम् अन्तःकरणम् आत्मा देहश्च संयतौ यस्य सः यतचित्तात्मा, निराशीः वीततृष्णः अपरिग्रहः परिग्रहरहितश्चेत्यर्थः~। संन्यासित्वेऽपि त्यक्तसर्वपरिग्रहः सन् युञ्जीत इत्यर्थः~॥~१०~॥\par
 अथेदानीं योगं युञ्जतः आसनाहारविहारादीनां योगसाधनत्वेन नियमो वक्तव्यः, प्राप्तयोगस्य लक्षणं तत्फलादि च, इत्यत आरभ्यते~। तत्र आसनमेव तावत् प्रथममुच्यते —} 
\begin{center}{\bfseries शुचौ देशे प्रतिष्ठाप्य स्थिरमासनमात्मनः~।\\नात्युच्छ्रितं नातिनीचं चैलाजिनकुशोत्तरम्~॥~११~॥}\end{center} 
शुचौ शुद्धे विविक्ते स्वभावतः संस्कारतो वा, देशे स्थाने प्रतिष्ठाप्य स्थिरम् अचलम् आत्मनः आसनं नात्युच्छ्रितं नातीव उच्छ्रितं न अपि अतिनीचम्~, तच्च चैलाजिनकुशोत्तरं चैलम् अजिनं कुशाश्च उत्तरे यस्मिन् आसने तत् आसनं चैलाजिनकुशोत्तरम्~। पाठक्रमाद्विपरीतः अत्र क्रमः चैलादीनाम्~॥~११~॥\par
 प्रतिष्ठाप्य, किम्~? —} 
\begin{center}{\bfseries तत्रैकाग्रं मनः कृत्वा यतचित्तेन्द्रियक्रियः~।\\उपविश्यासने युञ्ज्याद्योगमात्मविशुद्धये~॥~१२~॥}\end{center} 
तत्र तस्मिन् आसने उपविश्य योगं युञ्ज्यात्~। कथम्~? सर्वविषयेभ्यः उपसंहृत्य एकाग्रं मनः कृत्वा यतचित्तेन्द्रियक्रियः चित्तं च इन्द्रियाणि च चित्तेन्द्रियाणि तेषां क्रियाः संयता यस्य सः यतचित्तेन्द्रियक्रियः~। स किमर्थं योगं युञ्ज्यात् इत्याह — आत्मविशुद्धये अन्तःकरणस्य विशुद्ध्यर्थमित्येतत्~॥~१२~॥\par
 बाह्यमासनमुक्तम्~; अधुना शरीरधारणं कथम् इत्युच्यते —} 
\begin{center}{\bfseries समं कायशिरोग्रीवं धारयन्नचलं स्थिरः~।\\सम्प्रेक्ष्य नासिकाग्रं स्वं दिशश्चानवलोकयन्~॥~१३~॥}\end{center} 
समं कायशिरोग्रीवं कायश्च शिरश्च ग्रीवा च कायशिरोग्रीवं तत् समं धारयन् अचलं च~। समं धारयतः चलनं सम्भवति~; अतः विशिनष्टि — अचलमिति~। स्थिरः स्थिरो भूत्वा इत्यर्थः~। स्वं नासिकाग्रं सम्प्रेक्ष्य सम्यक् प्रेक्षणं दर्शनं कृत्वेव इति~। इवशब्दो लुप्तो द्रष्टव्यः~। न हि स्वनासिकाग्रसम्प्रेक्षणमिह विधित्सितम्~। किं तर्हि~? चक्षुषो दृष्टिसंनिपातः~। स च अन्तःकरणसमाधानापेक्षो विवक्षितः~। स्वनासिकाग्रसम्प्रेक्षणमेव चेत् विवक्षितम्~, मनः तत्रैव समाधीयेत, नात्मनि~। आत्मनि हि मनसः समाधानं वक्ष्यति ‘आत्मसंस्थं मनः कृत्वा’\footnote{भ. गी. ६~। २५} इति~। तस्मात् इवशब्दलोपेन अक्ष्णोः दृष्टिसंनिपात एव ‘सम्प्रेक्ष्य’ इत्युच्यते~। दिशश्च अनवलोकयन् दिशां च अवलोकनमन्तराकुर्वन् इत्येतत्~॥~१३~॥\par
 किञ्च —} 
\begin{center}{\bfseries प्रशान्तात्मा विगतभीर्ब्रह्मचारिव्रते स्थितः~।\\मनः संयम्य मच्चित्तो युक्त आसीत मत्परः~॥~१४~॥}\end{center} 
प्रशान्तात्मा प्रकर्षेण शान्तः आत्मा अन्तःकरणं यस्य सोऽयं प्रशान्तात्मा, विगतभीः विगतभयः, ब्रह्मचारिव्रते स्थितः ब्रह्मचारिणो व्रतं ब्रह्मचर्यं गुरुशुश्रूषाभिक्षान्नभुक्त्यादि तस्मिन् स्थितः, तदनुष्ठाता भवेदित्यर्थः~। किञ्च, मनः संयम्य मनसः वृत्तीः उपसंहृत्य इत्येतत्~, मच्चित्तः मयि परमेश्वरे चित्तं यस्य सोऽयं मच्चित्तः, युक्तः समाहितः सन् आसीत उपविशेत्~। मत्परः अहं परो यस्य सोऽयं मत्परो भवति~। कश्चित् रागी स्त्रीचित्तः, न तु स्त्रियमेव परत्वेन गृह्णाति~; किं तर्हि~? राजानं महादेवं वा~। अयं तु मच्चित्तो मत्परश्च~॥~१४~॥\par
 अथेदानीं योगफलमुच्यते —} 
\begin{center}{\bfseries युञ्जन्नेवं सदात्मानं योगी नियतमानसः~।\\शान्तिं निर्वाणपरमां मत्संस्थामधिगच्छति~॥~१५~॥}\end{center} 
युञ्जन् समाधानं कुर्वन् एवं यतोक्तेन विधानेन सदा आत्मानं सर्वदा योगी नियतमानसः नियतं संयतं मानसं मनो यस्य सोऽयं नियतमानसः, शान्तिम् उपरतिं निर्वाणपरमां निर्वाणं मोक्षः तत् परमा निष्ठा यस्याः शान्तेः सा निर्वाणपरमा तां निर्वाणपरमाम्~, मत्संस्थां मदधीनाम् अधिगच्छति प्राप्नोति~॥~१५~॥\par
 इदानीं योगिनः आहारादिनियम उच्यते —} 
\begin{center}{\bfseries नात्यश्नतस्तु योगोऽस्ति न चैकान्तमनश्नतः~।\\न चातिस्वप्नशीलस्य जाग्रतो नैव चार्जुन~॥~१६~॥}\end{center} 
न अत्यश्नतः आत्मसंमितमन्नपरिमाणमतीत्याश्नतः अत्यश्नतः न योगः अस्ति~। न च एकान्तम् अनश्नतः योगः अस्ति~। ‘यदु ह वा आत्मसंमितमन्नं तदवति तन्न हिनस्ति यद्भूयो हिनस्ति तद्यत् कनीयोऽन्नं न तदवति’\footnote{श. ब्रा.~? } इति श्रुतेः~। तस्मात् योगी न आत्मसंमितात् अन्नात् अधिकं न्यूनं वा अश्नीयात्~। अथवा, योगिनः योगशास्त्रे परिपठीतात् अन्नपरिमाणात् अतिमात्रमश्नतः योगो नास्ति~। उक्तं हि — ‘अर्धं सव्यञ्जनान्नस्य तृतीयमुदकस्य च~। वायोः सञ्चरणार्थं तु चतुर्थमवशेषयेत्’\footnote{~? } इत्यादिपरिमाणम्~। तथा — न च अतिस्वप्नशीलस्य योगो भवति नैव च अतिमात्रं जाग्रतो भवति च अर्जुन~॥~१६~॥\par
 कथं पुनः योगो भवति इत्युच्यते —} 
\begin{center}{\bfseries युक्ताहारविहारस्य युक्तचेष्टस्य कर्मसु~।\\युक्तस्वप्नावबोधस्य योगो भवति दुःखहा~॥~१७~॥}\end{center} 
युक्ताहारविहारस्य आह्रियते इति आहारः अन्नम्~, विहरणं विहारः पादक्रमः, तौ युक्तौ नियतपरिमाणौ यस्य सः युक्ताहारविहारः तस्य, तथा युक्तचेष्टस्य युक्ता नियता चेष्टा यस्य कर्मसु तस्य, तथा युक्तस्वप्नावबोधस्य युक्तौ स्वप्नश्च अवबोधश्च तौ नियतकालौ यस्य तस्य, युक्ताहारविहारस्य युक्तचेष्टस्य कर्मसु युक्तस्वप्नावबोधस्य योगिनो योगो भवति दुःखहा दुःखानि सर्वाणि हन्तीति दुःखहा, सर्वसंसारदुःखक्षयकृत् योगः भवतीत्यर्थः~॥~१७~॥\par
 अथ अधुना कदा युक्तो भवति इत्युच्यते —} 
\begin{center}{\bfseries यदा विनियतं चित्तमात्मन्येवावतिष्ठते~।\\निःस्पृहः सर्वकामेभ्यो युक्त इत्युच्यते तदा~॥~१८~॥}\end{center} 
यदा विनियतं विशेषेण नियतं संयतम् एकाग्रतामापन्नं चित्तं हित्वा बाह्यार्थचिन्ताम् आत्मन्येव केवले अवतिष्ठते, स्वात्मनि स्थितिं लभते इत्यर्थः~। निःस्पृहः सर्वकामेभ्यः निर्गता दृष्टादृष्टविषयेभ्यः स्पृहा तृष्णा यस्य योगिनः सः युक्तः समाहितः इत्युच्यते तदा तस्मिन्काले~॥~१८~॥\par
 तस्य योगिनः समाहितं यत् चित्तं तस्योपमा उच्यते —} 
\begin{center}{\bfseries यदा दीपो निवातस्थो नेङ्गते सोपमा स्मृता~।\\योगिनो यतचित्तस्य युञ्जतो योगमात्मनः~॥~१९~॥}\end{center} 
यथा दीपः प्रदीपः निवातस्थः निवाते वातवर्जिते देशे स्थितः न इङ्गते न चलति, सा उपमा उपमीयते अनया इत्युपमा योगज्ञैः चित्तप्रचारदर्शिभिः स्मृता चिन्तिता योगिनो यतचित्तस्य संयतान्तःकरणस्य युञ्जतो योगम् अनुतिष्ठतः आत्मनः समाधिमनुतिष्ठत इत्यर्थः~॥~१९~॥\par
 एवं योगाभ्यासबलादेकाग्रीभूतं निवातप्रदीपकल्पं सत् —} 
\begin{center}{\bfseries यत्रोपरमते चित्तं निरुद्धं योगसेवया~।\\यत्र चैवात्मनात्मानं पश्यन्नात्मनि तुष्यति~॥~२०~॥}\end{center} 
यत्र यस्मिन् काले उपरमते चित्तम् उपरतिं गच्छति निरुद्धं सर्वतो निवारितप्रचारं योगसेवया योगानुष्ठानेन, यत्र चैव यस्मिंश्च काले आत्मना समाधिपरिशुद्धेन अन्तःकरणेन आत्मानं परं चैतन्यं ज्योतिःस्वरूपं पश्यन् उपलभमानः स्वे एव आत्मनि तुष्यति तुष्टिं भजते~॥~२०~॥\par
 किञ्च —} 
\begin{center}{\bfseries सुखमात्यन्तिकं यत्तद्बुद्धिग्राह्यमतीन्द्रियम्~।\\वेत्ति यत्र न चैवायं स्थितश्चलति तत्त्वतः~॥~२१~॥}\end{center} 
सुखम् आत्यन्तिकं अत्यन्तमेव भवति इत्यात्यन्तिकम् अनन्तमित्यर्थः, यत् तत् बुद्धिग्राह्यं बुद्ध्यैव इन्द्रियनिरपेक्षया गृह्यते इति बुद्धिग्राह्यम् अतीन्द्रियम् इन्द्रियगोचरातीतम् अविषयजनितमित्यर्थः, वेत्ति तत् ईदृशं सुखमनुभवति यत्र यस्मिन् काले, न च एव अयं विद्वान् आत्मस्वरूपे स्थितः तस्मात् नैव चलति तत्त्वतः तत्त्वस्वरूपात् न प्रच्यवते इत्यर्थः~॥~२१~॥\par
 किञ्च —} 
\begin{center}{\bfseries यं लब्ध्वा चापरं लाभं मन्यते नाधिकं ततः~।\\यस्मिन्स्थितो न दुःखेन गुरुणापि विचाल्यते~॥~२२~॥}\end{center} 
यं लब्ध्वा यम् आत्मलाभं लब्ध्वा प्राप्य च अपरम् अन्यत् लाभं लाभान्तरं ततः अधिकम् अस्तीति न मन्यते न चिन्तयति~। किञ्च, यस्मिन् आत्मतत्त्वे स्थितः दुःखेन शस्त्रनिपातादिलक्षणेन गुरुणा महता अपि न विचाल्यते~॥~२२~॥\par
 ‘यत्रोपरमते’\footnote{भ. गी. ६~। २०} इत्याद्यारभ्य यावद्भिः विशेषणैः विशिष्ट आत्मावस्थाविशेषः योगः उक्तः —} 
\begin{center}{\bfseries तं विद्याद्दुःखसंयोगवियोगं योगसंज्ञितम्~।\\स निश्चयेन योक्तव्यो योगोऽनिर्विण्णचेतसा~॥~२३~॥}\end{center} 
तं विद्यात् विजानीयात् दुःखसंयोगवियोगं दुःखैः संयोगः दुःखसंयोगः, तेन वियोगः दुःखसंयोगवियोगः, तं दुःखसंयोगवियोगं योग इत्येव संज्ञितं विपरीतलक्षणेन विद्यात् विजानीयादित्यर्थः~। योगफलमुपसंहृत्य पुनरन्वारम्भेण योगस्य कर्तव्यता उच्यते निश्चयानिर्वेदयोः योगसाधनत्वविधानार्थम्~। स यथोक्तफलो योगः निश्चयेन अध्यवसायेन योक्तव्यः अनिर्विण्णचेतसा न निर्विण्णम् अनिर्विण्णम्~। किं तत्~? चेतः तेन निर्वेदरहितेन चेतसा चित्तेनेत्यर्थः~॥~२३~॥\par
 किञ्च —} 
\begin{center}{\bfseries सङ्कल्पप्रभवान्कामांस्त्यक्त्वा सर्वानशेषतः~।\\मनसैवेन्द्रियग्रामं विनियम्य समन्ततः~॥~२४~॥}\end{center} 
सङ्कल्पप्रभवान् सङ्कल्पः प्रभवः येषां कामानां ते सङ्कल्पप्रभवाः कामाः तान् त्यक्त्वा परित्यज्य सर्वान् अशेषतः निर्लेपेन~। किञ्च, मनसैव विवेकयुक्तेन इन्द्रियग्रामम् इन्द्रियसमुदायं विनियम्य नियमनं कृत्वा समन्ततः समन्तात्~॥~२४~॥\par
 \begin{center}{\bfseries शनैः शनैरुपरमेद्बुद्ध्या धृतिगृहीतया~।\\आत्मसंस्थं मनः कृत्वा न किञ्चिदपि चिन्तयेत्~॥~२५~॥}\end{center} 
शनैः शनैः न सहसा उपरमेत् उपरतिं कुर्यात्~। कया~? बुद्ध्या~। किंविशिष्टया~? धृतिगृहीतया धृत्या धैर्येण गृहीतया धृतिगृहीतया धैर्येण युक्तया इत्यर्थः~। आत्मसंस्थम् आत्मनि संस्थितम् ‘आत्मैव सर्वं न ततोऽन्यत् किञ्चिदस्ति’ इत्येवमात्मसंस्थं मनः कृत्वा न किञ्चिदपि चिन्तयेत्~। एष योगस्य परमो विधिः~॥~२५~॥\par
 तत्र एवमात्मसंस्थं मनः कर्तुं प्रवृत्तो योगी —} 
\begin{center}{\bfseries यतो यतो निश्चरति मनश्चञ्चलमस्थिरम्~।\\ततस्ततो नियम्यैतदात्मन्येव वशं नयेत्~॥~२६~॥}\end{center} 
यतो यतः यस्माद्यस्मात् निमित्तात् शब्दादेः निश्चरति निर्गच्छति स्वभावदोषात् मनः चञ्चलम् अत्यर्थं चलम्~, अत एव अस्थिरम्~, ततस्ततः तस्मात्तस्मात् शब्दादेः निमित्तात् नियम्य तत्तन्निमित्तं याथात्म्यनिरूपणेन आभासीकृत्य वैराग्यभावनया च एतत् मनः आत्मन्येव वशं नयेत् आत्मवश्यतामापादयेत्~। एवं योगाभ्यासबलात् योगिनः आत्मन्येव प्रशाम्यति मनः~॥~२६~॥\par
 \begin{center}{\bfseries प्रशान्तमनसं ह्येनं योगिनं सुखमुत्तमम्~।\\उपैति शान्तरजसं ब्रह्मभूतमकल्मषम्~॥~२७~॥}\end{center} 
प्रशान्तमनसं प्रकर्षेण शान्तं मनः यस्य सः प्रशान्तमनाः तं प्रशान्तमनसं हि एनं योगिनं सुखम् उत्तमं निरतिशयम् उपैति उपगच्छति शान्तरजसं प्रक्षीणमोहादिक्लेशरजसमित्यर्थः, ब्रह्मभूतं जीवन्मुक्तम्~, ‘ब्रह्मैव सर्वम्’ इत्येवं निश्चयवन्तं ब्रह्मभूतम् अकल्मषं धर्माधर्मादिवर्जितम्~॥~२७~॥\par
 \begin{center}{\bfseries युञ्जन्नेवं सदात्मानं योगी विगतकल्मषः~।\\सुखेन ब्रह्मसंस्पर्शमत्यन्तं सुखमश्नुते~॥~२८~॥}\end{center} 
युञ्जन् एवं यथोक्तेन क्रमेण योगी योगान्तरायवर्जितः सदा सर्वदा आत्मानं विगतकल्मषः विगतपापः, सुखेन अनायासेन ब्रह्मसंस्पर्शं ब्रह्मणा परेण संस्पर्शो यस्य तत् ब्रह्मसंस्पर्शं सुखम् अत्यन्तम् अन्तमतीत्य वर्तत इत्यत्यन्तम् उत्कृष्टं निरतिशयम् अश्नुते व्याप्नोति~॥~२८~॥\par
 इदानीं योगस्य यत् फलं ब्रह्मैकत्वदर्शनं सर्वसंसारविच्छेदकारणं तत् प्रदर्श्यते —} 
\begin{center}{\bfseries सर्वभूतस्थमात्मानं सर्वभूतानि चात्मनि~।\\ईक्षते योगयुक्तात्मा सर्वत्र समदर्शनः~॥~२९~॥}\end{center} 
सर्वभूतस्थं सर्वेषु भूतेषु स्थितं स्वम् आत्मानं सर्वभूतानि च आत्मनि ब्रह्मादीनि स्तम्बपर्यन्तानि च सर्वभूतानि आत्मनि एकतां गतानि ईक्षते पश्यति योगयुक्तात्मा समाहितान्तःकरणः सर्वत्र समदर्शनः सर्वेषु ब्रह्मादिस्थावरान्तेषु विषमेषु सर्वभूतेषु समं निर्विशेषं ब्रह्मात्मैकत्वविषयं दर्शनं ज्ञानं यस्य स सर्वत्र समदर्शनः~॥~२९~॥\par
 एतस्य आत्मैकत्वदर्शनस्य फलम् उच्यते —} 
\begin{center}{\bfseries यो मां पश्यति सर्वत्र सर्वं च मयि पश्यति~।\\तस्याहं न प्रणश्यामि स च मे न प्रणश्यति~॥~३०~॥}\end{center} 
यो मां पश्यति वासुदेवं सर्वस्य आत्मानं सर्वत्र सर्वेषु भूतेषु सर्वं च ब्रह्मादिभूतजातं मयि सर्वात्मनि पश्यति, तस्य एवं आत्मैकत्वदर्शिनः अहम् ईश्वरो न प्रणश्यामि न परोक्षतां गमिष्यामि~। स च मे न प्रणश्यति स च विद्वान् मम वासुदेवस्य न प्रणश्यति न परोक्षो भवति, तस्य च मम च एकात्मकत्वात्~; स्वात्मा हि नाम आत्मनः प्रिय एव भवति, यस्माच्च अहमेव सर्वात्मैकत्वदर्शी~॥~३०~॥\par
 इत्येतत् पूर्वश्लोकार्थं सम्यग्दर्शनमनूद्य तत्फलं मोक्षः अभिधीयते —} 
\begin{center}{\bfseries सर्वभूतस्थितं यो मां भजत्येकत्वमास्थितः~।\\सर्वथा वर्तमानोऽपि स योगी मयि वर्तते~॥~३१~॥}\end{center} 
सर्वथा सर्वप्रकारैः वर्तमानोऽपि सम्यग्दर्शी योगी मयि वैष्णवे परमे पदे वर्तते, नित्यमुक्त एव सः, न मोक्षं प्रति केनचित् प्रतिबध्यते इत्यर्थः~॥~३१~॥\par
 किञ्च अन्यत् —} 
\begin{center}{\bfseries आत्मौपम्येन सर्वत्र समं पश्यति योऽर्जुन~।\\शुखं वा यदि वा दुःखं स योगी परमो मतः~॥~३२~॥}\end{center} 
आत्मौपम्येन आत्मा स्वयमेव उपमीयते अनया इत्युपमा तस्या उपमाया भावः औपम्यं तेन आत्मौपम्येन, सर्वत्र सर्वभूतेषु समं तुल्यं पश्यति यः अर्जुन, स च किं समं पश्यति इत्युच्यते — यथा मम सुखम् इष्टं तथा सर्वप्राणिनां सुखम् अनुकूलम्~। वाशब्दः चार्थे~। यदि वा यच्च दुःखं मम प्रतिकूलम् अनिष्टं यथा तथा सर्वप्राणिनां दुःखम् अनिष्टं प्रतिकूलं इत्येवम् आत्मौपम्येन सुखदुःखे अनुकूलप्रतिकूले तुल्यतया सर्वभूतेषु समं पश्यति, न कस्यचित् प्रतिकूलमाचरति, अहिंसक इत्यर्थः~। यः एवमहिंसकः सम्यग्दर्शननिष्ठः स योगी परमः उत्कृष्टः मतः अभिप्रेतः सर्वयोगिनां मध्ये~॥~३२~॥\par
 एतस्य यथोक्तस्य सम्यग्दर्शनलक्षणस्य योगस्य दुःखसम्पाद्यतामालक्ष्य शुश्रूषुः ध्रुवं तत्प्राप्त्युपायमर्जुन उवाच —}\\ 
\begin{center}{\bfseries अर्जुन उवाच —\\ योऽयं योगस्त्वया प्रोक्तः साम्येन मधुसूदन~।\\एतस्याहं न पश्यामि चञ्चलत्वात्स्थितिं स्थिराम्~॥~३३~॥}\end{center} 
यः अयं योगः त्वया प्रोक्तः साम्येन समत्वेन हे मधुसूदन एतस्य योगस्य अहं न पश्यामि नोपलभे, चञ्चलत्वात् मनसः~। किम्~? स्थिराम् अचलां स्थितिम्~॥~३३~॥\par
 प्रसिद्धमेतत् —} 
\begin{center}{\bfseries चञ्चलं हि मनः कृष्ण प्रमाथि बलवद्दृढम्~।\\तस्याहं निग्रहं मन्ये वायोरिव सुदुष्करम्~॥~३४~॥}\end{center} 
चञ्चलं हि मनः~। कृष्ण इति कृषतेः विलेखनार्थस्य रूपम्~। भक्तजनपापादिदोषाकर्षणात् कृष्णः, तस्य सम्बुद्धिः हे कृष्ण~। हि यस्मात् मनः चञ्चलं न केवलमत्यर्थं चञ्चलम्~, प्रमाथि च प्रमथनशीलम्~, प्रमथ्नाति शरीरम् इन्द्रियाणि च विक्षिपत् सत् परवशीकरोति~। किञ्च — बलवत् प्रबलम्~, न केनचित् नियन्तुं शक्यम्~, दुर्निवारत्वात्~। किञ्च — दृढं तन्तुनागवत् अच्छेद्यम्~। तस्य एवंभूतस्य मनसः अहं निग्रहं निरोधं मन्ये वायोरिव यथा वायोः दुष्करो निग्रहः ततोऽपि दुष्करं मन्ये इत्यभिप्रायः~॥~३४~॥\par
 श्रीभगवानुवाच, एवम् एतत् यथा ब्रवीषि —} 
\begin{center}{\bfseries श्रीभगवानुवाच —\\ असंशयं महाबाहो मनो दुर्निग्रहं चलम्~।\\अभ्यासेन तु कौन्तेय वैराग्येण च गृह्यते~॥~३५~॥}\end{center} 
असंशयं नास्ति संशयः ‘मनो दुर्निग्रहं चलम्’ इत्यत्र हे महाबाहो~। किन्तु अभ्यासेन तु अभ्यासो नाम चित्तभूमौ कस्याञ्चित् समानप्रत्ययावृत्तिः चित्तस्य~। वैराग्येण वैराग्यं नाम दृष्टादृष्टेष्टभोगेषु दोषदर्शनाभ्यासात् वैतृष्ण्यम्~। तेन च वैराग्येण गृह्यते विक्षेपरूपः प्रचारः चित्तस्य~। एवं तत् मनः गृह्यते निगृह्यते निरुध्यते इत्यर्थः~॥~३५~॥\par
 यः पुनः असंयतात्मा, तेन —} 
\begin{center}{\bfseries असंयतात्मना योगो दुष्प्राप इति मे मतिः~।\\वश्यात्मना तु यतता शक्योऽवाप्तुमुपायतः~॥~३६~॥}\end{center} 
असंयतात्मना अभ्यासवैराग्याभ्यामसंयतः आत्मा अन्तःकरणं यस्य सोऽयम् असंयतात्मा तेन असंयतात्मना योगो दुष्प्रापः दुःखेन प्राप्यत इति मे मतिः~। यस्तु पुनः वश्यात्मा अभ्यासवैराग्याभ्यां वश्यत्वमापादितः आत्मा मनः यस्य सोऽयं वश्यात्मा तेन वश्यात्मना तु यतता भूयोऽपि प्रयत्नं कुर्वता शक्यः अवाप्तुं योगः उपायतः यथोक्तादुपायात्~॥~३६~॥\par
 तत्र योगाभ्यासाङ्गीकरणेन इहलोकपरलोकप्राप्तिनिमित्तानि कर्माणि संन्यस्तानि, योगसिद्धिफलं च मोक्षसाधनं सम्यग्दर्शनं न प्राप्तमिति, योगी योगमार्गात् मरणकाले चलितचित्तः इति तस्य नाशमशङ्क्य अर्जुन उवाच —}\\ 
\begin{center}{\bfseries अर्जुन उवाच —\\ अयतिः श्रद्धयोपेतो योगाच्चलितमानसः~।\\अप्राप्य योगसंसिद्धिं कां गतिं कृष्ण गच्छति~॥~३७~॥}\end{center} 
अयतिः अप्रयत्नवान् योगमार्गे श्रद्धया आस्तिक्यबुद्ध्या च उपेतः योगात् अन्तकाले च चलितं मानसं मनो यस्य सः चलितमानसः भ्रष्टस्मृतिः सः अप्राप्य योगसंसिद्धिं योगफलं सम्यग्दर्शनं कां गतिं हे कृष्ण गच्छति~॥~३७~॥\par
 \begin{center}{\bfseries कच्चिन्नोभयविभ्रष्टश्छिन्नाभ्रमिव नश्यति~।\\अप्रतिष्ठो महाबाहो विमूढो ब्रह्मणः पथि~॥~३८~॥}\end{center} 
कच्चित् किं न उभयविभ्रष्टः कर्ममार्गात् योगमार्गाच्च विभ्रष्टः सन्  छिन्नाभ्रमिव नश्यति, किं वा न नश्यति अप्रतिष्ठो निराश्रयः हे महाबाहो विमूढः सन् ब्रह्मणः पथि ब्रह्मप्राप्तिमार्गे~॥~३८~॥\par
 \begin{center}{\bfseries एतन्मे संशयं कृष्ण च्छेत्तुमर्हस्यशेषतः~।\\त्वदन्यः संशयस्यास्य च्छेत्ता न ह्युपपद्यते~॥~३९~॥}\end{center} 
एतत् मे मम संशयं कृष्ण च्छेत्तुम् अपनेतुम् अर्हसि अशेषतः~। त्वदन्यः त्वत्तः अन्यः ऋषिः देवो वा च्छेत्ता नाशयिता संशयस्य अस्य न हि यस्मात् उपपद्यते न सम्भवति~। अतः त्वमेव च्छेत्तुमर्हसि इत्यर्थः~॥~३९~॥\par
 {\bfseries श्रीभगवानुवाच —}\\
\begin{center}{\bfseries पार्थ नैवेह नामुत्र विनाशस्तस्य विद्यते~।\\न हि कल्याणकृत्कश्चिद्दुर्गतिं तात गच्छति~॥~४०~॥}\end{center} 
हे पार्थ नैव इह लोके नामुत्र परस्मिन् वा लोके विनाशः तस्य विद्यते नास्ति~। नाशो नाम पूर्वस्मात् हीनजन्मप्राप्तिः स योगभ्रष्टस्य नास्ति~। न हि यस्मात् कल्याणकृत् शुभकृत् कश्चित् दुर्गतिं कुत्सितां गतिं हे तात, तनोति आत्मानं पुत्ररूपेणेति पिता तात उच्यते~। पितैव पुत्र इति पुत्रोऽपि तात उच्यते~। शिष्योऽपि पुत्र उच्यते~। यतो न गच्छति~॥~४०~॥\par
 किं तु अस्य भवति~? —} 
\begin{center}{\bfseries प्राप्य पुण्यकृतां लोकानुषित्वा शाश्वतीः समाः~।\\शुचीनां श्रीमतां गेहे योगभ्रष्टोऽभिजायते~॥~४१~॥}\end{center} 
योगमार्गे प्रवृत्तः संन्यासी सामर्थ्यात् प्राप्य गत्वा पुण्यकृताम् अश्वमेधादियाजिनां लोकान्~, तत्र च उषित्वा वासमनुभूय शाश्वतीः नित्याः समाः संवत्सरान्~, तद्भोगक्षये शुचीनां यथोक्तकारिणां श्रीमतां विभूतिमतां गेहे गृहे योगभ्रष्टः अभिजायते~॥~४१~॥\par
 \begin{center}{\bfseries अथवा योगिनामेव कुले भवति धीमताम्~।\\एतद्धि दुर्लभतरं लोके जन्म यदीदृशम्~॥~४२~॥}\end{center} 
अथवा श्रीमतां कुलात् अन्यस्मिन् योगिनामेव दरिद्राणां कुले भवति जायते धीमतां बुद्धिमताम्~। एतत् हि जन्म, यत् दरिद्राणां योगिनां कुले, दुर्लभतरं दुःखलभ्यतरं पूर्वमपेक्ष्य लोके जन्म यत् ईदृशं यथोक्तविशेषणे कुले~॥~४२~॥\par
 यस्मात् —} 
\begin{center}{\bfseries तत्र तं बुद्धिसंयोगं लभते पौर्वदेहिकम्~।\\यतते च ततो भूयः संसिद्धौ कुरुनन्दन~॥~४३~॥}\end{center} 
तत्र योगिनां कुले तं बुद्धिसंयोगं बुद्ध्या संयोगं बुद्धिसंयोगं लभते पौर्वदेहिकं पूर्वस्मिन् देहे भवं पौर्वदेहिकम्~। यतते च प्रयत्नं च करोति ततः तस्मात् पूर्वकृतात् संस्कारात् भूयः बहुतरं संसिद्धौ संसिद्धिनिमित्तं हे कुरुनन्दन~॥~४३~॥\par
 कथं पूर्वदेहबुद्धिसंयोग इति तदुच्यते —} 
\begin{center}{\bfseries पूर्वाभ्यासेन तेनैव ह्रियते ह्यवशोऽपि सः~।\\जिज्ञासुरपि योगस्य शब्दब्रह्मातिवर्तते~॥~४४~॥}\end{center} 
यः पूर्वजन्मनि कृतः अभ्यासः सः पूर्वाभ्यासः, तेनैव बलवता ह्रियते संसिद्धौ हि यस्मात् अवशोऽपि सः योगभ्रष्टः~; न कृतं चेत् योगाभ्यासजात् संस्कारात् बलवत्तरमधर्मादिलक्षणं कर्म, तदा योगाभ्यासजनितेन संस्कारेण ह्रियते~; अधर्मश्चेत् बलवत्तरः कृतः, तेन योगजोऽपि संस्कारः अभिभूयत एव, तत्क्षये तु योगजः संस्कारः स्वयमेव कार्यमारभते, न दीर्घकालस्थस्यापि विनाशः तस्य अस्ति इत्यर्थः~। अतः जिज्ञासुरपि योगस्य स्वरूपं ज्ञातुमिच्छन् अपि योगमार्गे प्रवृत्तः संन्यासी योगभ्रष्टः, सामर्थ्यात् सोऽपि शब्दब्रह्म वेदोक्तकर्मानुष्ठानफलम् अतिवर्तते अतिक्रामति अपाकरिष्यति~; किमुत बुद्ध्वा यः योगं तन्निष्ठः अभ्यासं कुर्यात्~॥~४४~॥\par
 कुतश्च योगित्वं श्रेयः इति —} 
\begin{center}{\bfseries प्रयत्नाद्यतमानस्तु योगी संशुद्धकिल्बिषः~।\\अनेकजन्मसंसिद्धस्ततो याति परां गतिम्~॥~४५~॥}\end{center} 
प्रयत्नात् यतमानः, अधिकं यतमान इत्यर्थः~। तत्र योगी विद्वान् संशुद्धकिल्बिषः विशुद्धकिल्बिषः संशुद्धपापः अनेकजन्मसंसिद्धः अनेकेषु जन्मसु किञ्चित्किञ्चित् संस्कारजातम् उपचित्य तेन उपचितेन अनेकजन्मकृतेन संसिद्धः अनेकजन्मसंसिद्धः ततः लब्धसम्यग्दर्शनः सन् याति परां प्रकृष्टां गतिम्~॥~४५~॥\par
 यस्मादेवं तस्मात् —} 
\begin{center}{\bfseries तपस्विभ्योऽधिको योगी ज्ञानिभ्योऽपि मतोऽधिकः~।\\कर्मिभ्यश्चाधिको योगी तस्माद्योगी भवार्जुन~॥~४६~॥}\end{center} 
तपस्विभ्यः अधिकः योगी, ज्ञानिभ्योऽपि ज्ञानमत्र शास्त्रार्थपाण्डित्यम्~, तद्वद्भ्योऽपि मतः ज्ञातः अधिकः श्रेष्ठः इति~। कर्मिभ्यः, अग्निहोत्रादि कर्म, तद्वद्भ्यः अधिकः योगी विशिष्टः यस्मात् तस्मात् योगी भव अर्जुन~॥~४६~॥\par
 \begin{center}{\bfseries योगिनामपि सर्वेषां मद्गतेनान्तरात्मना~।\\श्रद्धावान्भजते यो मां स मे युक्ततमो मतः~॥~४७~॥}\end{center} 
योगिनामपि सर्वेषां रुद्रादित्यादिध्यानपराणां मध्ये मद्गतेन मयि वासुदेवे समाहितेन अन्तरात्मना अन्तःकरणेन श्रद्धावान् श्रद्दधानः सन् भजते सेवते यो माम्~, स मे मम युक्ततमः अतिशयेन युक्तः मतः अभिप्रेतः इति~॥~४७~॥\par
 
इति श्रीमत्परमहंसपरिव्राजकाचार्यस्य श्रीगोविन्दभगवत्पूज्यपादशिष्यस्य श्रीमच्छङ्करभगवतः कृतौ श्रीमद्भगवद्गीताभाष्ये षष्ठोऽध्यायः~॥\par
 
‘योगिनामपि सर्वेषां मद्गतेनान्तरात्मना~। श्रद्धावान्भजते यो मां स मे युक्ततमो मतः’\footnote{भ. गी. ६~। ४७} इति प्रश्नबीजम् उपन्यस्य, स्वयमेव ‘ईदृशं मदीयं तत्त्वम्~, एवं मद्गतान्तरात्मा स्यात्’ इत्येतत् विवक्षुः श्रीभगवानुवाच —}\\ 
\begin{center}{\bfseries श्रीभगवानुवाच —\\ मय्यासक्तमनाः पार्थ योगं युञ्जन्मदाश्रयः~।\\असंशयं समग्रं मां यथा ज्ञास्यसि तच्छृणु~॥~१~॥}\end{center} 
मयि वक्ष्यमाणविशेषणे परमेश्वरे आसक्तं मनः यस्य सः मय्यासक्तमनाः, हे पार्थ योगं युञ्जन् मनःसमाधानं कुर्वन्~, मदाश्रयः अहमेव परमेश्वरः आश्रयो यस्य सः मदाश्रयः~। यो हि कश्चित् पुरुषार्थेन केनचित् अर्थी भवति स तत्साधनं कर्म अग्निहोत्रादि तपः दानं वा किञ्चित् आश्रयं प्रतिपद्यते, अयं तु योगी मामेव आश्रयं प्रतिपद्यते, हित्वा अन्यत् साधनान्तरं मय्येव आसक्तमनाः भवति~। यः त्वं एवंभूतः सन् असंशयं समग्रं समस्तं विभूतिबलशक्त्यैश्वर्यादिगुणसम्पन्नं मां यथा येन प्रकारेण ज्ञास्यसि संशयमन्तरेण ‘एवमेव भगवान्’ इति, तत् शृणु उच्यमानं मया~॥~१~॥\par
 तच्च मद्विषयम् —} 
\begin{center}{\bfseries ज्ञानं तेऽहं सविज्ञानमिदं वक्ष्याम्यशेषतः~।\\यज्ज्ञात्वा नेह भूयोऽन्यज्ज्ञातव्यमवशिष्यते~॥~२~॥}\end{center} 
ज्ञानं ते तुभ्यम् अहं सविज्ञानं विज्ञानसहितं स्वानुभवयुक्तम् इदं वक्ष्यामि कथयिष्यामि अशेषतः कार्‌त्स्न्येन~। तत् ज्ञानं विवक्षितं स्तौति श्रोतुः अभिमुखीकरणाय — यत् ज्ञात्वा यत् ज्ञानं ज्ञात्वा न इह भूयः पुनः अन्यत् ज्ञातव्यं पुरुषार्थसाधनम् अवशिष्यते नावशिष्टं भवति~। इति मत्तत्त्वज्ञो यः, सः सर्वज्ञो भवतीत्यर्थः~। अतो विशिष्टफलत्वात् दुर्लभं ज्ञानम्~॥~२~॥\par
 कथमित्युच्यते —} 
\begin{center}{\bfseries मनुष्याणां सहस्रेषु कश्चिद्यतति सिद्धये~।\\यततामपि सिद्धानां कश्चिन्मां वेत्ति तत्त्वतः~॥~३~॥}\end{center} 
मनुष्याणां मध्ये सहस्रेषु अनेकेषु कश्चित् यतति प्रयत्नं करोति सिद्धये सिद्ध्यर्थम्~। तेषां यततामपि सिद्धानाम्~, सिद्धा एव हि ते ये मोक्षाय यतन्ते, तेषां कश्चित् एव हि मां वेत्ति तत्त्वतः यथावत्~॥~३~॥\par
 श्रोतारं प्ररोचनेन अभिमुखीकृत्याह —} 
\begin{center}{\bfseries भूमिरापोऽनलो वायुः खं मनो बुद्धिरेव च~।\\अहङ्कार इतीयं मे भिन्ना प्रकृतिरष्टधा~॥~४~॥}\end{center} 
भूमिः इति पृथिवीतन्मात्रमुच्यते, न स्थूला, ‘भिन्ना प्रकृतिरष्टधा’ इति वचनात्~। तथा अबादयोऽपि तन्मात्राण्येव उच्यन्ते — आपः अनलः वायुः खम्~। मनः इति मनसः कारणमहङ्कारो गृह्यते~। बुद्धिः इति अहङ्कारकारणं महत्तत्त्वम्~। अहङ्कारः इति अविद्यासंयुक्तमव्यक्तम्~। यथा विषसंयुक्तमन्नं विषमित्युच्यते, एवमहङ्कारवासनावत् अव्यक्तं मूलकारणमहङ्कार इत्युच्यते, प्रवर्तकत्वात् अहङ्कारस्य~। अहङ्कार एव हि सर्वस्य प्रवृत्तिबीजं दृष्टं लोके~। इतीयं यथोक्ता प्रकृतिः मे मम ऐश्वरी मायाशक्तिः अष्टधा भिन्ना भेदमागता~॥~४~॥\par
 \begin{center}{\bfseries अपरेयमितस्त्वन्यां प्रकृतिं विद्धि मे पराम्~।\\जीवभूतां महाबाहो ययेदं धार्यते जगत्~॥~५~॥}\end{center} 
अपरा न परा निकृष्टा अशुद्धा अनर्थकरी संसारबन्धनात्मिका इयम्~। इतः अस्याः यथोक्तायाः तु अन्यां विशुद्धां प्रकृतिं मम आत्मभूतां विद्धि मे परां प्रकृष्टां जीवभूतां क्षेत्रज्ञलक्षणां प्राणधारणनिमित्तभूतां हे महाबाहो, यया प्रकृत्या इदं धार्यते जगत् अन्तः प्रविष्टया~॥~५~॥\par
 \begin{center}{\bfseries एतद्योनीनि भूतानि सर्वाणीत्युपधारय~।\\अहं कृत्स्नस्य जगतः प्रभवः प्रलयस्तथा~॥~६~॥}\end{center} 
एतद्योनीनि एते परापरे क्षेत्रक्षेत्रज्ञलक्षणे प्रकृती योनिः येषां भूतानां तानि एतद्योनीनि, भूतानि सर्वाणि इति एवम् उपधारय जानीहि~। यस्मात् मम प्रकृती योनिः कारणं सर्वभूतानाम्~, अतः अहं कृत्स्नस्य समस्तस्य जगतः प्रभवः उत्पत्तिः प्रलयः विनाशः तथा~। प्रकृतिद्वयद्वारेण अहं सर्वज्ञः ईश्वरः जगतः कारणमित्यर्थः~॥~६~॥\par
 यतः तस्मात् —} 
\begin{center}{\bfseries मत्तः परतरं नान्यत्किञ्चिदस्ति धनञ्जय~।\\मयि सर्वमिदं प्रोतं सूत्रे मणिगणा इव~॥~७~॥}\end{center} 
मत्तः परमेश्वरात् परतरम् अन्यत् कारणान्तरं किञ्चित् नास्ति न विद्यते, अहमेव जगत्कारणमित्यर्थः, हे धनञ्जय~। यस्मादेवं तस्मात् मयि परमेश्वरे सर्वाणि भूतानि सर्वमिदं जगत् प्रोतं अनुस्यूतम् अनुगतम् अनुविद्धं ग्रथितमित्यर्थ, दीर्घतन्तुषु पटवत्~, सूत्रे च मणिगणा इव~॥~७~॥\par
 केन केन धर्मेण विशिष्टे त्वयि सर्वमिदं प्रोतमित्युच्यते —} 
\begin{center}{\bfseries रसोऽहमप्सु कौन्तेय प्रभास्मि शशिसूर्ययोः~।\\प्रणवः सर्ववेदेषु शब्दः खे पौरुषं नृषु~॥~८~॥}\end{center} 
रसः अहम्~, अपां यः सारः स रसः, तस्मिन् रसभूते मयि आपः प्रोता इत्यर्थः~। एवं सर्वत्र~। यथा अहम् अप्सु रसः, एवं प्रभा अस्मि शशिसूर्ययोः~। प्रणवः ओङ्कारः सर्ववेदेषु, तस्मिन् प्रणवभूते मयि सर्वे वेदाः प्रोताः~। तथा खे आकाशे शब्दः सारभूतः, तस्मिन् मयि खं प्रोतम्~। तथा पौरुषं पुरुषस्य भावः पौरुषं यतः पुम्बुद्धिः नृषु, तस्मिन् मयि पुरुषाः प्रोताः~॥~८~॥\par
 \begin{center}{\bfseries पुण्यो गन्धः पृथिव्यां च तेजश्चास्मि विभावसौ~।\\जीवनं सर्वभूतेषु तपश्चास्मि तपस्विषु~॥~९~॥}\end{center} 
पुण्यः सुरभिः गन्धः पृथिव्यां च अहम्~, तस्मिन् मयि गन्धभूते पृथिवी प्रोता~। पुण्यत्वं गन्धस्य स्वभावत एव पृथिव्यां दर्शितम् अबादिषु रसादेः पुण्यत्वोपलक्षणार्थम्~। अपुण्यत्वं तु गन्धादीनाम् अविद्याधर्माद्यपेक्षं संसारिणां भूतविशेषसंसर्गनिमित्तं भवति~। तेजश्च दीप्तिश्च अस्मि विभावसौ अग्नौ~। तथा जीवनं सर्वभूतेषु, येन जीवन्ति सर्वाणि भूतानि तत् जीवनम्~। तपश्च अस्मि तपस्विषु, तस्मिन् तपसि मयि तपस्विनः प्रोताः~॥~९~॥\par
 \begin{center}{\bfseries बीजं मां सर्वभूतानां विद्धि पार्थ सनातनम्~।\\बुद्धिर्बुद्धिमतामस्मि तेजस्तेजस्विनामहम्~॥~१०~॥}\end{center} 
बीजं प्ररोहकारणं मां विद्धि सर्वभूतानां हे पार्थ सनातनं चिरन्तनम्~। किञ्च, बुद्धिः विवेकशक्तिः अन्तःकरणस्य बुद्धिमतां विवेकशक्तिमताम् अस्मि, तेजः प्रागल्भ्यं तद्वतां तेजस्विनाम् अहम्~॥~१०~॥\par
 \begin{center}{\bfseries बलं बलवतां चाहं कामरागविवर्जितम्~।\\धर्माविरुद्धो भूतेषु कामोऽस्मि भरतर्षभ~॥~११~॥}\end{center} 
बलं सामर्थ्यम् ओजो बलवताम् अहम्~, तच्च बलं कामरागविवर्जितम्~, कामश्च रागश्च कामरागौ — कामः तृष्णा असंनिकृष्टेषु विषयेषु, रागो रञ्जना प्राप्तेषु विषयेषु — ताभ्यां कामरागाभ्यां विवर्जितं देहादिधारणमात्रार्थं बलं सत्त्वमहमस्मि~; न तु यत्संसारिणां तृष्णारागकारणम्~। किञ्च — धर्माविरुद्धः धर्मेण शास्त्रार्थेन अविरुद्धो यः प्राणिषु भूतेषु कामः, यथा देहधारणमात्राद्यर्थः अशनपानादिविषयः, स कामः अस्मि हे भरतर्षभ~॥~११~॥\par
 किञ्च —} 
\begin{center}{\bfseries ये चैव सात्त्विका भावा राजसास्तमसाश्च ये~।\\मत्त एवेति तान्विद्धि न त्वहं तेषु ते मयि~॥~१२~॥}\end{center} 
ये चैव सात्त्विकाः सत्त्वनिर्वृत्ताः भावाः पदार्थाः, राजसाः रजोनिर्वृत्ताः, तामसाः तमोनिर्वृत्ताश्च, ये केचित् प्राणिनां स्वकर्मवशात् जायन्ते भावाः, तान् मत्त एव जायमानान् इति एवं विद्धि सर्वान् समस्तानेव~। यद्यपि ते मत्तः जायन्ते, तथापि न तु अहं तेषु तदधीनः तद्वशः, यथा संसारिणः~। ते पुनः मयि मद्वशाः मदधीनाः~॥~१२~॥\par
 एवंभूतमपि परमेश्वरं नित्यशुद्धबुद्धमुक्तस्वभावं सर्वभूतात्मानं निर्गुणं संसारदोषबीजप्रदाहकारणं मां नाभिजानाति जगत् इति अनुक्रोशं दर्शयति भगवान्~। तच्च किंनिमित्तं जगतः अज्ञानमित्युच्यते —} 
\begin{center}{\bfseries त्रिभिर्गुणमयैर्भावैरेभिः सर्वमिदं जगत्~।\\मोहितं नाभिजानाति मामेभ्यः परमव्ययम्~॥~१३~॥}\end{center} 
त्रिभिः गुणमयैः गुणविकारैः रागद्वेषमोहादिप्रकारैः भावैः पदार्थैः एभिः यथोक्तैः सर्वम् इदं प्राणिजातं जगत् मोहितम् अविवेकितामापादितं सत् न अभिजानाति माम्~, एभ्यः यथोक्तेभ्यः गुणेभ्यः परं व्यतिरिक्तं विलक्षणं च अव्ययं व्ययरहितं जन्मादिसर्वभावविकारवर्जितम् इत्यर्थः~॥~१३~॥\par
 कथं पुनः दैवीम् एतां त्रिगुणात्मिकां वैष्णवीं मायामतिक्रामति इत्युच्यते —} 
\begin{center}{\bfseries दैवी ह्येषा गुणमयी मम माया दुरत्यया~।\\मामेव ये प्रपद्यन्ते मायामेतां तरन्ति ते~॥~१४~॥}\end{center} 
दैवी देवस्य मम ईश्वरस्य विष्णोः स्वभावभूता हि यस्मात् एषा यथोक्ता गुणमयी मम माया दुरत्यया दुःखेन अत्ययः अतिक्रमणं यस्याः सा दुरत्यया~। तत्र एवं सति सर्वधर्मान् परित्यज्य मामेव मायाविनं स्वात्मभूतं सर्वात्मना ये प्रपद्यन्ते ते मायाम् एतां सर्वभूतमोहिनीं तरन्ति अतिक्रामन्ति~; ते संसारबन्धनात् मुच्यन्ते इत्यर्थः~॥~१४~॥\par
 यदि त्वां प्रपन्नाः मायामेतां तरन्ति, कस्मात् त्वामेव सर्वे न प्रपद्यन्ते इत्युच्यते —} 
\begin{center}{\bfseries न मां दुष्कृतिनो मूढाः प्रपद्यन्ते नराधमाः~।\\माययापहृतज्ञाना आसुरं भावमाश्रिताः~॥~१५~॥}\end{center} 
न मां परमेश्वरं नारायणं दुष्कृतिनः पापकारिणः मूढाः प्रपद्यन्ते नराधमाः नराणां मध्ये अधमाः निकृष्टाः~। ते च मायया अपहृतज्ञानाः संमुषितज्ञानाः आसुरं भावं हिंसानृतादिलक्षणम् आश्रिताः~॥~१५~॥\par
 ये पुनर्नरोत्तमाः पुण्यकर्माणः —} 
\begin{center}{\bfseries चतुर्विधा भजन्ते मां जनाः सुकृतिनोऽर्जुन~।\\आर्तो जिज्ञासुरर्थार्थी ज्ञानी च भरतर्षभ~॥~१६~॥}\end{center} 
चतुर्विधाः चतुःप्रकाराः भजन्ते सेवंते मां जनाः सुकृतिनः पुण्यकर्माणः हे अर्जुन~। आर्तः आर्तिपरिगृहीतः तस्करव्याघ्ररोगादिना अभिभूतः आपन्नः, जिज्ञासुः भगवत्तत्त्वं ज्ञातुमिच्छति यः, अर्थार्थी धनकामः, ज्ञानी विष्णोः तत्त्वविच्च हे भरतर्षभ~॥~१६~॥\par
 \begin{center}{\bfseries तेषां ज्ञानी नित्ययुक्त एकभक्तिर्विशिष्यते~।\\प्रियो हि ज्ञानिनोऽत्यर्थमहं स च मम प्रियः~॥~१७~॥}\end{center} 
तेषां चतुर्णां मध्ये ज्ञानी तत्त्ववित् तत्ववित्त्वात् नित्ययुक्तः भवति एकभक्तिश्च, अन्यस्य भजनीयस्य अदर्शनात्~; अतः स एकभक्तिः विशिष्यते विशेषम् आधिक्यम् आपद्यते, अतिरिच्यते इत्यर्थः~। प्रियो हि यस्मात् अहम् आत्मा ज्ञानिनः, अतः तस्य अहम् अत्यर्थं प्रियः~; प्रसिद्धं हि लोके ‘आत्मा प्रियो भवति’ इति~। तस्मात् ज्ञानिनः आत्मत्वात् वासुदेवः प्रियो भवतीत्यर्थः~। स च ज्ञानी मम वासुदेवस्य आत्मैवेति मम अत्यर्थं प्रियः~॥~१७~॥\par
 न तर्हि आर्तादयः त्रयः वासुदेवस्य प्रियाः~? न~; किं तर्हि~? — } 
\begin{center}{\bfseries उदाराः सर्व एवैते ज्ञानी त्वात्मैव मे मतम्~।\\आस्थितः स हि युक्तात्मा मामेवानुत्तमां गतिम्~॥~१८~॥}\end{center} 
उदाराः उत्कृष्टाः सर्व एव एते, त्रयोऽपि मम प्रिया एवेत्यर्थः~। न हि कश्चित् मद्भक्तः मम वासुदेवस्य अप्रियः भवति~। ज्ञानी तु अत्यर्थं प्रियो भवतीति विशेषः~। तत् कस्मात् इत्यत आह — ज्ञानी तु आत्मैव न अन्यो मत्तः इति मे मम मतं निश्चयः~। आस्थितः आरोढुं प्रवृत्तः सः ज्ञानी हि यस्मात् ‘अहमेव भगवान् वासुदेवः न अन्योऽस्मि’ इत्येवं युक्तात्मा समाहितचित्तः सन् मामेव परं ब्रह्म गन्तव्यम् अनुत्तमां गन्तुं प्रवृत्त इत्यर्थः~॥~१८~॥\par
 ज्ञानी पुनरपि स्तूयते —} 
\begin{center}{\bfseries बहूनां जन्मनामन्ते ज्ञानवान्मां प्रपद्यते~।\\वासुदेवः सर्वमिति स महात्मा सुदुर्लभः~॥~१९~॥}\end{center} 
बहूनां जन्मनां ज्ञानार्थसंस्काराश्रयाणाम् अन्ते समाप्तौ ज्ञानवान् प्राप्तपरिपाकज्ञानः मां वासुदेवं प्रत्यगात्मानं प्रत्यक्षतः प्रपद्यते~। कथम्~? वासुदेवः सर्वम् इति~। यः एवं सर्वात्मानं मां नारायणं प्रतिपद्यते, सः महात्मा~; न तत्समः अन्यः अस्ति, अधिको वा~। अतः सुदुर्लभः, ‘मनुष्याणां सहस्रेषु’\footnote{भ. गी. ७~। ३} इति हि उक्तम्~॥~१९~॥\par
 आत्मैव सर्वो वासुदेव इत्येवमप्रतिपत्तौ कारणमुच्यते —} 
\begin{center}{\bfseries कामैस्तैस्तैर्हृतज्ञानाः प्रपद्यन्तेऽन्यदेवताः~।\\तं तं नियममास्थाय प्रकृत्या नियताः स्वया~॥~२०~॥}\end{center} 
कामैः तैस्तैः पुत्रपशुस्वर्गादिविषयैः हृतज्ञानाः अपहृतविवेकविज्ञानाः प्रपद्यन्ते अन्यदेवताः प्राप्नुवन्ति वासुदेवात् आत्मनः अन्याः देवताः~; तं तं नियमं देवताराधने प्रसिद्धो यो यो नियमः तं तम् आस्थाय आश्रित्य प्रकृत्या स्वभावेन जन्मान्तरार्जितसंस्कारविशेषेण नियताः नियमिताः स्वया आत्मीयया~॥~२०~॥\par
 तेषां च कामीनाम् —} 
\begin{center}{\bfseries यो यो यां यां तनुं भक्तः श्रद्धयार्चितुमिच्छति~।\\तस्य तस्याचलां श्रद्धां तामेव विदधाम्यहम्~॥~२१~॥}\end{center} 
यः यः कामी यां यां देवतातनुं श्रद्धया संयुक्तः भक्तश्च सन् अर्चितुं पूजयितुम् इच्छति, तस्य तस्य कामिनः अचलां स्थिरां श्रद्धां तामेव विदधामि स्थिरीकरोमि~॥~२१~॥\par
 ययैव पूर्वं प्रवृत्तः स्वभावतो यः यां देवतातनुं श्रद्धया अर्चितुम् इच्छति —} 
\begin{center}{\bfseries स तया श्रद्धया युक्तस्तस्या राधनमीहते~।\\लभते च ततः कामान्मयैव विहितान्हि तान्~॥~२२~॥}\end{center} 
स तया मद्विहितया श्रद्धया युक्तः सन् तस्याः देवतातन्वाः राधनम् आराधनम् ईहते चेष्टते~। लभते च ततः तस्याः आराधितायाः देवतातन्वाः कामान् ईप्सितान् मयैव परमेश्वरेण सर्वज्ञेन कर्मफलविभागज्ञतया विहितान् निर्मितान् तान्~, हि यस्मात् ते भगवता विहिताः कामाः तस्मात् तान् अवश्यं लभते इत्यर्थः~। ‘हितान्’ इति पदच्छेदे हितत्वं कामानामुपचरितं कल्प्यम्~; न हि कामा हिताः कस्यचित्~॥~२२~॥\par
 यस्मात् अन्तवत्साधनव्यापारा अविवेकिनः कामिनश्च ते, अतः —} 
\begin{center}{\bfseries अन्तवत्तु फलं तेषां तद्भवत्यल्पमेधसाम्~।\\देवान्देवयजो यान्ति मद्भक्ता यान्ति मामपि~॥~२३~॥}\end{center} 
अन्तवत् विनाशि तु फलं तेषां तत् भवति अल्पमेधसां अल्पप्रज्ञानाम्~। देवान्देवयजो यान्ति देवान् यजन्त इति देवयजः, ते देवान् यान्ति, मद्भक्ता यान्ति मामपि~। एवं समाने अपि आयासे मामेव न प्रपद्यन्ते अनन्तफलाय, अहो खलु कष्टं वर्तन्ते, इत्यनुक्रोशं दर्शयति भगवान्~॥~२३~॥\par
 किंनिमित्तं मामेव न प्रपद्यन्ते इत्युच्यते —} 
\begin{center}{\bfseries अव्यक्तं व्यक्तिमापन्नं मन्यन्ते मामबुद्धयः~।\\परं भावमजानन्तो ममाव्ययमनुत्तमम्~॥~२४~॥}\end{center} 
अव्यक्तम् अप्रकाशं व्यक्तिम् आपन्नं प्रकाशं गतम् इदानीं मन्यन्ते मां नित्यप्रसिद्धमीश्वरमपि सन्तम् अबुद्धयः अविवेकिनः परं भावं परमात्मस्वरूपम् अजानन्तः अविवेकिनः मम अव्ययं व्ययरहितम् अनुत्तमं निरतिशयं मदीयं भावमजानन्तः मन्यन्ते इत्यर्थः~॥~२४~॥\par
 तदज्ञानं किंनिमित्तमित्युच्यते —} 
\begin{center}{\bfseries नाहं प्रकाशः सर्वस्य योगमायासमावृतः~।\\मूढोऽयं नाभिजानाति लोको मामजमव्ययम्~॥~२५~॥}\end{center} 
न अहं प्रकाशः सर्वस्य लोकस्य, केषाञ्चिदेव मद्भक्तानां प्रकाशः अहमित्यभिप्रायः~। योगमायासमावृतः योगः गुणानां युक्तिः घटनं सैव माया योगमाया, तया योगमायया समावृतः, सञ्छन्नः इत्यर्थः~। अत एव मूढो लोकः अयं न अभिजानाति माम् अजम् अव्ययम्~॥~} 
यया योगमायया समावृतं मां लोकः नाभिजानाति, नासौ योगमाया मदीया सती मम ईश्वरस्य मायाविनो ज्ञानं प्रतिबध्नाति, यथा अन्यस्यापि मायाविनः मायाज्ञानं तद्वत्~॥~२५~॥\par
 यतः एवम्~, अतः —} 
\begin{center}{\bfseries वेदाहं समतीतानि वर्तमानानि चार्जुन~।\\भविष्याणि च भूतानि मां तु वेद न कश्चन~॥~२६~॥}\end{center} 
अहं तु वेद जाने समतीतानि समतिक्रान्तानि भूतानि, वर्तमानानि च अर्जुन, भविष्याणि च भूतानि वेद अहम्~। मां तु वेद न कश्चन मद्भक्तं मच्छरणम् एकं मुक्त्वा~; मत्तत्त्ववेदनाभावादेव न मां भजते~॥~२६~॥\par
 केन पुनः मत्तत्त्ववेदनप्रतिबन्धेन प्रतिबद्धानि सन्ति जायमानानि सर्वभूतानि मां न विदन्ति इत्यपेक्षायामिदमाह —} 
\begin{center}{\bfseries इच्छाद्वेषसमुत्थेन द्वन्द्वमोहेन भारत~।\\सर्वभूतानि संमोहं सर्गे यान्ति परन्तप~॥~२७~॥}\end{center} 
इच्छाद्वेषसमुत्थेन इच्छा च द्वेषश्च इच्छाद्वेषौ ताभ्यां समुत्तिष्ठतीति इच्छाद्वेषसमुत्थः तेन इच्छाद्वेषसमुत्थेन~। केनेति विशेषापेक्षायामिदमाह — द्वन्द्वमोहेन द्वन्द्वनिमित्तः मोहः द्वन्द्वमोहः तेन~। तावेव इच्छाद्वेषौ शीतोष्णवत् परस्परविरुद्धौ सुखदुःखतद्धेतुविषयौ यथाकालं सर्वभूतैः सम्बध्यमानौ द्वन्द्वशब्देन अभिधीयेते~। यत्र यदा इच्छाद्वेषौ सुखदुःखतद्धेतुसम्प्राप्त्या लब्धात्मकौ भवतः, तदा तौ सर्वभूतानां प्रज्ञायाः स्ववशापादनद्वारेण परमार्थात्मतत्त्वविषयज्ञानोत्पत्तिप्रतिबन्धकारणं मोहं जनयतः~। न हि इच्छाद्वेषदोषवशीकृतचित्तस्य यथाभूतार्थविषयज्ञानमुत्पद्यते बहिरपि~; किमु वक्तव्यं ताभ्यामाविष्टबुद्धेः संमूढस्य प्रत्यगात्मनि बहुप्रतिबन्धे ज्ञानं नोत्पद्यत इति~। अतः तेन इच्छाद्वेषसमुत्थेन द्वन्द्वमोहेन, भारत भरतान्वयज, सर्वभूतानि संमोहितानि सन्ति संमोहं संमूढतां सर्गे जन्मनि, उत्पत्तिकाले इत्येतत्~, यान्ति गच्छन्ति हे परन्तप~। मोहवशान्येव सर्वभूतानि जायमानानि जायन्ते इत्यभिप्रायः~। यतः एवम्~, अतः तेन द्वन्द्वमोहेन प्रतिबद्धप्रज्ञानानि सर्वभूतानि संमोहितानि मामात्मभूतं न जानन्ति~; अत एव आत्मभावे मां न भजन्ते~॥~२७~॥\par
 के पुनः अनेन द्वन्द्वमोहेन निर्मुक्ताः सन्तः त्वां विदित्वा यथाशास्त्रमात्मभावेन भजन्ते इत्यपेक्षितमर्थं दर्शितुम् उच्यते —} 
\begin{center}{\bfseries येषां त्वन्तगतं पापं जनानां पुण्यकर्मणाम्~।\\ते द्वन्द्वमोहनिर्मुक्ता भजन्ते मां दृढव्रताः~॥~२८~॥}\end{center} 
येषां तु पुनः अन्तगतं समाप्तप्रायं क्षीणं पापं जनानां पुण्यकर्मणां पुण्यं कर्म येषां सत्त्वशुद्धिकारणं विद्यते ते पुण्यकर्माणः तेषां पुण्यकर्मणाम्~, ते द्वन्द्वमोहनिर्मुक्ताः यथोक्तेन द्वन्द्वमोहेन निर्मुक्ताः भजन्ते मां परमात्मानं दृढव्रताः~। ‘एवमेव परमार्थतत्त्वं नान्यथा’ इत्येवं सर्वपरित्यागव्रतेन निश्चितविज्ञानाः दृढव्रताः उच्यन्ते~॥~२८~॥\par
 ते किमर्थं भजन्ते इत्युच्यते —} 
\begin{center}{\bfseries जरामरणमोक्षाय मामाश्रित्य यतन्ति ये~।\\ते ब्रह्म तद्विदुः कृत्स्नमध्यात्मं कर्म चाखिलम्~॥~२९~॥}\end{center} 
जरामरणमोक्षाय जरामरणयोः मोक्षार्थं मां परमेश्वरम् आश्रित्य मत्समाहितचित्ताः सन्तः यतन्ति प्रयतन्ते ये, ते यत् ब्रह्म परं तत् विदुः कृत्स्नं समस्तम् अध्यात्मं प्रत्यगात्मविषयं वस्तु तत् विदुः, कर्म च अखिलं समस्तं विदुः~॥~२९~॥\par
 \begin{center}{\bfseries साधिभूताधिदैवं मां साधियज्ञं च ये विदुः~।\\प्रयाणकालेऽपि च मां ते विदुर्युक्तचेतसः~॥~३०~॥}\end{center} 
साधिभूताधिदैवम् अधिभूतं च अधिदैवं च अधिभूताधिदैवम्~, सह अधिभूताधिदैवेन वर्तते इति साधिभूताधिदैवं च मां ये विदुः, साधियज्ञं च सह अधियज्ञेन साधियज्ञं ये विदुः, प्रयाणकाले मरणकाले अपि च मां ते विदुः युक्तचेतसः समाहितचित्ता इति~॥~३०~॥\par
 
इति श्रीमत्परमहंसपरिवारजकाचार्यस्य श्रीगोविन्दभगवत्पूजयपादशिष्यस्य श्रीमच्छङ्करभगवतः कृतौ श्रीमद्भगवद्गीताभाष्ये सप्तमोऽध्यायः~॥\par
 
‘ते ब्रह्म तद्विदुः कृत्स्नम्’\footnote{भ. गी. ७~। २९} इत्यादिना भगवता अर्जुनस्य प्रश्नबीजानि उपदिष्टानि~। अतः तत्प्रश्नार्थम् अर्जुनः उवाच —}\\ 
\begin{center}{\bfseries अर्जुन उवाच —\\ किं तद्ब्रह्म किमध्यात्मं किं कर्म पुरुषोत्तम~।\\अधिभूतं च किं प्रोक्तमधिदैवं किमुच्यते~॥~१~॥}\\[10pt]
{\bfseries अधियज्ञः कथं कोऽत्र देहेऽस्मिन्मधुसूदन~।\\प्रयाणकाले च कथं ज्ञेयोऽसि नियतात्मभिः~॥~२~॥}\end{center}
 एषां प्रश्नानां यथाक्रमं निर्णयाय श्रीभगवानुवाच —}\\ 
\begin{center}{\bfseries श्रीभगवानुवाच —\\ अक्षरं ब्रह्म परमं स्वभावोऽध्यात्ममुच्यते~।\\भूतभावोद्भवकरो विसर्गः कर्मसंज्ञितः~॥~३~॥}\end{center} 
अक्षरं न क्षरतीति अक्षरं परमात्मा, ‘एतस्य वा अक्षरस्य प्रशासने गार्गि’\footnote{बृ. उ. ३~। ८~। ९} इति श्रुतेः~। ओङ्कारस्य च ‘ओमित्येकाक्षरं ब्रह्म’\footnote{भ. गी. ८~। १३} इति परेण विशेषणात् अग्रहणम्~। परमम् इति च निरतिशये ब्रह्मणि अक्षरे उपपन्नतरम् विशेषणम्~। तस्यैव परस्य ब्रह्मणः प्रतिदेहं प्रत्यगात्मभावः स्वभावः, स्वो भावः स्वभावः अध्यात्मम् उच्यते~। आत्मानं देहम् अधिकृत्य प्रत्यगात्मतया प्रवृत्तं परमार्थब्रह्मावसानं वस्तु स्वभावः अध्यात्मम् उच्यते अध्यात्मशब्देन अभिधीयते~। भूतभावोद्भवकरः भूतानां भावः भूतभावः तस्य उद्भवः भूतभावोद्भवः तं करोतीति भूतभावोद्भवकरः, भूतवस्तूत्पत्तिकर इत्यर्थः~। विसर्गः विसर्जनं देवतोद्देशेन चरुपुरोडाशादेः द्रव्यस्य परित्यागः~; स एष विसर्गलक्षणो यज्ञः कर्मसंज्ञितः कर्मशब्दित इत्येतत्~। एतस्मात् हि बीजभूतात् वृष्ट्यादिक्रमेण स्थावरजङ्गमानि भूतानि उद्भवन्ति~॥~३~॥\par
 \begin{center}{\bfseries अधिभूतं क्षरो भावः पुरुषश्चाधिदैवतम्~।\\अधियज्ञोऽहमेवात्र देहे देहभृतां वर~॥~४~॥}\end{center} 
अधिभूतं प्राणिजातम् अधिकृत्य भवतीति~। कोऽसौ~? क्षरः क्षरतीति क्षरः विनाशी, भावः यत्किञ्चित् जनिमत् वस्तु इत्यर्थः~। पुरुषः पूर्णम् अनेन सर्वमिति, पुरि शयनात् वा, पुरुषः आदित्यान्तर्गतो हिरण्यगर्भः, सर्वप्राणिकरणानाम् अनुग्राहकः, सः अधिदैवतम्~। अधियज्ञः सर्वयज्ञाभिमानिनी विष्ण्वाख्या देवता, ‘यज्ञो वै विष्णुः’\footnote{तै. सं. १~। ७~। ४} इति श्रुतेः~। स हि विष्णुः अहमेव~; अत्र अस्मिन् देहे यो यज्ञः तस्य अहम् अधियज्ञः~; यज्ञो हि देहनिर्वर्त्यत्वेन देहसमवायी इति देहाधिकरणो भवति, देहभृतां वर~॥~४~॥\par
 \begin{center}{\bfseries अन्तकाले च मामेव स्मरन्मुक्त्वा कलेबरम्~।\\यः प्रयाति स मद्भावं याति नास्त्यत्र संशयः~॥~५~॥}\end{center} 
अन्तकाले मरणकाले च मामेव परमेश्वरं विष्णुं स्मरन् मुक्त्वा परित्यज्य कलेबरं शरीरं यः प्रयाति गच्छति, सः मद्भावं वैष्णवं तत्त्वं याति~। नास्ति न विद्यते अत्र अस्मिन् अर्थे संशयः — याति वा न वा इति~॥~५~॥\par
 न मद्विषय एव अयं नियमः~। किं तर्हि~? —} 
\begin{center}{\bfseries यं यं वापि स्मरन्भावं त्यजत्यन्ते कलेबरम्~।\\तं तमेवैति कौन्तेय सदा तद्भावभावितः~॥~६~॥}\end{center} 
यं यं वापि यं यं भावं देवताविशेषं स्मरन् चिन्तयन् त्यजति परित्यजति अन्ते अन्तकाले प्राणवियोगकाले कलेबरं शरीरं तं तमेव स्मृतं भावमेव एति नान्यं कौन्तेय, सदा सर्वदा तद्भावभावितः तस्मिन् भावः तद्भावः स भावितः स्मर्यमाणतया अभ्यस्तः येन सः तद्भावभावितः सन्~॥~६~॥\par
 यस्मात् एवम् अन्त्या भावना देहान्तरप्राप्तौ कारणम् —} 
\begin{center}{\bfseries तस्मात्सर्वेषु कालेषु मामनुस्मर युध्य च~।\\मय्यर्पितमनोबुद्धिर्मामेवैष्यस्यसंशयः~॥~७~॥}\end{center} 
तस्मात् सर्वेषु कालेषु माम् अनुस्मर यथाशास्त्रम्~। युध्य च युद्धं च स्वधर्मं कुरु~। मयि वासुदेवे अर्पिते मनोबुद्धी यस्य तव स त्वं मयि अर्पितमनोबुद्धिः सन् मामेव यथास्मृतम् एष्यसि आगमिष्यसि~; असंशयः न संशयः अत्र विद्यते~॥~७~॥\par
 किञ्च—} 
\begin{center}{\bfseries अभ्यासयोगयुक्तेन चेतसा नान्यगामिना~।\\परमं पुरुषं दिव्यं याति पार्थानुचिन्तयन्~॥~८~॥}\end{center} 
अभ्यासयोगयुक्तेन मयि चित्तसमर्पणविषयभूते एकस्मिन् तुल्यप्रत्ययावृत्तिलक्षणः विलक्षणप्रत्ययानन्तरितः अभ्यासः स चाभ्यासो योगः तेन युक्तं तत्रैव व्यापृतं योगिनः चेतः तेन, चेतसा नान्यगामिना न अन्यत्र विषयान्तरे गन्तुं शीलम् अस्येति नान्यगामि तेन नान्यगामिना, परमं निरतिशयं पुरुषं दिव्यं दिवि सूर्यमण्डले भवं याति गच्छति हे पार्थ अनुचिन्तयन् शास्त्राचार्योपदेशम् अनुध्यायन् इत्येतत्~॥~८~॥\par
 किंविशिष्टं च पुरुषं याति इति उच्यते —} 
\begin{center}{\bfseries कविं पुराणमनुशासितारमणोरणीयांसमनुस्मरेद्यः~।\\सर्वस्य धातारमचिन्त्यरूपमादित्यवर्णं तमसः परस्तात्~॥~९~॥}\end{center} 
कविं क्रान्तदर्शिनं सर्वज्ञं पुराणं चिरन्तनम् अनुशासितारं सर्वस्य जगतः प्रशासितारम् अणोः सूक्ष्मादपि अणीयांसं सूक्ष्मतरम् अनुस्मरेत् अनुचिन्तयेत् यः कश्चित्~, सर्वस्य कर्मफलजातस्य धातारं विधातारं विचित्रतया प्राणिभ्यो विभक्तारम्~, अचिन्त्यरूपं न अस्य रूपं नियतं विद्यमानमपि केनचित् चिन्तयितुं शक्यते इति अचिन्त्यरूपः तम्~, आदित्यवर्णम् आदित्यस्येव नित्यचैतन्यप्रकाशो वर्णो यस्य तम् आदित्यवर्णम्~, तमसः परस्तात् अज्ञानलक्षणात् मोहान्धकारात् परं तम् अनुचिन्तयन् याति इति पूर्वेण सम्बन्धः~॥~९~॥\par
 किञ्च —} 
\begin{center}{\bfseries प्रयाणकाले मनसाचलेन भक्त्या युक्तो योगबलेन चैव~।\\भ्रुवोर्मध्ये प्राणमावेश्य सम्यक्स तं परं पुरुषमुपैति दिव्यम्~॥~१०~॥}\end{center} 
प्रयाणकाले मरणकाले मनसा अचलेन चलनवर्जितेन भक्त्या युक्तः भजनं भक्तिः तया युक्तः योगबलेन चैव योगस्य बलं योगबलं समाधिजसंस्कारप्रचयजनितचित्तस्थैर्यलक्षणं योगबलं तेन च युक्तः इत्यर्थः, पूर्वं हृदयपुण्डरीके वशीकृत्य चित्तं ततः ऊर्ध्वगामिन्या नाड्या भूमिजयक्रमेण भ्रुवोः मध्ये प्राणम् आवेश्य स्थापयित्वा सम्यक् अप्रमत्तः सन्~, सः एवं विद्वान् योगी ‘कविं पुराणम्’\footnote{भ. गी. ८~। ९} इत्यादिलक्षणं तं परं परतरं पुरुषम् उपैति प्रतिपद्यते दिव्यं द्योतनात्मकम्~॥~१०~॥\par
 पुनरपि वक्ष्यमाणेन उपायेन प्रतिपित्सितस्य ब्रह्मणो वेदविद्वदनादिविशेषणविशेष्यस्य अभिधानं करोति भगवान् —} 
\begin{center}{\bfseries यदक्षरं वेदविदो वदन्ति विशन्ति यद्यतयो वीतरागाः~।\\यदिच्छन्तो ब्रह्मचर्यं चरन्ति तत्ते पदं सङ्ग्रहेण प्रवक्ष्ये~॥~११~॥}\end{center} 
यत् अक्षरं न क्षरतीति अक्षरम् अविनाशि वेदविदः वेदार्थज्ञाः वदन्ति, ‘तद्वा एतदक्षरं गार्गि ब्राह्मणा अभिवदन्ति’\footnote{बृ. उ. ३~। ८~। ८} इति श्रुतेः, सर्वविशेषनिवर्तकत्वेन अभिवदन्ति ‘अस्थूलमनणु’ इत्यादि~। किञ्च — विशन्ति प्रविशन्ति सम्यग्दर्शनप्राप्तौ सत्यां यत् यतयः यतनशीलाः संन्यासिनः वीतरागाः वीतः विगतः रागः येभ्यः ते वीतरागाः~। यच्च अक्षरमिच्छन्तः — ज्ञातुम् इति वाक्यशेषः — ब्रह्मचर्यं गुरौ चरन्ति आचरन्ति, तत् ते पदं तत् अक्षराख्यं पदं पदनीयं ते तव सङ्ग्रहेण सङ्ग्रहः सङ्क्षेपः तेन सङ्क्षेपेण प्रवक्ष्ये कथयिष्यामि~॥~११~॥\par
 ‘स यो ह वै तद्भगवन्मनुष्येषु प्रायणान्तमोङ्कारमभिध्यायीत कतमम् वाव स तेन लोकं जयतीति~। ’\footnote{प्र. उ. ५~। १}‘तस्मै स होवाच एतद्वै सत्यकाम परं चापरं च ब्रह्म यदोङ्कारः’\footnote{प्र. उ. ५~। २} इत्युपक्रम्य ‘यः पुनरेतं त्रिमात्रेणोमित्येतेनैवाक्षरेण परं पुरुषमभिध्यायीत — स सामभिरुन्नीयते ब्रह्मलोकम्’\footnote{प्र. उ. ५~। ५} इत्यादिना वचनेन, ‘अन्यत्र धर्मादन्यत्राधर्मात्’\footnote{क. उ. १~। २~। १४} इति च उपक्रम्य ‘सर्वे वेदा यत्पदमामनन्ति~। तपांसि सर्वाणि च यद्वदन्ति~। यदिच्छन्तो ब्रह्मचर्यं चरन्ति तत्ते पदं सङ्ग्रहेण ब्रवीम्योमित्येतत्’\footnote{क. उ. १~। २~। १५} इत्यादिभिश्च वचनैः परस्य ब्रह्मणो वाचकरूपेण, प्रतिमावत् प्रतीकरूपेण वा, परब्रह्मप्रतिपत्तिसाधनत्वेन मन्दमध्यमबुद्धीनां विवक्षितस्य ओङ्कारस्य उपासनं कालान्तरे मुक्तिफलम् उक्तं यत्~, तदेव इहापि ‘कविं पुराणमनुशासितारम्’\footnote{भ. गी. ८~। ९} ‘यदक्षरं वेदविदो वदन्ति’\footnote{भ. गी. ८~। ११} इति च उपन्यस्तस्य परस्य ब्रह्मणः पूर्वोक्तरूपेण प्रतिपत्त्युपायभूतस्य ओङ्कारस्य कालान्तरमुक्तिफलम् उपासनं योगधारणासहितं वक्तव्यम्~, प्रसक्तानुप्रसक्तं च यत्किञ्चित्~, इत्येवमर्थः उत्तरो ग्रन्थ आरभ्यते —} 
\begin{center}{\bfseries सर्वद्वाराणि संयम्य मनो हृदि निरुध्य च~।\\मूर्ध्न्याधायात्मनः प्राणमास्थितो योगधारणाम्~॥~१२~॥}\end{center} 
सर्वद्वाराणि सर्वाणि च तानि द्वाराणि च सर्वद्वाराणि उपलब्धौ, तानि सर्वाणि संयम्य संयमनं कृत्वा मनः हृदि हृदयपुण्डरीके निरुध्य निरोधं कृत्वा निष्प्रचारमापाद्य, तत्र वशीकृतेन मनसा हृदयात् ऊर्ध्वगामिन्या नाड्या ऊर्ध्वमारुह्य मूर्ध्नि आधाय आत्मनः प्राणम् आस्थितः प्रवृत्तः योगधारणां धारयितुम्~॥~१२~॥\par
 तत्रैव च धारयन् —} 
\begin{center}{\bfseries ओमित्येकाक्षरं ब्रह्म व्याहरन्मामनुस्मरन्~।\\यः प्रयाति त्यजन्देहं स याति परमां गतिम्~॥~१३~॥}\end{center} 
ओमिति एकाक्षरं ब्रह्म ब्रह्मणः अभिधानभूतम् ओङ्कारं व्याहरन् उच्चारयन्~, तदर्थभूतं माम् ईश्वरम् अनुस्मरन् अनुचिन्तयन् यः प्रयाति म्रियते, सः त्यजन् परित्यजन् देहं शरीरम् — ‘त्यजन् देहम्’ इति प्रयाणविशेषणार्थम् देहत्यागेन प्रयाणम् आत्मनः, न स्वरूपनाशेनेत्यर्थः — सः एवं याति गच्छति परमां प्रकृष्टां गतिम्~॥~१३~॥\par
 किञ्च —} 
\begin{center}{\bfseries अनन्यचेताः सततं यो मां स्मरति नित्यशः~।\\तस्याहं सुलभः पार्थ नित्ययुक्तस्य योगिनः~॥~१४~॥}\end{center} 
अनन्यचेताः न अन्यविषये चेतः यस्य सोऽयम् अनन्यचेताः, योगी सततं सर्वदा यः मां परमेश्वरं स्मरति नित्यशः~। सततम् इति नैरन्तर्यम् उच्यते, नित्यशः इति दीर्घकालत्वम् उच्यते~। न षण्मासं संवत्सरं वा~; किं तर्हि~? यावज्जीवं नैरन्तर्येण यः मां स्मरतीत्यर्थः~। तस्य योगिनः अहं सुलभः सुखेन लभ्यः हे पार्थ, नित्ययुक्तस्य सदा समाहितचित्तस्य योगिनः~। यतः एवम्~, अतः अनन्यचेताः सन् मयि सदा समाहितः भवेत्~॥~१४~॥\par
 तव सौलभ्येन किं स्यात् इत्युच्यते~; शृणु तत् मम सौलभ्येन यत् भवति —} 
\begin{center}{\bfseries मामुपेत्य पुनर्जन्म दुःखालयमशाश्वतम्~।\\नाप्नुवन्ति महात्मानः संसिद्धिं परमां गताः~॥~१५~॥}\end{center} 
माम् उपेत्य माम् ईश्वरम् उपेत्य मद्भावमापद्य पुनर्जन्म पुनरुत्पत्तिं नाप्नुवन्ति न प्राप्नुवन्ति~। किंविशिष्टं पुनर्जन्म न प्राप्नुवन्ति इति, तद्विशेषणमाह — दुःखालयं दुःखानाम् आध्यात्मिकादीनां आलयम् आश्रयम् आलीयन्ते यस्मिन् दुःखानि इति दुःखालयं जन्म~। न केवलं दुःखालयम्~, अशाश्वतम् अनवस्थितस्वरूपं च~। नाप्नुवन्ति ईदृशं पुनर्जन्म महात्मानः यतयः संसिद्धिं मोक्षाख्यां परमां प्रकृष्टां गताः प्राप्ताः~। ये पुनः मां न प्राप्नुवन्ति ते पुनः आवर्तन्ते~॥~१५~॥\par
 किं पुनः त्वत्तः अन्यत् प्राप्ताः पुनरावर्तन्ते इति, उच्यते —} 
\begin{center}{\bfseries आ ब्रह्मभुवनाल्लोकाः पुनरावर्तिनोऽर्जुन~।\\मामुपेत्य तु कौन्तेय पुनर्जन्म न विद्यते~॥~१६~॥}\end{center} 
आ ब्रह्मभुवनात् भवन्ति अस्मिन् भूतानि इति भुवनम्~, ब्रह्मणो भुवनं ब्रह्मभुवनम्~, ब्रह्मलोक इत्यर्थः, आ ब्रह्मभुवनात् सह ब्रह्मभुवनेन लोकाः सर्वे पुनरावर्तिनः पुनरावर्तनस्वभावाः हे अर्जुन~। माम् एकम् उपेत्य तु कौन्तेय पुनर्जन्म पुनरुत्पत्तिः न विद्यते~॥~१६~॥\par
 ब्रह्मलोकसहिताः लोकाः कस्मात् पुनरावर्तिनः~? कालपरिच्छिन्नत्वात्~। कथम्~? —} 
\begin{center}{\bfseries सहस्रयुगपर्यन्तमहर्यद्ब्रह्मणो विदुः~।\\रात्रिं युगसहस्रान्तां तेऽहोरात्रविदो जनाः~॥~१७~॥}\end{center} 
सहस्रयुगपर्यन्तं सहस्राणि युगानि पर्यन्तः पर्यवसानं यस्य अह्नः तत् अहः सहस्रयुगपर्यन्तम्~, ब्रह्मणः प्रजापतेः विराजः विदुः, रात्रिम् अपि युगसहस्रान्तां अहःपरिमाणामेव~। के विदुरित्याह — ते अहोरात्रविदः कालसङ्ख्याविदो जनाः इत्यर्थः~। यतः एवं कालपरिच्छिन्नाः ते, अतः पुनरावर्तिनो लोकाः~॥~१७~॥\par
 प्रजापतेः अहनि यत् भवति रात्रौ च, तत् उच्यते —} 
\begin{center}{\bfseries अव्यक्ताद्व्यक्तयः सर्वाः प्रभवन्त्यहरागमे~।\\रात्र्यागमे प्रलीयन्ते तत्रैवाव्यक्तसंज्ञके~॥~१८~॥}\end{center} 
अव्यक्तात् अव्यक्तं प्रजापतेः स्वापावस्था तस्मात् अव्यक्तात् व्यक्तयः व्यज्यन्त इति व्यक्तयः स्थावरजङ्गमलक्षणाः सर्वाः प्रजाः प्रभवन्ति अभिव्यज्यन्ते, अह्नः आगमः अहरागमः तस्मिन् अहरागमे काले ब्रह्मणः प्रबोधकाले~। तथा रात्र्यागमे ब्रह्मणः स्वापकाले प्रलीयन्ते सर्वाः व्यक्तयः तत्रैव पूर्वोक्ते अव्यक्तसंज्ञके~॥~१८~॥\par
 अकृताभ्यागमकृतविप्रणाशदोषपरिहारार्थम्~, बन्धमोक्षशास्त्रप्रवृत्तिसाफल्यप्रदर्शनार्थम् अविद्यादिक्लेशमूलकर्माशयवशाच्च अवशः भूतग्रामः भूत्वा भूत्वा प्रलीयते इत्यतः संसारे वैराग्यप्रदर्शनार्थं च इदमाह —} 
\begin{center}{\bfseries भूतग्रामः स एवायं भूत्वा भूत्वा प्रलीयते~।\\रात्र्यागमेऽवशः पार्थ प्रभवत्यहरागमे~॥~१९~॥}\end{center} 
भूतग्रामः भूतसमुदायः स्थावरजङ्गमलक्षणः यः पूर्वस्मिन् कल्पे आसीत् स एव अयं नान्यः~। भूत्वा भूत्वा अहरागमे, प्रलीयते पुनः पुनः रात्र्यागमे अह्नः क्षये अवशः अस्वतन्त्र एव, हे पार्थ, प्रभवति जायते अवश एव अहरागमे~॥~१९~॥\par
 यत् उपन्यस्तम् अक्षरम्~, तस्य प्राप्त्युपायो निर्दिष्टः ‘ओमित्येकाक्षरं ब्रह्म’\footnote{भ. गी. ८~। १३} इत्यादिना~। अथ इदानीम् अक्षरस्यैव स्वरूपनिर्दिदिक्षया इदम् उच्यते, अनेन योगमार्गेण इदं गन्तव्यमिति —} 
\begin{center}{\bfseries परस्तस्मात्तु भावोऽन्योऽव्यक्तोऽव्यक्तात्सनातनः~।\\यः स सर्वेषु भूतेषु नश्यत्सु न विनश्यति~॥~२०~॥}\end{center} 
परः व्यतिरिक्तः भिन्नः~; कुतः~? तस्मात् पूर्वोक्तात्~। तु—शब्दः अक्षरस्य विवक्षितस्य अव्यक्तात् वैलक्षण्यविशेषणार्थः~। भावः अक्षराख्यं परं ब्रह्म~। व्यतिरिक्तत्वे सत्यपि सालक्षण्यप्रसङ्गोऽस्तीति तद्विनिवृत्त्यर्थम् आह — अन्यः इति~। अन्यः विलक्षणः~। स च अव्यक्तः अनिन्द्रियगोचरः~। ‘परस्तस्मात्’ इत्युक्तम्~; कस्मात् पुनः परः~? पूर्वोक्तात् भूतग्रामबीजभूतात् अविद्यालक्षणात् अव्यक्तात्~। अन्यः विलक्षणः भावः इत्यभिप्रायः~। सनातनः चिरन्तनः यः सः भावः सर्वेषु भूतेषु ब्रह्मादिषु नश्यत्सु न विनश्यति~॥~२०~॥\par
 \begin{center}{\bfseries अव्यक्तोऽक्षर इत्युक्तस्तमाहुः परमां गतिम्~।\\यं प्राप्य न निवर्तन्ते तद्धाम परमं मम~॥~२१~॥}\end{center} 
योऽसौ अव्यक्तः अक्षरः इत्युक्तः, तमेव अक्षरसंज्ञकम् अव्यक्तं भावम् आहुः परमां प्रकृष्टां गतिम्~। यं परं भावं प्राप्य गत्वा न निवर्तन्ते संसाराय, तत् धाम स्थानं परमं प्रकृष्टं मम, विष्णोः परमं पदमित्यर्थः~॥~२१~॥\par
 तल्लब्धेः उपायः उच्यते —} 
\begin{center}{\bfseries पुरुषः स परः पार्थ भक्त्या लभ्यस्त्वनन्यया~।\\यस्यान्तःस्थानि भूतानि येन सर्वमिदं ततम्~॥~२२~॥}\end{center} 
पुरुषः पुरि शयनात् पूर्णत्वाद्वा, स परः पार्थ, परः निरतिशयः, यस्मात् पुरुषात् न परं किञ्चित्~। सः भक्त्या लभ्यस्तु ज्ञानलक्षणया अनन्यया आत्मविषयया~। यस्य पुरुषस्य अन्तःस्थानि मध्यस्थानि भूतानि कार्यभूतानि~; कार्यं हि कारणस्य अन्तर्वर्ति भवति~। येन पुरुषेण सर्वं इदं जगत् ततं व्याप्तम् आकाशेनेव घटादि~॥~२२~॥\par
 प्रकृतानां योगिनां प्रणवावेशितब्रह्मबुद्धीनां कालान्तरमुक्तिभाजां ब्रह्मप्रतिपत्तये उत्तरो मार्गो वक्तव्य इति ‘यत्र काले’ इत्यादि विवक्षितार्थसमर्पणार्थम् उच्यते, आवृत्तिमार्गोपन्यासः इतरमार्गस्तुत्यर्थः —} 
\begin{center}{\bfseries यत्र काले त्वनावृत्तिमावृत्तिं चैव योगिनः~।\\प्रयाता यान्ति तं कालं वक्ष्यामि भरतर्षभ~॥~२३~॥}\end{center} 
यत्र काले प्रयाताः इति व्यवहितेन सम्बन्धः~। यत्र यस्मिन् काले तु अनावृत्तिम् अपुनर्जन्म आवृत्तिं तद्विपरीतां चैव~। योगिनः इति योगिनः कर्मिणश्च उच्यन्ते, कर्मिणस्तु गुणतः — ‘कर्मयोगेन योगिनाम्’\footnote{भ. गी. ३~। ३} इति विशेषणात् — योगिनः~। यत्र काले प्रयाताः मृताः योगिनः अनावृत्तिं यान्ति, यत्र काले च प्रयाताः आवृत्तिं यान्ति, तं कालं वक्ष्यामि भरतर्षभ~॥~२३~॥\par
 तं कालमाह —} 
\begin{center}{\bfseries अग्निर्ज्योतिरहः शुक्लः षण्मासा उत्तरायणम्~।\\तत्र प्रयाता गच्छन्ति ब्रह्म ब्रह्मविदो जनाः~॥~२४~॥}\end{center} 
अग्निः कालाभिमानिनी देवता~। तथा ज्योतिरपि देवतैव कालाभिमानिनी~। अथवा, अग्निज्योतिषी यथाश्रुते एव देवते~। भूयसा तु निर्देशो ‘यत्र काले’ ‘तं कालम्’ इति आम्रवणवत्~। तथा अहः देवता अहरभिमानिनी~; शुक्लः शुक्लपक्षदेवता~; षण्मासा उत्तरायणम्~, तत्रापि देवतैव मार्गभूता इति स्थितः अन्यत्र अयं न्यायः~। तत्र तस्मिन् मार्गे प्रयाताः मृताः गच्छन्ति ब्रह्म ब्रह्मविदो ब्रह्मोपासकाः ब्रह्मोपासनपरा जनाः~। ‘क्रमेण’ इति वाक्यशेषः~। न हि सद्योमुक्तिभाजां सम्यग्दर्शननिष्ठानां गतिः आगतिर्वा क्वचित् अस्ति, ‘न तस्य प्राणा उत्क्रामन्ति’\footnote{बृ. उ. ४~। ४~। ६} इति श्रुतेः~। ब्रह्मसंलीनप्राणा एव ते ब्रह्ममया ब्रह्मभूता एव ते~॥~२४~॥\par
 \begin{center}{\bfseries धूमो रात्रिस्तथा कृष्णः षण्मासा दक्षिणायनम्~।\\ तत्र चान्द्रमसं ज्योतिर्योगी प्राप्य निवर्तते~॥~२५~॥}\end{center} 
धूमो रात्रिः धूमाभिमानिनी रात्र्यभिमानिनी च देवता~। तथा कृष्णः कृष्णपक्षदेवता~। षण्मासा दक्षिणायनम् इति च पूर्ववत् देवतैव~। तत्र चन्द्रमसि भवं चान्द्रमसं ज्योतिः फलम् इष्टादिकारी योगी कर्मी प्राप्य भुक्त्वा तत्क्षयात् इह पुनः निवर्तते~॥~२५~॥\par
 \begin{center}{\bfseries शुक्लकृष्णे गती ह्येते जगतः शाश्वते मते~।\\एकया यात्यनावृत्तिमन्ययावर्तते पुनः~॥~२६~॥}\end{center} 
शुक्लकृष्णे शुक्ला च कृष्णा च शुक्लकृष्णे, ज्ञानप्रकाशकत्वात् शुक्ला, तदभावात् कृष्णा~; एते शुक्लकृष्णे हि गती जगतः इति अधिकृतानां ज्ञानकर्मणोः, न जगतः सर्वस्यैव एते गती सम्भवतः~; शाश्वते नित्ये, संसारस्य नित्यत्वात्~, मते अभिप्रेते~। तत्र एकया शुक्लया याति अनावृत्तिम्~, अन्यया इतरया आवर्तते पुनः भूयः~॥~२६~॥\par
 \begin{center}{\bfseries नैते सृती पार्थ जानन्योगी मुह्यति कश्चन~।\\तस्मात्सर्वेषु कालेषु योगयुक्तो भवार्जुन~॥~२७~॥}\end{center} 
न एते यथोक्ते सृती मार्गौ पार्थ जानन् संसाराय एका, अन्या मोक्षाय इति, योगी न मुह्यति कश्चन कश्चिदपि~। तस्मात् सर्वेषु कालेषु योगयुक्तः समाहितो भव अर्जुन~॥~२७~॥\par
 शृणु तस्य योगस्य माहात्म्यम् —} 
\begin{center}{\bfseries वेदेषु यज्ञेषु तपःसु चैव दानेषु यत्पुण्यफलं प्रदिष्टम्~।\\अत्येति तत्सर्वमिदं विदित्वा योगी परं स्थानमुपैति चाद्यम्~॥~२८~॥}\end{center} 
वेदेषु सम्यगधीतेषु यज्ञेषु च साद्गुण्येन अनुष्ठितेषु तपःसु च सुतप्तेषु दानेषु च सम्यग्दत्तेषु, एतेषु यत् पुण्यफलं प्रदिष्टं शास्त्रेण, अत्येति अतीत्य गच्छति तत् सर्वं फलजातम्~; इदं विदित्वा सप्तप्रश्ननिर्णयद्वारेण उक्तम् अर्थं सम्यक् अवधार्य अनुष्ठाय योगी, परम् उत्कृष्टम् ऐश्वरं स्थानम् उपैति च प्रतिपद्यते आद्यम् आदौ भवम्~, कारणं ब्रह्म इत्यर्थः~॥~२८~॥\par
 
इति श्रीमत्परमहंसपरिव्राजकाचार्यस्य श्रीगोविन्दभगवत्पूज्यपादशिष्यस्य श्रीमच्छङ्करभगवतः कृतौ श्रीमद्भगवद्गीताभाष्ये अष्टमोऽध्यायः~॥\par
 
अष्टमे नाडीद्वारेण धारणायोगः सगुणः उक्तः~। तस्य च फलम् अग्न्यर्चिरादिक्रमेण कालान्तरे ब्रह्मप्राप्तिलक्षणमेव अनावृत्तिरूपं निर्दिष्टम्~। तत्र ‘अनेनैव प्रकारेण मोक्षप्राप्तिफलम् अधिगम्यते, न अन्यथा’ इति तदाशङ्काव्याविवर्तयिषया श्रीभगवान् उवाच —}\\ 
\begin{center}{\bfseries श्रीभगवानुवाच —\\ इदं तु ते गुह्यतमं\\ प्रवक्ष्याम्यनसूयवे~।\\ज्ञानं विज्ञानसहितं\\ यज्ज्ञात्वा मोक्ष्यसेऽशुभात्~॥~१~॥}\end{center} 
इदं ब्रह्मज्ञानं वक्ष्यमाणम् उक्तं च पूर्वेषु अध्यायेषु, } 
तत् बुद्धौ संनिधीकृत्य इदम् इत्याह~। तु—शब्दो विशेषनिर्धारणार्थः~। इदमेव तु सम्यग्ज्ञानं साक्षात् मोक्षप्राप्तिसाधनम् ‘वासुदेवः सर्वमिति’\footnote{भ. गी. ७~। १९} ‘आत्मैवेदं सर्वम्’\footnote{छा. उ. ७~। २५~। २} ‘एकमेवाद्वितीयम्’\footnote{छा. उ. ६~। २~। १} इत्यादिश्रुतिस्मृतिभ्यः~; नान्यत्~, ‘अथ ते येऽन्यथातो विदुः अन्यराजानः ते क्षय्यलोका भवन्ति’\footnote{छा. उ. ७~। २५~। २} इत्यादिश्रुतिभ्यश्च~। ते तुभ्यं गुह्यतमं गोप्यतमं प्रवक्ष्यामि कथयिष्यामि अनसूयवे असूयारहिताय~। किं तत्~? ज्ञानम्~। किंविशिष्टम्~? विज्ञानसहितम् अनुभवयुक्तम्~, यत् ज्ञात्वा प्राप्य मोक्ष्यसे अशुभात् संसारबन्धनात्~॥~१~॥\par
 तच्च —} 
\begin{center}{\bfseries राजविद्या राजगुह्यं पवित्रमिदमुत्तमम्~।\\प्रत्यक्षावगमं धर्म्यं सुसुखं कर्तुमव्ययम्~॥~२~॥}\end{center} 
राजविद्या विद्यानां राजा, दीप्त्यतिशयवत्त्वात्~; दीप्यते हि इयम् अतिशयेन ब्रह्मविद्या सर्वविद्यानाम्~। तथा राजगुह्यं गुह्यानां राजा~। पवित्रं पावनं इदम् उत्तमं सर्वेषां पावनानां शुद्धिकारणं ब्रह्मज्ञानम् उत्कृष्टतमम्~। अनेकजन्मसहस्रसञ्चितमपि धर्माधर्मादि समूलं कर्म क्षणमात्रादेव भस्मीकरोति इत्यतः किं तस्य पावनत्वं वक्तव्यम्~। किञ्च — प्रत्यक्षावगमं प्रत्यक्षेण सुखादेरिव अवगमो यस्य तत् प्रत्यक्षावगमम्~। अनेकगुणवतोऽपि धर्मविरुद्धत्वं दृष्टम्~, न तथा आत्मज्ञानं धर्मविरोधि, किन्तु धर्म्यं धर्मादनपेतम्~। एवमपि, स्याद्दुःखसम्पाद्यमित्यत आह — सुसुखं कर्तुम्~, यथा रत्नविवेकविज्ञानम्~। तत्र अल्पायासानामन्येषां कर्मणां सुखसम्पाद्यानाम् अल्पफलत्वं दुष्कराणां च महाफलत्वं दृष्टमिति, इदं तु सुखसम्पाद्यत्वात् फलक्षयात् व्येति इति प्राप्ते, आह — अव्ययम् इति~। न अस्य फलतः कर्मवत् व्ययः अस्तीति अव्ययम्~। अतः श्रद्धेयम् आत्मज्ञानम्~॥~२~॥\par
 ये पुनः —} 
\begin{center}{\bfseries अश्रद्दधानाः पुरुषा धर्मस्यास्य परन्तप~।\\अप्राप्य मां निवर्तन्ते मृत्युसंसारवर्त्मनि~॥~३~॥}\end{center} 
अश्रद्दधानाः श्रद्धाविरहिताः आत्मज्ञानस्य धर्मस्य अस्य स्वरूपे तत्फले च नास्तिकाः पापकारिणः, असुराणाम् उपनिषदं देहमात्रात्मदर्शनमेव प्रतिपन्नाः असुतृपः पापाः पुरुषाः अश्रद्दधानाः, परन्तप, अप्राप्य मां परमेश्वरम्~, मत्प्राप्तौ नैव आशङ्का इति मत्प्राप्तिमार्गभेदभक्तिमात्रमपि अप्राप्य इत्यर्थः~। निवर्तन्ते निश्चयेन वर्तन्ते~; क्व~? — मृत्युसंसारवर्त्मनि मृत्युयुक्तः संसारः मृत्युसंसारः तस्य वर्त्म नरकतिर्यगादिप्राप्तिमार्गः, तस्मिन्नेव वर्तन्ते इत्यर्थः~॥~३~॥\par
 स्तुत्या अर्जुनमभिमुखीकृत्य आह —} 
\begin{center}{\bfseries मया ततमिदं सर्वं जगतदव्यक्तमूर्तिना~।\\मत्स्थानि सर्वभूतानि न चाहं तेष्ववस्थितः~॥~४~॥}\end{center} 
मया मम यः परो भावः तेन ततं व्याप्तं सर्वम् इदं जगत् अव्यक्तमूर्तिना न व्यक्ता मूर्तिः स्वरूपं यस्य मम सोऽहमव्यक्तमूर्तिः तेन मया अव्यक्तमूर्तिना, करणागोचरस्वरूपेण इत्यर्थः~। तस्मिन् मयि अव्यक्तमूर्तौ स्थितानि मत्स्थानि, सर्वभूतानि ब्रह्मादीनि स्तम्बपर्यन्तानि~। न हि निरात्मकं किञ्चित् भूतं व्यवहाराय अवकल्पते~। अतः मत्स्थानि मया आत्मना आत्मवत्त्वेन स्थितानि, अतः मयि स्थितानि इति उच्यन्ते~। तेषां भूतानाम् अहमेव आत्मा इत्यतः तेषु स्थितः इति मूढबुद्धीनां अवभासते~; अतः ब्रवीमि — न च अहं तेषु भूतेषु अवस्थितः, मूर्तवत् संश्लेषाभावेन आकाशस्यापि अन्तरतमो हि अहम्~। न हि असंसर्गि वस्तु क्वचित् आधेयभावेन अवस्थितं भवति~॥~४~॥\par
 अत एव असंसर्गित्वात् मम —} 
\begin{center}{\bfseries न च मत्स्थानि भूतानि पश्य मे योगमैश्वरम्~।\\भूतभृन्न च भूतस्थो ममात्मा भूतभावनः~॥~५~॥}\end{center} 
न च मत्स्थानि भूतानि ब्रह्मादीनि~। पश्य मे योगं युक्तिं घटनं मे मम ऐश्वरम् ईश्वरस्य इमम् ऐश्वरम्~, योगम् आत्मनो याथात्म्यमित्यर्थः~। तथा च श्रुतिः असंसर्गित्वात् असङ्गतां दर्शयति — ‘ असङ्गो न हि सज्जते’\footnote{बृ. उ. ३~। ९~। २६} इति~। इदं च आश्चर्यम् अन्यत् पश्य — भूतभृत् असङ्गोऽपि सन् भूतानि बिभर्ति~; न च भूतस्थः, यथोक्तेन न्यायेन दर्शितत्वात् भूतस्थत्वानुपपत्तेः~। कथं पुनरुच्यते ‘असौ मम आत्मा’ इति~? विभज्य देहादिसङ्घातं तस्मिन् अहङ्कारम् अध्यारोप्य लोकबुद्धिम् अनुसरन् व्यपदिशति ‘मम आत्मा’ इति, न पुनः आत्मनः आत्मा अन्यः इति लोकवत् अजानन्~। तथा भूतभावनः भूतानि भावयति उत्पादयति वर्धयतीति वा भूतभावनः~॥~५~॥\par
 यथोक्तेन श्लोकद्वयेन उक्तम् अर्थं दृष्टान्तेन उपपादयन् आह —} 
\begin{center}{\bfseries यथाकाशस्थितो नित्यं वायुः सर्वत्रगो महान्~।\\तथा सर्वाणि भूतानि मत्स्थानीत्युपधारय~॥~६~॥}\end{center} 
यथा लोके आकाशस्थितः आकाशे स्थितः नित्यं सदा वायुः सर्वत्र गच्छतीति सर्वत्रगः महान् परिमाणतः, तथा आकाशवत् सर्वगते मयि असंश्लेषेणैव स्थितानि इत्येवम् उपधारय विजानीहि~॥~६~॥\par
 एवं वायुः आकाशे इव मयि स्थितानि सर्वभूतानि स्थितिकाले~; तानि —} 
\begin{center}{\bfseries सर्वभूतानि कौन्तेय प्रकृतिं यान्ति मामिकाम्~।\\कल्पक्षये पुनस्तानि कल्पादौ विसृजाम्यहम्~॥~७~॥}\end{center} 
सर्वभूतानि कौन्तेय प्रकृतिं त्रिगुणात्मिकाम् अपरां निकृष्टां यान्ति मामिकां मदीयां कल्पक्षये प्रलयकाले~। पुनः भूयः तानि भूतानि उत्पत्तिकाले कल्पादौ विसृजामि उत्पादयामि अहं पूर्ववत्~॥~७~॥\par
 एवम् अविद्यालक्षणाम् —} 
\begin{center}{\bfseries प्रकृतिं स्वामवष्टभ्य विसृजामि पुनः पुनः~।\\भूतग्राममिमं कृत्स्नमवशं प्रकृतेर्वशात्~॥~८~॥}\end{center} 
प्रकृतिं स्वां स्वीयाम् अवष्टभ्य वशीकृत्य विसृजामि पुनः पुनः प्रकृतितो जातं भूतग्रामं भूतसमुदायम् इमं वर्तमानं कृत्स्नं समग्रम् अवशम् अस्वतन्त्रम्~, अविद्यादिदोषैः परवशीकृतम्~, प्रकृतेः वशात् स्वभाववशात्~॥~८~॥\par
 तर्हि तस्य ते परमेश्वरस्य, भूतग्रामम् इमं विषमं विदधतः, तन्निमित्ताभ्यां धर्माधर्माभ्यां सम्बन्धः स्यादिति, इदम् आह भगवान् —} 
\begin{center}{\bfseries न च मां तानि कर्माणि निबध्नन्ति धनञ्जय~।\\उदासीनवदासीनमसक्तं तेषु कर्मसु~॥~९~॥}\end{center} 
न च माम् ईश्वरं तानि भूतग्रामस्य विषमसर्गनिमित्तानि कर्माणि निबध्नन्ति धनञ्जय~। तत्र कर्मणां असम्बन्धित्वे कारणमाह — उदासीनवत् आसीनं यथा उदासीनः उपेक्षकः कश्चित् तद्वत् आसीनम्~, आत्मनः अविक्रियत्वात्~, असक्तं फलासङ्गरहितम्~, अभिमानवर्जितम् ‘अहं करोमि’ इति तेषु कर्मसु~। अतः अन्यस्यापि कर्तृत्वाभिमानाभावः फलासङ्गाभावश्च असम्बन्धकारणम्~, अन्यथा कर्मभिः बध्यते मूढः कोशकारवत् इत्यभिप्रायः~॥~९~॥\par
 तत्र ‘भूतग्राममिमं विसृजामि’\footnote{भ. गी. ९~। ८} ‘उदासीनवदासीनम्’\footnote{भ. गी. ९~। ९} इति च विरुद्धम् उच्यते, इति तत्परिहारार्थम् आह —} 
\begin{center}{\bfseries मयाध्यक्षेण प्रकृतिः सूयते सचराचरम्~।\\हेतुनानेन कौन्तेय जगद्विपरिवर्तते~॥~१०~॥}\end{center} 
मया अध्यक्षेण सर्वतो दृशिमात्रस्वरूपेण अविक्रियात्मना अध्यक्षेण मया, मम माया त्रिगुणात्मिका अविद्यालक्षणा प्रकृतिः सूयते उत्पादयति सचराचरं जगत्~। तथा च मन्त्रवर्णः — ‘एको देवः सर्वभूतेषु गूढः सर्वव्यापी सर्वभूतान्तरात्मा~। कर्माध्यक्षः सर्वभूताधिवासः साक्षी चेता केवलो निर्गुणश्च’\footnote{श्वे. उ. ६~। ११} इति~। हेतुना निमित्तेन अनेन अध्यक्षत्वेन कौन्तेय जगत् सचराचरं व्यक्ताव्यक्तात्मकं विपरिवर्तते सर्वावस्थासु~। दृशिकर्मत्वापत्तिनिमित्ता हि जगतः सर्वा प्रवृत्तिः — अहम् इदं भोक्ष्ये, पश्यामि इदम्~, शृणोमि इदम्~, सुखमनुभवामि, दुःखमनुभवामि, तदर्थमिदं करिष्ये, इदं ज्ञास्यामि, इत्याद्या अवगतिनिष्ठा अवगत्यवसानैव~। ‘यो अस्याध्यक्षः परमे व्योमन्’\footnote{ऋ. १०~। १२९~। ७},\footnote{तै. ब्रा. २~। ८~। ९} इत्यादयश्च मन्त्राः एतमर्थं दर्शयन्ति~। ततश्च एकस्य देवस्य सर्वाध्यक्षभूतचैतन्यमात्रस्य परमार्थतः सर्वभोगानभिसम्बन्धिनः अन्यस्य चेतनान्तरस्य अभावे भोक्तुः अन्यस्य अभावात्~। किंनिमित्ता इयं सृष्टिः इत्यत्र प्रश्नप्रतिवचने अनुपपन्ने, ‘को अद्धा वेद क इह प्रवोचत्~। कुत आजाता कुत इयं विसृष्टिः’\footnote{ऋ. १०~। १२९~। ६},\footnote{तै. ब्रा. २~। ८~। ९} इत्यादिमन्त्रवर्णेभ्यः~। दर्शितं च भगवता — ‘अज्ञानेनावृतं ज्ञानं तेन मुह्यन्ति जन्तवः’\footnote{भ. गी. ५~। १५} इति~॥~१०~॥\par
 एवं मां नित्यशुद्धबुद्धमुक्तस्वभावं सर्वज्ञं सर्वजन्तूनाम् आत्मानमपि सन्तम् —} 
\begin{center}{\bfseries अवजानन्ति मां मूढा मानुषीं तनुमाश्रितम्~।\\परं भावमजानन्तो मम भूतमहेश्वरम्~॥~११~॥}\end{center} 
अवजानन्ति अवज्ञां परिभवं कुर्वन्ति मां मूढाः अविवेकिनः मानुषीं मनुष्यसम्बन्धिनीं तनुं देहम् आश्रितम्~, मनुष्यदेहेन व्यवहरन्तमित्येतत्~, परं प्रकृष्टं भावं परमात्मतत्त्वम् आकाशकल्पम् आकाशादपि अन्तरतमम् अजानन्तो मम भूतमहेश्वरं सर्वभूतानां महान्तम् ईश्वरं स्वात्मानम्~। ततश्च तस्य मम अवज्ञानभावनेन आहताः ते वराकाः~॥~११~॥\par
 कथम्~? —} 
\begin{center}{\bfseries मोघाशा मोघकर्माणो मोघज्ञाना विचेतसः~।\\राक्षसीमासुरीं चैव प्रकृतिं मोहिनीं श्रिताः~॥~१२~॥}\end{center} 
मोघाशाः वृथा आशाः आशिषः येषां ते मोघाशाः, तथा मोघकर्माणः यानि च अग्निहोत्रादीनि तैः अनुष्ठीयमानानि कर्माणि तानि च, तेषां भगवत्परिभवात्~, स्वात्मभूतस्य अवज्ञानात्~, मोघान्येव निष्फलानि कर्माणि भवन्तीति मोघकर्माणः~। तथा मोघज्ञानाः मोघं निष्फलं ज्ञानं येषां ते मोघज्ञानाः, ज्ञानमपि तेषां निष्फलमेव स्यात्~। विचेतसः विगतविवेकाश्च ते भवन्ति इत्यभिप्रायः~। किञ्च — ते भवन्ति राक्षसीं रक्षसां प्रकृतिं स्वभावम् आसुरीम् असुराणां च प्रकृतिं मोहिनीं मोहकरीं देहात्मवादिनीं श्रिताः आश्रिताः, छिन्द्धि, भिन्द्धि, पिब, खाद, परस्वमपहर, इत्येवं वदनशीलाः क्रूरकर्माणो भवन्ति इत्यर्थः, ‘असुर्या नाम ते लोकाः’\footnote{ई. उ. ३} इति श्रुतेः~॥~१२~॥\par
 ये पुनः श्रद्दधानाः भगवद्भक्तिलक्षणे मोक्षमार्गे प्रवृत्ताः —} 
\begin{center}{\bfseries महात्मानस्तु मां पार्थ दैवीं प्रकृतिमाश्रिताः~।\\भजन्त्यनन्यमनसो ज्ञात्वा भूतादिमव्ययम्~॥~१३~॥}\end{center} 
महात्मानस्तु अक्षुद्रचित्ताः माम् ईश्वरं पार्थ दैवीं देवानां प्रकृतिं शमदमदयाश्रद्धादिलक्षणाम् आश्रिताः सन्तः भजन्ति सेवंते अनन्यमनसः अनन्यचित्ताः ज्ञात्वा भूतादिं भूतानां वियदादीनां प्राणिनां च आदिं कारणम् अव्ययम्~॥~१३~॥\par
 कथम्~? —} 
\begin{center}{\bfseries सततं कीर्तयन्तो मां यतन्तश्च दृढव्रताः~।\\नमस्यन्तश्च मां भक्त्या नित्ययुक्ता उपासते~॥~१४~॥}\end{center} 
सततं सर्वदा भगवन्तं ब्रह्मस्वरूपं मां कीर्तयन्तः, यतन्तश्च इन्द्रियोपसंहारशमदमदयाहिंसादिलक्षणैः धर्मैः प्रयतन्तश्च, दृढव्रताः दृढं स्थिरम् अचाल्यं व्रतं येषां ते दृढव्रताः नमस्यन्तश्च मां हृदयेशयम् आत्मानं भक्त्या नित्ययुक्ताः सन्तः उपासते सेवंते~॥~१४~॥\par
 ते केन केन प्रकारेण उपासते इत्युच्यते —} 
\begin{center}{\bfseries ज्ञानयज्ञेन चाप्यन्ये यजन्तो मामुपासते~।\\एकत्वेन पृथक्त्वेन बहुधा विश्वतोमुखम्~॥~१५~॥}\end{center} 
ज्ञानयज्ञेन ज्ञानमेव भगवद्विषयं यज्ञः तेन ज्ञानयज्ञेन, यजन्तः पूजयन्तः माम् ईश्वरं च अपि अन्ये अन्याम् उपासनां परित्यज्य उपासते~। तच्च ज्ञानम् — एकत्वेन ‘एकमेव परं ब्रह्म’ इति परमार्थदर्शनेन यजन्तः उपासते~। केचिच्च पृथक्त्वेन ‘आदित्यचन्द्रादिभेदेन स एव भगवान् विष्णुः अवस्थितः’ इति उपासते~। केचित् ‘बहुधा अवस्थितः स एव भगवान् सर्वतोमुखः विश्वरूपः’ इति तं विश्वरूपं सर्वतोमुखं बहुधा बहुप्रकारेण उपासते~॥~१५~॥\par
 यदि बहुभिः प्रकारैः उपासते, कथं त्वामेव उपासते इति, अत आह —} 
\begin{center}{\bfseries अहं क्रतुरहं यज्ञः स्वधाहमहमौषधम्~।\\ मन्त्रोऽहमहमेवाज्यमहमग्निरहं हुतम्~॥~१६~॥}\end{center} 
अहं क्रतुः श्रौतकर्मभेदः अहमेव~। अहं यज्ञः स्मार्तः~। किञ्च स्वधा अन्नम् अहम्~, पितृभ्यो यत् दीयते~। अहम् औषधं सर्वप्राणिभिः यत् अद्यते तत् औषधशब्दशब्दितं व्रीहियवादिसाधारणम्~। अथवा स्वधा इति सर्वप्राणिसाधारणम् अन्नम्~, औषधम् इति व्याध्युपशमनार्थं भेषजम्~। मन्त्रः अहम्~, येन पितृभ्यो देवताभ्यश्च हविः दीयते~। अहमेव आज्यं हविश्च~। अहम् अग्निः, यस्मिन् हूयते हविः सः अग्निः अहम्~। अहं हुतं हवनकर्म च~॥~१६~॥\par
 किञ्च —} 
\begin{center}{\bfseries पिताहमस्य जगतो माता धाता पितामहः~।\\वेद्यं पवित्रमोङ्कार ऋक्साम यजुरेव च~॥~१७~॥}\end{center} 
पिता जनयिता अहम् अस्य जगतः, माता जनयित्री, धाता कर्मफलस्य प्राणिभ्यो विधाता, पितामहः पितुः पिता, वेद्यं वेदितव्यम्~, पवित्रं पावनम् ओङ्कारः, ऋक् साम यजुः एव च~॥~१७~॥\par
 किञ्च—} 
\begin{center}{\bfseries गतिर्भर्ता प्रभुः साक्षी निवासः शरणं सुहृत्~।\\प्रभवः प्रलयः स्थानं निधानं बीजमव्ययम्~॥~१८~॥}\end{center} 
गतिः कर्मफलम्~, भर्ता पोष्टा, प्रभुः स्वामी, साक्षी प्राणिनां कृताकृतस्य, निवासः यस्मिन् प्राणिनो निवसन्ति, शरणम् आर्तानाम्~, प्रपन्नानामार्तिहरः~। सुहृत् प्रत्युपकारानपेक्षः सन् उपकारी, प्रभवः उत्पत्तिः जगतः, प्रलयः प्रलीयते अस्मिन् इति, तथा स्थानं तिष्ठति अस्मिन् इति, निधानं निक्षेपः कालान्तरोपभोग्यं प्राणिनाम्~, बीजं प्ररोहकारणं प्ररोहधर्मिणाम्~, अव्ययं यावत्संसारभावित्वात् अव्ययम्~, न हि अबीजं किञ्चित् प्ररोहति~; नित्यं च प्ररोहदर्शनात् बीजसन्ततिः न व्येति इति गम्यते~॥~१८~॥\par
 किञ्च —} 
\begin{center}{\bfseries तपाम्यहमहं वर्षं निगृह्णाम्युत्सृजामि च~।\\अमृतं चैव मृत्युश्च सदसच्चाहमर्जुन~॥~१९~॥}\end{center} 
तपामि अहम् आदित्यो भूत्वा कैश्चित् रश्मिभिः उल्बणैः~। अहं वर्षं कैश्चित् रश्मिभिः उत्सृजामि~। उत्सृज्य पुनः निगृह्णामि कैश्चित् रश्मिभिः अष्टभिः मासैः पुनः उत्सृजामि प्रावृषि~। अमृतं चैव देवानाम्~, मृत्युश्च मर्त्यानाम्~। सत् यस्य यत् सम्बन्धितया विद्यमानं तत्~, तद्विपरीतम् असच्च एव अहम् अर्जुन~। न पुनः अत्यन्तमेव असत् भगवान्~, स्वयं कार्यकारणे वा सदसती ये पूर्वोक्तैः निवृत्तिप्रकारैः एकत्वपृथक्त्वादिविज्ञानैः यज्ञैः मां पूजयन्तः उपासते ज्ञानविदः, ते यथाविज्ञानं मामेव प्राप्नुवन्ति~॥~१९~॥\par
  ये पुनः अज्ञाः कामकामाः —} 
\begin{center}{\bfseries त्रैविद्या मां सोमपाः पूतपापा यज्ञैरिष्ट्वा स्वर्गतिं प्रार्थयन्ते~।\\ते पुण्यमासाद्य सुरेन्द्रलोकमश्नन्ति दिव्यान्दिवि देवभोगान्~॥~२०~॥}\end{center} 
त्रैविद्याः ऋग्यजुःसामविदः मां वस्वादिदेवरूपिणं सोमपाः सोमं पिबन्तीति सोमपाः, तेनैव सोमपानेन पूतपापाः शुद्धकिल्बिषाः, यज्ञैः अग्निष्टोमादिभिः इष्ट्वा पूजयित्वा स्वर्गतिं स्वर्गगमनं स्वरेव गतिः स्वर्गतिः ताम्~, प्रार्थयन्ते~। ते च पुण्यं पुण्यफलम् आसाद्य सम्प्राप्य सुरेन्द्रलोकं शतक्रतोः स्थानम् अश्नन्ति भुञ्जते दिव्यान् दिवि भवान् अप्राकृतान् देवभोगान् देवानां भोगान्~॥~२०~॥\par
 \begin{center}{\bfseries ते तं भुक्त्वा स्वर्गलोकं विशालं क्षीणे पुण्ये मर्त्यलोकं विशन्ति~।\\एवं त्रयीधर्ममनुप्रपन्ना गतागतं कामकामा लभन्ते~॥~२१~॥}\end{center} 
ते तं भुक्त्वा स्वर्गलोकं विशालं विस्तीर्णं क्षीणे पुण्ये मर्त्यलोकं विशन्ति आविशन्ति~। एवं यथोक्तेन प्रकारेण त्रयीधर्मं केवलं वैदिकं कर्म अनुप्रपन्नाः गतागतं गतं च आगतं च गतागतं गमनागमनं कामकामाः कामान् कामयन्ते इति कामकामाः लभन्ते गतागतमेव, न तु स्वातन्त्र्यं क्वचित् लभन्ते इत्यर्थः~॥~२१~॥\par
 ये पुनः निष्कामाः सम्यग्दर्शिनः —} 
\begin{center}{\bfseries अनन्याश्चिन्तयन्तो मां ये जनाः पर्युपासते~।\\तेषां नित्याभियुक्तानां योगक्षेमं वहाम्यहम्~॥~२२~॥}\end{center} 
अनन्याः अपृथग्भूताः परं देवं नारायणम् आत्मत्वेन गताः सन्तः चिन्तयन्तः मां ये जनाः संन्यासिनः पर्युपासते, तेषां परमार्थदर्शिनां नित्याभियुक्तानां सतताभियोगिनां योगक्षेमं योगः अप्राप्तस्य प्रापणं क्षेमः तद्रक्षणं तदुभयं वहामि प्रापयामि अहम्~; ‘ज्ञानी त्वात्मैव मे मतम्’\footnote{भ. गी. ७~। १८} ‘स च मम प्रियः’\footnote{भ. गी. ७~। १७} यस्मात्~, तस्मात् ते मम आत्मभूताः प्रियाश्च इति~॥~} 
ननु अन्येषामपि भक्तानां योगक्षेमं वहत्येव भगवान्~। सत्यं वहत्येव~; किन्तु अयं विशेषः — अन्ये ये भक्ताः ते आत्मार्थं स्वयमपि योगक्षेमम् ईहन्ते~; अनन्यदर्शिनस्तु न आत्मार्थं योगक्षेमम् ईहन्ते~; न हि ते जीविते मरणे वा आत्मनः गृद्धिं कुर्वन्ति~; केवलमेव भगवच्छरणाः ते~; अतः भगवानेव तेषां योगक्षेमं वहतीति~॥~२२~॥\par
 ननु अन्या अपि देवताः त्वमेव चेत्~, तद्भक्ताश्च त्वामेव यजन्ते~। सत्यमेवम् —} 
\begin{center}{\bfseries येऽप्यन्यदेवताभक्ता यजन्ते श्रद्धयान्विताः~।\\तेऽपि मामेव कौन्तेय यजन्त्यविधिपूर्वकम्~॥~२३~॥}\end{center} 
येऽपि अन्यदेवताभक्ताः अन्यासु देवतासु भक्ताः अन्यदेवताभक्ताः सन्तः यजन्ते पूजयन्ति श्रद्धया आस्तिक्यबुद्ध्या अन्विताः अनुगताः, तेऽपि मामेव कौन्तेय यजन्ति अविधिपूर्वकम् अविधिः अज्ञानं तत्पूर्वकं यजन्ते इत्यर्थः~॥~२३~॥\par
 कस्मात् ते अविधिपूर्वकं यजन्ते इत्युच्यते~; यस्मात् —} 
\begin{center}{\bfseries अहं हि सर्वयज्ञानां भोक्ता च प्रभुरेव च~।\\न तु मामभिजानन्ति तत्त्वेनातश्च्यवन्ति ते~॥~२४~॥}\end{center} 
अहं हि सर्वयज्ञानां श्रौतानां स्मार्तानां च सर्वेषां यज्ञानां देवतात्मत्वेन भोक्ता च प्रभुः एव च~। मत्स्वामिको हि यज्ञः, ‘अधियज्ञोऽहमेवात्र’\footnote{भ. गी. ८~। ४} इति हि उक्तम्~। तथा न तु माम् अभिजानन्ति तत्त्वेन यथावत्~। अतश्च अविधिपूर्वकम् इष्ट्वा यागफलात् च्यवन्ति प्रच्यवन्ते ते~॥~२४~॥\par
 येऽपि अन्यदेवताभक्तिमत्त्वेन अविधिपूर्वकं यजन्ते, तेषामपि यागफलं अवश्यंभावि~। कथम्~? —} 
\begin{center}{\bfseries यान्ति देवव्रता देवान्पितॄन्यान्ति पितृव्रताः~।\\भूतानि यान्ति भूतेज्या यान्ति मद्याजिनोऽपि माम्~॥~२५~॥}\end{center} 
यान्ति गच्छन्ति देवव्रताः देवेषु व्रतं नियमो भक्तिश्च येषां ते देवव्रताः देवान् यान्ति~। पितॄन् अग्निष्वात्तादीन् यान्ति पितृव्रताः श्राद्धादिक्रियापराः पितृभक्ताः~। भूतानि विनायकमातृगणचतुर्भगिन्यादीनि यान्ति भूतेज्याः भूतानां पूजकाः~। यान्ति मद्याजिनः मद्यजनशीलाः वैष्णवाः मामेव यान्ति~। समाने अपि आयासे मामेव न भजन्ते अज्ञानात्~, तेन ते अल्पफलभाजः भवन्ति इत्यर्थः~॥~२५~॥\par
 न केवलं मद्भक्तानाम् अनावृत्तिलक्षणम् अनन्तफलम्~, सुखाराधनश्च अहम्~। कथम्~? —} 
\begin{center}{\bfseries पत्रं पुष्पं फलं तोयं यो मे भक्त्या प्रयच्छति~।\\तदहं भक्त्युपहृतमश्नामि प्रयतात्मनः~॥~२६~॥}\end{center} 
पत्रं पुष्पं फलं तोयम् उदकं यः मे मह्यं भक्त्या प्रयच्छति, तत् अहं पत्रादि भक्त्या उपहृतं भक्तिपूर्वकं प्रापितं भक्त्युपहृतम् अश्नामि गृह्णामि प्रयतात्मनः शुद्धबुद्धेः~॥~२६~॥\par
 यतः एवम्~, अतः —} 
\begin{center}{\bfseries यत्करोषि यदश्नासि यज्जुहोषि ददासि यत्~।\\यत्तपस्यसि कौन्तेय तत्कुरुष्व मदर्पणम्~॥~२७~॥}\end{center} 
यत् करोषि स्वतः प्राप्तम्~, यत् अश्नासि, यच्च जुहोषि हवनं निर्वर्तयसि श्रौतं स्मार्तं वा, यत् ददासि प्रयच्छसि ब्राह्मणादिभ्यः हिरण्यान्नाज्यादि, यत् तपस्यसि तपः चरसि कौन्तेय, तत् कुरुष्व मदर्पणं मत्समर्पणम्~॥~२७~॥\par
 एवं कुर्वतः तव यत् भवति, तत् शृणु —} 
\begin{center}{\bfseries शुभाशुभफलैरेवं मोक्ष्यसे कर्मबन्धनैः~।\\संन्यासयोगयुक्तात्मा विमुक्तो मामुपैष्यसि~॥~२८~॥}\end{center} 
शुभाशुभफलैः शुभाशुभे इष्टानिष्टे फले येषां तानि शुभाशुभफलानि कर्माणि तैः शुभाशुभफलैः कर्मबन्धनैः कर्माण्येव बन्धनानि कर्मबन्धनानि तैः कर्मबन्धनैः एवं मदर्पणं कुर्वन् मोक्ष्यसे~। सोऽयं संन्यासयोगो नाम, संन्यासश्च असौ मत्समर्पणतया कर्मत्वात् योगश्च असौ इति, तेन संन्यासयोगेन युक्तः आत्मा अन्तःकरणं यस्य तव सः त्वं संन्यासयोगयुक्तात्मा सन् विमुक्तः कर्मबन्धनैः जीवन्नेव पतिते चास्मिन् शरीरे माम् उपैष्यसि आगमिष्यसि~॥~२८~॥\par
 रागद्वेषवान् तर्हि भगवान्~, यतो भक्तान् अनुगृह्णाति, न इतरान् इति~। तत् न —} 
\begin{center}{\bfseries समोऽहं सर्वभूतेषु न मे द्वेष्योऽस्ति न प्रियः~।\\ये भजन्ति तु मां भक्त्या मयि ते तेषु चाप्यहम्~॥~२९~॥}\end{center} 
समः तुल्यः अहं सर्वभूतेषु~। न मे द्वेष्यः अस्ति न प्रियः~। अग्निवत् अहम् — दूरस्थानां यथा अग्निः शीतं न अपनयति, समीपम् उपसर्पतां अपनयति~; तथा अहं भक्तान् अनुगृह्णामि, न इतरान्~। ये भजन्ति तु माम् ईश्वरं भक्त्या मयि ते — स्वभावत एव, न मम रागनिमित्तम् — वर्तन्ते~। तेषु च अपि अहं स्वभावत एव वर्ते, न इतरेषु~। न एतावता तेषु द्वेषो मम~॥~२९~॥\par
 शृणु मद्भक्तेर्माहात्म्यम् —} 
\begin{center}{\bfseries अपि चेत्सुदुराचारो भजते मामनन्यभाक्~।\\साधुरेव स मन्तव्यः सम्यग्व्यवसितो हि सः~॥~३०~॥}\end{center} 
अपि चेत् यद्यपि सुदुराचारः सुष्ठु दुराचारः अतीव कुत्सिताचारोऽपि भजते माम् अनन्यभाक् अनन्यभक्तिः सन्~, साधुरेव सम्यग्वृत्त एव सः मन्तव्यः ज्ञातव्यः~; सम्यक् यथावत् व्यवसितो हि सः, यस्मात् साधुनिश्चयः सः~॥~३०~॥\par
 उत्सृज्य च बाह्यां दुराचारतां अन्तः सम्यग्व्यवसायसामर्थ्यात् —} 
\begin{center}{\bfseries क्षिप्रं भवति धर्मात्मा शश्वच्छान्तिं निगच्छति~।\\कौन्तेय प्रतिजानीहि न मे भक्तः प्रणश्यति~॥~३१~॥}\end{center} 
क्षिप्रं शीघ्रं भवति धर्मात्मा धर्मचित्तः एव~। शश्वत् नित्यं शान्तिं च उपशमं निगच्छति प्राप्नोति~। शृणु परमार्थम्~, कौन्तेय प्रतिजानीहि निश्चितां प्रतिज्ञां कुरु, न मे मम भक्तः मयि समर्पितान्तरात्मा मद्भक्तः न प्रणश्यति इति~॥~३१~॥\par
 किञ्च —} 
\begin{center}{\bfseries मां हि पार्थ व्यपाश्रित्य येऽपि स्युः पापयोनयः~।\\स्त्रियो वैश्यास्तथा शूद्रास्तेऽपि यान्ति परां गतिम्~॥~३२~॥}\end{center} 
मां हि यस्मात् पार्थ व्यपाश्रित्य माम् आश्रयत्वेन गृहीत्वा येऽपि स्युः भवेयुः पापयोनयः पापा योनिः येषां ते पापयोनयः पापजन्मानः~। के ते इति, आह — स्त्रियः वैश्याः तथा शूद्राः तेऽपि यान्ति गच्छन्ति परां प्रकृष्टां गतिम्~॥~३२~॥\par
 \begin{center}{\bfseries किं पुनर्ब्राह्मणाः पुण्या भक्ता राजर्षयस्तथा~।\\अनित्यमसुखं लोकमिमं प्राप्य भजस्व माम्~॥~३३~॥}\end{center} 
किं पुनः ब्राह्मणाः पुण्याः पुण्ययोनयः भक्ताः राजर्षयः तथा~। राजानश्च ते ऋषयश्च राजर्षयः~। यतः एवम्~, अतः अनित्यं क्षणभङ्गुरम् असुखं च सुखवर्जितम् इमं लोकं मनुष्यलोकं प्राप्य पुरुषार्थसाधनं दुर्लभं मनुष्यत्वं लब्ध्वा भजस्व सेवस्व माम्~॥~३३~॥\par
 कथम् —} 
\begin{center}{\bfseries मन्मना भव मद्भक्तो मद्याजी मां नमस्कुरु~।\\मामेवैष्यसि युक्त्वैवमात्मानं मत्परायणः~॥~३४~॥}\end{center} 
मयि वासुदेवे मनः यस्य तव स त्वं मन्मनाः भव~। तथा मद्भक्तः भव मद्याजी मद्यजनशीलः भव~। माम् एव च नमस्कुरु~। माम् एव ईश्वरम् एष्यसि आगमिष्यसि युक्त्वा समाधाय चित्तम्~। एवम् आत्मानम्~, अहं हि सर्वेषां भूतानाम् आत्मा, परा च गतिः, परम् अयनम्~, तं माम् एवंभूतम्~, एष्यसि इति अतीतेन सम्बन्धः, मत्परायणः सन् इत्यर्थः~॥~३४~॥\par
 
इति श्रीमत्परमहंसपरिव्राजकाचार्यस्य श्रीगोविन्दभगवत्पूज्यपादशिष्यस्य श्रीमच्छङ्करभगवतः कृतौ श्रीमद्भगवद्गीताभाष्ये नवमोऽध्यायः~॥\par
 
सप्तमे अध्याये भगवतस्तत्त्वं विभूतयश्च प्रकाशिताः, नवमे च~। अथ इदानीं येषु येषु भावेषु चिन्त्यो भगवान्~, ते ते भावा वक्तव्याः, तत्त्वं च भगवतो वक्तव्यम् उक्तमपि, दुर्विज्ञेयत्वात्~, इत्यतः श्रीभगवानुवाच —}\\ 
\begin{center}{\bfseries श्रीभगवानुवाच —\\ भूय एव महाबाहो शृणु मे परमं वचः~।\\यत्तेऽहं प्रीयमाणाय वक्ष्यामि हितकाम्यया~॥~१~॥}\end{center} 
भूयः एव भूयः पुनः हे महाबाहो शृणु मे मदीयं परमं प्रकृष्टं निरतिशयवस्तुनः प्रकाशकं वचः वाक्यं यत् परमं ते तुभ्यं प्रीयमाणाय — मद्वचनात् प्रीयसे त्वम् अतीव अमृतमिव पिबन्~, ततः — वक्ष्यामि हितकाम्यया हितेच्छया~॥~१~॥\par
 किमर्थम् अहं वक्ष्यामि इत्यत आह – } 
\begin{center}{\bfseries न मे विदुः सुरगणाः प्रभवं न महर्षयः~।\\अहमादिर्हि देवानां महर्षीणां च सर्वशः~॥~२~॥}\end{center} 
न मे विदुः न जानन्ति सुरगणाः ब्रह्मादयः~। किं ते न विदुः~? मम प्रभवं प्रभावं प्रभुशक्त्यतिशयम्~, अथवा प्रभवं प्रभवनम् उत्पत्तिम्~। नापि महर्षयः भृग्वादयः विदुः~। कस्मात् ते न विदुरित्युच्यते — अहम् आदिः कारणं हि यस्मात् देवानां महर्षीणां च सर्वशः सर्वप्रकारैः~॥~२~॥\par
 किञ्च —} 
\begin{center}{\bfseries यो मामजमनादिं च वेत्ति लोकमहेश्वरम्~।\\असंमूढः स मर्त्येषु सर्वपापैः प्रमुच्यते~॥~३~॥}\end{center} 
यः माम् अजम् अनादिं च, यस्मात् अहम् आदिः देवानां महर्षीणां च, न मम अन्यः आदिः विद्यते~; अतः अहम् अजः अनादिश्च~; अनादित्वम् अजत्वे हेतुः, तं माम् अजम् अनादिं च यः वेत्ति विजानाति लोकमहेश्वरं लोकानां महान्तम् ईश्वरं तुरीयम् अज्ञानतत्कार्यवर्जितम् असंमूढः संमोहवर्जितः सः मर्त्येषु मनुष्येषु, सर्वपापैः सर्वैः पापैः मतिपूर्वामतिपूर्वकृतैः प्रमुच्यते प्रमोक्ष्यते~॥~३~॥\par
 इतश्चाहं महेश्वरो लोकानाम् —} 
\begin{center}{\bfseries बुद्धिर्ज्ञानमसंमोहः क्षमा सत्यं दमः शमः~।\\सुखं दुःखं भवोऽभावो भयं चाभयमेव च~॥~४~॥}\end{center} 
बुद्धिः अन्तःकरणस्य सूक्ष्माद्यर्थावबोधनसामर्थ्यम्~, तद्वन्तं बुद्धिमानिति हि वदन्ति~। ज्ञानम् आत्मादिपदार्थानामवबोधः~। असंमोहः प्रत्युत्पन्नेषु बोद्धव्येषु विवेकपूर्विका प्रवृत्तिः~। क्षमा आक्रुष्टस्य ताडितस्य वा अविकृतचित्तता~। सत्यं यथादृष्टस्य यथाश्रुतस्य च आत्मानुभवस्य परबुद्धिसङ्क्रान्तये तथैव उच्चार्यमाणा वाक् सत्यम् उच्यते~। दमः बाह्येन्द्रियोपशमः~। शमः अन्तःकरणस्य उपशमः~। सुखम् आह्लादः~। दुःखं सन्तापः~। भवः उद्भवः~। अभावः तद्विपर्ययः~। भयं च त्रासः, अभयमेव च तद्विपरीतम्~॥~४~॥\par
 \begin{center}{\bfseries अहिंसा समता तुष्टिस्तपो दानं यशोऽयशः~।\\भवन्ति भावा भूतानां मत्त एव पृथग्विधाः~॥~५~॥}\end{center} 
अहिंसा अपीडा प्राणिनाम्~। समता समचित्तता~। तुष्टिः सन्तोषः पर्याप्तबुद्धिर्लाभेषु~। तपः इन्द्रियसंयमपूर्वकं शरीरपीडनम्~। दानं यथाशक्ति संविभागः~। यशः धर्मनिमित्ता कीर्तिः~। अयशस्तु अधर्मनिमित्ता अकीर्तिः~। भवन्ति भावाः यथोक्ताः बुद्ध्यादयः भूतानां प्राणिनां मत्तः एव ईश्वरात् पृथग्विधाः नानाविधाः स्वकर्मानुरूपेण~॥~५~॥\par
 किञ्च —} 
\begin{center}{\bfseries महर्षयः सप्त पूर्वे चत्वारो मनवस्तथा~।\\मद्भावा मानसा जाता येषां लोक इमाः प्रजाः~॥~६~॥}\end{center} 
महर्षयः सप्त भृग्वादयः पूर्वे अतीतकालसम्बन्धिनः, चत्वारः मनवः तथा सावर्णा इति प्रसिद्धाः, ते च मद्भावाः मद्गतभावनाः वैष्णवेन सामर्थ्येन उपेताः, मानसाः मनसैव उत्पादिताः मया जाताः उत्पन्नाः, येषां मनूनां महर्षीणां च सृष्टिः लोके इमाः स्थावरजङ्गमलक्षणाः प्रजाः~॥~६~॥\par
 \begin{center}{\bfseries एतां विभूतिं योगं च मम यो वेत्ति तत्त्वतः~।\\सोऽविकम्पेन योगेन युज्यते नात्र संशयः~॥~७~॥}\end{center} 
एतां यथोक्तां विभूतिं विस्तारं योगं च युक्तिं च आत्मनः घटनम्~, अथवा योगैश्वर्यसामर्थ्यं सर्वज्ञत्वं योगजं योगः उच्यते, मम मदीयं योगं यः वेत्ति तत्त्वतः तत्त्वेन यथावदित्येतत्~, सः अविकम्पेन अप्रचलितेन योगेन सम्यग्दर्शनस्थैर्यलक्षणेन युज्यते सम्बध्यते~। न अत्र संशयः न अस्मिन् अर्थे संशयः अस्ति~॥~७~॥\par
 कीदृशेन अविकम्पेन योगेन युज्यते इत्युच्यते —} 
\begin{center}{\bfseries अहं सर्वस्य प्रभवो मत्तः सर्वं प्रवर्तते~।\\इति मत्वा भजन्ते मां बुधा भावसमन्विताः~॥~८~॥}\end{center} 
अहं परं ब्रह्म वासुदेवाख्यं सर्वस्य जगतः प्रभवः उत्पत्तिः~। मत्तः एव स्थितिनाशक्रियाफलोपभोगलक्षणं विक्रियारूपं सर्वं जगत् प्रवर्तते~। इति एवं मत्वा भजन्ते सेवंते मां बुधाः अवगतपरमार्थतत्त्वाः, भावसमन्विताः भावः भावना परमार्थतत्त्वाभिनिवेशः तेन समन्विताः संयुक्ताः इत्यर्थः~॥~८~॥\par
 किञ्च —} 
\begin{center}{\bfseries मच्चित्ता मद्गतप्राणा बोधयन्तः परस्परम्~।\\कथयन्तश्च मां नित्यं तुष्यन्ति च रमन्ति च~॥~९~॥}\end{center} 
मच्चित्ताः, मयि चित्तं येषां ते मच्चित्ताः, मद्गतप्राणाः मां गताः प्राप्ताः चक्षुरादयः प्राणाः येषां ते मद्गतप्राणाः, मयि उपसंहृतकरणाः इत्यर्थः~। अथवा, मद्गतप्राणाः मद्गतजीवनाः इत्येतत्~। बोधयन्तः अवगमयन्तः परस्परम् अन्योन्यम्~, कथयन्तश्च ज्ञानबलवीर्यादिधर्मैः विशिष्टं माम्~, तुष्यन्ति च परितोषम् उपयान्ति च रमन्ति च रतिं च प्राप्नुवन्ति प्रियसङ्गत्येव~॥~९~॥\par
 ये यथोक्तैः प्रकारैः भजन्ते मां भक्ताः सन्तः —} 
\begin{center}{\bfseries तेषां सततयुक्तानां भजतां प्रीतिपूर्वकम्~।\\ददामि बुद्धियोगं तं येन मामुपयान्ति ते~॥~१०~॥}\end{center} 
तेषां सततयुक्तानां नित्याभियुक्तानां निवृत्तसर्वबाह्यैषणानां भजतां सेवमानानाम्~। किम् अर्थित्वादिना कारणेन~? नेत्याह — प्रीतिपूर्वकं प्रीतिः स्नेहः तत्पूर्वकं मां भजतामित्यर्थः~। ददामि प्रयच्छामि बुद्धियोगं बुद्धिः सम्यग्दर्शनं मत्तत्त्वविषयं तेन योगः बुद्धियोगः तं बुद्धियोगम्~, येन बुद्धियोगेन सम्यग्दर्शनलक्षणेन मां परमेश्वरम् आत्मभूतम् आत्मत्वेन उपयान्ति प्रतिपद्यन्ते~। के~? ते ये मच्चित्तत्वादिप्रकारैः मां भजन्ते~॥~१०~॥\par
 किमर्थम्~, कस्य वा, त्वत्प्राप्तिप्रतिबन्धहेतोः नाशकं बुद्धियोगं तेषां त्वद्भक्तानां ददासि इत्यपेक्षायामाह —} 
\begin{center}{\bfseries तेषामेवानुकम्पार्थमहमज्ञानजं तमः~।\\नाशयाम्यात्मभावस्थो ज्ञानदीपेन भास्वता~॥~११~॥}\end{center} 
तेषामेव कथं नु नाम श्रेयः स्यात् इति अनुकम्पार्थं दयाहेतोः अहम् अज्ञानजम् अविवेकतः जातं मिथ्याप्रत्ययलक्षणं मोहान्धकारं तमः नाशयामि, आत्मभावस्थः आत्मनः भावः अन्तःकरणाशयः तस्मिन्नेव स्थितः सन् ज्ञानदीपेन विवेकप्रत्ययरूपेण भक्तिप्रसादस्नेहाभिषिक्तेन मद्भावनाभिनिवेशवातेरितेन ब्रह्मचर्यादिसाधनसंस्कारवत्प्रज्ञावर्तिना विरक्तान्तःकरणाधारेण विषयव्यावृत्तचित्तरागद्वेषाकलुषितनिवातापवरकस्थेन नित्यप्रवृत्तैकाग्र्यध्यानजनितसम्यग्दर्शनभास्वता ज्ञानदीपेनेत्यर्थः~॥~११~॥\par
 यथोक्तां भगवतः विभूतिं योगं च श्रुत्वा अर्जुन उवाच —}\\ 
\begin{center}{\bfseries अर्जुन उवाच —\\ परं ब्रह्म परं धाम पवित्रं परमं भवान्~।\\पुरुषं शाश्वतं दिव्यमादिदेवमजं विभुम्~॥~१२~॥}\end{center} 
परं ब्रह्म परमात्मा परं धाम परं तेजः पवित्रं पावनं परमं प्रकृष्टं भवान्~। पुुुरुषं शाश्वतं नित्यं दिव्यं दिवि भवम् आदिदेवं सर्वदेवानाम् आदौ भवम् आदिदेवम् अजं विभुं विभवनशीलम्~॥~१२~॥\par
 ईदृशम् —} 
\begin{center}{\bfseries आहुस्त्वामृषयः सर्वे देवर्षिर्नारदस्तथा~।\\असितो देवलो व्यासः स्वयं चैव ब्रवीषि मे~॥~१३~॥}\end{center} 
आहुः कथयन्ति त्वाम् ऋषयः वसिष्ठादयः सर्वे देवर्षिः नारदः तथा~। असितः देवलोऽपि एवमेवाह, व्यासश्च, स्वयं चैव त्वं च ब्रवीषि मे~॥~१३~॥\par
 \begin{center}{\bfseries सर्वमेतदृतं मन्ये यन्मां वदसि केशव~।\\न हि ते भगवन्व्यक्तिं विदुर्देवा न दानवाः~॥~१४~॥}\end{center} 
सर्वमेतत् यथोक्तम् ऋषिभिः त्वया च एतत् ऋतं सत्यमेव मन्ये, यत् मां प्रति वदसि भाषसे हे केशव~। न हि ते तव भगवन् व्यक्तिं प्रभवं विदुः न देवाः, न दानवाः~॥~१४~॥\par
 यतः त्वं देवादीनाम् आदिः, अतः —} 
\begin{center}{\bfseries स्वयमेवात्मनात्मानं वेत्थ त्वं पुरुषोत्तम~।\\भूतभावन भूतेश देवदेव जगत्पते~॥~१५~॥}\end{center} 
स्वयमेव आत्मना आत्मानं वेत्थ जानासि त्वं निरतिशयज्ञानैश्वर्यबलादिशक्तिमन्तम् ईश्वरं पुरुषोत्तम~। भूतानि भावयतीति भूतभावनः, हे भूतभावन~। भूतेश भूतानाम् ईशितः~। हे देवदेव जगत्पते~॥~१५~॥\par
 \begin{center}{\bfseries वक्तुमर्हस्यशेषेण\\ दिव्या ह्यात्मविभूतयः~।\\याभिर्विभूतिभिर्लोका—\\ निमांस्त्वं व्याप्य तिष्ठसि~॥~१६~॥}\end{center} 
वक्तुं कथयितुम् अर्हसि अशेषेण~। दिव्याः हि आत्मविभूतयः~। आत्मनो विभूतयो याः ताः वक्तुम् अर्हसि~। याभिः विभूतिभिः आत्मनो माहात्म्यविस्तरैः इमान् लोकान् त्वं व्याप्य तिष्ठसि~॥~१६~॥\par
 \begin{center}{\bfseries कथं विद्यामहं योगिंस्त्वां सदा परिचिन्तयन्~।\\केषु केषु च भावेषु चिन्त्योऽसि भगवन्मया~॥~१७~॥}\end{center} 
कथं विद्यां विजानीयाम् अहं हे योगिन् त्वां सदा परिचिन्तयन्~। केषु केषु च भावेषु वस्तुषु चिन्त्यः असि ध्येयः असि भगवन् मया~॥~१७~॥\par
 \begin{center}{\bfseries विस्तरेणात्मनो योगं विभूतिं च जनार्दन~।\\भूयः कथय तृप्तिर्हि शृण्वतो नास्ति मेऽमृतम्~॥~१८~॥}\end{center} 
विस्तरेण आत्मनः योगं योगैश्वर्यशक्तिविशेषं विभूतिं च विस्तरं ध्येयपदार्थानां हे जनार्दन, अर्दतेः गतिकर्मणः रूपम्~, असुराणां देवप्रतिपक्षभूतानां जनानां नरकादिगमयितृत्वात् जनार्दनः अभ्युदयनिःश्रेयसपुरुषार्थप्रयोजनं सर्वैः जनैः याच्यते इति वा~। भूयः पूर्वम् उक्तमपि कथय~; तृप्तिः परितोषः हि यस्मात् नास्ति मे मम शृण्वतः त्वन्मुखनिःसृतवाक्यामृतम्~॥~१८~॥\par
 {\bfseries श्रीभगवानुवाच —}\\
\begin{center}{\bfseries हन्त ते कथयिष्यामि दिव्या ह्यात्मविभूतयः~।\\प्राधान्यतः कुरुश्रेष्ठ नास्त्यन्तो विस्तरस्य मे~॥~१९~॥}\end{center} 
हन्त इदानीं ते तव दिव्याः दिवि भवाः आत्मविभूतयः आत्मनः मम विभूतयः याः ताः कथयिष्यामि इत्येतत्~। प्राधान्यतः यत्र यत्र प्रधाना या या विभूतिः तां तां प्रधानां प्राधान्यतः कथयिष्यामि अहं कुरुश्रेष्ठ~। अशेषतस्तु वर्षशतेनापि न शक्या वक्तुम्~, यतः नास्ति अन्तः विस्तरस्य मे मम विभूतीनाम् इत्यर्थः~॥~१९~॥\par
 तत्र प्रथममेव तावत् शृणु —} 
\begin{center}{\bfseries अहमात्मा गुडाकेश सर्वभूताशयस्थितः~।\\अहमादिश्च मध्यं च भूतानामन्त एव च~॥~२०~॥}\end{center} 
अहम् आत्मा प्रत्यगात्मा गुडाकेश, गुडाका निद्रा तस्याः ईशः गुडाकेशः, जितनिद्रः इत्यर्थः~; घनकेश इति वा~। सर्वभूताशयस्थितः सर्वेषां भूतानाम् आशये अन्तर्हृदि स्थितः अहम् आत्मा प्रत्यगात्मा नित्यं ध्येयः~। तदशक्तेन च उत्तरेषु भावेषु चिन्त्यः अहम्~; यस्मात् अहम् एव आदिः भूतानां कारणं तथा मध्यं च स्थितिः अन्तः प्रलयश्च~॥~२०~॥\par
 एवं च ध्येयोऽहम् —} 
\begin{center}{\bfseries आदित्यानामहं विष्णुर्ज्योतिषां रविरंशुमान्~।\\मरीचिर्मरुतामस्मि नक्षत्राणामहं शशी~॥~२१~॥}\end{center} 
आदित्यानां द्वादशानां विष्णुः नाम आदित्यः अहम्~। ज्योतिषां रविः प्रकाशयितॄणाम् अंशुमान् रश्मिमान्~। मरीचिः नाम मरुतां मरुद्देवताभेदानाम् अस्मि~। नक्षत्राणाम् अहं शशी चन्द्रमाः~॥~२१~॥\par
 \begin{center}{\bfseries वेदानां सामवेदोऽस्मि देवानामस्मि वासवः~।\\इन्द्रियाणां मनश्चास्मि भूतानामस्मि चेतना~॥~२२~॥}\end{center} 
वेदानां मध्ये सामवेदः अस्मि~। देवानां रुद्रादित्यादीनां वासवः इन्द्रः अस्मि~। इन्द्रियाणां एकादशानां चक्षुरादीनां मनश्च अस्मि सङ्कल्पविकल्पात्मकं मनश्चास्मि~। भूतानाम् अस्मि चेतना कार्यकरणसङ्घाते नित्याभिव्यक्ता बुद्धिवृत्तिः चेतना~॥~२२~॥\par
 \begin{center}{\bfseries रुद्राणां शङ्करश्चास्मि वित्तेशो यक्षरक्षसाम्~।\\वसूनां पावकश्चास्मि मेरुः शिखरिणामहम्~॥~२३~॥}\end{center} 
रुद्राणाम् एकादशानां शङ्करश्च अस्मि~। वित्तेशः कुबेरः यक्षरक्षसां यक्षाणां रक्षसां च~। वसूनाम् अष्टानां पावकश्च अस्मि अग्निः~। मेरुः शिखरिणां शिखरवताम् अहम्~॥~२३~॥\par
 \begin{center}{\bfseries पुरोधसां च मुख्यं मां विद्धि पार्थ बृहस्पतिम्~।\\सेनानीनामहं स्कन्दः सरसामस्मि सागरः~॥~२४~॥}\end{center} 
पुरोधसां च राजपुरोहितानां च मुख्यं प्रधानं मां विद्धि हे पार्थ बृहस्पतिम्~। स हि इन्द्रस्येति मुख्यः स्यात् पुरोधाः~। सेनानीनां सेनापतीनाम् अहं स्कन्दः देवसेनापतिः~। सरसां यानि देवखातानि सरांसि तेषां सरसां सागरः अस्मि भवामि~॥~२४~॥\par
 \begin{center}{\bfseries महर्षीणां भृगुरहं गिरामस्म्येकमक्षरम्~।\\यज्ञानां जपयज्ञोऽस्मि स्थावराणां हिमालयः~॥~२५~॥}\end{center} 
महर्षीणां भृगुः अहम्~। गिरां वाचां पदलक्षणानाम् एकम् अक्षरम् ओङ्कारः अस्मि~। यज्ञानां जपयज्ञः अस्मि, स्थावराणां स्थितिमतां हिमालयः~॥~२५~॥\par
 \begin{center}{\bfseries अश्वत्थः सर्ववृक्षाणां देवर्षीणां च नारदः~।\\गन्धर्वाणां चित्ररथः सिद्धानां कपिलो मुनिः~॥~२६~॥}\end{center} 
अश्वत्थः सर्ववृक्षाणाम्~, देवर्षीणां च नारदः देवाः एव सन्तः ऋषित्वं प्राप्ताः मन्त्रदर्शित्वात्ते देवर्षयः, तेषां नारदः अस्मि~। गन्धर्वाणां चित्ररथः नाम गन्धर्वः अस्मि~। सिद्धानां जन्मनैव धर्मज्ञानवैराग्यैश्वर्यातिशयं प्राप्तानां कपिलो मुनिः~॥~२६~॥\par
 \begin{center}{\bfseries उच्चैःश्रवसमश्वानां विद्धि माममृतोद्भवम्~।\\ऐरावतं गजेन्द्राणां नराणां च नराधिपम्~॥~२७~॥}\end{center} 
उच्चैःश्रवसम् अश्वानां उच्चैःश्रवाः नाम अश्वराजः तं मां विद्धि विजानीहि अमृतोद्भवम् अमृतनिमित्तमथनोद्भवम्~। ऐरावतम् इरावत्याः अपत्यं गजेन्द्राणां हस्तीश्वराणाम्~, तम् ‘मां विद्धि’ इति अनुवर्तते~। नराणां च मनुष्याणां नराधिपं राजानं मां विद्धि जानीहि~॥~२७~॥\par
 \begin{center}{\bfseries आयुधानामहं वज्रं धेनूनामस्मि कामधुक्~।\\प्रजनश्चास्मि कन्दर्पः सर्पाणामस्मि वासुकिः~॥~२८~॥}\end{center} 
आयुधानाम् अहं वज्रं दधीच्यस्थिसम्भवम्~। धेनूनां दोग्ध्रीणाम् अस्मि कामधुक् वसिष्ठस्य सर्वकामानां दोग्ध्री, सामान्या वा कामधुक्~। प्रजनः प्रजनयिता अस्मि कन्दर्पः कामः सर्पाणां सर्पभेदानाम् अस्मि वासुकिः सर्पराजः~॥~२८~॥\par
 \begin{center}{\bfseries अनन्तश्चास्मि नागानां वरुणो यादसामहम्~।\\पितॄणामर्यमा चास्मि यमः संयमतामहम्~॥~२९~॥}\end{center} 
अनन्तश्च अस्मि नागानां नागविशेषाणां नागराजश्च अस्मि~। वरुणो यादसाम् अहम् अब्देवतानां राजा अहम्~। पितॄणाम् अर्यमा नाम पितृराजश्च अस्मि~। यमः संयमतां संयमनं कुर्वताम् अहम्~॥~२९~॥\par
 \begin{center}{\bfseries प्रह्लादश्चास्मि दैत्यानां कालः कलयतामहम्~।\\मृगाणां च मृगेन्द्रोऽहं वैनतेयश्च पक्षिणाम्~॥~३०~॥}\end{center} 
प्रह्लादो नाम च अस्मि दैत्यानां दितिवंश्यानाम्~। कालः कलयतां कलनं गणनं कुर्वताम् अहम्~। मृगाणां च मृगेन्द्रः सिंहो व्याघ्रो वा अहम्~। वैनतेयश्च गरुत्मान् विनतासुतः पक्षिणां पतत्रिणाम्~॥~३०~॥\par
 \begin{center}{\bfseries पवनः पवतामस्मि रामः शस्त्रभृतामहम्~।\\झषाणां मकरश्चास्मि स्रोतसामस्मि जाह्नवी~॥~३१~॥}\end{center} 
पवनः वायुः पवतां पावयितॄणाम् अस्मि~। रामः शस्त्रभृताम् अहं शस्त्राणां धारयितॄणां दाशरथिः रामः अहम्~। झषाणां मत्स्यादीनां मकरः नाम जातिविशेषः अहम्~। स्रोतसां स्रवन्तीनाम् अस्मि जाह्नवी गङ्गा~॥~३१~॥\par
 \begin{center}{\bfseries सर्गाणामादिरन्तश्च मध्यं चैवाहमर्जुन~।\\अध्यात्मविद्या विद्यानां वादः प्रवदतामहम्~॥~३२~॥}\end{center} 
सर्गाणां सृष्टीनाम् आदिः अन्तश्च मध्यं चैव अहम् उत्पत्तिस्थितिलयाः अहम् अर्जुन~। भूतानां जीवाधिष्ठितानामेव आदिः अन्तश्च इत्याद्युक्तम् उपक्रमे, इह तु सर्वस्यैव सर्गमात्रस्य इति विशेषः~। अध्यात्मविद्या विद्यानां मोक्षार्थत्वात् प्रधानमस्मि~। वादः अर्थनिर्णयहेतुत्वात् प्रवदतां प्रधानम्~, अतः सः अहम् अस्मि~। प्रवत्त्कृद्वारेण वदनभेदानामेव वादजल्पवितण्डानाम् इह ग्रहणं प्रवदताम् इति~॥~३२~॥\par
 \begin{center}{\bfseries अक्षराणामकारोऽस्मि\\ द्वन्द्वः सामासिकस्य च~।\\अहमेवाक्षयः कालो\\ धाताहं विश्वतोमुखः~॥~३३~॥}\end{center} 
अक्षराणां वर्णानाम् अकारः वर्णः अस्मि~। द्वन्द्वः समासः अस्मि सामासिकस्य च समाससमूहस्य~। किञ्च अहमेव अक्षयः अक्षीणः कालः प्रसिद्धः क्षणाद्याख्यः, अथवा परमेश्वरः कालस्यापि कालः अस्मि~। धाता अहं कर्मफलस्य विधाता सर्वजगतः विश्वतोमुखः सर्वतोमुखः~॥~३३~॥\par
 \begin{center}{\bfseries मृत्युः सर्वहरश्चाह—\\ मुद्भवश्च भविष्यताम्~।\\कीर्तिः श्रीर्वाक्च नारीणां\\ स्मृतिर्मेधा धृतिः क्षमा~॥~३४~॥}\end{center} 
मृत्युः द्विविधः धनादिहरः प्राणहरश्च~; तत्र यः प्राणहरः, स सर्वहरः उच्यते~; सः अहम् इत्यर्थः~। अथवा, परः ईश्वरः प्रलये सर्वहरणात् सर्वहरः, सः अहम्~। उद्भवः उत्कर्षः अभ्युदयः तत्प्राप्तिहेतुश्च अहम्~। केषाम्~? भविष्यतां भाविकल्याणानाम्~, उत्कर्षप्राप्तियोग्यानाम् इत्यर्थः~। कीर्तिः श्रीः वाक् च नारीणां स्मृतिः मेधा धृतिः क्षमा इत्येताः उत्तमाः स्त्रीणाम् अहम् अस्मि, यासाम् आभासमात्रसम्बन्धेनापि लोकः कृतार्थमात्मानं मन्यते~॥~३४~॥\par
 \begin{center}{\bfseries बृहत्साम तथा साम्नां\\ गायत्री च्छन्दसामहम्~।\\मासानां मार्गशीर्षोऽह -\\ मृतूनां कुसुमाकरः~॥~३५~॥}\end{center} 
बृहत्साम तथा साम्नां प्रधानमस्मि~। गायत्री च्छन्दसाम् अहं गायत्र्यादिच्छन्दोविशिष्टानामृचां गायत्री ऋक् अहम् अस्मि इत्यर्थः~। मासानां मार्गशीर्षः अहम्~, ऋतूनां कुसुमाकरः वसन्तः~॥~३५~॥\par
 \begin{center}{\bfseries द्यूतं छलयतामस्मि\\ तेजस्तेजस्विनामहम्~।\\जयोऽस्मि व्यवसायोऽस्मि\\ सत्त्वं सत्त्ववतामहम्~॥~३६~॥}\end{center} 
द्यूतम् अक्षदेवनादिलक्षणं छलयतां छलस्य कर्तॄणाम् अस्मि~। तेजस्विनां तेजः अहम्~। जयः अस्मि जेतॄणाम्~, व्यवसायः अस्मि व्यवसायिनाम्~, सत्त्वं सत्त्ववतां सात्त्विकानाम् अहम्~॥~३६~॥\par
 \begin{center}{\bfseries वृष्णीनां वासुदेवोऽस्मि\\ पाण्डवानां धनञ्जयः~।\\मुनीनामप्यहं व्यासः\\ कवीनामुशना कविः~॥~३७~॥}\end{center} 
वृष्णीनां यादवानां वासुदेवः अस्मि अयमेव अहं त्वत्सखः~। पाण्डवानां धनञ्जयः } 
त्वमेव~। मुनीनां मननशीलानां सर्वपदार्थज्ञानिनाम् अपि अहं व्यासः, कवीनां क्रान्तदर्शिनाम् उशना कविः अस्मि~॥~३७~॥\par
 \begin{center}{\bfseries दण्डो दमयतामस्मि\\ नीतिरस्मि जिगीषताम्~।\\मौनं चैवास्मि गुह्यानां\\ ज्ञानं ज्ञानवतामहम्~॥~३८~॥}\end{center} 
दण्डः दमयतां दमयितॄणाम् अस्मि अदान्तानां दमनकारणम्~। नीतिः अस्मि जिगीषतां जेतुमिच्छताम्~। मौनं चैव अस्मि गुह्यानां गोप्यानाम्~। ज्ञानं ज्ञानवताम् अहम्~॥~३८~॥\par
 \begin{center}{\bfseries यच्चापि सर्वभूतानां\\ बीजं तदहमर्जुन~।\\न तदस्ति विना यत्स्या—\\ न्मया भूतं चराचरम्~॥~३९~॥}\end{center} 
यच्चापि सर्वभूतानां बीजं प्ररोहकारणम्~, तत् अहम् अर्जुन~। प्रकरणोपसंहारार्थं विभूतिसङ्क्षेपमाह — न तत् अस्ति भूतं चराचरं चरम् अचरं वा, मया विना यत् स्यात् भवेत्~। मया अपकृष्टं परित्यक्तं निरात्मकं शून्यं हि तत् स्यात्~। अतः मदात्मकं सर्वमित्यर्थः~॥~३९~॥\par
 \begin{center}{\bfseries नान्तोऽस्ति मम दिव्यानां\\ विभूतीनां परन्तप~।\\एष तूद्देशतः प्रोक्तो\\ विभूतेर्विस्तरो मया~॥~४०~॥}\end{center} 
न अन्तः अस्ति मम दिव्यानां विभूतीनां विस्तराणां परन्तप~। न हि ईश्वरस्य सर्वात्मनः दिव्यानां विभूतीनाम् इयत्ता शक्या वक्तुं ज्ञातुं वा केनचित्~। एष तु उद्देशतः एकदेशेन प्रोक्तः विभूतेः विस्तरः मया~॥~४०~॥\par
 \begin{center}{\bfseries यद्यद्विभूतिमत्सत्त्वं श्रीमदूर्जितमेव वा~।\\तत्तदेवावगच्छ त्वं मम तेजोंशसम्भवम्~॥~४१~॥}\end{center} 
यद्यत् लोके विभूतिमत् विभूतियुक्तं सत्त्वं वस्तु श्रीमत् ऊर्जितमेव वा श्रीर्लक्ष्मीः तया सहितम् उत्साहोपेतं वा, तत्तदेव अवगच्छ त्वं जानीहि मम ईश्वरस्य तेजोंशसम्भवं तेजसः अंशः एकदेशः सम्भवः यस्य तत् तेजोंशसम्भवमिति अवगच्छ त्वम्~॥~४१~॥\par
 \begin{center}{\bfseries अथवा बहुनैतेन\\ किं ज्ञातेन तवार्जुन~।\\विष्टभ्याहमिदं कृत्स्न—\\ मेकांशेन स्थितो जगत्~॥~४२~॥}\end{center} 
अथवा बहुना एतेन एवमादिना किं ज्ञातेन तव अर्जुन स्यात् सावशेषेण~। अशेषतः त्वम् उच्यमानम् अर्थं शृणु — विष्टभ्य विशेषतः स्तम्भनं दृढं कृत्वा इदं कृत्स्नं जगत् एकांशेन एकावयवेन एकपादेन, सर्वभूतस्वरूपेण इत्येतत्~; तथा च मन्त्रवर्णः — ‘पादोऽस्य विश्वा भूतानि’\footnote{ऋ. १०~। ८~। ९०~। ३} इति~; स्थितः अहम् इति~॥~४२~॥\par
 
इति श्रीमत्परमहंसपरिव्राजकाचार्यस्य श्रीगोविन्दभगवत्पूज्यपादशिष्यस्य श्रीमच्छङ्करभगवतः कृतौ श्रीमद्भगवद्गीताभाष्ये दशमोऽध्यायः~॥\par
 
भगवतो विभूतय उक्ताः~। तत्र च ‘विष्टभ्याहमिदं कृत्स्नमेकांशेन स्थितो जगत्’\footnote{भ. गी. १०~। ४२} इति भगवता अभिहितं श्रुत्वा, यत् जगदात्मरूपम् आद्यमैश्वरं तत् साक्षात्कर्तुमिच्छन्~, अर्जुन उवाच —}\\ 
\begin{center}{\bfseries अर्जुन उवाच —\\ मदनुग्रहाय परमं गुह्यमध्यात्मसंज्ञितम्~।\\यत्त्वयोक्तं वचस्तेन मोहोऽयं विगतो मम~॥~१~॥}\end{center} 
मदनुग्रहाय ममानुग्रहार्थं परमं निरतिशयं गुह्यं गोप्यम् अध्यात्मसंज्ञितम् आत्मानात्मविवेकविषयं यत् त्वया उक्तं वचः वाक्यं तेन ते वचसा मोहः अयं विगतः मम, अविवेकबुद्धिः अपगता इत्यर्थः~॥~१~॥\par
 किञ्च —} 
\begin{center}{\bfseries भवाप्ययौ हि भूतानां श्रुतौ विस्तरशो मया~।\\त्वत्तः कमलपत्राक्ष माहात्म्यमपि चाव्ययम्~॥~२~॥}\end{center} 
भवः उत्पत्तिः अप्ययः प्रलयः तौ भवाप्ययौ हि भूतानां श्रुतौ विस्तरशः मया, न सङ्क्षेपतः, त्वत्तः त्वत्सकाशात्~, कमलपत्राक्ष कमलस्य पत्रं कमलपत्रं तद्वत् अक्षिणी यस्य तव स त्वं कमलपत्राक्षः हे कमलपत्राक्ष, महात्मनः भावः माहात्म्यमपि च अव्ययम् अक्षयम् ‘श्रुतम्’ इति अनुवर्तते~॥~२~॥\par
 \begin{center}{\bfseries एवमेतद्यथात्थ त्वमात्मानं परमेश्वर~।\\द्रष्टुमिच्छामि ते रूपमैश्वरं पुरुषोत्तम~॥~३~॥}\end{center} 
एवमेतत् नान्यथा यथा येन प्रकारेण आत्थ कथयसि त्वम् आत्मानं परमेश्वर~। तथापि द्रष्टुमिच्छामि ते तव ज्ञानैश्वर्यशक्तिबलवीर्यतेजोभिः सम्पन्नम् ऐश्वरं वैष्णवं रूपं पुरुषोत्तम~॥~३~॥\par
 \begin{center}{\bfseries मन्यसे यदि तच्छक्यं मया द्रष्टुमिति प्रभो~।\\योगेश्वर ततो मे त्वं दर्शयात्मानमव्ययम्~॥~४~॥}\end{center} 
मन्यसे चिन्तयसि यदि मया अर्जुनेन तत् शक्यं द्रष्टुम् इति प्रभो, स्वामिन्~, योगेश्वर योगिनो योगाः, तेषां ईश्वरः योगेश्वरः, हे योगेश्वर~। यस्मात् अहम् अतीव अर्थी द्रष्टुम्~, ततः तस्मात् मे मदर्थं दर्शय त्वम् आत्मानम् अव्ययम्~॥~४~॥\par
 एवं चोदितः अर्जुनेन भगवान् उवाच —}\\ 
\begin{center}{\bfseries श्रीभगवानुवाच —\\ पश्य मे पार्थ रूपाणि शतशोऽथ सहस्रशः~।\\नानाविधानि दिव्यानि नानावर्णाकृतीनि च~॥~५~॥}\end{center} 
पश्य मे पार्थ, रूपाणि शतशः अथ सहस्रशः, अनेकशः इत्यर्थः~। तानि च नानाविधानि अनेकप्रकाराणि दिवि भवानि दिव्यानि अप्राकृतानि नानावर्णाकृतीनि च नाना विलक्षणाः नीलपीतादिप्रकाराः वर्णाः तथा आकृतयश्च अवयवसंस्थानविशेषाः येषां रूपाणां तानि नानावर्णाकृतीनि च~॥~५~॥\par
 \begin{center}{\bfseries पश्यादित्यान्वसून्रुद्रा—\\ नश्विनौ मरुतस्तथा~।\\बहून्यदृष्टपूर्वाणि\\ पश्याश्चर्याणि भारत~॥~६~॥}\end{center} 
पश्य आदित्यान् द्वादश, वसून् अष्टौ, रुद्रान् एकादश, अश्विनौ द्वौ, मरुतः सप्त सप्त गणाः ये तान्~। तथा च बहूनि अन्यान्यपि अदृष्टपूर्वाणि मनुष्यलोके त्वया, त्वत्तः अन्येन वा केनचित्~, पश्य आश्चर्याणि अद्भुतानि भारत~॥~६~॥\par
 न केवलम् एतावदेव —} 
\begin{center}{\bfseries इहैकस्थं जगत्कृत्स्नं पश्याद्य सचराचरम्~।\\मम देहे गुडाकेश यच्चान्यद्द्रष्टुमिच्छसि~॥~७~॥}\end{center} 
इह एकस्थम् एकस्मिन्नेव स्थितं जगत् कृत्स्नं समस्तं पश्य अद्य इदानीं सचराचरं सह चरेण अचरेण च वर्तते मम देहे गुडाकेश~। यच्च अन्यत् जयपराजयादि, यत् शङ्कसे, ‘यद्वा जयेम यदि वा नो जयेयुः’\footnote{भ. गी. २~। ६} इति यत् अवोचः, तदपि द्रष्टुं यदि इच्छसि~॥~७~॥\par
 किं तु —} 
\begin{center}{\bfseries न तु मां शक्यसे द्रष्टुमनेनैव स्वचक्षुषा~।\\दिव्यं ददामि ते चक्षुः पश्य मे योगमैश्वरम्~॥~८~॥}\end{center} 
न तु मां विश्वरूपधरं शक्यसे द्रष्टुम् अनेनैव प्राकृतेन स्वचक्षुषा स्वकीयेन चक्षुषा~। येन तु शक्यसे द्रष्टुं दिव्येन, तत् दिव्यं ददामि ते तुभ्यं चक्षुः~। तेन पश्य मे योगम् ऐश्वरम् ईश्वरस्य मम ऐश्वरं योगं योगशक्त्यतिशयम् इत्यर्थः~॥~८~॥\par
\begin{center}{\bfseries सञ्जय उवाच —\\ एवमुक्त्वा ततो राजन्महायोगेश्वरो हरिः~।\\दर्शयामास पार्थाय परमं रूपमैश्वरम्~॥~९~॥}\end{center} 
एवं यथोक्तप्रकारेण उक्त्वा ततः अनन्तरं राजन् धृतराष्ट्र, महायोगेश्वरः महांश्च असौ योगेश्वरश्च हरिः नारायणः दर्शयामास दर्शितवान् पार्थाय पृथासुताय परमं रूपं विश्वरूपम् ऐश्वरम्~॥~९~॥\par
 \begin{center}{\bfseries अनेकवक्त्रनयनमनेकाद्भुतदर्शनम्~।\\अनेकदिव्याभरणं दिव्यानेकोद्यतायुधम्~॥~१०~॥}\end{center} 
अनेकवक्त्रनयनम् अनेकानि वक्त्राणि नयनानि च यस्मिन् रूपे तत् अनेकवक्त्रनयनम्~, अनेकाद्भुतदर्शनम् अनेकानि अद्भुतानि विस्मापकानि दर्शनानि यस्मिन् रूपे तत् अनेकाद्भुतदर्शनं रूपम्~, तथा अनेकदिव्याभरणम् अनेकानि दिव्यानि आभरणानि यस्मिन् तत् अनेकदिव्याभरणम्~, तथा दिव्यानेकोद्यतायुधं दिव्यानि अनेकानि अस्यादीनि उद्यतानि आयुधानि यस्मिन् तत् दिव्यानेकोद्यतायुधम्~, ‘दर्शयामास’ इति पूर्वेण सम्बन्धः~॥~१०~॥\par
 किञ्च —} 
\begin{center}{\bfseries दिव्यमाल्याम्बरधरं दिव्यगन्धानुलेपनम्~।\\सर्वाश्चर्यमयं देवमनन्तं विश्वतोमुखम्~॥~११~॥}\end{center} 
दिव्यमाल्याम्बरधरं दिव्यानि माल्यानि पुष्पाणि अम्बराणि वस्त्राणि च ध्रियन्ते येन ईश्वरेण तं दिव्यमाल्याम्बरधरम्~, दिव्यगन्धानुलेपनं दिव्यं गन्धानुलेपनं यस्य तं दिव्यगन्धानुलेपनम्~, सर्वाश्चर्यमयं सर्वाश्चर्यप्रायं देवम् अनन्तं न अस्य अन्तः अस्ति इति अनन्तः तम्~, विश्वतोमुखं सर्वतोमुखं सर्वभूतात्मभूतत्वात्~, तं दर्शयामास~। ‘अर्जुनः ददर्श’ इति वा अध्याह्रियते~॥~११~॥\par
 या पुनर्भगवतः विश्वरूपस्य भाः, तस्या उपमा उच्यते —} 
\begin{center}{\bfseries दिवि सूर्यसहस्रस्य\\ भवेद्युगपदुत्थिता~।\\यदि भाः सदृशी सा स्या—\\ द्भासस्तस्य महात्मनः~॥~१२~॥}\end{center} 
दिवि अन्तरिक्षे तृतीयस्यां वा दिवि सूर्याणां सहस्रं सूर्यसहस्रं तस्य युगपदुत्थितस्य सूर्यसहस्रस्य या युगपदुत्थिता भाः, सा यदि, सदृशी स्यात् तस्य महात्मनः विश्वरूपस्यैव भासः~। यदि वा न स्यात्~, ततः विश्वरूपस्यैव भाः अतिरिच्यते इत्यभिप्रायः~॥~१२~॥\par
 किञ्च —} 
\begin{center}{\bfseries तत्रैकस्थं जगत्कृत्स्नं प्रविभक्तमनेकधा~।\\अपश्यद्देवदेवस्य शरीरे पाण्डवस्तदा~॥~१३~॥}\end{center} 
तत्र तस्मिन् विश्वरूपे एकस्मिन् स्थितम् एकस्थं जगत् कृत्स्नं प्रविभक्तम् अनेकधा देवपितृमनुष्यादिभेदैः अपश्यत् दृष्टवान् देवदेवस्य हरेः शरीरे पाण्डवः अर्जुनः तदा~॥~१३~॥\par
 \begin{center}{\bfseries ततः स विस्मयाविष्टो हृष्टरोमा धनञ्जयः~।\\प्रणम्य शिरसा देवं कृताञ्जलिरभाषत~॥~१४~॥}\end{center} 
ततः तं दृष्ट्वा सः विस्मयेन आविष्टः विस्मयाविष्टः हृष्टानि रोमाणि यस्य सः अयं हृष्टरोमा च अभवत् धनञ्जयः~। प्रणम्य प्रकर्षेण नमनं कृत्वा प्रह्वीभूतः सन् शिरसा देवं विश्वरूपधरं कृताञ्जलिः नमस्कारार्थं सम्पुटीकृतहस्तः सन् अभाषत उक्तवान्~॥~१४~॥\par
 कथम्~? यत् त्वया दर्शितं विश्वरूपम्~, तत् अहं पश्यामीति स्वानुभवमाविष्कुर्वन् अर्जुन उवाच —}\\ 
\begin{center}{\bfseries अर्जुन उवाच —\\पश्यामि देवांस्तव देव देहे\\ सर्वांस्तथा भूतविशेषसङ्घान्~।\\ब्रह्माणमीशं कमलासनस्थ—\\ मृषींश्च सर्वानुरगांश्च दिव्यान्~॥~१५~॥}\end{center} 
पश्यामि उपलभे हे देव, तव देहे देवान् सर्वान्~, तथा भूतविशेषसङ्घान् भूतविशेषाणां स्थावरजङ्गमानां नानासंस्थानविशेषाणां सङ्घाः भूतविशेषसङ्घाः तान्~, किञ्च — ब्रह्माणं चतुर्मुखम् ईशम् ईशितारं प्रजानां कमलासनस्थं पृथिवीपद्ममध्ये मेरुकर्णिकासनस्थमित्यर्थः, ऋषींश्च वसिष्ठादीन् सर्वान्~, उरगांश्च वासुकिप्रभृतीन् दिव्यान् दिवि भवान्~॥~१५~॥\par
 \begin{center}{\bfseries अनेकबाहूदरवक्त्रनेत्रं\\ पश्यामि त्वा सर्वतोऽनन्तरूपम्~।\\नान्तं न मध्यं न पुनस्तवादिं\\ पश्यामि विश्वेश्वर विश्वरूप~॥~१६~॥}\end{center} 
अनेकबाहूदरवक्त्रनेत्रम् अनेके बाहवः उदराणि वक्त्राणि नेत्राणि च यस्य तव सः त्वम् अनेकबाहूदरवक्त्रनेत्रः तम् अनेकबाहूदरवक्त्रनेत्रम्~। पश्यामि त्वा त्वां सर्वतः सर्वत्र अनन्तरूपम् अनन्तानि रूपाणि अस्य इति अनन्तरूपः तम् अनन्तरूपम्~। न अन्तम्~, अन्तः अवसानम्~, न मध्यम्~, मध्यं नाम द्वयोः कोट्योः अन्तरम्~, न पुनः तव आदिम् — न देवस्य अन्तं पश्यामि, न मध्यं पश्यामि, न पुनः आदिं पश्यामि, हे विश्वेश्वर विश्वरूप~॥~१६~॥\par
 किञ्च —} 
\begin{center}{\bfseries किरीटिनं गदिनं चक्रिणं च\\ तेजोराशिं सर्वतोदीप्तिमन्तम्~।\\पश्यामि त्वां दुर्निरीक्ष्यं समन्ता—\\ द्दीप्तानलार्कद्युतिमप्रमेयम्~॥~१७~॥}\end{center} 
किरीटिनं किरीटं नाम शिरोभूषणविशेषः तत् यस्य अस्ति सः किरीटी तं किरीटिनम्~, तथा गदिनं गदा अस्य विद्यते इति गदी तं गदिनम्~, तथा चक्रिणं चक्रम् अस्य अस्तीति चक्री तं चक्रिणं च, तेजोराशिं तेजःपुञ्जं सर्वतोदीप्तिमन्तं सर्वतोदीप्तिः अस्य अस्तीति सर्वतोदीप्तिमान्~, तं सर्वतोदीप्तिमन्तं पश्यामि त्वां दुर्निरीक्ष्यं दुःखेन निरीक्ष्यः दुर्निरीक्ष्यः तं दुर्निरीक्ष्यं समन्तात् समन्ततः सर्वत्र दीप्तानलार्कद्युतिम् अनलश्च अर्कश्च अनलार्कौ दीप्तौ अनलार्कौ दीप्तानलार्कौ तयोः दीप्तानलार्कयोः द्युतिरिव द्युतिः तेजः यस्य तव स त्वं दीप्तानलार्कद्युतिः तं त्वां दीप्तानलार्कद्युतिम्~, अप्रमेयं न प्रमेयम् अशक्यपरिच्छेदम् इत्येतत्~॥~१७~॥\par
 इत एव ते योगशक्तिदर्शनात् अनुमिनोमि —} 
\begin{center}{\bfseries त्वमक्षरं परमं वेदितव्यं\\ त्वमस्य विश्वस्य परं निधानम्~।\\त्वमव्ययः शाश्वतधर्मगोप्ता\\ सनातनस्त्वं पुरुषो मतो मे~॥~१८~॥}\end{center} 
त्वम् अक्षरं न क्षरतीति, परमं ब्रह्म वेदितव्यं ज्ञातव्यं मुमुक्षुभिः~। त्वम् अस्य विश्वस्य समस्तस्य जगतः परं प्रकृष्टं निधानं निधीयते अस्मिन्निति निधानं परः आश्रयः इत्यर्थः~। किञ्च, त्वम् अव्ययः न तव व्ययो विद्यते इति अव्ययः, शाश्वतधर्मगोप्ता शश्वद्भवः शाश्वतः नित्यः धर्मः तस्य गोप्ता शाश्वतधर्मगोप्ता~। सनातनः चिरन्तनः त्वं पुरुषः परमः मतः अभिप्रेतः मे मम~॥~१८~॥\par
 किञ्च —} 
\begin{center}{\bfseries अनादिमध्यान्तमनन्तवीर्य—\\ मनन्तबाहुं शशिसूर्यनेत्रम्~।\\पश्यामि त्वां दीप्तहुताशवक्त्रं\\ स्वतेजसा विश्वमिदं तपन्तम्~॥~१९~॥}\end{center} 
अनादिमध्यान्तम् आदिश्च मध्यं च अन्तश्च न विद्यते यस्य सः अयम् अनादिमध्यान्तः तं त्वां अनादिमध्यान्तम्~, अनन्तवीर्यं न तव वीर्यस्य अन्तः अस्ति इति अनन्तवीर्यः तं त्वाम् अनन्तवीर्यम्~, तथा अनन्तबाहुम् अनन्ताः बाहवः यस्य तव सः त्वम्~, अनन्तबाहुः तं त्वाम् अनन्तबाहुम्~, शशिसूर्यनेत्रं शशिशूर्यौ नेत्रे यस्य तव सः त्वं शशिसूर्यनेत्रः तं त्वां शशिसूर्यनेत्रं चन्द्रादित्यनयनम्~, पश्यामि त्वां दीप्तहुताशवक्त्रं दीप्तश्च असौ हुताशश्च वक्त्रं यस्य तव सः त्वं दीप्तहुताशवक्त्रः तं त्वां दीप्तहुताशवक्त्रम्~, स्वतेजसा विश्वम् इदं समस्तं तपन्तम्~॥~१९~॥\par
 \begin{center}{\bfseries द्यावापृथिव्योरिदमन्तरं हि\\ व्याप्तं त्वयैकेन दिशश्च सर्वाः~।\\दृष्ट्वाद्भुतं रूपमिदं तवोग्रं\\ लोकत्रयं प्रव्यथितं महात्मन्~॥~२०~॥}\end{center} 
द्यावापृथिव्योः इदम् अन्तरं हि अन्तरिक्षं व्याप्तं त्वया एकेन विश्वरूपधरेण दिशश्च सर्वाः व्याप्ताः~। दृष्ट्वा उपलभ्य अद्भुतं विस्मापकं रूपम् इदं तव उग्रं क्रूरं लोकानां त्रयं लोकत्रयं प्रव्यथितं भीतं प्रचलितं वा हे महात्मन् अक्षुद्रस्वभाव~॥~२०~॥\par
 अथ अधुना पुरा ‘यद्वा जयेम यदि वा नो जयेयुः’\footnote{भ. गी. २~। ६} इति अर्जुनस्य यः संशयः आसीत्~, तन्निर्णयाय पाण्डवजयम् ऐकान्तिकं दर्शयामि इति प्रवृत्तो भगवान्~। तं पश्यन् आह — किञ्च —} 
\begin{center}{\bfseries अमी हि त्वा सुरसङ्घा विशन्ति\\ केचिद्भीताः प्राञ्जलयो गृणन्ति~।\\स्वस्तीत्युक्त्वा महर्षिसिद्धसङ्घाः\\ स्तुवन्ति त्वां स्तुतिभिः पुष्कलाभिः~॥~२१~॥}\end{center} 
अमी हि युध्यमाना योद्धारः त्वा त्वां सुरसङ्घाः ये अत्र भूभारावताराय अवतीर्णाः वस्वादिदेवसङ्घाः मनुष्यसंस्थानाः त्वां विशन्ति प्रविशन्तः दृश्यन्ते~। तत्र केचित् भीताः प्राञ्जलयः सन्तो गृणन्ति स्तुवन्ति त्वाम् अन्ये पलायनेऽपि अशक्ताः सन्तः~। युद्धे प्रत्युपस्थिते उत्पातादिनिमित्तानि उपलक्ष्य स्वस्ति अस्तु जगतः इति उक्त्वा महर्षिसिद्धसङ्घाः महर्षीणां सिद्धानां च सङ्घाः स्तुवन्ति त्वां स्तुतिभिः पुष्कलाभिः सम्पूर्णाभिः~॥~२१~॥\par
 किञ्चान्यत् —} 
\begin{center}{\bfseries रुद्रादित्या वसवो ये च साध्या\\ विश्वेऽश्विनौ मरुतश्चोष्मपाश्च~।\\गन्धर्वयक्षासुरसिद्धसङ्घा\\ वीक्षन्ते त्वां विस्मिताश्चैव सर्वे~॥~२२~॥}\end{center} 
रुद्रादित्याः वसवो ये च साध्याः रुद्रादयः गणाः विश्वेदेवाः अश्विनौ च देवौ मरुतश्च ऊष्मपाश्च पितरः, गन्धर्वयक्षासुरसिद्धसङ्घाः गन्धर्वाः हाहाहूहूप्रभृतयः यक्षाः कुबेरप्रभृतयः असुराः विरोचनप्रभृतयः सिद्धाः कपिलादयः तेषां सङ्घाः गन्धर्वयक्षासुरसिद्धसङ्घाः, ते वीक्षन्ते पश्यन्ति त्वां विस्मिताः विस्मयमापन्नाः सन्तः ते एव सर्वे~॥~२२~॥\par
 यस्मात् —} 
\begin{center}{\bfseries रूपं महत्ते बहुवक्त्रनेत्रं\\ महाबाहो बहुबाहूरुपादम्~।\\बहूदरं बहुदंष्ट्राकरालं\\ दृष्ट्वा लोकाः प्रव्यथितास्तथाहम्~॥~२३~॥}\end{center} 
रूपं महत् अतिप्रमाणं ते तव बहुवक्त्रनेत्रं बहूनि वक्त्राणि मुखानि नेत्राणि चक्षूंषि च यस्मिन् तत् रूपं बहुवक्त्रनेत्रम्~, हे महाबाहो, बहुबाहूरुपादं बहवो बाहवः ऊरवः पादाश्च यस्मिन् रूपे तत् बहुबाहूरुपादम्~, किञ्च, बहूदरं बहूनि उदराणि यस्मिन्निति बहूदरम्~, बहुदंष्ट्राकरालं बह्वीभिः दंष्ट्राभिः करालं विकृतं तत् बहुदंष्ट्राकरालम्~, दृष्ट्वा रूपम् ईदृशं लोकाः लौकिकाः प्राणिनः प्रव्यथिताः प्रचलिताः भयेन~; तथा अहमपि~॥~२३~॥\par
 तत्रेदं कारणम् —} 
\begin{center}{\bfseries नभःस्पृशं दीप्तमनेकवर्णं\\ व्यात्ताननं दीप्तविशालनेत्रम्~।\\दृष्ट्वा हि त्वां प्रव्यथितान्तरात्मा\\ धृतिं न विन्दामि शमं च विष्णो~॥~२४~॥}\end{center} 
नभःस्पृशं द्युस्पर्शम् इत्यर्थः, दीप्तं प्रज्वलितम्~, अनेकवर्णम् अनेके वर्णाः भयङ्कराः नानासंस्थानाः यस्मिन् त्वयि तं त्वाम् अनेकवर्णम्~, व्यात्ताननं व्यात्तानि विवृतानि आननानि मुखानि यस्मिन् त्वयि तं त्वां व्यात्ताननम्~, दीप्तविशालनेत्रं दीप्तानि प्रज्वलितानि विशालानि विस्तीर्णानि नेत्राणि यस्मिन् त्वयि तं त्वां दीप्तविशालनेत्रं दृष्ट्वा हि त्वां प्रव्यथितान्तरात्मा प्रव्यथितः प्रभीतः अन्तरात्मा मनः यस्य मम सः अहं प्रव्यथितान्तरात्मा सन् धृतिं धैर्यं न विन्दामि न लभे शमं च उपशमनं मनस्तुष्टिं हे विष्णो~॥~२४~॥\par
 कस्मात् —} 
\begin{center}{\bfseries दंष्ट्राकरालानि च ते मुखानि\\ दृष्ट्वैव कालानलसंनिभानि~।\\दिशो न जाने न लभे च शर्म\\ प्रसीद देवेश जगन्निवास~॥~२५~॥}\end{center} 
दंष्ट्राकरालानि दंष्ट्राभिः करालानि विकृतानि ते तव मुखानि दृष्ट्वैव उपलभ्य कालानलसंनिभानि प्रलयकाले लोकानां दाहकः अग्निः कालानलः तत्सदृशानि कालानलसंनिभानि मुखानि दृष्ट्वेत्येतत्~। दिशः पूर्वापरविवेकेन न जाने दिङ्मूढो जातः अस्मि~। अतः न लभे च न उपलभे च शर्म सुखम्~। अतः प्रसीद प्रसन्नो भव हे देवेश, जगन्निवास~॥~२५~॥\par
 येभ्यो मम पराजयाशङ्का या आसीत् सा च अपगता~। यतः —} 
\begin{center}{\bfseries अमी च त्वां धृतराष्ट्रस्य पुत्राः\\ सर्वे सहैवावनिपालसङ्घैः~।\\भीष्मो द्रोणः सूतपुत्रस्तथासौ\\ सहास्मदीयैरपि योधमुख्यैः~॥~२६~॥}\end{center} 
अमी च त्वां धृतराष्ट्रस्य पुत्राः दुर्योधनप्रभृतयः — ‘त्वरमाणाः विशन्ति’ इति व्यवहितेन सम्बन्धः — सर्वे सहैव सहिताः अवनिपालसङ्घैः अवनिं पृथ्वीं पालयन्तीति अवनिपालाः तेषां सङ्घैः, किञ्च भीष्मो द्रोणः सूतपुत्रः कर्णः तथा असौ सह अस्मदीयैरपि धृष्टद्युम्नप्रभृतिभिः योधमुख्यैः योधानां मुख्यैः प्रधानैः सह~॥~२६~॥\par
 किञ्च —} 
\begin{center}{\bfseries वक्त्राणि ते त्वरमाणा विशन्ति\\ दंष्ट्राकरालानि भयानकानि~।\\केचिद्विलग्ना दशनान्तरेषु\\ सन्दृश्यन्ते चूर्णितैरुत्तमाङ्गैः~॥~२७~॥}\end{center} 
वक्त्राणि मुखानि ते तव त्वरमाणाः त्वरायुक्ताः सन्तः विशन्ति, किंविशिष्टानि मुखानि~? दंष्ट्राकरालानि भयानकानि भयङ्कराणि~। किञ्च, केचित् मुखानि प्रविष्टानां मध्ये विलग्नाः दशनान्तरेषु मांसमिव भक्षितं सन्दृश्यन्ते उपलभ्यन्ते चूर्णितैः चूर्णीकृतैः उत्तमाङ्गैः शिरोभिः~॥~२७~॥\par
 कथं प्रविशन्ति मुखानि इत्याह —} 
\begin{center}{\bfseries यथा नदीनां बहवोऽम्बुवेगाः\\ समुद्रमेवाभिमुखा द्रवन्ति~।\\तथा तवामी नरलोकवीरा\\ विशन्ति वक्त्राण्यभिविज्वलन्ति~॥~२८~॥}\end{center} 
यथा नदीनां स्रवन्तीनां बहवः अनेके अम्बूनां वेगाः अम्बुवेगाः त्वराविशेषाः समुद्रमेव अभिमुखाः प्रतिमुखाः द्रवन्ति प्रविशन्ति, तथा तद्वत् तव अमी भीष्मादयः नरलोकवीराः मनुष्यलोके शूराः विशन्ति वक्त्राणि अभिविज्वलन्ति प्रकाशमानानि~॥~२८~॥\par
 ते किमर्थं प्रविशन्ति कथं च इत्याह —} 
\begin{center}{\bfseries यथा प्रदीप्तं ज्वलनं पतङ्गा\\ विशन्ति नाशाय समृद्धवेगाः~।\\तथैव नाशाय विशन्ति लोका—\\ स्तवापि वक्त्राणि समृद्धवेगाः~॥~२९~॥}\end{center} 
यथा प्रदीप्तं ज्वलनम् अग्निं पतङ्गाः पक्षिणः विशन्ति नाशाय विनाशाय समृद्धवेगाः समृद्धः उद्भूतः वेगः गतिः येषां ते समृद्धवेगाः, तथैव नाशाय विशन्ति लोकाः प्राणिनः तवापि वक्त्राणि समृद्धवेगाः~॥~२९~॥\par
 त्वं पुनः —} 
\begin{center}{\bfseries लेलिह्यसे ग्रसमानः समन्ता—\\ ल्लोकान्समग्रान्वदनैर्ज्वलद्भिः~।\\तेजोभिरापूर्य जगत्समग्रं\\ भासस्तवोग्राः प्रतपन्ति विष्णो~॥~३०~॥}\end{center} 
लेलिह्यसे आस्वादयसि ग्रसमानः अन्तः प्रवेशयन् समन्तात् समन्ततः लोकान् समग्रान् समस्तान् वदनैः वक्त्रैः ज्वलद्भिः दीप्यमानैः तेजोभिः आपूर्य संव्याप्य जगत् समग्रं सह अग्रेण समस्तम् इत्येतत्~। किञ्च, भासः दीप्तयः तव उग्राः क्रूराः प्रतपन्ति प्रतापं कुर्वन्ति हे विष्णो व्यापनशील~॥~३०~॥\par
 यतः एवमुग्रस्वभावः, अतः —} 
\begin{center}{\bfseries आख्याहि मे को भवानुग्ररूपो\\ नमोऽस्तु ते देववर प्रसीद~।\\विज्ञातुमिच्छामि भवन्तमाद्यं\\ न हि प्रजानामि तव प्रवृत्तिम्~॥~३१~॥}\end{center} 
आख्याहि कथय मे मह्यं कः भवान् उग्ररूपः क्रूराकारः, नमः अस्तु ते तुभ्यं हे देववर देवानां प्रधान, प्रसीद प्रसादं कुरु~। विज्ञातुं विशेषेण ज्ञातुम् इच्छामि भवन्तम् आद्यम् आदौ भवम् आद्यम्~, न हि यस्मात् प्रजानामि तव त्वदीयां प्रवृत्तिं चेष्टाम्~॥~३१~॥\par
 {\bfseries श्रीभगवानुवाच —}\\
\begin{center}{\bfseries कालोऽस्मि लोकक्षयकृत्प्रवृद्धो\\ लोकान्समाहर्तुमिह प्रवृत्तः~।\\ऋतेऽपि त्वा न भविष्यन्ति सर्वे\\ येऽवस्थिताः प्रत्यनीकेषु योधाः~॥~३२~॥}\end{center} 
कालः अस्मि लोकक्षयकृत् लोकानां क्षयं करोतीति लोकक्षयकृत् प्रवृद्धः वृद्धिं गतः~। यदर्थं प्रवृद्धः तत् शृणु — लोकान् समाहर्तुं संहर्तुम् इह अस्मिन् काले प्रवृत्तः~। ऋतेऽपि विनापि त्वा त्वां न भविष्यन्ति भीष्मद्रोणकर्णप्रभृतयः सर्वे, येभ्यः तव आशङ्का, ये अवस्थिताः प्रत्यनीकेषु अनीकमनीकं प्रति प्रत्यनीकेषु प्रतिपक्षभूतेषु अनीकेषु योधाः योद्धारः~॥~३२~॥\par
 यस्मात् एवम् —} 
\begin{center}{\bfseries तस्मात्त्वमुत्तिष्ठ यशो लभस्व\\ जित्वा शत्रून्भुङ्क्ष्व राज्यं समृद्धम्~।\\मयैवैते निहताः पूर्वमेव\\ निमित्तमात्रं भव सव्यसाचिन्~॥~३३~॥}\end{center} 
तस्मात् त्वम् उत्तिष्ठ ‘भीष्मप्रभृतयः अतिरथाः अजेयाः देवैरपि, अर्जुनेन जिताः’ इति यशः लभस्व~; केवलं पुण्यैः हि तत् प्राप्यते~। जित्वा शत्रून् दुर्योधनप्रभृतीन् भुङ्क्ष्व राज्यं समृद्धम् असपत्नम् अकण्टकम्~। मया एव एते निहताः निश्चयेन हताः प्राणैः वियोजिताः पूर्वमेव~। निमित्तमात्रं भव त्वं हे सव्यसाचिन्~, सव्येन वामेनापि हस्तेन शराणां क्षेप्ता सव्यसाची इति उच्यते अर्जुनः~॥~३३~॥\par
 \begin{center}{\bfseries द्रोणं च भीष्मं च जयद्रथं च\\ कर्णं तथान्यानपि योधवीरान्~।\\मया हतांस्त्वं जहि मा व्यथिष्ठा\\ युध्यस्व जेतासि रणे सपत्नान्~॥~३४~॥}\end{center} 
द्रोणं च, येषु येषु योधेषु अर्जुनस्य आशङ्का तांस्तान् व्यपदिशति भगवान्~, मया हतानिति~। तत्र द्रोणभीष्मयोः तावत् प्रसिद्धम् आशङ्काकारणम्~। द्रोणस्तु धनुर्वेदाचार्यः दिव्यास्त्रसम्पन्नः, आत्मनश्च विशेषतः गुरुः गरिष्ठः~। भीष्मश्च स्वच्छन्दमृत्युः दिव्यास्त्रसम्पन्नश्च परशुरामेण द्वन्द्वयुद्धम् अगमत्~, न च पराजितः~। तथा जयद्रथः, यस्य पिता तपः चरति ‘मम पुत्रस्य शिरः भूमौ निपातयिष्यति यः, तस्यापि शिरः पतिष्यति’ इति~। कर्णोऽपि वासवदत्तया शक्त्या त्वमोघया सम्पन्नः सूर्यपुत्रः कानीनः यतः, अतः तन्नाम्नैव निर्देशः~। मया हतान् त्वं जहि निमित्तमात्रेण~। मा व्यथिष्ठाः तेभ्यः भयं मा कार्षीः~। युध्यस्व जेतासि दुर्योधनप्रभृतीन् रणे युद्धे सपत्नान् शत्रून्~॥~३४~॥\par
 {\bfseries सञ्जय उवाच —}\\
\begin{center}{\bfseries एतच्छ्रुत्वा वचनं केशवस्य\\ कृताञ्जलिर्वेपमानः किरीटी~।\\नमस्कृत्वा भूय एवाह कृष्णं\\ सगद्गदं भीतभीतः प्रणम्य~॥~३५~॥}\end{center} 
एतत् श्रुत्वा वचनं केशवस्य पूर्वोक्तं कृताञ्जलिः सन् वेपमानः कम्पमानः किरीटी नमस्कृत्वा, भूयः पुनः एव आह उक्तवान् कृष्णं सगद्गदं भयाविष्टस्य दुःखाभिघातात् स्नेहाविष्टस्य च हर्षोद्भवात्~, अश्रुपूर्णनेत्रत्वे सति श्लेष्मणा कण्ठावरोधः~; ततश्च वाचः अपाटवं मन्दशब्दत्वं यत् स गद्गदः तेन सह वर्तत इति सगद्गदं वचनम् आह इति वचनक्रियाविशेषणम् एतत्~। भीतभीतः पुनः पुनः भयाविष्टचेताः सन् प्रणम्य प्रह्वः भूत्वा, ‘आह’ इति व्यवहितेन सम्बन्धः~॥~} 
अत्र अवसरे सञ्जयवचनं साभिप्रायम्~। कथम्~? द्रोणादिषु अर्जुनेन निहतेषु अजेयेषु चतुर्षु, निराश्रयः दुर्योधनः निहतः एव इति मत्वा धृतराष्ट्रः जयं प्रति निराशः सन् सन्धिं करिष्यति, ततः शान्तिः उभयेषां भविष्यति इति~। तदपि न अश्रौषीत् धृतराष्ट्रः भवितव्यवशात्~॥~३५~॥\par
 {\bfseries अर्जुन उवाच —}\\
\begin{center}{\bfseries स्थाने हृषीकेश तव प्रकीर्त्या\\ जगत्प्रहृष्यत्यनुरज्यते च~।\\रक्षांसि भीतानि दिशो द्रवन्ति\\ सर्वे नमस्यन्ति च सिद्धसङ्घाः~॥~३६~॥}\end{center} 
स्थाने युक्तम्~। किं तत्~? तव प्रकीर्त्या त्वन्माहात्म्यकीर्तनेन श्रुतेन, हे हृषीकेश, यत् जगत् प्रहृष्यति प्रहर्षम् उपैति, तत् स्थाने युक्तम्~, इत्यर्थः~। अथवा विषयविशेषणं स्थाने इति~। युक्तः हर्षादिविषयः भगवान्~, यतः ईश्वरः सर्वात्मा सर्वभूतसुहृच्च इति~। तथा अनुरज्यते अनुरागं च उपैति~; तच्च विषये इति व्याख्येयम्~। किञ्च, रक्षांसि भीतानि भयाविष्टानि दिशः द्रवन्ति गच्छन्ति~; तच्च स्थाने विषये~। सर्वे नमस्यन्ति नमस्कुर्वन्ति च सिद्धसङ्घाः सिद्धानां समुदायाः कपिलादीनाम्~, तच्च स्थाने~॥~३६~॥\par
 भगवतो हर्षादिविषयत्वे हेतुं दर्शयति —} 
\begin{center}{\bfseries कस्माच्च ते न नमेरन्महात्म—\\ न्गरीयसे ब्रह्मणोऽप्यादिकर्त्रे~।\\अनन्त देवेश जगन्निवास\\ त्वमक्षरं सदसत्तत्परं यत्~॥~३७~॥}\end{center} 
कस्माच्च हेतोः ते तुभ्यं न नमेरन् नमस्कुर्युः हे महात्मन्~, गरीयसे गुरुतराय~; यतः ब्रह्मणः हिरण्यगर्भस्य अपि आदिकर्ता कारणम् अतः तस्मात् आदिकर्त्रे~। कथम् एते न नमस्कुर्युः~? अतः हर्षादीनां नमस्कारस्य च स्थानं त्वं अर्हः विषयः इत्यर्थः~। हे अनन्त देवेश हे जगन्निवास त्वम् अक्षरं तत् परम्~, यत् वेदान्तेषु श्रूयते~। किं तत्~? सदसत् इति~। सत् विद्यमानम्~, असत् च यत्र नास्ति इति बुद्धिः~; ते उपधानभूते सदसती यस्य अक्षरस्य, यद्द्वारेण सदसती इति उपचर्यते~। परमार्थतस्तु सदसतोः परं तत् अक्षरं यत् अक्षरं वेदविदः वदन्ति~। तत् त्वमेव, न अन्यत् इति अभिप्रायः~॥~३७~॥\par
 पुनरपि स्तौति —} 
\begin{center}{\bfseries त्वमादिदेवः पुरुषः पुराण—\\ स्त्वमस्य विश्वस्य परं निधानम्~।\\वेत्तासि वेद्यं च परं च धाम\\ त्वया ततं विश्वमनन्तरूप~॥~३८~॥}\end{center} 
त्वम् आदिदेवः, जगतः स्रष्टृत्वात्~। पुरुषः, पुरि शयनात् पुराणः चिरन्तनः त्वम् एव अस्य विश्वस्य परं प्रकृष्टं निधानं निधीयते अस्मिन् जगत् सर्वं महाप्रलयादौ इति~। किञ्च, वेत्ता असि, वेदिता असि सर्वस्यैव वेद्यजातस्य~। यत् च वेद्यं वेदनार्हं तच्च असि परं च धाम परमं पदं वैष्णवम्~। त्वया ततं व्याप्तं विश्वं समस्तम्~, हे अनन्तरूप अन्तो न विद्यते तव रूपाणाम्~॥~३८~॥\par
 किञ्च —} 
\begin{center}{\bfseries वायुर्यमोऽग्निर्वरुणः शशाङ्कः\\ प्रजापतिस्त्वं प्रपितामहश्च~।\\नमो नमस्तेऽस्तु सहस्रकृत्वः\\ पुनश्च भूयोऽपि नमो नमस्ते~॥~३९~॥}\end{center} 
वायुः त्वं यमश्च अग्निः वरुणः अपां पतिः शशाङ्कः चन्द्रमाः प्रजापतिः त्वं कश्यपादिः प्रपितामहश्च पितामहस्यापि पिता प्रपितामहः, ब्रह्मणोऽपि पिता इत्यर्थः~। नमो नमः ते तुभ्यम् अस्तु सहस्रकृत्वः~। पुनश्च भूयोऽपि नमो नमः ते~। बहुशो नमस्कारक्रियाभ्यासावृत्तिगणनं कृत्वसुचा उच्यते~। ‘पुनश्च’ ‘भूयोऽपि’ इति श्रद्धाभक्त्यतिशयात् अपरितोषम् आत्मनः दर्शयति~॥~३९~॥\par
 तथा —} 
\begin{center}{\bfseries नमः पुरस्तादथ पृष्ठतस्ते\\ नमोऽस्तु ते सर्वत एव सर्व~।\\अनन्तवीर्यामितविक्रमस्त्वं\\ सर्वं समाप्नोषि ततोऽसि सर्वः~॥~४०~॥}\end{center} 
नमः पुरस्तात् पूर्वस्यां दिशि तुभ्यम्~, अथ पृष्ठतः ते पृष्ठतः अपि च ते नमोऽस्तु, ते सर्वत एव सर्वासु दिक्षु सर्वत्र स्थिताय हे सर्व~। अनन्तवीर्यामितविक्रमः अनन्तं वीर्यम् अस्य, अमितः विक्रमः अस्य~। वीर्यं सामर्थ्यं विक्रमः पराक्रमः~। वीर्यवानपि कश्चित् शत्रुवधादिविषये न पराक्रमते, मन्दपराक्रमो वा~। त्वं तु अनन्तवीर्यः अमितविक्रमश्च इति अनन्तवीर्यामितविक्रमः~। सर्वं समस्तं जगत् समाप्तोषि सम्यक् एकेन आत्मना व्याप्नोषि यतः, ततः तस्मात् असि भवसि सर्वः त्वम्~, त्वया विनाभूतं न किञ्चित् अस्ति इति अभिप्रायः~॥~४०~॥\par
 यतः अहं त्वन्माहात्म्यापरिज्ञानात् अपराद्धः, अतः —} 
\begin{center}{\bfseries सखेति मत्वा प्रसभं यदुक्तं\\ हे कृष्ण हे यादव हे सखेति~।\\अजानता महिमानं तवेदं\\ मया प्रमादात्प्रणयेन वापि~॥~४१~॥}\end{center} 
सखा समानवयाः इति मत्वा ज्ञात्वा विपरीतबुद्ध्या प्रसभम् अभिभूय प्रसह्य यत् उक्तं हे कृष्ण हे यादव हे सखेति च अजानता अज्ञानिना मूढेन~; किम् अजानता इति आह — महिमानं महात्म्यं तव इदम् ईश्वरस्य विश्वरूपम्~। ‘तव इदं महिमानम् अजानता’ इति वैयधिकरण्येन सम्बन्धः~। ‘तवेमम्’ इति पाठः यदि अस्ति, तदा सामानाधिकरण्यमेव~। मया प्रमादात् विक्षिप्तचित्ततया, प्रणयेन वापि, प्रणयो नाम स्नेहनिमित्तः विस्रम्भः तेनापि कारणेन यत् उक्तवान् अस्मि~॥~४१~॥\par
 \begin{center}{\bfseries यच्चावहासार्थमसत्कृतोऽसि\\ विहारशय्यासनभोजनेषु~।\\एकोऽथवाप्यच्युत तत्समक्षं\\ तत्क्षामये त्वामहमप्रमेयम्~॥~४२~॥}\end{center} 
यच्च अवहासार्थं परिहासप्रयोजनाय असत्कृतः परिभूतः असि भवसि~; क्व~? विहारशय्यासनभोजनेषु, विहरणं विहारः पादव्यायामः, शयनं शय्या, आसनम् आस्थायिका, भोजनम् अदनम्~, इति एतेषु विहारशय्यासनभोजनेषु, एकः परोक्षः सन् असत्कृतः असि परिभूतः असि~; अथवापि हे अच्युत, तत् समक्षम्~, तच्छब्दः क्रियाविशेषणार्थः, प्रत्यक्षं वा असत्कृतः असि तत् सर्वम् अपराधजातं क्षामये क्षमां कारये त्वाम् अहम् अप्रमेयं प्रमाणातीतम्~॥~४२~॥\par
 यतः त्वम् —} 
\begin{center}{\bfseries पितासि लोकस्य चराचरस्य\\ त्वमस्य पूज्यश्च गुरुर्गरीयान्~।\\न त्वत्समोऽस्त्यभ्यधिकः कुतोऽन्यो\\ लोकत्रयेऽप्यप्रतिमप्रभाव~॥~४३~॥}\end{center} 
पिता असि जनयिता असि लोकस्य प्राणिजातस्य चराचरस्य स्थावरजङ्गमस्य~। न केवलं त्वम् अस्य जगतः पिता, पूज्यश्च पूजार्हः, यतः गुरुः गरीयान् गुरुतरः~। कस्मात् गुरुतरः त्वम् इति आह — न त्वत्समः त्वत्तुल्यः अस्ति~। न हि ईश्वरद्वयं सम्भवति, अनेकेश्वरत्वे व्यवहारानुपपत्तेः~। त्वत्सम एव तावत् अन्यः न सम्भवति~; कुतः एव अन्यः अभ्यधिकः स्यात् लोकत्रयेऽपि सर्वस्मिन्~? अप्रतिमप्रभाव प्रतिमीयते यया सा प्रतिमा, न विद्यते प्रतिमा यस्य तव प्रभावस्य सः त्वम् अप्रतिमप्रभावः, हे अप्रतिमप्रभाव निरतिशयप्रभाव इत्यर्थः~॥~४३~॥\par
 यतः एवम् —} 
\begin{center}{\bfseries तस्मात्प्रणम्य प्रणिधाय कायं\\ प्रसादये त्वामहमीशमीड्यम्~।\\पितेव पुत्रस्य सखेव सख्युः\\ प्रियः प्रियायार्हसि देव सोढुम्~॥~४४~॥}\end{center} 
तस्मात् प्रणम्य नमस्कृत्य, प्रणिधाय प्रकर्षेण नीचैः धृत्वा कायं शरीरम्~, प्रसादये प्रसादं कारये त्वाम् अहम् ईशम् ईशितारम्~, ईड्यं स्तुत्यम्~। त्वं पुनः पुत्रस्य अपराधं पिता यथा क्षमते, सर्वं सखा इव सख्युः अपराधम्~, यथा वा प्रियः प्रियायाः अपराधं क्षमते, एवम् अर्हसि हे देव सोढुं प्रसहितुम् क्षन्तुम् इत्यर्थः~॥~४४~॥\par
 \begin{center}{\bfseries अदृष्टपूर्वं हृषितोऽस्मि दृष्ट्वा\\ भयेन च प्रव्यथितं मनो मे~।\\तदेव मे दर्शय देव रूपं\\ प्रसीद देवेश जगन्निवास~॥~४५~॥}\end{center} 
अदृष्टपूर्वं न कदाचिदपि दृष्टपूर्वम् इदं विश्वरूपं तव मया अन्यैर्वा, तत् अहं दृष्ट्वा हृषितः अस्मि~। भयेन च प्रव्यथितं मनः मे~। अतः तदेव मे मम दर्शय हे देव रूपं यत् मत्सखम्~। प्रसीद देवेश, जगन्निवास जगतो निवासो जगन्निवासः, हे जगन्निवास~॥~४५~॥\par
 \begin{center}{\bfseries किरीटिनं गदिनं चक्रहस्त—\\ मिच्छामि त्वां द्रष्टुमहं तथैव~।\\तेनैव रूपेण चतुर्भुजेन\\ सहस्रबाहो भव विश्वमूर्ते~॥~४६~॥}\end{center} 
किरीटिनं किरीटवन्तं तथा गदिनं गदावन्तं चक्रहस्तम् इच्छामि त्वां प्रार्थये त्वां द्रष्टुम् अहं तथैव, पूर्ववत् इत्यर्थः~। यतः एवम्~, तस्मात् तेनैव रूपेण वसुदेवपुत्ररूपेण चतुर्भुजेन, सहस्रबाहो वार्तमानिकेन विश्वरूपेण, भव विश्वमूर्ते~; उपसंहृत्य विश्वरूपम्~, तेनैव रूपेण भव इत्यर्थः~॥~४६~॥\par
 अर्जुनं भीतम् उपलभ्य, उपसंहृत्य विश्वरूपम्~, प्रियवचनेन आश्वासयन् श्रीभगवान् उवाच —}\\ 
\begin{center}{\bfseries श्रीभगवानुवाच —\\ मया प्रसन्नेन तवार्जुनेदं\\ रूपं परं दर्शितमात्मयोगात्~।\\तेजोमयं विश्वमनन्तमाद्यं\\ यन्मे त्वदन्येन न दृष्टपूर्वम्~॥~४७~॥}\end{center} 
मया प्रसन्नेन, प्रसादो नाम त्वयि अनुग्रहबुद्धिः, तद्वता प्रसन्नेन मया तव हे अर्जुन, इदं परं रूपं विश्वरूपं दर्शितम् आत्मयोगात् आत्मनः ऐश्वर्यस्य सामर्थ्यात्~। तेजोमयं तेजःप्रायं विश्वं समस्तम् अनन्तम् अन्तरहितं आदौ भवम् आद्यं यत् रूपं मे मम त्वदन्येन त्वत्तः अन्येन केनचित् न दृष्टपूर्वम्~॥~४७~॥\par
 आत्मनः मम रूपदर्शनेन कृतार्थ एव त्वं संवृत्तः इति तत् स्तौति —} 
\begin{center}{\bfseries न वेदयज्ञाध्ययनैर्न दानै—\\ र्न च क्रियाभिर्न तपोभिरुग्रैः~।\\एवंरूपः शक्य अहं नृलोके\\ द्रष्टुं त्वदन्येन कुरुप्रवीर~॥~४८~॥}\end{center} 
न वेदयज्ञाध्ययनैः चतुर्णामपि वेदानाम् अध्ययनैः यथावत् यज्ञाध्ययनैश्च — वेदाध्ययनैरेव यज्ञाध्ययनस्य सिद्धत्वात् पृथक् यज्ञाध्ययनग्रहणं यज्ञविज्ञानोपलक्षणार्थम् — तथा न दानैः तुलापुरुषादिभिः, न च क्रियाभिः अग्निहोत्रादिभिः श्रौतादिभिः, न अपि तपोभिः उग्रैः चान्द्रायणादिभिः उग्रैः घोरैः, एवंरूपः यथादर्शितं विश्वरूपं यस्य सोऽहम् एवंरूपः न शक्यः अहं नृलोके मनुष्यलोके द्रष्टुं त्वदन्येन त्वत्तः अन्येन कुरुप्रवीर~॥~४८~॥\par
 \begin{center}{\bfseries मा ते व्यथा मा च विमूढभावो\\ दृष्ट्वा रूपं घोरमीदृङ्ममेदम्~।\\व्यपेतभीः प्रीतमनाः पुनस्त्वं\\ तदेव मे रूपमिदं प्रपश्य~॥~४९~॥}\end{center} 
मा ते व्यथा मा भूत् ते भयम्~, मा च विमूढभावः विमूढचित्तता, दृष्ट्वा उपलभ्य रूपं घोरम् ईदृक् यथादर्शितं मम इदम्~। व्यपेतभीः विगतभयः, प्रीतमनाश्च सन् पुनः भूयः त्वं तदेव चतुर्भुजं रूपं शङ्खचक्रगदाधरं तव इष्टं रूपम् इदं प्रपश्य~॥~४९~॥\par
\begin{center}{\bfseries सञ्जय उवाच —\\ इत्यर्जुनं वासुदेवस्तथोक्त्वा\\ स्वकं रूपं दर्शयामास भूयः~।\\आश्वासयामास च भीतमेनं\\ भूत्वा पुनःसौम्यवपुर्महात्मा~॥~५०~॥}\end{center} 
इति एवम् अर्जुनं वासुदेवः तथाभूतं वचनम् उक्त्वा, स्वकं वसुदेवस्य गृहे जातं रूपं दर्शयामास दर्शितवान् भूयः पुनः~। आश्वासयामास च आश्वासितवान् भीतम् एनम्~, भूत्वा पुनः सौम्यवपुः प्रसन्नदेहः महात्मा~॥~५०~॥\par
\begin{center}{\bfseries अर्जुन उवाच —\\ दृष्ट्वेदं मानुषं रूपं\\ तव सौम्यं जनार्दन~।\\इदानीमस्मि संवृत्तः\\ सचेताः प्रकृतिं गतः~॥~५१~॥}\end{center} 
दृष्ट्वा इदं मानुषं रूपं मत्सखं प्रसन्नं तव सौम्यं जनार्दन, इदानीम् अधुना अस्मि संवृत्तः सञ्जातः~। किम्~? सचेताः प्रसन्नचित्तः प्रकृतिं स्वभावं गतश्च अस्मि~॥~५१~॥\par
\begin{center}{\bfseries श्रीभगवानुवाच —\\ सुदुर्दर्शमिदं रूपं\\ दृष्टवानसि यन्मम~।\\देवा अप्यस्य रूपस्य\\ नित्यं दर्शनकाङ्क्षिणः~॥~५२~॥}\end{center} 
सुदुर्दर्शं सुष्ठु दुःखेन दर्शनम् अस्य इति सुदुर्दर्शम्~, इदं रूपं दृष्टवान् असि यत् मम, देवादयः अपि अस्य मम रूपस्य नित्यं सर्वदा दर्शनकाङ्क्षिणः~; दर्शनेप्सवोऽपि न त्वमिव दृष्टवन्तः, न द्रक्ष्यन्ति च इति अभिप्रायः~॥~५२~॥\par
 कस्मात्~? —} 
\begin{center}{\bfseries नाहं वेदैर्न तपसा\\ न दानेन न चेज्यया~।\\शक्य एवंविधो द्रष्टुं\\ दृष्टवानसि मां यथा~॥~५३~॥}\end{center} 
न अहं वेदैः ऋग्यजुःसामाथर्ववेदैः चतुर्भिरपि, न तपसा उग्रेण चान्द्रायणादिना, न दानेन गोभूहिरण्यादिना, न च इज्यया यज्ञेन पूजया वा शक्यः एवंविधः यथादर्शितप्रकारः द्रष्टुं दृष्टावान् असि मां यथा त्वम्~॥~५३~॥\par
 कथं पुनः शक्यः इति उच्यते —} 
\begin{center}{\bfseries भक्त्या त्वनन्यया शक्य\\ अहमेवंविधोऽर्जुन~।\\ज्ञातुं द्रष्टुं च तत्त्वेन\\ प्रवेष्टुं च परन्तप~॥~५४~॥}\end{center} 
भक्त्या तु किंविशिष्टया इति आह — अनन्यया अपृथग्भूतया, भगवतः अन्यत्र पृथक् न कदाचिदपि या भवति सा त्वनन्या भक्तिः~। सर्वैरपि करणैः वासुदेवादन्यत् न उपलभ्यते यया, सा अनन्या भक्तिः, तया भक्त्या शक्यः अहम् एवंविधः विश्वरूपप्रकारः हे अर्जुन, ज्ञातुं शास्त्रतः~। न केवलं ज्ञातुं शास्त्रतः, द्रष्टुं च साक्षात्कर्तुं तत्त्वेन तत्त्वतः, प्रवेष्टुं च मोक्षं च गन्तुं परन्तप~॥~५४~॥\par
 अधुना सर्वस्य गीताशास्त्रस्य सारभूतः अर्थः निःश्रेयसार्थः अनुष्ठेयत्वेन समुच्चित्य उच्यते —} 
\begin{center}{\bfseries मत्कर्मकृन्मत्परमो\\ मद्भक्तः सङ्गवर्जितः~।\\निर्वैरः सर्वभूतेषु\\ यः स मामेति पाण्डव~॥~५५~॥}\end{center} 
मत्कर्मकृत् मदर्थं कर्म मत्कर्म, तत् करोतीति मत्कर्मकृत्~। मत्परमः — करोति भृत्यः स्वामिकर्म, न तु आत्मनः परमा प्रेत्य गन्तव्या गतिरिति स्वामिनं प्रतिपद्यते~; अयं तु मत्कर्मकृत् मामेव परमां गतिं प्रतिपद्यते इति मत्परमः, अहं परमः परा गतिः यस्य सोऽयं मत्परमः~। तथा मद्भक्तः मामेव सर्वप्रकारैः सर्वात्मना सर्वोत्साहेन भजते इति मद्भक्तः~। सङ्गवर्जितः धनपुत्रमित्रकलत्रबन्धुवर्गेषु सङ्गवर्जितः सङ्गः प्रीतिः स्नेहः तद्वर्जितः~। निर्वैरः निर्गतवैरः सर्वभूतेषु शत्रुभावरहितः आत्मनः अत्यन्तापकारप्रवृत्तेष्वपि~। यः ईदृशः मद्भक्तः सः माम् एति, अहमेव तस्य परा गतिः, न अन्या गतिः काचित् भवति~। अयं तव उपदेशः इष्टः मया उपदिष्टः हे पाण्डव इति~॥~५५~॥\par
 
इति श्रीमत्परमहंसपरिव्राजकाचार्यस्य श्रीगोविन्दभगवत्पूज्यपादशिष्यस्य श्रीमच्छङ्करभगवतः कृतौ श्रीमद्भगवद्गीताभाष्ये एकादशोऽध्यायः~॥\par
 
द्वितीयाध्यायप्रभृतिषु विभूत्यन्तेषु अध्यायेषु परमात्मनः ब्रह्मणः अक्षरस्य विध्वस्तसर्वोपाधिविशेषस्य उपासनम् उक्तम्~; सर्वयोगैश्वर्यसर्वज्ञानशक्तिमत्सत्त्वोपाधेः ईश्वरस्य तव च उपासनं तत्र तत्र उक्तम्~। विश्वरूपाध्याये तु ऐश्वरम् आद्यं समस्तजगदात्मरूपं विश्वरूपं त्वदीयं दर्शितम् उपासनार्थमेव त्वया~। तच्च दर्शयित्वा उक्तवानसि ‘मत्कर्मकृत्’\footnote{भ. गी. ११~। ५५} इत्यादि~। अतः अहम् अनयोः उभयोः पक्षयोः विशिष्टतरबुभुत्सया त्वां पृच्छामि इति अर्जुन उवाच —}\\ 
\begin{center}{\bfseries अर्जुन उवाच —\\ एवं सततयुक्ता ये भक्तास्त्वां पर्युपासते~।\\ये चाप्यक्षरमव्यक्तं तेषां के योगवित्तमाः~॥~१~॥}\end{center} 
एवम् इति अतीतानन्तरश्लोकेन उक्तम् अर्थं परामृशति ‘मत्कर्मकृत्’\footnote{भ. गी. ११~। ५५} इत्यादिना~। एवं सततयुक्ताः, नैरन्तर्येण भगवत्कर्मादौ यथोक्ते अर्थे समाहिताः सन्तः प्रवृत्ता इत्यर्थः~। ये भक्ताः अनन्यशरणाः सन्तः त्वां यथादर्शितं विश्वरूपं पर्युपासते ध्यायन्ति~; ये चान्येऽपि त्यक्तसर्वैषणाः संन्यस्तसर्वकर्माणः यथाविशेषितं ब्रह्म अक्षरं निरस्तसर्वोपाधित्वात् अव्यक्तम् अकरणगोचरम्~। यत् हि करणगोचरं तत् व्यक्तम् उच्यते, अञ्जेः धातोः तत्कर्मकत्वात्~; इदं तु अक्षरं तद्विपरीतम्~, शिष्टैश्च उच्यमानैः विशेषणैः विशिष्टम्~, तत् ये चापि पर्युपासते, तेषाम् उभयेषां मध्ये के योगवित्तमाः~? के अतिशयेन योगविदः इत्यर्थः~॥~१~॥\par
 श्रीभगवान् उवाच — ये तु अक्षरोपासकाः सम्यग्दर्शिनः निवृत्तैषणाः, ते तावत् तिष्ठन्तु~; तान् प्रति यत् वक्तव्यम्~, तत् उपरिष्टात् वक्ष्यामः~। ये तु इतरे —} 
\begin{center}{\bfseries श्रीभगवानुवाच —\\ मय्यावेश्य मनो ये मां नित्ययुक्ता उपासते~।\\श्रद्धया परयोपेतास्ते मे युक्ततमा मताः~॥~२~॥}\end{center} 
मयि विश्वरूपे परमेश्वरे आवेश्य समाधाय मनः, ये भक्ताः सन्तः, मां सर्वयोगेश्वराणाम् अधीश्वरं सर्वज्ञं विमुक्तरागादिक्लेशतिमिरदृष्टिम्~, नित्ययुक्ताः अतीतानन्तराध्यायान्तोक्तश्लोकार्थन्यायेन सततयुक्ताः सन्तः उपासते श्रद्धया परया प्रकृष्टया उपेताः, ते मे मम मताः अभिप्रेताः युक्ततमाः इति~। नैरन्तर्येण हि ते मच्चित्ततया अहोरात्रम् अतिवाहयन्ति~। अतः युक्तं तान् प्रति युक्ततमाः इति वक्तुम्~॥~२~॥\par
 किमितरे युक्ततमाः न भवन्ति~? न~; किन्तु तान् प्रति यत् वक्तव्यम्~, तत् शृणु —} 
\begin{center}{\bfseries ये त्वक्षरमनिर्देश्यमव्यक्तं पर्युपासते~।\\सर्वत्रगमचिन्त्यं च कूटस्थमचलं ध्रुवम्~॥~३~॥}\end{center} 
ये तु अक्षरम् अनिर्देश्यम्~, अव्यक्तत्वात् अशब्दगोचर इति न निर्देष्टुं शक्यते, अतः अनिर्देश्यम्~, अव्यक्तं न केनापि प्रमाणेन व्यज्यत इत्यव्यक्तं पर्युपासते परि समन्तात् उपासते~। उपासनं नाम यथाशास्त्रम् उपास्यस्य अर्थस्य विषयीकरणेन सामीप्यम् उपगम्य तैलधारावत् समानप्रत्ययप्रवाहेण दीर्घकालं यत् आसनम्~, तत् उपासनमाचक्षते~। अक्षरस्य विशेषणमाह उपास्यस्य — सर्वत्रगं व्योमवत् व्यापि अचिन्त्यं च अव्यक्तत्वादचिन्त्यम्~। यद्धि करणगोचरम्~, तत् मनसापि चिन्त्यम्~, तद्विपरीतत्वात् अचिन्त्यम् अक्षरम्~, कूटस्थं दृश्यमानगुणम् अन्तर्दोषं वस्तु कूटम्~। ‘कूटरूपम्’ ’ कूटसाक्ष्यम्’ इत्यादौ कूटशब्दः प्रसिद्धः लोके~। तथा च अविद्याद्यनेकसंसारबीजम् अन्तर्दोषवत् मायाव्याकृतादिशब्दवाच्यतया ‘मायां तु प्रकृतिं विद्यान्मायिनं तु महेश्वरम्’\footnote{श्वे. उ. ४~। १०} ‘मम माया दुरत्यया’\footnote{भ. गी. ७~। १४} इत्यादौ प्रसिद्धं यत् तत् कूटम्~, तस्मिन् कूटे स्थितं कूटस्थं तदध्यक्षतया~। अथवा, राशिरिव स्थितं कूटस्थम्~। अत एव अचलम्~। यस्मात् अचलम्~, तस्मात् ध्रुवम्~, नित्यमित्यर्थः~॥~३~॥\par
 \begin{center}{\bfseries संनियम्येन्द्रियग्रामं सर्वत्र समबुद्धयः~।\\ते प्राप्नुवन्ति मामेव सर्वभूतहिते रताः~॥~४~॥}\end{center} 
सन्नियम्य सम्यक् नियम्य उपसंहृत्य इन्द्रियग्रामम् इन्द्रियसमुदायं सर्वत्र सर्वस्मिन् काले समबुद्धयः समा तुल्या बुद्धिः येषाम् इष्टानिष्टप्राप्तौ ते समबुद्धयः~। ते ये एवंविधाः ते प्राप्नुवन्ति मामेव सर्वभूतहिते रताः~। न तु तेषां वक्तव्यं किञ्चित् ‘मां ते प्राप्नुवन्ति’ इति~; ‘ज्ञानी त्वात्मैव मे मतम्’\footnote{भ. गी. ७~। १८} इति हि उक्तम्~। न हि भगवत्स्वरूपाणां सतां युक्ततमत्वमयुक्ततमत्वं वा वाच्यम्~॥~४~॥\par
 किं तु —} 
\begin{center}{\bfseries क्लेशोऽधिकतरस्तेषामव्यक्तासक्तचेतसाम्~।\\अव्यक्ता हि गतिर्दुःखं देहवद्भिरवाप्यते~॥~५~॥}\end{center} 
क्लेशः अधिकतरः, यद्यपि मत्कर्मादिपराणां क्लेशः अधिक एव क्लेशः अधिकतरस्तु अक्षरात्मनां परमात्मदर्शिनां देहाभिमानपरित्यागनिमित्तः~। अव्यक्तासक्तचेतसाम् अव्यक्ते आसक्तं चेतः येषां ते अव्यक्तासक्तचेतसः तेषाम् अव्यक्तासक्तचेतसाम्~। अव्यक्ता हि यस्मात् या गतिः अक्षरात्मिका दुःखं सा देहवद्भिः देहाभिमानवद्भिः अवाप्यते, अतः क्लेशः अधिकतरः~॥~५~॥\par
 अक्षरोपासकानां यत् वर्तनम्~, तत् उपरिष्टाद्वक्ष्यामः —} 
\begin{center}{\bfseries ये तु सर्वाणि कर्माणि मयि संन्यस्य मत्पराः~।\\अनन्येनैव योगेन मां ध्यायन्त उपासते~॥~६~॥}\end{center} 
ये तु सर्वाणि कर्माणि मयि ईश्वरे संन्यस्य मत्पराः अहं परः येषां ते मत्पराः सन्तः अनन्येनैव अविद्यमानम् अन्यत् आलम्बनं विश्वरूपं देवम् आत्मानं मुक्त्वा यस्य सः अनन्यः तेन अनन्येनैव~; केन~? योगेन समाधिना मां ध्यायन्तः चिन्तयन्तः उपासते~॥~६~॥\par
 तेषां किम्~? —} 
\begin{center}{\bfseries तेषामहं समुद्धर्ता मृत्युसंसारसागरात्~।\\भवामि न चिरात्पार्थ मय्यावेशितचेतसाम्~॥~७~॥}\end{center} 
तेषां मदुपासनैकपराणाम् अहम् ईश्वरः समुद्धर्ता~। कुतः इति आह — मृत्युसंसारसागरात् मृत्युयुक्तः संसारः मृत्युसंसारः, स एव सागर इव सागरः, दुस्तरत्वात्~, तस्मात् मृत्युसंसारसागरात् अहं तेषां समुद्धर्ता भवामि न चिरात्~। किं तर्हि~? क्षिप्रमेव हे पार्थ, मयि आवेशितचेतसां मयि विश्वरूपे आवेशितं समाहितं चेतः येषां ते मय्यावेशितचेतसः तेषाम्~॥~७~॥\par
 यतः एवम्~, तस्मात् —} 
\begin{center}{\bfseries मय्येव मन आधत्स्व मयि बुद्धिं निवेशय~।\\निवसिष्यसि मय्येव अत ऊर्ध्वं न संशयः~॥~८~॥}\end{center} 
मयि एव विश्वरूपे ईश्वरे मनः सङ्कल्पविकल्पात्मकं आधत्स्व स्थापय~। मयि एव अध्यवसायं कुर्वतीं बुद्धिम् आधत्स्व निवेशय~। ततः ते किं स्यात् इति शृणु — निवसिष्यसि निवत्स्यसि निश्चयेन मदात्मना मयि निवासं करिष्यसि एव अतः शरीरपातात् ऊर्ध्वम्~। न संशयः संशयः अत्र न कर्तव्यः~॥~८~॥\par
 \begin{center}{\bfseries अथ चित्तं समाधातुं\\ न शक्नोषि मयि स्थिरम्~।\\अभ्यासयोगेन ततो\\ मामिच्छाप्तुं धनञ्जय~॥~९~॥}\end{center} 
अथ एवं यथा अवोचं तथा मयि चित्तं समाधातुं स्थापयितुं स्थिरम् अचलं न शक्नोषि चेत्~, ततः पश्चात् अभ्यासयोगेन, चित्तस्य एकस्मिन् आलम्बने सर्वतः समाहृत्य पुनः पुनः स्थापनम् अभ्यासः, तत्पूर्वको योगः समाधानलक्षणः तेन अभ्यासयोगेन मां विश्वरूपम् इच्छ प्रार्थयस्व आप्तुं प्राप्तुं हे धनञ्जय~॥~९~॥\par
 \begin{center}{\bfseries अभ्यासेऽप्यसमर्थोऽसि\\ मत्कर्मपरमो भव~।\\मदर्थमपि कर्माणि\\ कुर्वन्सिद्धिमवाप्स्यसि~॥~१०~॥}\end{center} 
अभ्यासे अपि असमर्थः असि अशक्तः असि, तर्हि मत्कर्मपरमः भव मदर्थं कर्म मत्कर्म तत्परमः मत्कर्मपरमः, मत्कर्मप्रधानः इत्यर्थः~। अभ्यासेन विना मदर्थमपि कर्माणि केवलं कुर्वन् सिद्धिं सत्त्वशुद्धियोगज्ञानप्राप्तिद्वारेण अवाप्स्यसि~॥~१०~॥\par
 \begin{center}{\bfseries अथैतदप्यशक्तोऽसि कर्तुं मद्योगमाश्रितः~।\\सर्वकर्मफलत्यागं ततः कुरु यतात्मवान्~॥~११~॥}\end{center} 
अथ पुनः एतदपि यत् उक्तं मत्कर्मपरमत्वम्~, तत् कर्तुम् अशक्तः असि, मद्योगम् आश्रितः मयि क्रियमाणानि कर्माणि संन्यस्य यत् करणं तेषाम् अनुष्ठानं सः मद्योगः, तम् आश्रितः सन्~, सर्वकर्मफलत्यागं सर्वेषां कर्मणां फलसंन्यासं सर्वकर्मफलत्यागं ततः अनन्तरं कुरु यतात्मवान् संयतचित्तः सन् इत्यर्थः~॥~११~॥\par
 इदानीं सर्वकर्मफलत्यागं स्तौति —} 
\begin{center}{\bfseries श्रेयो हि ज्ञानमभ्यासा—\\ ज्ज्ञानाद्ध्यानं विशिष्यते~।\\ध्यानात्कर्मफलत्याग—\\ स्त्यागाच्छान्तिरनन्तरम्~॥~१२~॥}\end{center} 
श्रेयः हि प्रशस्यतरं ज्ञानम्~। कस्मात्~? विवेकपूर्वकात् अभ्यासात्~। तस्मादपि ज्ञानात् ज्ञानपूर्वकं ध्यानं विशिष्यते~। ज्ञानवतो ध्यानात् अपि कर्मफलत्यागः, ‘विशिष्यते’ इति अनुषज्यते~। एवं कर्मफलत्यागात् पूर्वविशेषणवतः शान्तिः उपशमः सहेतुकस्य संसारस्य अनन्तरमेव स्यात्~, न तु कालान्तरम् अपेक्षते~॥~} 
अज्ञस्य कर्मणि प्रवृत्तस्य पूर्वोपदिष्टोपायानुष्ठानाशक्तौ सर्वकर्मणां फलत्यागः श्रेयःसाधनम् उपदिष्टम्~, न प्रथममेव~। अतश्च ‘श्रेयो हि ज्ञानमभ्यासात्’ इत्युत्तरोत्तरविशिष्टत्वोपदेशेन सर्वकर्मफलत्यागः स्तूयते, सम्पन्नसाधनानुष्ठानाशक्तौ अनुष्ठेयत्वेन श्रुतत्वात्~। केन साधर्म्येण स्तुतित्वम्~? ‘यदा सर्वे प्रमुच्यन्ते’\footnote{क. उ. २~। ३~। १४} इति सर्वकामप्रहाणात् अमृतत्वम् उक्तम्~; तत् प्रसिद्धम्~। कामाश्च सर्वे श्रौतस्मार्तकर्मणां फलानि~। तत्त्यागे च विदुषः ध्याननिष्ठस्य अनन्तरैव शान्तिः इति सर्वकामत्यागसामान्यम् अज्ञकर्मफलत्यागस्य अस्ति इति तत्सामान्यात् सर्वकर्मफलत्यागस्तुतिः इयं प्ररोचनार्था~। यथा अगस्त्येन ब्राह्मणेन समुद्रः पीतः इति इदानीन्तनाः अपि ब्राह्मणाः ब्राह्मणत्वसामान्यात् स्तूयन्ते, एवं कर्मफलत्यागात् कर्मयोगस्य श्रेयःसाधनत्वमभिहितम्~॥~१२~॥\par
 अत्र च आत्मेश्वरभेदमाश्रित्य विश्वरूपे ईश्वरे चेतःसमाधानलक्षणः योगः उक्तः, ईश्वरार्थं कर्मानुष्ठानादि च~। ‘अथैतदप्यशक्तोऽसि’\footnote{भ. गी. १२~। ११} इति अज्ञानकार्यसूचनात् न अभेददर्शिनः अक्षरोपासकस्य कर्मयोगः उपपद्यते इति दर्शयति~; तथा कर्मयोगिनः अक्षरोपासनानुपपत्तिम्~। ‘ते प्राप्नुवन्ति मामेव’\footnote{भ. गी. १२~। ४} इति अक्षरोपासकानां कैवल्यप्राप्तौ स्वातन्त्र्यम् उक्त्वा, इतरेषां पारतन्त्र्यात् ईश्वराधीनतां दर्शितवान् ‘तेषामहं समुद्धर्ता’\footnote{भ. गी. १२~। ७} इति~। यदि हि ईश्वरस्य आत्मभूताः ते मताः अभेददर्शित्वात्~, अक्षरस्वरूपाः एव ते इति समुद्धरणकर्मवचनं तान् प्रति अपेशलं स्यात्~। यस्माच्च अर्जुनस्य अत्यन्तमेव हितैषी भगवान् तस्य सम्यग्दर्शनानन्वितं कर्मयोगं भेददृष्टिमन्तमेव उपदिशति~। न च आत्मानम् ईश्वरं प्रमाणतः बुद्ध्वा कस्यचित् गुणभावं जिगमिषति कश्चित्~, विरोधात्~। तस्मात् अक्षरोपासकानां सम्यग्दर्शननिष्ठानां संन्यासिनां त्यक्तसर्वैषणानाम् ‘अद्वेष्टा सर्वभूतानाम्’ इत्यादिधर्मपूगं साक्षात् अमृतत्वकारणं वक्ष्यामीति प्रवर्तते —} 
\begin{center}{\bfseries अद्वेष्टा सर्वभूतानां मैत्रः करुण एव च~।\\निर्ममो निरहङ्कारः समदुःखसुखः क्षमी~॥~१३~॥}\end{center} 
अद्वेष्टा सर्वभूतानां न द्वेष्टा, आत्मनः दुःखहेतुमपि न किञ्चित् द्वेष्टि, सर्वाणि भूतानि आत्मत्वेन हि पश्यति~। मैत्रः मित्रभावः मैत्री मित्रतया वर्तते इति मैत्रः~। करुणः एव च, करुणा कृपा दुःखितेषु दया, तद्वान् करुणः, सर्वभूताभयप्रदः, संन्यासी इत्यर्थः~। निर्ममः ममप्रत्ययवर्जितः~। निरहङ्कारः निर्गताहंप्रत्ययः~। समदुःखसुखः समे दुःखसुखे द्वेषरागयोः अप्रवर्तके यस्य सः समदुःखसुखः~। क्षमी क्षमावान्~, आक्रुष्टः अभिहतो वा अविक्रियः एव आस्ते~॥~१३~॥\par
 \begin{center}{\bfseries सन्तुष्टः सततं योगी\\ यतात्मा दृढनिश्चयः~।\\मय्यर्पितमनोबुद्धि—\\ र्यो मद्भक्तः स मे प्रियः~॥~१४~॥}\end{center} 
सन्तुष्टः सततं नित्यं देहस्थितिकारणस्य लाभे अलाभे च उत्पन्नालंप्रत्ययः~। तथा गुणवल्लाभे विपर्यये च सन्तुष्टः~। सततं योगी समाहितचित्तः~। यतात्मा संयतस्वभावः~। दृढनिश्चयः दृढः स्थिरः निश्चयः अध्यवसायः यस्य आत्मतत्त्वविषये स दृढनिश्चयः~। मय्यर्पितमनोबुद्धिः सङ्कल्पविकल्पात्मकं मनः, अध्यवसायलक्षणा बुद्धिः, ते मय्येव अर्पिते स्थापिते यस्य संन्यासिनः सः मय्यर्पितमनोबुद्धिः~। यः ईदृशः मद्भक्तः सः मे प्रियः~। ‘प्रियो हि ज्ञानिनोऽत्यर्थमहं स च मम प्रियः’\footnote{भ. गी. ७~। १७} इति सप्तमे अध्याये सूचितम्~, तत् इह प्रपञ्च्यते~॥~१४~॥\par
 \begin{center}{\bfseries यस्मान्नोद्विजते लोको\\ लोकान्नोद्विजते च यः~।\\ हर्षामर्षभयोद्वेगै—\\ र्मुक्तो यः स च मे प्रियः~॥~१५~॥}\end{center} 
यस्मात् संन्यासिनः न उद्विजते न उद्वेगं गच्छति न सन्तप्यते न सङ्क्षुभ्यति लोकः, तथा लोकात् न उद्विजते च यः, हर्षामर्षभयोद्वेगैः हर्षश्च अमर्षश्च भयं च उद्वेगश्च तैः हर्षामर्षभयोद्वेगैः मुक्तः~; हर्षः प्रियलाभे अन्तःकरणस्य उत्कर्षः रोमाञ्चनाश्रुपातादिलिङ्गः, अमर्षः असहिष्णुता, भयं त्रासः, उद्वेगः उद्विग्नता, तैः मुक्तः यः स च मे प्रियः~॥~१५~॥\par
 \begin{center}{\bfseries अनपेक्षः शुचिर्दक्ष\\ उदासीनो गतव्यथः~।\\सर्वारम्भपरित्यागी\\ यो मद्भक्तः स मे प्रियः~॥~१६~॥}\end{center} 
देहेन्द्रियविषयसम्बन्धादिषु अपेक्षाविषयेषु अनपेक्षः निःस्पृहः~। शुचिः बाह्येन आभ्यन्तरेण च शौचेन सम्पन्नः~। दक्षः प्रत्युत्पन्नेषु कार्येषु सद्यः यथावत् प्रतिपत्तुं समर्थः~। उदासीनः न कस्यचित् मित्रादेः पक्षं भजते यः, सः उदासीनः यतिः~। गतव्यथः गतभयः~। सर्वारम्भपरित्यागी आरभ्यन्त इति आरम्भाः इहामुत्रफलभोगार्थानि कामहेतूनि कर्माणि सर्वारम्भाः, तान् परित्यक्तुं शीलम् अस्येति सर्वारम्भपरित्यागी यः मद्भक्तः सः मे प्रियः~॥~१६~॥\par
 किञ्च —} 
\begin{center}{\bfseries यो न हृष्यति न द्वेष्टि\\ न शोचति न काङ्क्षति~।\\शुभाशुभपरित्यागी\\ भक्तिमान्यः स मे प्रियः~॥~१७~॥}\end{center} 
यः न हृष्यति इष्टप्राप्तौ, न द्वेष्टि अनिष्टप्राप्तौ, न शोचति प्रियवियोगे, न च अप्राप्तं काङ्क्षति, शुभाशुभे कर्मणी परित्यक्तुं शीलम् अस्य इति शुभाशुभपरित्यागी भक्तिमान् यः सः मे प्रियः~॥~१७~॥\par
 \begin{center}{\bfseries समः शत्रौ च मित्रे च\\ तथा मानापमानयोः~।\\शीतोष्णसुखदुःखेषु\\ समः सङ्गविवर्जितः~॥~१८~॥}\end{center} 
समः शत्रौ च मित्रे च, तथा मानापमानयोः पूजापरिभवयोः, शीतोष्णसुखदुःखेषु समः, सर्वत्र च सङ्गविवर्जितः~॥~१८~॥\par
 किञ्च —} 
\begin{center}{\bfseries तुल्यनिन्दास्तुतिर्मौनी\\ सन्तुष्टो येन केनचित्~।\\अनिकेतः स्थिरमति—\\ र्भक्तिमान्मे प्रियो नरः~॥~१९~॥}\end{center} 
तुल्यनिन्दास्तुतिः निन्दा च स्तुतिश्च निन्दास्तुती ते तुल्ये यस्य सः तुल्यनिन्दास्तुतिः~। मौनी मौनवान् संयतवाक्~। सन्तुष्टः येन केनचित् शरीरस्थितिहेतुमात्रेण~; तथा च उक्तम् — ‘येन केनचिदाच्छन्नो येन केनचिदाशितः~। यत्र क्वचन शायी स्यात्तं देवा ब्राह्मणं विदुः’\footnote{मो. ध. २४५~। १२} इति~। किञ्च, अनिकेतः निकेतः आश्रयः निवासः नियतः न विद्यते यस्य सः अनिकेतः, ‘नागारे’\footnote{~? } इत्यादिस्मृत्यन्तरात्~। स्थिरमतिः स्थिरा परमार्थविषया यस्य मतिः सः स्थिरमतिः~। भक्तिमान् मे प्रियः नरः~॥~१९~॥\par
 ‘अद्वेष्टा सर्वभूतानाम्’\footnote{भ. गी. १२~। १३}, इत्यादिना अक्षरोपासकानां निवृत्तसर्वैषणानां सन्यासिनां परमार्थज्ञाननिष्ठानां धर्मजातं प्रक्रान्तम् उपसंह्रियते —} 
\begin{center}{\bfseries ये तु धर्म्यामृतमिदं\\ यथोक्तं पर्युपासते~।\\श्रद्दधाना मत्परमा\\ भक्तास्तेऽतीव मे प्रियाः~॥~२०~॥}\end{center} 
ये तु संन्यासिनः धर्म्यामृतं धर्मादनपेतं धर्म्यं च तत् अमृतं च तत्~, अमृतत्वहेतुत्वात्~, इदं यथोक्तम् ‘अद्वेष्टा सर्वभूतानाम्’\footnote{भ. गी. १२~। १३} इत्यादिना पर्युपासते अनुतिष्ठन्ति श्रद्दधानाः सन्तः मत्परमाः यथोक्तः अहं अक्षरात्मा परमः निरतिशया गतिः येषां ते मत्परमाः, मद्भक्ताः च उत्तमां परमार्थज्ञानलक्षणां भक्तिमाश्रिताः, ते अतीव मे प्रियाः~। ‘प्रियो हि ज्ञानिनोऽत्यर्थम्’\footnote{भ. गी. ७~। १७} इति यत् सूचितं तत् व्याख्याय इह उपसंहृतम् ‘भक्तास्तेऽतीव मे प्रियाः’ इति~। यस्मात् धर्म्यामृतमिदं यथोक्तमनुतिष्ठन् भगवतः विष्णोः परमेश्वरस्य अतीव प्रियः भवति, तस्मात् इदं धर्म्यामृतं मुमुक्षुणा यत्नतः अनुष्ठेयं विष्णोः प्रियं परं धाम जिगमिषुणा इति वाक्यार्थः~॥~२०~॥\par
 
इति श्रीमत्परमहंसपरिव्राजकाचार्यस्य श्रीगोविन्दभगवत्पूज्यपादशिष्यस्य श्रीमच्छङ्करभगवतः कृतौ श्रीमद्भगवद्गीताभाष्ये द्वादशोऽध्यायः~॥\par
 
सप्तमे अध्याये सूचिते द्वे प्रकृती ईश्वरस्य — त्रिगुणात्मिका अष्टधा भिन्ना अपरा, संसारहेतुत्वात्~; परा च अन्या जीवभूता क्षेत्रज्ञलक्षणा ईश्वरात्मिका — याभ्यां प्रकृतिभ्यामीश्वरः जगदुत्पत्तिस्थितिलयहेतुत्वं प्रतिपद्यते~। तत्र क्षेत्रक्षेत्रज्ञलक्षणप्रकृतिद्वयनिरूपणद्वारेण तद्वतः ईश्वरस्य तत्त्वनिर्धारणार्थं क्षेत्राध्यायः आरभ्यते~। अतीतानन्तराध्याये च ‘अद्वेष्टा सर्वभूतानाम्’\footnote{भ. गी. १२~। १३} इत्यादिना यावत् अध्यायपरिसमाप्तिः तावत् तत्त्वज्ञानिनां संन्यासिनां निष्ठा यथा ते वर्तन्ते इत्येतत् उक्तम्~। केन पुनः ते तत्त्वज्ञानेन युक्ताः यथोक्तधर्माचरणात् भगवतः प्रिया भवन्तीति एवमर्थश्च अयमध्यायः आरभ्यते~। प्रकृतिश्च त्रिगुणात्मिका सर्वकार्यकरणविषयाकारेण परिणता पुरुषस्य भोगापवर्गार्थकर्तव्यतया देहेन्द्रियाद्याकारेण संहन्यते~। सोऽयं सङ्घातः इदं शरीरम्~। तदेतत् भगवान् उवाच —}\\ 
\begin{center}{\bfseries श्रीभगवानुवाच —\\इदं शरीरं कौन्तेय क्षेत्रमित्यभिधीयते~।\\एतद्यो वेत्ति तं प्राहुः क्षेत्रज्ञ इति तद्विदः~॥~१~॥}\end{center} 
इदम् इति सर्वनाम्ना उक्तं विशिनष्टि शरीरम् इति~। हे कौन्तेय, क्षतत्राणात्~, क्षयात्~, क्षरणात्~, क्षेत्रवद्वा अस्मिन् कर्मफलनिष्पत्तेः क्षेत्रम् इति — इतिशब्दः एवंशब्दपदार्थकः — क्षेत्रम् इत्येवम् अभिधीयते कथ्यते~। एतत् शरीरं क्षेत्रं यः वेत्ति विजानाति, आपादतलमस्तकं ज्ञानेन विषयीकरोति, स्वाभाविकेन औपदेशिकेन वा वेदनेन विषयीकरोति विभागशः, तं वेदितारं प्राहुः कथयन्ति क्षेत्रज्ञः इति — इतिशब्दः एवंशब्दपदार्थकः एव पूर्ववत् — क्षेत्रज्ञः इत्येवम् आहुः~। के~? तद्विदः तौ क्षेत्रक्षेत्रज्ञौ ये विदन्ति ते तद्विदः~॥~१~॥\par
 एवं क्षेत्रक्षेत्रज्ञौ उक्तौ~। किम् एतावन्मात्रेण ज्ञानेन ज्ञातव्यौ इति~? न इति उच्यते —} 
\begin{center}{\bfseries क्षेत्रज्ञं चापि मां विद्धि सर्वक्षेत्रेषु भारत~।\\क्षेत्रक्षेत्रज्ञयोर्ज्ञानं यत्तज्ज्ञानं मतं मम~॥~२~॥}\end{center} 
क्षेत्रज्ञं यथोक्तलक्षणं चापि मां परमेश्वरम् असंसारिणं विद्धि जानीहि~। सर्वक्षेत्रेषु यः क्षेत्रज्ञः ब्रह्मादिस्तम्बपर्यन्तानेकक्षेत्रोपाधिप्रविभक्तः, तं निरस्तसर्वोपाधिभेदं सदसदादिशब्दप्रत्ययागोचरं विद्धि इति अभिप्रायः~। हे भारत, यस्मात् क्षेत्रक्षेत्रज्ञेश्वरयाथात्म्यव्यतिरेकेण न ज्ञानगोचरम् अन्यत् अवशिष्टम् अस्ति, तस्मात् क्षेत्रक्षेत्रज्ञयोः ज्ञेयभूतयोः यत् ज्ञानं क्षेत्रक्षेत्रज्ञौ येन ज्ञानेन विषयीक्रियेते, तत् ज्ञानं सम्यग्ज्ञानम् इति मतम् अभिप्रायः मम ईश्वरस्य विष्णोः~॥~} 
ननु सर्वक्षेत्रेषु एक एव ईश्वरः, न अन्यः तद्व्यतिरिक्तः भोक्ता विद्यते चेत्~, ततः ईस्वरस्य संसारित्वं प्राप्तम्~; ईश्वरव्यतिरेकेण वा संसारिणः अन्यस्य अभावात् संसाराभावप्रसङ्गः~। तच्च उभयमनिष्टम्~, बन्धमोक्षतद्धेतुशास्त्रानर्थक्यप्रसङ्गात्~, प्रत्यक्षादिप्रमाणविरोधाच्च~। प्रत्यक्षेण तावत् सुखदुःखतद्धेतुलक्षणः संसारः उपलभ्यते~; जगद्वैचित्र्योपलब्धेश्च धर्माधर्मनिमित्तः संसारः अनुमीयते~। सर्वमेतत् अनुपपन्नमात्मेश्वरैकत्वे~॥~} 
न~; ज्ञानाज्ञानयोः अन्यत्वेनोपपत्तेः — ‘दूरमेते विपरीते विषूची अविद्या या च विद्येति ज्ञाता’\footnote{क. उ. १~। २~। ४}~। तथा तयोः विद्याविद्याविषययोः फलभेदोऽपि विरुद्धः निर्दिष्टः — ‘श्रेयश्च प्रेयश्च’\footnote{क. उ. १~। २~। २} इति~; विद्याविषयः श्रेयः, प्रेयस्तु अविद्याकार्यम् इति~। तथा च व्यासः — ‘द्वाविमावथ पन्थानौ’\footnote{मो. ध. २४१~। ६} इत्यादि, ‘इमौ द्वावेव पन्थानौ’ इत्यादि च~। इह च द्वे निष्ठे उक्ते~। अविद्या च सह कार्येण हातव्या इति श्रुतिस्मृतिन्यायेभ्यः अवगम्यते~। श्रुतयः तावत् — ‘इह चेदवेदीदथ सत्यमस्ति न चेदिहावेदीन्महती विनष्टिः’\footnote{के. उ. २~। ५} ‘तमेवं विद्वानमृत इह भवति~। नान्यः पन्था विद्यतेऽयनाय’\footnote{तै. आ. ३~। १३} ‘विद्वान्न बिभेति कुतश्चन’\footnote{तै. उ. २~। ९~। १}~। अविदुषस्तु — ‘अथ तस्य भयं भवति’\footnote{तै. उ. २~। ७~। १}, ‘अविद्यायामन्तरे वर्तमानाः’\footnote{क. उ. १~। २~। ५}, ‘ब्रह्म वेद ब्रह्मैव भवति’ ‘अन्योऽसावन्योऽहमस्मीति न स वेद यथा पशुरेवं स देवानाम्’\footnote{बृ. उ. १~। ४~। १०} आत्मवित् यः ‘स इदं सर्वं भवति’\footnote{बृ. उ. १~। ४~। १०}~; ‘यदा चर्मवत्’\footnote{श्वे. उ. ६~। २०} इत्याद्याः सहस्रशः~। स्मृतयश्च — ‘अज्ञानेनावृतं ज्ञानं तेन मुह्यन्ति जन्तवः’\footnote{भ. गी. ५~। १५} ‘इहैव तैर्जितः सर्गो येषां साम्ये स्थितं मनः’\footnote{भ. गी. ५~। १९} ‘समं पश्यन् हि सर्वत्र’\footnote{भ. गी. १३~। २८} इत्याद्याः~। न्यायतश्च — ‘सर्पान्कुशाग्राणि तथोदपानं ज्ञात्वा मनुष्याः परिवर्जयन्ति~। अज्ञानतस्तत्र पतन्ति केचिज्ज्ञाने फलं पश्य यथाविशिष्टम्’\footnote{मो. ध. २०१~। १७}~। तथा च — देहादिषु आत्मबुद्धिः अविद्वान् रागद्वेषादिप्रयुक्तः धर्माधर्मानुष्ठानकृत् जायते म्रियते च इति अवगम्यते~; देहादिव्यतिरिक्तात्मदर्शिनः रागद्वेषादिप्रहाणापेक्षधर्माधर्मप्रवृत्त्युपशमात् मुच्यन्ते इति न केनचित् प्रत्याख्यातुं शक्यं न्यायतः~। तत्र एवं सति, क्षेत्रज्ञस्य ईश्वरस्यैव सतः अविद्याकृतोपाधिभेदतः संसारित्वमिव भवति, यथा देहाद्यात्मत्वमात्मनः~। सर्वजन्तूनां हि प्रसिद्धः देहादिषु अनात्मसु आत्मभावः निश्चितः अविद्याकृतः, यथा स्थाणौ पुरुषनिश्चयः~; न च एतावता पुरुषधर्मः स्थाणोः भवति, स्थाणुधर्मो वा पुरुषस्य, तथा न चैतन्यधर्मो देहस्य, देहधर्मो वा चेतनस्य सुखदुःखमोहात्मकत्वादिः आत्मनः न युक्तः~; अविद्याकृतत्वाविशेषात्~, जरामृत्युवत्~॥~} 
न, अतुल्यत्वात्~; इति चेत् — स्थाणुपुरुषौ ज्ञेयावेव सन्तौ ज्ञात्रा अन्योन्यस्मिन् अध्यस्तौ अविद्यया~; देहात्मनोस्तु ज्ञेयज्ञात्रोरेव इतरेतराध्यासः, इति न समः दृष्टान्तः~। अतः देहधर्मः ज्ञेयोऽपि ज्ञातुरात्मनः भवतीति चेत्~, न~; अचैतन्यादिप्रसङ्गात्~। यदि हि ज्ञेयस्य देहादेः क्षेत्रस्य धर्माः सुखदुःखमोहेच्छादयः ज्ञातुः भवन्ति, तर्हि, ‘ज्ञेयस्य क्षेत्रस्य धर्माः केचित् आत्मनः भवन्ति अविद्याध्यारोपिताः, जरामरणादयस्तु न भवन्ति’ इति विशेषहेतुः वक्तव्यः~। ‘न भवन्ति’ इति अस्ति अनुमानम् — अविद्याध्यारोपितत्वात् जरामरणादिवत् इति, हेयत्वात्~, उपादेयत्वाच्च इत्यादि~। तत्र एवं सति, कर्तृत्वभोक्तृत्वलक्षणः संसारः ज्ञेयस्थः ज्ञातरि अविद्यया अध्यारोपितः इति, न तेन ज्ञातुः किञ्चित् दुष्यति, यथा बालैः अध्यारोपितेन आकाशस्य तलमलिनत्वादिना~॥~} 
एवं च सति, सर्वक्षेत्रेष्वपि सतः भगवतः क्षेत्रज्ञस्य ईश्वरस्य संसारित्वगन्धमात्रमपि नाशङ्क्यम्~। न हि क्वचिदपि लोके अविद्याध्यस्तेन धर्मेण कस्यचित् उपकारः अपकारो वा दृष्टः~॥~} 
यत्तु उक्तम् — न समः दृष्टान्तः इति, तत् असत्~। कथम्~? अविद्याध्यासमात्रं हि दृष्टान्तदार्ष्टान्तिकयोः साधर्म्यं विवक्षितम्~। तत् न व्यभिचरति~। यत्तु ज्ञातरि व्यभिचरति इति मन्यसे, तस्यापि अनैकान्तिकत्वं दर्शितं जरादिभिः~॥~} 
अविद्यावत्त्वात् क्षेत्रज्ञस्य संसारित्वम् इति चेत्~, न~; अविद्यायाः तामसत्वात्~। तामसो हि प्रत्ययः, आवरणात्मकत्वात् अविद्या विपरीतग्राहकः, संशयोपस्थापको वा, अग्रहणात्मको वा~; विवेकप्रकाशभावे तदभावात्~, तामसे च आवरणात्मके तिमिरादिदोषे सति अग्रहणादेः अविद्यात्रयस्य उपलब्धेः~॥~} 
अत्र आह — एवं तर्हि ज्ञातृधर्मः अविद्या~। न~; करणे चक्षुषि तैमिरिकत्वादिदोषोपलब्धेः~। यत्तु मन्यसे — ज्ञातृधर्मः अविद्या, तदेव च अविद्याधर्मवत्त्वं क्षेत्रज्ञस्य संसारित्वम्~; तत्र यदुक्तम् ‘ईश्वर एव क्षेत्रज्ञः, न संसारी’ इत्येतत् अयुक्तमिति — तत् न~; यथा करणे चक्षुषि विपरीतग्राहकादिदोषस्य दर्शनात्~। न विपरीतादिग्रहणं तन्निमित्तं वा तैमिरिकत्वादिदोषः ग्रहीतुः, चक्षुषः संस्कारेण तिमिरे अपनीते ग्रहीतुः अदर्शनात् न ग्रहीतुर्धर्मः यथा~; तथा सर्वत्रैव अग्रहणविपरीतसंशयप्रत्ययास्तन्निमित्ताः करणस्यैव कस्यचित् भवितुमर्हन्ति, न ज्ञातुः क्षेत्रज्ञस्य~। संवेद्यत्वाच्च तेषां प्रदीपप्रकाशवत् न ज्ञातृधर्मत्वम् — संवेद्यत्वादेव स्वात्मव्यतिरिक्तसंवेद्यत्वम्~; सर्वकरणवियोगे च कैवल्ये सर्ववादिभिः अविद्यादिदोषवत्त्वानभ्युपगमात्~। आत्मनः यदि क्षेत्रज्ञस्य अग्न्युष्णवत् स्वः धर्मः, ततः न कदाचिदपि तेन वियोगः स्यात्~। अविक्रियस्य च व्योमवत् सर्वगतस्य अमूर्तस्य आत्मनः केनचित् संयोगवियोगानुपपत्तेः, सिद्धं क्षेत्रज्ञस्य नित्यमेव ईश्वरत्वम्~; ‘अनादित्वान्निर्गुणत्वात्’\footnote{भ. गी. १३~। ३१} इत्यादीश्वरवचनाच्च~॥~} 
ननु एवं सति संसारसंसारित्वाभावे शास्त्रानर्थक्यादिदोषः स्यादिति चेत्~, न~; सर्वैरभ्युपगतत्वात्~। सर्वैः आत्मवादिभिः अभ्युपगतः दोषः न एकेन परिहर्तव्यः भवति~। कथम् अभ्युपगतः इति~? मुक्तात्मनां हि संसारसंसारित्वव्यवहाराभावः सर्वैरेव आत्मवादिभिः इष्यते~। न च तेषां शास्त्रानर्थक्यादिदोषप्राप्तिः अभ्युपगता~। तथा नः क्षेत्रज्ञानाम् ईश्वरैकत्वे सति, शास्त्रानर्थक्यं भवतु~; अविद्याविषये च अर्थवत्त्वम् — यथा द्वैतिनां सर्वेषां बन्धावस्थायामेव शास्त्राद्यर्थवत्त्वम्~, न मुक्तावस्थायाम्~, एवम्~॥~} 
ननु आत्मनः बन्धमुक्तावस्थे परमार्थत एव वस्तुभूते द्वैतिनां सर्वेषाम्~। अतः हेयोपादेयतत्साधनसद्भावे शास्त्राद्यर्थवत्त्वं स्यात्~। अद्वैतिनां पुनः, द्वैतस्य अपरमार्थत्वात्~, अविद्याकृतत्वात् बन्धावस्थायाश्च आत्मनः अपरमार्थत्वे निर्विषयत्वात्~, शास्त्राद्यानर्थक्यम् इति चेत्~, न~; आत्मनः अवस्थाभेदानुपपत्तेः~। यदि तावत् आत्मनः बन्धमुक्तावस्थे, युगपत् स्याताम्~, क्रमेण वा~। युगपत् तावत् विरोधात् न सम्भवतः स्थितिगती इव एकस्मिन्~। क्रमभावित्वे च, निर्निमित्तत्वे अनिर्मोक्षप्रसङ्गः~। अन्यनिमित्तत्वे च स्वतः अभावात् अपरमार्थत्वप्रसङ्गः~। तथा च सति अभ्युपगमहानिः~। किञ्च, बन्धमुक्तावस्थयोः पौर्वापर्यनिरूपणायां बन्धावस्था पूर्वं प्रकल्प्या, अनादिमती अन्तवती च~; तच्च प्रमाणविरुद्धम्~। तथा मोक्षावस्था आदिमती अनन्ता च प्रमाणविरुद्धैव अभ्युपगम्यते~। न च अवस्थावतः अवस्थान्तरं गच्छतः नित्यत्वम् उपपादयितुं शक्यम्~। अथ अनित्यत्वदोषपरिहाराय बन्धमुक्तावस्थाभेदो न कल्प्यते, अतः द्वैतिनामपि शास्त्रानर्थक्यादिदोषः अपरिहार्य एव~; इति समानत्वात् न अद्वैतवादिना परिहर्तव्यः दोषः~॥~} 
न च शास्त्रानर्थक्यम्~, यथाप्रसिद्धाविद्वत्पुरुषविषयत्वात् शास्त्रस्य~। अविदुषां हि फलहेत्वोः अनात्मनोः आत्मदर्शनम्~, न विदुषाम्~; विदुषां हि फलहेतुभ्याम् आत्मनः अन्यत्वदर्शने सति, तयोः अहमिति आत्मदर्शनानुपपत्तेः~। न हि अत्यन्तमूढः उन्मत्तादिरपि जलाग्न्योः छायाप्रकाशयोर्वा ऐकात्म्यं पश्यति~; किमुत विवेकी~। तस्मात् न विधिप्रतिषेधशास्त्रं तावत् फलहेतुभ्याम् आत्मनः अन्यत्वदर्शिनः भवति~। न हि ‘देवदत्त, त्वम् इदं कुरु’ इति कस्मिंश्चित् कर्मणि नियुक्ते, विष्णुमित्रः ‘अहं नियुक्तः’ इति तत्रस्थः नियोगं शृण्वन्नपि प्रतिपद्यते~। वियोगविषयविवेकाग्रहणात् तु उपपद्यते प्रतिपत्तिः~; तथा फलहेत्वोरपि~॥~} 
ननु प्राकृतसम्बन्धापेक्षया युक्तैव प्रतिपत्तिः शास्त्रार्थविषया — फलहेतुभ्याम् अन्यात्मविषयदर्शनेऽपि सति — इष्टफलहेतौ प्रवर्तितः अस्मि, अनिष्टफलहेतोश्च निवर्तितः अस्मीति~; यथा पितृपुत्रादीनाम् इतरेतरात्मान्यत्वदर्शने सत्यपि अन्योन्यनियोगप्रतिषेधार्थप्रतिपत्तिः~। न~; व्यतिरिक्तात्मदर्शनप्रतिपत्तेः प्रागेव फलहेत्वोः आत्माभिमानस्य सिद्धत्वात्~। प्रतिपन्ननियोगप्रतिषेधार्थो हि फलहेतुभ्याम् आत्मनः अन्यत्वं प्रतिपद्यते, न पूर्वम्~। तस्मात् विधिप्रतिषेधशास्त्रम् अविद्वद्विषयम् इति सिद्धम्~॥~} 
ननु ‘स्वर्गकामो यजेत’\footnote{~? } ‘न कलञ्जं भक्षयेत्’\footnote{~? } इत्यादौ आत्मव्यतिरेकदर्शिनाम् अप्रवृत्तौ, केवलदेहाद्यात्मदृष्टीनां च~; अतः कर्तुः अभावात् शास्त्रानर्थक्यमिति चेत्~, न~; यथाप्रसिद्धित एव प्रवृत्तिनिवृत्त्युपपत्तेः~। ईश्वरक्षेत्रज्ञैकत्वदर्शी ब्रह्मवित् तावत् न प्रवर्तते~। तथा नैरात्म्यवाद्यपि नास्ति परलोकः इति न प्रवर्तते~। यथाप्रसिद्धितस्तु विधिप्रतिषेधशास्त्रश्रवणान्यथानुपपत्त्या अनुमितात्मास्तित्वः आत्मविशेषानभिज्ञः कर्मफलसञ्जाततृष्णः श्रद्दधानतया च प्रवर्तते~। इति सर्वेषां नः प्रत्यक्षम्~। अतः न शास्त्रानर्थक्यम्~॥~} 
विवेकिनाम् अप्रवृत्तिदर्शनात् तदनुगामिनाम् अप्रवृत्तौ शास्त्रानर्थक्यम् इति चेत्~, न~; } 
कस्यचिदेव विवेकोपपत्तेः~। अनेकेषु हि प्राणिषु कश्चिदेव विवेकी स्यात्~, यथेदानीम्~। न च विवेकिनम् अनुवर्तन्ते मूढाः, रागादिदोषतन्त्रत्वात् प्रवृत्तेः, अभिचरणादौ च प्रवृत्तिदर्शनात्~, स्वाभाव्याच्च प्रवृत्तेः — ‘स्वभावस्तु प्रवर्तते’\footnote{भ. गी. ५~। १४} इति हि उक्तम्~॥~} 
तस्मात् अविद्यामात्रं संसारः यथादृष्टविषयः एव~। न क्षेत्रज्ञस्य केवलस्य अविद्या तत्कार्यं च~। न च मिथ्याज्ञानं परमार्थवस्तु दूषयितुं समर्थम्~। न हि ऊषरदेशं स्नेहेन पङ्कीकर्तुं शक्नोति मरीच्युदकम्~। तथा अविद्या क्षेत्रज्ञस्य न किञ्चित् कर्तुं शक्नोति~। अतश्चेदमुक्तम् — ‘क्षेत्रज्ञं चापि मां विद्धि’\footnote{भ. गी. १३~। २}, ‘अज्ञानेनावृतं ज्ञानम्’\footnote{भ. गी. ५~। १५} इति च~॥~} 
अथ किमिदं संसारिणामिव ‘अहमेवं’ ‘ममैवेदम्’ इति पण्डितानामपि~? शृणु~; इदं तत् पाण्डित्यम्~, यत् क्षेत्रे एव आत्मदर्शनम्~। यदि पुनः क्षेत्रज्ञम् अविक्रियं पश्येयुः, ततः न भोगं कर्म वा आकाङ्क्षेयुः ‘मम स्यात्’ इति~। विक्रियैव भोगकर्मणी~। अथ एवं सति, फलार्थित्वात् अविद्वान् प्रवर्तते~। विदुषः पुनः अविक्रियात्मदर्शिनः फलार्थित्वाभावात् प्रवृत्त्यनुपपत्तौ कार्यकरणसङ्घातव्यापारोपरमे निवृत्तिः उपचर्यते~॥~} 
इदं च अन्यत् पाण्डित्यं केषाञ्चित् अस्तु — क्षेत्रज्ञः ईश्वर एव~। क्षेत्रं च अन्यत् क्षेत्रज्ञस्यैव विषयः~। अहं तु संसारी सुखी दुःखी च~। संसारोपरमश्च मम कर्तव्यः क्षेत्रक्षेत्रज्ञविज्ञानेन, ध्यानेन च ईश्वरं क्षेत्रज्ञं साक्षात्कृत्वा तत्स्वरूपावस्थानेनेति~। यश्च एवं बुध्यते, यश्च बोधयति, नासौ क्षेत्रज्ञः इति~। एवं मन्वानः यः सः पण्डितापशदः, संसारमोक्षयोः शास्त्रस्य च अर्थवत्त्वं करोमीति~; आत्महा स्वयं मूढः अन्यांश्च व्यामोहयति शास्त्रार्थसम्प्रदायरहितत्वात्~, श्रुतहानिम् अश्रुतकल्पनां च कुर्वन्~। तस्मात् असम्प्रदायवित् सर्वशास्त्रविदपि मूर्खवदेव उपेक्षणीयः~॥~} 
यत्तूक्तम् ‘ईश्वरस्य क्षेत्रज्ञैकत्वे संसारित्वं प्राप्नोति, क्षेत्रज्ञानां च ईश्वरैकत्वे संसारिणः अभावात् संसाराभावप्रसङ्गः’ इति, एतौ दोषौ प्रत्युक्तौ ‘विद्याविद्ययोः वैलक्षण्याभ्युपगमात्’ इति~। कथम्~? अविद्यापरिकल्पितदोषेण तद्विषयं वस्तु पारमार्थिकं न दुष्यतीति~। तथा च दृष्टान्तः दर्शितः — मरीच्यम्भसा ऊषरदेशो न पङ्कीक्रियते इति~। संसारिणः अभावात् संसाराभावप्रसङ्गदोषोऽपि संसारसंसारिणोः अविद्याकल्पितत्वोपपत्त्या प्रत्युक्तः~॥~} 
ननु अविद्यावत्त्वमेव क्षेत्रज्ञस्य संसारित्वदोषः~। तत्कृतं च सुखित्वदुःखित्वादि प्रत्यक्षम् उपलभ्यते इति चेत्~, न~; ज्ञेयस्य क्षेत्रधर्मत्वात्~, ज्ञातुः क्षेत्रज्ञस्य तत्कृतदोषानुपपत्तेः~। यावत् किञ्चित् क्षेत्रज्ञस्य दोषजातम् अविद्यमानम् आसञ्जयसि, तस्य ज्ञेयत्वोपपत्तेः क्षेत्रधर्मत्वमेव, न क्षेत्रज्ञधर्मत्वम्~। न च तेन क्षेत्रज्ञः दुष्यति, ज्ञेयेन ज्ञातुः संसर्गानुपपत्तेः~। यदि हि संसर्गः स्यात्~, ज्ञेयत्वमेव नोपपद्येत~। यदि आत्मनः धर्मः अविद्यावत्त्वं दुःखित्वादि च कथं भोः प्रत्यक्षम् उपलभ्यते, कथं वा क्षेत्रज्ञधर्मः~। ‘ज्ञेयं च सर्वं क्षेत्रं ज्ञातैव क्षेत्रज्ञः’ इति अवधारिते, ‘अविद्यादुःखित्वादेः क्षेत्रज्ञविशेषणत्वं क्षेत्रज्ञ धर्मत्वं तस्य च प्रत्यक्षोपलभ्यत्वम्’ इति विरुद्धम् उच्यते अविद्यामात्रावष्टम्भात् केवलम्~॥~} 
अत्र आह — सा अविद्या कस्य इति~। यस्य दृश्यते तस्य एव~। कस्य दृश्यते इति~। अत्र उच्यते — ‘अविद्या कस्य दृश्यते~? ’ इति प्रश्नः निरर्थकः~। कथम्~? दृश्यते चेत् अविद्या, तद्वन्तमपि पश्यसि~। न च तद्वति उपलभ्यमाने ‘सा कस्य~? ’ इति प्रश्नो युक्तः~। न हि गोमति उपलभ्यमाने ‘गावः कस्य~? ’ इति प्रश्नः अर्थवान् भवति~। ननु विषमो दृष्टान्तः~। गवां तद्वतश्च प्रत्यक्षत्वात् तत्सम्बन्धोऽपि प्रत्यक्ष इति प्रश्नो निरर्थकः~। न तथा अविद्या तद्वांश्च प्रत्यक्षौ, यतः प्रश्नः निरर्थकः स्यात्~। अप्रत्यक्षेण अविद्यावता अविद्यासम्बन्धे ज्ञाते, किं तव स्यात्~? अविद्यायाः अनर्थहेतुत्वात् परिहर्तव्या स्यात्~। यस्य अविद्या, सः तां परिहरिष्यति~। ननु ममैव अविद्या~। जानासि तर्हि अविद्यां तद्वन्तं च आत्मानम्~। जानामि, न तु प्रत्यक्षेण~। अनुमानेन चेत् जानासि, कथं सम्बन्धग्रहणम्~? न हि तव ज्ञातुः ज्ञेयभूतया अविद्यया तत्काले सम्बन्धः ग्रहीतुं शक्यते, अविद्याया विषयत्वेनैव ज्ञातुः उपयुक्तत्वात्~। न च ज्ञातुः अविद्यायाश्च सम्बन्धस्य यः ग्रहीता, ज्ञानं च अन्यत् तद्विषयं सम्भवति~; अनवस्थाप्राप्तेः~। यदि ज्ञात्रापि ज्ञेयसम्बन्धो ज्ञायते, अन्यः ज्ञाता कल्प्यः स्यात्~, तस्यापि अन्यः, तस्यापि अन्यः इति अनवस्था अपरिहार्या~। यदि पुनः अविद्या ज्ञेया, अन्यद्वा ज्ञेयं ज्ञेयमेव~। तथा ज्ञातापि ज्ञातैव, न ज्ञेयं भवति~। यदा च एवम्~, अविद्यादुःखित्वाद्यैः न ज्ञातुः क्षेत्रज्ञस्य किञ्चित् दुष्यति~॥~} 
ननु अयमेव दोषः, यत् दोषवत्क्षेत्रविज्ञातृत्वम्~; न च विज्ञानस्वरूपस्यैव अविक्रियस्य विज्ञातृत्वोपचारात्~; यथा उष्णतामात्रेण अग्नेः तप्तिक्रियोपचारः तद्वत्~। यथा अत्र भगवता क्रियाकारकफलात्मत्वाभावः आत्मनि स्वत एव दर्शितः — अविद्याध्यारोपितः एव क्रियाकारकादिः आत्मनि उपचर्यते~; तथा तत्र तत्र ‘य एवं वेत्ति हन्तारम्’\footnote{भ. गी. २~। १९}, ‘प्रकृतेः क्रियमाणानि गुणैः कर्माणि सर्वशः’\footnote{भ. गी. ३~। २७}, ‘नादत्ते कस्यचित्पापम्’\footnote{भ. गी. ५~। १५} इत्यादिप्रकरणेषु दर्शितः~। तथैव च व्याख्यातम् अस्माभिः~। उत्तरेषु च प्रकरणेषु दर्शयिष्यामः~॥~} 
हन्त~। तर्हि आत्मनि क्रियाकारकफलात्मतायाः स्वतः अभावे, अविद्यया च अध्यारोपितत्वे, कर्माणि अविद्वत्कर्तव्यान्येव, न विदुषाम् इति प्राप्तम्~। सत्यम् एवं प्राप्तम्~, एतदेव च ‘न हि देहभृता शक्यम्’\footnote{भ. गी. १८~। ११} इत्यत्र दर्शयिष्यामः~। सर्वशास्त्रार्थोपसंहारप्रकरणे च ‘समासेनैव कौन्तेय निष्ठा ज्ञानस्य या परा’\footnote{भ. गी. १८~। ५०} इत्यत्र विशेषतः दर्शयिष्यामः~। अलम् इह बहुप्रपञ्चनेन, इति उपसंह्रियते~॥~२~॥\par
 ‘इदं शरीरम्’ इत्यादिश्लोकोपदिष्टस्य क्षेत्राध्यायार्थस्य सङ्ग्रहश्लोकः अयम् उपन्यस्यते ‘तत्क्षेत्रं यच्च’ इत्यादि, व्याचिख्यासितस्य हि अर्थस्य सङ्ग्रहोपन्यासः न्याय्यः इति — } 
\begin{center}{\bfseries तत्क्षेत्रं यच्च यादृक्च यद्विकारि यतश्च यत्~।\\स च यो यत्प्रभावश्च तत्समासेन मे शृणु~॥~३~॥}\end{center} 
यत् निर्दिष्टम् ‘इदं शरीरम्’ इति तत् तच्छब्देन परामृशति~। यच्च इदं निर्दिष्टं क्षेत्रं तत् यादृक् यादृशं स्वकीयैः धर्मैः~। च - शब्दः समुच्चयार्थः~। यद्विकारि यः विकारः यस्य तत् यद्विकारि, यतः यस्मात् च यत्~, कार्यम् उत्पद्यते इति वाक्यशेषः~। स च यः क्षेत्रज्ञः निर्दिष्टः सः यत्प्रभावः ये प्रभावाः उपाधिकृताः शक्तयः यस्य सः यत्प्रभावश्च~। तत् क्षेत्रक्षेत्रज्ञयोः याथात्म्यं यथाविशेषितं समासेन सङ्क्षेपेण मे मम वाक्यतः शृणु, श्रुत्वा अवधारय इत्यर्थः~॥~३~॥\par
 तत् क्षेत्रक्षेत्रज्ञयाथात्म्यं विवक्षितं स्तौति श्रोतृबुद्धिप्ररोचनार्थम् —} 
\begin{center}{\bfseries ऋषिभिर्बहुधा गीतं छन्दोभिर्विविधैः पृथक्~।\\ब्रह्मसूत्रपदैश्चैव हेतुमद्भिर्विनिश्चितैः~॥~४~॥}\end{center} 
ऋषिभिः वसिष्ठादिभिः बहुधा बहुप्रकारं गीतं कथितम्~। छन्दोभिः छन्दांसि ऋगादीनि तैः छन्दोभिः विविधैः नानाभावैः नानाप्रकारैः पृथक् विवेकतः गीतम्~। किञ्च, ब्रह्मसूत्रपदैश्च एव ब्रह्मणः सूचकानि वाक्यानि ब्रह्मसूत्राणि तैः पद्यते गम्यते ज्ञायते इति तानि पदानि उच्यन्ते तैरेव च क्षेत्रक्षेत्रज्ञयाथात्म्यम् ‘गीतम्’ इति अनुवर्तते~। ‘आत्मेत्येवोपासीत’\footnote{बृ. उ. १~। ४~। ७} इत्येवमादिभिः ब्रह्मसूत्रपदैः आत्मा ज्ञायते, हेतुमद्भिः युक्तियुक्तैः विनिश्चितैः निःसंशयरूपैः निश्चितप्रत्ययोत्पादकैः इत्यर्थः~॥~४~॥\par
 स्तुत्या अभिमुखीभूताय अर्जुनाय आह भगवान् —} 
\begin{center}{\bfseries महाभूतान्यहङ्कारो बुद्धिरव्यक्तमेव च~।\\इन्द्रियाणि दशैकं च पञ्च चेन्द्रियगोचराः~॥~५~॥}\end{center} 
महाभूतानि महान्ति च तानि सर्वविकारव्यापकत्वात् भूतानि च सूक्ष्माणि~। स्थूलानि तु इन्द्रियगोचरशब्देन अभिधायिष्यन्ते अहङ्कारः महाभूतकारणम् अहंप्रत्ययलक्षणः~। अहङ्कारकारणं बुद्धिः अध्यवसायलक्षणा~। तत्कारणम् अव्यक्तमेव च, न व्यक्तम् अव्यक्तम् अव्याकृतम् ईश्वरशक्तिः ‘मम माया दुरत्यया’\footnote{भ. गी. ७~। १४} इत्युक्तम्~। एवशब्दः प्रकृत्यवधारणार्थः एतावत्येव अष्टधा भिन्ना प्रकृतिः~। च - शब्दः भेदसमुच्चयार्थः~। इन्द्रियाणि दश, श्रोत्रादीनि पञ्च बुद्ध्युत्पादकत्वात् बुद्धीन्द्रियाणि, वाक्पाण्यादीनि पञ्च कर्मनिवर्तकत्वात् कर्मेन्द्रियाणि~; तानि दश~। एकं च~; किं तत्~? मनः एकादशं सङ्कल्पाद्यात्मकम्~। पञ्च च इन्द्रियगोचराः शब्दादयो विषयाः~। तानि एतानि साङ्ख्याः चतुर्विंशतितत्त्वानि आचक्षते~॥~५~॥\par
 	अथ इदानीम् आत्मगुणा इति यानाचक्षते वैशेषिकाः तेपि क्षेत्रधर्मा एव न तु क्षेत्रज्ञस्य इत्याह भगवान् - } 
    \begin{center}{\bfseries इच्छा द्वेषः सुखं दुःखं सङ्घातश्चेतना धृतिः~।\\एतत्क्षेत्रं समासेन सविकारमुदाहृतम्~॥~६~॥}\end{center} 
इच्छा, यज्जातीयं सुखहेतुमर्थम् उपलब्धवान् पूर्वम्~, पुनः तज्जातीयमुपलभमानः तमादातुमिच्छति सुखहेतुरिति~; सा इयं इच्छा अन्तःकरणधर्मः ज्ञेयत्वात् क्षेत्रम्~। तथा द्वेषः, यज्जातीयमर्थं दुःखहेतुत्वेन अनुभूतवान्~, पुनः तज्जातीयमर्थमुपलभमानः तं द्वेष्टि~; सोऽयं द्वेषः ज्ञेयत्वात् क्षेत्रमेव~। तथा सुखम् अनुकूलं प्रसन्नसत्त्वात्मकं ज्ञेयत्वात् क्षेत्रमेव~। दुःखं प्रतिकूलात्मकम्~; ज्ञेयत्वात् तदपि क्षेत्रम्~। सङ्घातः देहेन्द्रियाणां संहतिः~। तस्यामभिव्यक्तान्तःकरणवृत्तिः, तप्त इव लोहपिण्डे अग्निः आत्मचैतन्याभासरसविद्धा चेतना~; सा च क्षेत्रं ज्ञेयत्वात्~। धृतिः यया अवसादप्राप्तानि देहेन्द्रियाणि ध्रियन्ते~; सा च ज्ञेयत्वात् क्षेत्रम्~। सर्वान्तःकरणधर्मोपलक्षणार्थम् इच्छादिग्रहणम्~। यत उक्तमुपसंहरति — एतत् क्षेत्रं समासेन सविकारं सह विकारेण महदादिना उदाहृतम् उक्तम् यस्य क्षेत्रभेदजातस्य संहतिः ‘इदं शरीरं क्षेत्रम्’\footnote{भ. गी. १३~। १} इति उक्तम्~, तत् क्षेत्रं व्याख्यातं महाभूतादिभेदभिन्नं धृत्यन्तम्~।~॥~६~॥\par
  क्षेत्रज्ञः वक्ष्यमाणविशेषणः — यस्य सप्रभावस्य क्षेत्रज्ञस्य परिज्ञानात् अमृतत्वं भवति, तम् ‘ज्ञेयं यत्तत्प्रवक्ष्यामि’\footnote{भ. गी. १३~। १२} इत्यादिना सविशेषणं स्वयमेव वक्ष्यति भगवान्~। अधुना तु तज्ज्ञानसाधनगणममानित्वादिलक्षणम्~, यस्मिन् सति तज्ज्ञेयविज्ञाने योग्यः अधिकृतः भवति, यत्परः संन्यासी ज्ञाननिष्ठः उच्यते, तम् अमानित्वादिगणं ज्ञानसाधनत्वात् ज्ञानशब्दवाच्यं विदधाति भगवान् — } 
\begin{center}{\bfseries अमानित्वमदम्भित्व—\\ महिंसा क्षान्तिरार्जवम्~।\\आचार्योपासनं शौचं\\ स्थैर्यमात्मविनिग्रहः~॥~७~॥}\end{center} 
अमानित्वं मानिनः भावः मानित्वमात्मनः श्लाघनम्~, तदभावः अमानित्वम्~। अदम्भित्वं स्वधर्मप्रकटीकरणं दम्भित्वम्~, तदभावः अदम्भित्वम्~। अहिंसा अहिंसनं प्राणिनामपीडनम्~। क्षान्तिः परापराधप्राप्तौ अविक्रिया~। आर्जवम् ऋजुभावः अवक्रत्वम्~। आचार्योपासनं मोक्षसाधनोपदेष्टुः आचार्यस्य शुश्रूषादिप्रयोगेण सेवनम्~। शौचं कायमलानां मृज्जलाभ्यां प्रक्षालनम्~; अन्तश्च मनसः प्रतिपक्षभावनया रागादिमलानामपनयनं शौचम्~। स्थैर्यं स्थिरभावः, मोक्षमार्गे एव कृताध्यवसायत्वम्~। आत्मविनिग्रहः आत्मनः अपकारकस्य आत्मशब्दवाच्यस्य कार्यकरणसङ्घातस्य विनिग्रहः स्वभावेन सर्वतः प्रवृत्तस्य सन्मार्गे एव निरोधः आत्मविनिग्रहः~॥~७~॥\par
 किञ्च —} 
\begin{center}{\bfseries इन्द्रियार्थेषु वैराग्यमनहङ्कार एव च~।\\जन्ममृत्युजराव्याधिदुःखदोषानुदर्शनम्~॥~८~॥}\end{center} 
इन्द्रियार्थेषु शब्दादिषु दृष्टादृष्टेषु भोगेषु विरागभावो वैराग्यम् अनहङ्कारः अहङ्काराभावः एव च जन्ममृत्युजराव्याधिदुःखदोषानुदर्शनं जन्म च मृत्युश्च जरा च व्याधयश्च दुःखानि च तेषु जन्मादिदुःखान्तेषु प्रत्येकं दोषानुदर्शनम्~। जन्मनि गर्भवासयोनिद्वारनिःसरणं दोषः, तस्य अनुदर्शनमालोचनम्~। तथा मृत्यौ दोषानुदर्शनम्~। तथा जरायां प्रज्ञाशक्तितेजोनिरोधदोषानुदर्शनं परिभूतता चेति~। तथा } 
व्याधिषु शिरोरोगादिषु दोषानुदर्शनम्~। तथा दुःखेषु अध्यात्माधिभूताधिदैवनिमित्तेषु~। अथवा दुःखान्येव दोषः दुःखदोषः तस्य जन्मादिषु पूर्ववत् अनुदर्शनम् — दुःखं जन्म, दुःखं मृत्युः, दुःखं जरा, दुःखं व्याधयः~। दुःखनिमित्तत्वात् जन्मादयः दुःखम्~, न पुनः स्वरूपेणैव दुःखमिति~। एवं जन्मादिषु दुःखदोषानुदर्शनात् देहेन्द्रियादिविषयभोगेषु वैराग्यमुपजायते~। ततः प्रत्यगात्मनि प्रवृत्तिः करणानामात्मदर्शनाय~। एवं ज्ञानहेतुत्वात् ज्ञानमुच्यते जन्मादिदुःखदोषानुदर्शनम्~॥~८~॥\par
 किञ्च —} 
\begin{center}{\bfseries असक्तिरनभिष्वङ्गः पुत्रदारगृहादिषु~।\\नित्यं च समचित्तत्वमिष्टानिष्टोपपत्तिषु~॥~९~॥}\end{center} 
असक्तिः सक्तिः सङ्गनिमित्तेषु विषयेषु प्रीतिमात्रम्~, तदभावः असक्तिः~। अनभिष्वङ्गः अभिष्वङ्गाभावः~। अभिष्वङ्गो नाम आसक्तिविशेष एव अनन्यात्मभावनालक्षणः~; यथा अन्यस्मिन् सुखिनि दुःखिनि वा ‘अहमेव सुखी, दुःखी च, ’ जीवति मृते वा ‘अहमेव जीवामि मरिष्यामि च’ इति~। क्व इति आह — पुत्रदारगृहादिषु, पुत्रेषु दारेषु गृहेषु आदिग्रहणात् अन्येष्वपि अत्यन्तेष्टेषु दासवर्गादिषु~। तच्च उभयं ज्ञानार्थत्वात् ज्ञानमुच्यते~। नित्यं च समचित्तत्वं तुल्यचित्तता~। क्व~? इष्टानिष्टोपपत्तिषु इष्टानामनिष्टानां च उपपत्तयः सम्प्राप्तयः तासु इष्टानिष्टोपपत्तिषु नित्यमेव तुल्यचित्तता~। इष्टोपपत्तिषु न हृष्यति, न कुप्यति च अनिष्टोपपत्तिषु~। तच्च एतत् नित्यं समचित्तत्वं ज्ञानम्~॥~९~॥\par
 किञ्च — } 
\begin{center}{\bfseries मयि चानन्ययोगेन भक्तिरव्यभिचारिणी~।\\विविक्तदेशसेवित्वमरतिर्जनसंसदि~॥~१०~॥}\end{center} 
मयि च ईश्वरे अनन्ययोगेन अपृथक्समाधिना ‘न अन्यो भगवतो वासुदेवात् परः अस्ति, अतः स एव नः गतिः’ इत्येवं निश्चिता अव्यभिचारिणी बुद्धिः अनन्ययोगः, तेन भजनं भक्तिः न व्यभिचरणशीला अव्यभिचारिणी~। सा च ज्ञानम्~। विविक्तदेशसेवित्वम्~, विविक्तः स्वभावतः संस्कारेण वा अशुच्यादिभिः सर्पव्याघ्रादिभिश्च रहितः अरण्यनदीपुलिनदेवगृहादिभिर्विविक्तो देशः, तं सेवितुं शीलमस्य इति विविक्तदेशसेवी, तद्भावः विविक्तदेशसेवित्वम्~। विविक्तेषु हि देशेषु चित्तं प्रसीदति यतः ततः आत्मादिभावना विविक्ते उपजायते~। अतः विविक्तदेशसेवित्वं ज्ञानमुच्यते~। अरतिः अरमणं जनसंसदि, जनानां प्राकृतानां संस्कारशून्यानाम् अविनीतानां संसत् समवायः जनसंसत्~; न संस्कारवतां विनीतानां संसत्~; तस्याः ज्ञानोपकारकत्वात्~। अतः प्राकृतजनसंसदि अरतिः ज्ञानार्थत्वात् ज्ञानम्~॥~१०~॥\par
 किञ्च — } 
\begin{center}{\bfseries अध्यात्मज्ञाननित्यत्वं तत्त्वज्ञानार्थदर्शनम्~।\\एतज्ज्ञानमिति प्रोक्तमज्ञानं यदतोऽन्यथा~॥~११~॥}\end{center} 
अध्यात्मज्ञाननित्यत्वम् आत्मादिविषयं ज्ञानम् अध्यात्मज्ञानम्~, तस्मिन् नित्यभावः नित्यत्वम्~। अमानित्वादीनां ज्ञानसाधनानां भावनापरिपाकनिमित्तं तत्त्वज्ञानम्~, तस्य अर्थः मोक्षः संसारोपरमः~; तस्य आलोचनं तत्त्वज्ञानार्थदर्शनम्~; तत्त्वज्ञानफलालोचने हि तत्साधनानुष्ठाने प्रवृत्तिः स्यादिति~। एतत् अमानित्वादितत्त्वज्ञानार्थदर्शनान्तमुक्तं ज्ञानम् इति प्रोक्तं ज्ञानार्थत्वात्~। अज्ञानं यत् अतः अस्मात् यथोक्तात् अन्यथा विपर्ययेण~। मानित्वं दम्भित्वं हिंसा अक्षान्तिः अनार्जवम् इत्यादि अज्ञानं विज्ञेयं परिहरणाय, संसारप्रवृत्तिकारणत्वात् इति~॥~११~॥\par
 यथोक्तेन ज्ञानेन ज्ञातव्यं किम् इत्याकाङ्क्षायामाह — ‘ज्ञेयं यत्तत्’ इत्यादि~। ननु यमाः नियमाश्च अमानित्वादयः~। न तैः ज्ञेयं ज्ञायते~। न हि अमानित्वादि कस्यचित् वस्तुनः परिच्छेदकं दृष्टम्~। सर्वत्रैव च यद्विषयं ज्ञानं तदेव तस्य ज्ञेयस्य परिच्छेदकं दृश्यते~। न हि अन्यविषयेण ज्ञानेन अन्यत् उपलभ्यते, यथा घटविषयेण ज्ञानेन अग्निः~। नैष दोषः, ज्ञाननिमित्तत्वात् ज्ञानमुच्यते इति हि अवोचाम~; ज्ञानसहकारिकारणत्वाच्च —} 
\begin{center}{\bfseries ज्ञेयं यत्तत्प्रवक्ष्यामि यज्ज्ञात्वामृतमश्नुते~।\\अनादिमत्परं ब्रह्म न सत्तन्नासदुच्यते~॥~१२~॥}\end{center} 
ज्ञेयं ज्ञातव्यं यत् तत् प्रवक्ष्यामि प्रकर्षेण यथावत् वक्ष्यामि~। किम्फलं तत् इति प्ररोचनेन श्रोतुः अभिमुखीकरणाय आह — यत् ज्ञेयं ज्ञात्वा अमृतम् अमृतत्वम् अश्नुते, न पुनः म्रियते इत्यर्थः~। अनादिमत् आदिः अस्य अस्तीति आदिमत्~, न आदिमत् अनादिमत्~; किं तत्~? परं निरतिशयं ब्रह्म, ‘ज्ञेयम्’ इति प्रकृतम्~॥~} 
अत्र केचित् ‘अनादि मत्परम्’ इति पदं छिन्दन्ति, बहुव्रीहिणा उक्ते अर्थे मतुपः आनर्थक्यम् अनिष्टं स्यात् इति~। अर्थविशेषं च दर्शयन्ति — अहं वासुदेवाख्या परा शक्तिः यस्य तत् मत्परम् इति~। सत्यमेवमपुनरुक्तं स्यात्~, अर्थः चेत् सम्भवति~। न तु अर्थः सम्भवति, ब्रह्मणः सर्वविशेषप्रतिषेधेनैव विजिज्ञापयिषितत्वात् ‘न सत्तन्नासदुच्यते’ इति~। विशिष्टशक्तिमत्त्वप्रदर्शनं विशेषप्रतिषेधश्च इति विप्रतिषिद्धम्~। तस्मात् मतुपः बहुव्रीहिणा समानार्थत्वेऽपि प्रयोगः श्लोकपूरणार्थः~॥~} 
अमृतत्वफलं ज्ञेयं मया उच्यते इति प्ररोचनेन अभिमुखीकृत्य आह — न सत् तत् ज्ञेयमुच्यते इति न अपि असत् तत् उच्यते~॥~} 
ननु महता परिकरबन्धेन कण्ठरवेण उद्घुष्य ‘ज्ञेयं प्रवक्ष्यामि’ इति, अननुरूपमुक्तं ‘न सत्तन्नासदुच्यते’ इति~। न, अनुरूपमेव उक्तम्~। कथम्~? सर्वासु हि उपनिषत्सु ज्ञेयं ब्रह्म ‘नेति नेति’\footnote{बृ. उ. २~। ३~। ६} ‘अस्थूलमनणु’\footnote{बृ. उ. ३~। ८~। ८} इत्यादिविशेषप्रतिषेधेनैव निर्दिश्यते, न ‘इदं तत्’ इति, वाचः अगोचरत्वात्~॥~} 
ननु न तदस्ति, यद्वस्तु अस्तिशब्देन नोच्यते~। अथ अस्तिशब्देन नोच्यते, नास्ति तत् ज्ञेयम्~। विप्रतिषिद्धं च — ‘ज्ञेयं तत्~, ’ ‘अस्तिशब्देन नोच्यते’ इति च~। न तावन्नास्ति, नास्तिबुद्ध्यविषयत्वात्~॥~} 
ननु सर्वाः बुद्धयः अस्तिनास्तिबुद्ध्यनुगताः एव~। तत्र एवं सति ज्ञेयमपि अस्तिबुद्ध्यनुगतप्रत्ययविषयं वा स्यात्~, नास्तिबुद्ध्यनुगतप्रत्ययविषयं वा स्यात्~। न, अतीन्द्रियत्वेन उभयबुद्ध्यनुगतप्रत्ययाविषयत्वात्~। यद्धि इन्द्रियगम्यं वस्तु घटादिकम्~, तत् अस्तिबुद्ध्यनुगतप्रत्ययविषयं वा स्यात्~, नास्तिबुद्ध्यनुगतप्रत्ययविषयं वा स्यात्~। इदं तु ज्ञेयम् अतीन्द्रियत्वेन शब्दैकप्रमाणगम्यत्वात् न घटादिवत् उभयबुद्ध्यनुगतप्रत्ययविषयम् इत्यतः ‘न सत्तन्नासत्’ इति उच्यते~॥~} 
यत्तु उक्तम् — विरुद्धमुच्यते, ‘ज्ञेयं तत्’ ‘न सत्तन्नासदुच्यते’ इति — न विरुद्धम्~, ‘अन्यदेव तद्विदितादथो अविदितादधि’\footnote{के. उ. १~। ४} इति श्रुतेः~। श्रुतिरपि विरुद्धार्था इति चेत् — यथा यज्ञाय शालामारभ्य ‘यद्यमुष्मिंल्लोकेऽस्ति वा न वेति’\footnote{तै. सं. ६~। १~। १~। १} इत्येवमिति चेत्~, न~; विदिताविदिताभ्यामन्यत्वश्रुतेः अवश्यविज्ञेयार्थप्रतिपादनपरत्वात् ‘यद्यमुष्मिन्’ इत्यादि तु विधिशेषः अर्थवादः~। उपपत्तेश्च सदसदादिशब्दैः ब्रह्म नोच्यते इति~। सर्वो हि शब्दः अर्थप्रकाशनाय प्रयुक्तः, श्रूयमाणश्च श्रोतृभिः, जातिक्रियागुणसम्बन्धद्वारेण सङ्केतग्रहणसव्यपेक्षः अर्थं प्रत्याययति~; न अन्यथा, अदृष्टत्वात्~। तत् यथा — ‘गौः’ ‘अश्वः’ इति वा जातितः, ‘पचति’ ‘पठति’ इति वा क्रियातः, ‘शुक्लः’ ‘कृष्णः’ इति वा गुणतः, ‘धनी’ ‘गोमान्’ इति वा सम्बन्धतः~। न तु ब्रह्म जातिमत्~, अतः न सदादिशब्दवाच्यम्~। नापि गुणवत्~, येन गुणशब्देन उच्येत, निर्गुणत्वात्~। नापि क्रियाशब्दवाच्यं निष्क्रियत्वात् ‘निष्कलं निष्क्रियं शान्तम्’\footnote{श्वे. उ. ६~। १९} इति श्रुतेः~। न च सम्बन्धी, एकत्वात्~। अद्वयत्वात् अविषयत्वात् आत्मत्वाच्च न केनचित् शब्देन उच्यते इति युक्तम्~; ‘यतो वाचो निवर्तन्ते’\footnote{तै. उ. २~। ९~। १} इत्यादिश्रुतिभिश्च~॥~१२~॥\par
 सच्छब्दप्रत्ययाविषयत्वात् असत्त्वाशङ्कायां ज्ञेयस्य सर्वप्राणिकरणोपाधिद्वारेण तदस्तित्वं प्रतिपादयन् तदाशङ्कानिवृत्त्यर्थमाह — } 
\begin{center}{\bfseries सर्वतःपाणिपादं तत्सर्वतोक्षिशिरोमुखम्~।\\सर्वतःश्रुतिमल्लोके सर्वमावृत्य तिष्ठति~॥~१३~॥}\end{center} 
सर्वतःपाणिपादं सर्वतः पाणयः पादाश्च अस्य इति सर्वतःपाणिपादं तत् ज्ञेयम्~। सर्वप्राणिकरणोपाधिभिः क्षेत्रज्ञस्य अस्तित्वं विभाव्यते~। क्षेत्रज्ञश्च क्षेत्रोपाधितः उच्यते~। क्षेत्रं च पाणिपादादिभिः अनेकधा भिन्नम्~। क्षेत्रोपाधिभेदकृतं विशेषजातं मिथ्यैव क्षेत्रज्ञस्य, इति तदपनयनेन ज्ञेयत्वमुक्तम् ‘न सत्तन्नासदुच्यते’ इति~। उपाधिकृतं मिथ्यारूपमपि अस्तित्वाधिगमाय ज्ञेयधर्मवत् परिकल्प्य उच्यते ‘सर्वतःपाणिपादम्’ इत्यादि~। तथा हि सम्प्रदायविदां वचनम् — ‘अध्यारोपापवादाभ्यां निष्प्रपञ्चं प्रपञ्च्यते’\footnote{~? } इति~। सर्वत्र सर्वदेहावयवत्वेन गम्यमानाः पाणिपादादयः ज्ञेयशक्तिसद्भावनिमित्तस्वकार्याः इति ज्ञेयसद्भावे लिङ्गानि ‘ज्ञेयस्य’ इति उपचारतः उच्यन्ते~। तथा व्याख्येयम् अन्यत्~। सर्वतःपाणिपादं तत् ज्ञेयम्~। सर्वतोक्षिशिरोमुखं सर्वतः अक्षीणि शिरांसि मुखानि च यस्य तत् सर्वतोक्षिशिरोमुखम्~; सर्वतःश्रुतिमत् श्रुतिः श्रवणेन्द्रियम्~, तत् यस्य तत् श्रुतिमत्~, लोके प्राणिनिकाये, सर्वम् आवृत्य संव्याप्य तिष्ठति स्थितिं लभते~॥~१३~॥\par
 उपाधिभूतपाणिपादादीन्द्रियाध्यारोपणात् ज्ञेयस्य तद्वत्ताशङ्का मा भूत् इत्येवमर्थः श्लोकारम्भः —} 
\begin{center}{\bfseries सर्वेन्द्रियगुणाभासं सर्वेन्द्रियविवर्जितम्~।\\असक्तं सर्वभृच्चैव निर्गुणं गुणभोक्तृ च~॥~१४~॥}\end{center} 
सर्वेन्द्रियगुणाभासं सर्वाणि च तानि इन्द्रियाणि श्रोत्रादीनि बुद्धीन्द्रियकर्मेन्द्रियाख्यानि, अन्तःकरणे च बुद्धिमनसी, ज्ञेयोपाधित्वस्य तुल्यत्वात्~, सर्वेन्द्रियग्रहणेन गृह्यन्ते~। अपि च, अन्तःकरणोपाधिद्वारेणैव श्रोत्रादीनामपि उपाधित्वम् इत्यतः अन्तःकरणबहिष्करणोपाधिभूतैः सर्वेन्द्रियगुणैः अध्यवसायसङ्कल्पश्रवणवचनादिभिः अवभासते इति सर्वेन्द्रियगुणाभासं सर्वेन्द्रियव्यापारैः व्यापृतमिव तत् ज्ञेयम् इत्यर्थः~; ‘ध्यायतीव लेलायतीव’\footnote{बृ. उ. ४~। ३~। ७} इति श्रुतेः~। कस्मात् पुनः कारणात् न व्यापृतमेवेति गृह्यते इत्यतः आह — सर्वेन्द्रियविवर्जितम्~, सर्वकरणरहितमित्यर्थः~। अतः न करणव्यापारैः व्यापृतं तत् ज्ञेयम्~। यस्तु अयं मन्त्रः — ‘अपाणिपादो जवनो ग्रहीता पश्यत्यचक्षुः स शृणोत्यकर्णः’\footnote{श्वे. उ. ३~। १९} इत्यादिः, स सर्वेन्द्रियोपाधिगुणानुगुण्यभजनशक्तिमत् तत् ज्ञेयम् इत्येवं प्रदर्शनार्थः, न तु साक्षादेव जवनादिक्रियावत्त्वप्रदर्शनार्थः~। ‘अन्धो मणिमविन्दत्’\footnote{तै. आ. १~। ११} इत्यादिमन्त्रार्थवत् तस्य मन्त्रस्य अर्थः~। यस्मात् सर्वकरणवर्जितं ज्ञेयम्~, तस्मात् असक्तं सर्वसंश्लेषवर्जितम्~। यद्यपि एवम्~, तथापि सर्वभृच्च एव~। सदास्पदं हि सर्वं सर्वत्र सद्बुद्ध्यनुगमात्~। न हि मृगतृष्णिकादयोऽपि निरास्पदाः भवन्ति~। अतः सर्वभृत् सर्वं बिभर्ति इति~। स्यात् इदं च अन्यत् ज्ञेयस्य सत्त्वाधिगमद्वारम् — निर्गुणं सत्त्वरजस्तमांसि गुणाः तैः वर्जितं तत् ज्ञेयम्~, तथापि गुणभोक्तृ च गुणानां सत्त्वरजस्तमसां शब्दादिद्वारेण सुखदुःखमोहाकारपरिणतानां भोक्तृ च उपलब्धृ च तत् ज्ञेयम् इत्यर्थः~॥~१४~॥\par
 किञ्च —} 
\begin{center}{\bfseries बहिरन्तश्च भूतानामचरं चरमेव च~।\\सूक्ष्मत्वात्तदविज्ञेयं दूरस्थं चान्तिके च तत्~॥~१५~॥}\end{center} 
बहिः त्वक्पर्यन्तं देहम् आत्मत्वेन अविद्याकल्पितम् अपेक्ष्य तमेव अवधिं कृत्वा बहिः उच्यते~। तथा प्रत्यगात्मानमपेक्ष्य देहमेव अवधिं कृत्वा अन्तः उच्यते~। ‘बहिरन्तश्च’ इत्युक्ते मध्ये अभावे प्राप्ते, इदमुच्यते — अचरं चरमेव च, यत् चराचरं देहाभासमपि तदेव ज्ञेयं यथा रज्जुसर्पाभासः~। यदि अचरं चरमेव च स्यात् व्यवहारविषयं सर्वं ज्ञेयम्~, किमर्थम् ‘इदम्’ इति सर्वैः न विज्ञेयम् इति~? उच्यते — सत्यं सर्वाभासं तत्~; तथापि व्योमवत् सूक्ष्मम्~। अतः सूक्ष्मत्वात् स्वेन रूपेण तत् ज्ञेयमपि अविज्ञेयम् अविदुषाम्~। विदुषां तु, ‘आत्मैवेदं सर्वम्’\footnote{छा. उ. ७~। २५~। २} ‘ब्रह्मैवेदं सर्वम्’ इत्यादिप्रमाणतः नित्यं विज्ञातम्~। अविज्ञाततया दूरस्थं वर्षसहस्रकोट्यापि अविदुषाम् अप्राप्यत्वात्~। अन्तिके च तत्~, आत्मत्वात् विदुषाम्~॥~१५~॥\par
 किञ्च —} 
\begin{center}{\bfseries अविभक्तं च भूतेषु विभक्तमिव च स्थितम्~।\\भूतभर्तृ च तज्ज्ञेयं ग्रसिष्णु प्रभविष्णु च~॥~१६~॥}\end{center} 
अविभक्तं च प्रतिदेहं व्योमवत् तदेकम्~। भूतेषु सर्वप्राणिषु विभक्तमिव च स्थितं देहेष्वेव विभाव्यमानत्वात्~। भूतभर्तृ च भूतानि बिभर्तीति तत् ज्ञेयं भूतभर्तृ च स्थितिकाले~। प्रलयकाले गृसिष्णु ग्रसनशीलम्~। उत्पत्तिकाले प्रभविष्णु च प्रभवनशीलं यथा रज्ज्वादिः सर्पादेः मिथ्याकल्पितस्य~॥~१६~॥\par
 किञ्च, सर्वत्र विद्यमानमपि सत् न उपलभ्यते चेत्~, ज्ञेयं तमः तर्हि~? न~। किं तर्हि~? —} 
\begin{center}{\bfseries ज्योतिषामपि तज्ज्योतिस्तमसः परमुच्यते~।\\ज्ञानं ज्ञेयं ज्ञानगम्यं हृदि सर्वस्य विष्ठितम्~॥~१७~॥}\end{center} 
ज्योतिषाम् आदित्यादीनामपि तत् ज्ञेयं ज्योतिः~। आत्मचैतन्यज्योतिषा इद्धानि हि आदित्यादीनि ज्योतींषि दीप्यन्ते, ‘येन सूर्यस्तपति तेजसेद्धः’\footnote{तै. ब्रा. ३~। १२~। ९} ‘तस्य भासा सर्वमिदं विभाति’\footnote{मु. उ. २~। २~। ११} इत्यादिश्रुतिभ्यः~; स्मृतेश्च इहैव — ‘यदादित्यगतं तेजः’\footnote{भ. गी. १५~। १२} इत्यादेः~। तमसः अज्ञानात् परम् अस्पृष्टम् उच्यते~। ज्ञानादेः दुःसम्पादनबुद्ध्या प्राप्तावसादस्य उत्तम्भनार्थमाह — ज्ञानम् अमानित्वादि~; ज्ञेयम् ‘ज्ञेयं यत् तत् प्रवक्ष्यामि’\footnote{भ. गी. १३~। १२} इत्यादिना उक्तम्~; ज्ञानगम्यम् ज्ञेयमेव ज्ञातं सत् ज्ञानफलमिति ज्ञानगम्यमुच्यते~; ज्ञायमानं तु ज्ञेयम्~। तत् एतत् त्रयमपि हृदि बुद्धौ सर्वस्य प्राणिजातस्य विष्ठितं विशेषेण स्थितम्~। तत्रैव हि त्रयं विभाव्यते~॥~१७~॥\par
 यथोक्तार्थोपसंहारार्थः अयं श्लोकः आरभ्यते —} 
\begin{center}{\bfseries इति क्षेत्रं तथा ज्ञानं ज्ञेयं चोक्तं समासतः~।\\मद्भक्त एतद्विज्ञाय मद्भावायोपपद्यते~॥~१८~॥}\end{center} 
इति एवं क्षेत्रं महाभूतादि धृत्यन्तं तथा ज्ञानम् अमानित्वादि तत्त्वज्ञानार्थदर्शनपर्यन्तं ज्ञेयं च ‘ज्ञेयं यत् तत्’\footnote{भ. गी. १३~। १२} इत्यादि ‘तमसः परमुच्यते’\footnote{भ. गी. १३~। १७} इत्येवमन्तम् उक्तं समासतः सङ्क्षेपतः~। एतावान् सर्वः हि वेदार्थः गीतार्थश्च उपसंहृत्य उक्तः~। अस्मिन् सम्यग्दर्शने कः अधिक्रियते इति उच्यते — मद्भक्तः मयि ईश्वरे सर्वज्ञे परमगुरौ वासुदेवे समर्पितसर्वात्मभावः, यत् पश्यति शृणोति स्पृशति वा ‘सर्वमेव भगवान् वासुदेवः’ इत्येवंग्रहाविष्टबुद्धिः मद्भक्तः स एतत् यथोक्तं सम्यग्दर्शनं विज्ञाय, मद्भावाय मम भावः मद्भावः परमात्मभावः तस्मै मद्भावाय उपपद्यते मोक्षं गच्छति~॥~१८~॥\par
 तत्र सप्तमे ईश्वरस्य द्वे प्रकृती उपन्यस्ते, परापरे क्षेत्रक्षेत्रज्ञलक्षणे~; ‘एतद्योनीनि भूतानि’\footnote{भ. गी. ७~। ६} इति च उक्तम्~। क्षेत्रक्षेत्रज्ञप्रकृतिद्वययोनित्वं कथं भूतानामिति अयमर्थः अधुना उच्यते —} 
\begin{center}{\bfseries प्रकृतिं पुरुषं चैव विद्ध्यनादी उभावपि~।\\विकारांश्च गुणांश्चैव विद्धि प्रकृतिसम्भवान्~॥~१९~॥}\end{center} 
प्रकृतिं पुरुषं चैव ईश्वरस्य प्रकृती तौ प्रकृतिपुरुषौ उभावपि अनादी विद्धि, न विद्यते आदिः ययोः तौ अनादी~। नित्येश्वरत्वात् ईश्वरस्य तत्प्रकृत्योरपि युक्तं नित्यत्वेन भवितुम्~। प्रकृतिद्वयवत्त्वमेव हि ईश्वरस्य ईश्वरत्वम्~। याभ्यां प्रकृतिभ्याम् ईश्वरः जगदुत्पत्तिस्थितिप्रलयहेतुः, ते द्वे अनादी सत्यौ संसारस्य कारणम्~॥~} 
न आदी अनादी इति तत्पुरुषसमासं केचित् वर्णयन्ति~। तेन हि किल ईश्वरस्य कारणत्वं सिध्यति~। यदि पुनः प्रकृतिपुरुषावेव नित्यौ स्यातां तत्कृतमेव जगत् न ईश्वरस्य जगतः कर्तृत्वम्~। तत् असत्~; प्राक् प्रकृतिपुरुषयोः उत्पत्तेः ईशितव्याभावात् ईश्वरस्य अनीश्वरत्वप्रसङ्गात्~, संसारस्य निर्निमित्तत्वे अनिर्मोक्षप्रसङ्गात् शास्त्रानर्थक्यप्रसङ्गात् बन्धमोक्षाभावप्रसङ्गाच्च~। नित्यत्वे पुनः ईश्वरस्य प्रकृत्योः सर्वमेतत् उपपन्नं भवेत्~। कथम्~? } 
विकारांश्च गुणांश्चैव वक्ष्यमाणान्विकारान् बुद्ध्यादिदेहेन्द्रियान्तान् गुणांश्च सुखदुःखमोहप्रत्ययाकारपरिणतान् विद्धि जानीहि प्रकृतिसम्भवान्~, प्रकृतिः ईश्वरस्य विकारकारणशक्तिः त्रिगुणात्मिका माया, सा सम्भवो येषां विकाराणां गुणानां च तान् विकारान् गुणांश्च विद्धि प्रकृतिसम्भवान् प्रकृतिपरिणामान्~॥~१९~॥\par
 के पुनः ते विकाराः गुणाश्च प्रकृतिसम्भवाः —} 
\begin{center}{\bfseries कार्यकरणकर्तृत्वे हेतुः प्रकृतिरुच्यते~।\\पुरुषः सुखदुःखानां भोक्तृत्वे हेतुरुच्यते~॥~२०~॥}\end{center} 
कार्यकरणकर्तृत्वे — कार्यं शरीरं करणानि तत्स्थानि त्रयोदश~। देहस्यारम्भकाणि भूतानि पञ्च विषयाश्च प्रकृतिसम्भवाः विकाराः पूर्वोक्ताः इह कार्यग्रहणेन गृह्यन्ते~। गुणाश्च प्रकृतिसम्भवाः सुखदुःखमोहात्मकाः करणाश्रयत्वात् करणग्रहणेन गृह्यन्ते~। तेषां कार्यकरणानां कर्तृत्वम् उत्पादकत्वं यत् तत् कार्यकरणकर्तृत्वं तस्मिन् कार्यकरणकर्तृत्वे हेतुः कारणम् आरम्भकत्वेन प्रकृतिः उच्यते~। एवं कार्यकरणकर्तृत्वेन संसारस्य कारणं प्रकृतिः~। कार्यकारणकर्तृत्वे इत्यस्मिन्नपि पाठे, कार्यं यत् यस्य परिणामः तत् तस्य कार्यं विकारः विकारि कारणं तयोः विकारविकारिणोः कार्यकारणयोः कर्तृत्वे इति~। अथवा, षोडश विकाराः कार्यं सप्त प्रकृतिविकृतयः कारणम् तान्येव कार्यकारणान्युच्यन्ते तेषां कर्तृत्वे हेतुः प्रकृतिः उच्यते, आरम्भकत्वेनैव~। पुरुषश्च संसारस्य कारणं यथा स्यात् तत् उच्यते — पुरुषः जीवः क्षेत्रज्ञः भोक्ता इति पर्यायः, सुखदुःखानां भोग्यानां भोक्तृत्वे उपलब्धृत्वे हेतुः उच्यते~॥~} 
कथं पुनः अनेन कार्यकरणकर्तृत्वेन सुखदुःखभोक्तृत्वेन च प्रकृतिपुरुषयोः संसारकारणत्वमुच्यते इति, अत्र उच्यते — कार्यकरणसुखदुःखरूपेण हेतुफलात्मना प्रकृतेः परिणामाभावे, पुरुषस्य च चेतनस्य असति तदुपलब्धृत्वे, कुतः संसारः स्यात्~? यदा पुनः कार्यकरणसुखदुःखस्वरूपेण हेतुफलात्मना परिणतया प्रकृत्या भोग्यया पुरुषस्य तद्विपरीतस्य भोक्तृत्वेन अविद्यारूपः संयोगः स्यात्~, तदा संसारः स्यात् इति~। अतः यत् प्रकृतिपुरुषयोः कार्यकरणकर्तृत्वेन सुखदुःखभोक्तृत्वेन च संसारकारणत्वमुक्तम्~, तत् युक्तम्~। कः पुनः अयं संसारो नाम~? सुखदुःखसम्भोगः संसारः~। पुरुषस्य च सुखदुःखानां सम्भोक्तृत्वं संसारित्वमिति~॥~२०~॥\par
 यत् पुरुषस्य सुखदुःखानां भोक्तृत्वं संसारित्वम् इति उक्तं तस्य तत् किंनिमित्तमिति उच्यते —} 
\begin{center}{\bfseries पुरुषः प्रकृतिस्थो हि भुङ्क्ते प्रकृतिजान्गुणान्~।\\कारणं गुणसङ्गोऽस्य सदसद्योनिजन्मसु~॥~२१~॥}\end{center} 
पुरुषः भोक्ता प्रकृतिस्थः प्रकृतौ अविद्यालक्षणायां कार्यकरणरूपेण परिणतायां स्थितः प्रकृतिस्थः, प्रकृतिमात्मत्वेन गतः इत्येतत्~, हि यस्मात्~, तस्मात् भुङ्क्ते उपलभते इत्यर्थः~। प्रकृतिजान् प्रकृतितः जातान् सुखदुःखमोहाकाराभिव्यक्तान् गुणान् ‘सुखी, दुःखी, मूढः, पण्डितः अहम्’ इत्येवम्~। सत्यामपि अविद्यायां सुखदुःखमोहेषु गुणेषु भुज्यमानेषु यः सङ्गः आत्मभावः संसारस्य सः प्रधानं कारणं जन्मनः, ‘सः यथाकामो भवति तत्क्रतुर्भवति’\footnote{बृ. उ. ४~। ४~। ५} इत्यादिश्रुतेः~। तदेतत् आह — कारणं हेतुः गुणसङ्गः गुणेषु सङ्गः अस्य पुरुषस्य भोक्तुः सदसद्योनिजन्मसु, सत्यश्च असत्यश्च योनयः सदसद्योनयः तासु सदसद्योनिषु जन्मानि सदसद्योनिजन्मानि, तेषु सदसद्योनिजन्मसु विषयभूतेषु कारणं गुणसङ्गः~। अथवा, सदसद्योनिजन्मसु अस्य संसारस्य कारणं गुणसङ्गः इति संसारपदमध्याहार्यम्~। सद्योनयः देवादियोनयः~; असद्योनयः पश्वादियोनयः~। सामर्थ्यात् सदसद्योनयः मनुष्ययोनयोऽपि अविरुद्धाः द्रष्टव्याः~॥~} 
एतत् उक्तं भवति — प्रकृतिस्थत्वाख्या अविद्या, गुणेषु च सङ्गः कामः, संसारस्य कारणमिति~। तच्च परिवर्जनाय उच्यते~। अस्य च निवृत्तिकारणं ज्ञानवैराग्ये ससंन्यासे गीताशास्त्रे प्रसिद्धम्~। तच्च ज्ञानं पुरस्तात् उपन्यस्तं क्षेत्रक्षेत्रज्ञविषयम् ‘यज्ज्ञात्वामृतमश्नुते’\footnote{भ. गी. १३~। १२} इति~। उक्तं च अन्यापोहेन अतद्धर्माध्यारोपेण च~॥~२१~॥\par
 तस्यैव पुनः साक्षात् निर्देशः क्रियते —} 
\begin{center}{\bfseries उपद्रष्टानुमन्ता च भर्ता भोक्ता महेश्वरः~।\\परमात्मेति चाप्युक्तो देहेऽस्मिन्पुरुषः परः~॥~२२~॥}\end{center} 
उपद्रष्टा समीपस्थः सन् द्रष्टा स्वयम् अव्यापृतः~। यथा ऋत्विग्यजमानेषु यज्ञकर्मव्यापृतेषु तटस्थः अन्यः अव्यापृतः यज्ञविद्याकुशलः ऋत्विग्यजमानव्यापारगुणदोषाणाम् ईक्षिता, तद्वच्च कार्यकरणव्यापारेषु अव्यापृतः अन्यः तद्विलक्षणः तेषां कार्यकरणानां सव्यापाराणां सामीप्येन द्रष्टा उपद्रष्टा~। अथवा, देहचक्षुर्मनोबुद्ध्यात्मानः द्रष्टारः, तेषां बाह्यः द्रष्टा देहः, ततः आरभ्य अन्तरतमश्च प्रत्यक् समीपे आत्मा द्रष्टा, यतः परः अन्तरतमः नास्ति द्रष्टा~; सः अतिशयसामीप्येन द्रष्टृत्वात् उपद्रष्टा स्यात्~। यज्ञोपद्रष्टृवद्वा सर्वविषयीकरणात् उपद्रष्टा~। अनुमन्ता च, अनुमोदनम् अनुमननं कुर्वत्सु तत्क्रियासु परितोषः, तत्कर्ता अनुमन्ता च~। अथवा, अनुमन्ता, कार्यकरणप्रवृत्तिषु स्वयम् अप्रवृत्तोऽपि प्रवृत्त इव तदनुकूलः विभाव्यते, तेन अनुमन्ता~। अथवा, प्रवृत्तान् स्वव्यापारेषु तत्साक्षिभूतः कदाचिदपि न निवारयति इति अनुमन्ता~। भर्ता, भरणं नाम देहेन्द्रियमनोबुद्धीनां संहतानां चैतन्यात्मपारार्थ्येन निमित्तभूतेन चैतन्याभासानां यत् स्वरूपधारणम्~, तत् चैतन्यात्मकृतमेव इति भर्ता आत्मा इति उच्यते~। भोक्ता, अग्न्युष्णवत् नित्यचैतन्यस्वरूपेण बुद्धेः सुखदुःखमोहात्मकाः प्रत्ययाः सर्वविषयविषयाः चैतन्यात्मग्रस्ता इव जायमानाः विभक्ताः विभाव्यन्ते इति भोक्ता आत्मा उच्यते~। महेश्वरः, सर्वात्मत्वात् स्वतन्त्रत्वाच्च महान् ईश्वरश्च इति महेश्वरः~। परमात्मा, देहादीनां बुद्ध्यन्तानां प्रत्यगात्मत्वेन कल्पितानाम् अविद्यया परमः उपद्रष्टृत्वादिलक्षणः आत्मा इति परमात्मा~। सः अतः ‘परमात्मा’ इत्यनेन शब्देन च अपि उक्तः कथितः श्रुतौ~। क्व असौ~? अस्मिन् देहे पुरुषः परः अव्यक्तात्~, ‘उत्तमः पुरुषस्त्वन्यः परमात्मेत्युदाहृतः’\footnote{भ. गी. १५~। १७} इति यः वक्ष्यमाणः‘क्षेत्रज्ञं चापि मां विद्धि’\footnote{भ. गी. १३~। २} इति उपन्यस्तः व्याख्याय उपसंहृतश्च~॥~२२~॥\par
  तमेतं यथोक्तलक्षणम् आत्मानम् —} 
\begin{center}{\bfseries य एवं वेत्ति पुरुषं प्रकृतिं च गुणैः सह~।\\सर्वथा वर्तमानोऽपि न स भूयोऽभिजायते~॥~२३~॥}\end{center} 
यः एवं यथोक्तप्रकारेण वेत्ति पुरुषं साक्षात् अहमिति प्रकृतिं च यथोक्ताम् अविद्यालक्षणां गुणैः स्वविकारैः सह निवर्तिताम् अभावम् आपादितां विद्यया, सर्वथा सर्वप्रकारेण वर्तमानोऽपि सः भूयः पुनः पतिते अस्मिन् विद्वच्छरीरे देहान्तराय न अभिजायते न उत्पद्यते, देहान्तरं न गृह्णाति इत्यर्थः~। अपिशब्दात् किमु वक्तव्यं स्ववृत्तस्थो न जायते इति अभिप्रायः~॥~} 
ननु, यद्यपि ज्ञानोत्पत्त्यनन्तरं पुनर्जन्माभाव उक्तः, तथापि प्राक् ज्ञानोत्पत्तेः कृतानां कर्मणाम् उत्तरकालभाविनां च, यानि च अतिक्रान्तानेकजन्मकृतानि तेषां च, फलमदत्त्वा नाशो न युक्त इति, स्युः त्रीणि जन्मानि, कृतविप्रणाशो हि न युक्त इति, यथा फले प्रवृत्तानाम् आरब्धजन्मनां कर्मणाम्~। न च कर्मणां विशेषः अवगम्यते~। तस्मात् त्रिप्रकाराण्यपि कर्माणि त्रीणि जन्मानि आरभेरन्~; संहतानि वा सर्वाणि एकं जन्म आरभेरन्~। अन्यथा कृतविनाशे सति सर्वत्र अनाश्वासप्रसङ्गः, शास्त्रानर्थक्यं च स्यात्~। इत्यतः इदमयुक्तमुक्तम् ‘न स भूयोऽभिजायते’ इति~। न~; ‘क्षीयन्ते चास्य कर्माणि’\footnote{मु. उ. २~। २~। ९} ‘ब्रह्म वेद ब्रह्मैव भवति’\footnote{मु. उ. ३~। २~। ९} ‘तस्य तावदेव चिरम्’\footnote{छा. उ. ६~। १४~। २} ‘इषीकातूलवत् सर्वाणि कर्माणि प्रदूयन्ते’\footnote{छा. उ. ५~। २४~। ३} इत्यादिश्रुतिशतेभ्यः उक्तो विदुषः सर्वकर्मदाहः~। इहापि च उक्तः ‘यथैधांसि’\footnote{भ. गी. ४~। ३७} इत्यादिना सर्वकर्मदाहः, वक्ष्यति च~। उपपत्तेश्च — अविद्याकामक्लेशबीजनिमित्तानि हि कर्माणि जन्मान्तराङ्कुरम् आरभन्ते~; इहापि च ‘साहङ्काराभिसन्धीनि कर्माणि फलारम्भकाणि, न इतराणि’ इति तत्र तत्र भगवता उक्तम्~। ‘बीजान्यग्न्युपदग्धानि न रोहन्ति यथा पुनः~। ज्ञानदग्धैस्तथा क्लेशैर्नात्मा सम्पद्यते पुनः’\footnote{मो. २११~। १७} इति च~। अस्तु तावत् ज्ञानोत्पत्त्युत्तरकालकृतानां कर्मणां ज्ञानेन दाहः ज्ञानसहभावित्वात्~। न तु इह जन्मनि ज्ञानोत्पत्तेः प्राक् कृतानां कर्मणां अतीतजन्मकृतानां च दाहः युक्तः~। न~; ‘सर्वकर्माणि’\footnote{भ. गी. ४~। ३७} इति विशेषणात्~। ज्ञानोत्तरकालभाविनामेव सर्वकर्मणाम् इति चेत्~, न~; सङ्कोचे कारणानुपपत्तेः~। यत्तु उक्तम् ‘यथा वर्तमानजन्मारम्भकाणि कर्माणि न क्षीयन्ते फलदानाय प्रवृत्तान्येव सत्यपि ज्ञाने, तथा अनारब्धफलानामपि कर्मणां क्षयो न युक्तः’ इति, तत् असत्~। कथम्~? तेषां मुक्तेषुवत् प्रवृत्तफलत्वात्~। यथा पूर्वं लक्ष्यवेधाय मुक्तः इषुः धनुषः लक्ष्यवेधोत्तरकालमपि आरब्धवेगक्षयात् पतनेनैव निवर्तते, एवं शरीरारम्भकं कर्म शरीरस्थितिप्रयोजने निवृत्तेऽपि, आ संस्कारवेगक्षयात् पूर्ववत् वर्तते एव~। यथा स एव इषुः प्रवृत्तिनिमित्तानारब्धवेगस्तु अमुक्तो धनुषि प्रयुक्तोऽपि उपसंह्रियते, तथा अनारब्धफलानि कर्माणि स्वाश्रयस्थान्येव ज्ञानेन निर्बीजीक्रियन्ते इति, पतिते अस्मिन् विद्वच्छरीरे ‘न स भूयोऽभिजायते’ इति युक्तमेव उक्तमिति सिद्धम्~॥~२३~॥\par
 अत्र आत्मदर्शने उपायविकल्पाः इमे ध्यानादयः उच्यन्ते —} 
\begin{center}{\bfseries ध्यानेनात्मनि पश्यन्ति केचिदात्मानमात्मना~।\\अन्ये साङ्ख्येन योगेन कर्मयोगेन चापरे~॥~२४~॥}\end{center} 
ध्यानेन, ध्यानं नाम शब्दादिभ्यो विषयेभ्यः श्रोत्रादीनि करणानि मनसि उपसंहृत्य, मनश्च प्रत्यक्चेतयितरि, एकाग्रतया यत् चिन्तनं तत् ध्यानम्~; तथा, ध्यायतीव बकः, ध्यायतीव पृथिवी, ध्यायन्तीव पर्वताः इति उपमोपादानात्~। तैलधारावत् सन्ततः अविच्छिन्नप्रत्ययो ध्यानम्~; तेन ध्यानेन आत्मनि बुद्धौ पश्यन्ति आत्मानं प्रत्यक्चेतनम् आत्मना स्वेनैव प्रत्यक्चेतनेन ध्यानसंस्कृतेन अन्तःकरणेन केचित् योगिनः~। अन्ये साङ्ख्येन योगेन, साङ्ख्यं नाम ‘इमे सत्त्वरजस्तमांसि गुणाः मया दृश्या अहं तेभ्योऽन्यः तद्व्यापारसाक्षिभूतः नित्यः गुणविलक्षणः आत्मा’ इति चिन्तनम् एषः साङ्ख्यो योगः, तेन ‘पश्यन्ति आत्मानमात्मना’ इति वर्तते~। कर्मयोगेन, कर्मैव योगः, ईश्वरार्पणबुद्ध्या अनुष्ठीयमानं घटनरूपं योगार्थत्वात् योगः उच्यते गुणतः~; तेन सत्त्वशुद्धिज्ञानोत्पत्तिद्वारेण च अपरे~॥~२४~॥\par
 \begin{center}{\bfseries अन्ये त्वेवमजानन्तः श्रुत्वान्येभ्य उपासते~।\\तेऽपि चातितरन्त्येव मृत्युं श्रुतिपरायणाः~॥~२५~॥}\end{center} 
अन्ये तु एषु विकल्पेषु अन्यतमेनापि एवं यथोक्तम् आत्मानम् अजानन्तः अन्येभ्यः आचार्येभ्यः श्रुत्वा ‘इदमेव चिन्तयत’ इति उक्ताः उपासते श्रद्दधानाः सन्तः चिन्तयन्ति~। तेऽपि च अतितरन्त्येव अतिक्रामन्त्येव मृत्युम्~, मृत्युयुक्तं संसारम् इत्येतत्~। श्रुतिपरायणाः श्रुतिः श्रवणं परम् अयनं गमनं मोक्षमार्गप्रवृत्तौ परं साधनं येषां ते श्रुतिपरायणाः~; केवलपरोपदेशप्रमाणाः स्वयं विवेकरहिताः इत्यभिप्रायः~। किमु वक्तव्यम् प्रमाणं प्रति स्वतन्त्राः विवेकिनः मृत्युम् अतितरन्ति इति अभिप्रायः~॥~२५~॥\par
 क्षेत्रज्ञेश्वरैकत्वविषयं ज्ञानं मोक्षसाधनम् ‘यज्ज्ञात्वामृतमश्नुते’\footnote{भ. गी. १३~। १२} इत्युक्तम्~, तत् कस्मात् हेतोरिति, तद्धेतुप्रदर्शनार्थं श्लोकः आरभ्यते —} 
\begin{center}{\bfseries यावत्सञ्जायते किञ्चित्सत्त्वं स्थावरजङ्गमम्~।\\क्षेत्रक्षेत्रज्ञसंयोगात्तद्विद्धि भरतर्षभ~॥~२६~॥}\end{center} 
यावत् यत् किञ्चित् सञ्जायते समुत्पद्यते सत्त्वं वस्तु~; किम् अविशेषेण~? नेत्याह — स्थावरजङ्गमं स्थावरं जङ्गमं च क्षेत्रक्षेत्रज्ञसंयोगात् तत् जायते इत्येवं विद्धि जानीहि भरतर्षभ~॥~} 
कः पुनः अयं क्षेत्रक्षेत्रज्ञयोः संयोगः अभिप्रेतः~? न तावत् रज्ज्वेव घटस्य अवयवसंश्लेषद्वारकः सम्बन्धविशेषः संयोगः क्षेत्रेण क्षेत्रज्ञस्य सम्भवति, आकाशवत् निरवयवत्वात्~। नापि समवायलक्षणः तन्तुपटयोरिव क्षेत्रक्षेत्रज्ञयोः इतरेतरकार्यकारणभावानभ्युपगमात् इति, उच्यते — क्षेत्रक्षेत्रज्ञयोः विषयविषयिणोः भिन्नस्वभावयोः इतरेतरतद्धर्माध्यासलक्षणः संयोगः क्षेत्रक्षेत्रज्ञस्वरूपविवेकाभावनिबन्धनः, रज्जुशुक्तिकादीनां तद्विवेकज्ञानाभावात् अध्यारोपितसर्परजतादिसंयोगवत्~। सः अयं अध्यासस्वरूपः क्षेत्रक्षेत्रज्ञसंयोगः मिथ्याज्ञानलक्षणः~। यथाशास्त्रं क्षेत्रक्षेत्रज्ञलक्षणभेदपरिज्ञानपूर्वकं प्राक् दर्शितरूपात् क्षेत्रात् मुञ्जादिव इषीकां यथोक्तलक्षणं क्षेत्रज्ञं प्रविभज्य ‘न सत्तन्नासदुच्यते’\footnote{भ. गी. १३~। १२} इत्यनेन निरस्तसर्वोपाधिविशेषं ज्ञेयं ब्रह्मस्वरूपेण यः पश्यति, क्षेत्रं च मायानिर्मितहस्तिस्वप्नदृष्टवस्तुगन्धर्वनगरादिवत् ‘असदेव सदिव अवभासते’ इति एवं निश्चितविज्ञानः यः, तस्य यथोक्तसम्यग्दर्शनविरोधात् अपगच्छति मिथ्याज्ञानम्~। तस्य जन्महेतोः अपगमात् ‘य एवं वेत्ति पुरुषं प्रकृतिं च गुणैः सह’\footnote{भ. गी. १३~। २३} इत्यनेन ‘विद्वान् भूयः न अभिजायते’ इति यत् उक्तम्~, तत् उपपन्नमुक्तम्~॥~२६~॥\par
 ‘न स भूयोऽभिजायते’\footnote{भ. गी. १३~। २३} इति सम्यग्दर्शनफलम् अविद्यादिसंसारबीजनिवृत्तिद्वारेण जन्माभावः उक्तः~। जन्मकारणं च अविद्यानिमित्तकः क्षेत्रक्षेत्रज्ञसंयोगः उक्तः~; अतः तस्याः अविद्यायाः निवर्तकं सम्यग्दर्शनम् उक्तमपि पुनः शब्दान्तरेण उच्यते —} 
\begin{center}{\bfseries समं सर्वेषु भूतेषु तिष्ठन्तं परमेश्वरम्~।\\विनश्यत्स्वविनश्यन्तं यः पश्यति स पश्यति~॥~२७~॥}\end{center} 
समं निर्विशेषं तिष्ठन्तं स्थितिं कुर्वन्तम्~; क्व~? सर्वेषु समस्तेषु भूतेषु ब्रह्मादिस्थावरान्तेषु प्राणिषु~; कम्~? परमेश्वरं देहेन्द्रियमनोबुद्ध्यव्यक्तात्मनः अपेक्ष्य परमेश्वरः, तं सर्वेषु भूतेषु समं तिष्ठन्तम्~। तानि विशिनष्टि विनश्यत्सु इति, तं च परमेश्वरम् अविनश्यन्तम् इति, भूतानां परमेश्वरस्य च अत्यन्तवैलक्षण्यप्रदर्शनार्थम्~। कथम्~? सर्वेषां हि भावविकाराणां जनिलक्षणः भावविकारो मूलम्~; जन्मोत्तरकालभाविनः अन्ये सर्वे भावविकाराः विनाशान्ताः~; विनाशात् परो न कश्चित् अस्ति भावविकारः, भावाभावात्~। सति हि धर्मिणि धर्माः भवन्ति~। अतः अन्त्यभावविकाराभावानुवादेन पूर्वभाविनः सर्वे भावविकाराः प्रतिषिद्धाः भवन्ति सह कार्यैः~। तस्मात् सर्वभूतैः वैलक्षण्यम् अत्यन्तमेव परमेश्वरस्य सिद्धम्~, निर्विशेषत्वम् एकत्वं च~। यः एवं यथोक्तं परमेश्वरं पश्यति, सः पश्यति~॥~} 
ननु सर्वोऽपि लोकः पश्यति, किं विशेषणेन इति~। सत्यं पश्यति~; किं तु विपरीतं पश्यति~। अतः विशिनष्टि — स एव पश्यतीति~। यथा तिमिरदृष्टिः अनेकं चन्द्रं पश्यति, तमपेक्ष्य एकचन्द्रदर्शी विशिष्यते — स एव पश्यतीति~; तथा इहापि एकम् अविभक्तं यथोक्तं आत्मानं यः पश्यति, सः विभक्तानेकात्मविपरीतदर्शिभ्यः विशिष्यते — स एव पश्यतीति~। इतरे पश्यन्तोऽपि न पश्यन्ति, विपरीतदर्शित्वात् अनेकचन्द्रदर्शिवत् इत्यर्थः~॥~२७~॥\par
 यथोक्तस्य सम्यग्दर्शनस्य फलवचनेन स्तुतिः कर्तव्या इति श्लोकः आरभ्यते —} 
\begin{center}{\bfseries समं पश्यन्हि सर्वत्र\\ समवस्थितमीश्वरम्~।\\न हिनस्त्यात्मनात्मानं\\ ततो याति परां गतिम्~॥~२८~॥}\end{center} 
समं पश्यन् उपलभमानः हि यस्मात् सर्वत्र सर्वभूतेषु समवस्थितं तुल्यतया अवस्थितम् ईश्वरम् अतीतानन्तरश्लोकोक्तलक्षणमित्यर्थः~। समं पश्यन् किम्~? न हिनस्ति हिंसां न करोति आत्मना स्वेनैव स्वमात्मानम्~। ततः तदहिंसनात् याति परां प्रकृष्टां गतिं मोक्षाख्याम्~॥~} 
ननु नैव कश्चित् प्राणी स्वयं स्वम् आत्मानं हिनस्ति~। कथम् उच्यते अप्राप्तम् ‘न हिनस्ति’ इति~? यथा ‘न पृथिव्यामग्निश्चेतव्यो नान्तरिक्षे’\footnote{तै. सं. ५~। २~। ७} इत्यादि~। नैष दोषः, अज्ञानाम् आत्मतिरस्करणोपपत्तेः~। सर्वो हि अज्ञः अत्यन्तप्रसिद्धं साक्षात् अपरोक्षात् आत्मानं तिरस्कृत्य अनात्मानम् आत्मत्वेन परिगृह्य, तमपि धर्माधर्मौ कृत्वा उपात्तम् आत्मानं हत्वा अन्यम् आत्मानम् उपादत्ते नवं तं चैवं हत्वा अन्यमेवं तमपि हत्वा अन्यम् इत्येवम् उपात्तमुपात्तम् आत्मानं हन्ति, इति आत्महा सर्वः अज्ञः~। यस्तु परमार्थात्मा, असावपि सर्वदा अविद्यया हत इव, विद्यमानफलाभावात्~, इति सर्वे आत्महनः एव अविद्वांसः~। यस्तु इतरः यथोक्तात्मदर्शी, सः उभयथापि आत्मना आत्मानं न हिनस्ति न हन्ति~। ततः याति परां गतिम् यथोक्तं फलं तस्य भवति इत्यर्थः~॥~२८~॥\par
 सर्वभूतस्थम् ईश्वरं समं पश्यन् ‘न हिनस्ति आत्मना आत्मानम्’ इति उक्तम्~। तत् अनुपपन्नं स्वगुणकर्मवैलक्षण्यभेदभिन्नेषु आत्मसु, इत्येतत् आशङ्क्य आह —} 
\begin{center}{\bfseries प्रकृत्यैव च कर्माणि क्रियमाणानि सर्वशः~।\\यः पश्यति तथात्मानमकर्तारं स पश्यति~॥~२९~॥}\end{center} 
प्रकृत्या प्रकृतिः भगवतः माया त्रिगुणात्मिका, ‘मायां तु प्रकृतिं विद्यात्’\footnote{श्वे. उ. ४~। १०} इति मन्त्रवर्णात्~, तया प्रकृत्यैव च न अन्येन महदादिकार्यकारणाकारपरिणतया कर्माणि वाङ्मनःकायारभ्याणि क्रियमाणानि निर्वर्त्यमानानि सर्वशः सर्वप्रकारैः यः पश्यति उपलभते, तथा आत्मानं क्षेत्रज्ञम् अकर्तारं सर्वोपाधिविवर्जितं सः पश्यति, सः परमार्थदर्शी इत्यभिप्रायः~; निर्गुणस्य अकर्तुः निर्विशेषस्य आकाशस्येव भेदे प्रमाणानुपपत्तिः इत्यर्थः~॥~२९~॥\par
 पुनरपि तदेव सम्यग्दर्शनं शब्दान्तरेण प्रपञ्चयति —} 
\begin{center}{\bfseries यदा भूतपृथग्भावमेकस्थमनुपश्यति~।\\तत एव च विस्तारं ब्रह्म सम्पद्यते तदा~॥~३०~॥}\end{center} 
यदा यस्मिन् काले भूतपृथग्भावं भूतानां पृथग्भावं पृथक्त्वम् एकस्मिन् आत्मनि स्थितं एकस्थम् अनुपश्यति शास्त्राचार्योपदेशम्~, अनु आत्मानं प्रत्यक्षत्वेन पश्यति ‘आत्मैव इदं सर्वम्’\footnote{छा. उ. ७~। २५~। २} इति, तत एव च तस्मादेव च विस्तारं उत्पत्तिं विकासम् ‘आत्मतः प्राण आत्मत आशा आत्मतः स्मर आत्मत आकाश आत्मतस्तेज आत्मत आप आत्मत आविर्भावतिरोभावावात्मतोऽन्नम्’\footnote{छा. उ. ७~। २६~। १} इत्येवमादिप्रकारैः विस्तारं यदा पश्यति, ब्रह्म सम्पद्यते ब्रह्मैव भवति तदा तस्मिन् काले इत्यर्थः~॥~३०~॥\par
 एकस्य आत्मानः सर्वदेहात्मत्वे तद्दोषसम्बन्धे प्राप्ते, इदम् उच्यते —} 
\begin{center}{\bfseries अनादित्वान्निर्गुणत्वा—\\ त्परमात्मायमव्ययः~।\\शरीरस्थोऽपि कौन्तेय\\ न करोति न लिप्यते~॥~३१~॥}\end{center} 
अनादित्वात् अनादेः भावः अनादित्वम्~, आदिः कारणम्~, तत् यस्य नास्ति तत् आनादि~। यद्धि आदिमत् तत् स्वेन आत्मना व्येति~; अयं तु अनादित्वात् निरवयव इति कृत्वा न व्येति~। तथा निर्गुणत्वात्~। सगुणो हि गुणव्ययात् व्येति~; अयं तु निर्गुणत्वाच्च न व्येति~; इति परमात्मा अयम् अव्ययः~; न अस्य व्ययो विद्यते इति अव्ययः~। यत एवमतः शरीरस्थोऽपि, शरीरेषु आत्मनः उपलब्धिः भवतीति शरीरस्थः उच्यते~; तथापि न करोति~। तदकरणादेव तत्फलेन न लिप्यते~। यो हि कर्ता, सः कर्मफलेन लिप्यते~। अयं तु अकर्ता, अतः न फलेन लिप्यते इत्यर्थः~॥~} 
कः पुनः देहेषु करोति लिप्यते च~? यदि तावत् अन्यः परमात्मनो देही करोति लिप्यते च, ततः इदम् अनुपपन्नम् उक्तं क्षेत्रज्ञेश्वरैकत्वम् ‘क्षेत्रज्ञं चापि मां विद्धि’\footnote{भ. गी. १३~। २} इत्यादि~। अथ नास्ति ईश्वरादन्यो देही, कः करोति लिप्यते च~? इति वाच्यम्~; परो वा नास्ति इति सर्वथा दुर्विज्ञेयं दुर्वाच्यं च इति भगवत्प्रोक्तम् औपनिषदं दर्शनं परित्यक्तं वैशेषिकैः साङ्ख्यार्हतबौद्धैश्च~। तत्र अयं परिहारो भगवता स्वेनैव उक्तः ‘स्वभावस्तु प्रवर्तते’\footnote{भ. गी. ५~। १४} इति~। अविद्यामात्रस्वभावो हि करोति लिप्यते इति व्यवहारो भवति, न तु परमार्थत एकस्मिन् परमात्मनि तत् अस्ति~। अतः एतस्मिन् परमार्थसाङ्ख्यदर्शने स्थितानां ज्ञाननिष्ठानां परमहंसपरिव्राजकानां तिरस्कृताविद्याव्यवहाराणां कर्माधिकारो नास्ति इति तत्र तत्र दर्शितं भगवता~॥~३१~॥\par
 किमिव न करोति न लिप्यते इति अत्र दृष्टान्तमाह —} 
\begin{center}{\bfseries यथा सर्वगतं सौक्ष्म्या—\\ दाकाशं नोपलिप्यते~।\\सर्वत्रावस्थितो देहे\\ तथात्मा नोपलिप्यते~॥~३२~॥}\end{center} 
यथा सर्वगतं व्यापि अपि सत् सौक्ष्म्यात् सूक्ष्मभावात् आकाशं खं न उपलिप्यते न सम्बध्यते, सर्वत्र अवस्थितः देहे तथा आत्मा न उपलिप्यते~॥~३२~॥\par
 किञ्च —} 
\begin{center}{\bfseries यथा प्रकाशयत्येकः कृत्स्नं लोकमिमं रविः~।\\क्षेत्रं क्षेत्री तथा कृत्स्नं प्रकाशयति भारत~॥~३३~॥}\end{center} 
यथा प्रकाशयति अवभासयति एकः कृत्स्नं लोकम् इमं रविः सविता आदित्यः, तथा तद्वत् महाभूतादि धृत्यन्तं क्षेत्रम् एकः सन् प्रकाशयति~। कः~? क्षेत्री परमात्मा इत्यर्थः~। रविदृष्टान्तः अत्र आत्मनः उभयार्थोऽपि भवति — रविवत् सर्वक्षेत्रेषु एक एव आत्मा, अलेपकश्च इति~॥~३३~॥\par
 समस्ताध्यायार्थोपसंहारार्थः अयं श्लोकः —} 
\begin{center}{\bfseries क्षेत्रक्षेत्रज्ञयोरेवमन्तरं ज्ञानचक्षुषा~।\\भूतप्रकृतिमोक्षं च ये विदुर्यान्ति ते परम्~॥~३४~॥}\end{center} 
क्षेत्रक्षेत्रज्ञयोः यथाव्याख्यातयोः एवं यथाप्रदर्शितप्रकारेण अन्तरम् इतरेतरवैलक्षण्यविशेषं ज्ञानचक्षुषा शास्त्राचार्यप्रसादोपदेशजनितम् आत्मप्रत्ययिकं ज्ञानं चक्षुः, तेन ज्ञानचक्षुषा, भूतप्रकृतिमोक्षं च, भूतानां प्रकृतिः अविद्यालक्षणा अव्यक्ताख्या, तस्याः भूतप्रकृतेः मोक्षणम् अभावगमनं च ये विदुः विजानन्ति, यान्ति गच्छन्ति ते परं परमात्मतत्त्वं ब्रह्म, न पुनः देहं आददते इत्यर्थः~॥~३४~॥\par
 
इति श्रीमत्परमहंसपरिव्राजकाचार्यस्य श्रीगोविन्दभगवत्पूज्यपादशिष्यस्य श्रीमच्छङ्करभगवतः कृतौ श्रीमद्भगवद्गीताभाष्ये त्रयोदशोऽध्यायः~॥\par
 
सर्वम् उत्पद्यमानं क्षेत्रक्षेत्रज्ञसंयोगात् उत्पद्यते इति उक्तम्~। तत् कथमिति, तत्प्रदर्शनार्थम् ‘परं भूयः’ इत्यादिः अध्यायः आरभ्यते~। अथवा, ईश्वरपरतन्त्रयोः क्षेत्रक्षेत्रज्ञयोः जगत्कारणत्वं न तु साङ्ख्यानामिव स्वतन्त्रयोः इत्येवमर्थम्~। प्रकृतिस्थत्वं गुणेषु च सङ्गः संसारकारणम् इति उक्तम्~। कस्मिन् गुणे कथं सङ्गः~? के वा गुणाः~? कथं वा ते बध्नन्ति इति~? गुणेभ्यश्च मोक्षणं कथं स्यात्~? मुक्तस्य च लक्षणं वक्तव्यम्~, इत्येवमर्थं च भगवान् उवाच —}\\ 
\begin{center}{\bfseries श्रीभगवानुवाच —\\ परं भूयः प्रवक्ष्यामि\\ ज्ञानानां ज्ञानमुत्तमम्~।\\यज्ज्ञात्वा मुनयः सर्वे\\ परां सिद्धिमितो गताः~॥~१~॥}\end{center} 
परं ज्ञानम् इति व्यवहितेन सम्बन्धः, भूयः पुनः पूर्वेषु सर्वेष्वध्यायेषु असकृत् उक्तमपि प्रवक्ष्यामि~। तच्च परं परवस्तुविषयत्वात्~। किं तत्~? ज्ञानं सर्वेषां ज्ञानानाम् उत्तमम्~, उत्तमफलत्वात्~। ज्ञानानाम् इति न अमानित्वादीनाम्~; किं तर्हि~? यज्ञादिज्ञेयवस्तुविषयाणाम् इति~। तानि न मोक्षाय, इदं तु मोक्षाय इति परोत्तमशब्दाभ्यां स्तौति श्रोतृबुद्धिरुच्युत्पादनार्थम्~। यत् ज्ञात्वा यत् ज्ञानं ज्ञात्वा प्राप्य मुनयः संन्यासिनः मननशीलाः सर्वे परां सिद्धिं मोक्षाख्याम् इतः अस्मात् देहबन्धनात् ऊर्ध्वं गताः प्राप्ताः~॥~१~॥\par
 अस्याश्च सिद्धेः ऐकान्तिकत्वं दर्शयति —} 
\begin{center}{\bfseries इदं ज्ञानमुपाश्रित्य मम साधर्म्यमागताः~।\\सर्गेऽपि नोपजायन्ते प्रलये न व्यथन्ति च~॥~२~॥}\end{center} 
इदं ज्ञानं यथोक्तमुपाश्रित्य, ज्ञानसाधनम् अनुष्ठाय इत्येतत्~, मम परमेश्वरस्य साधर्म्यं मत्स्वरूपताम् आगताः प्राप्ताः इत्यर्थः~। न तु समानधर्मता साधर्म्यम्~, क्षेत्रज्ञेश्वरयोः भेदानभ्युपगमात् गीताशास्त्रे~। फलवादश्च अयं स्तुत्यर्थम् उच्यते~। सर्गेऽपि सृष्टिकालेऽपि न उपजायन्ते~। न उत्पद्यन्ते~। प्रलये ब्रह्मणोऽपि विनाशकाले न व्यथन्ति च व्यथां न आपद्यन्ते, न च्यवन्ति इत्यर्थः~॥~२~॥\par
 क्षेत्रक्षेत्रज्ञसंयोगः ईदृशः भूतकारणम् इत्याह —} 
\begin{center}{\bfseries मम योनिर्महद्ब्रह्म तस्मिन्गर्भं दधाम्यहम्~।\\सम्भवः सर्वभूतानां ततो भवति भारत~॥~३~॥}\end{center} 
मम स्वभूता मदीया माया त्रिगुणात्मिका प्रकृतिः योनिः सर्वभूतानां कारणम्~। सर्वकार्येभ्यो महत्त्वात् भरणाच्च स्वविकाराणां महत् ब्रह्म इति योनिरेव विशिष्यते~। तस्मिन् महति ब्रह्मणि योनौ गर्भं हिरण्यगर्भस्य जन्मनः बीजं सर्वभूतजन्मकारणं बीजं दधामि निक्षिपामि क्षेत्रक्षेत्रज्ञप्रकृतिद्वयशक्तिमान् ईश्वरः अहम्~, अविद्याकामकर्मोपाधिस्वरूपानुविधायिनं क्षेत्रज्ञं क्षेत्रेण संयोजयामि इत्यर्थः~। सम्भवः उत्पत्तिः सर्वभूतानां हिरण्यगर्भोत्पत्तिद्वारेण ततः तस्मात् गर्भाधानात् भवति हे भारत~॥~३~॥\par
 \begin{center}{\bfseries सर्वयोनिषु कौन्तेय मूर्तयः सम्भवन्ति याः~।\\तासां ब्रह्म महद्योनिरहं बीजप्रदः पिता~॥~४~॥}\end{center} 
देवपितृमनुष्यपशुमृगादिसर्वयोनिषु कौन्तेय, मूर्तयः देहसंस्थानलक्षणाः मूर्छिताङ्गावयवाः मूर्तयः सम्भवन्ति याः, तासां मूर्तीनां ब्रह्म महत् सर्वावस्थं योनिः कारणम् अहम् ईश्वरः बीजप्रदः गर्भाधानस्य कर्ता पिता~॥~४~॥\par
 के गुणाः कथं बध्नन्तीति, उच्यते —} 
\begin{center}{\bfseries सत्त्वं रजस्तम इति गुणाः प्रकृतिसम्भवाः~।\\निबध्नन्ति महाबाहो देहे देहिनमव्ययम्~॥~५~॥}\end{center} 
सत्त्वं रजः तमः इति एवंनामानः~। गुणाः इति पारिभाषिकः शब्दः, न रूपादिवत् द्रव्याश्रिताः गुणाः~। न च गुणगुणिनोः अन्यत्वमत्र विवक्षितम्~। तस्मात् गुणा इव नित्यपरतन्त्राः क्षेत्रज्ञं प्रति अविद्यात्मकत्वात् क्षेत्रज्ञं निबध्नन्तीव~। तम् आस्पदीकृत्य आत्मानं प्रतिलभन्ते इति निबध्नन्ति इति उच्यते~। ते च प्रकृतिसम्भवाः भगवन्मायासम्भवाः निबध्नन्ति इव हे महाबाहो, महान्तौ समर्थतरौ आजानुप्रलम्बौ बाहू यस्य सः महाबाहुः, हे महाबाहो देहे शरीरे देहिनं देहवन्तम् अव्ययम्~, अव्ययत्वं च उक्तम् ‘अनादित्वात्’\footnote{भ. गी. १३~। ३१} इत्यादिश्लोकेन~। ननु ‘देही न लिप्यते’\footnote{भ. गी. १३~। ३१} इत्युक्तम्~। तत् कथम् इह निबध्नन्ति इति अन्यथा उच्यते~? परिहृतम् अस्माभिः इवशब्देन निबध्नन्ति इव इति~॥~५~॥\par
 तत्र सत्त्वादीनां सत्त्वस्यैव तावत् लक्षणम् उच्यते —} 
\begin{center}{\bfseries तत्र सत्त्वं निर्मलत्वात्प्रकाशकमनामयम्~।\\सुखसङ्गेन बध्नाति ज्ञानसङ्गेन चानघ~॥~६~॥}\end{center} 
निर्मलत्वात् स्फटिकमणिरिव प्रकाशकम् अनामयं निरुपद्रवं सत्त्वं तन्निबध्नाति~। कथम्~? सुखसङ्गेन ‘सुखी अहम्’ इति विषयभूतस्य सुखस्य विषयिणि आत्मनि संश्लेषापादनं मृषैव सुखे सञ्जनम् इति~। सैषा अविद्या~। न हि विषयधर्मः विषयिणः भवति~। इच्छादि च धृत्यन्तं क्षेत्रस्यैव विषयस्य धर्मः इति उक्तं भगवता~। अतः अविद्ययैव स्वकीयधर्मभूतया विषयविषय्यविवेकलक्षणया अस्वात्मभूते सुखे सञ्जयति इव, आसक्तमिव करोति, असङ्गं सक्तमिव करोति, असुखिनं सुखिनमिव~। तथा ज्ञानसङ्गेन च, ज्ञानमिति सुखसाहचर्यात् क्षेत्रस्यैव विषयस्य अन्तःकरणस्य धर्मः, न आत्मनः~; आत्मधर्मत्वे सङ्गानुपपत्तेः, बन्धानुपपत्तेश्च~। सुखे इव ज्ञानादौ सङ्गः मन्तव्यः~। हे अनघ अव्यसन~॥~६~॥\par
 \begin{center}{\bfseries रजो रागात्मकं विद्धि तृष्णासङ्गसमुद्भवम्~।\\तन्निबध्नाति कौन्तेय कर्मसङ्गेन देहिनम्~॥~७~॥}\end{center} 
रजः रागात्मकं रञ्जनात् रागः गैरिकादिवद्रागात्मकं विद्धि जानीहि~। तृष्णासङ्गसमुद्भवं तृष्णा अप्राप्ताभिलाषः, आसङ्गः प्राप्ते विषये मनसः प्रीतिलक्षणः संश्लेषः, तृष्णासङ्गयोः समुद्भवं तृष्णासङ्गसमुद्भवम्~। तन्निबध्नाति तत् रजः निबध्नाति कौन्तेय कर्मसङ्गेन, दृष्टादृष्टार्थेषु कर्मसु सञ्जनं तत्परता कर्मसङ्गः, तेन निबध्नाति रजः देहिनम्~॥~७~॥\par
 \begin{center}{\bfseries तमस्त्वज्ञानजं विद्धि मोहनं सर्वदेहिनाम्~।\\प्रमादालस्यनिद्राभिस्तन्निबध्नाति भारत~॥~८~॥}\end{center} 
तमः तृतीयः गुणः अज्ञानजम् अज्ञानात् जातम् अज्ञानजं विद्धि मोहनं मोहकरम् अविवेककरं सर्वदेहिनां सर्वेषां देहवताम्~। प्रमादालस्यनिद्राभिः प्रमादश्च आलस्यं च निद्रा च प्रमादालस्यनिद्राः ताभिः प्रमादालस्यनिद्राभिः तत् तमः निबध्नाति भारत~॥~८~॥\par
 पुनः गुणानां व्यापारः सङ्क्षेपतः उच्यते —} 
\begin{center}{\bfseries सत्त्वं सुखे सञ्जयति रजः कर्मणि भारत~।\\ज्ञानमावृत्य तु तमः प्रमादे सञ्जयत्युत~॥~९~॥}\end{center} 
सत्त्वं सुखे सञ्जयति संश्लेषयति, रजः कर्मणि हे भारत सञ्जयति इति अनुवर्तते~। ज्ञानं सत्त्वकृतं विवेकम् आवृत्य आच्छाद्य तु तमः स्वेन आवरणात्मना प्रमादे सञ्जयति उत प्रमादः नाम प्राप्तकर्तव्याकरणम्~॥~९~॥\par
 उक्तं कार्यं कदा कुर्वन्ति गुणा इति उच्यते —} 
\begin{center}{\bfseries रजस्तमश्चाभिभूय सत्त्वं भवति भारत~।\\रजः सत्त्वं तमश्चैव तमः सत्त्वं रजस्तथा~॥~१०~॥}\end{center} 
रजः तमश्च उभावपि अभिभूय सत्त्‌वं भवति उद्भवति वर्धते यदा, तदा लब्धात्मकं सत्त्वं स्वकार्यं ज्ञानसुखादि आरभते हे भारत~। तथा रजोगुणः सत्त्वं तमश्च एव उभावपि अभिभूय वर्धते यदा, तदा कर्म कृष्यादि स्वकार्यम् आरभते~। तमआख्यो गुणः सत्त्वं रजश्च उभावपि अभिभूय तथैव वर्धते यदा, तदा ज्ञानावरणादि स्वकार्यम् आरभते~॥~१०~॥\par
 यदा यो गुणः उद्भूतः भवति, तदा तस्य किं लिङ्गमिति उच्यते —} 
\begin{center}{\bfseries सर्वद्वारेषु देहेऽस्मिन्प्रकाश उपजायते~।\\ज्ञानं यदा तदा विद्याद्विवृद्धं सत्त्वमित्युत~॥~११~॥}\end{center} 
सर्वद्वारेषु, आत्मनः उपलब्धिद्वाराणि श्रोत्रादीनि सर्वाणि करणानि, तेषु सर्वद्वारेषु अन्तःकरणस्य बुद्धेः वृत्तिः प्रकाशः देहे अस्मिन् उपजायते~। तदेव ज्ञानम्~। यदा एवं प्रकाशो ज्ञानाख्यः उपजायते, तदा ज्ञानप्रकाशेन लिङ्गेन विद्यात् विवृद्धम् उद्भूतं सत्त्वम् इति उत अपि~॥~११~॥\par
 रजसः उद्भूतस्य इदं चिह्नम् —} 
\begin{center}{\bfseries लोभः प्रवृत्तिरारम्भः कर्मणामशमः स्पृहा~।\\रजस्येतानि जायन्ते विवृद्धे भरतर्षभ~॥~१२~॥}\end{center} 
लोभः परद्रव्यादित्सा, प्रवृत्तिः प्रवर्तनं सामान्यचेष्टा, आरम्भः~; कस्य~? कर्मणाम्~। अशमः अनुपशमः, हर्षरागादिप्रवृत्तिः, स्पृहा सर्वसामान्यवस्तुविषया तृष्णा — रजसि गुणे विवृद्धे एतानि लिङ्गानि जायन्ते हे भरतर्षभ~॥~१२~॥\par
 \begin{center}{\bfseries अप्रकाशोऽप्रवृत्तिश्च प्रमादो मोह एव च~।\\तमस्येतानि जायन्ते विवृद्धे कुरुनन्दन~॥~१३~॥}\end{center} 
अप्रकाशः अविवेकः, अत्यन्तम् अप्रवृत्तिश्च प्रवृत्त्यभावः तत्कार्यं प्रमादो मोह एव च अविवेकः मूढता इत्यर्थः~। तमसि गुणे विवृद्धे एतानि लिङ्गानि जायन्ते हे कुरुनन्दन~॥~१३~॥\par
 मरणद्वारेणापि यत् फलं प्राप्यते, तदपि सङ्गरागहेतुकं सर्वं गौणमेव इति दर्शयन् आह —} 
\begin{center}{\bfseries यदा सत्त्वे प्रवृद्धे तु प्रलयं याति देहभृत्~।\\तदोत्तमविदां लोकानमलान्प्रतिपद्यते~॥~१४~॥}\end{center} 
यदा सत्त्वे प्रवृद्धे उद्भूते तु प्रलयं मरणं याति प्रतिपद्यते देहभृत् आत्मा, तदा उत्तमविदां महदादितत्त्वविदाम् इत्येतत्~, लोकान् अमलान् मलरहितान् प्रतिपद्यते प्राप्नोति इत्येतत्~॥~१४~॥\par
 \begin{center}{\bfseries रजसि प्रलयं गत्वा कर्मसङ्गिषु जायते~।\\तथा प्रलीनस्तमसि मूढयोनिषु जायते~॥~१५~॥}\end{center} 
रजसि गुणे विवृद्धे प्रलयं मरणं गत्वा प्राप्य कर्मसङ्गिषु कर्मासक्तियुक्तेषु मनुष्येषु जायते~। तथा तद्वदेव प्रलीनः मृतः तमसि विवृद्धे मूढयोनिषु पश्वादियोनिषु जायते~॥~१५~॥\par
 अतीतश्लोकार्थस्यैव सङ्क्षेपः उच्यते —} 
\begin{center}{\bfseries कर्मणः सुकृतस्याहुः सात्त्विकं निर्मलं फलम्~।\\रजसस्तु फलं दुःखमज्ञानं तमसः फलम्~॥~१६~॥}\end{center} 
कर्मणः सुकृतस्य सात्त्विकस्य इत्यर्थः, आहुः शिष्टाः सात्त्विकम् एव निर्मलं फलम् इति~। रजसस्तु फलं दुःखं राजसस्य कर्मणः इत्यर्थः, कर्माधिकारात् फलम् अपि दुःखम् एव, कारणानुरूप्यात् राजसमेव~। तथा अज्ञानं तमसः तामसस्य कर्मणः अधर्मस्य पूर्ववत्~॥~१६~॥\par
 किञ्च गुणेभ्यो भवति —} 
\begin{center}{\bfseries सत्त्वात्सञ्जायते ज्ञानं रजसो लोभ एव च~।\\प्रमादमोहौ तमसो भवतोऽज्ञानमेव च~॥~१७~॥}\end{center} 
सत्त्वात् लब्धात्मकात् सञ्जायते समुत्पद्यते ज्ञानम्~, रजसो लोभ एव च, प्रमादमोहौ च उभौ तमसो भवतः, अज्ञानमेव च भवति~॥~१७~॥\par
 किञ्च —} 
\begin{center}{\bfseries ऊर्ध्वं गच्छन्ति सत्त्वस्था\\ मध्ये तिष्ठन्ति राजसाः~।\\जघन्यगुणवृत्तस्था\\ अधो गच्छन्ति तामसाः~॥~१८~॥}\end{center} 
ऊर्ध्वं गच्छन्ति देवलोकादिषु उत्पद्यन्ते सत्त्वस्थाः सत्त्वगुणवृत्तस्थाः~। मध्ये तिष्ठन्ति मनुष्येषु उत्पद्यन्ते राजसाः~। जघन्यगुणवृत्तस्थाः जघन्यश्च असौ गुणश्च जघन्यगुणः तमः, तस्य वृत्तं निद्रालस्यादि, तस्मिन् स्थिताः जघन्यगुणवृत्तस्थाः मूढाः अधः गच्छन्ति पश्वादिषु उत्पद्यन्ते तामसाः~॥~१८~॥\par
 पुरुषस्य प्रकृतिस्थत्वरूपेण मिथ्याज्ञानेन युक्तस्य भोग्येषु गुणेषु सुखदुःखमोहात्मकेषु ‘सुखी दुःखी मूढः अहम् अस्मि’ इत्येवंरूपः यः सङ्गः तत्कारणं पुरुषस्य सदसद्योनिजन्मप्राप्तिलक्षणस्य संसारस्य इति समासेन पूर्वाध्याये यत् उक्तम्~, तत् इह ‘सत्त्वं रजस्तम इति गुणाः प्रकृतिसम्भवाः’\footnote{भ. गी. १४~। ५} इति आरभ्य गुणस्वरूपम्~, गुणवृत्तम्~, स्ववृत्तेन च गुणानां बन्धकत्वम्~, गुणवृत्तनिबद्धस्य च पुरुषस्य या गतिः, इत्येतत् सर्वं मिथ्याज्ञानमूलं बन्धकारणं विस्तरेण उक्त्वा, अधुना सम्यग्दर्शनान्मोक्षो वक्तव्यः इत्यत आह भगवान् —} 
\begin{center}{\bfseries नान्यं गुणेभ्यः कर्तारं यदा द्रष्टानुपश्यति~।\\गुणेभ्यश्च परं वेत्ति मद्भावं सोऽधिगच्छति~॥~१९~॥}\end{center} 
न अन्यं कार्यकरणविषयाकारपरिणतेभ्यः गुणेभ्यः कर्तारम् अन्यं यदा द्रष्टा विद्वान् सन् न अनुपश्यति, गुणा एव सर्वावस्थाः सर्वकर्मणां कर्तारः इत्येवं पश्यति, गुणेभ्यश्च परं गुणव्यापारसाक्षिभूतं वेत्ति, मद्भावं मम भावं सः द्रष्टा अधिगच्छति~॥~१९~॥\par
 कथम् अधिगच्छति इति, उच्यते —} 
\begin{center}{\bfseries गुणानेतानतीत्य त्रीन्देही देहसमुद्भवान्~।\\जन्ममृत्युजरादुःखैर्विमुक्तोऽमृतमश्नुते~॥~२०~॥}\end{center} 
गुणान् एतान् यथोक्तान् अतीत्य जीवन्नेव अतिक्रम्य मायोपाधिभूतान् त्रीन् देही देहसमुद्भवान् देहोत्पत्तिबीजभूतान् जन्ममृत्युजरादुःखैः जन्म च मृत्युश्च जरा च दुःखानि च जन्ममृत्युजरादुःखानि तैः जीवन्नेव विमुक्तः सन् विद्वान् अमृतम् अश्नुते, एवं मद्भावम् अधिगच्छति इत्यर्थः~॥~२०~॥\par
 जीवन्नेव गुणान् अतीत्य अमृतम् अश्नुते इति प्रश्नबीजं प्रतिलभ्य, अर्जुन उवाच —}\\ 
\begin{center}{\bfseries अर्जुन उवाच —\\ कैर्लिङ्गैस्त्रीन्गुणानेतानतीतो भवति प्रभो~।\\किमाचारः कथं चैतांस्त्रीन्गुणानतिवर्तते~॥~२१~॥}\end{center} 
कैः लिङ्गैः चिह्नैः त्रीन् एतान् व्याख्यातान् गुणान् अतीतः अतिक्रान्तः भवति प्रभो, किमाचारः कः अस्य आचारः इति किमाचारः कथं केन च प्रकारेण एतान् त्रीन् गुणान् अतिवर्तते अतीत्य वर्तते~॥~२१~॥\par
 गुणातीतस्य लक्षणं गुणातीतत्वोपायं च अर्जुनेन पृष्टः अस्मिन् श्लोके प्रश्नद्वयार्थं प्रतिवचनं भगवान् उवाच~। यत् तावत् ‘कैः लिङ्गैः युक्तो गुणातीतो भवति’ इति, तत् शृणु —} 
\begin{center}{\bfseries श्रीभगवानुवाच —\\ प्रकाशं च प्रवृत्तिं च मोहमेव च पाण्डव~।\\न द्वेष्टि सम्प्रवृत्तानि न निवृत्तानि काङ्क्षति~॥~२२~॥}\end{center} 
प्रकाशं च सत्त्वकार्यं प्रवृत्तिं च रजःकार्यं मोहमेव च तमःकार्यम् इत्येतानि न द्वेष्टि सम्प्रवृत्तानि सम्यग्विषयभावेन उद्भूतानि — ‘मम तामसः प्रत्ययो जातः, तेन अहं मूढः~; तथा राजसी प्रवृत्तिः मम उत्पन्ना दुःखात्मिका, तेन अहं रजसा प्रवर्तितः प्रचलितः स्वरूपात्~; कष्टं मम वर्तते यः अयं मत्स्वरूपावस्थानात् भ्रंशः~; तथा सात्त्विको गुणः प्रकाशात्मा मां विवेकित्वम् आपादयन् सुखे च सञ्जयन् बध्नाति’ इति तानि द्वेष्टि असम्यग्दर्शित्वेन~। तत् एवं गुणातीतो न द्वेष्टि सम्प्रवृत्तानि~। यथा च सात्त्विकादिपुरुषः सत्त्वादिकार्याणि आत्मानं प्रति प्रकाश्य निवृत्तानि काङ्क्षति, न तथा गुणातीतो निवृत्तानि काङ्क्षति इत्यर्थः~। एतत् न परप्रत्यक्षं लिङ्गम्~। किं तर्हि~? स्वात्मप्रत्यक्षत्वात् आत्मार्थमेव एतत् लक्षणम्~। न हि स्वात्मविषयं द्वेषमाकाङ्क्षां वा परः पश्यति~॥~२२~॥\par
 अथ इदानीम् ‘गुणातीतः किमाचारः~? ’ इति प्रश्नस्य प्रतिवचनम् आह —} 
\begin{center}{\bfseries उदासीनवदासीनो गुणैर्यो न विचाल्यते~।\\गुणा वर्तन्त इत्येव योऽवतिष्ठति नेङ्गते~॥~२३~॥}\end{center} 
उदासीनवत् यथा उदासीनः न कस्यचित् पक्षं भजते, तथा अयं गुणातीतत्वोपायमार्गेऽवस्थितः आसीनः आत्मवित् गुणैः यः संन्यासी न विचाल्यते विवेकदर्शनावस्थातः~। तदेतत् स्फुटीकरोति — गुणाः कार्यकरणविषयाकारपरिणताः अन्योऽन्यस्मिन् वर्तन्ते इति यः अवतिष्ठति~। छन्दोभङ्गभयात् परस्मैपदप्रयोगः~। योऽनुतिष्ठतीति वा पाठान्तरम्~। न इङ्गते न चलति, स्वरूपावस्थ एव भवति इत्यर्थः~॥~२३~॥\par
 किञ्च —} 
\begin{center}{\bfseries समदुःखसुखः स्वस्थः समलोष्टाश्मकाञ्चनः~।\\तुल्यप्रियाप्रियो धीरस्तुल्यनिन्दात्मसंस्तुतिः~॥~२४~॥}\end{center} 
समदुःखसुखः समे दुःखसुखे यस्य सः समदुःखसुखः, स्वस्थः स्वे आत्मनि स्थितः प्रसन्नः, समलोष्टाश्मकाञ्चनः लोष्टं च अश्मा च काञ्चनं च लोष्टाश्मकाञ्चनानि समानि यस्य सः समलोष्टाश्मकाञ्चनः, तुल्यप्रियाप्रियः प्रियं च अप्रियं च प्रियाप्रिये तुल्ये समे यस्य सोऽयं तुल्यप्रियाप्रियः, धीरः धीमान्~, तुल्यनिन्दात्मसंस्तुतिः निन्दा च आत्मसंस्तुतिश्च निन्दात्मसंस्तुती, तुल्ये निन्दात्मसंस्तुती यस्य यतेः सः तुल्यनिन्दात्मसंस्तुतिः~॥~२४~॥\par
 किञ्च —} 
\begin{center}{\bfseries मानापमानयोस्तुल्यस्तुल्यो मित्रारिपक्षयोः~।\\सर्वारम्भपरित्यागी गुणातीतः स उच्यते~॥~२५~॥}\end{center} 
मानापमानयोः तुल्यः समः निर्विकारः~; तुल्यः मित्रारिपक्षयोः, यद्यपि उदासीना भवन्ति केचित् स्वाभिप्रायेण, तथापि पराभिप्रायेण मित्रारिपक्षयोरिव भवन्ति इति तुल्यो मित्रारिपक्षयोः इत्याह~। सर्वारम्भपरित्यागी, दृष्टादृष्टार्थानि कर्माणि आरभ्यन्ते इति आरम्भाः, सर्वान् आरम्भान् परित्यक्तुं शीलम् अस्य इति सर्वारम्भपरित्यागी, देहधारणमात्रनिमित्तव्यतिरेकेण सर्वकर्मपरित्यागी इत्यर्थः~। गुणातीतः सः उच्यते ‘उदासीनवत्’\footnote{भ. गी. १४~। २३} इत्यादि ‘गुणातीतः स उच्यते’\footnote{भ. गी. १४~। २५} इत्येतदन्तम् उक्तं यावत् यत्नसाध्यं तावत् संन्यासिनः अनुष्ठेयं गुणातीतत्वसाधनं मुमुक्षोः~; स्थिरीभूतं तु स्वसंवेद्यं सत् गुणातीतस्य यतेः लक्षणं भवति इति~।
॥~२५~॥\par
 
 अधुना ‘कथं च त्रीन्गुणानतिवर्तते~? ’\footnote{भ. गी. १४~। २१} इत्यस्य प्रश्नस्य प्रतिवचनम् आह —} 
\begin{center}{\bfseries मां च योऽव्यभिचारेण भक्तियोगेन सेवते~।\\स गुणान्समतीत्यैतान्ब्रह्मभूयाय कल्पते~॥~२६~॥}\end{center} 
मां च ईश्वरं नारायणं सर्वभूतहृदयाश्रितं यो यतिः कर्मी वा अव्यभिचारेण न कदाचित् यो व्यभिचरति भक्तियोगेन भजनं भक्तिः सैव योगः तेन भक्तियोगेन सेवते, सः गुणान् समतीत्य एतान् यथोक्तान् ब्रह्मभूयाय, भवनं भूयः, ब्रह्मभूयाय ब्रह्मभवनाय मोक्षाय कल्पते समर्थो भवति इत्यर्थः~॥~२६~॥\par
 कुत एतदिति उच्यते —} 
\begin{center}{\bfseries ब्रह्मणो हि प्रतिष्ठाहममृतस्याव्ययस्य च~।\\शाश्वतस्य च धर्मस्य सुखस्यैकान्तिकस्य च~॥~२७~॥}\end{center} 
ब्रह्मणः परमात्मनः हि यस्मात् प्रतिष्ठा अहं प्रतितिष्ठति अस्मिन् इति प्रतिष्ठा अहं प्रत्यगात्मा~। कीदृशस्य ब्रह्मणः~? अमृतस्य अविनाशिनः अव्ययस्य अविकारिणः शाश्वतस्य च नित्यस्य धर्मस्य धर्मज्ञानस्य ज्ञानयोगधर्मप्राप्यस्य सुखस्य आनन्दरूपस्य ऐकान्तिकस्य अव्यभिचारिणः अमृतादिस्वभावस्य परमानन्दरूपस्य परमात्मनः प्रत्यगात्मा प्रतिष्ठा, सम्यग्ज्ञानेन परमात्मतया निश्चीयते~। तदेतत् ‘ब्रह्मभूयाय कल्पते’\footnote{भ. गी. १४~। २६} इति उक्तम्~। यया च ईश्वरशक्त्या भक्तानुग्रहादिप्रयोजनाय ब्रह्म प्रतिष्ठते प्रवर्तते, सा शक्तिः ब्रह्मैव अहम्~, शक्तिशक्तिमतोः अनन्यत्वात् इत्यभिप्रायः~। अथवा, ब्रह्मशब्दवाच्यत्वात् सविकल्पकं ब्रह्म~। तस्य ब्रह्मणो निर्विकल्पकः अहमेव नान्यः प्रतिष्ठा आश्रयः~। किंविशिष्टस्य~? अमृतस्य अमरणधर्मकस्य अव्ययस्य व्ययरहितस्य~। किञ्च, शाश्वतस्य च नित्यस्य धर्मस्य ज्ञाननिष्ठालक्षणस्य सुखस्य तज्जनितस्य ऐकान्तिकस्य एकान्तनियतस्य च, ‘प्रतिष्ठा अहम्’ इति वर्तते~॥~२७~॥\par
 
इति श्रीमत्परमहंसपरिव्राजकाचार्यस्य श्रीगोविन्दभगवत्पूज्यपादशिष्यस्य श्रीमच्छङ्करभगवतः कृतौ श्रीमद्भगवद्गीताभाष्ये चतुर्दशोऽध्यायः~॥\par
 
यस्मात् मदधीनं कर्मिणां कर्मफलं ज्ञानिनां च ज्ञानफलम्~, अतः भक्तियोगेन मां ये सेवंते ते मम प्रसादात् ज्ञानप्राप्तिक्रमेण गुणातीताः मोक्षं गच्छन्ति~। किमु वक्तव्यम् आत्मनः तत्त्वमेव सम्यक् विजानन्तः इति अतः भगवान् अर्जुनेन अपृष्टोऽपि आत्मनः तत्त्वं विवक्षुः उवाच ‘ऊर्ध्वमूलम्’ इत्यादिना~। तत्र तावत् वृक्षरूपककल्पनया वैराग्यहेतोः संसारस्वरूपं वर्णयति — विरक्तस्य हि संसारात् भगवत्तत्त्वज्ञाने अधिकारः, न अन्यस्येति~॥~} 
\begin{center}{\bfseries श्रीभगवानुवाच —\\ऊर्ध्वमूलमधःशाखमश्वत्थं प्राहुरव्ययम्~।\\छन्दांसि यस्य पर्णानि यस्तं वेद स वेदवित्~॥~१~॥}\end{center} 
ऊर्ध्वमूलं कालतः सूक्ष्मत्वात् कारणत्वात् नित्यत्वात् महत्त्वाच्च ऊर्ध्वम्~; उच्यते ब्रह्म अव्यक्तं मायाशक्तिमत्~, तत् मूलं अस्येति सोऽयं संसारवृक्षः ऊर्ध्वमूलः~। श्रुतेश्च — ‘ऊर्ध्वमूलोऽवाक्शाख एषोऽश्वत्थः सनातनः’\footnote{क. उ. २~। ३~। १} इति~। पुराणे च —} 
‘अव्यक्तमूलप्रभवस्तस्यैवानुग्रहोच्छ्रितः~। बुद्धिस्कन्धमयश्चैव इन्द्रियान्तरकोटरः~॥\\ महाभूतविशाखश्च विषयैः पत्रवांस्तथा~। धर्माधर्मसुपुष्पश्च सुखदुःखफलोदयः~॥\\ आजीव्यः सर्वभूतानां ब्रह्मवृक्षः सनातनः~। एतद्ब्रह्मवनं चैव ब्रह्माचरति नित्यशः~॥\\ एतच्छित्त्वा च भित्त्वा च ज्ञानेन परमासिना~। ततश्चात्मरतिं प्राप्य तस्मान्नावर्तते पुनः~॥~’ 
 इत्यादि~। तम् ऊर्ध्वमूलं संसारं मायामयं वृक्षम् अधःशाखं महदहङ्कारतन्मात्रादयः शाखा इव अस्य अधः भवन्तीति सोऽयं अधःशाखः, तम् अधःशाखम् न श्वोऽपि स्थाता इति अश्वत्थः तं क्षणप्रध्वंसिनम् अश्वत्थं प्राहुः कथयन्ति ~।\\ } 
 अव्ययं संसारमायायाः अनादिकालप्रवृत्तत्वात् सोऽयं संसारवृक्षः अव्ययः, अनाद्यन्तदेहादिसन्तानाश्रयः हि सुप्रसिद्धः, तम् अव्ययम्~। तस्यैव संसारवृक्षस्य इदम् अन्यत् विशेषणम् — छन्दांसि यस्य पर्णानि, छन्दांसि च्छादनात् ऋग्यजुःसामलक्षणानि यस्य संसारवृक्षस्य पर्णानीव पर्णानि~। यथा वृक्षस्य परिरक्षणार्थानि पर्णानि, तथा वेदाः संसारवृक्षपरिरक्षणार्थाः, धर्माधर्मतद्धेतुफलप्रदर्शनार्थत्वात्~। यथाव्याख्यातं संसारवृक्षं समूलं यः तं वेद सः वेदवित्~, वेदार्थवित् इत्यर्थः~। न हि समूलात् संसारवृक्षात् अस्मात् ज्ञेयः अन्यः अणुमात्रोऽपि अवशिष्टः अस्ति इत्यतः सर्वज्ञः सर्ववेदार्थविदिति समूलसंसारवृक्षज्ञानं स्तौति~॥~१~॥\par
 तस्य एतस्य संसारवृक्षस्य अपरा अवयवकल्पना उच्यते —} 
\begin{center}{\bfseries अधश्चोर्ध्वं प्रसृतास्तस्य शाखा\\ गुणप्रवृद्धा विषयप्रवालाः~।\\अधश्च मूलान्यनुसन्ततानि\\ कर्मानुबन्धीनि मनुष्यलोके~॥~२~॥}\end{center} 
अधः मनुष्यादिभ्यो यावत् स्थावरम् ऊर्ध्वं च यावत् ब्रह्मणः विश्वसृजो धाम इत्येतदन्तं यथाकर्म यथाश्रुतं ज्ञानकर्मफलानि, तस्य वृक्षस्य शाखा इव शाखाः प्रसृताः प्रगताः, गुणप्रवृद्धाः गुणैः सत्त्वरजस्तमोभिः प्रवृद्धाः स्थूलीकृताः उपादानभूतैः, विषयप्रवालाः विषयाः शब्दादयः प्रवालाः इव देहादिकर्मफलेभ्यः शाखाभ्यः अङ्कुरीभवन्तीव, तेन विषयप्रवालाः शाखाः~। संसारवृक्षस्य परममूलं उपादानकारणं पूर्वम् उक्तम्~। अथ इदानीं कर्मफलजनितरागद्वेषादिवासनाः मूलानीव धर्माधर्मप्रवृत्तिकारणानि अवान्तरभावीनि तानि अधश्च देवाद्यपेक्षया मूलानि अनुसन्ततानि अनुप्रविष्टानि कर्मानुबन्धीनि कर्म धर्माधर्मलक्षणम् अनुबन्धः पश्चाद्भावि, येषाम् उद्भूतिम् अनु उद्भवति, तानि कर्मानुबन्धीनि मनुष्यलोके विशेषतः~। अत्र हि मनुष्याणां कर्माधिकारः प्रसिद्धः~॥~२~॥\par
 यस्तु अयं वर्णितः संसारवृक्षः —} 
\begin{center}{\bfseries न रूपमस्येह तथोपलभ्यते\\ नान्तो न चादिर्न च सम्प्रतिष्ठा~।\\अश्वत्थमेनं सुविरूढमूल—\\ मसङ्गशस्त्रेण दृढेन छित्त्वा~॥~३~॥}\end{center} 
न रूपम् अस्य इह यथा उपवर्णितं तथा नैव उपलभ्यते, स्वप्नमरीच्युदकमायागन्धर्वनगरसमत्वात्~; दृष्टनष्टस्वरूपो हि स इति अत एव न अन्तः न पर्यन्तः निष्ठा परिसमाप्तिर्वा विद्यते~। तथा न च आदिः, ‘इतः आरभ्य अयं प्रवृत्तः’ इति न केनचित् गम्यते~। न च सम्प्रतिष्ठा स्थितिः मध्यम् अस्य न केनचित् उपलभ्यते~। अश्वत्थम् एनं यथोक्तं सुविरूढमूलं सुष्ठु विरूढानि विरोहं गतानि सुदृढानि मूलानि यस्य तम् एनं सुविरूढमूलम्~, असङ्गशस्त्रेण असङ्गः पुत्रवित्तलोकैषणाभ्यः व्युत्थानं तेन असङ्गशस्त्रेण दृढेन परमात्माभिमुख्यनिश्चयदृढीकृतेन पुनः पुनः विवेकाभ्यासाश्मनिशितेन च्छित्वा संसारवृक्षं सबीजम् उद्धृत्य~॥~३~॥\par
 \begin{center}{\bfseries ततः पदं तत्परिमार्गितव्यं\\ यस्मिन्गता न निवर्तन्ति भूयः~।\\तमेव चाद्यं पुरुषं प्रपद्ये\\ यतः प्रवृत्तिः प्रसृता पुराणी~॥~४~॥}\end{center} 
ततः पश्चात् यत् पदं वैष्णवं तत् परिमार्गितव्यम्~, परिमार्गणम् अन्वेषणं ज्ञातव्यमित्यर्थः~। यस्मिन् पदे गताः प्रविष्टाः न निवर्तन्ति न आवर्तन्ते भूयः पुनः संसाराय~। कथं परिमार्गितव्यमिति आह — तमेव च यः पदशब्देन उक्तः आद्यम् आदौ भवम् आद्यं पुरुषं प्रपद्ये इत्येवं परिमार्गितव्यं तच्छरणतया इत्यर्थः~। कः असौ पुरुषः इति, उच्यते — यतः यस्मात् पुरुषात् संसारमायावृक्षप्रवृत्तिः प्रसृता निःसृता, ऐन्द्रजालिकादिव माया, पुराणी चिरन्तनी~॥~४~॥\par
 कथम्भूताः तत् पदं गच्छन्तीति, उच्यते —} 
\begin{center}{\bfseries निर्मानमोहा जितसङ्गदोषा\\ अध्यात्मनित्या विनिवृत्तकामाः~।\\द्वन्द्वैर्विमुक्ताः सुखदुःखसंज्ञै—\\ र्गच्छन्त्यमूढाः पदमव्ययं तत्~॥~५~॥}\end{center} 
निर्मानमोहाः मानश्च मोहश्च मानमोहौ, तौ निर्गतौ येभ्यः ते निर्मानमोहाः मानमोहवर्जिताः~। जितसङ्गदोषाः सङ्ग एव दोषः सङ्गदोषः, जितः सङ्गदोषः यैः ते जितसङ्गदोषाः~। अध्यात्मनित्याः परमात्मस्वरूपालोचननित्याः तत्पराः~। विनिवृत्तकामाः विशेषतो निर्लेपेन निवृत्ताः कामाः येषां ते विनिवृत्तकामाः यतयः संन्यासिनः द्वन्द्वैः प्रियाप्रियादिभिः विमुक्ताः सुखदुःखसंज्ञैः परित्यक्ताः गच्छन्ति अमूढाः मोहवर्जिताः पदम् अव्ययं तत् यथोक्तम्~॥~५~॥\par
 तदेव पदं पुनः विशेष्यते —} 
\begin{center}{\bfseries न तद्भासयते सूर्यो न शशाङ्को न पावकः~।\\यद्गत्वा न निवर्तन्ते तद्धाम परमं मम~॥~६~॥}\end{center} 
तत् धाम इति व्यवहितेन धाम्ना सम्बध्यते~। तत् धाम तेजोरूपं पदं न भासयते सूर्यः आदित्यः सर्वावभासनशक्तिमत्त्वेऽपि सति~। तथा न शशाङ्कः चन्द्रः, न पावकः न अग्निरपि~। यत् धाम वैष्णवं पदं गत्वा प्राप्य न निवर्तन्ते, यच्च सूर्यादिः न भासयते, तत् धाम पदं परमं विष्णोः मम पदम्~, ~॥~६~॥\par
 यत् गत्वा न निवर्तन्ते इत्युक्तम्ननु सर्वा हि गतिः आगत्यन्ता, ‘संयोगाः विप्रयोगान्ताः’ इति प्रसिद्धम्~। कथम् उच्यते ‘तत् धाम गतानां नास्ति निवृत्तिः’ इति~? शृणु तत्र कारणम् —} 
\begin{center}{\bfseries ममैवांशो जीवलोके जीवभूतः सनातनः~।\\मनःषष्ठानीन्द्रियाणि प्रकृतिस्थानि कर्षति~॥~७~॥}\end{center} 
ममैव परमात्मनः नारायणस्य, अंशः भागः अवयवः एकदेशः इति अनर्थान्तरं जिवलोके जीवानां लोके संसारे जीवभूतः कर्ता भोक्ता इति प्रसिद्धः सनातनः चिरन्तनः~; यथा जलसूर्यकः सूर्यांशः जलनिमित्तापाये सूर्यमेव गत्वा न निवर्तते च तेनैव आत्मना गच्छति, एवमेव~; यथा घटाद्युपाधिपरिच्छिन्नो घटाद्याकाशः आकाशांशः सन् घटादिनिमित्तापाये आकाशं प्राप्य न निवर्तते~। अतः उपपन्नम् उक्तम् ‘यद्गत्वा न निवर्तन्ते’\footnote{भ. गी. १५~। ६} इति~। ननु निरवयवस्य परमात्मनः कुतः अवयवः एकदेशः अंशः इति~? सावयवत्वे च विनाशप्रसङ्गः अवयवविभागात्~। नैष दोषः, अविद्याकृतोपाधिपरिच्छिन्नः एकदेशः अंश इव कल्पितो यतः~। दर्शितश्च अयमर्थः क्षेत्राध्याये विस्तरशः~। स च जीवो मदंशत्वेन कल्पितः कथं संसरति उत्क्रामति च इति, उच्यते — मनःषष्ठानि इन्द्रियाणि श्रोत्रादीनि प्रकृतिस्थानि स्वस्थाने कर्णशष्कुल्यादौ प्रकृतौ स्थितानि कर्षति आकर्षति~॥~७~॥\par
 कस्मिन् काले~? —} 
\begin{center}{\bfseries शरीरं यदवाप्नोति यच्चाप्युत्क्रामतीश्वरः~।\\गृहीत्वैतानि संयाति वायुर्गन्धानिवाशयात्~॥~८~॥}\end{center} 
यच्चापि यदा चापि उत्क्रामति ईश्वरः देहादिसङ्घातस्वामी जीवः, तदा ‘कर्षति’ इति श्लोकस्य द्वितीयपादः अर्थवशात् प्राथम्येन सम्बध्यते~। यदा च पूर्वस्मात् शरीरात् शरीरान्तरम् अवाप्नोति तदा गृहीत्वा एतानि मनःषष्ठानि इन्द्रियाणि संयाति सम्यक् याति गच्छति~। किमिव इति, आह — वायुः पवनः गन्धानिव आशयात् पुष्पादेः~॥~८~॥\par
 कानि पुनः तानि —} 
\begin{center}{\bfseries श्रोत्रं चक्षुः स्पर्शनं च रसनं घ्राणमेव च~।\\अधिष्ठाय मनश्चायं विषयानुपसेवते~॥~९~॥}\end{center} 
श्रोत्रं चक्षुः स्पर्शनं च त्वगिन्द्रियं रसनं घ्राणमेव च मनश्च षष्ठं प्रत्येकम् इन्द्रियेण सह, अधिष्ठाय देहस्थः विषयान् शब्दादीन् उपसेवते~॥~९~॥\par
 एवं देहगतं देहात् —} 
\begin{center}{\bfseries उत्क्रामन्तं स्थितं वापि भुञ्जानं वा गुणान्वितम्~।\\विमूढा नानुपश्यन्ति पश्यन्ति ज्ञानचक्षुषः~॥~१०~॥}\end{center} 
उत्क्रामन्तं देहं पूर्वोपात्तं परित्यजन्तं स्थितं वापि देहे तिष्ठन्तं भुञ्जानं वा शब्दादींश्च उपलभमानं गुणान्वितं सुखदुःखमोहाद्यैः गुणैः अन्वितम् अनुगतं संयुक्तमित्यर्थः~। एवंभूतमपि एनम् अत्यन्तदर्शनगोचरप्राप्तं विमूढाः दृष्टादृष्टविषयभोगबलाकृष्टचेतस्तया अनेकधा मूढाः न अनुपश्यन्ति — अहो कष्टं वर्तते इति अनुक्रोशति च भगवान् — ये तु पुनः प्रमाणजनितज्ञानचक्षुषः ते एनं पश्यन्ति ज्ञानचक्षुषः विविक्तदृष्टयः इत्यर्थः~॥~१०~॥\par
 केचित्तु —} 
\begin{center}{\bfseries यतन्तो योगिनश्चैनं पश्यन्त्यात्मन्यवस्थितम्~।\\यतन्तोऽप्यकृतात्मानो नैनं पश्यन्त्यचेतसः~॥~११~॥}\end{center} 
यतन्तः प्रयत्नं कुर्वन्तः योगिनश्च समाहितचित्ताः एनं प्रकृतम् आत्मानं पश्यन्ति ‘अयम् अहम् अस्मि’ इति उपलभन्ते आत्मनि स्वस्यां बुद्धौ अवस्थितम्~। यतन्तोऽपि शास्त्रादिप्रमाणैः, अकृतात्मानः असंस्कृतात्मानः तपसा इन्द्रियजयेन च, दुश्चरितात् अनुपरताः, अशान्तदर्पाः, प्रयत्नं कुर्वन्तोऽपि न एवं पश्यन्ति अचेतसः अविवेकिनः~॥~११~॥\par
 यत् पदं सर्वस्य अवभासकमपि अग्न्यादित्यादिकं ज्योतिः न अवभासयते, यत् प्राप्ताश्च मुमुक्षवः पुनः संसाराभिमुखाः न निवर्तन्ते, यस्य च पदस्य उपाधिभेदम् अनुविधीयमानाः जीवाः — घटाकाशादयः इव आकाशस्य — अंशाः, तस्य पदस्य सर्वात्मत्वं सर्वव्यवहारास्पदत्वं च विवक्षुः चतुर्भिः श्लोकैः विभूतिसङ्क्षेपमाह भगवान् —} 
\begin{center}{\bfseries यदादित्यगतं तेजो जगद्भासयतेऽखिलम्~।\\यच्चन्द्रमसि यच्चाग्नौ तत्तेजो विद्धि मामकम्~॥~१२~॥}\end{center} 
यत् आदित्यगतम् आदित्याश्रयम्~। किं तत्~? तेजः दीप्तिः प्रकाशः जगत् भासयते प्रकाशयति अखिलं समस्तम्~; यत् चन्द्रमसि शशभृति तेजः अवभासकं वर्तते, यच्च अग्नौ हुतवहे, तत् तेजः विद्धि विजानीहि मामकं मदीयं मम विष्णोः तत् ज्योतिः~। अथवा, आदित्यगतं तेजः चैतन्यात्मकं ज्योतिः, यच्चन्द्रमसि, यच्च अग्नौ वर्तते तत् तेजः विद्धि मामकं मदीयं मम विष्णोः तत् ज्योतिः~॥~} 
ननु स्थावरेषु जङ्गमेषु च तत् समानं चैतन्यात्मकं ज्योतिः~। तत्र कथम् इदं विशेषणम् — ‘यदादित्यगतम्’ इत्यादि~। नैष दोषः, सत्त्वाधिक्यात् आविस्तरत्वोपपत्तेः~। आदित्यादिषु हि सत्त्वं अत्यन्तप्रकाशम् अत्यन्तभास्वरम्~; अतः तत्रैव आविस्तरं ज्योतिः इति तत् विशिष्यते, न तु तत्रैव तत् अधिकमिति~। यथा हि श्लोके तुल्येऽपि मुखसंस्थाने न काष्ठकुड्यादौ मुखम् आविर्भवति, आदर्शादौ तु स्वच्छे स्वच्छतरे च तारतम्येन आविर्भवति~; तद्वत्~॥~१२~॥\par
 किञ्च —} 
\begin{center}{\bfseries गामाविश्य च भूतानि\\ धारयाम्यहमोजसा~।\\पुष्णामि चौषधीः सर्वाः\\ सोमो भूत्वा रसात्मकः~॥~१३~॥}\end{center} 
गां पृथिवीम् आविश्य प्रविश्य धारयामि भूतानि जगत् अहम् ओजसा बलेन~; यत् बलं कामरागविवर्जितम् ऐश्वरं रूपं जगद्विधारणाय पृथिव्याम् आविष्टं येन पृथिवी गुर्वी न अधः पतति न विदीर्यते च~। तथा च मन्त्रवर्णः — ‘येन द्यौरुग्रा पृथिवी च दृढा’\footnote{तै. सं. ४~। १~। ८} इति, ‘स दाधार पृथिवीम्’\footnote{तै. सं. ४~। १~। ८} इत्यादिश्च~। अतः गामाविश्य च भूतानि चराचराणि धारयामि इति युक्तमुक्तम्~। किञ्च, पृथिव्यां जाताः ओषधीः सर्वाः व्रीहियवाद्याः पुष्णामि पुष्टिमतीः रसस्वादुमतीश्च करोमि सोमो भूत्वा रसात्मकः सोमः सन् रसात्मकः रसस्वभावः~। सर्वरसानाम् आकरः सोमः~। स हि सर्वरसात्मकः सर्वाः ओषधीः स्वात्मरसान् अनुप्रवेशयन् पुष्णाति~॥~१३~॥\par
 किञ्च —} 
\begin{center}{\bfseries अहं वैश्वानरो भूत्वा प्राणिनां देहमाश्रितः~।\\प्राणापानसमायुक्तः पचाम्यन्नं चतुर्विधम्~॥~१४~॥}\end{center} 
अहमेव वैश्वानरः उदरस्थः अग्निः भूत्वा — ‘अयमग्निर्वैश्वानरो योऽयमन्तः पुरुषे येनेदमन्नं पच्यते’\footnote{बृ. उ. ५~। ९~। १} इत्यादिश्रुतेः~; वैश्वानरः सन् प्राणिनां प्राणवतां देहम् आश्रितः प्रविष्टः प्राणापानसमायुक्तः प्राणापानाभ्यां समायुक्तः संयुक्तः पचामि पक्तिं करोमि अन्नम् अशनं चतुर्विधं चतुष्प्रकारं भोज्यं भक्ष्यं चोष्यं लेह्यं च~। ‘भोक्ता वैश्वानरः अग्निः, अग्नेः भोज्यम् अन्नं सोमः, तदेतत् उभयम् अग्नीषोमौ सर्वम्’ इति पश्यतः अन्नदोषलेपः न भवति~॥~१४~॥\par
 किञ्च —} 
\begin{center}{\bfseries सर्वस्य चाहं हृदि संनिविष्टो\\ मत्तः स्मृतिर्ज्ञानमपोहनं च~।\\वेदैश्च सर्वैरहमेव वेद्यो\\ वेदान्तकृद्वेदविदेव चाहम्~॥~१५~॥}\end{center} 
सर्वस्य च प्राणिजातस्य अहम् आत्मा सन् हृदि बुद्धौ संनिविष्टः~। अतः मत्तः आत्मनः सर्वप्राणिनां स्मृतिः ज्ञानं तदपोहनं च अपगमनं च~; येषां यथा पुण्यकर्मणां पुण्यकर्मानुरोधेन ज्ञानस्मृती भवतः, तथा पापकर्मणां पापकर्मानुरूपेण स्मृतिज्ञानयोः अपोहनं च अपायनम् अपगमनं च~। वेदैश्च सर्वैः अहमेव परमात्मा वेद्यः वेदितव्यः~। वेदान्तकृत् वेदान्तार्थसम्प्रदायकृत् इत्यर्थः, वेदवित् वेदार्थवित् एव च अहम्~॥~१५~॥\par
 भगवतः ईश्वरस्य नारायणाख्यस्य विभूतिसङ्क्षेपः उक्तः विशिष्टोपाधिकृतः ‘यदादित्यगतं तेजः’\footnote{भ. गी. १५~। १२} इत्यादिना~। अथ अधुना तस्यैव क्षराक्षरोपाधिप्रविभक्ततया निरुपाधिकस्य केवलस्य स्वरूपनिर्दिधारयिषया उत्तरे श्लोकाः आरभ्यन्ते~। तत्र सर्वमेव अतीतानागताध्यायार्थजातं त्रिधा राशीकृत्य आह —} 
\begin{center}{\bfseries द्वाविमौ पुरुषौ लोके क्षरश्चाक्षर एव च~।\\क्षरः सर्वाणि भूतानि कूटस्थोऽक्षर उच्यते~॥~१६~॥}\end{center} 
द्वौ इमौ पृथग्राशीकृतौ पुरुषौ इति उच्येते लोके संसारे — क्षरश्च क्षरतीति क्षरः विनाशी इति एको राशिः~; अपरः पुरुषः अक्षरः तद्विपरीतः, भगवतः मायाशक्तिः, क्षराख्यस्य पुरुषस्य उत्पत्तिबीजम् अनेकसंसारिजन्तुकामकर्मादिसंस्काराश्रयः, अक्षरः पुरुषः उच्यते~। कौ तौ पुरुषौ इति आह स्वयमेव भगवान् — क्षरः सर्वाणि भूतानि, समस्तं विकारजातम् इत्यर्थः~। कूटस्थः कूटः राशी राशिरिव स्थितः~। अथवा, कूटः माया वञ्चना जिह्मता कुटिलता इति पर्यायाः, अनेकमायावञ्चनादिप्रकारेण स्थितः कूटस्थः, संसारबीजानन्त्यात् न क्षरति इति अक्षरः उच्यते~॥~१६~॥\par
 आभ्यां क्षराक्षराभ्यां अन्यः विलक्षणः क्षराक्षरोपाधिद्वयदोषेण अस्पृष्टः नित्यशुद्धबुद्धमुक्तस्वभावः —} 
\begin{center}{\bfseries उत्तमः पुरुषस्त्वन्यः परमात्मेत्युदाहृतः~।\\यो लोकत्रयमाविश्य बिभर्त्यव्यय ईश्वरः~॥~१७~॥}\end{center} 
उत्तमः उत्कृष्टतमः पुरुषस्तु अन्यः अत्यन्तविलक्षणः आभ्यां परमात्मा इति परमश्च असौ देहाद्यविद्याकृतात्मभ्यः, आत्मा च सर्वभूतानां प्रत्यक्चेतनः, इत्यतः परमात्मा इति उदाहृतः उक्तः वेदान्तेषु~। स एव विशिष्यते यः लोकत्रयं भूर्भुवःस्वराख्यं स्वकीयया चैतन्यबलशक्त्या आविश्य प्रविश्य बिभर्ति स्वरूपसद्भावमात्रेण बिभर्ति धारयति~; अव्ययः न अस्य व्ययः विद्यते इति अव्ययः~। कः~? ईश्वरः सर्वज्ञः नारायणाख्यः ईशनशीलः~॥~१७~॥\par
 यथाव्याख्यातस्य ईश्वरस्य ‘पुरुषोत्तमः’ इत्येतत् नाम प्रसिद्धम्~। तस्य नामनिर्वचनप्रसिद्ध्या अर्थवत्त्वं नाम्नो दर्शयन् ‘निरतिशयः अहम् ईश्वरः’ इति आत्मानं दर्शयति भगवान् —} 
\begin{center}{\bfseries यस्मात्क्षरमतीतोऽह—\\ मक्षरादपि चोत्तमः~।\\अतोऽस्मि लोके वेदे च\\ प्रथितः पुरुषोत्तमः~॥~१८~॥}\end{center} 
यस्मात् क्षरम् अतीतः अहं संसारमायावृक्षम् अश्वत्थाख्यम् अतिक्रान्तः अहम् अक्षरादपि संसारमायारूपवृक्षबीजभूतादपि च उत्तमः उत्कृष्टतमः ऊर्ध्वतमो वा, अतः ताभ्यां क्षराक्षराभ्याम् उत्तमत्वात् अस्मि लोके वेदे च प्रथितः प्रख्यातः~। पुरुषोत्तमः इत्येवं मां भक्तजनाः विदुः~। कवयः काव्यादिषु च इदं नाम निबध्नन्ति~। पुरुषोत्तम इत्यनेनाभिधानेनाभिगृणन्ति~॥~१८~॥\par
 अथ इदानीं यथानिरुक्तम् आत्मानं यो वेद, तस्य इदं फलम् उच्यते —} 
\begin{center}{\bfseries यो मामेवमसंमूढो जानाति पुरुषोत्तमम्~।\\स सर्वविद्भजति मां सर्वभावेन भारत~॥~१९~॥}\end{center} 
यः माम् ईश्वरं यथोक्तविशेषणम् एवं यथोक्तेन प्रकारेण असंमूढः संमोहवर्जितः सन् जानाति ‘अयम् अहम् अस्मि’ इति पुरुषोत्तमं सः सर्ववित् सर्वात्मना सर्वं वेत्तीति सर्वज्ञः सर्वभूतस्थं भजति मां सर्वभावेन सर्वात्मतया हे भारत~॥~१९~॥\par
 अस्मिन् अध्याये भगवत्तत्त्वज्ञानं मोक्षफलम् उक्त्वा,अथ इदानीं तत् स्तौति —} 
\begin{center}{\bfseries इति गुह्यतमं शास्त्र—\\ मिदमुक्तं मयानघ~।\\एतद्बुद्ध्वा बुद्धिमान्स्या—\\ त्कृतकृत्यश्च भारत~॥~२०~॥}\end{center} 
इति एतत् गुह्यतमं गोप्यतमम्~, अत्यन्तरहस्यं इत्येतत्~। किं तत्~? शास्त्रम्~। यद्यपि गीताख्यं समस्तम् ‘शास्त्रम्’ उच्यते, तथापि अयमेव अध्यायः इह ‘शास्त्रम्’ इति उच्यते स्तुत्यर्थं प्रकरणात्~। सर्वो हि गीताशास्त्रार्थः अस्मिन् अध्याये समासेन उक्तः~। न केवलं गीताशास्त्रार्थ एव, किन्तु सर्वश्च वेदार्थः इह परिसमाप्तः~। ‘यस्तं वेद स वेदवित्’\footnote{भ. गी. १५~। १} ‘वेदैश्च सर्वैरहमेव वेद्यः’\footnote{भ. गी. १५~। १५} इति च उक्तम्~। इदम् उक्तं कथितं मया हे अनघ अपाप~। एतत् शास्त्रं यथादर्शितार्थं बुद्ध्वा बुद्धिमान् स्यात् भवेत् न अन्यथा कृतकृत्यश्च भारत कृतं कृत्यं कर्तव्यं येन सः कृतकृत्यः~; विशिष्टजन्मप्रसूतेन ब्राह्मणेन यत् कर्तव्यं तत् सर्वं भगवत्तत्त्वे विदिते कृतं भवेत् इत्यर्थः~; न च अन्यथा कर्तव्यं परिसमाप्यते कस्यचित् इत्यभिप्रायः~। ‘सर्वं कर्माखिलं पार्थ ज्ञाने परिसमाप्यते’\footnote{भ. गी. ४~। ३३} इति च उक्तम्~। ‘एतद्धि जन्मसामग्र्यं ब्राह्मणस्य विशेषतः~। प्राप्यैतत्कृतकृत्यो हि द्विजो भवति नान्यथा’\footnote{मनु. १२~। ९३} इति च मानवं वचनम्~। यतः एतत् परमार्थतत्त्वं मत्तः श्रुतवान् असि, अतः कृतार्थः त्वं भारत इति~॥~२०~॥\par
 
इति श्रीमत्परमहंसपरिव्राजकाचार्यस्य श्रीगोविन्दभगवत्पूज्यपादशिष्यस्य श्रीमच्छङ्करभगवतः कृतौ श्रीमद्भगवद्गीताभाष्ये पञ्चदशोऽध्यायः~॥\par
 
दैवी आसुरी राक्षसी इति प्राणिनां प्रकृतयः नवमे अध्याये सूचिताः~। तासां विस्तरेण प्रदर्शनाय ‘अभयं सत्त्वसंशुद्धिः’ इत्यादिः अध्यायः आरभ्यते~। तत्र संसारमोक्षाय दैवी प्रकृतिः, निबन्धाय आसुरी राक्षसी च इति दैव्याः आदानाय प्रदर्शनं क्रियते, इतरयोः परिवर्जनाय च~॥~} 
\begin{center}{\bfseries श्रीभगवानुवाच —\\अभयं सत्त्वसंशुद्धिर्ज्ञानयोगव्यवस्थितिः~।\\दानं दमश्च यज्ञश्च स्वाध्यायस्तप आर्जवम्~॥~१~॥}\end{center} 
अभयम् अभीरुता~। सत्त्वसंशुद्धिः सत्त्वस्य अन्तःकरणस्य संशुद्धिः संव्यवहारेषु परवञ्चनामायानृतादिपरिवर्जनं शुद्धसत्त्वभावेन व्यवहारः इत्यर्थः~। ज्ञानयोगव्यवस्थितिः ज्ञानं शास्त्रतः आचार्यतश्च आत्मादिपदार्थानाम् अवगमः, अवगतानाम् इन्द्रियाद्युपसंहारेण एकाग्रतया स्वात्मसंवेद्यतापादनं योगः, तयोः ज्ञानयोगयोः व्यवस्थितिः व्यवस्थानं तन्निष्ठता~। एषा प्रधाना दैवी सात्त्विकी सम्पत्~। यत्र येषाम् अधिकृतानां या प्रकृतिः सम्भवति, सात्त्विकी सा उच्यते~। दानं यथाशक्ति संविभागः अन्नादीनाम्~। दमश्च बाह्यकरणानाम् उपशमः~; अन्तःकरणस्य उपशमं शान्तिं वक्ष्यति~। यज्ञश्च श्रौतः अग्निहोत्रादिः~। स्मार्तश्च देवयज्ञादिः, स्वाध्यायः ऋग्वेदाद्यध्ययनम् अदृष्टार्थम्~। तपः वक्ष्यमाणं शारीरादि~। आर्जवम् ऋजुत्वं सर्वदा~॥~१~॥\par
 किञ्च —} 
\begin{center}{\bfseries अहिंसा सत्यमक्रोधस्त्यागः शान्तिरपैशुनम्~।\\दया भूतेष्वलोलुप्त्वं मार्दवं ह्रीरचापलम्~॥~२~॥}\end{center} 
अहिंसा अहिंसनं प्राणिनां पीडावर्जनम्~। सत्यम् अप्रियानृतवर्जितं यथाभूतार्थवचनम्~। अक्रोधः परैः आक्रुष्टस्य अभिहतस्य वा प्राप्तस्य क्रोधस्य उपशमनम्~। त्यागः संन्यासः, पूर्वं दानस्य उक्तत्वात्~। शान्तिः अन्तःकरणस्य उपशमः~। अपैशुनं अपिशुनता~; परस्मै पररन्ध्रप्रकटीकरणं पैशुनम्~, तदभावः अपैशुनम्~। दया कृपा भूतेषु दुःखितेषु~। अलोलुप्त्वम् इन्द्रियाणां विषयसंनिधौ अविक्रिया~। मार्दवं मृदुता अक्रौर्यम्~। ह्रीः लज्जा~। अचापलम् असति प्रयोजने वाक्पाणिपादादीनाम् अव्यापारयितृत्वम्~॥~२~॥\par
 किञ्च —} 
\begin{center}{\bfseries तेजः क्षमा धृतिः शौचमद्रोहो नातिमानिता~।\\भवन्ति सम्पदं दैवीमभिजातस्य भारत~॥~३~॥}\end{center} 
तेजः प्रागल्भ्यं न त्वग्गता दीप्तिः~। क्षमा आक्रुष्टस्य ताडितस्य वा अन्तर्विक्रियानुत्पत्तिः, उत्पन्नायां विक्रियायाम् उपशमनम् अक्रोधः इति अवोचाम~। इत्थं क्षमायाः अक्रोधस्य च विशेषः~। धृतिः देहेन्द्रियेषु अवसादं प्राप्तेषु तस्य प्रतिषेधकः अन्तःकरणवृत्तिविशेषः, येन उत्तम्भितानि करणानि देहश्च न अवसीदन्ति~। शौचं द्विविधं मृज्जलकृतं बाह्यम् आभ्यन्तरं च मनोबुद्ध्योः नैर्मल्यं मायारागादिकालुष्याभावः~; एवं द्विविधं शौचम्~। अद्रोहः परजिघांसाभावः अहिंसनम्~। नातिमानिता अत्यर्थं मानः अतिमानः, सः यस्य विद्यते सः अतिमानी, तद्भावः अतिमानिता, तदभावः नातिमानिता आत्मनः पूज्यतातिशयभावनाभाव इत्यर्थः~। भवन्ति अभयादीनि एतदन्तानि सम्पदं अभिजातस्य~। किंविशिष्टां सम्पदम्~? दैवीं देवानां या सम्पत् ताम् अभिलक्ष्य जातस्य देवविभूत्यर्हस्य भाविकल्याणस्य इत्यर्थः, हे भारत~॥~३~॥\par
 अथ इदानीं आसुरी सम्पत् उच्यते —} 
\begin{center}{\bfseries दम्भो दर्पोऽतिमानश्च क्रोधः पारुष्यमेव च~।\\अज्ञानं चाभिजातस्य पार्थ सम्पदमासुरीम्~॥~४~॥}\end{center} 
दम्भः धर्मध्वजित्वम्~। दर्पः विद्याधनस्वजनादिनिमित्तः उत्सेकः~। अतिमानः पूर्वोक्तः~। क्रोधश्च~। पारुष्यमेव च परुषवचनम् — यथा काणम् ‘चक्षुष्मान्’ विरूपम् ‘रूपवान्’ हीनाभिजनम् ‘उत्तमाभिजनः’ इत्यादि~। अज्ञानं च अविवेकज्ञानं कर्तव्याकर्तव्यादिविषयमिथ्याप्रत्ययः~। अभिजातस्य पार्थ~। किम् अभिजातस्येति, आह — सम्पदम् आसुरीम् असुराणां सम्पत् आसुरी ताम् अभिजातस्य इत्यर्थः~॥~४~॥\par
 अनयोः सम्पदोः कार्यम् उच्यते —} 
\begin{center}{\bfseries दैवी सम्पद्विमोक्षाय निबन्धायासुरी मता~।\\मा शुचः सम्पदं दैवीमभिजातोऽसि पाण्डव~॥~५~॥}\end{center} 
दैवी सम्पत् या, सा विमोक्षाय संसारबन्धनात्~। निबन्धाय नियतः बन्धः निबन्धः तदर्थम् आसुरी सम्पत् मता अभिप्रेता~। तथा राक्षसी च~। तत्र एवम् उक्ते सति अर्जुनस्य अन्तर्गतं भावम् ‘किम् अहम् आसुरसम्पद्युक्तः~? किं वा दैवसम्पद्युक्तः~? ’ इत्येवम् आलोचनारूपम् आलक्ष्य आह भगवान् — मा शुचः शोकं मा कार्षीः~। सम्पदं दैवीम् अभिजातः असि अभिलक्ष्य जातोऽसि, भाविकल्याणः त्वम् असि इत्यर्थः, हे पाण्डव~॥~५~॥\par
 \begin{center}{\bfseries द्वौ भूतसर्गौ लोकेऽस्मिन्दैव आसुर एव च~।\\दैवो विस्तरशः प्रोक्त आसुरं पार्थ मे शृणु~॥~६~॥}\end{center} 
द्वौ द्विसङ्ख्याकौ भूतसर्गौ भूतानां मनुष्याणां सर्गौ सृष्टी भूतसर्गौ सृज्येतेति सर्गौ भूतान्येव सृज्यमानानि दैवासुरसम्पद्द्वययुक्तानि इति द्वौ भूतसर्गौ इति उच्यते, ‘द्वया ह वै प्राजापत्या देवाश्चासुराश्च’\footnote{बृ. उ. १~। ३~। १} इति श्रुतेः~। लोके अस्मिन्~, संसारे इत्यर्थः, सर्वेषां द्वैविध्योपपत्तेः~। कौ तौ भूतसर्गौ इति, उच्यते — प्रकृतावेव दैव आसुर एव च~। उक्तयोरेव पुनः अनुवादे प्रयोजनम् आह — दैवः भूतसर्गः ‘अभयं सत्त्वसंशुद्धिः’\footnote{भ. गी. १६~। १} इत्यादिना विस्तरशः विस्तरप्रकारैः प्रोक्तः कथितः, न तु आसुरः विस्तरशः~; अतः तत्परिवर्जनार्थम् आसुरं पार्थ, मे मम वचनात् उच्यमानं विस्तरशः शृणु अवधारय~॥~६~॥\par
 आ अध्यायपरिसमाप्तेः आसुरी सम्पत् प्राणिविशेषणत्वेन प्रदर्श्यते, प्रत्यक्षीकरणेन च शक्यते तस्याः परिवर्जनं कर्तुमिति —} 
\begin{center}{\bfseries प्रवृत्तिं च निवृत्तिं च जना न विदुरासुराः~।\\न शौचं नापि चाचारो न सत्यं तेषु विद्यते~॥~७~॥}\end{center} 
प्रवृत्तिं च प्रवर्तनं यस्मिन् पुरुषार्थसाधने कर्तव्ये प्रवृत्तिः ताम्~, निवृत्तिं च एतद्विपरीतां यस्मात् अनर्थहेतोः निवर्तितव्यं सा निवृत्तिः तां च, जनाः आसुराः न विदुः न जानन्ति~। न केवलं प्रवृत्तिनिवृत्ती एव ते न विदुः, न शौचं नापि च आचारः न सत्यं तेषु विद्यते~; अशौचाः अनाचाराः मायाविनः अनृतवादिनो हि आसुराः~॥~७~॥\par
 किञ्च —} 
\begin{center}{\bfseries असत्यमप्रतिष्ठं ते जगदाहुरनीश्वरम्~।\\अपरस्परसम्भूतं किमन्यत्कामहैतुकम्~॥~८~॥}\end{center} 
असत्यं यथा वयम् अनृतप्रायाः तथा इदं जगत् सर्वम् असत्यम्~, अप्रतिष्ठं च न अस्य धर्माधर्मौ प्रतिष्ठा अतः अप्रतिष्ठं च, इति ते आसुराः जनाः जगत् आहुः, अनीश्वरम् न च धर्माधर्मसव्यपेक्षकः अस्य शासिता ईश्वरः विद्यते इति अतः अनीश्वरं जगत् आहुः~। किञ्च, अपरस्परसम्भूतं कामप्रयुक्तयोः स्त्रीपुरुषयोः अन्योन्यसंयोगात् जगत् सर्वं सम्भूतम्~। किमन्यत् कामहैतुकं कामहेतुकमेव कामहैतुकम्~। किमन्यत् जगतः कारणम्~? न किञ्चित् अदृष्टं धर्माधर्मादि कारणान्तरं विद्यते जगतः ‘काम एव प्राणिनां कारणम्’ इति लोकायतिकदृष्टिः इयम्~॥~८~॥\par
 \begin{center}{\bfseries एतां दृष्टिमवष्टभ्य नष्टात्मानोऽल्पबुद्धयः~।\\प्रभवन्त्युग्रकर्माणः क्षयाय जगतोऽहिताः~॥~९~॥}\end{center} 
एतां दृष्टिम् अवष्टभ्य आश्रित्य नष्टात्मानः नष्टस्वभावाः विभ्रष्टपरलोकसाधनाः अल्पबुद्धयः विषयविषया अल्पैव बुद्धिः येषां ते अल्पबुद्धयः प्रभवन्ति उद्भवन्ति उग्रकर्माणः क्रूरकर्माणः हिंसात्मकाः~। क्षयाय जगतः प्रभवन्ति इति सम्बन्धः~। जगतः अहिताः, शत्रवः इत्यर्थः~॥~९~॥\par
 ते च —} 
\begin{center}{\bfseries काममाश्रित्य दुष्पूरं दम्भमानमदान्विताः~।\\मोहाद्गृहीत्वासद्ग्राहान्प्रवर्तन्तेऽशुचिव्रताः~॥~१०~॥}\end{center} 
कामम् इच्छाविशेषम् आश्रित्य अवष्टभ्य दुष्पूरम् अशक्यपूरणं दम्भमानमदान्विताः दम्भश्च मानश्च मदश्च दम्भमानमदाः तैः अन्विताः दम्भमानमदान्विताः मोहात् अविवेकतः गृहीत्वा उपादाय असद्ग्राहान् अशुभनिश्चयान् प्रवर्तन्ते लोके अशुचिव्रताः अशुचीनि व्रतानि येषां ते अशुचिव्रताः~॥~१०~॥\par
 किञ्च —} 
\begin{center}{\bfseries चिन्तामपरिमेयां च प्रलयान्तामुपाश्रिताः~।\\कामोपभोगपरमा एतावदिति निश्चिताः~॥~११~॥}\end{center} 
चिन्ताम् अपरिमेयां च, न परिमातुं शक्यते यस्याः चिन्तायाः इयत्ता सा अपरिमेया, ताम् अपरिमेयाम्~, प्रलयान्तां मरणान्ताम् उपाश्रिताः, सदा चिन्तापराः इत्यर्थः~। कामोपभोगपरमाः, काम्यन्ते इति कामाः विषयाः शब्दादयः तदुपभोगपरमाः ‘अयमेव परमः पुरुषार्थः यः कामोपभोगः’ इत्येवं निश्चितात्मानः, एतावत् इति निश्चिताः~॥~११~॥\par
 \begin{center}{\bfseries आशापाशशतैर्बद्धाः कामक्रोधपरायणाः~।\\ईहन्ते कामभोगार्थमन्यायेनार्थसञ्चयान्~॥~१२~॥}\end{center} 
आशापाशशतैः आशा एव पाशाः तच्छतैः बद्धाः नियन्त्रिताः सन्तः सर्वतः आकृष्यमाणाः, कामक्रोधपरायणाः कामक्रोधौ परम् अयनम् आश्रयः येषां ते कामक्रोधपरायणाः, ईहन्ते चेष्टन्ते कामभोगार्थं कामभोगप्रयोजनाय न धर्मार्थम्~, अन्यायेन परस्वापहरणादिना इत्यर्थः~; किम्~? अर्थसञ्चयान् अर्थप्रचयान्~॥~१२~॥\par
 ईदृशश्च तेषाम् अभिप्रायः —} 
\begin{center}{\bfseries इदमद्य मया लब्धमिदं प्राप्स्ये मनोरथम्~।\\इदमस्तीदमपि मे भविष्यति पुनर्धनम्~॥~१३~॥}\end{center} 
इदं द्रव्यं अद्य इदानीं मया लब्धम्~। इदं च अन्यत् प्राप्स्ये मनोरथं मनस्तुष्टिकरम्~। इदं च अस्ति इदमपि मे भविष्यति आगामिनि संवत्सरे पुनः धनं तेन अहं धनी विख्यातः भविष्यामि इति~॥~१३~॥\par
 \begin{center}{\bfseries असौ मया हतः शत्रु—\\ र्हनिष्ये चापरानपि~।\\ईश्वरोऽहमहं भोगी\\ सिद्धोऽहं बलवान्सुखी~॥~१४~॥}\end{center} 
असौ देवदत्तनामा मया हतः दुर्जयः शत्रुः~। हनिष्ये च अपरान् अन्यान् वराकान् अपि~। किम् एते करिष्यन्ति तपस्विनः~; सर्वथापि नास्ति मत्तुल्यः~। कथम्~? ईश्वरः अहम्~, अहं भोगी~। सर्वप्रकारेण च सिद्धः अहं सम्पन्नः पुत्रैः नप्तृभिः, न केवलं मानुषः, बलवान् सुखी च अहमेव~; अन्ये तु भूमिभारायावतीर्णाः~॥~१४~॥\par
 \begin{center}{\bfseries आढ्योऽभिजनवानस्मि\\ कोऽन्योऽस्ति सदृशो मया~।\\यक्ष्ये दास्यामि मोदिष्य\\ इत्यज्ञानविमोहिताः~॥~१५~॥}\end{center} 
आढ्यः धनेन, अभिजनवान् सप्तपुरुषं श्रोत्रियत्वादिसम्पन्नः — तेनापि न मम तुल्यः अस्ति कश्चित्~। कः अन्यः अस्ति सदृशः तुल्यः मया~? किञ्च, यक्ष्ये यागेनापि अन्यान् अभिभविष्यामि, दास्यामि नटादिभ्यः, मोदिष्ये हर्षं च अतिशयं प्राप्स्यामि, इति एवम् अज्ञानविमोहिताः अज्ञानेन विमोहिताः विविधम् अविवेकभावम् आपन्नाः~॥~१५~॥\par
 \begin{center}{\bfseries अनेकचित्तविभ्रान्ता मोहजालसमावृताः~।\\प्रसक्ताः कामभोगेषु पतन्ति नरकेऽशुचौ~॥~१६~॥}\end{center} 
अनेकचित्तविभ्रान्ताः उक्तप्रकारैः अनेकैः चित्तैः विविधं भ्रान्ताः अनेकचित्तविभ्रान्ताः, मोहजालसमावृताः मोहः अविवेकः अज्ञानं तदेव जालमिव आवरणात्मकत्वात्~, तेन समावृताः~। प्रसक्ताः कामभोगेषु तत्रैव निषण्णाः सन्तः तेन उपचितकल्मषाः पतन्ति नरके अशुचौ वैतरण्यादौ~॥~१६~॥\par
 \begin{center}{\bfseries आत्मसम्भाविताः स्तब्धा धनमानमदान्विताः~।\\यजन्ते नामयज्ञैस्ते दम्भेनाविधिपूर्वकम्~॥~१७~॥}\end{center} 
आत्मसम्भाविताः सर्वगुणविशिष्टतया आत्मनैव सम्भाविताः आत्मसम्भाविताः, न साधुभिः~। स्तब्धाः अप्रणतात्मानः~। धनमानमदान्विताः धननिमित्तः मानः मदश्च, ताभ्यां धनमानमदाभ्याम् अन्विताः~। यजन्ते नामयज्ञैः नाममात्रैः यज्ञैः ते दम्भेन धर्मध्वजितया अविधिपूर्वकं विधिविहिताङ्गेतिकर्तव्यतारहितम्~॥~१७~॥\par
 \begin{center}{\bfseries अहङ्कारं बलं दर्पं कामं क्रोधं च संश्रिताः~।\\मामात्मपरदेहेषु प्रद्विषन्तोऽभ्यसूयकाः~॥~१८~॥}\end{center} 
अहङ्कारं अहङ्करणम् अहङ्कारः, विद्यमानैः अविद्यमानैश्च गुणैः आत्मनि अध्यारोपितैः ‘विशिष्टमात्मानमहम्’ इति मन्यते, सः अहङ्कारः अविद्याख्यः कष्टतमः, सर्वदोषाणां मूलं सर्वानर्थप्रवृत्तीनां च, तम्~। तथा बलं पराभिभवनिमित्तं कामरागान्वितम्~। दर्पं दर्पो नाम यस्य उद्भवे धर्मम् अतिक्रामति सः अयम् अन्तःकरणाश्रयः दोषविशेषः~। कामं स्त्र्यादिविषयम्~। क्रोधम् अनिष्टविषयम्~। एतान् अन्यांश्च महतो दोषान् संश्रिताः~। किञ्च ते माम् ईश्वरम् आत्मपरदेहेषु स्वदेहे परदेहेषु च तद्बुद्धिकर्मसाक्षिभूतं मां प्रद्विषन्तः, मच्छासनातिवर्तित्वं प्रद्वेषः, तं कुर्वन्तः अभ्यसूयकाः सन्मार्गस्थानां गुणेषु असहमानाः~॥~१८~॥\par
 \begin{center}{\bfseries तानहं द्विषतः क्रूरान्संसारेषु नराधमान्~।\\क्षिपाम्यजस्रमशुभानासुरीष्वेव योनिषु~॥~१९~॥}\end{center} 
तान् अहं सन्मार्गप्रतिपक्षभूतान् साधुद्वेषिणः द्विषतश्च मां क्रूरान् संसारेषु एव अनेकनरकसंसरणमार्गेषु नराधमान् अधर्मदोषवत्त्वात् क्षिपामि प्रक्षिपामि अजस्रं सन्ततम् अशुभान् अशुभकर्मकारिणः आसुरीष्वेव क्रूरकर्मप्रायासु व्याघ्रसिंहादियोनिषु ‘क्षिपामि’ इत्यनेन सम्बन्धः~॥~१९~॥\par
 \begin{center}{\bfseries आसुरीं योनिमापन्ना\\ मूढा जन्मनि जन्मनि~।\\मामप्राप्यैव कौन्तेय\\ ततो यान्त्यधमां गतिम्~॥~२०~॥}\end{center} 
आसुरीं योनिम् आपन्नाः प्रतिपन्नाः मूढाः अविवेकिनः जन्मनि जन्मनि प्रतिजन्म तमोबहुलास्वेव योनिषु जायमानाः अधो गच्छन्तो मूढाः माम् ईश्वरम् अप्राप्य अनासाद्य एव हे कौन्तेय, ततः तस्मादपि यान्ति अधमां गतिं निकृष्टतमां गतिम्~। ‘माम् अप्राप्यैव’ इति न मत्प्राप्तौ काचिदपि आशङ्का अस्ति, अतः मच्छिष्टसाधुमार्गम् अप्राप्य इत्यर्थः~॥~२०~॥\par
 सर्वस्या आसुर्याः सम्पदः सङ्क्षेपः अयम् उच्यते, यस्मिन् त्रिविधे सर्वः आसुरीसम्पद्भेदः अनन्तोऽपि अन्तर्भवति~। यत्परिहारेण परिहृतश्च भवति, यत् मूलं सर्वस्य अनर्थस्य, तत् एतत् उच्यते —} 
\begin{center}{\bfseries त्रिविधं नरकस्येदं\\ द्वारं नाशनमात्मनः~।\\कामः क्रोधस्तथा लोभ—\\ स्तस्मादेतत्त्रयं त्यजेत्~॥~२१~॥}\end{center} 
त्रिविधं त्रिप्रकारं नरकस्य प्राप्तौ इदं द्वारं नाशनम् आत्मनः, यत् द्वारं प्रविशन्नेव नश्यति आत्मा~; कस्मैचित् पुरुषार्थाय योग्यो न भवति इत्येतत्~, अतः उच्यते ‘द्वारं नाशनमात्मनः’ इति~। किं तत्~? कामः क्रोधः तथा लोभः~। तस्मात् एतत् त्रयं त्यजेत्~। यतः एतत् द्वारं नाशनम् आत्मनः तस्मात् कामादित्रयमेतत् त्यजेत्~॥~२१~॥\par
 त्यागस्तुतिरियम् —} 
\begin{center}{\bfseries एतैर्विमुक्तः कौन्तेय\\ तमोद्वारैस्त्रिभिर्नरः~।\\आचरत्यात्मनः श्रेय—\\ स्ततो याति परां गतिम्~॥~२२~॥}\end{center} 
एतैः विमुक्तः कौन्तेय तमोद्वारैः तमसः नरकस्य दुःखमोहात्मकस्य द्वाराणि कामादयः तैः, एतैः त्रिभिः विमुक्तः नरः आचरति अनुतिष्ठति~। किम्~? आत्मनः श्रेयः~। यत्प्रतिबद्धः पूर्वं न आचचार, तदपगमात् आचरति~। ततः तदाचरणात् याति परां गतिं मोक्षमपि इति~॥~२२~॥\par
 सर्वस्य एतस्य आसुरीसम्पत्परिवर्जनस्य श्रेयआचरणस्य च शास्त्रं कारणम्~। शास्त्रप्रमाणात् उभयं शक्यं कर्तुम्~, न अन्यथा~। अतः —} 
\begin{center}{\bfseries यः शास्त्रविधिमुत्सृज्य\\ वर्तते कामकारतः~।\\न स सिद्धिमवाप्नोति\\ न सुखं न परां गतिम्~॥~२३~॥}\end{center} 
यः शास्त्रविधिं शास्त्रं वेदः तस्य विधिं कर्तव्याकर्तव्यज्ञानकारणं विधिप्रतिषेधाख्यम् उत्सृज्य त्यक्त्वा वर्तते कामकारतः कामप्रयुक्तः सन्~, न सः सिद्धिं पुरुषार्थयोग्यताम् अवाप्नोति, न अपि अस्मिन् लोके सुखं न अपि परां प्रकृष्टां गतिं स्वर्गं मोक्षं वा~॥~२३~॥\par
 \begin{center}{\bfseries तस्माच्छास्त्रं प्रमाणं ते\\ कार्याकार्यव्यवस्थितौ~।\\ज्ञात्वा शास्त्रविधानोक्तं\\ कर्म कर्तुमिहार्हसि~॥~२४~॥}\end{center} 
तस्मात् शास्त्रं प्रमाणं ज्ञानसाधनं ते तव कार्याकार्यव्यवस्थितौ कर्तव्याकर्तव्यव्यवस्थायाम्~। अतः ज्ञात्वा बुद्ध्वा शास्त्रविधानोक्तं विधिः विधानं शास्त्रमेव विधानं शास्त्रविधानम् ‘कुर्यात्~, न कुर्यात्’ इत्येवंलक्षणम्~, तेन उक्तं स्वकर्म यत् तत् कर्तुम् इह अर्हसि, इह इति कर्माधिकारभूमिप्रदर्शनार्थम् इति~॥~२४~॥\par
 
इति श्रीमत्परमहंसपरिव्राजकाचार्यस्य श्रीगोविन्दभगवत्पूज्यपादशिष्यस्य श्रीमच्छङ्करभगवतः कृतौ श्रीमद्भगवद्गीताभाष्ये षोडशोऽध्यायः~॥\par
 
‘तस्माच्छास्त्रं प्रमाणं ते’\footnote{भ. गी. १६~। २४} इति भगवद्वाक्यात् लब्धप्रश्नबीजः अर्जुन उवाच —}\\ 
\begin{center}{\bfseries अर्जुन उवाच —\\ ये शास्त्रविधिमुत्सृज्य\\ यजन्ते श्रद्धयान्विताः~।\\तेषां निष्ठा तु का कृष्ण\\ सत्त्वमाहो रजस्तमः~॥~१~॥}\end{center} 
ये केचित् अविशेषिताः शास्त्रविधिं शास्त्रविधानं श्रुतिस्मृतिशास्त्रचोदनाम् उत्सृज्य परित्यज्य यजन्ते देवादीन् पूजयन्ति श्रद्धया अन्विताः श्रद्धया आस्तिक्यबुद्ध्या अन्विताः संयुक्ताः सन्तः — श्रुतिलक्षणं स्मृतिलक्षणं वा कञ्चित् शास्त्रविधिम् अपश्यन्तः वृद्धव्यवहारदर्शनादेव श्रद्दधानतया ये देवादीन् पूजयन्ति, ते इह ‘ये शास्त्रविधिमुत्सृज्य यजन्ते श्रद्धयान्विताः’ इत्येवं गृह्यन्ते~। ये पुनः कञ्चित् शास्त्रविधिं उपलभमाना एव तम् उत्सृज्य अयथाविधि देवादीन् पूजयन्ति, ते इह ‘ये शास्त्रविधिमुत्सृज्य यजन्ते’ इति न परिगृह्यन्ते~। कस्मात्~? श्रद्धया अन्वितत्वविशेषणात्~। देवादिपूजाविधिपरं किञ्चित् शास्त्रं पश्यन्त एव तत् उत्सृज्य अश्रद्दधानतया तद्विहितायां देवादिपूजायां श्रद्धया अन्विताः प्रवर्तन्ते इति न शक्यं कल्पयितुं यस्मात्~, तस्मात् पूर्वोक्ता एव ‘ये शास्त्रविधिमुत्सृज्य यजन्ते श्रद्धयान्विताः’ इत्यत्र गृह्यन्ते तेषाम् एवंभूतानां निष्ठा तु का कृष्ण सत्त्वम् आहो रजः तमः, किं सत्त्वं निष्ठा अवस्थानम्~, आहोस्वित् रजः, अथवा तमः इति~। एतत् उक्तं भवति — या तेषां देवादिविषया पूजा, सा किं सात्त्विकी, आहोस्वित् राजसी, उत तामसी इति~॥~१~॥\par
 सामान्यविषयः अयं प्रश्नः न अप्रविभज्यं प्रतिवचनम् अर्हतीति श्रीभगवानुवाच —}\\ 
\begin{center}{\bfseries श्रीभगवानुवाच —\\ त्रिविधा भवति श्रद्धा\\ देहिनां सा स्वभावजा~।\\सात्त्विकी राजसी चैव\\ तामसी चेति तां शृणु~॥~२~॥}\end{center} 
त्रिविधा त्रिप्रकारा भवति श्रद्धा, यस्यां निष्ठायां त्वं पृच्छसि, देहिनां शरीरिणां सा स्वभावजा~; जन्मान्तरकृतः धर्मादिसंस्कारः मरणकाले अभिव्यक्तः स्वभावः उच्यते, ततो जाता स्वभावजा~। सात्त्विकी सत्त्वनिर्वृत्ता देवपूजादिविषया~; राजसी रजोनिर्वृत्ता यक्षरक्षःपूजादिविषया~; तामसी तमोनिर्वृत्ता प्रेतपिशाचादिपूजाविषया~; एवं त्रिविधां ताम् उच्यमानां श्रद्धां शृणु अवधारय~॥~२~॥\par
 सा इयं त्रिविधा भवति —} 
\begin{center}{\bfseries सत्त्वानुरूपा सर्वस्य\\ श्रद्धा भवति भारत~।\\श्रद्धामयोऽयं पुरुषो\\ यो यच्छ्रद्धः स एव सः~॥~३~॥}\end{center} 
सत्त्वानुरूपा विशिष्टसंस्कारोपेतान्तःकरणानुरूपा सर्वस्य प्राणिजातस्य श्रद्धा भवति भारत~। यदि एवं ततः किं स्यादिति, उच्यते — श्रद्धामयः अयं श्रद्धाप्रायः पुरुषः संसारी जीवः~। कथम्~? यः यच्छ्रद्धः या श्रद्धा यस्य जीवस्य सः यच्छ्रद्धः स एव तच्छ्रद्धानुरूप एव सः जीवः~॥~३~॥\par
 ततश्च कार्येण लिङ्गेन देवादिपूजया सत्त्वादिनिष्ठा अनुमेया इत्याह —} 
\begin{center}{\bfseries यजन्ते सात्त्विका देवा—\\ न्यक्षरक्षांसि राजसाः~।\\प्रेतान्भूतगणांश्चान्ये\\ यजन्ते तामसा जनाः~॥~४~॥}\end{center} 
यजन्ते पूजयन्ति सात्त्विकाः सत्त्वनिष्ठाः देवान्~, यक्षरक्षांसि राजसाः, प्रेतान् भूतगणांश्च सप्तमातृकादींश्च अन्ये यजन्ते तामसाः जनाः~॥~४~॥\par
 एवं कार्यतो निर्णीताः सत्त्वादिनिष्ठाः शास्त्रविध्युत्सर्गे~। तत्र कश्चिदेव सहस्रेषु देवपूजादिपरः सत्त्वनिष्ठो भवति, बाहुल्येन तु रजोनिष्ठाः तमोनिष्ठाश्चैव प्राणिनो भवन्ति~। कथम्~? —} 
\begin{center}{\bfseries अशास्त्रविहितं घोरं तप्यन्ते ये तपो जनाः~।\\दम्भाहङ्कारसंयुक्ताः कामरागबलान्विताः~॥~५~॥}\end{center} 
अशास्त्रविहितं न शास्त्रविहितम् अशास्त्रविहितं घोरं पीडाकरं प्राणिनाम् आत्मनश्च तपः तप्यन्ते निर्वर्तयन्ति ये जनाः ते च दम्भाहङ्कारसंयुक्ताः, दम्भश्च अहङ्कारश्च दम्भाहङ्कारौ, ताभ्यां संयुक्ताः दम्भाहङ्कारसंयुक्ताः, कामरागबलान्विताः कामश्च रागश्च कामरागौ तत्कृतं बलं कामरागबलं तेन अन्विताः कामरागबलान्विताः~॥~५~॥\par
 \begin{center}{\bfseries कर्शयन्तः शरीरस्थं\\ भूतग्राममचेतसः~।\\मां चैवान्तःशरीरस्थं\\ तान्विद्ध्यासुरनिश्चयान्~॥~६~॥}\end{center} 
कर्शयन्तः कृशीकुर्वन्तः शरीरस्थं भूतग्रामं करणसमुदायम् अचेतसः अविवेकिनः मां चैव तत्कर्मबुद्धिसाक्षिभूतम् अन्तःशरीरस्थं नारायणं कर्शयन्तः, मदनुशासनाकरणमेव मत्कर्शनम्~, तान् विद्धि आसुरनिश्चयान् आसुरो निश्चयो येषां ते आसुरनिश्चयाः तान् परिहरणार्थं विद्धि इति उपदेशः~॥~६~॥\par
 आहाराणां च रस्यस्निग्धादिवर्गत्रयरूपेण भिन्नानां यथाक्रमं सात्त्विकराजसतामसपुरुषप्रियत्वदर्शनम् इह क्रियते रस्यस्निग्धादिषु आहारविशेषेषु आत्मनः प्रीत्यतिरेकेण लिङ्गेन सात्त्विकत्वं राजसत्वं तामसत्वं च बुद्ध्वा रजस्तमोलिङ्गानाम् आहाराणां परिवर्जनार्थं सत्त्वलिङ्गानां च उपादानार्थम्~। तथा यज्ञादीनामपि सत्त्वादिगुणभेदेन त्रिविधत्वप्रतिपादनम् इह ‘राजसतामसान् बुद्ध्वा कथं नु नाम परित्यजेत्~, सात्त्विकानेव अनुतिष्ठेत्’ इत्येवमर्थम्~। आह —} 
\begin{center}{\bfseries आहारस्त्वपि सर्वस्य त्रिविधो भवति प्रियः~।\\यज्ञस्तपस्तथा दानं तेषां भेदमिमं शृणु~॥~७~॥}\end{center} 
आहारस्त्वपि सर्वस्य भोक्तुः प्राणिनः त्रिविधो भवति प्रियः इष्टः, तथा यज्ञः, तथा तपः, तथा दानम्~। तेषाम् आहारादीनां भेदम् इमं वक्ष्यमाणं शृणु~॥~७~॥\par
 \begin{center}{\bfseries आयुःसत्त्वबलारोग्य—\\ सुखप्रीतिविवर्धनाः~।\\रस्याः स्निग्धाः स्थिरा हृद्या\\ आहाराः सात्त्विकप्रियाः~॥~८~॥}\end{center} 
आयुश्च सत्त्वं च बलं च आरोग्यं च सुखं च प्रीतिश्च आयुःसत्त्वबलारोग्यसुखप्रीतयः तासां विवर्धनाः आयुःसत्त्वबलारोग्यसुखप्रीतिविवर्धनाः, ते च रस्याः रसोपेताः, स्निग्धाः स्नेहवन्तः, स्थिराः चिरकालस्थायिनः देहे, हृद्याः हृदयप्रियाः आहाराः सात्त्विकप्रियाः सात्त्विकस्य इष्टाः~॥~८~॥\par
 \begin{center}{\bfseries कट्वम्ललवणात्युष्णतीक्ष्णरूक्षविदाहिनः~।\\आहारा राजसस्येष्टा दुःखशोकामयप्रदाः~॥~९~॥}\end{center} 
कट्वम्ललवणात्युष्णतीक्ष्णरूक्षविदाहिनः इत्यत्र अतिशब्दः कट्वादिषु सर्वत्र योज्यः, अतिकटुः अतितीक्ष्णः इत्येवम्~। कटुश्च अम्लश्च लवणश्च अत्युष्णश्च तीक्ष्णश्च रूक्षश्च विदाही च ते आहाराः राजसस्य इष्टाः, दुःखशोकामयप्रदाः दुःखं च शोकं च आमयं च प्रयच्छन्तीति दुःखशोकामयप्रदाः~॥~९~॥\par
 \begin{center}{\bfseries यातयामं गतरसं पूति पर्युषितं च यत्~।\\उच्छिष्टमपि चामेध्यं भोजनं तामसप्रियम्~॥~१०~॥}\end{center} 
यातयामं मन्दपक्वम्~, निर्वीर्यस्य गतरसशब्देन उक्तत्वात्~। गतरसं रसवियुक्तम्~, पूति दुर्गन्धि, पर्युषितं च पक्वं सत् रात्र्यन्तरितं च यत्~, उच्छिष्टमपि भुक्तशिष्टम् उच्छिष्टम्~, अमेध्यम् अयज्ञार्हम्~, भोजनम् ईदृशं तामसप्रियम्~॥~१०~॥\par
 अथ इदानीं यज्ञः त्रिविधः उच्यते —} 
\begin{center}{\bfseries अफलाकाङ्क्षिभिर्यज्ञो विधिदृष्टो य इज्यते~।\\यष्टव्यमेवेति मनः समाधाय स सात्त्विकः~॥~११~॥}\end{center} 
अफलाकाङ्क्षिभिः अफलार्थिभिः यज्ञः विधिदृष्टः शास्त्रचोदनादृष्टो यः यज्ञः इज्यते निर्वर्त्यते, यष्टव्यमेवेति यज्ञस्वरूपनिर्वर्तनमेव कार्यम् इति मनः समाधाय, न अनेन पुरुषार्थो मम कर्तव्यः इत्येवं निश्चित्य, सः सात्त्विकः यज्ञः उच्यते~॥~११~॥\par
 \begin{center}{\bfseries अभिसन्धाय तु फलं दम्भार्थमपि चैव यत्~।\\इज्यते भरतश्रेष्ठ तं यज्ञं विद्धि राजसम्~॥~१२~॥}\end{center} 
अभिसन्धाय तु उद्दिश्य फलं दम्भार्थमपि चैव यत् इज्यते भरतश्रेष्ठ तं यज्ञं विद्धि राजसम्~॥~१२~॥\par
 \begin{center}{\bfseries विधिहीनमसृष्टान्नं मन्त्रहीनमदक्षिणम्~।\\श्रद्धाविरहितं यज्ञं तामसं परिचक्षते~॥~१३~॥}\end{center} 
विधिहीनं यथाचोदितविपरीतम्~, असृष्टान्नं ब्राह्मणेभ्यो न सृष्टं न दत्तम् अन्नं यस्मिन् यज्ञे सः असृष्टान्नः तम् असृष्टान्नम्~, मन्त्रहीनं मन्त्रतः स्वरतो वर्णतो वा वियुक्तं मन्त्रहीनम्~, अदक्षिणम् उक्तदक्षिणारहितम्~, श्रद्धाविरहितं यज्ञं तामसं परिचक्षते तमोनिर्वृत्तं कथयन्ति~॥~१३~॥\par
 अथ इदानीं तपः त्रिविधम् उच्यते —} 
\begin{center}{\bfseries देवद्विजगुरुप्राज्ञपूजनं शौचमार्जवम्~।\\ब्रह्मचर्यमहिंसा च शारीरं तप उच्यते~॥~१४~॥}\end{center} 
देवाश्च द्विजाश्च गुरवश्च प्राज्ञाश्च देवद्विजगुरुप्राज्ञाः तेषां पूजनं देवद्विजगुरुप्राज्ञपूजनम्~, शौचम्~, आर्जवम् ऋजुत्वम्~, ब्रह्मचर्यम् अहिंसा च शरीरनिर्वर्त्यं शारीरं शरीरप्रधानैः सर्वैरेव कार्यकरणैः कर्त्रादिभिः साध्यं शारीरं तपः उच्यते~। ‘पञ्चैते तस्य हेतवः’\footnote{भ. गी. १८~। १५} इति हि वक्ष्यति~॥~१४~॥\par
 \begin{center}{\bfseries अनुद्वेगकरं वाक्यं सत्यं प्रियहितं च यत्~।\\स्वाध्यायाभ्यसनं चैव वाङ्मयं तप उच्यते~॥~१५~॥}\end{center} 
अनुद्वेगकरं प्राणिनाम् अदुःखकरं वाक्यं सत्यं प्रियहितं च यत् प्रियहिते दृष्टादृष्टार्थे~। अनुद्वेगकरत्वादिभिः धर्मैः वाक्यं विशेष्यते~। विशेषणधर्मसमुच्चयार्थः च—शब्दः~। परप्रत्ययार्थं प्रयुक्तस्य वाक्यस्य सत्यप्रियहितानुद्वेगकरत्वानाम् अन्यतमेन द्वाभ्यां त्रिभिर्वा हीनता स्याद्यदि, न तद्वाङ्मयं तपः~। तथा सत्यवाक्यस्य इतरेषाम् अन्यतमेन द्वाभ्यां त्रिभिर्वा विहीनतायां न वाङ्मयतपस्त्वम्~। तथा प्रियवाक्यस्यापि इतरेषाम् अन्यतमेन द्वाभ्यां त्रिभिर्वा विहीनस्य न वाङ्मयतपस्त्वम्~। तथा हितवाक्यस्यापि इतरेषाम् अन्यतमेन द्वाभ्यां त्रिभिर्वा विहीनस्य न वाङ्मयतपस्त्वम्~। किं पुनः तत् तपः~? यत् सत्यं वाक्यम् अनुद्वेगकरं प्रियं हितं च, तत् तपः वाङ्मयम्~; यथा ‘शान्तो भव वत्स, स्वाध्यायं योगं च अनुतिष्ठ, तथा ते श्रेयो भविष्यति’ इति~। स्वाध्यायाभ्यसनं चैव यथाविधि वाङ्मयं तपः उच्यते~॥~१५~॥\par
 \begin{center}{\bfseries मनःप्रसादः सौम्यत्वं मौनमात्मविनिग्रहः~।\\भावसंशुद्धिरित्येतत्तपो मानसमुच्यते~॥~१६~॥}\end{center} 
मनःप्रसादः मनसः प्रशान्तिः, स्वच्छतापादनं प्रसादः, सौम्यत्वं यत् सौमनस्यम् आहुः — मुखादिप्रसादादिकार्योन्नेया अन्तःकरणस्य वृत्तिः~। मौनं वाङ्‌नियमोऽपि मनःसंयमपूर्वको भवति इति कार्येण कारणम् उच्यते मनःसंयमो मौनमिति~। आत्मविनिग्रहः मनोनिरोधः सर्वतः सामान्यरूपः आत्मविनिग्रहः, वाग्विषयस्यैव मनसः संयमः मौनम् इति विशेषः~। भावसंशुद्धिः परैः व्यवहारकाले अमायावित्वं भावसंशुद्धिः~। इत्येतत् तपः मानसम् उच्यते~॥~१६~॥\par
 यथोक्तं कायिकं वाचिकं मानसं च तपः तप्तं नरैः सत्त्वादिगुणभेदेन कथं त्रिविधं भवतीति, उच्यते —} 
\begin{center}{\bfseries श्रद्धया परया तप्तं तपस्तत्त्रिविधं नरैः~।\\अफलकाङ्क्षिभिर्युक्तैः सात्त्विकं परिचक्षते~॥~१७~॥}\end{center} 
श्रद्धया आस्तिक्यबुद्ध्या परया प्रकृष्टया तप्तम् अनुष्ठितं तपः तत् प्रकृतं त्रिविधं त्रिप्रकारं त्र्यधिष्ठानं नरैः अनुष्ठातृभिः अफलाकाङ्क्षिभिः फलाकाङ्क्षारहितैः युक्तैः समाहितैः — यत् ईदृशं तपः, तत् सात्त्विकं सत्त्वनिर्वृत्तं परिचक्षते कथयन्ति शिष्टाः~॥~१७~॥\par
 \begin{center}{\bfseries सत्कारमानपूजार्थं तपो दम्भेन चैव यत्~।\\क्रियते तदिह प्रोक्तं राजसं चलमध्रुवम्~॥~१८~॥}\end{center} 
सत्कारः साधुकारः ‘साधुः अयं तपस्वी ब्राह्मणः’ इत्येवमर्थम्~, मानो माननं प्रत्युत्थानाभिवादनादिः तदर्थम्~, पूजा पादप्रक्षालनार्चनाशयितृत्वादिः तदर्थं च तपः सत्कारमानपूजार्थम्~, दम्भेन चैव यत् क्रियते तपः तत् इह प्रोक्तं कथितं राजसं चलं कादाचित्कफलत्वेन अध्रुवम्~॥~१८~॥\par
 \begin{center}{\bfseries मूढग्राहेणात्मनो यत्पीडया क्रियते तपः~।\\परस्योत्सादनार्थं वा तत्तामसमुदाहृतम्~॥~१९~॥}\end{center} 
मूढग्राहेण अविवेकनिश्चयेन आत्मनः पीडया यत् क्रियते तपः परस्य उत्सादनार्थं विनाशार्थं वा, तत् तामसं तपः उदाहृतम्~॥~१९~॥\par
 इदानीं दानत्रैविध्यम् उच्यते —} 
\begin{center}{\bfseries दातव्यमिति यद्दानं\\ दीयतेऽनुपकारिणे~।\\देशे काले च पात्रे च\\ तद्दानं सात्त्विकं स्मृतम्~॥~२०~॥}\end{center} 
दातव्यमिति एवं मनः कृत्वा यत् दानं दीयते अनुपकारिणे प्रत्युपकारासमर्थाय, समर्थायापि निरपेक्षं दीयते, देशे पुण्ये कुरुक्षेत्रादौ, काले सङ्क्रान्त्यादौ, पात्रे च षडङ्गविद्वेदपारग इत्यादौ, तत् दानं सात्त्विकं स्मृतम्~॥~२०~॥\par
 \begin{center}{\bfseries यत्तु प्रत्युपकारार्थं\\ फलमुद्दिश्य वा पुनः~।\\दीयते च परिक्लिष्टं\\ तद्दानं राजसं स्मृतम्~॥~२१~॥}\end{center} 
यत्तु दानं प्रत्युपकारार्थं काले तु अयं मां प्रत्युपकरिष्यति इत्येवमर्थम्~, फलं वा अस्य दानस्य मे भविष्यति अदृष्टम् इति, तत् उद्दिश्य पुनः दीयते च परिक्लिष्टं खेदसंयुक्तम्~, तत् दानं राजसं स्मृतम्~॥~२१~॥\par
 \begin{center}{\bfseries अदेशकाले यद्दानमपात्रेभ्यश्च दीयते~।\\असत्कृतमवज्ञातं तत्तामसमुदाहृतम्~॥~२२~॥}\end{center} 
अदेशकाले अदेशे अपुण्यदेशे म्लेच्छाशुच्यादिसङ्कीर्णे अकाले पुण्यहेतुत्वेन अप्रख्याते सङ्क्रान्त्यादिविशेषरहिते अपात्रेभ्यश्च मूर्खतस्करादिभ्यः, देशादिसम्पत्तौ वा असत्कृतं प्रियवचनपादप्रक्षालनपूजादिरहितम् अवज्ञातं पात्रपरिभवयुक्तं च यत् दानम्~, तत् तामसम् उदाहृतम्~॥~२२~॥\par
 यज्ञदानतपःप्रभृतीनां साद्गुण्यकरणाय अयम् उपदेशः उच्यते —} 
\begin{center}{\bfseries ओं तत्सदिति निर्देशो ब्रह्मणस्त्रिविधः स्मृतः~।\\ब्राह्मणास्तेन वेदाश्च यज्ञाश्च विहिताः पुरा~॥~२३~॥}\end{center} 
ओं तत् सत् इति एवं निर्देशः, निर्दिश्यते अनेनेति निर्देशः, त्रिविधो नामनिर्देशः ब्रह्मणः स्मृतः चिन्तितः वेदान्तेषु ब्रह्मविद्भिः~। ब्राह्मणाः तेन निर्देशेन त्रिविधेन वेदाश्च यज्ञाश्च विहिताः निर्मिताः पुरा पूर्वम् इति निर्देशस्तुत्यर्थम् उच्यते~॥~२३~॥\par
 \begin{center}{\bfseries तस्मादोमित्युदाहृत्य यज्ञदानतपःक्रियाः~।\\प्रवर्तन्ते विधानोक्ताः सततं ब्रह्मवादिनाम्~॥~२४~॥}\end{center} 
तस्मात् ‘ओम् इति उदाहृत्य’ उच्चार्य यज्ञदानतपःक्रियाः यज्ञादिस्वरूपाः क्रियाः प्रवर्तन्ते विधानोक्ताः शास्त्रचोदिताः सततं सर्वदा ब्रह्मवादिनां ब्रह्मवदनशीलानाम्~॥~२४~॥\par
 \begin{center}{\bfseries तदित्यनभिसन्धाय\\ फलं यज्ञतपःक्रियाः~।\\दानक्रियाश्च विविधाः\\ क्रियन्ते मोक्षकाङ्क्षिभिः~॥~२५~॥}\end{center} 
तत् इति अनभिसन्धाय, ‘तत्’ इति ब्रह्माभिधानम् उच्चार्य अनभिसन्धाय च यज्ञादिकर्मणः फलं यज्ञतपःक्रियाः यज्ञक्रियाश्च तपःक्रियाश्च यज्ञतपःक्रियाः दानक्रियाश्च विविधाः क्षेत्रहिरण्यप्रदानादिलक्षणाः क्रियन्ते निर्वर्त्यन्ते मोक्षकाङ्क्षिभिः मोक्षार्थिभिः मुमुक्षुभिः~॥~२५~॥\par
 ओन्तच्छब्दयोः विनियोगः उक्तः~। अथ इदानीं सच्छब्दस्य विनियोगः कथ्यते —} 
\begin{center}{\bfseries सद्भावे साधुभावे च सदित्येतत्प्रयुज्यते~।\\प्रशस्ते कर्मणि तथा सच्छब्दः पार्थ युज्यते~॥~२६~॥}\end{center} 
सद्भावे, असतः सद्भावे यथा अविद्यमानस्य पुत्रस्य जन्मनि, तथा साधुभावे च असद्वृत्तस्य असाधोः सद्वृत्तता साधुभावः तस्मिन् साधुभावे च सत् इत्येतत् अभिधानं ब्रह्मणः प्रयुज्यते अभिधीयते~। प्रशस्ते कर्मणि विवाहादौ च तथा सच्छब्दः पार्थ, युज्यते प्रयुज्यते इत्येतत्~॥~२६~॥\par
 \begin{center}{\bfseries यज्ञे तपसि दाने च स्थितिः सदिति चोच्यते~।\\कर्म चैव तदर्थीयं सदित्येवाभिधीयते~॥~२७~॥}\end{center} 
यज्ञे यज्ञकर्मणि या स्थितिः, तपसि च या स्थितिः, दाने च या स्थितिः, सा सत् इति च उच्यते विद्वद्भिः~। कर्म च एव तदर्थीयं यज्ञदानतपोर्थीयम्~; अथवा, यस्य अभिधानत्रयं प्रकृतं तदर्थीयं यज्ञदानतपोर्थीयम् ईश्वरार्थीयम् इत्येतत्~; सत् इत्येव अभिधीयते~। तत् एतत् यज्ञदानतपआदि कर्म असात्त्विकं विगुणमपि श्रद्धापूर्वकं ब्रह्मणः अभिधानत्रयप्रयोगेण सगुणं सात्त्विकं सम्पादितं भवति~॥~२७~॥\par
 तत्र च सर्वत्र श्रद्धाप्रधानतया सर्वं सम्पाद्यते यस्मात्~, तस्मात् —} 
\begin{center}{\bfseries अश्रद्धया हुतं दत्तं तपस्तप्तं कृतं च यत्~।\\असदित्युच्यते पार्थ न च तत्प्रेत्य नो इह~॥~२८~॥}\end{center} 
अश्रद्धया हुतं हवनं कृतम्~, अश्रद्धया दत्तं ब्राह्मणेभ्यः, अश्रद्धया तपः तप्तम् अनुष्ठितम्~, तथा अश्रद्धयैव कृतं यत् स्तुतिनमस्कारादि, तत् सर्वम् असत् इति उच्यते, मत्प्राप्तिसाधनमार्गबाह्यत्वात् पार्थ~। न च तत् बहुलायासमपि प्रेत्य फलाय नो अपि इहार्थम्~, साधुभिः निन्दितत्वात् इति~॥~२८~॥\par
 
इति श्रीमत्परमहंसपरिव्राजकाचार्यस्य श्रीगोविन्दभगवत्पूज्यपादशिष्यस्य श्रीमच्छङ्करभगवतः कृतौ श्रीमद्भगवद्गीताभाष्ये सप्तदशोऽध्यायः~॥\par
 
सर्वस्यैव गीताशास्त्रस्य अर्थः अस्मिन् अध्याये उपसंहृत्य सर्वश्च वेदार्थो वक्तव्यः इत्येवमर्थः अयम् अध्यायः आरभ्यते~। सर्वेषु हि अतीतेषु अध्यायेषु उक्तः अर्थः अस्मिन् अध्याये अवगम्यते~। अर्जुनस्तु संन्यासत्यागशब्दार्थयोरेव विशेषबुभुत्सुः उवाच —}\\ 
\begin{center}{\bfseries अर्जुन उवाच —\\ संन्यासस्य महाबाहो तत्त्वमिच्छामि वेदितुम्~।\\त्यागस्य च हृषीकेश पृथक्केशिनिषूदन~॥~१~॥}\end{center} 
संन्यासस्य संन्यासशब्दार्थस्य इत्येतत्~, हे महाबाहो, तत्त्वं तस्य भावः तत्त्वम्~, याथात्म्यमित्येतत्~, इच्छामि वेदितुं ज्ञातुम्~, त्यागस्य च त्यागशब्दार्थस्येत्येतत्~, हृषीकेश, पृथक् इतरेतरविभागतः केशिनिषूदन केशिनामा हयच्छद्मा कश्चित् असुरः तं निषूदितवान् भगवान् वासुदेवः, तेन तन्नाम्ना सम्बोध्यते अर्जुनेन~॥~१~॥\par
 संन्यासत्यागशब्दौ तत्र तत्र निर्दिष्टौ, न निर्लुठितार्थौ पूर्वेषु अध्यायेषु~। अतः अर्जुनाय पृष्टवते तन्निर्णयाय भगवान् उवाच —}\\ 
\begin{center}{\bfseries श्रीभगवानुवाच —\\ काम्यानां कर्मणां न्यासं संन्यासं कवयो विदुः~।\\सर्वकर्मफलत्यागं प्राहुस्त्यागं विचक्षणाः~॥~२~॥}\end{center} 
काम्यानाम् अश्वमेधादीनां कर्मणां न्यासं संन्यासशब्दार्थम्~, अनुष्ठेयत्वेन प्राप्तस्य अनुष्ठानम्~, कवयः पण्डिताः केचित् विदुः विजानन्ति~। नित्यनैमित्तिकानाम् अनुष्ठीयमानानां सर्वकर्मणाम् आत्मसम्बन्धितया प्राप्तस्य फलस्य परित्यागः सर्वकर्मफलत्यागः तं प्राहुः कथयन्ति त्यागं त्यागशब्दार्थं विचक्षणाः पण्डिताः~। यदि काम्यकर्मपरित्यागः फलपरित्यागो वा अर्थः वक्तव्यः, सर्वथा परित्यागमात्रं संन्यासत्यागशब्दयोः एकः अर्थः स्यात्~, न घटपटशब्दाविव जात्यन्तरभूतार्थौ~॥~} 
ननु नित्यनैमित्तिकानां कर्मणां फलमेव नास्ति इति आहुः~। कथम् उच्यते तेषां फलत्यागः, यथा वन्ध्यायाः पुत्रत्यागः~? नैष दोषः, नित्यानामपि कर्मणां भगवता फलवत्त्वस्य इष्टत्वात्~। वक्ष्यति हि भगवान् ‘अनिष्टमिष्टं मिश्रं च’\footnote{भ. गी. १८~। १२} इति ‘न तु संन्यासिनाम्’\footnote{भ. गी. १८~। १२} इति च~। संन्यासिनामेव हि केवलं कर्मफलासम्बन्धं दर्शयन् असंन्यासिनां नित्यकर्मफलप्राप्तिम् ‘भवत्यत्यागिनां प्रेत्य’\footnote{भ. गी. १८~। १२} इति दर्शयति~॥~२~॥\par
 \begin{center}{\bfseries त्याज्यं दोषवदित्येके कर्म प्राहुर्मनीषिणः~।\\यज्ञदानतपःकर्म न त्याज्यमिति चापरे~॥~३~॥}\end{center} 
त्याज्यं त्यक्तव्यं दोषवत् दोषः अस्य अस्तीति दोषवत्~। किं तत्~? कर्म बन्धहेतुत्वात् सर्वमेव~। अथवा, दोषः यथा रागादिः त्यज्यते, तथा त्याज्यम् इति एके कर्म प्राहुः मनीषिणः पण्डिताः साङ्ख्यादिदृष्टिम् आश्रिताः, अधिकृतानां कर्मिणामपि इति~। तत्रैव यज्ञदानतपःकर्म न त्याज्यम् इति च अपरे~॥~} 
कर्मिणः एव अधिकृताः, तान् अपेक्ष्य एते विकल्पाः, न तु ज्ञाननिष्ठान् व्युत्थायिनः संन्यासिनः अपेक्ष्य~। ‘ज्ञानयोगेन साङ्ख्यानां निष्ठा मया पुरा प्रोक्ता’\footnote{भ. गी. ३~। ३} इति कर्माधिकारात् अपोद्धृताः ये, न तान् प्रति चिन्ता~॥~} 
ननु ‘कर्मयोगेन योगिनाम्’\footnote{भ. गी. ३~। ३} इति अधिकृताः पूर्वं विभक्तनिष्ठाः अपि इह सर्वशास्त्रार्थोपसंहारप्रकरणे यथा विचार्यन्ते, तथा साङ्ख्या अपि ज्ञाननिष्ठाः विचार्यन्ताम् इति~। न, तेषां मोहदुःखनिमित्तत्यागानुपपत्तेः~। न कायक्लेशनिमित्तं दुःखं साङ्ख्याः आत्मनि पश्यन्ति, इच्छादीनां क्षेत्रधर्मत्वेनैव दर्शितत्वात्~। अतः ते न कायक्लेशदुःखभयात् कर्म परित्यजन्ति~। नापि ते कर्माणि आत्मनि पश्यन्ति, येन नियतं कर्म मोहात् परित्यजेयुः~। गुणानां कर्म ‘नैव किञ्चित्करोमि’\footnote{भ. गी. ५~। ८} इति हि ते संन्यस्यन्ति~। ‘सर्वकर्माणि मनसा संन्यस्य’\footnote{भ. गी. ५~। १३} इत्यादिभिः तत्त्वविदः संन्यासप्रकारः उक्तः~। तस्मात् ये अन्ये अधिकृताः कर्मणि अनात्मविदः, येषां च मोहनिमित्तः त्यागः सम्भवति कायक्लेशभयाच्च, ते एव तामसाः त्यागिनः राजसाश्च इति निन्द्यन्ते कर्मिणाम् अनात्मज्ञानां कर्मफलत्यागस्तुत्यर्थम्~; ‘सर्वारम्भपरित्यागी’\footnote{भ. गी. १२~। १६} ‘मौनी सन्तुष्टो येन केनचित्~। अनिकेतः स्थिरमतिः’\footnote{भ. गी. १२~। १९} इति गुणातीतलक्षणे च परमार्थसंन्यासिनः विशेषितत्वात्~। वक्ष्यति च ‘निष्ठा ज्ञानस्य या परा’\footnote{भ. गी. १८~। ५८} इति~। तस्मात् ज्ञाननिष्ठाः संन्यासिनः न इह विवक्षिताः~। कर्मफलत्यागः एव सात्त्विकत्वेन गुणेन तामसत्वाद्यपेक्षया संन्यासः उच्यते, न मुख्यः सर्वकर्मसंन्यासः~॥~} 
सर्वकर्मसंन्यासासम्भवे च ‘न हि देहभृता’\footnote{भ. गी. १८~। ११} इति हेतुवचनात् मुख्य एव इति चेत्~, न~; हेतुवचनस्य स्तुत्यर्थत्वात्~। यथा ‘त्यागाच्छान्तिरनन्तरम्’\footnote{भ. गी. १२~। १२} इति कर्मफलत्यागस्तुतिरेव यथोक्तानेकपक्षानुष्ठानाशक्तिमन्तम् अर्जुनम् अज्ञं प्रति विधानात्~; तथा इदमपि ‘न हि देहभृता शक्यम्’\footnote{भ. गी. १८~। ११} इति कर्मफलत्यागस्तुत्यर्थम्~; न ‘सर्वकर्माणि मनसा संन्यस्य नैव कुर्वन्न कारयन्नास्ते’\footnote{भ. गी. ५~। १३} इत्यस्य पक्षस्य अपवादः केनचित् दर्शयितुं शक्यः~। तस्मात् कर्मणि अधिकृतान् प्रत्येव एषः संन्यासत्यागविकल्पः~। ये तु परमार्थदर्शिनः साङ्ख्याः, तेषां ज्ञाननिष्ठायामेव सर्वकर्मसंन्यासलक्षणायाम् अधिकारः, न अन्यत्र, इति न ते विकल्पार्हाः~। तच्च उपपादितम् अस्माभिः ‘वेदाविनाशिनम्’\footnote{भ. गी. २~। २१} इत्यस्मिन्प्रदेशे, तृतीयादौ च~॥~३~॥\par
 तत्र एतेषु विकल्पभेदेषु —} 
\begin{center}{\bfseries निश्चयं शृणु मे तत्र\\ त्यागे भरतसत्तम~।\\त्यागो हि पुरुषव्याघ्र\\ त्रिविधः सम्प्रकीर्तितः~॥~४~॥}\end{center} 
निश्चयं शृणु अवधारय मे मम वचनात्~; तत्र त्यागे त्यागसंन्यासविकल्पे यथादर्शिते भरतसत्तम भरतानां साधुतम~। त्यागो हि, त्यागसंन्यासशब्दवाच्यो हि यः अर्थः सः एक एवेति अभिप्रेत्य आह — त्यागो हि इति~। पुरुषव्याघ्र, त्रिविधः त्रिप्रकारः तामसादिप्रकारैः सम्प्रकीर्तितः शास्त्रेषु सम्यक् कथितः यस्मात् तामसादिभेदेन त्यागसंन्यासशब्दवाच्यः अर्थः अधिकृतस्य कर्मिणः अनात्मज्ञस्य त्रिविधः सम्भवति, न परमार्थदर्शिनः, इत्ययमर्थः दुर्ज्ञानः, तस्मात् अत्र तत्त्वं न अन्यः वक्तुं समर्थः~। तस्मात् निश्चयं परमार्थशास्त्रार्थविषयम् अध्यवसायम् ऐश्वरं मे मत्तः शृणु~॥~४~॥\par
 कः पुनः असौ निश्चयः इति, आह — } 
\begin{center}{\bfseries यज्ञदानतपःकर्म न त्याज्यं कार्यमेव तत्~।\\यज्ञो दानं तपश्चैव पावनानि मनीषिणाम्~॥~५~॥}\end{center} 
यज्ञः दानं तपः इत्येतत् त्रिविधं कर्म न त्याज्यं न त्यक्तव्यम्~, कार्यं करणीयम् एव तत्~। कस्मात्~? यज्ञः दानं तपश्चैव पावनानि विशुद्धिकराणि मनीषिणां फलानभिसन्धीनाम् इत्येतत्~॥~५~॥\par
 \begin{center}{\bfseries एतान्यपि तु कर्माणि\\ सङ्गं त्यक्त्वा फलानि च~।\\कर्तव्यानीति मे पार्थ\\ निश्चितं मतमुत्तमम्~॥~६~॥}\end{center} 
एतान्यपि तु कर्माणि यज्ञदानतपांसि पावनानि उक्तानि सङ्गम् आसक्तिं तेषु त्यक्त्वा फलानि च तेषां परित्यज्य कर्तव्यानि इति अनुष्ठेयानि इति मे मम निश्चितं मतम् उत्तमम्~॥~} 
‘निश्चयं शृणु मे तत्र’\footnote{भ. गी. १८~। ४} इति प्रतिज्ञाय, पावनत्वं च हेतुम् उक्त्वा, ‘एतान्यपि कर्माणि कर्तव्यानि’ इत्येतत् ‘निश्चितं मतमुत्तमम्’ इति प्रतिज्ञातार्थोपसंहार एव, न अपूर्वार्थं वचनम्~, ‘एतान्यपि’ इति प्रकृतसंनिकृष्टार्थत्वोपपत्तेः~। सासङ्गस्य फलार्थिनः बन्धहेतवः एतान्यपि कर्माणि मुमुक्षोः कर्तव्यानि इति अपिशब्दस्य अर्थः~। न तु अन्यानि कर्माणि अपेक्ष्य ‘एतान्यपि’ इति उच्यते~॥~} 
अन्ये तु वर्णयन्ति — नित्यानां कर्मणां फलाभावात् ‘सङ्गं त्यक्त्वा फलानि च’ इति न उपपद्यते~। अतः ‘एतान्यपि’ इति यानि काम्यानि कर्मणि नित्येभ्यः अन्यानि, एतानि अपि कर्तव्यानि, किमुत यज्ञदानतपांसि नित्यानि इति~। तत् असत्~, नित्यानामपि कर्मणाम् इह फलवत्त्वस्य उपपादितत्वात् ‘यज्ञो दानं तपश्चैव पावनानि’\footnote{भ. गी. १८~। ५} इत्यादिना वचनेन~। नित्यान्यपि कर्माणि बन्धहेतुत्वाशङ्कया जिहासोः मुमुक्षोः कुतः काम्येषु प्रसङ्गः~? ‘दूरेण ह्यवरं कर्म’\footnote{भ. गी. २~। ४९} इति च निन्दितत्वात्~, ‘यज्ञार्थात् कर्मणोऽन्यत्र’\footnote{भ. गी. ३~। ९} इति च काम्यकर्मणां बन्धहेतुत्वस्य निश्चितत्वात्~, ‘त्रैगुण्यविषया वेदाः’\footnote{भ. गी. २~। ४५} ‘त्रैविद्या मां सोमपाः’\footnote{भ. गी. ९~। २०} ‘क्षीणे पुण्ये मर्त्यलोकं विशन्ति’\footnote{भ. गी. ९~। २१} इति च, दूरव्यवहितत्वाच्च, न काम्येषु ‘एतान्यपि’ इति व्यपदेशः~॥~६~॥\par
 तस्मात् अज्ञस्य अधिकृतस्य मुमुक्षोः —} 
\begin{center}{\bfseries नियतस्य तु संन्यासः कर्मणो नोपपद्यते~।\\मोहात्तस्य परित्यागस्तामसः परिकीर्तितः~॥~७~॥}\end{center} 
नियतस्य तु नित्यस्य संन्यासः परित्यागः कर्मणः न उपपद्यते, अज्ञस्य पावनत्वस्य इष्टत्वात्~। मोहात् अज्ञानात् तस्य नियतस्य परित्यागः — नियतं च अवश्यं कर्तव्यम्~, त्यज्यते च, इति विप्रतिषिद्धम्~; अतः मोहनिमित्तः परित्यागः तामसः परिकीर्तितः मोहश्च तमः इति~॥~७~॥\par
 	किंच
	} 
\begin{center}{\bfseries दुःखमित्येव यत्कर्म कायक्लेशभयात्त्यजेत्~।\\स कृत्वा राजसं त्यागं नैव त्यागफलं लभेत्~॥~८~॥}\end{center} 
दुःखम् इति एव यत् कर्म कायक्लेशभयात् शरीरदुःखभयात् त्यजेत्~, सः कृत्वा राजसं रजोनिर्वर्त्यं त्यागं नैव त्यागफलं ज्ञानपूर्वकस्य सर्वकर्मत्यागस्य फलं मोक्षाख्यं न लभेत् नैव लभेत~॥~८~॥\par
 कः पुनः सात्त्विकः त्यागः इति, आह —} 
\begin{center}{\bfseries कार्यमित्येव यत्कर्म\\ नियतं क्रियतेऽर्जुन~।\\सङ्गं त्यक्त्वा फलं चैव\\ स त्यागः सात्त्विको मतः~॥~९~॥}\end{center} 
कार्यं कर्तव्यम् इत्येव यत् कर्म नियतं नित्यं क्रियते निर्वर्त्यते हे अर्जुन, सङ्गं त्यक्त्वा फलं च एव~। एतत् नित्यानां कर्मणां फलवत्त्वे भगवद्वचनं प्रमाणम् अवोचाम~। अथवा, यद्यपि फलं न श्रूयते नित्यस्य कर्मणः, तथापि नित्यं कर्म कृतम् आत्मसंस्कारं प्रत्यवायपरिहारं वा फलं करोति आत्मनः इति कल्पयत्येव अज्ञः~। तत्र तामपि कल्पनां निवारयति ‘फलं त्यक्त्वा’ इत्यनेन~। अतः साधु उक्तम् ‘सङ्गं त्यक्त्वा फलं च’ इति~। सः त्यागः नित्यकर्मसु सङ्गफलपरित्यागः सात्त्विकः सत्त्वनिर्वृत्तः मतः अभिप्रेतः~॥~} 
ननु कर्मपरित्यागः त्रिविधः संन्यासः इति च प्रकृतः~। तत्र तामसो राजसश्च उक्तः त्यागः~। कथम् इह सङ्गफलत्यागः तृतीयत्वेन उच्यते~? यथा त्रयो ब्राह्मणाः आगताः, तत्र षडङ्गविदौ द्वौ, क्षत्रियः तृतीयः इति तद्वत्~। नैष दोषः त्यागसामान्येन स्तुत्यर्थत्वात्~। अस्ति हि कर्मसंन्यासस्य फलाभिसन्धित्यागस्य च त्यागत्वसामान्यम्~। तत्र राजसतामसत्वेन कर्मत्यागनिन्दया कर्मफलाभिसन्धित्यागः सात्त्विकत्वेन स्तूयते ‘स त्यागः सात्त्विको मतः’ इति~॥~९~॥\par
 यस्तु अधिकृतः सङ्गं त्यक्त्वा फलाभिसन्धिं च नित्यं कर्म करोति, तस्य फलरागादिना अकलुषीक्रियमाणम् अन्तःकरणं नित्यैश्च कर्मभिः संस्क्रियमाणं विशुध्यति~। तत् विशुद्धं प्रसन्नम् आत्मालोचनक्षमं भवति~। तस्यैव नित्यकर्मानुष्ठानेन विशुद्धान्तःकरणस्य आत्मज्ञानाभिमुखस्य क्रमेण यथा तन्निष्ठा स्यात्~, तत् वक्तव्यमिति आह —} 
\begin{center}{\bfseries न द्वेष्ट्यकुशलं कर्म\\ कुशले नानुषज्जते~।\\त्यागी सत्त्वसमाविष्टो\\ मेधावी च्छिन्नसंशयः~॥~१०~॥}\end{center} 
न द्वेष्टि अकुशलम् अशोभनं काम्यं कर्म, शरीरारम्भद्वारेण संसारकारणम्~, ‘किमनेन~? ’ इत्येवम्~। कुशले शोभने नित्ये कर्मणि सत्त्वशुद्धिज्ञानोत्पत्तितन्निष्ठाहेतुत्वेन ‘मोक्षकारणम् इदम्’ इत्येवं न अनुषज्जते अनुषङ्गं प्रीतिं न करोति इत्येतत्~। कः पुनः असौ~? त्यागी पूर्वोक्तेन सङ्गफलत्यागेन तद्वान् त्यागी, यः कर्मणि सङ्गं त्यक्त्वा तत्फलं च नित्यकर्मानुष्ठायी सः त्यागी~। कदा पुनः असौ अकुशलं कर्म न द्वेष्टि, कुशले च न अनुषज्जते इति, उच्यते — सत्त्वसमाविष्टः यदा सत्त्वेन आत्मानात्मविवेकविज्ञानहेतुना समाविष्टः संव्याप्तः, संयुक्त इत्येतत्~। अत एव च मेधावी मेधया आत्मज्ञानलक्षणया प्रज्ञया संयुक्तः तद्वान् मेधावी~। मेधावित्वादेव च्छिन्नसंशयः छिन्नः अविद्याकृतः संशयः यस्य ‘आत्मस्वरूपावस्थानमेव परं निःश्रेयससाधनम्~, न अन्यत् किञ्चित्’ इत्येवं निश्चयेन च्छिन्नसंशयः~॥~} 
यः अधिकृतः पुरुषः पूर्वोक्तेन प्रकारेण कर्मयोगानुष्ठानेन क्रमेण संस्कृतात्मा सन् जन्मादिविक्रियारहितत्वेन निष्क्रियम् आत्मानम् आत्मत्वेन सम्बुद्धः, सः सर्वकर्माणि मनसा संन्यस्य नैव कुर्वन् न कारयन् आसीनः नैष्कर्म्यलक्षणां ज्ञाननिष्ठाम् अश्नुते इत्येतत्~। पूर्वोक्तस्य कर्मयोगस्य प्रयोजनम् अनेनैव श्लोकेन उक्तम्~॥~१०~॥\par
 यः पुनः अधिकृतः सन् देहात्माभिमानित्वेन देहभृत् अज्ञः अबाधितात्मकर्तृत्वविज्ञानतया ‘अहं कर्ता’ इति निश्चितबुद्धिः तस्य अशेषकर्मपरित्यागस्य अशक्यत्वात् कर्मफलत्यागेन चोदितकर्मानुष्ठाने एव अधिकारः, न तत्त्यागे इति एतम् अर्थं दर्शयितुम् आह —} 
\begin{center}{\bfseries न हि देहभृता शक्यं त्यक्तुं कर्माण्यशेषतः~।\\यस्तु कर्मफलत्यागी स त्यागीत्यभिधीयते~॥~११~॥}\end{center} 
न हि यस्मात् देहभृता, देहं बिभर्तीति देहभृत्~, देहात्माभिमानवान् देहभृत् उच्यते, न विवेकी~; स हि ‘वेदाविनाशिनम्’\footnote{भ. गी. २~। २१} इत्यादिना कर्तृत्वाधिकारात् निवर्तितः~। अतः तेन देहभृता अज्ञेन न शक्यं त्यक्तुं संन्यसितुं कर्माणि अशेषतः निःशेषेण~। तस्मात् यस्तु अज्ञः अधिकृतः नित्यानि कर्माणि कुर्वन् कर्मफलत्यागी कर्मफलाभिसन्धिमात्रसंन्यासी सः त्यागी इति अभिधीयते कर्मी अपि सन् इति स्तुत्यभिप्रायेण~। तस्मात् परमार्थदर्शिनैव अदेहभृता देहात्मभावरहितेन अशेषकर्मसंन्यासः शक्यते कर्तुम्~॥~११~॥\par
 किं पुनः तत् प्रयोजनम्~, यत् सर्वकर्मसंन्यासात् स्यादिति, उच्यते —} 
\begin{center}{\bfseries अनिष्टमिष्टं मिश्रं च\\ त्रिविधं कर्मणः फलम्~।\\भवत्यत्यागिनां प्रेत्य\\ न तु संन्यासिनां क्वचित्~॥~१२~॥}\end{center} 
अनिष्टं नरकतिर्यगादिलक्षणम्~, इष्टं देवादिलक्षणम्~, मिश्रम् इष्टानिष्टसंयुक्तं मनुष्यलक्षणं च, तत्र त्रिविधं त्रिप्रकारं कर्मणः धर्माधर्मलक्षणस्य फलं बाह्यानेककारकव्यापारनिष्पन्नं सत् अविद्याकृतम् इन्द्रजालमायोपमं महामोहकरं प्रत्यगात्मोपसर्पि इह — फल्गुतया लयम् अदर्शनं गच्छतीति फलनिर्वचनम् — तत् एतत् एवंलक्षणं फलं भवति अत्यागिनाम् अज्ञानां कर्मिणां अपरमार्थसंन्यासिनां प्रेत्य शरीरपातात् ऊर्ध्वम्~। न तु संन्यासिनां परमार्थसंन्यासिनां परमहंसपरिव्राजकानां केवलज्ञाननिष्ठानां क्वचित्~। न हि केवलसम्यग्दर्शननिष्ठा अविद्यादिसंसारबीजं न उन्मूलयति कदाचित् इत्यर्थः~॥~१२~॥\par
 
अतः परमार्थदर्शिनः एव अशेषकर्मसंन्यासित्वं सम्भवति, अविद्याध्यारोपितत्वात् आत्मनि क्रियाकारकफलानाम्~; न तु अज्ञस्य अधिष्ठानादीनि क्रियाकर्तृकारकाणि आत्मत्वेनैव पश्यतः अशेषकर्मसंन्यासः सम्भवति तदेतत् उत्तरैः श्लोकैः दर्शयति —} 
\begin{center}{\bfseries पञ्चैतानि महाबाहो\\ कारणानि निबोध मे~।\\साङ्ख्ये कृतान्ते प्रोक्तानि\\ सिद्धये सर्वकर्मणाम्~॥~१३~॥}\end{center} 
पञ्च एतानि वक्ष्यमाणानि हे महाबाहो, कारणानि निर्वर्तकानि~। निबोध मे मम इति उत्तरत्र चेतःसमाधानार्थम्~, वस्तुवैषम्यप्रदर्शनार्थं च~। तानि च कारणानि ज्ञातव्यतया स्तौति — साङ्ख्ये ज्ञातव्याः पदार्थाः सङ्ख्यायन्ते यस्मिन् शास्त्रे तत् साङ्ख्यं वेदान्तः~। कृतान्ते इति तस्यैव विशेषणम्~। कृतम् इति कर्म उच्यते, तस्य अन्तः परिसमाप्तिः यत्र सः कृतान्तः, कर्मान्तः इत्येतत्~। ‘यावानर्थ उदपाने’\footnote{भ. गी. २~। ४६} ‘सर्वं कर्माखिलं पार्थ ज्ञाने परिसमाप्यते’\footnote{भ. गी. ४~। ३३} इति आत्मज्ञाने सञ्जाते सर्वकर्मणां निवृत्तिं दर्शयति~। अतः तस्मिन् आत्मज्ञानार्थे साङ्ख्ये कृतान्ते वेदान्ते प्रोक्तानि कथितानि सिद्धये निष्पत्त्यर्थं सर्वकर्मणाम्~॥~१३~॥\par
 कानि तानीति, उच्यते —} 
\begin{center}{\bfseries अधिष्ठानं तथा कर्ता करणं च पृथग्विधम्~।\\विविधाश्च पृथक्चेष्टा दैवं चैवात्र पञ्चमम्~॥~१४~॥}\end{center} 
अधिष्ठानम् इच्छाद्वेषसुखदुःखज्ञानादीनाम् अभिव्यक्तेराश्रयः अधिष्ठानं शरीरम्~, तथा कर्ता उपाधिलक्षणः भोक्ता, करणं च श्रोत्रादि शब्दाद्युपलब्धये पृथग्विधं नानाप्रकारं तत् द्वादशसङ्ख्यं विविधाश्च पृथक्चेष्टाः वायवीयाः प्राणापानाद्याः दैवं चैव दैवमेव च अत्र एतेषु चतुर्षु पञ्चमं पञ्चानां पूरणम् आदित्यादि चक्षुराद्यनुग्राहकम्~॥~१४~॥\par
 \begin{center}{\bfseries शरीरवाङ्मनोभिर्य—\\ त्कर्म प्रारभते नरः~।\\न्याय्यं वा विपरीतं वा\\ पञ्चैते तस्य हेतवः~॥~१५~॥}\end{center} 
शरीरवाङ्मनोभिः यत् कर्म त्रिभिः एतैः प्रारभते निर्वर्तयति नरः, न्याय्यं वा धर्म्यं शास्त्रीयम्~, विपरीतं वा अशास्त्रीयम् अधर्म्यं यच्चापि निमिषितचेष्टितादि जीवनहेतुः तदपि पूर्वकृतधर्माधर्मयोरेव कार्यमिति न्याय्यविपरीतयोरेव ग्रहणेन गृहीतम्~, पञ्च एते यथोक्ताः तस्य सर्वस्यैव कर्मणो हेतवः कारणानि~॥~} 
ननु एतानि अधिष्ठानादीनि सर्वकर्मणां निर्वर्तकानि~। कथम् उच्यते ‘शरीरवाङ्मनोभिः यत् कर्म प्रारभते’ इति~? नैष दोषः~; विधिप्रतिषेधलक्षणं सर्वं कर्म शरीरादित्रयप्रधानम्~; तदङ्गतया दर्शनश्रवणादि च जीवनलक्षणं त्रिधैव राशीकृतम् उच्यते शरीरादिभिः आरभ्यते इति~। फलकालेऽपि तत्प्रधानैः साधनैः भुज्यते इति पञ्चानामेव हेतुत्वं न विरुध्यते इति~॥~१५~॥\par
 \begin{center}{\bfseries तत्रैवं सति कर्तारमात्मानं केवलं तु यः~।\\पश्यत्यकृतबुद्धित्वान्न स पश्यति दुर्मतिः~॥~१६~॥}\end{center} 
तत्र इति प्रकृतेन सम्बध्यते~। एवं सति एवं यथोक्तैः पञ्चभिः हेतुभिः निर्वर्त्ये सति कर्मणि~। तत्रैवं सति इति दुर्मतित्वस्य हेतुत्वेन सम्बध्यते~। तत्र एतेषु आत्मानन्यत्वेन अविद्यया परिकल्पितैः क्रियमाणस्य कर्मणः ‘अहमेव कर्ता’ इति कर्तारम् आत्मानं केवलं शुद्धं तु यः पश्यति अविद्वान्~; कस्मात्~? वेदान्ताचार्योपदेशन्यायैः अकृतबुद्धित्वात् असंस्कृतबुद्धित्वात्~; योऽपि देहादिव्यतिरिक्तात्मवादी आत्मानमेव केवलं कर्तारं पश्यति, असावपि अकृतबुद्धिः~; अतः अकृतबुद्धित्वात् न सः पश्यति आत्मनः तत्त्वं कर्मणो वा इत्यर्थः~। अतः दुर्मतिः, कुत्सिता विपरीता दुष्टा अजस्रं जननमरणप्रतिपत्तिहेतुभूता मतिः अस्य इति दुर्मतिः~। सः पश्यन्नपि न पश्यति, यथा तैमिरिकः अनेकं चन्द्रम्~, यथा वा अभ्रेषु धावत्सु चन्द्रं धावन्तम्~, यथा वा वाहने उपविष्टः अन्येषु धावत्सु आत्मानं धावन्तम्~॥~१६~॥\par
 कः पुनः सुमतिः यः सम्यक् पश्यतीति, उच्यते —} 
\begin{center}{\bfseries यस्य नाहङ्कृतो भावो बुद्धिर्यस्य न लिप्यते~।\\हत्वापि स इमांल्लोकान्न हन्ति न निबध्यते~॥~१७~॥}\end{center} 
यस्य शास्त्राचार्योपदेशन्यायसंस्कृतात्मनः न भवति अहङ्कृतः ‘अहं कर्ता’ इत्येवंलक्षणः भावः भावना प्रत्ययः — एते एव पञ्च अधिष्ठानादयः अविद्यया आत्मनि कल्पिताः सर्वकर्मणां कर्तारः, न अहम्~, अहं तु तद्व्यापाराणां साक्षिभूतः ‘अप्राणो ह्यमनाः शुभ्रो ह्यक्षरात्परतः परः’\footnote{मु. उ. २~। १~। २} केवलः अविक्रियः इत्येवं पश्यतीति एतत् — बुद्धिः अन्तःकरणं यस्य आत्मनः उपाधिभूता न लिप्यते न अनुशयिनी भवति — ‘इदमहमकार्षम्~, तेन अहं नरकं गमिष्यामि’ इत्येवं यस्य बुद्धिः न लिप्यते — सः सुमतिः, सः पश्यति~। हत्वा अपि सः इमान् लोकान्~, सर्वान् इमान् प्राणिनः इत्यर्थः, न हन्ति हननक्रियां न करोति, न निबध्यते नापि तत्कार्येण अधर्मफलेन सम्बध्यते~॥~} 
ननु हत्वापि न हन्ति इति विप्रतिषिद्धम् उच्यते यद्यपि स्तुतिः~। नैष दोषः, लौकिकपारमार्थिकदृष्ट्यपेक्षया तदुपपत्तेः~। देहाद्यात्मबुद्ध्या ‘हन्ता अहम्’ इति लौकिकीं दृष्टिम् आश्रित्य ‘हत्वापि’ इति आह~। यथादर्शितां पारमार्थिकीं दृष्टिम् आश्रित्य ‘न हन्ति न निबध्यते’ इति~। एतत् उभयम् उपपद्यते एव~॥~} 
ननु अधिष्ठानादिभिः सम्भूय करोत्येव आत्मा, ‘कर्तारमात्मानं केवलं तु’\footnote{भ. गी. १८~। १६} इति केवलशब्दप्रयोगात्~। नैष दोषः, आत्मनः अविक्रियस्वभावत्वे अधिष्ठानादिभिः, संहतत्वानुपपत्तेः~। विक्रियावतो हि अन्यैः संहननं सम्भवति, संहत्य वा कर्तृत्वं स्यात्~। न तु अविक्रियस्य आत्मनः केनचित् संहननम् अस्ति इति न सम्भूय कर्तृत्वम् उपपद्यते~। अतः केवलत्वम् आत्मनः स्वाभाविकमिति केवलशब्दः अनुवादमात्रम्~। अविक्रियत्वं च आत्मनः श्रुतिस्मृतिन्यायप्रसिद्धम्~। ‘अविकार्योऽयमुच्यते’\footnote{भ. गी. २~। २५} ‘गुणैरेव कर्माणि क्रियन्ते’\footnote{भ. गी. ३~। २७} ‘शरीरस्थोऽपि न करोति’\footnote{भ. गी. १३~। ३१} इत्यादि असकृत् उपपादितं गीतास्वेव तावत्~। श्रुतिषु च ‘ध्यायतीव लेलायतीव’\footnote{बृ. उ. ४~। ३~। ७} इत्येवमाद्यासु~। न्यायतश्च — निरवयवम् अपरतन्त्रम् अविक्रियम् आत्मतत्त्वम् इति राजमार्गः~। विक्रियावत्त्वाभ्युपगमेऽपि आत्मनः स्वकीयैव विक्रिया स्वस्य भवितुम् अर्हति, न अधिष्ठानादीनां कर्माणि आत्मकर्तृकाणि स्युः~। न हि परस्य कर्म परेण अकृतम् आगन्तुम् अर्हति~। यत्तु अविद्यया गमितम्~, न तत् तस्य~। यथा रजतत्वं न शुक्तिकायाः~; यथा वा तलमलिनत्वं बालैः गमितम् अविद्यया, न आकाशस्य, तथा अधिष्ठानादिविक्रियापि तेषामेव, न आत्मनः~। तस्मात् युक्तम् उक्तम् ‘अहङ्कृतत्वबुद्धिलेपाभावात् विद्वान् न हन्ति न निबध्यते’ इति~। ‘नायं हन्ति न हन्यते’\footnote{भ. गी. २~। १९} इति प्रतिज्ञाय ‘न जायते’\footnote{भ. गी. २~। २०} इत्यादिहेतुवचनेन अविक्रियत्वम् आत्मनः उक्त्वा, ‘वेदाविनाशिनम्’\footnote{भ. गी. २~। २१} इति विदुषः कर्माधिकारनिवृत्तिं शास्त्रादौ सङ्क्षेपतः उक्त्वा, मध्ये प्रसारितां तत्र तत्र प्रसङ्गं कृत्वा इह उपसंहरति शास्त्रार्थपिण्डीकरणाय ‘विद्वान् न हन्ति न निबध्यते’ इति~। एवं च सति देहभृत्त्वाभिमानानुपपत्तौ अविद्याकृताशेषकर्मसंन्यासोपपत्तेः संन्यासिनाम् अनिष्टादि त्रिविधं कर्मणः फलं न भवति इति उपपन्नम्~; तद्विपर्ययाच्च इतरेषां भवति इत्येतच्च अपरिहार्यम् इति एषः गीताशास्त्रार्थः उपसंहृतः~। स एषः सर्ववेदार्थसारः निपुणमतिभिः पण्डितैः विचार्य प्रतिपत्तव्यः इति तत्र तत्र प्रकरणविभागेन दर्शितः अस्माभिः शास्त्रन्यायानुसारेण~॥~१७~॥\par
 अथ इदानीं कर्मणां प्रवर्तकम् उच्यते —} 
\begin{center}{\bfseries ज्ञानं ज्ञेयं परिज्ञाता त्रिविधा कर्मचोदना~।\\करणं कर्म कर्तेति त्रिविधः कर्मसङ्ग्रहः~॥~१८~॥}\end{center} 
ज्ञानं ज्ञायते अनेन इति सर्वविषयम् अविशेषेण उच्यते~। तथा ज्ञेयं ज्ञातव्यम्~, तदपि सामान्येनैव सर्वम् उच्यते~। तथा परिज्ञाता उपाधिलक्षणः अविद्याकल्पितः भोक्ता~। इति एतत् त्रयम् अविशेषेण सर्वकर्मणां प्रवर्तिका त्रिविधा त्रिप्रकारा कर्मचोदना~। ज्ञानादीनां हि त्रयाणां संनिपाते हानोपादानादिप्रयोजनः सर्वकर्मारम्भः स्यात्~। ततः पञ्चभिः अधिष्ठानादिभिः आरब्धं वाङ्मनःकायाश्रयभेदेन त्रिधा राशीभूतं त्रिषु करणादिषु सङ्गृह्यते इत्येतत् उच्यते — करणं क्रियते अनेन इति बाह्यं श्रोत्रादि, अन्तःस्थं बुद्ध्यादि, कर्म ईप्सिततमं कर्तुः क्रियया व्याप्यमानम्~, कर्ता करणानां व्यापारयिता उपाधिलक्षणः, इति त्रिविधः त्रिप्रकारः कर्मसङ्ग्रहः, सङ्गृह्यते अस्मिन्निति सङ्ग्रहः, कर्मणः सङ्ग्रहः कर्मसङ्ग्रहः, कर्म एषु हि त्रिषु समवैति, तेन अयं त्रिविधः कर्मसङ्ग्रहः~॥~१८~॥\par
 अथ इदानीं क्रियाकारकफलानां सर्वेषां गुणात्मकत्वात् सत्त्वरजस्तमोगुणभेदतः त्रिविधः भेदः वक्तव्य इति आरभ्यते —} 
\begin{center}{\bfseries ज्ञानं कर्म च कर्ता च\\ त्रिधैव गुणभेदतः~।\\प्रोच्यते गुणसङ्ख्याने\\ यथावच्छृणु तान्यपि~॥~१९~॥}\end{center} 
ज्ञानं कर्म च, कर्म क्रिया, न कारकं पारिभाषिकम् ईप्सिततमं कर्म, कर्ता च निर्वर्तकः क्रियाणां त्रिधा एव, अवधारणं गुणव्यतिरिक्तजात्यन्तराभावप्रदर्शनार्थं गुणभेदतः सत्त्वादिभेदेन इत्यर्थः~। प्रोच्यते कथ्यते गुणसङ्ख्याने कापिले शास्त्रे तदपि गुणसङ्ख्यानशास्त्रं गुणभोक्तृविषये प्रमाणमेव~। परमार्थब्रह्मैकत्वविषये यद्यपि विरुध्यते, तथापि ते हि कापिलाः गुणगौणव्यापारनिरूपणे अभियुक्ताः इति तच्छास्त्रमपि वक्ष्यमाणार्थस्तुत्यर्थत्वेन उपादीयते इति न विरोधः~। यथावत् यथान्यायं यथाशास्त्रं शृणु तान्यपि ज्ञानादीनि तद्भेदजातानि गुणभेदकृतानि शृणु, वक्ष्यमाणे अर्थे मनःसमाधिं कुरु इत्यर्थः~॥~१९~॥\par
 ज्ञानस्य तु तावत् त्रिविधत्वम् उच्यते —} 
\begin{center}{\bfseries सर्वभूतेषु येनैकं\\ भावमव्ययमीक्षते~।\\अविभक्तं विभक्तेषु\\ तज्ज्ञानं विद्धि सात्त्विकम्~॥~२०~॥}\end{center} 
सर्वभूतेषु अव्यक्तादिस्थावरान्तेषु भूतेषु येन ज्ञानेन एकं भावं वस्तु — भावशब्दः वस्तुवाची, एकम् आत्मवस्तु इत्यर्थः~; अव्ययं न व्येति स्वात्मना स्वधर्मेण वा, कूटस्थम् इत्यर्थः~; ईक्षते पश्यति येन ज्ञानेन, तं च भावम् अविभक्तं प्रतिदेहं विभक्तेषु देहभेदेषु न विभक्तं तत् आत्मवस्तु, व्योमवत् निरन्तरमित्यर्थः~; तत् ज्ञानं साक्षात् सम्यग्दर्शनम् अद्वैतात्मविषयं सात्त्विकं विद्धि इति~॥~२०~॥\par
 यानि द्वैतदर्शनानि तानि असम्यग्भूतानि राजसानि तामसानि च इति न साक्षात् संसारोच्छित्तये भवन्ति —} 
\begin{center}{\bfseries पृथक्त्वेन तु यज्ज्ञानं\\ नानाभावान्पृथग्विधान्~।\\वेत्ति सर्वेषु भूतेषु\\ तज्ज्ञानं विद्धि राजसम्~॥~२१~॥}\end{center} 
पृथक्त्वेन तु भेदेन प्रतिशरीरम् अन्यत्वेन यत् ज्ञानं नानाभावान् भिन्नान् आत्मनः पृथग्विधान् पृथक्प्रकारान् भिन्नलक्षणान् इत्यर्थः, वेत्ति विजानाति यत् ज्ञानं सर्वेषु भूतेषु, ज्ञानस्य कर्तृत्वासम्भवात् येन ज्ञानेन वेत्ति इत्यर्थः, तत् ज्ञानं विद्धि राजसं रजोगुणनिर्वृत्तम्~॥~२१~॥\par
 \begin{center}{\bfseries यत्तु कृत्स्नवदेकस्मिन्कार्ये सक्तमहैतुकम्~।\\अतत्त्वार्थवदल्पं च तत्तामसमुदाहृतम्~॥~२२~॥}\end{center} 
यत् ज्ञानं कृत्स्नवत् समस्तवत् सर्वविषयमिव एकस्मिन् कार्ये देहे बहिर्वा प्रतिमादौ सक्तम् ‘एतावानेव आत्मा ईश्वरो वा, न अतः परम् अस्ति’ इति, यथा नग्नक्षपणकादीनां शरीरान्तर्वर्ती देहपरिमाणो जीवः, ईश्वरो वा पाषाणदार्वादिमात्रम्~, इत्येवम् एकस्मिन् कार्ये सक्तम् अहैतुकं हेतुवर्जितं निर्युक्तिकम्~, अतत्त्वार्थवत् अयथाभूतार्थवत्~, यथाभूतः अर्थः तत्त्वार्थः, सः अस्य ज्ञेयभूतः अस्तीति तत्त्वार्थवत्~, न तत्त्वार्थवत् अतत्त्वार्थवत्~; अहैतुकत्वादेव अल्पं च, अल्पविषयत्वात् अल्पफलत्वाद्वा~। तत् तामसम् उदाहृतम्~। तामसानां हि प्राणिनाम् अविवेकिनाम् ईदृशं ज्ञानं दृश्यते~॥~२२~॥\par
 अथ इदानीं कर्मणः त्रैविध्यम् उच्यते —} 
\begin{center}{\bfseries नियतं सङ्गरहितमरागद्वेषतःकृतम्~।\\अफलप्रेप्सुना कर्म यत्तत्सात्त्विकमुच्यते~॥~२३~॥}\end{center} 
नियतं नित्यं सङ्गरहितम् आसक्तिवर्जितम् अरागद्वेषतःकृतं रागप्रयुक्तेन द्वेषप्रयुक्तेन च कृतं रागद्वेषतःकृतम्~, तद्विपरीतम् अरागद्वेषतःकृतम्~, अफलप्रेप्सुना फलं प्रेप्सतीति फलप्रेप्सुः फलतृष्णः तद्विपरीतेन अफलप्रेप्सुना कर्त्रा कृतं कर्म यत्~, तत् सात्त्विकम् उच्यते~॥~२३~॥\par
 \begin{center}{\bfseries यत्तु कामेप्सुना कर्म साहङ्कारेण वा पुनः~।\\क्रियते बहुलायासं तद्राजसमुदाहृतम्~॥~२४~॥}\end{center} 
यत्तु कामेप्सुना कर्मफलप्रेप्सुना इत्यर्थः, कर्म साहङ्कारेण इति न तत्त्वज्ञानापेक्षया~। किं तर्हि~? लौकिकश्रोत्रियनिरहङ्कारापेक्षया~। यो हि परमार्थनिरहङ्कारः आत्मवित्~, न तस्य कामेप्सुत्वबहुलायासकर्तृत्वप्राप्तिः अस्ति~। सात्त्विकस्यापि कर्मणः अनात्मवित् साहङ्कारः कर्ता, किमुत राजसतामसयोः~। लोके अनात्मविदपि श्रोत्रियो निरहङ्कारः उच्यते ‘निरहङ्कारः अयं ब्राह्मणः’ इति~। तस्मात् तदपेक्षयैव ‘साहङ्कारेण वा’ इति उक्तम्~। पुनःशब्दः पादपूरणार्थः~। क्रियते बहुलायासं कर्त्रा महता आयासेन निर्वर्त्यते, तत् कर्म राजसम् उदाहृतम्~॥~२४~॥\par
 \begin{center}{\bfseries अनुबन्धं क्षयं हिंसामनपेक्ष्य च पौरुषम्~।\\मोहादारभ्यते कर्म यत्तत्तामसमुच्यते~॥~२५~॥}\end{center} 
अनुबन्धं पश्चाद्भावि यत् वस्तु सः अनुबन्धः उच्यते तं च अनुबन्धम्~, क्षयं यस्मिन् कर्मणि क्रियमाणे शक्तिक्षयः अर्थक्षयो वा स्यात् तं क्षयम्~, हिंसां प्राणिबाधां च~; अनपेक्ष्य च पौरुषं पुरुषकारम् ‘शक्नोमि इदं कर्म समापयितुम्’ इत्येवम् आत्मसामर्थ्यम्~, इत्येतानि अनुबन्धादीनि अनपेक्ष्य पौरुषान्तानि मोहात् अविवेकतः आरभ्यते कर्म यत्~, तत् तामसं तमोनिर्वृत्तम् उच्यते~॥~२५~॥\par
 इदानीं कर्तृभेदः उच्यते —} 
\begin{center}{\bfseries मुक्तसङ्गोऽनहंवादी\\ धृत्युत्साहसमन्वितः~।\\सिद्ध्यसिद्ध्योर्निर्विकारः\\ कर्ता सात्त्विक उच्यते~॥~२६~॥}\end{center} 
मुक्तसङ्गः मुक्तः परित्यक्तः सङ्गः येन सः मुक्तसङ्गः, अनहंवादी न अहंवदनशीलः, धृत्युत्साहसमन्वितः धृतिः धारणम् उत्साहः उद्यमः ताभ्यां समन्वितः संयुक्तः धृत्युत्साहसमन्वितः, सिद्ध्यसिद्ध्योः क्रियमाणस्य कर्मणः फलसिद्धौ असिद्धौ च सिद्ध्यसिद्ध्योः निर्विकारः, केवलं शास्त्रप्रमाणेन प्रयुक्तः न फलरागादिना यः सः निर्विकारः उच्यते~। एवंभूतः कर्ता यः सः सात्त्विकः उच्यते~॥~२६~॥\par
 \begin{center}{\bfseries रागी कर्मफलप्रेप्सुर्लुब्धो हिंसात्मकोऽशुचिः~।\\हर्षशोकान्वितः कर्ता राजसः परिकीर्तितः~॥~२७~॥}\end{center} 
रागी रागः अस्य अस्तीति रागी, कर्मफलप्रेप्सुः कर्मफलार्थी इत्यर्थः, लुब्धः परद्रव्येषु सञ्जाततृष्णः, तीर्थादौ स्वद्रव्यापरित्यागी वा, हिंसात्मकः परपीडाकरस्वभावः, अशुचिः बाह्याभ्यन्तरशौचवर्जितः, हर्षशोकान्वितः इष्टप्राप्तौ हर्षः अनिष्टप्राप्तौ इष्टवियोगे च शोकः ताभ्यां हर्षशोकाभ्याम् अन्वितः संयुक्तः, तस्यैव च कर्मणः सम्पत्तिविपत्तिभ्यां हर्षशोकौ स्याताम्~, ताभ्यां संयुक्तो यः कर्ता सः राजसः परिकीर्तितः~॥~२७~॥\par
 \begin{center}{\bfseries अयुक्तः प्राकृतः स्तब्धः\\ शठो नैकृतिकोऽलसः~।\\विषादी दीर्घसूत्री च\\ कर्ता तामस उच्यते~॥~२८~॥}\end{center} 
अयुक्तः न युक्तः असमाहितः, प्राकृतः अत्यन्तासंस्कृतबुद्धिः बालसमः, स्तब्धः दण्डवत् न नमति कस्मैचित्~, शठः मायावी शक्तिगूहनकारी, नैकृतिकः परविभेदनपरः, अलसः अप्रवृत्तिशीलः कर्तव्येष्वपि, विषादी विषादवान् सर्वदा अवसन्नस्वभावः, दीर्घसूत्री च कर्तव्यानां दीर्घप्रसारणः, सर्वदा मन्दस्वभावः, यत् अद्य श्वो वा कर्तव्यं तत् मासेनापि न करोति, यश्च एवंभूतः, सः कर्ता तामसः उच्यते~॥~२८~॥\par
 \begin{center}{\bfseries बुद्धेर्भेदं धृतेश्चैव गुणतस्त्रिविधं शृणु~।\\प्रोच्यमानमशेषेण पृथक्त्वेन धनञ्जय~॥~२९~॥}\end{center} 
बुद्धेः भेदं धृतेश्चैव भेदं गुणतः सत्त्वादिगुणतः त्रिविधं शृणु इति सूत्रोपन्यासः~। प्रोच्यमानं कथ्यमानम् अशेषेण निरवशेषतः यथावत् पृथक्त्वेन विवेकतः धनञ्जय, दिग्विजये मानुषं दैवं च प्रभूतं धनं जितवान्~, तेन असौ धनञ्जयः अर्जुनः~॥~२९~॥\par
 \begin{center}{\bfseries प्रवृत्तिं च निवृत्तिं च\\ कार्याकार्ये भयाभये~।\\बन्धं मोक्षं च या वेत्ति\\ बुद्धिः सा पार्थ सात्त्विकी~॥~३०~॥}\end{center} 
प्रवृत्तिं च प्रवृत्तिः प्रवर्तनं बन्धहेतुः कर्ममार्गः शास्त्रविहितविषयः, निवृत्तिं च निर्वृत्तिः मोक्षहेतुः संन्यासमार्गः — बन्धमोक्षसमानवाक्यत्वात् प्रवृत्तिनिवृत्ती कर्मसंन्यासमार्गौ इति अवगम्यते — कार्याकार्ये विहितप्रतिषिद्धे लौकिके वैदिके वा शास्त्रबुद्धेः कर्तव्याकर्तव्ये करणाकरणे इत्येतत्~; कस्य~? देशकालाद्यपेक्षया दृष्टादृष्टार्थानां कर्मणाम्~। भयाभये बिभेति अस्मादिति भयं चोरव्याघ्रादि, न भयं अभयम्~, भयं च अभयं च भयाभये, दृष्टादृष्टविषययोः भयाभययोः कारणे इत्यर्थः~। बन्धं सहेतुकं मोक्षं च सहेतुकं या वेत्ति विजानाति बुद्धिः, सा पार्थ सात्त्विकी~। तत्र ज्ञानं बुद्धेः वृत्तिः~; बुद्धिस्तु वृत्तिमती~। धृतिरपि वृत्तिविशेषः एव बुद्धेः~॥~३०~॥\par
 \begin{center}{\bfseries यया धर्ममधर्मं च\\ कार्यं चाकार्यमेव च~।\\ अयथावत्प्रजानाति\\ बुद्धिः सा पार्थ राजसी~॥~३१~॥}\end{center} 
यया धर्मं शास्त्रचोदितम् अधर्मं च तत्प्रतिषिद्धं कार्यं च अकार्यमेव च पूर्वोक्ते एव कार्याकार्ये अयथावत् न यथावत् सर्वतः निर्णयेन न प्रजानाति, बुद्धिः सा पार्थ, राजसी~॥~३१~॥\par
 \begin{center}{\bfseries अधर्मं धर्ममिति या\\ मन्यते तमसावृता~।\\सर्वार्थान्विपरीतांश्च\\ बुद्धिः सा पार्थ तामसी~॥~३२~॥}\end{center} 
अधर्मं प्रतिषिद्धं धर्मं विहितम् इति या मन्यते जानाति तमसा आवृता सती, सर्वार्थान् सर्वानेव ज्ञेयपदार्थान् विपरीतांश्च विपरीतानेव विजानाति, बुद्धिः सा पार्थ, तामसी~॥~३२~॥\par
 \begin{center}{\bfseries धृत्या यया धारयते\\ मनःप्राणेन्द्रियक्रियाः~।\\योगेनाव्यभिचारिण्या\\ धृतिः सा पार्थ सात्त्विकी~॥~३३~॥}\end{center} 
धृत्या यया — अव्यभिचारिण्या इति व्यवहितेन सम्बन्धः, धारयते~; किम्~? मनःप्राणेन्द्रियक्रियाः मनश्च प्राणाश्च इन्द्रियाणि च मनःप्राणेन्द्रियाणि, तेषां क्रियाः चेष्टाः, ताः उच्छास्त्रमार्गप्रवृत्तेः धारयते धारयति — धृत्या हि धार्यमाणाः उच्छास्त्रमार्गविषयाः न भवन्ति — योगेन समाधिना, अव्यभिचारिण्या, नित्यसमाध्यनुगतया इत्यर्थः~। एतत् उक्तं भवति — अव्यभिचारिण्या धृत्या मनःप्राणेन्द्रियक्रियाः धार्यमाणाः योगेन धारयतीति~। या एवंलक्षणा धृतिः, सा पार्थ, सात्त्विकी~॥~३३~॥\par
 \begin{center}{\bfseries यया तु धर्मकामार्था—\\ न्धृत्या धारयतेऽर्जुन~।\\प्रसङ्गेन फलाकाङ्क्षी\\ धृतिः सा पार्थ राजसी~॥~३४~॥}\end{center} 
यया तु धर्मकामार्थान् धर्मश्च कामश्च अर्थश्च धर्मकामार्थाः तान् धर्मकामार्थान् धृत्या यया धारयते मनसि नित्यमेव कर्तव्यरूपान् अवधारयति हे अर्जुन, प्रसङ्गेन यस्य यस्य धर्मादेः धारणप्रसङ्गः तेन तेन प्रसङ्गेन फलाकाङ्क्षी च भवति यः पुरुषः, तस्य धृतिः या, सा पार्थ, राजसी~॥~३४~॥\par
 \begin{center}{\bfseries यया स्वप्नं भयं शोकं\\ विषादं मदमेव च~।\\न विमुञ्चति दुर्मेधा\\ धृतिः सा तामसी मता~॥~३५~॥}\end{center} 
यया स्वप्नं निद्रां भयं त्रासं शोकं विषादं विषण्णतां मदं विषयसेवाम् आत्मनः बहुमन्यमानः मत्त इव मदम् एव च मनसि नित्यमेव कर्तव्यरूपतया कुर्वन् न विमुञ्चति धारयत्येव दुर्मेधाः कुत्सितमेधाः पुरुषः यः, तस्य धृतिः या, सा तामसी मता~॥~३५~॥\par
 गुणभेदेन क्रियाणां कारकाणां च त्रिविधो भेदः उक्तः~। अथ इदानीं फलस्य सुखस्य त्रिविधो भेदः उच्यते —} 
\begin{center}{\bfseries सुखं त्विदानीं त्रिविधं\\ शृणु मे भरतर्षभ~।\\अभ्यासाद्रमते यत्र\\ दुःखान्तं च निगच्छति~॥~३६~॥}\end{center} 
सुखं तु इदानीं त्रिविधं शृणु, समाधानं कुरु इत्येतत्~, मे मम भरतर्षभ~। अभ्यासात् परिचयात् आवृत्तेः रमते रतिं प्रतिपद्यते यत्र यस्मिन् सुखानुभवे दुःखान्तं च दुःखावसानं दुःखोपशमं च निगच्छति निश्चयेन प्राप्नोति~॥~३६~॥\par
 \begin{center}{\bfseries यत्तदग्रे विषमिव\\ परिणामेऽमृतोपमम्~।\\तत्सुखं सात्त्विकं प्रोक्त—\\ मात्मबुद्धिप्रसादजम्~॥~३७~॥}\end{center} 
यत् तत् सुखम् अग्रे पूर्वं प्रथमसंनिपाते ज्ञानवैराग्यध्यानसमाध्यारम्भे अत्यन्तायासपूर्वकत्वात् विषमिव दुःखात्मकं भवति, परिणामे ज्ञानवैराग्यादिपरिपाकजं सुखम् अमृतोपमम्~, तत् सुखं सात्त्विकं प्रोक्तं विद्वद्भिः, आत्मनः बुद्धिः आत्मबुद्धिः, आत्मबुद्धेः प्रसादः नैर्मल्यं सलिलस्य इव स्वच्छता, ततः जातं आत्मबुद्धिप्रसादजम्~। आत्मविषया वा आत्मावलम्बना वा बुद्धिः आत्मबुद्धिः, तत्प्रसादप्रकर्षाद्वा जातमित्येतत्~। तस्मात् सात्त्विकं तत्~॥~३७~॥\par
 \begin{center}{\bfseries विषयेन्द्रियसंयोगाद्यत्तदग्रेऽमृतोपमम्~।\\परिणामे विषमिव तत्सुखं राजसं स्मृतम्~॥~३८~॥}\end{center} 
विषयेन्द्रियसंयोगात् जायते यत् सुखम् तत् सुखम् अग्रे प्रथमक्षणे अमृतोपमम् अमृतसमम्~, परिणामे विषमिव, बलवीर्यरूपप्रज्ञामेधाधनोत्साहहानिहेतुत्वात् अधर्मतज्जनितनरकादिहेतुत्वाच्च परिणामे तदुपभोगपरिणामान्ते विषमिव, तत् सुखं राजसं स्मृतम्~॥~३८~॥\par
 \begin{center}{\bfseries यदग्रे चानुबन्धे च सुखं मोहनमात्मनः~।\\निद्रालस्यप्रमादोत्थं तत्तामसमुदाहृतम्~॥~३९~॥}\end{center} 
यत् अग्रे च अनुबन्धे च अवसानोत्तरकाले च सुखं मोहनं मोहकरम् आत्मनः निद्रालस्यप्रमादोत्थं निद्रा च आलस्यं च प्रमादश्च तेभ्यः समुत्तिष्ठतीति निद्रालस्यप्रमादोत्थम्~, तत् तामसम् उदाहृतम्~॥~३९~॥\par
 अथ इदानीं प्रकरणोपसंहारार्थः श्लोकः आरभ्यते —} 
\begin{center}{\bfseries न तदस्ति पृथिव्यां वा दिवि देवेषु वा पुनः~।\\सत्त्वं प्रकृतिजैर्मुक्तं यदेभिः स्यात्त्रिभिर्गुणैः~॥~४०~॥}\end{center} 
न तत् अस्ति तत् नास्ति पृथिव्यां वा मनुष्यादिषु सत्त्वं प्राणिजातम् अन्यद्वा अप्राणि, दिवि देवेषु वा पुनः सत्त्वम्~, प्रकृतिजैः प्रकृतितः जातैः एभिः त्रिभिः गुणैः सत्त्वादिभिः मुक्तं परित्यक्तं यत् स्यात्~, न तत् अस्ति इति पूर्वेण सम्बन्धः~॥~४०~॥\par
 सर्वः संसारः क्रियाकारकफललक्षणः सत्त्वरजस्तमोगुणात्मकः अविद्यापरिकल्पितः समूलः अनर्थः उक्तः, वृक्षरूपकल्पनया च ‘ऊर्ध्वमूलम्’\footnote{भ. गी. १५~। १} इत्यादिना,तं च  ‘असङ्गशस्त्रेण दृढेन च्छित्त्वा' 'ततः पदं तत्परिमार्गितव्यम्’\footnote{भ. गी. १५~। ३},\footnote{भ. गी. १५~। ४} इति च उक्तम्~। तत्र च सर्वस्य त्रिगुणात्मकत्वात् संसारकारणनिवृत्त्यनुपपत्तौ प्राप्तायाम्~, यथा तन्निवृत्तिः स्यात् तथा वक्तव्यम्~, सर्वश्च गीताशास्त्रार्थः उपसंहर्तव्यः, एतावानेव च सर्ववेदस्मृत्यर्थः पुरुषार्थम् इच्छद्भिः अनुष्ठेयः इत्येवमर्थम् ‘ब्राह्मणक्षत्रियविशाम्’ इत्यादिः आरभ्यते —} 
\begin{center}{\bfseries ब्राह्मणक्षत्रियविशां शूद्राणां च परन्तप~।\\कर्माणि प्रविभक्तानि स्वभावप्रभवैर्गुणैः~॥~४१~॥}\end{center} 
ब्राह्मणाश्च क्षत्रियाश्च विशश्च ब्राह्मणक्षत्रियविशः, तेषां ब्राह्मणक्षत्रियविशां शूद्राणां च — शूद्राणाम् असमासकरणम् एकजातित्वे सति वेदानधिकारात् — हे परन्तप, कर्माणि प्रविभक्तानि इतरेतरविभागेन व्यवस्थापितानि~। केन~? स्वभावप्रभवैः गुणैः, स्वभावः ईश्वरस्य प्रकृतिः त्रिगुणात्मिका माया सा प्रभवः येषां गुणानां ते स्वभावप्रभवाः, तैः, शमादीनि कर्माणि प्रविभक्तानि ब्राह्मणादीनाम्~। अथवा ब्राह्मणस्वभावस्य सत्त्वगुणः प्रभवः कारणम्~, तथा क्षत्रियस्वभावस्य सत्त्वोपसर्जनं रजः प्रभवः, वैश्यस्वभावस्य तमउपसर्जनं रजः प्रभवः, शूद्रस्वभावस्य रजउपसर्जनं तमः प्रभवः, प्रशान्त्यैश्वर्येहामूढतास्वभावदर्शनात् चतुर्णाम्~। अथवा, जन्मान्तरकृतसंस्कारः प्राणिनां वर्तमानजन्मनि स्वकार्याभिमुखत्वेन अभिव्यक्तः स्वभावः, सः प्रभवो येषां गुणानां ते स्वभावप्रभवाः गुणाः~; गुणप्रादुर्भावस्य निष्कारणत्वानुपपत्तेः~। ‘स्वभावः कारणम्’ इति च कारणविशेषोपादानम्~। एवं स्वभावप्रभवैः प्रकृतिभवैः सत्त्वरजस्तमोभिः गुणैः स्वकार्यानुरूपेण शमादीनि कर्माणि प्रविभक्तानि~॥~} 
ननु शास्त्रप्रविभक्तानि शास्त्रेण विहितानि ब्राह्मणादीनां शमादीनि कर्माणि~; कथम् उच्यते सत्त्वादिगुणप्रविभक्तानि इति~? नैष दोषः~; शास्त्रेणापि ब्राह्मणादीनां सत्त्वादिगुणविशेषापेक्षयैव शमादीनि कर्माणि प्रविभक्तानि, न गुणानपेक्षया, इति शास्त्रप्रविभक्तान्यपि कर्माणि गुणप्रविभक्तानि इति उच्यते~॥~४१~॥\par
 कानि पुनः तानि कर्माणि इति, उच्यते —} 
\begin{center}{\bfseries शमो दमस्तपः शौचं\\ क्षान्तिरार्जवमेव च~।\\ज्ञानं विज्ञानमास्तिक्यं\\ ब्रह्मकर्म स्वभावजम्~॥~४२~॥}\end{center} 
शमः दमश्च यथाव्याख्यातार्थौ, तपः यथोक्तं शारीरादि, शौचं व्याख्यातम्~, क्षान्तिः क्षमा, आर्जवम् ऋजुता एव च ज्ञानं विज्ञानम्~, आस्तिक्यम् आस्तिकभावः श्रद्दधानता आगमार्थेषु, ब्रह्मकर्म ब्राह्मणजातेः कर्म स्वभावजम् — यत् उक्तं स्वभावप्रभवैर्गुणैः प्रविभक्तानि इति तदेवोक्तं स्वभावजम् इति~॥~४२~॥\par
 \begin{center}{\bfseries शौर्यं तेजो धृतिर्दाक्ष्यं युद्धे चाप्यपलायनम्~।\\दानमीश्वरभावश्च क्षात्रं कर्म स्वभावजम्~॥~४३~॥}\end{center} 
शौर्यं शूरस्य भावः, तेजः प्रागल्भ्यम्~, धृतिः धारणम्~, सर्वावस्थासु अनवसादः भवति यया धृत्या उत्तम्भितस्य, दाक्ष्यं दक्षस्य भावः, सहसा प्रत्युत्पन्नेषु कार्येषु अव्यामोहेन प्रवृत्तिः, युद्धे चापि अपलायनम् अपराङ्मुखीभावः शत्रुभ्यः, दानं देयद्रव्येषु मुक्तहस्तता, ईश्वरभावश्च ईश्वरस्य भावः, प्रभुशक्तिप्रकटीकरणम् ईशितव्यान् प्रति, क्षात्रं कर्म क्षत्रियजातेः विहितं कर्म क्षात्रं कर्म स्वभावजम्~॥~४३~॥\par
 \begin{center}{\bfseries कृषिगौरक्ष्यवाणिज्यं वैश्यकर्म स्वभावजम्~।\\परिचर्यात्मकं कर्म शूद्रस्यापि स्वभावजम्~॥~४४~॥}\end{center} 
कृषिगौरक्ष्यवाणिज्यं कृषिश्च गौरक्ष्यं च वाणिज्यं च कृषिगौरक्ष्यवाणिज्यम्~, कृषिः भूमेः विलेखनम्~, गौरक्ष्यं गाः रक्षतीति गोरक्षः तस्य भावः गौरक्ष्यम्~, पाशुपाल्यम् इत्यर्थः, वाणिज्यं वणिक्कर्म क्रयविक्रयादिलक्षणं वैश्यकर्म वैश्यजातेः कर्म वैश्यकर्म स्वभावजम्~। परिचर्यात्मकं शुश्रूषास्वभावं कर्म शूद्रस्यापि स्वभावजम्~॥~४४~॥\par
 एतेषां जातिविहितानां कर्मणां सम्यगनुष्ठितानां स्वर्गप्राप्तिः फलं स्वभावतः, ‘वर्णा आश्रमाश्च स्वकर्मनिष्ठाः प्रेत्य कर्मफलमनुभूय ततः शेषेण विशिष्टदेशजातिकुलधर्मायुःश्रुतवृत्तवित्तसुखमेधसो जन्म प्रतिपद्यन्ते’\footnote{गौ. ध. २~। २~। २९},\footnote{मै. गौ. ध. ११~। ३१} इत्यादिस्मृतिभ्यः~; पुराणे च वर्णिनाम् आश्रमिणां च लोकफलभेदविशेषस्मरणात्~। कारणान्तरात्तु इदं वक्ष्यमाणं फलम् —} 
\begin{center}{\bfseries स्वे स्वे कर्मण्यभिरतः\\ संसिद्धिं लभते नरः~।\\स्वकर्मनिरतः सिद्धिं\\ यथा विन्दति तच्छृणु~॥~४५~॥}\end{center} 
स्वे स्वे यथोक्तलक्षणभेदे कर्मणि अभिरतः तत्परः संसिद्धिं स्वकर्मानुष्ठानात् अशुद्धिक्षये सति कायेन्द्रियाणां ज्ञाननिष्ठायोग्यतालक्षणां संसिद्धिं लभते प्राप्नोति नरः अधिकृतः पुरुषः~; किं स्वकर्मानुष्ठानत एव साक्षात् संसिद्धिः~? न~; कथं तर्हि~? स्वकर्मनिरतः सिद्धिं यथा येन प्रकारेण विन्दति, तत् शृणु~॥~४५~॥\par
 \begin{center}{\bfseries यतः प्रवृत्तिर्भूतानां\\ येन सर्वमिदं ततम्~।\\स्वकर्मणा तमभ्यर्च्य\\ सिद्धिं विन्दति मानवः~॥~४६~॥}\end{center} 
यतः यस्मात् प्रवृत्तिः उत्पत्तिः चेष्टा वा यस्मात् अन्तर्यामिणः ईश्वरात् भूतानां प्राणिनां स्यात्~, येन ईश्वरेण सर्वम् इदं ततं जगत् व्याप्तं स्वकर्मणा पूर्वोक्तेन प्रतिवर्णं तम् ईश्वरम् अभ्यर्च्य पूजयित्वा आराध्य केवलं ज्ञाननिष्ठायोग्यतालक्षणां सिद्धिं विन्दति मानवः मनुष्यः~॥~४६~॥\par
 यतः एवम्~, अतः —} 
\begin{center}{\bfseries श्रेयान्स्वधर्मो विगुणः\\ परधर्मात्स्वनुष्ठितात्~।\\स्वभावनियतं कर्म\\ कुर्वन्नाप्नोति किल्बिषम्~॥~४७~॥}\end{center} 
श्रेयान् प्रशस्यतरः स्वो धर्मः स्वधर्मः, विगुणोऽपि इति अपिशब्दो द्रष्टव्यः, परधर्मात्~। स्वभावनियतं स्वभावेन नियतम्~, यदुक्तं स्वभावजमिति, तदेवोक्तं स्वभावनियतम् इति~; यथा विषजातस्य कृमेः विषं न दोषकरम्~, तथा स्वभावनियतं कर्म कुर्वन् न आप्नोति किल्बिषं पापम्~॥~४७~॥\par
 स्वभावनियतं कर्म कुर्वाणो विषजः इव कृमिः किल्बिषं न आप्नोतीति उक्तम्~; परधर्मश्च भयावहः इति, अनात्मज्ञश्च ‘न हि कश्चित्क्षणमपि अकर्मकृत्तिष्ठति’\footnote{भ. गी. ३~। ५} इति~। अतः —} 
\begin{center}{\bfseries सहजं कर्म कौन्तेय\\ सदोषमपि न त्यजेत्~।\\सर्वारम्भा हि दोषेण\\ धूमेनाग्निरिवावृताः~॥~४८~॥}\end{center} 
सहजं सह जन्मनैव उत्पन्नम्~। किं तत्~? कर्म कौन्तेय सदोषमपि त्रिगुणात्मकत्वात् न त्यजेत्~। सर्वारम्भाः आरभ्यन्त इति आरम्भाः, सर्वकर्माणि इत्येतत्~; प्रकरणात् ये केचित् आरम्भाः स्वधर्माः परधर्माश्च, ते सर्वे हि यस्मात् — त्रिगुणात्मकत्वम् अत्र हेतुः — त्रिगुणात्मकत्वात् दोषेण धूमेन सहजेन अग्निरिव, आवृताः~। सहजस्य कर्मणः स्वधर्माख्यस्य परित्यागेन परधर्मानुष्ठानेऽपि दोषात् नैव मुच्यते~; भयावहश्च परधर्मः~। न च शक्यते अशेषतः त्यक्तुम् अज्ञेन कर्म यतः, तस्मात् न त्यजेत् इत्यर्थः~॥~} 
किम् अशेषतः त्यक्तुम् अशक्यं कर्म इति न त्यजेत्~? किं वा सहजस्य कर्मणः त्यागे दोषो भवतीति~? किं च अतः~? यदि तावत् अशेषतः त्यक्तुम् अशक्यम् इति न त्याज्यं सहजं कर्म, एवं तर्हि अशेषतः त्यागे गुण एव स्यादिति सिद्धं भवति~। सत्यम् एवम्~; अशेषतः त्याग एव न उपपद्यते इति चेत्~, किं नित्यप्रचलितात्मकः पुरुषः, यथा साङ्ख्यानां गुणाः~? किं वा क्रियैव कारकम्~, यथा बौद्धानां स्कन्धाः क्षणप्रध्वंसिनः~? उभयथापि कर्मणः अशेषतः त्यागः न सम्भवति~। अथ तृतीयोऽपि पक्षः — यदा करोति तदा सक्रियं वस्तु~। यदा न करोति, तदा निष्क्रियं तदेव~। तत्र एवं सति शक्यं कर्म अशेषतः त्यक्तुम्~। अयं तु अस्मिन् तृतीये पक्षे विशेषः — न नित्यप्रचलितं वस्तु, नापि क्रियैव कारकम्~। किं तर्हि~? व्यवस्थिते द्रव्ये अविद्यमाना क्रिया उत्पद्यते, विद्यमाना च विनश्यति~। शुद्धं तत् द्रव्यं शक्तिमत् अवतिष्ठते~। इति एवम् आहुः काणादाः~। तदेव च कारकम् इति~। अस्मिन् पक्षे को दोषः इति~। अयमेव तु दोषः — यतस्तु अभागवतं मतम् इदम्~। कथं ज्ञायते~? यतः आह भगवान् ‘नासतो विद्यते भावः’\footnote{भ. गी. २~। १६} इत्यादि~। काणादानां हि असतः भावः, सतश्च अभावः, इति इदं मतम् अभागवतम्~। अभागवतमपि न्यायवच्चेत् को दोषः इति चेत्~, उच्यते — दोषवत्तु इदम्~, सर्वप्रमाणविरोधात्~। कथम्~? यदि तावत् द्व्यणुकादि द्रव्यं प्राक् उत्पत्तेः अत्यन्तमेव असत्~, उत्पन्नं च स्थितं कञ्चित् कालं पुनः अत्यन्तमेव असत्त्वम् आपद्यते, तथा च सति असदेव सत् जायते, सदेव असत्त्वम् आपद्यते, अभावः भावो भवति, भावश्च अभावो भवति~; तत्र अभावः जायमानः प्राक् उत्पत्तेः शशविषाणकल्पः समवाय्यसमवायिनिमित्ताख्यं कारणम् अपेक्ष्य जायते इति~। न च एवम् अभावः उत्पद्यते, कारणं च अपेक्षते इति शक्यं वक्तुम्~, असतां शशविषाणादीनाम् अदर्शनात्~। भावात्मकाश्चेत् घटादयः उत्पद्यमानाः, किञ्चित् अभिव्यक्तिमात्रे कारणम् अपेक्ष्य उत्पद्यन्ते इति शक्यं प्रतिपत्तुम्~। किञ्च, असतश्च सतश्च सद्भावे असद्भावे न क्वचित् प्रमाणप्रमेयव्यवहारेषु विश्वासः कस्यचित् स्यात्~, ‘सत् सदेव असत् असदेव’ इति निश्चयानुपपत्तेः~॥~} 
किञ्च, उत्पद्यते इति द्व्यणुकादेः द्रव्यस्य स्वकारणसत्तासम्बन्धम् आहुः~। प्राक् उत्पत्तेश्च असत्~, पश्चात् कारणव्यापारम् अपेक्ष्य स्वकारणैः परमाणुभिः सत्तया च समवायलक्षणेन सम्बन्धेन सम्बध्यते~। सम्बद्धं सत् कारणसमवेतं सत् भवति~। तत्र वक्तव्यं कथम् असतः स्वं कारणं भवेत् सम्बन्धो वा केनचित् स्यात्~? न हि वन्ध्यापुत्रस्य स्वं कारणं सम्बन्धो वा केनचित् प्रमाणतः कल्पयितुं शक्यते~॥~} 
ननु नैवं वैशेषिकैः अभावस्य सम्बन्धः कल्प्यते~। द्व्यणुकादीनां हि द्रव्याणां स्वकारणसमवायलक्षणः सम्बन्धः सतामेव उच्यते इति~। न~; सम्बन्धात् प्राक् सत्त्वानभ्युपगमात्~। न हि वैशेषिकैः कुलालदण्डचक्रादिव्यापारात् प्राक् घटादीनाम् अस्तित्वम् इष्यते~। न च मृद एव घटाद्याकारप्राप्तिम् इच्छन्ति~। ततश्च असत एव सम्बन्धः पारिशेष्यात् इष्टो भवति~॥~} 
ननु असतोऽपि समवायलक्षणः सम्बन्धः न विरुद्धः~। न~; वन्ध्यापुत्रादीनाम् अदर्शनात्~। घटादेरेव प्रागभावस्य स्वकारणसम्बन्धो भवति न वन्ध्यापुत्रादेः, अभावस्य तुल्यत्वेऽपि इति विशेषः अभावस्य वक्तव्यः~। एकस्य अभावः, द्वयोः अभावः, सर्वस्य अभावः, प्रागभावः, प्रध्वंसाभावः, इतरेतराभावः, अत्यन्ताभावः इति लक्षणतो न केनचित् विशेषो दर्शयितुं शक्यः~। असति च विशेषे घटस्य प्रागभावः एव कुलालादिभिः घटभावम् आपद्यते सम्बध्यते च भावेन कपालाख्येन, सम्बद्धश्च सर्वव्यवहारयोग्यश्च भवति, न तु घटस्यैव प्रध्वंसाभावः अभावत्वे सत्यपि, इति प्रध्वंसाद्यभावानां न क्वचित् व्यवहारयोग्यत्वम्~, प्रागभावस्यैव द्व्यणुकादिद्रव्याख्यस्य उत्पत्त्यादिव्यवहारार्हत्वम् इत्येतत् असमञ्जसम्~; अभावत्वाविशेषात् अत्यन्तप्रध्वंसाभावयोरिव~॥~} 
ननु नैव अस्माभिः प्रागभावस्य भावापत्तिः उच्यते~। भावस्यैव तर्हि भावापत्तिः~; यथा घटस्य घटापत्तिः, पटस्य वा पटापत्तिः~। एतदपि अभावस्य भावापत्तिवदेव प्रमाणविरुद्धम्~। साङ्ख्यस्यापि यः परिणामपक्षः सोऽपि अपूर्वधर्मोत्पत्तिविनाशाङ्गीकरणात् वैशेषिकपक्षात् न विशिष्यते~। अभिव्यक्तितिरोभावाङ्गीकरणेऽपि अभिव्यक्तितिरोभावयोः विद्यमानत्वाविद्यमानत्वनिरूपणे पूर्ववदेव प्रमाणविरोधः~। एतेन कारणस्यैव संस्थानम् उत्पत्त्यादि इत्येतदपि प्रत्युक्तम्~॥~} 
पारिशेष्यात् सत् एकमेव वस्तु अविद्यया उत्पत्तिविनाशादिधर्मैः अनेकधा नटवत् विकल्प्यते इति~। इदं भागवतं मतम् उक्तम् ‘नासतो विद्यते भावः’\footnote{भ. गी. २~। १६} इत्यस्मिन् श्लोके, सत्प्रत्ययस्य अव्यभिचारात्~, व्यभिचाराच्च इतरेषामिति~॥~} 
कथं तर्हि आत्मनः अविक्रियत्वे अशेषतः कर्मणः त्यागः न उपपद्यते इति~? यदि वस्तुभूताः गुणाः, यदि वा अविद्याकल्पिताः, तद्धर्मः कर्म, तदा आत्मनि अविद्याध्यारोपितमेव इति अविद्वान् ‘न हि कश्चित् क्षणमपि अशेषतः त्यक्तुं शक्नोति’\footnote{भ. गी. ३~। ५} इति उक्तम्~। विद्वांस्तु पुनः विद्यया अविद्यायां निवृत्तायां शक्नोत्येव अशेषतः कर्म परित्यक्तुम्~, अविद्याध्यारोपितस्य शेषानुपपत्तेः~। न हि तैमिरिकदृष्ट्या अध्यारोपितस्य द्विचन्द्रादेः तिमिरापगमेऽपि शेषः अवतिष्ठते~। एवं च सति इदं वचनम् उपपन्नम् ‘सर्वकर्माणि मनसा’\footnote{भ. गी. ५~। १३} इत्यादि, ‘स्वे स्वे कर्मण्यभिरतः संसिद्धिं लभते नरः’\footnote{भ. गी. १८~। ४५} ‘स्वकर्मणा तमभ्यर्च्य सिद्धिं विन्दति मानवः’\footnote{भ. गी. १८~। ४६} इति च~॥~४८~॥\par
 या कर्मजा सिद्धिः उक्ता ज्ञाननिष्ठायोग्यतालक्षणा, तस्याः फलभूता नैष्कर्म्यसिद्धिः ज्ञाननिष्ठालक्षणा च वक्तव्येति श्लोकः आरभ्यते —} 
\begin{center}{\bfseries असक्तबुद्धिः सर्वत्र\\ जितात्मा विगतस्पृहः~।\\नैष्कर्म्यसिद्धिं परमां\\ संन्यासेनाधिगच्छति~॥~४९~॥}\end{center} 
असक्तबुद्धिः असक्ता सङ्गरहिता बुद्धिः अन्तःकरणं यस्य सः असक्तबुद्धिः सर्वत्र पुत्रदारादिषु आसक्तिनिमित्तेषु, जितात्मा जितः वशीकृतः आत्मा अन्तःकरणं यस्य सः जितात्मा, विगतस्पृहः विगता स्पृहा तृष्णा देहजीवितभोगेषु यस्मात् सः विगतस्पृहः, यः एवंभूतः आत्मज्ञः सः नैष्कर्म्यसिद्धिं निर्गतानि कर्माणि यस्मात् निष्क्रियब्रह्मात्मसम्बोधात् सः निष्कर्मा तस्य भावः नैष्कर्म्यम्~, नैष्कर्म्यं च तत् सिद्धिश्च सा नैष्कर्म्यसिद्धिः, निष्कर्मत्वस्य वा निष्क्रियात्मरूपावस्थानलक्षणस्य सिद्धिः निष्पत्तिः, तां नैष्कर्म्यसिद्धिं परमां प्रकृष्टां कर्मजसिद्धिविलक्षणां सद्योमुक्त्यवस्थानरूपां संन्यासेन सम्यग्दर्शनेन तत्पूर्वकेण वा सर्वकर्मसंन्यासेन, अधिगच्छति प्राप्नोति~। तथा च उक्तम् — ‘सर्वकर्माणि मनसा संन्यस्य नैव कुर्वन्न कारयन्नास्ते’\footnote{भ. गी. ५~। १३} इति~॥~४९~॥\par
 पूर्वोक्तेन स्वकर्मानुष्ठानेन ईश्वराभ्यर्चनरूपेण जनितां प्रागुक्तलक्षणां सिद्धिं प्राप्तस्य उत्पन्नात्मविवेकज्ञानस्य केवलात्मज्ञाननिष्ठारूपा नैष्कर्म्यलक्षणा सिद्धिः येन क्रमेण भवति, तत् वक्तव्यमिति आह —} 
\begin{center}{\bfseries सिद्धिं प्राप्तो यथा ब्रह्म तथाप्नोति निबोध मे~।\\समासेनैव कौन्तेय निष्ठा ज्ञानस्य या परा~॥~५०~॥}\end{center} 
सिद्धिं प्राप्तः स्वकर्मणा ईश्वरं समभ्यर्च्य तत्प्रसादजां कायेन्द्रियाणां ज्ञाननिष्ठायोग्यतालक्षणां सिद्धिं प्राप्तः — सिद्धिं प्राप्तः इति तदनुवादः उत्तरार्थः~। किं तत् उत्तरम्~, यदर्थः अनुवादः इति, उच्यते — यथा येन प्रकारेण ज्ञाननिष्ठारूपेण ब्रह्म परमात्मानम् आप्नोति, तथा तं प्रकारं ज्ञाननिष्ठाप्राप्तिक्रमं मे मम वचनात् निबोध त्वं निश्चयेन अवधारय इत्येतत्~। किं विस्तरेण~? न इति आह — समासेनैव सङ्क्षेपेणैव हे कौन्तेय, यथा ब्रह्म प्राप्नोति तथा निबोधेति~। अनेन या प्रतिज्ञाता ब्रह्मप्राप्तिः, ताम् इदन्तया दर्शयितुम् आह — ‘निष्ठा ज्ञानस्य या परा’ इति~। निष्ठा पर्यवसानं परिसमाप्तिः इत्येतत्~। कस्य~? ब्रह्मज्ञानस्य या परा~। कीदृशी सा~? यादृशम् आत्मज्ञानम्~। कीदृक् तत्~? यादृशः आत्मा~। कीदृशः सः~? यादृशो भगवता उक्तः, उपनिषद्वाक्यैश्च न्यायतश्च~॥~} 
ननु विषयाकारं ज्ञानम्~। न ज्ञानविषयः, नापि आकारवान् आत्मा इष्यते क्वचित्~। ननु ‘आदित्यवर्णम्’\footnote{श्वे. उ. ३~। ८} ‘भारूपः’\footnote{छा. उ. ३~। १४~। २} ‘स्वयञ्ज्योतिः’\footnote{बृ. उ. ४~। ३~। ९} इति आकारवत्त्वम् आत्मनः श्रूयते~। न~; तमोरूपत्वप्रतिषेधार्थत्वात् तेषां वाक्यानाम् — द्रव्यगुणाद्याकारप्रतिषेधे आत्मनः तमोरूपत्वे प्राप्ते तत्प्रतिषेधार्थानि ‘आदित्यवर्णम्’\footnote{श्वे. उ. ३~। ८} इत्यादीनि वाक्यानि~। ‘अरूपम्’\footnote{क. उ. १~। ३~। १५} इति च विशेषतः रूपप्रतिषेधात्~। अविषयत्वाच्च — ‘न सन्दृशे तिष्ठति रूपमस्य न चक्षुषा पश्यति कश्चनैनम्’\footnote{श्वे. उ. ४~। २०} ‘अशब्दमस्पर्शम्’\footnote{क. उ. १~। ३~। १५} इत्यादेः~। तस्मात् आत्माकारं ज्ञानम् इति अनुपपन्नम्~॥~} 
कथं तर्हि आत्मनः ज्ञानम्~? सर्वं हि यद्विषयं यत् ज्ञानम्~, तत् तदाकारं भवति~। निराकारश्च आत्मा इत्युक्तम्~। ज्ञानात्मनोश्च उभयोः निराकारत्वे कथं तद्भावनानिष्ठा इति~? न~; अत्यन्तनिर्मलत्वातिस्वच्छत्वातिसूक्ष्मत्वोपपत्तेः आत्मनः~। बुद्धेश्च आत्मवत् नैर्मल्याद्युपपत्तेः आत्मचैतन्याकाराभासत्वोपपत्तिः~। बुद्ध्याभासं मनः, तदाभासानि इन्द्रियाणि, इन्द्रियाभासश्च देहः~। अतः लौकिकैः देहमात्रे एव आत्मदृष्टिः क्रियते~॥~} 
देहचैतन्यवादिनश्च लोकायतिकाः ‘चैतन्यविशिष्टः कायः पुरुषः’ इत्याहुः~। तथा अन्ये इन्द्रियचैतन्यवादिनः, अन्ये मनश्चैतन्यवादिनः, अन्ये बुद्धिचैतन्यवादिनः~। ततोऽपि आन्तरम् अव्यक्तम् अव्याकृताख्यम् अविद्यावस्थम् आत्मत्वेन प्रतिपन्नाः केचित्~। सर्वत्र बुद्ध्यादिदेहान्ते आत्मचैतन्याभासता आत्मभ्रान्तिकारणम् इत्यतश्च आत्मविषयं ज्ञानं न विधातव्यम्~। किं तर्हि~? नामरूपाद्यनात्माध्यारोपणनिवृत्तिरेव कार्या, आत्मचैतन्यविज्ञानं कार्यम्~, अविद्याध्यारोपितसर्वपदार्थाकारैः अविशिष्टतया दृश्यमानत्वात् इति~। अत एव हि विज्ञानवादिनो बौद्धाः विज्ञानव्यतिरेकेण वस्त्वेव नास्तीति प्रतिपन्नाः, प्रमाणान्तरनिरपेक्षतां च स्वसंविदितत्वाभ्युपगमेन~। तस्मात् अविद्याध्यारोपितनिराकरणमात्रं ब्रह्मणि कर्तव्यम्~, न तु ब्रह्मविज्ञाने यत्नः, अत्यन्तप्रसिद्धत्वात्~। अविद्याकल्पितनामरूपविशेषाकारापहृतबुद्धीनाम् अत्यन्तप्रसिद्धं सुविज्ञेयम् आसन्नतरम् आत्मभूतमपि, अप्रसिद्धं दुर्विज्ञेयम् अतिदूरम् अन्यदिव च प्रतिभाति अविवेकिनाम्~। बाह्याकारनिवृत्तबुद्धीनां तु लब्धगुर्वात्मप्रसादानां न अतः परं सुखं सुप्रसिद्धं सुविज्ञेयं स्वासन्नतरम् अस्ति~। तथा चोक्तम् — ‘प्रत्यक्षावगमं धर्म्यम्’\footnote{भ. गी. ९~। २} इत्यादि~॥~} 
केचित्तु पण्डितंमन्याः ‘निराकारत्वात् आत्मवस्तु न उपैति बुद्धिः~। अतः दुःसाध्या सम्यग्ज्ञाननिष्ठा’ इत्याहुः। सत्यम्~; एवं गुरुसम्प्रदायरहितानाम् अश्रुतवेदान्तानाम् अत्यन्तबहिर्विषयासक्तबुद्धीनां सम्यक्प्रमाणेषु अकृतश्रमाणाम्~। तद्विपरीतानां तु लौकिकग्राह्यग्राहकद्वैतवस्तुनि सद्बुद्धिः नितरां दुःसम्पादा, आत्मचैतन्यव्यतिरेकेण वस्त्वन्तरस्य अनुपलब्धेः, यथा च ‘एतत् एवमेव, न अन्यथा’ इति अवोचाम~; उक्तं च भगवता ‘यस्यां जाग्रति भूतानि सा निशा पश्यतो मुनेः’\footnote{भ. गी. २~। ६९} इति~। तस्मात् बाह्याकारभेदबुद्धिनिवृत्तिरेव आत्मस्वरूपावलम्बनकारणम्~। न हि आत्मा नाम कस्यचित् कदाचित् अप्रसिद्धः प्राप्यः हेयः उपादेयो वा~; अप्रसिद्धे हि तस्मिन् आत्मनि स्वार्थाः सर्वाः प्रवृत्तयः व्यर्थाः प्रसज्येरन्~। न च देहाद्यचेतनार्थत्वं शक्यं कल्पयितुम्~। न च सुखार्थं सुखम्~, दुःखार्थं दुःखम्~। आत्मावगत्यवसानार्थत्वाच्च सर्वव्यवहारस्य~। तस्मात् यथा स्वदेहस्य परिच्छेदाय न प्रमाणान्तरापेक्षा, ततोऽपि आत्मनः अन्तरतमत्वात् तदवगतिं प्रति न प्रमाणान्तरापेक्षा~; इति आत्मज्ञाननिष्ठा विवेकिनां सुप्रसिद्धा इति सिद्धम्~॥~} 
येषामपि निराकारं ज्ञानम् अप्रत्यक्षम्~, तेषामपि ज्ञानवशेनैव ज्ञेयावगतिरिति ज्ञानम् अत्यन्तप्रसिद्धं सुखादिवदेव इति अभ्युपगन्तव्यम्~। जिज्ञासानुपपत्तेश्च — अप्रसिद्धं चेत् ज्ञानम्~, ज्ञेयवत् जिज्ञास्येत~। यथा ज्ञेयं घटादिलक्षणं ज्ञानेन ज्ञाता व्याप्तुम् इच्छति, तथा ज्ञानमपि ज्ञानान्तरेण ज्ञातव्यम् आप्तुम् इच्छेत्~। न एतत् अस्ति~। अतः अत्यन्तप्रसिद्धं ज्ञानम्~, ज्ञातापि अत एव प्रसिद्धः इति~। तस्मात् ज्ञाने यत्नो न कर्तव्यः, किं तु अनात्मनि आत्मबुद्धिनिवृत्तावेव~। तस्मात् ज्ञाननिष्ठा सुसम्पाद्या~॥~५०~॥\par
 सा इयं ज्ञानस्य परा निष्ठा उच्यते, कथं कार्या इति —} 
\begin{center}{\bfseries बुद्ध्या विशुद्धया युक्तो\\ धृत्यात्मानं नियम्य च~।\\शब्दादीन्विषयांस्त्यक्त्वा\\ रागद्वेषौ व्युदस्य च~॥~५१~॥}\end{center} 
बुद्ध्या अध्यवसायलक्षणया विशुद्धया मायारहितया युक्तः सम्पन्नः, धृत्या धैर्येण आत्मानं कार्यकरणसङ्घातं नियम्य च नियमनं कृत्वा वशीकृत्य, शब्दादीन् शब्दः आदिः येषां तान् विषयान् त्यक्त्वा, सामर्थ्यात् शरीरस्थितिमात्रहेतुभूतान् केवलान् मुक्त्वा ततः अधिकान् सुखार्थान् त्यक्त्वा इत्यर्थः, शरीरस्थित्यर्थत्वेन प्राप्तेषु रागद्वेषौ व्युदस्य च परित्यज्य च~॥~५१~॥\par
 ततः —} 
\begin{center}{\bfseries विविक्तसेवी लघ्वाशी\\ यतवाक्कायमानसः~।\\ध्यानयोगपरो नित्यं\\ वैराग्यं समुपाश्रितः~॥~५२~॥}\end{center} 
विविक्तसेवी अरण्यनदीपुलिनगिरिगुहादीन् विविक्तान् देशान् सेवितुं शीलम् अस्य इति विविक्तसेवी, लघ्वाशी लघ्वशनशीलः — विविक्तसेवालघ्वशनयोः निद्रादिदोषनिवर्तकत्वेन चित्तप्रसादहेतुत्वात् ग्रहणम्~; यतवाक्कायमानसः वाक् च कायश्च मानसं च यतानि संयतानि यस्य ज्ञाननिष्ठस्य सः ज्ञाननिष्ठः यतिः यतवाक्कायमानसः स्यात्~। एवम् उपरतसर्वकरणः सन् ध्यानयोगपरः ध्यानम् आत्मस्वरूपचिन्तनम्~, योगः आत्मविषये एकाग्रीकरणम् तौ परत्वेन कर्तव्यौ यस्य सः ध्यानयोगपरः नित्यं नित्यग्रहणं मन्त्रजपाद्यन्यकर्तव्याभावप्रदर्शनार्थम्~, वैराग्यं विरागस्य भावः दृष्टादृष्टेषु विषयेषु वैतृष्ण्यं समुपाश्रितः सम्यक् उपाश्रितः नित्यमेव इत्यर्थः~॥~५२~॥\par
 किञ्च —} 
\begin{center}{\bfseries अहङ्कारं बलं दर्पं\\ कामं क्रोधं परिग्रहम्~।\\विमुच्य निर्ममः शान्तो\\ ब्रह्मभूयाय कल्पते~॥~५३~॥}\end{center} 
अहङ्कारम् अहङ्करणम् अहङ्कारः देहादिषु तम्~, बलं सामर्थ्यं कामरागसंयुक्तम् — न इतरत् शरीरादिसामर्थ्यं स्वाभाविकत्वेन तत्त्यागस्य अशक्यत्वात् — दर्पं दर्पो नाम हर्षानन्तरभावी धर्मातिक्रमहेतुः ‘हृष्टो दृप्यति दृप्तो धर्ममतिक्रामति’\footnote{आ. ध. सू. १~। १३~। ४} इति स्मरणात्~; तं च, कामम् इच्छां क्रोधं द्वेषं परिग्रहम् इन्द्रियमनोगतदोषपरित्यागेऽपि शरीरधारणप्रसङ्गेन धर्मानुष्ठाननिमित्तेन वा बाह्यः परिग्रहः प्राप्तः, तं च विमुच्य परित्यज्य, परमहंसपरिव्राजको भूत्वा, देहजीवनमात्रेऽपि निर्गतममभावः निर्ममः, अत एव शान्तः उपरतः, यः संहृतहर्षायासः यतिः ज्ञाननिष्ठः ब्रह्मभूयाय ब्रह्मभवनाय कल्पते समर्थो भवति~॥~५३~॥\par
 अनेन क्रमेण —} 
\begin{center}{\bfseries ब्रह्मभूतः प्रसन्नात्मा\\ न शोचति न काङ्क्षति~।\\समः सर्वेषु भूतेषु\\ मद्भक्तिं लभते पराम्~॥~५४~॥}\end{center} 
ब्रह्मभूतः ब्रह्मप्राप्तः प्रसन्नात्मा लब्धाध्यात्मप्रसादस्वभावः न शोचति, किञ्चित् अर्थवैकल्यम् आत्मनः वैगुण्यं वा उद्दिश्य न शोचति न सन्तप्यते~; न काङ्क्षति, न हि अप्राप्तविषयाकाङ्क्षा ब्रह्मविदः उपपद्यते~; अतः ब्रह्मभूतस्य अयं स्वभावः अनूद्यते — न शोचति न काङ्क्षति इति~। ‘न हृष्यति’ इति वा पाठान्तरम्~। समः सर्वेषु भूतेषु, आत्मौपम्येन सर्वभूतेषु सुखं दुःखं वा सममेव पश्यति इत्यर्थः~। न आत्मसमदर्शनम् इह, तस्य वक्ष्यमाणत्वात् ‘भक्त्या मामभिजानाति’\footnote{भ. गी. १८~। ५५} इति~। एवंभूतः ज्ञाननिष्ठः, मद्भक्तिं मयि परमेश्वरे भक्तिं भजनं पराम् उत्तमां ज्ञानलक्षणां चतुर्थीं लभते, ‘चतुर्विधा भजन्ते माम्’\footnote{भ. गी. ७~। १६} इति हि उक्तम्~॥~५४~॥\par
 ततः ज्ञानलक्षणया —} 
\begin{center}{\bfseries भक्त्या मामभिजानाति\\ यावान्यश्चास्मि तत्त्वतः~।\\ततो मां तत्त्वतो ज्ञात्वा\\ विशते तदनन्तरम्~॥~५५~॥}\end{center} 
भक्त्या माम् अभिजानाति यावान् अहम् उपाधिकृतविस्तरभेदः, यश्च अहम् अस्मि विध्वस्तसर्वोपाधिभेदः उत्तमः पुरुषः आकाशकल्पः, तं माम् अद्वैतं चैतन्यमात्रैकरसम् अजरम् अभयम् अनिधनं तत्त्वतः अभिजानाति~। ततः माम् एवं तत्त्वतः ज्ञात्वा विशते तदनन्तरं मामेव ज्ञानानन्तरम्~। नात्र ज्ञानप्रवेशक्रिये भिन्ने विवक्षिते ‘ज्ञात्वा विशते तदनन्तरम्’ इति~। किं तर्हि~? फलान्तराभावात् ज्ञानमात्रमेव, ‘क्षेत्रज्ञं चापि मां विद्धि’\footnote{भ. गी. १३~। २} इति उक्तत्वात्~॥~} 
ननु विरुद्धम् इदम् उक्तम् ‘ज्ञानस्य या परा निष्ठा तया माम् अभिजानाति’ इति~। कथं विरुद्धम् इति चेत्~, उच्यते — यदैव यस्मिन् विषये ज्ञानम् उत्पद्यते ज्ञातुः, तदैव तं विषयम् अभिजानाति ज्ञाता इति न ज्ञाननिष्ठां ज्ञानावृत्तिलक्षणाम् अपेक्षते इति~; अतश्च ज्ञानेन न अभिजानाति, ज्ञानावृत्त्या तु ज्ञाननिष्ठया अभिजानातीति~। नैष दोषः~; ज्ञानस्य स्वात्मोत्पत्तिपरिपाकहेतुयुक्तस्य प्रतिपक्षविहीनस्य यत् आत्मानुभवनिश्चयावसानत्वं तस्य निष्ठाशब्दाभिलापात्~। शास्त्राचार्योपदेशेन ज्ञानोत्पत्तिहेतुं सहकारिकारणं बुद्धिविशुद्धत्वादि अमानित्वादिगुणं च अपेक्ष्य जनितस्य क्षेत्रज्ञपरमात्मैकत्वज्ञानस्य कर्तृत्वादिकारकभेदबुद्धिनिबन्धनसर्वकर्मसंन्याससहितस्य स्वात्मानुभवनिश्चयरूपेण यत् अवस्थानम्~, सा परा ज्ञाननिष्ठा इति उच्यते~। सा इयं ज्ञाननिष्ठा आर्तादिभक्तित्रयापेक्षया परा चतुर्थी भक्तिरिति उक्ता~। तया परया भक्त्या भगवन्तं तत्त्वतः अभिजानाति, यदनन्तरमेव ईश्वरक्षेत्रज्ञभेदबुद्धिः अशेषतः निवर्तते~। अतः ज्ञाननिष्ठालक्षणया भक्त्या माम् अभिजानातीति वचनं न विरुध्यते~। } 
अत्र च सर्वं निवृत्तिविधायि शास्त्रं वेदान्तेतिहासपुराणस्मृतिलक्षणं न्यायप्रसिद्धम् अर्थवत् भवति — ‘विदित्वा . . . व्युत्थायाथ भिक्षाचर्यं चरन्ति’\footnote{बृ. उ. ३~। ५~। १} ‘तस्मान्न्यासमेषां तपसामतिरिक्तमाहुः’\footnote{तै. ना. ७९} ‘न्यास एवात्यरेचयत्’\footnote{तै. ना. ७८} इति~। ‘संन्यासः कर्मणां न्यासः’\footnote{~? } ‘वेदानिमं च लोकममुं च परित्यज्य’\footnote{आ. ध. २~। ९~। १३} ‘त्यज धर्ममधर्मं च’\footnote{मो. ध. ३२९~। ४०} इत्यादि~। इह च प्रदर्शितानि वाक्यानि~। न च तेषां वाक्यानाम् आनर्थक्यं युक्तम्~; न च अर्थवादत्वम्~, स्वप्रकरणस्थत्वात्~, प्रत्यगात्माविक्रियस्वरूपनिष्ठत्वाच्च मोक्षस्य~। न हि पूर्वसमुद्रं जिगमिषोः प्रातिलोम्येन प्रत्यक्समुद्रजिगमिषुणा समानमार्गत्वं सम्भवति~। प्रत्यगात्मविषयप्रत्ययसन्तानकरणाभिनिवेशश्च ज्ञाननिष्ठा~; सा च प्रत्यक्समुद्रगमनवत् कर्मणा सहभावित्वेन विरुध्यते~। पर्वतसर्षपयोरिव अन्तरवान् विरोधः प्रमाणविदां निश्चितः~। तस्मात् सर्वकर्मसंन्यासेनैव ज्ञाननिष्ठा कार्या इति सिद्धम्~॥~५५~॥\par
 स्वकर्मणा भगवतः अभ्यर्चनभक्तियोगस्य सिद्धिप्राप्तिः फलं ज्ञाननिष्ठायोग्यता, यन्निमित्ता ज्ञाननिष्ठा मोक्षफलावसाना~। सः भगवद्भक्तियोगः अधुना स्तूयते शास्त्रार्थोपासंहारप्रकरणे शास्त्रार्थनिश्चयदार्ढ्याय —} 
\begin{center}{\bfseries सर्वकर्माण्यपि सदा कुर्वाणो मद्व्यपाश्रयः~।\\मत्प्रसादादवाप्नोति शाश्वतं पदमव्ययम्~॥~५६~॥}\end{center} 
सर्वकर्माण्यपि प्रतिषिद्धान्यपि सदा कुर्वाणः अनुतिष्ठन् मद्व्यपाश्रयः अहं वासुदेवः ईश्वरः व्यपाश्रयो व्यपाश्रयणं यस्य सः मद्व्यपाश्रयः मय्यर्पितसर्वभावः इत्यर्थः~। सोऽपि मत्प्रसादात् मम ईश्वरस्य प्रसादात् अवाप्नोति शाश्वतं नित्यं वैष्णवं पदम् अव्ययम्~॥~५६~॥\par
 यस्मात् एवम् —} 
\begin{center}{\bfseries चेतसा सर्वकर्माणि मयि संन्यस्य मत्परः~।\\बुद्धियोगमपाश्रित्य मच्चित्तः सततं भव~॥~५७~॥}\end{center} 
चेतसा विवेकबुद्ध्या सर्वकर्माणि दृष्टादृष्टार्थानि मयि ईश्वरे संन्यस्य ‘यत् करोषि यदश्नासि’\footnote{भ. गी. ९~। २७} इति उक्तन्यायेन, मत्परः अहं वासुदेवः परो यस्य तव सः त्वं मत्परः सन् मय्यर्पितसर्वात्मभावः बुद्धियोगं समाहितबुद्धित्वं बुद्धियोगः तं बुद्धियोगम् अपाश्रित्य अपाश्रयः अनन्यशरणत्वं मच्चित्तः मय्येव चित्तं यस्य तव सः त्वं मच्चित्तः सततं सर्वदा भव~॥~५७~॥\par
 \begin{center}{\bfseries मच्चित्तः सर्वदुर्गाणि\\ मत्प्रसादात्तरिष्यसि~।\\अथ चेत्त्वमहङ्कारा—\\ न्न श्रोष्यसि विनङ्क्ष्यसि~॥~५८~॥}\end{center} 
मच्चित्तः सर्वदुर्गाणि सर्वाणि दुस्तराणि संसारहेतुजातानि मत्प्रसादात् तरिष्यसि अतिक्रमिष्यसि~। अथ चेत् यदि त्वं मदुक्तम् अहङ्कारात् ‘पण्डितः अहम्’ इति न श्रोष्यसि न ग्रहीष्यसि, ततः त्वं विनङ्क्ष्यसि विनाशं गमिष्यसि~॥~५८~॥\par
 इदं च त्वया न मन्तव्यम् ‘स्वतन्त्रः अहम्~, किमर्थं परोक्तं करिष्यामि~? ’ इति — } 
\begin{center}{\bfseries यद्यहङ्कारमाश्रित्य\\ न योत्स्य इति मन्यसे~।\\मिथ्यैष व्यवसायस्ते\\ प्रकृतिस्त्वां नियोक्ष्यति~॥~५९~॥}\end{center} 
यदि चेत् त्वम् अहङ्कारम् आश्रित्य न योत्स्ये इति न युद्धं करिष्यामि इति मन्यसे चिन्तयसि निश्चयं करोषि, मिथ्या एषः व्यवसायः निश्चयः ते तव~; यस्मात् प्रकृतिः क्षत्रियस्वभावः त्वां नियोक्ष्यति~॥~५९~॥\par
 यस्माच्च —} 
\begin{center}{\bfseries स्वभावजेन कौन्तेय\\ निबद्धः स्वेन कर्मणा~।\\कर्तुं नेच्छसि यन्मोहा—\\ त्करिष्यस्यवशोऽपि तत्~॥~६०~॥}\end{center} 
स्वभावजेन शौर्यादिना यथोक्तेन कौन्तेय निबद्धः निश्चयेन बद्धः स्वेन आत्मीयेन कर्मणा कर्तुं न इच्छसि यत् कर्म, मोहात् अविवेकतः करिष्यसि अवशोऽपि परवश एव तत् कर्म~॥~६०~॥\par
 यस्मात् —} 
\begin{center}{\bfseries ईश्वरः सर्वभूतानां हृद्देशेऽर्जुन तिष्ठति~।\\भ्रामयन्सर्वभूतानि यन्त्रारूढानि मायया~॥~६१~॥}\end{center} 
ईश्वरः ईशनशीलः नारायणः सर्वभूतानां सर्वप्राणिनां हृद्देशे हृदयदेशे अर्जुन शुक्लान्तरात्मस्वभावः विशुद्धान्तःकरणः — ‘अहश्च कृष्णमहरर्जुनं च’\footnote{ऋ. मं. ६~। १~। ९~। १} इति दर्शनात् — तिष्ठति स्थितिं लभते~। तेषु सः कथं तिष्ठतीति, आह — भ्रामयन् भ्रमणं कारयन् सर्वभूतानि यन्त्रारूढानि यन्त्राणि आरूढानि अधिष्ठितानि इव — इति इवशब्दः अत्र द्रष्टव्यः — यथा दारुकृतपुरुषादीनि यन्त्रारूढानि~। मायया च्छद्मना भ्रामयन् तिष्ठति इति सम्बन्धः~॥~६१~॥\par
 \begin{center}{\bfseries तमेव शरणं गच्छ\\ सर्वभावेन भारत~।\\तत्प्रसादात्परां शान्तिं\\ स्थानं प्राप्स्यसि शाश्वतम्~॥~६२~॥}\end{center} 
तमेव ईश्वरं शरणम् आश्रयं संसारार्तिहरणार्थं गच्छ आश्रय सर्वभावेन सर्वात्मना हे भारत~। ततः तत्प्रसादात् ईश्वरानुग्रहात् परां प्रकृष्टां शान्तिम् उपरतिं स्थानं च मम विष्णोः परमं पदं प्राप्स्यसि शाश्वतं नित्यम्~॥~६२~॥\par
 \begin{center}{\bfseries इति ते ज्ञानमाख्यातं गुह्याद्गुह्यतरं मया~।\\विमृश्यैतदशेषेण यथेच्छसि तथा कुरु~॥~६३~॥}\end{center} 
इति एतत् ते तुभ्यं ज्ञानम् आख्यातं कथितं गुह्यात् गोप्यात् गुह्यतरम् अतिशयेन गुह्यं रहस्यम् इत्यर्थः, मया सर्वज्ञेन ईश्वरेण~। विमृश्य विमर्शनम् आलोचनं कृत्वा एतत् यथोक्तं शास्त्रम् अशेषेण समस्तं यथोक्तं च अर्थजातं यथा इच्छसि तथा कुरु~॥~६३~॥\par
 भूयोऽपि मया उच्यमानं शृणु —} 
\begin{center}{\bfseries सर्वगुह्यतमं भूयः\\ शृणु मे परमं वचः~।\\इष्टोऽसि मे दृढमिति\\ ततो वक्ष्यामि ते हितम्~॥~६४~॥}\end{center} 
सर्वगुह्यतमं सर्वेभ्यः गुह्येभ्यः अत्यन्तगुह्यतमम् अत्यन्तरहस्यम्~, उक्तमपि असकृत् भूयः पुनः शृणु मे मम परमं प्रकृष्टं वचः वाक्यम्~। न भयात् नापि अर्थकारणाद्वा वक्ष्यामि~; किं तर्हि~? इष्टः प्रियः असि मे मम दृढम् अव्यभिचारेण इति कृत्वा ततः तेन कारणेन वक्ष्यामि कथयिष्यामि ते तव हितं परमं ज्ञानप्राप्तिसाधनम्~, तद्धि सर्वहितानां हिततमम्~॥~६४~॥\par
 किं तत् इति, आह —} 
\begin{center}{\bfseries मन्मना भव मद्भक्तो\\ मद्याजी मां नमस्कुरु~।\\मामेवैष्यसि सत्यं ते\\ प्रतिजाने प्रियोऽसि मे~॥~६५~॥}\end{center} 
मन्मनाः भव मच्चित्तः भव~। मद्भक्तः भव मद्भजनो भव~। मद्याजी मद्यजनशीलो भव~। मां नमस्कुरु नमस्कारम् अपि ममैव कुरु~। तत्र एवं वर्तमानः वासुदेवे एव समर्पितसाध्यसाधनप्रयोजनः मामेव एष्यसि आगमिष्यसि~। सत्यं ते तव प्रतिजाने, सत्यां प्रतिज्ञां करोमि एतस्मिन् वस्तुनि इत्यर्थः~; यतः प्रियः असि मे~। एवं भगवतः सत्यप्रतिज्ञत्वं बुद्ध्वा भगवद्भक्तेः अवश्यंभावि मोक्षफलम् अवधार्य भगवच्छरणैकपरायणः भवेत् इति वाक्यार्थः~॥~६५~॥\par
 कर्मयोगनिष्ठायाः परमरहस्यम् ईश्वरशरणताम् उपसंहृत्य, अथ इदानीं कर्मयोगनिष्ठाफलं सम्यग्दर्शनं सर्ववेदान्तसारविहितं वक्तव्यमिति आह —} 
\begin{center}{\bfseries सर्वधर्मान्परित्यज्य\\ मामेकं शरणं व्रज~।\\अहं त्वा सर्वपापेभ्यो\\ मोक्षयिष्यामि मा शुचः~॥~६६~॥}\end{center} 
सर्वधर्मान् सर्वे च ते धर्माश्च सर्वधर्माः तान् — धर्मशब्देन अत्र अधर्मोऽपि गृह्यते, नैष्कर्म्यस्य विवक्षितत्वात्~, ‘नाविरतो दुश्चरितात्’\footnote{क. उ. १~। २~। २४} ‘त्यज धर्ममधर्मं च’\footnote{मो. ध. ३२९~। ४०} इत्यादिश्रुतिस्मृतिभ्यः — सर्वधर्मान् परित्यज्य संन्यस्य सर्वकर्माणि इत्येतत्~। माम् एकं सर्वात्मानं समं सर्वभूतस्थितम् ईश्वरम् अच्युतं गर्भजन्मजरामरणवर्जितम् ‘अहमेव’ इत्येवं शरणं व्रज, न मत्तः अन्यत् अस्ति इति अवधारय इत्यर्थः~। अहं त्वा त्वाम् एवं निश्चितबुद्धिं सर्वपापेभ्यः सर्वधर्माधर्मबन्धनरूपेभ्यः मोक्षयिष्यामि स्वात्मभावप्रकाशीकरणेन~। उक्तं च ‘नाशयाम्यात्मभावस्थो ज्ञानदीपेन भास्वता’\footnote{भ. गी. १०~। ११} इति~। अतः मा शुचः शोकं मा कार्षीः इत्यर्थः~॥~} 
अस्मिन्गीताशास्त्रे परमनिःश्रेयससाधनं निश्चितं किं ज्ञानम्~, कर्म वा, आहोस्वित् उभयम्~? इति~। कुतः संशयः~? ‘यज्ज्ञात्वामृतमश्नुते’\footnote{भ. गी. १३~। १२} ‘ततो मां तत्त्वतो ज्ञात्वा विशते तदनन्तरम्’\footnote{भ. गी. १८~। ५५} इत्यादीनि वाक्यानि केवलाज्ज्ञानात् निःश्रेयसप्राप्तिं दर्शयन्ति~। ‘कर्मण्येवाधिकारस्ते’\footnote{भ. गी. २~। ४७} ‘कुरु कर्मैव’\footnote{भ. गी. ४~। १५} इत्येवमादीनि कर्मणामवश्यकर्तव्यतां दर्शयन्ति~। एवं ज्ञानकर्मणोः कर्तव्यत्वोपदेशात् समुच्चितयोरपि निःश्रेयसहेतुत्वं स्यात् इति भवेत् संशयः कस्यचित्~। किं पुनरत्र मीमांसाफलम्~? ननु एतदेव — एषामन्यतमस्य परमनिःश्रेयससाधनत्वावधारणम्~; अतः विस्तीर्णतरं मीमांस्यम् एतत्~॥~} 
आत्मज्ञानस्य तु केवलस्य निःश्रेयसहेतुत्वम्~, भेदप्रत्ययनिवर्तकत्वेन कैवल्यफलावसायित्वात्~। क्रियाकारकफलभेदबुद्धिः अविद्यया आत्मनि नित्यप्रवृत्ता — ‘मम कर्म, अहं कर्तामुष्मै फलायेदं कर्म करिष्यामि’ इति इयम् अविद्या अनादिकालप्रवृत्ता~। अस्या अविद्यायाः निवर्तकम् ‘अयमहमस्मि केवलोऽकर्ता अक्रियोऽफलः~; न मत्तोऽन्योऽस्ति कश्चित्’ इत्येवंरूपम् आत्मविषयं ज्ञानम् उत्पद्यमानम्~, कर्मप्रवृत्तिहेतुभूतायाः भेदबुद्धेः निवर्तकत्वात्~। तु - शब्दः पक्षव्यावृत्त्यर्थः — न केवलेभ्यः कर्मभ्यः, न च ज्ञानकर्मभ्यां समुच्चिताभ्यां निःश्रेयसप्राप्तिः इति पक्षद्वयं निवर्तयति~। अकार्यत्वाच्च निःश्रेयसस्य कर्मसाधनत्वानुपपत्तिः~। न हि नित्यं वस्तु कर्मणा ज्ञानेन वा क्रियते~। केवलं ज्ञानमपि अनर्थकं तर्हि~? न, अविद्यानिवर्तकत्वे सति दृष्टकैवल्यफलावसानत्वात्~। अविद्यातमोनिवर्तकस्य ज्ञानस्य दृष्टं कैवल्यफलावसानत्वम्~, रज्ज्वादिविषये सर्पाद्यज्ञानतमोनिवर्तकप्रदीपप्रकाशफलवत्~। विनिवृत्तसर्पादिविकल्परज्जुकैवल्यावसानं हि प्रकाशफलम्~; तथा ज्ञानम्~। दृष्टार्थानां च च्छिदिक्रियाग्निमन्थनादीनां व्यापृतकर्त्रादिकारकाणां द्वैधीभावाग्निदर्शनादिफलात् अन्यफले कर्मान्तरे वा व्यापारानुपपत्तिः यथा, तथा दृष्टार्थायां ज्ञाननिष्ठाक्रियायां व्यापृतस्य ज्ञात्रादिकारकस्य आत्मकैवल्यफलात् कर्मान्तरे प्रवृत्तिः अनुपपन्ना इति न ज्ञाननिष्ठा कर्मसहिता उपपद्यते~। भुज्यग्निहोत्रादिक्रियावत्स्यात् इति चेत्~, न~; कैवल्यफले ज्ञाने क्रियाफलार्थित्वानुपपत्तेः~। कैवल्यफले हि ज्ञाने प्राप्ते, सर्वतःसम्प्लुतोदकफले कूपतटाकादिक्रियाफलार्थित्वाभाववत्~, फलान्तरे तत्साधनभूतायां वा क्रियायाम् अर्थित्वानुपपत्तिः~। न हि राज्यप्राप्तिफले कर्मणि व्यापृतस्य क्षेत्रमात्रप्राप्तिफले व्यापारः उपपद्यते, तद्विषयं वा अर्थित्वम्~। तस्मात् न कर्मणोऽस्ति निःश्रेयससाधनत्वम्~। न च ज्ञानकर्मणोः समुच्चितयोः~। नापि ज्ञानस्य कैवल्यफलस्य कर्मसाहाय्यापेक्षा, अविद्यानिवर्तकत्वेन विरोधात्~। न हि तमः तमसः निवर्तकम्~। अतः केवलमेव ज्ञानं निःश्रेयससाधनम् इति~। न~; नित्याकरणे प्रत्यवायप्राप्तेः, कैवल्यस्य च नित्यत्वात्~। यत् तावत् केवलाज्ज्ञानात् कैवल्यप्राप्तिः इत्येतत्~, तत् असत्~; यतः नित्यानां कर्मणां श्रुत्युक्तानाम् अकरणे प्रत्यवायः नरकादिप्राप्तिलक्षणः स्यात्~। ननु एवं तर्हि कर्मभ्यो मोक्षो नास्ति इति अनिर्मोक्ष एव~। नैष दोषः~; नित्यत्वात् मोक्षस्य~। नित्यानां कर्मणाम् अनुष्ठानात् प्रत्यवायस्य अप्राप्तिः, प्रतिषिद्धस्य च अकरणात् अनिष्टशरीरानुपपत्तिः, काम्यानां च वर्जनात् इष्टाशरीरानुपपत्तिः, वर्तमानशरीरारम्भकस्य च कर्मणः फलोपभोगक्षये पतिते अस्मिन् शरीरे देहान्तरोत्पत्तौ च कारणाभावात् आत्मनः रागादीनां च अकरणे स्वरूपावस्थानमेव कैवल्यमिति अयत्नसिद्धं कैवल्यम् इति~। अतिक्रान्तानेकजन्मान्तरकृतस्य स्वर्गनरकादिप्राप्तिफलस्य अनारब्धकार्यस्य उपभोगानुपपत्तेः क्षयाभावः इति चेत्~, न~; नित्यकर्मानुष्ठानायासदुःखोपभोगस्य तत्फलोपभोगत्वोपपत्तेः~। प्रायश्चित्तवद्वा पूर्वोपात्तदुरितक्षयार्थं नित्यं कर्म~। आरब्धानां च कर्मणाम् उपभोगेनैव क्षीणत्वात् अपूर्वाणां च कर्मणाम् अनारम्भे अयत्नसिद्धं कैवल्यमिति~। न~; ‘तमेव विदित्वातिमृत्युमेति नान्यः पन्था विद्यतेऽयनाय’\footnote{श्वे. उ. ३~। ८} इति विद्याया अन्यः पन्थाः मोक्षाय न विद्यते इति श्रुतेः, चर्मवदाकाशवेष्टनासम्भववत् अविदुषः मोक्षासम्भवश्रुतेः, ‘ज्ञानात्कैवल्यमाप्नोति’\footnote{~? } इति च पुराणस्मृतेः~; अनारब्धफलानां पुण्यानां कर्मणां क्षयानुपपत्तेश्च~। यथा पूर्वोपात्तानां दुरितानाम् अनारब्धफलानां सम्भवः, तथा पुण्यानाम् अनारब्धफलानां स्यात्सम्भवः~। तेषां च देहान्तरम् अकृत्वा क्षयानुपपत्तौ मोक्षानुपपत्तिः~। धर्माधर्महेतूनां च रागद्वेषमोहानाम् अन्यत्र आत्मज्ञानात् उच्छेदानुपपत्तेः धर्माधर्मोच्छेदानुपपत्तिः~। नित्यानां च कर्मणां पुण्यफलत्वश्रुतेः, ‘वर्णा आश्रमाश्च स्वकर्मनिष्ठाः’\footnote{गौ. ध. सू. २~। २~। २९} इत्यादिस्मृतेश्च कर्मक्षयानुपपत्तिः~॥~} 
ये तु आहुः — नित्यानि कर्माणि दुःखरूपत्वात् पूर्वकृतदुरितकर्मणां फलमेव, न तु तेषां स्वरूपव्यतिरेकेण अन्यत् फलम् अस्ति, अश्रुतत्वात्~, जीवनादिनिमित्ते च विधानात् इति~। न अप्रवृत्तानां कर्मणां फलदानासम्भवात्~; दुःखफलविशेषानुपपत्तिश्च स्यात्~। यदुक्तं पूर्वजन्मकृतदुरितानां कर्मणां फलं नित्यकर्मानुष्ठानायासदुःखं भुज्यत इति, तदसत्~। न हि मरणकाले फलदानाय अनङ्कुरीभूतस्य कर्मणः फलम् अन्यकर्मारब्धे जन्मनि उपभुज्यते इति उपपत्तिः~। अन्यथा स्वर्गफलोपभोगाय अग्निहोत्रादिकर्मारब्धे जन्मनि नरकफलोपभोगानुपपत्तिः न स्यात्~। तस्य दुरितस्य दुःखविशेषफलत्वानुपपत्तेश्च — अनेकेषु हि दुरितेषु सम्भवत्सु भिन्नदुःखसाधनफलेषु नित्यकर्मानुष्ठानायासदुःखमात्रफलेषु कल्प्यमानेषु द्वन्द्वरोगादिबाधनं निर्निमित्तं न हि शक्यते कल्पयितुम्~, नित्यकर्मानुष्ठानायासदुःखमेव पूर्वोपात्तदुरितफलं न शिरसा पाषाणवहनादिदुःखमिति~। अप्रकृतं च इदम् उच्यते — नित्यकर्मानुष्ठानायासदुःखं पूर्वकृतदुरितकर्मफलम् इति~। कथम्~? अप्रसूतफलस्य हि पूर्वकृतदुरितस्य क्षयः न उपपद्यत इति प्रकृतम्~। तत्र प्रसूतफलस्य कर्मणः फलं नित्यकर्मानुष्ठानायासदुःखम् आह भवान्~, न अप्रसूतफलस्येति~। अथ सर्वमेव पूर्वकृतं दुरितं प्रसूतफलमेव इति मन्यते भवान्~, ततः नित्यकर्मानुष्ठानायासदुःखमेव फलम् इति विशेषणम् अयुक्तम्~। नित्यकर्मविध्यानर्थक्यप्रसङ्गश्च, उपभोगेनैव प्रसूतफलस्य दुरितकर्मणः क्षयोपपत्तेः~। किञ्च, श्रुतस्य नित्यस्य कर्मणः दुःखं चेत् फलम्~, नित्यकर्मानुष्ठानायासादेव तत् दृश्यते व्यायामादिवत्~; तत् अन्यस्य इति कल्पनानुपपत्तिः~। जीवनादिनिमित्ते च विधानात्~, नित्यानां कर्मणां प्रायश्चित्तवत् पूर्वकृतदुरितफलत्वानुपपत्तिः~। यस्मिन् पापकर्मणि निमित्ते यत् विहितं प्रायश्चित्तम् न तु तस्य पापस्य तत् फलम्~। अथ तस्यैव पापस्य निमित्तस्य प्रायश्चित्तदुःखं फलम्~, जीवनादिनिमित्तेऽपि नित्यकर्मानुष्ठानायासदुःखं जीवनादिनिमित्तस्यैव फलं प्रसज्येत, नित्यप्रायश्चित्तयोः नैमित्तिकत्वाविशेषात्~। किञ्च अन्यत् — नित्यस्य काम्यस्य च अग्निहोत्रादेः अनुष्ठानायासदुःखस्य तुल्यत्वात् नित्यानुष्ठानायासदुःखमेव पूर्वकृतदुरितस्य फलम्~, न तु काम्यानुष्ठानायासदुःखम् इति विशेषो नास्तीति तदपि पूर्वकृतदुरितफलं प्रसज्येत~। तथा च सति नित्यानां फलाश्रवणात् तद्विधानान्यथानुपपत्तेश्च नित्यानुष्ठानायासदुःखं पूर्वकृतदुरितफलम् इति अर्थापत्तिकल्पना च अनुपपन्ना, एवं विधानान्यथानुपपत्तेः अनुष्ठानायासदुःखव्यतिरिक्तफलत्वानुमानाच्च नित्यानाम्~। विरोधाच्च~; विरुद्धं च इदम् उच्यते — नित्यकर्मणा अनुष्टीयमानेन अन्यस्य कर्मणः फलं भुज्यते इति अभ्युपगम्यमाने स एव उपभोगः नित्यस्य कर्मणः फलम् इति, नित्यस्य कर्मणः फलाभाव इति च विरुद्धम् उच्यते~। किञ्च, काम्याग्निहोत्रादौ अनुष्ठीयमाने नित्यमपि अग्निहोत्रादि तन्त्रेणैव अनुष्ठितं भवतीति तदायासदुःखेनैव काम्याग्निहोत्रादिफलम् उपक्षीणं स्यात्~, तत्तन्त्रत्वात्~। अथ काम्याग्निहोत्रादिफलम् अन्यदेव स्वर्गादि, तदनुष्ठानायासदुःखमपि भिन्नं प्रसज्येत~। न च तदस्ति, दृष्टविरोधात्~; न हि काम्यानुष्ठानायासदुःखात् केवलनित्यानुष्ठानायासदुःखं भिन्नं दृश्यते~। किञ्च अन्यत् — अविहितमप्रतिषिद्धं च कर्म तत्कालफलम्~, न तु शास्त्रचोदितं प्रतिषिद्धं वा तत्कालफलं भवेत्~। तदा स्वर्गादिष्वपि अदृष्टफलाशासनेन उद्यमो न स्यात् — अग्निहोत्रादीनामेव कर्मस्वरूपाविशेषे अनुष्ठानायासदुःखमात्रेण उपक्षयः नित्यानाम्~; स्वर्गादिमहाफलत्वं काम्यानाम्~, अङ्गेतिकर्तव्यताद्याधिक्ये तु असति, फलकामित्वमात्रेणेति~। तस्माच्च न नित्यानां कर्मणाम् अदृष्टफलाभावः कदाचिदपि उपपद्यते~। अतश्च अविद्यापूर्वकस्य कर्मणः विद्यैव शुभस्य अशुभस्य वा क्षयकारणम् अशेषतः, न नित्यकर्मानुष्ठानम्~। अविद्याकामबीजं हि सर्वमेव कर्म~। तथा च उपपादितमविद्वद्विषयं कर्म, विद्वद्विषया च सर्वकर्मसंन्यासपूर्विका ज्ञाननिष्ठा — ‘उभौ तौ न विजानीतः’\footnote{भ. गी. २~। १९} ‘वेदाविनाशिनं नित्यम्’\footnote{भ. गी. २~। २१} ‘ज्ञानयोगेन साङ्ख्यानां कर्मयोगेन योगिनाम्’\footnote{भ. गी. ३~। ३} ‘अज्ञानां कर्मसङ्गिनाम्’\footnote{भ. गी. ३~। २६} ‘तत्त्ववित्तु महाबाहो गुणा गुणेषु वर्तन्ते इति मत्वा न सज्जते’\footnote{भ. गी. ३~। २८} ‘सर्वकर्माणि मनसा संन्यस्यास्ते’\footnote{भ. गी. ५~। १३} ‘नैव किञ्चित् करोमीति युक्तो मन्येत तत्त्ववित्’\footnote{भ. गी. ५~। ८}, अर्थात् अज्ञः करोमि इति~; आरुरुक्षोः कर्म कारणम्~, आरूढस्य योगस्थस्य शम एव कारणम्~; उदाराः त्रयोऽपि अज्ञाः, ‘ज्ञानी त्वात्मैव मे मतम्’\footnote{भ. गी. ७~। १८} ‘अज्ञाः कर्मिणः गतागतं कामकामाः लभन्ते’~; अनन्याश्चिन्तयन्तो मां नित्ययुक्ताः यथोक्तम् आत्मानम् आकाशकल्पम् उपासते~; ‘ददामि बुद्धियोगं तं येन मामुपयान्ति ते’, अर्थात् न कर्मिणः अज्ञाः उपयान्ति~। भगवत्कर्मकारिणः ये युक्ततमा अपि कर्मिणः अज्ञाः, ते उत्तरोत्तरहीनफलत्यागावसानसाधनाः~; अनिर्देश्याक्षरोपासकास्तु  ‘अद्वेष्टा सर्वभूतानाम्’\footnote{भ. गी. १२~। १३} इति आध्यायपरिसमाप्ति उक्तसाधनाः क्षेत्राध्यायाद्यध्यायत्रयोक्तज्ञानसाधनाश्च~। अधिष्ठानादिपञ्चकहेतुकसर्वकर्मसंन्यासिनां आत्मैकत्वाकर्तृत्वज्ञानवतां परस्यां ज्ञाननिष्ठायां वर्तमानानां भगवत्तत्त्वविदाम् अनिष्टादिकर्मफलत्रयं परमहंसपरिव्राजकानामेव लब्धभगवत्स्वरूपात्मैकत्वशरणानां न भवति~; भवत्येव अन्येषामज्ञानां कर्मिणामसंन्यासिनाम् इत्येषः गीताशास्त्रोक्तकर्तव्यार्थस्य विभागः~॥~} 
अविद्यापूर्वकत्वं सर्वस्य कर्मणः असिद्धमिति चेत्~, न~; ब्रह्महत्यादिवत्~। यद्यपि शास्त्रावगतं नित्यं कर्म, तथापि अविद्यावत एव भवति~। यथा प्रतिषेधशास्त्रावगतमपि ब्रह्महत्यादिलक्षणं कर्म अनर्थकारणम् अविद्याकामादिदोषवतः भवति, अन्यथा प्रवृत्त्यनुपपत्तेः, तथा नित्यनैमित्तिककाम्यान्यपीति~। देहव्यतिरिक्तात्मनि अज्ञाते प्रवृत्तिः नित्यादिकर्मसु अनुपपन्ना इति चेत्~, न~; चलनात्मकस्य कर्मणः अनात्मकर्तृकस्य ‘अहं करोमि’ इति प्रवृत्तिदर्शनात्~। देहादिसङ्घाते अहंप्रत्ययः गौणः, न मिथ्या इति चेत्~, न~; तत्कार्येष्वपि गौणत्वोपपत्तेः~। आत्मीये देहादिसङ्घाते अहंप्रत्ययः गौणः~; यथा आत्मीये पुत्रे ‘आत्मा वै पुत्रनामासि’\footnote{तै. आ. एका. २~। ११} इति, लोके च ‘मम प्राण एव अयं गौः’ इति, तद्वत्~। नैवायं मिथ्याप्रत्ययः~। मिथ्याप्रत्ययस्तु स्थाणुपुरुषयोः अगृह्यमाणविशेषयोः~। न गौणप्रत्ययस्य मुख्यकार्यार्थता, अधिकरणस्तुत्यर्थत्वात् लुप्तोपमाशब्देन~। यथा ‘सिंहो देवदत्तः’ ‘अग्निर्माणवकः’ इति सिंह इव अग्निरिव क्रौर्यपैङ्गल्यादिसामान्यवत्त्वात् देवदत्तमाणवकाधिकरणस्तुत्यर्थमेव, न तु सिंहकार्यम् अग्निकार्यं वा गौणशब्दप्रत्ययनिमित्तं किञ्चित्साध्यते~; मिथ्याप्रत्ययकार्यं तु अनर्थमनुभवति इति~। गौणप्रत्ययविषयं जानाति ‘नैष सिंहः देवदत्तः’, तथा ‘नायमग्निर्माणवकः’ इति~। तथा गौणेन देहादिसङ्घातेन आत्मना कृतं कर्म न मुख्येन अहंप्रत्ययविषयेण आत्मना कृतं स्यात्~। न हि गौणसिंहाग्निभ्यां कृतं कर्म मुख्यसिंहाग्निभ्यां कृतं स्यात्~। न च क्रौर्येण पैङ्गल्येन वा मुख्यसिंहाग्न्योः कार्यं किञ्चित् क्रियते, स्तुत्यर्थत्वेन उपक्षीणत्वात्~। स्तूयमानौ च जानीतः ‘न अहं सिंहः’ ‘न अहम् अग्निः’ इति~; न हि ‘सिंहस्य कर्म मम अग्नेश्च’ इति~। तथा ‘न सङ्घातस्य कर्म मम मुख्यस्य आत्मनः’ इति प्रत्ययः युक्ततरः स्यात्~; न पुनः ‘अहं कर्ता मम कर्म’ इति~। यच्च आहुः ‘आत्मीयैः स्मृतीच्छाप्रयत्नैः कर्महेतुभिरात्मा कर्म करोति’ इति, न~; तेषां मिथ्याप्रत्ययपूर्वकत्वात्~। मिथ्याप्रत्ययनिमित्तेष्टानिष्टानुभूतक्रियाफलजनितसंस्कारपूर्वकाः हि स्मृतीच्छाप्रयत्नादयः~। यथा अस्मिन् जन्मनि देहादिसङ्घाताभिमानरागद्वेषादिकृतौ धर्माधर्मौ तत्फलानुभवश्च, तथा अतीते अतीततरेऽपि जन्मनि इति अनादिरविद्याकृतः संसारः अतीतोऽनागतश्च अनुमेयः~। ततश्च सर्वकर्मसंन्याससहितज्ञाननिष्ठया आत्यन्तिकः संसारोपरम इति सिद्धम्~। अविद्यात्मकत्वाच्च देहाभिमानस्य, तन्निवृत्तौ देहानुपपत्तेः संसारानुपपत्तिः~। देहादिसङ्घाते आत्माभिमानः अविद्यात्मकः~। न हि लोके ‘गवादिभ्योऽन्योऽहम्~, मत्तश्चान्ये गवादयः’ इति जानन् तान् ‘अहम्’ इति मन्यते कश्चित्~। अजानंस्तु स्थाणौ पुरुषविज्ञानवत् अविवेकतः देहादिसङ्घाते कुर्यात् ‘अहम्’ इति प्रत्ययम्~, न विवेकतः जानन्~। यस्तु ‘आत्मा वै पुत्र नामासि’\footnote{तै. आ. एका. २~। ११} इति पुत्रे अहंप्रत्ययः, स तु जन्यजनकसम्बन्धनिमित्तः गौणः~। गौणेन च आत्मना भोजनादिवत् परमार्थकार्यं न शक्यते कर्तुम्~, गौणसिंहाग्निभ्यां मुख्यसिंहाग्निकार्यवत्~॥~} 
अदृष्टविषयचोदनाप्रामाण्यात् आत्मकर्तव्यं गौणैः देहेन्द्रियात्मभिः क्रियत एव इति चेत्~, न~; अविद्याकृतात्मत्वात्तेषाम्~। न च गौणाः आत्मानः देहेन्द्रियादयः~; किं तर्हि~? मिथ्याप्रत्ययेनैव अनात्मानः सन्तः आत्मत्वमापाद्यन्ते, तद्भावे भावात्~, तदभावे च अभावात्~। अविवेकिनां हि अज्ञानकाले बालानां दृश्यते ‘दीर्घोऽहम्’ ‘गौरोऽहम्’ इति देहादिसङ्घाते अहंप्रत्ययः~। न तु विवेकिनाम् ‘अन्योऽहं देहादिसङ्घातात्’ इति जानतां तत्काले देहादिसङ्घाते अहंप्रत्ययः भवति~। तस्मात् मिथ्याप्रत्ययाभावे अभावात् तत्कृत एव, न गौणः~। पृथग्गृह्यमाणविशेषसामान्ययोर्हि सिंहदेवदत्तयोः अग्निमाणवकयोर्वा गौणः प्रत्ययः शब्दप्रयोगो वा स्यात्~, न अगृह्यमाणविशेषसामान्ययोः~। यत्तु उक्तम् ‘श्रुतिप्रामाण्यात्’ इति, तत् न~; तत्प्रामाण्यस्य अदृष्टविषयत्वात्~। प्रत्यक्षादिप्रमाणानुपलब्धे हि विषये अग्निहोत्रादिसाध्यसाधनसम्बन्धे श्रुतेः प्रामाण्यम्~, न प्रत्यक्षादिविषये, अदृष्टदर्शनार्थविषयत्वात् प्रामाण्यस्य~। तस्मात् न दृष्टमिथ्याज्ञाननिमित्तस्य अहंप्रत्ययस्य देहादिसङ्घाते गौणत्वं कल्पयितुं शक्यम्~। न हि श्रुतिशतमपि ‘शीतोऽग्निरप्रकाशो वा’ इति ब्रुवत् प्रामाण्यमुपैति~। यदि ब्रूयात् ‘शीतोऽग्निरप्रकाशो वा’ इति, तथापि अर्थान्तरं श्रुतेः विवक्षितं कल्प्यम्~, प्रामाण्यान्यथानुपपत्तेः, न तु प्रमाणान्तरविरुद्धं स्ववचनविरुद्धं वा~। कर्मणः मिथ्याप्रत्ययवत्कर्तृकत्वात् कर्तुरभावे श्रुतेरप्रामाण्यमिति चेत्~, न~; ब्रह्मविद्यायामर्थवत्त्वोपपत्तेः~॥~} 
कर्मविधिश्रुतिवत् ब्रह्मविद्याविधिश्रुतेरपि अप्रामाण्यप्रसङ्ग इति चेत्~, न~; बाधकप्रत्ययानुपपत्तेः~। यथा ब्रह्मविद्याविधिश्रुत्या आत्मनि अवगते देहादिसङ्घाते अहंप्रत्ययः बाध्यते, तथा आत्मन्येव आत्मावगतिः न कदाचित् केनचित् कथञ्चिदपि बाधितुं शक्या, फलाव्यतिरेकादवगतेः, यथा अग्निः उष्णः प्रकाशश्च इति~। न च एवं कर्मविधिश्रुतेरप्रामाण्यम्~, पूर्वपूर्वप्रवृत्तिनिरोधेन उत्तरोत्तरापूर्वप्रवृत्तिजननस्य प्रत्यगात्माभिमुख्येन प्रवृत्त्युत्पादनार्थत्वात्~। मिथ्यात्वेऽपि उपायस्य उपेयसत्यतया सत्यत्वमेव स्यात्~, यथा अर्थवादानां विधिशेषाणाम्~; लोकेऽपि बालोन्मत्तादीनां पयआदौ पाययितव्ये चूडावर्धनादिवचनम्~। प्रकारान्तरस्थानां च साक्षादेव वा प्रामाण्यं सिद्धम्~, प्रागात्मज्ञानात् देहाभिमाननिमित्तप्रत्यक्षादिप्रामाण्यवत्~। यत्तु मन्यसे — स्वयमव्याप्रियमाणोऽपि आत्मा संनिधिमात्रेण करोति, तदेव मुख्यं कर्तृत्वमात्मनः~; यथा राजा युध्यमानेषु योधेषु युध्यत इति प्रसिद्धं स्वयमयुध्यमानोऽपि संनिधानादेव जितः पराजितश्चेति, तथा सेनापतिः वाचैव करोति~; क्रियाफलसम्बन्धश्च राज्ञः सेनापतेश्च दृष्टः~। यथा च ऋत्विक्कर्म यजमानस्य, तथा देहादीनां कर्म आत्मकृतं स्यात्~, फलस्य आत्मगामित्वात्~। यथा वा भ्रामकस्य लोहभ्रामयितृत्वात् अव्यापृतस्यैव मुख्यमेव कर्तृत्वम्~, तथा च आत्मनः इति~। तत् असत्~; अकुर्वतः कारकत्वप्रसङ्गात्~। कारकमनेकप्रकारमिति चेत्~, न~; राजप्रभृतीनां मुख्यस्यापि कर्तृत्वस्य दर्शनात्~। राजा तावत् स्वव्यापारेणापि युध्यते~; योधानां च योधयितृत्वे धनदाने च मुख्यमेव कर्तृत्वम्~, तथा जयपराजयफलोपभोगे~। यजमानस्यापि प्रधानत्यागे दक्षिणादाने च मुख्यमेव कर्तृत्वम्~। तस्मात् अव्यापृतस्य कर्तृत्वोपचारो यः, सः गौणः इति अवगम्यते~। यदि मुख्यं कर्तृत्वं स्वव्यापारलक्षणं नोपलभ्यते राजयजमानप्रभृतीनाम्~, तदा संनिधिमात्रेणापि कर्तृत्वं मुख्यं परिकल्प्येत~; यथा भ्रामकस्य लोहभ्रमणेन, न तथा राजयजमानादीनां स्वव्यापारो नोपलभ्यते~। तस्मात् संनिधिमात्रेण कर्तृत्वं गौणमेव~। तथा च सति तत्फलसम्बन्धोऽपि गौण एव स्यात्~। न गौणेन मुख्यं कार्यं निर्वर्त्यते~। तस्मात् असदेव एतत् गीयते ‘देहादीनां व्यापारेण अव्यापृतः आत्मा कर्ता भोक्ता च स्यात्’ इति~। भ्रान्तिनिमित्तं तु सर्वम् उपपद्यते, यथा स्वप्ने~; मायायां च एवम्~। न च देहाद्यात्मप्रत्ययभ्रान्तिसन्तानविच्छेदेषु सुषुप्तिसमाध्यादिषु कर्तृत्वभोक्तृत्वाद्यनर्थः उपलभ्यते~। तस्मात् भ्रान्तिप्रत्ययनिमित्तः एव अयं संसारभ्रमः, न तु परमार्थः~; इति सम्यग्दर्शनात् अत्यन्त एवोपरम इति सिद्धम्~॥~६६~॥\par
 सर्वं गीताशास्त्रार्थमुपसंहृत्य अस्मिन्नध्याये, विशेषतश्च अन्ते, इह शास्त्रार्थदार्ढ्याय सङ्क्षेपतः उपसंहारं कृत्वा, अथ इदानीं शास्त्रसम्प्रदायविधिमाह —} 
\begin{center}{\bfseries इदं ते नातपस्काय\\ नाभक्ताय कदाचन~।\\न चाशुश्रूषवे वाच्यं\\ न च मां योऽभ्यसूयति~॥~६७~॥}\end{center} 
इदं शास्त्रं ते तव हिताय मया उक्तं संसारविच्छित्तये अतपस्काय तपोरहिताय न वाच्यम् इति व्यवहितेन सम्बध्यते~। तपस्विनेऽपि अभक्ताय गुरौ देवे च भक्तिरहिताय कदाचन कस्याञ्चिदपि अवस्थायां न वाच्यम्~। भक्तः तपस्वी अपि सन् अशुश्रूषुः यो भवति तस्मै अपि न वाच्यम्~। न च यो मां वासुदेवं प्राकृतं मनुष्यं मत्वा अभ्यसूयति आत्मप्रशंसादिदोषाध्यारोपणेन ईश्वरत्वं मम अजानन् न सहते, असावपि अयोग्यः, तस्मै अपि न वाच्यम्~। भगवति अनसूयायुक्ताय तपस्विने भक्ताय शुश्रूषवे वाच्यं शास्त्रम् इति सामर्थ्यात् गम्यते~। तत्र ‘मेधाविने तपस्विने वा’\footnote{यास्क. नि. २~। १~। ६} इति अनयोः विकल्पदर्शनात् शुश्रूषाभक्तियुक्ताय तपस्विने तद्युक्ताय मेधाविने वा वाच्यम्~। शुश्रूषाभक्तिवियुक्ताय न तपस्विने नापि मेधाविने वाच्यम्~। भगवति असूयायुक्ताय समस्तगुणवतेऽपि न वाच्यम्~। गुरुशुश्रूषाभक्तिमते च वाच्यम् इत्येषः शास्त्रसम्प्रदायविधिः~॥~६७~॥\par
 सम्प्रदायस्य कर्तुः फलम् इदानीम् आह —} 
\begin{center}{\bfseries य इमं परमं गुह्यं\\ मद्भक्तेष्वभिधास्यति~।\\भक्तिं मयि परां कृत्वा\\ मामेवैष्यत्यसंशयः~॥~६८~॥}\end{center} 
यः इमं यथोक्तं परमं परमनिःश्रेयसार्थं केशवार्जुनयोः संवादरूपं ग्रन्थं गुह्यं गोप्यतमं मद्भक्तेषु मयि भक्तिमत्सु अभिधास्यति वक्ष्यति, ग्रन्थतः अर्थतश्च स्थापयिष्यतीत्यर्थः, यथा त्वयि मया~। भक्तेः पुनर्ग्रहणात् भक्तिमात्रेण केवलेन शास्त्रसम्प्रदाने पात्रं भवतीति गम्यते~। कथम् अभिधास्यति इति, उच्यते — भक्तिं मयि परां कृत्वा ‘भगवतः परमगुरोः अच्युतस्य शुश्रूषा मया क्रियते’ इत्येवं कृत्वेत्यर्थः~। तस्य इदं फलम् — मामेव एष्यति मुच्यते एव~। असंशयः अत्र संशयः न कर्तव्यः~॥~६८~॥\par
 किञ्च —} 
\begin{center}{\bfseries न च तस्मान्मनुष्येषु\\ कश्चिन्मे प्रियकृत्तमः~।\\भविता न च मे तस्मा—\\ दन्यः प्रियतरो भुवि~॥~६९~॥}\end{center} 
न च तस्मात् शास्त्रसम्प्रदायकृतः मनुष्येषु मनुष्याणां मध्ये कश्चित् मे मम प्रियकृत्तमः अतिशयेन प्रियकरः, अन्यः प्रियकृत्तमः, नास्त्येव इत्यर्थः वर्तमानेषु~। न च भविता भविष्यत्यपि काले तस्मात् द्वितीयः अन्यः प्रियतरः प्रियकृत्तरः भुवि लोकेऽस्मिन् न भविता~॥~६९~॥\par
 योऽपि —} 
\begin{center}{\bfseries अध्येष्यते च य इमं\\ धर्म्यं संवादमावयोः~।\\ज्ञानयज्ञेन तेनाह—\\ मिष्टः स्यामिति मे मतिः~॥~७०~॥}\end{center} 
अध्येष्यते च पठिष्यति यः इमं धर्म्यं धर्मादनपेतं संवादरूपं ग्रन्थं आवयोः, तेन इदं कृतं स्यात्~। ज्ञानयज्ञेन — विधिजपोपांशुमानसानां यज्ञानां ज्ञानयज्ञः मानसत्वात् विशिष्टतमः इत्यतः तेन ज्ञानयज्ञेन गीताशास्त्रस्य अध्ययनं स्तूयते~; फलविधिरेव वा, देवतादिविषयज्ञानयज्ञफलतुल्यम् अस्य फलं भवतीति — तेन अध्ययनेन अहम् इष्टः पूजितः स्यां भवेयम् इति मे मम मतिः निश्चयः~॥~७०~॥\par
 अथ श्रोतुः इदं फलम् —} 
\begin{center}{\bfseries श्रद्धावाननसूयश्च\\ शृणुयादपि यो नरः~।\\सोऽपि मुक्तः शुभांल्लोका—\\ न्प्राप्नुयात्पुण्यकर्मणाम्~॥~७१~॥}\end{center} 
श्रद्धावान् श्रद्दधानः अनसूयश्च असूयावर्जितः सन् इमं ग्रन्थं शृणुयादपि यो नरः, अपिशब्दात् किमुत अर्थज्ञानवान्~, सोऽपि पापात् मुक्तः शुभान् प्रशस्तान् लोकान् प्राप्नुयात् पुण्यकर्मणाम् अग्निहोत्रादिकर्मवताम्~॥~७१~॥\par
 शिष्यस्य शास्त्रार्थग्रहणाग्रहणविवेकबुभुत्सया पृच्छति~। तदग्रहणे ज्ञाते पुनः ग्राहयिष्यामि उपायान्तरेणापि इति प्रष्टुः अभिप्रायः~। यत्नान्तरं च आस्थाय शिष्यस्य कृतार्थता कर्तव्या इति आचार्यधर्मः प्रदर्शितो भवति —} 
\begin{center}{\bfseries कच्चिदेतच्छ्रुतं पार्थ\\ त्वयैकाग्रेण चेतसा~।\\कच्चिदज्ञानसंमोहः\\ प्रणष्टस्ते धनञ्जय~॥~७२~॥}\end{center} 
कच्चित् किम् एतत् मया उक्तं श्रुतं श्रवणेन अवधारितं पार्थ, त्वया एकाग्रेण चेतसा चित्तेन~? किं वा अप्रमादतः~? कच्चित् अज्ञानसंमोहः अज्ञाननिमित्तः संमोहः अविविक्तभावः अविवेकः स्वाभाविकः किं प्रणष्टः~? यदर्थः अयं शास्त्रश्रवणायासः तव, मम च उपदेष्टृत्वायासः प्रवृत्तः, ते तुभ्यं हे धनञ्जय~॥~७२~॥\par
 {\bfseries अर्जुन उवाच —}\\
\begin{center}{\bfseries नष्टो मोहः स्मृतिर्लब्धा\\ त्वत्प्रसादान्मयाच्युत~।\\स्थितोऽस्मि गतसन्देहः\\ करिष्ये वचनं तव~॥~७३~॥}\end{center} 
नष्टः मोहः अज्ञानजः समस्तसंसारानर्थहेतुः, सागर इव दुरुत्तरः~। स्मृतिश्च आत्मतत्त्वविषया लब्धा, यस्याः लाभात् सर्वहृदयग्रन्थीनां विप्रमोक्षः~; त्वत्प्रसादात् तव प्रसादात् मया त्वत्प्रसादम् आश्रितेन अच्युत~। अनेन मोहनाशप्रश्नप्रतिवचनेन सर्वशास्त्रार्थज्ञानफलम् एतावदेवेति निश्चितं दर्शितं भवति, यतः ज्ञानात् मोहनाशः आत्मस्मृतिलाभश्चेति~। तथा च श्रुतौ ‘अनात्मवित् शोचामि’\footnote{छा. उ. ७~। १~। ३} इति उपन्यस्य आत्मज्ञानेन सर्वग्रन्थीनां विप्रमोक्षः उक्तः~; ‘भिद्यते हृदयग्रन्थिः’\footnote{मु. उ. २~। २~। ९} ‘तत्र को मोहः कः शोकः एकत्वमनुपश्यतः’\footnote{ई. उ. ७} इति च मन्त्रवर्णः~। अथ इदानीं त्वच्छासने स्थितः अस्मि गतसन्देहः मुक्तसंशयः~। करिष्ये वचनं तव~। अहं त्वत्प्रसादात् कृतार्थः, न मे कर्तव्यम् अस्ति इत्यभिप्रायः~॥~७३~॥\par
 परिसमाप्तः शास्त्रार्थः~। अथ इदानीं कथासम्बन्धप्रदर्शनार्थं सञ्जयः उवाच —}\\ 
\begin{center}{\bfseries सञ्जय उवाच —\\ इत्यहं वासुदेवस्य\\ पार्थस्य च महात्मनः~।\\संवादमिममश्रौष—\\ मद्भुतं रोमहर्षणम्~॥~७४~॥}\end{center} 
इति एवम् अहं वासुदेवस्य पार्थस्य च महात्मनः संवादम् इमं यथोक्तम् अश्रौषं श्रुतवान् अस्मि अद्भुतम् अत्यन्तविस्मयकरं रोमहर्षणं रोमाञ्चकरम्~॥~७४~॥\par
 तं च इमम् —} 
\begin{center}{\bfseries व्यासप्रसादाच्छ्रुतवा—\\ निमं गुह्यतमं परम्~।\\योगं योगेश्वरात्कृष्णा—\\ त्साक्षात्कथयतः स्वयम्~॥~७५~॥}\end{center} 
व्यासप्रसादात् ततः दिव्यचक्षुर्लाभात् श्रुतवान् इमं संवादं गुह्यतमं परं योगम्~, योगार्थत्वात् ग्रन्थोऽपि योगः, संवादम् इमं योगमेव वा योगेश्वरात् कृष्णात् साक्षात् कथयतः स्वयम्~, न परम्परया~॥~७५~॥\par
 \begin{center}{\bfseries राजन् संस्मृत्य संस्मृत्य\\ संवादमिममद्भुतम्~।\\केशवार्जुनयोः पुण्यं\\ हृष्यामि च मुहुर्मुहुः~॥~७६~॥}\end{center} 
हे राजन् धृतराष्ट्र, संस्मृत्य संस्मृत्य प्रतिक्षणं संवादम् इमम् अद्भुतं केशवार्जुनयोः पुण्यम् इमं श्रवणेनापि पापहरं श्रुत्वा हृष्यामि च मुहुर्मुहुः प्रतिक्षणम्~॥~७६~॥\par
 \begin{center}{\bfseries तच्च संस्मृत्य संस्मृत्य\\ रूपमत्यद्भुतं हरेः~।\\विस्मयो मे महान्राजन्\\ हृष्यामि च पुनः पुनः~॥~७७~॥}\end{center} 
तच्च संस्मृत्य संस्मृत्य रूपम् अत्यद्भुतं हरेः विश्वरूपं विस्मयो मे महान् राजन्~, हृष्यामि च पुनः पुनः~॥~७७~॥\par
 किं बहुना —} 
\begin{center}{\bfseries यत्र योगेश्वरः कृष्णो\\ यत्र पार्थो धनुर्धरः~।\\तत्र श्रीर्विजयो भूति—\\ र्ध्रुवा नीतिर्मतिर्मम~॥~७८~॥}\end{center} 
यत्र यस्मिन् पक्षे योगेश्वरः सर्वयोगानाम् ईश्वरः, तत्प्रभवत्वात् सर्वयोगबीजस्य, कृष्णः, यत्र पार्थः यस्मिन् पक्षे धनुर्धरः गाण्डीवधन्वा, तत्र श्रीः तस्मिन् पाण्डवानां पक्षे श्रीः विजयः, तत्रैव भूतिः श्रियो विशेषः विस्तारः भूतिः, ध्रुवा अव्यभिचारिणी नीतिः नयः, इत्येवं मतिः मम इति~॥~७८~॥\par
 
इति श्रीमत्परमहंसपरिव्राजकाचार्यस्य श्रीगोविन्दभगवत्पूज्यपादशिष्यस्य श्रीमच्छङ्करभगवतः कृतौ श्रीमद्भगवद्गीताभाष्ये अष्टादशोऽध्यायः~॥\par
