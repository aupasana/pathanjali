॥ श्रीललिता त्रिशती पूर्व पीठिका ॥
अगस्त्य उवाच –
हयग्रीव दया सिन्धो भगवन्शिष्य वत्सल ।
त्वत्तः श्रुतमशेषेण श्रोतव्यं यद्यदस्तितत्॥ १॥
रहस्य नाम साहस्रमपि त्वत्तः श्रुतं मय ।
इतः परं मे नास्त्येव श्रोतव्यमिति निश्चयः ॥ २॥
तथापि मम चित्तस्य पर्याप्तिर्नैव जायते।
कार्त्स्न्यार्थः प्राप्य इत्येव शोचयिष्याम्यहं प्रभो ॥ ३॥
किमिदं कारणं ब्रूहि ज्ञातव्यांशोऽस्ति वा पुनः ।
अस्ति चेन्मम तद्ब्रूहि ब्रूहीत्युक्ता प्रणम्य तम्॥ ४॥
सूत उवाच -
समाललम्बे तत्पाद युगळं कलशोद्भवः ।
हयाननो भीतभीतः किमिदं किमिदं त्विति ॥ ५॥
मुञ्चमुञ्चेति तं चोक्का चिन्ताक्रान्तो बभूव सः ।
चिरं विचार्य निश्चिन्वन्वक्तव्यं न मयेत्यसौ ॥ ६॥
तष्णी स्थितः स्मरन्नाज्ञां ललिताम्बाकृतां पुरा ।
प्रणम्य विप्रं समुनिस्तत्पादावत्यजन्स्थितः ॥ ७॥
वर्षत्रयावधि तथा गुरुशिष्यौ तथा स्थितौ।
तछृंवन्तश्च पश्यन्तः सर्वे लोकाः सुविस्मिताः ॥ ८॥
तत्र श्रीललितादेवी कामेश्वरसमन्विता ।
प्रादुर्भूता हयग्रीवं रहस्येवमचोदयत्॥ ९॥
श्रीदेवी उवाच -
आश्वाननावयोः प्रीतिः शास्त्रविश्वासिनि त्वयि ।
राज्यं देयं शिरो देयं न देया षोडशाक्षरी ॥ १०॥
स्वमातृ जारवत्गोप्या विद्यैषत्यागमा जगुः ।
ततो ऽतिगोपनिया मे सर्वपूर्तिकरी स्तुतिः ॥ ११॥
मया कामेश्वरेणापि कृता साङ्गोपिता भृशम्।
मदाज्ञया वचोदेव्यश्चत्ररर्नामसहस्रकम्॥ १२॥
आवाभ्यां कथिता मुख्या सर्वपूर्तिकरी स्तुतिः ।
सर्वक्रियाणां वैकल्यपूर्तिर्यज्जपतो भवेत्॥ १३॥
सर्व पूर्तिकरं तस्मादिदं नाम कृतं मया ।
तद्ब्रूहि त्वमगस्त्याय पात्रमेव न संशयः ॥ १४॥
पत्न्यस्य लोपामुद्राख्या मामुपास्तेऽतिभक्तितः ।
अयञ्च नितरां भक्तस्तस्मादस्य वदस्व तत्॥ १५॥
अमुञ्चमानस्त्वद्वादौ वर्षत्रयमसौ स्थितः ।
एतज्ज्ञातुमतो भक्तया हितमेव निदर्शनम्॥ १६॥
चित्तपर्याप्तिरेतस्य नान्यथा सम्भविष्यती ।
सर्वपूर्तिकरं तस्मादनुज्ञातो मया वद ॥ १७॥
सूत उवाच -
इत्युक्तान्तरधदाम्बा कामेश्वरसमन्विता ।
अथोत्थाप्य हयग्रीवः पाणिभ्यां कुम्भसम्भवम्॥ १८॥
संस्थाप्य निकटेवाच उवाच भृश विस्मितः ।
हयग्रीव उवाच –
कृतार्थोऽसि कृतार्थोऽसि कृतार्थोऽसि घटोद्भव ॥ १९॥
त्वत्समो ललिताभक्तो नास्ति नास्ति जगत्रये ।
एनागस्त्य स्वयं देवी तववक्तव्यमन्वशात्॥ २०॥
सच्छिष्येन त्वया चाहं दृष्ट्वानस्मि तां शिवाम्।
यतन्ते दर्शनार्थाय ब्रह्मविष्ण्वीशपूर्वकाः ॥ २१॥
अतः परं ते वक्ष्यामि सर्वपूर्तिकरं स्थवम्।
यस्य स्मरण मात्रेण पर्याप्तिस्ते भवेद्धृदि ॥ २२॥
रहस्यनाम साह्स्रादपि गुह्यतमं मुने ।
आवश्यकं ततोऽप्येतल्ललितां समुपासितुम्॥ २३॥
तदहं सम्प्रवक्ष्यामि ललिताम्बानुशासनात्।
श्रीमत्पञ्चदशाक्षर्याः कादिवर्णान्क्रामन्मुने ॥ २४॥
पृथग्विंशति नामानि कथितानि घटोद्भव ।
आहत्य नाम्नां त्रिशती सर्वसम्पूर्तिकारणी ॥ २५॥
रहस्यादिरहस्यैषा गोपनीया प्रयत्नतः ।
तां शृणुष्व महाभाग सावधानेन चेतसा ॥ २६॥
केवलं नामबुद्धिस्ते न कार्य तेषु कुम्भज।
मन्त्रात्मकं एतेषां नाम्नां नामात्मतापि च ॥ २७॥
तस्मादेकाग्रमनसा श्रोतव्यं च त्वया सदा ।
सूत उवाच -
इति युक्ता तं हयग्रीवः प्रोचे नामशतत्रयम्॥ २८॥
॥ इति श्रीललिता त्रिशती स्तोत्र पूर्व पीठिका सम्पूर्णम्।

॥ न्यासः॥
अस्य श्रीललितात्रिशती स्तोत्र महामन्त्रस्य भगवान्हयग्रीव ऋषिः,
अनुष्टप्छन्दः, श्रीललितामहात्रिपुरसुन्दरी देवता,
ऐं बीजम् , सौः शक्तिः, क्लों कीलकम् , जपे परायणे विनियोगः ॥
मूलेन न्यासः ॥
॥ ध्यानम्॥
अतिमधुरचापहस्ताम्परिमितामोदसौभाग्याम्।
अरुणामतिशयकरुणामभिनवकुलसुन्दरीं वन्दे ॥
॥ लं इत्यादि पञ्च पूजा ॥

॥ अथ श्रीललितात्रिशती स्तोत्रम्॥
ककाररूपा कल्याणी कल्याणगुणशालिनी ।
कल्याणशैलनिलया कमनीया कलावती ॥ १॥
कमलाक्षी कल्मषघ्नी करुणामृत सागरा ।
कदम्बकाननावासा कदम्ब कुसुमप्रिया ॥ २॥
कन्दर्पविद्या कन्दर्प जनकापाङ्ग वीक्षणा ।
कर्पूरवीटी सौरभ्य कल्लोलित ककुप्तटा ॥ ३॥
कलिदोषहरा कञ्जलोचना कम्रविग्रहा ।
कर्मादि साक्षिणी कारयित्री कर्मफलप्रदा ॥ ४॥
एकाररूपा चैकाक्षर्येकानेकाक्षराकृतिः ।
एतत्तदित्यनिर्देश्या चैकानन्द चिदाकृतिः ॥ ५॥
एवमित्यागमाबोध्या चैकभक्ति मदर्चिता ।
एकाग्रचित्त निर्ध्याता चैषणा रहिताद्दृता ॥ ६॥
एलासुगन्धिचिकुरा चैनः कूट विनाशिनी ।
एकभोगा चैकरसा चैकैश्वर्य प्रदायिनी ॥ ७॥
एकातपत्र साम्राज्य प्रदा चैकान्तपूजिता ।
एधमानप्रभा चैजदनेकजगदीश्वरी ॥ ८॥
एकवीरादि संसेव्या चैकप्राभव शालिनी ।
ईकाररूपा चेशित्री चेप्सितार्थ प्रदायिनी ॥ ९॥
ईद्दृगित्य विनिर्देश्या चेश्वरत्व विधायिनी ।
ईशानादि ब्रह्ममयी चेशित्वाद्यष्ट सिद्धिदा ॥ १०॥
ईक्षित्रीक्षण सृष्टाण्ड कोटिरीश्वर वल्लभा ।
ईडिता चेश्वरार्धाङ्ग शरीरेशाधि देवता ॥ ११॥
ईश्वर प्रेरणकरी चेशताण्डव साक्षिणी ।
ईश्वरोत्सङ्ग निलया चेतिबाधा विनाशिनी ॥ १२॥
ईहाविराहिता चेश शक्ति रीषत्स्मितानना ।
लकाररूपा ललिता लक्ष्मी वाणी निषेविता ॥ १३॥
लाकिनी ललनारूपा लसद्दाडिम पाटला ।
ललन्तिकालसत्फाला ललाट नयनार्चिता ॥ १४॥
लक्षणोज्ज्वल दिव्याङ्गी लक्षकोट्यण्ड नायिका ।
लक्ष्यार्था लक्षणागम्या लब्धकामा लतातनुः ॥ १५॥
ललामराजदलिका लम्बिमुक्तालताञ्चिता ।
लम्बोदर प्रसूर्लभ्या लज्जाढ्या लयवर्जिता ॥ १६॥
ह्रींकार रूपा ह्रींकार निलया ह्रींपदप्रिया ।
ह्रींकार बीजा ह्रींकारमन्त्रा ह्रींकारलक्षणा ॥ १७॥
ह्रींकारजप सुप्रीता ह्रींमती ह्रींविभूषणा ।
ह्रींशीला ह्रींपदाराध्या ह्रींगर्भा ह्रींपदाभिधा ॥ १८॥
ह्रींकारवाच्या ह्रींकार पूज्या ह्रींकार पीठिका ।
ह्रींकारवेद्या ह्रींकारचिन्त्या ह्रीं ह्रींशरीरिणी ॥ १९॥
हकाररूपा हलधृत्पूजिता हरिणेक्षणा ।
हरप्रिया हराराध्या हरिब्रह्मेन्द्र वन्दिता ॥ २०॥
हयारूढा सेवितांघ्रिर्हयमेध समर्चिता ।
हर्यक्षवाहना हंसवाहना हतदानवा ॥ २१॥
हत्यादिपापशमनी हरिदश्वादि सेविता ।
हस्तिकुम्भोत्तुङ्क कुचा हस्तिकृत्ति प्रियाङ्गना ॥ २२॥
हरिद्राकुङ्कुमा दिग्धा हर्यश्वाद्यमरार्चिता ।
हरिकेशसखी हादिविद्या हालामदोल्लसा ॥ २३॥
सकाररूपा सर्वज्ञा सर्वेशी सर्वमङ्गला ।
सर्वकर्त्री सर्वभर्त्री सर्वहन्त्री सनातना ॥ २४॥
सर्वानवद्या सर्वाङ्ग सुन्दरी सर्वसाक्षिणी ।
सर्वात्मिका सर्वसौख्य दात्री सर्वविमोहिनी ॥ २५॥
सर्वाधारा सर्वगता सर्वावगुणवर्जिता ।
सर्वारुणा सर्वमाता सर्वभूषण भूषिता ॥ २६॥
ककारार्था कालहन्त्री कामेशी कामितार्थदा ।
कामसञ्जीविनी कल्या कठिनस्तन मण्डला ॥ २७॥
करभोरुः कलानाथ मुखी कचजिताम्भुदा ।
कटाक्षस्यन्दि करुणा कपालि प्राण नायिका ॥ २८॥
कारुण्य विग्रहा कान्ता कान्तिधूत जपावलिः ।
कलालापा कम्बुकण्ठी करनिर्जित पल्लवा ॥ २९॥
कल्पवल्ली समभुजा कस्तूरी तिलकाञ्चिता ।
हकारार्था हंसगतिर्हाटकाभरणोज्ज्वला ॥ ३०॥
हारहारि कुचाभोगा हाकिनी हल्यवर्जिता ।
हरित्पति समाराध्या हठात्कार हतासुरा ॥ ३१॥
हर्षप्रदा हविर्भोक्त्री हार्द सन्तमसापहा ।
हल्लीसलास्य सन्तुष्टा हंसमन्त्रार्थ रूपिणी ॥ ३२॥
हानोपादान निर्मुक्ता हर्षिणी हरिसोदरी ।
हाहाहूहू मुख स्तुत्या हानि वृद्धि विवर्जिता ॥ ३३॥
हय्यङ्गवीन हृदया हरिकोपारुणांशुका ।
लकाराख्या लतापूज्या लयस्थित्युद्भवेश्वरी ॥ ३४॥
लास्य दर्शन सन्तुष्टा लाभालाभ विवर्जिता ।
लङ्घ्येतराज्ञा लावण्य शालिनी लघु सिद्धिदा ॥ ३५॥
लाक्षारस सवर्णाभा लक्ष्मणाग्रज पूजिता ।
लभ्यतरा लब्ध भक्ति सुलभा लाङ्गलायुधा ॥ ३६॥
लग्नचामर हस्त श्रीशारदा परिवीजिता ।
लज्जापद समाराध्या लम्पटा लकुलेश्वरी ॥ ३७॥
लब्धमाना लब्धरसा लब्ध सम्पत्समुन्नतिः ।
ह्रींकारिणी च ह्रींकारी ह्रींमध्या ह्रींशिखामणिः ॥ ३८॥
ह्रींकारकुण्डाग्नि शिखा ह्रींकार शशिचन्द्रिका ।
ह्रींकार भास्कररुचिर्ह्रींकारांभोद चञ्चला ॥ ३९॥
ह्रींकार कन्दाङ्कुरिका ह्रींकारैक परायणाम्।
ह्रींकार दीर्घिकाहंसी ह्रींकारोद्यान केकिनी ॥ ४०॥
ह्रींकारारण्य हरिणी ह्रींकारावाल वल्लरी ।
ह्रींकार पञ्जरशुकी ह्रींकाराङ्गण दीपिका ॥ ४१॥
ह्रींकार कन्दरा सिंही ह्रींकाराम्भोज भृङ्गिका ।
ह्रींकार सुमनो माध्वी ह्रींकार तरुमंजरी ॥ ४२॥
सकाराख्या समरसा सकलागम संस्तुता ।
सर्ववेदान्त तात्पर्यभूमिः सदसदाश्रया ॥ ४३॥
सकला सच्चिदानन्दा साध्या सद्गतिदायिनी ।
सनकादिमुनिध्येया सदाशिव कुटुम्बिनी ॥ ४४॥
सकालाधिष्ठान रूपा सत्यरूपा समाकृतिः ।
सर्वप्रपञ्च निर्मात्री समनाधिक वर्जिता ॥ ४५॥
सर्वोत्तुङ्गा सङ्गहीना सगुणा सकलेश्वरी । सकलेष्टदा
ककारिणी काव्यलोला कामेश्वर मनोहरा ॥ ४६॥
कामेश्वरप्रणानाडी कामेशोत्सङ्ग वासिनी ।
कामेश्वरालिङ्गिताङ्गी कमेश्वर सुखप्रदा ॥ ४७॥
कामेश्वर प्रणयिनी कामेश्वर विलासिनी ।
कामेश्वर तपः सिद्धिः कामेश्वर मनः प्रिया ॥ ४८॥
कामेश्वर प्राणनाथा कामेश्वर विमोहिनी ।
कामेश्वर ब्रह्मविद्या कामेश्वर गृहेश्वरी ॥ ४९॥
कामेश्वराह्लादकरी कामेश्वर महेश्वरी ।
कामेश्वरी कामकोटि निलया काङ्क्षितार्थदा ॥ ५०॥
लकारिणी लब्धरूपा लब्धधीर्लब्ध वाञ्चिता ।
लब्धपाप मनोदूरा लब्धाहङ्कार दुर्गमा ॥ ५१॥
लब्धशक्तिर्लब्ध देहा लब्धैश्वर्य समुन्नतिः ।
लब्धवृद्धिर्लब्धलीला लब्धयौवन शालिनी ॥ ५२॥ लब्धबुधिः
लब्धातिशय सर्वाङ्ग सौन्दर्या लब्ध विभ्रमा ।
लब्धरागा लब्धपतिर्लब्ध नानागमस्थितिः ॥ ५३॥ लब्धगति
लब्ध भोगा लब्ध सुखा लब्ध हर्षाभि पूजिता ।
ह्रींकार मूर्तिर्ह्रीण्कार सौधशृङ्ग कपोतिका ॥ ५४॥
ह्रींकार दुग्धाब्धि सुधा ह्रींकार कमलेन्दिरा ।
ह्रींकारमणि दीपार्चिर्ह्रींकार तरुशारिका ॥ ५५॥
ह्रींकार पेटक मणिर्ह्रींकारदर्श बिम्बिता ।
ह्रींकार कोशासिलता ह्रींकारास्थान नर्तकी ॥ ५६॥
ह्रींकार शुक्तिका मुक्तामणिर्ह्रींकार बोधिता ।
ह्रींकारमय सौवर्णस्तम्भ विद्रुम पुत्रिका ॥ ५७॥
ह्रींकार वेदोपनिषद् ह्रींकाराध्वर दक्षिणा ।
ह्रींकार नन्दनाराम नवकल्पक वल्लरी ॥ ५८॥
ह्रींकार हिमवद्गङ्गा ह्रींकारार्णव कौस्तुभा ।
ह्रींकार मन्त्र सर्वस्वा ह्रींकारपर सौख्यदा ॥ ५९॥
॥ इति श्रीललितात्रिशतीस्तोत्रं सम्पूर्णम्॥

॥ श्रीललिता त्रिशती उत्तर पीठिका ॥
हयग्रीव उवाच -
इत्येवं ते मयाख्यातं देव्या नामशतत्रयम्।
रहस्यातिरहस्यत्वाद्गोपनीयं त्वया मुने ॥ १॥
शिववर्णानि नामानि श्रीदेव्या कथितानि हि ।
शक्तयक्षराणि नामानि कामेशकथितानि च ॥ २॥
उभयाक्षरनामानि ह्युभाभ्यां कथितानि वै ।
तदन्यैर्ग्रथितं स्तोत्रमेतस्य सदृशं किमु ॥ ३॥
नानेन सदृशं स्तोत्रं श्रीदेवी प्रीतिदायकम्।
लोकत्रयेऽपि कल्याणं सम्भवेन्नात्र संशयः॥ ४॥
सूत उवाच -
इति हयमुखगीतं स्तोत्रराजं निशम्य
प्रगलित कलुषोऽभृच्चित्तपर्याप्तिमेत्य ।
निजगुरुमथ नत्वा कुम्भजन्मा तदुक्तं
पुनरधिकरहस्यं ज्ञातुमेवं जगाद ॥ ५॥
अगस्त्य उवाच –
अश्वानन महाभाग रहस्यमपि मे वद ।
शिववर्णानि कान्यत्र शक्तिवर्णानि कानि हि ॥ ६॥
उभयोरपि वर्णानि कानि वा वद देशिक।
इति पृष्टः कुम्भजेन हयग्रीवोऽवदत्युनः ॥ ७॥
हयग्रीव उवाच -
तव गोप्यं किमस्तीह साक्षादम्बानुशासनात्।
इदं त्वतिरहस्यं ते वक्ष्यामि कुम्भज ॥ ८॥
एतद्विज्ञनमात्रेण श्रिविद्या सिद्धिदा भवेत्।
कत्रयं हद्बयं चैव शैवो भागः प्रकीर्तितः ॥ ९॥
शक्तयक्षराणि शेषाणिह्रीङ्कार उभयात्मकः ।
एवं विभागमज्ञात्वा ये विद्याजपशालिनः ॥ १०॥
न तेशां सिद्धिदा विद्या कल्पकोटिशतैरपि ।
चतुर्भिः शिवचक्रैश्च शक्तिचक्रैश्च पञ्चभिः ॥ ११॥
नव चक्रैश्ल संसिद्धं श्रीचक्रं शिवयोर्वपुः ।
त्रिकोणमष्टकोनं च दशकोणद्बयं तथा ॥ १२॥
चतुर्दशारं चैतानि शक्तिचक्राणि पञ्च च ।
बिन्दुश्चाष्टदलं पद्मं पद्मं षोडशपत्रकम्॥ १३॥
चतुरश्रं च चत्वारि शिवचक्राण्यनुक्रमात्।
त्रिकोणे बैन्दवं श्लिष्टं अष्टारेष्टदलाम्बुजम्॥ १४॥
दशारयोः षोडशारं भूगृहं भुवनाश्रके ।
शैवानामपि शाक्तानां चक्राणां च परस्परं ॥ १५॥
अविनाभावसम्बन्धं यो जानाति स चक्रवित्।
त्रिकोणरूपिणि शक्तिर्बिन्दुरूपपरः शिवः ॥ १६॥
अविनाभावसम्बन्धं तस्माद्विन्दुत्रिकोणयोः ।
एवं विभागमज्ञात्वा श्रीचक्रं यः समर्चयेत्॥ १७॥
न तत्फलमवाप्नोति ललिताम्बा न तुष्यति ।
ये च जानन्ति लोकेऽस्मिन्श्रीविद्याचक्रवेदिनः ॥ १८॥
सामन्यवेदिनः सर्वे विशेषज्ञोऽतिदुर्लभः ।
स्वयं विद्या विशेषज्ञो विशेषज्ञ समर्चयेत्॥ १९॥
तस्मैः देयं ततो ग्राह्यमशक्तस्तव्यदापयेत्।
अन्धम्तमः प्रविशन्ति ये ऽविद्यां समुपासते ॥ २०॥
इति श्रुतिरपाहैतानविद्योपासकान्पुनः ।
विद्यान्योपासकानेव निन्दत्यारुणिकी श्रुतिः ॥ २१॥
अश्रुता सश्रुतासश्व यज्चानों येऽप्ययञ्जनः ।
सवर्यन्तो नापेक्षन्ते इन्द्रमग्निश्च ये विदुः ॥ २२॥
सिकता इव संयन्ति रश्मिभिः समुदीरिताः ।
अस्माल्लोकादमुष्माच्चेत्याह चारण्यक श्रुतिः ॥ २३॥
यस्य नो पश्चिमं जन्म यदि वा शङ्करः स्वयम्।
तेनैव लभ्यते विद्या श्रीमत्पच्चदशाक्षरी ॥ २४॥
इति मन्त्रेषु बहुधा विद्याया महिमोच्यते ।
मोक्षैकहेतुविद्या तु श्रीविद्या नात्र संशयः ॥ २५॥
न शिल्पदि ज्ञानयुक्ते विद्वच्छव्धः प्रयुज्यते ।
मोक्षैकहेतुविद्या सा श्रीविद्यैव न संशयः ॥ २६॥
तस्माद्विद्याविदेवात्र विद्वान्विद्वानितीर्यते ।
स्वयं विद्याविदे दद्यात्ख्यापयेत्तद्गुणान्सुधीः ॥ २७॥
स्वयंविद्यारहस्यज्ञो विद्यामाहात्म्यमवेद्यपि
विद्याविदं नार्चयेच्चेत्को वा तं पूजयेज्जनः ॥ २८॥
प्रसङ्गादिदमुक्तं ते प्रकृतं शृणु कुम्भज ।
यः कीर्तयेत्सकृत्भक्तया दिव्यनामशतत्रयम्॥ २९॥
तस्य पुण्यमहं वक्ष्ये द्वं कुम्भसम्भव ।
रहस्यनामसाहस्रपाठे यत्फलमीरितम्॥ ३०॥
तत्फलं कोटिगुणितमेकनामजपाद्भवेत्।
कामेश्वरीकामेशाभ्यां कृतं नामशतत्रयम्॥ ३१॥
नान्येन तुलयेदेतत्स्तोत्रेणान्य कृतेन च ।
श्रियः परम्परा यस्य भावि वा चोत्तरोत्तरम्॥ ३२॥
तेनैव लभ्यते चैतत्पश्चाच्छेयः परीक्षयेत्।
अस्या नाम्नां त्रिशत्यास्तु महिमा केन वर्णयते ॥ ३३॥
या स्वयं शिवयोर्वक्तपद्माभ्यां परिनिःसृता ।
नित्यं षोडशसङ्ख्याकान्विप्रानादौ तु भोजयेत्॥ ३४॥
अभ्यक्ताम्सितिलतैलेन स्नातानुष्णेन वारिणा ।
अभ्यर्च गन्धपुष्पाद्यैः कामेश्वर्यादिनामभिः ॥ ३५॥
सूपापूपैः शर्कराद्मैः पायसैः फलसंयुतैः ।
विद्याविदो विशेषेण भोजयेत्पोडश द्विजान्॥ ३६॥
एवं नित्यार्चनं कुर्यातादौ ब्राह्मण भोजनम्।
त्रिशतीनामभिः पश्चाद्वाह्मणान्क्रमशोऽर्चयेत्॥ ३७॥
तैलाभ्यङ्गातिकं दत्वा विभवे सति भक्तितः ।
शुक्लप्रतिपदारभ्य पौर्णमास्यवधि क्रमात्॥ ३८॥
दिवसे दिवसे विप्रा भोज्या विम्शतीसङ्ख्यया ।
दशभिः पञ्चभिर्वापि त्रीभिरेकनवा दिनैः ॥ ३९॥
त्रिम्शत्पष्टिः शतं विप्राः सम्भोज्यस्तिशतं क्रमात्।
एवं यः कुरुते भक्तया जन्ममध्ये सकृन्नरः ॥ ४०॥
तस्यैव सफलं जन्म मुक्तिस्तस्य करे स्थिराः ।
रहस्यनाम साहस्त्र भोजनेऽप्येव्मेवहि ॥ ४१॥
आदौ नित्यबलिं कुर्यात्पश्चाद्वाह्मणभोजनम्।
रहस्यनामसाहस्रमहिमा यो मयोदितः ॥ ४२॥
सशिकराणुरत्रैकनामप्नो महिमवारिधेः ।
वाग्देवीरचिते नामसाहस्ने यद्यदीरितम्॥ ४३॥
तत्फलं कोटिगुणितं नाम्नोऽप्येकस्य कीर्तनात्।
एतन्यैर्जपैः स्तोत्रैरर्चनैर्यत्फलं भवेत्॥ ४४॥
तत्फलं कोटिगुणितं भवेन्नामशतत्रयात्।
वाग्देविरचितास्तोत्रे तादृशो महिमा यदि ॥ ४५॥
साक्षात्कामेशकामेशी कृते ऽस्मिन्गृहृतामिति ।
सकृत्सन्कीर्तनादेव नाम्नाम्नस्मिव्शतत्रये ॥ ४६॥
भवेच्चित्तस्य पर्यप्तिर्न्यूनमन्यानपेक्षिणी ।
न ज्ञातव्यमितोऽप्यन्यत्र जप्तव्यश्च कुम्भज ॥ ४७॥
यद्यत्साध्यतमं कार्य तत्तदर्थमिदञ्जपेत्।
तत्तत्फलमवाप्नोति पश्चात्कार्य परीक्षयेत्॥ ४८॥
ये ये प्रयोगास्तन्त्रेषु तैस्तैर्यत्साध्यते फलं ।
तत्सर्व सिद्धयति क्षिप्रं नामत्रिशतकीर्तनात्॥ ४९॥
आयुष्करं पुष्टिकरं पुत्रदं वश्यकारकम्।
विद्याप्रदं कीर्तिकरं सुखवित्वप्रदायकम्॥ ५०॥
सर्वसम्पत्प्रदं सर्वभोगदं सर्वसौख्यदम्।
सर्वाभिष्टप्रदं चैव देव्या नामशतत्रयम्॥ ५१॥
एतज्जपपरो भूयान्नान्यदिच्छेत्कदाचन ।
एतत्कीर्तनसन्तुष्टा श्रीदेवी ललिताम्बिका ॥ ५२॥
भक्तस्य यद्यदिष्टं स्यात्तत्तत्यूरयते ध्रुवं ।
तस्मात्कुभोद्भवमुने कीर्तय त्वमिदम्सदा ॥ ५३॥
नापरं किञ्चिदपि ते बोद्धव्यं नावशिष्यते ।
इति ते कथितं स्तोत्र ललिता प्रीतिदायकम्॥ ५४॥
नाविद्यावेदिने ब्रूयान्नाभक्ताय कदाचन ।
न शठाय न दुष्टाय नाविश्वासाय कहिर्चित्॥ ५६॥
यो ब्रूयात्रिशतीं नाम्नां तस्यानर्थो महान्भवेत्।
इत्याज्ञा शाङ्करी प्रोक्ता तस्माद्गोप्यमिदं त्वया ॥ ५७॥
ललिता प्रेरितेनैव मयोक्तम्स्तोत्रमुत्तमम्।
रहस्यनामसाहस्रादपि गोप्यमिदं मुने ॥ ५८॥
सूत उवाच -
एवमुक्त्वा हयग्रीवः कुम्भजं तापसोत्तमम्।
स्तोत्रेणानेन ललितां स्तुत्वा त्रिपुरसुन्दरी ॥
आनन्दलहरीमग्नरमानसः समवर्तत ॥ ५९॥
॥ इति श्री ब्रह्माण्डपुराणे उत्तराखण्डे
श्री हयग्रीवागस्त्यसंवादे
श्रीललितात्रिशती स्तोत्र कथनं सम्पूर्णम्॥
