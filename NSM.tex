%http://tex.stackexchange.com/questions/136900/insert-a-full-page-image

\begin{document}
विश्वनाथपञ्चाननभट्टाचार्यविरचिता न्यायसिद्धान्तमुक्तावली

चूडामणीकृतविधुर्वलयीकृतवासुकिः~।
भवो भवतु भव्याय लीलाताण्डवपण्डितः~॥१॥
निजनिर्मितकारिकावलीमतिसङ्क्षिप्तचिरन्तनोक्तिभिः~।
विशदीकरवाणि कौतुकान्ननु राजीवदयावशंवदः~॥२॥
सद्रव्या गुणगुम्फिता सुकृतिनां सत्कर्मणां ज्ञापिका
सत्सामान्यविशेषनित्यमिलिताऽभावप्रकर्षोज्ज्वला~।
विष्णोर्वक्षसि विश्वनाथकृतिना सिद्धान्तमुक्तावली
विन्यस्ता मनसो मुदं वितनुतां सद्युग्क्तिरेषा चिरम्~॥३॥
नूतनजलधररुचये गोपवधूटीदुकूलचौराय~।
तस्मै कृष्णाय नमः संसारमहीरुहस्य बीजाय~॥
विघ्नविघाताय कृतं मङ्गलं शिष्यशिक्षायै ग्रन्थतो निबध्नाति - नूतनेति~। ननु मङ्गलं न विघ्नध्वंसं प्रति न वा समाप्तिं प्रति कारणं, विनापि मङ्गलं नास्तिकादीनां प्रन्थेषु निर्विघ्नपरिसमाप्तिदर्शनादिति चेन्न~। अविगीतशिष्टाचारविषयत्वेन मङ्गलस्य सफलत्वे सिद्धे, तत्र च फलजिज्ञासायां सम्भवति
दृष्टफलकत्वेऽदृष्टफलकल्पनाया अन्याय्यत्वात् उपस्थितत्वाच्च समाप्तिरेव तत्फलं कल्प्यते~। इत्थं च यत्र मङ्गलं न दृश्यते, तत्रापि जन्मान्तरीयं तत् कल्प्यते~। यत्र च
सत्यपि मङ्गले समाप्तिर्न दृश्यते, तत्र बलवत्तरो विघ्नो विघ्नप्राचुर्यं वा बोध्यम्~। प्रचुरस्यैवाऽस्य बलवत्तरविघ्ननिवारणे कारणत्वं बोध्यम्~। विघ्नध्वंसस्तु मङ्गलस्य
द्वारमित्याहुः प्राञ्चः~।
नव्यास्तु मङ्गलस्य विघ्नध्वंस एव फलम् , समाप्तिस्तु बुद्धिप्रतिभादिकारणकलापात्~। न चैवं स्वतःसिद्धविघ्नविरहवता कृतस्य मङ्गलस्य निष्फलत्वापत्तिरिति
वाच्यम्~। इष्टापत्तेः, विघ्नशङ्कया तदाचरणात् , तथैव शिष्टाचारात्~। न च तस्य निष्फलत्वे तद्बोधकवेदाप्रामाण्यापत्तिरिति वाच्यम्~। सति विघ्ने तन्नाशस्यैव वेदबोधितत्वात्~।
अत एव पापभ्रमेण कृतस्य प्रायश्चित्तस्य निष्फलत्वेऽपि न तद्बोधकवेदाप्रामाण्यम्~। मङ्गलं च विघ्नध्वंसविशेषे कारणम् , विघ्नध्वंसविशेषे च विनायकस्तवपाठादि~।
क्वचिच्च विघ्नात्यन्ताभाव एव समाप्तिसाधनम्~। प्रतिबन्धकसंसर्गाभावस्यैव कार्यजनकत्वात्~। इत्थं च नास्तिकादीनां ग्रन्थेषु जन्मान्तरीयमङ्गलजन्यदुरितध्वंसः स्वतःसिद्धविघ्नात्यन्ताभावो वाऽस्तीति न व्यभिचार इति प्राहुः~॥
॥इति मङ्गलवादः॥
तस्मै कृष्णाय नमः संसारमहीरुहस्य बीजाय~॥१॥
संसारेति~। संसार एव महीरुहो वृक्षस्तस्य बीजाय - निमित्तकारणायेत्यर्थः~।
एतेन ईश्वरे प्रमाणमपि दर्शितं भवति~। तथा हि - यथा घटादिकार्यं कर्तृजन्यं, तथा क्षित्यङ्कुरादिकमपीति~। न च तत्कर्तृत्वमस्मदादीनां सम्भवतीत्यतस्तत्कर्तृत्वेनेश्वरसिद्धिः~।
न च शरीराजन्यत्वेन कर्त्रजन्यत्वसाधकेन सत्प्रतिपक्ष इति वाच्यम्~। अप्रयोजकत्वात्~। मम तु कर्तृत्वेन कार्यत्वेन कार्यकारणभाव एवाऽनुकूलस्तर्कः~। इत्थंञ्च "द्यावाभूमी
जनयन् देव एकः, विश्वस्य कर्ता भुवनस्य गोप्ता" इत्यादय आगमा अप्यनुसन्धेयाः~॥१॥
॥इतीश्वरानुमानम्॥
द्रव्यं गुणस्तथा कर्म सामान्यं सविशेषकम्~।
समवायस्तथाऽभावः पदार्थाः सप्त कीर्तिताः~॥२॥
द्रव्यमिति~। अत्र सप्तमस्याऽभावत्वकथनादेव षण्णां भावत्वं प्राप्तं, तेन (भावत्वेन) पृथगुपन्यासो न कृतः~। एते च पदार्था वैशेषिकनये प्रसिद्धाः, नैयायिकानामप्यविरुद्धाः~।
प्रतिपादितं चैवमेव भाष्ये~। अत एवोपमानचिन्तामणौ सप्तपदार्थभिन्नतया शक्तिसादृश्यादीनामतिरिक्तपदार्थत्वमाशङ्कितम्~।
ननु कथमेत एव पदार्थाः~? शक्तिसादृश्यादीनामप्यतिरिक्तपदार्थत्वात्~। तथा हि - मण्यादिसमवहितेन वह्निना दाहो न जन्यते, तच्छून्येन तु जन्यते, तत्र मण्यादिना
वन्हौ दाहानुकूलशक्तिर्नाश्यते, उत्तेजकेन मण्याद्यपसारणेन च जन्यते इति कल्प्यते~। एवं सादृश्यमप्यतिरिक्तः पदार्थः, तद्धि न षट्त्सु भावेष्वन्तर्भवति, सामान्येऽपि
सत्वात् , यथा गोत्वं नित्यं, तथाऽश्वत्वमपीति सादृश्यप्रतीतेः~। नाप्यभावे, सत्त्वेन प्रतीयमानत्वादिति चेत् , न~। मण्याद्यभावविशिष्टवह्न्यादेर्दाहादिकं प्रति स्वातन्त्र्येण
मण्यभावादेरेव वा हेतुत्वं कल्प्यते, अनेनैव सामञ्जस्येऽनन्तशक्ति-तत्प्रागभाव-ध्वंस कल्पनानौचित्यात्~। न चोत्तेजके सति प्रतिबन्धकसद्भावेऽपि कथं दाह इति वाच्यम्~।
उत्तेजकाभावविशिष्टमण्यभावस्य हेतुत्वात्~। सादृश्यमपि न पदार्थान्तरं, किन्तुं तद्भिन्नत्वे सति तद्गतभूयोधर्मवत्त्वम्~।
यथा चन्द्रभिन्नत्वे सति चन्द्रगताह्लादकत्वादिमत्त्वं मुखे चन्द्रसादृश्यमिति~॥१॥
॥इति पदार्थोद्देशग्रन्थः॥
क्षित्यप्तेजोमरुव्द्योमकालदिग्देहिनो मनः~।
द्रव्याणि-
द्रव्याणि विभजते - क्षित्यबिति~। क्षितिः - पृथिवी, आपः - जलानि, तेजः - वह्निः, वायुः - मरुत् , व्योम - आकाशः, कालः - समयः, दिक् -आशा, देही - आत्मा, मनः, एतानि
नव द्रव्याणीत्यर्थः~।
ननु द्रव्यत्वजातौ किं मानं~? न हि तत्र प्रत्यक्षं प्रमाणं, घृतजतुप्रभृतिषु द्रव्यत्वाग्रहादिति चेत् ,न~। कार्यसमवायिकारणतावच्छेदकतया, संयोगस्य, विभागस्य वा
समवायिकारणतावच्छेदकतया द्रव्यत्वजातिसिद्धेरिति~।
ननु दशमं द्रव्यं तमः कुतो नोक्तं~? तद्धि प्रत्यक्षेण गृह्यते~। तस्य च रूपवत्वात् कर्मवत्त्वाच्च द्रव्यत्वं~। तद्धि गन्धशून्यत्वात् न पृथिवी~। नीलरूपवत्वाच्च न
जलादिकम्~। तत्प्रत्यक्षे चाऽऽलोकनिरपेक्षं चक्षुः कारणमिति चेत् ,
न~। आवश्यकतेजोऽभावेनैवोपपत्तौ द्रव्यान्तरकल्पनाया अन्याय्यत्वात्~। रूपवत्ताप्रतीतिस्तु भ्रमरूपा~। कर्मवत्ताप्रतीतिरप्यालोकापसरणौपाधिकी भ्रान्तिरेव~।
तमसोऽतिरिक्तद्रव्यत्वे अनन्तावयवादिकल्पनागौरवं च स्यात्~। सुवर्णस्य यथा तेजस्यन्तर्भावस्तथाऽग्रे वक्ष्यते ॥२॥
॥इति द्रव्यविभागग्रन्थः॥
अथ गुणा रूपं रसो गन्धस्ततः परम् ॥३॥
गुणान् विभजते - अथ गुणा इति~। एते गुणाश्चतुर्विंशतिसङ्ख्याकाः कणादेन कण्ठतः, "च"शब्देन च दर्शिताः~। तत्र गुणत्वजातिसिद्धिरग्रे वक्ष्यते~।
॥इति गुणविभागग्रन्थः॥
स्पर्शः सङ्ख्या परिमितिः पृथक्त्वं च ततः परम्~।
संयोगश्च विभागश्च परत्वं चाऽपरत्वकम् ॥४॥
बुद्धिः सुखं दुःखमिच्छा द्वेषो यत्नो गुरुत्वकम्~।
द्रवत्वं स्नेह-संस्कारावदृष्टं शब्द एव च ॥५॥
उत्क्षेपणं ततोऽपक्षेपणमाकुञ्चनं तथा~।
प्रसारणं च गमनं कर्माण्येतानि पञ्च च ॥६॥
कर्माणि विभजते - उत्क्षेपणमिति~। कर्मत्वजातिस्तु प्रत्यक्षसिद्धा~। एवमुत्क्षेपणत्वादिकमपि~।
नन्वत्र भ्रमणादिकमपि पृथक्कर्म अधिकतया कुतो नोक्तमत आह -
भ्रमणं रेचनं स्यन्दनोध्र्वज्वलनमेव च~।
तिर्यग्गमनमप्यत्र गमनादेव लभ्यते ॥७॥
॥इति कर्मपदार्थविभागग्रन्थः॥
सामान्यं द्विविधं प्रोक्तं परं चाऽपरमेव च~।
द्रव्यादित्रिकवृत्तिस्तु सत्ता परतयोच्यते ॥८॥
सामान्यं निरूपयति - सामान्यमिति~। तल्लक्षणं तु नित्यत्वे सत्यनेकसमवेतत्वम्~। अनेकसमवेतत्वं संयोगादीनामप्यस्तीत्यत उक्तं - नित्यत्वे सतीति~। नित्यत्वे सति
समवेतत्वं गगनपरिमाणादीनामप्यस्तीत्यत उक्तमनेकेति~। नित्यत्वे सत्यनेकवृत्तित्वमत्यन्ताभावस्याऽप्यस्ति, अतो वृत्तित्वसामान्यं विहाय समवेतेत्युक्तम्~।
एकव्यक्तिमात्रवृत्तिस्तु न जातिः~। तथा चोक्तम् -
"व्यक्तेरभेदस्तुल्यत्वं सङ्करोऽथानवस्थितिः~।
रूपहानिरसम्बन्धो जातिबाधकसङ्ग्रहः ॥" इति~।
परभिन्ना तु या जातिः सैवाऽपरतयोच्यते~।
द्रव्यत्वादिकजातिस्तु परापरतयोच्यते ॥९॥
परत्वं - अधिकदेशवृत्तित्वं, अपरत्वं - अल्पदेशवृत्तित्वम्~। सकलजात्यपेक्षया सत्ताया अधिकदेशवृत्तित्वात् परत्वम् , तदपेक्षया चाऽन्यासां जातीनामपरत्वम्~। पृथिवीत्वाद्यपेक्षया
द्रव्यत्वस्याधिकदेशवृत्तित्वाद्व्यापकत्वात् परत्वं, सत्तापेक्षयाऽल्पदेशवृत्तित्वाद्व्याप्यत्वादपरत्वम् ॥७-८-६॥
॥इति सामान्यनिरूपणम्॥
व्यापकत्वात् पराऽपि स्याद् व्याप्यत्वादपराऽपि च~।
अन्त्यो नित्यद्रव्यवृत्तिर्विशेषः परिकीर्तितः ॥१०॥
विशेषं निरूपयति - अन्त्य इति~। अन्तेऽवसाने वर्तत इत्यन्त्यः, यदपेक्षया विशेषो नाऽस्तीत्यर्थः~। घटादीनां द्व्यणुकपर्यन्तानां तत्तदवयवभेदात् परस्परं भेदः,
परमाणूनां परस्परं भेदसाधको विशेष एव, स तु स्वत एव व्यावृत्तः~। तेन तत्र विशेषान्तरापेक्षा नाऽस्तीत्यर्थः ॥१०॥
॥इति विशेषनिरूपणम्॥
घटादीनां कपालादौ द्रव्येषु गुणकर्मणोः~।
तेषु जातेश्च सम्बन्धः समवायः प्रकीर्तितः ॥११॥
समवायं दर्शयति - घटादीनामिति~। अवयवावयविनोः, जातिव्यक्त्योः, गुणगुणिनोः, क्रियाक्रियावतोः, नित्यद्रव्यविशेषयोश्च यः सम्बन्धः स समवायः~।
समवायत्वं नित्यसम्बन्धत्वम्~। तत्र प्रमाणं तु, गुणक्रियादिविशिष्टबुद्धिर्विशेषणविशेष्यसम्बन्धविषया, विशिष्टबुद्धित्वात् , दण्डी पुरुष इति विशिष्टबुद्धिवदित्यनुमानेन
संयोगादिबाधात् समवायसिद्धिः~। न च स्वरूपसम्बन्धेन सिद्धसाधनं, अर्थान्तरं वा~। अनन्तस्वरूपाणां सम्बन्धत्वकल्पने गौरवात् , लाघवादेकसमवायसिद्धिः~।
न च समवायस्यैकत्वे वायौ रूपवत्ताबुद्धिप्रसङ्गः, तत्र रूपसमवायसत्त्वेऽपि रूपाभावात्~।
न चैवमभावस्याऽपि वैशिष्ट्यं सम्बन्धान्तरं सिध्येदिति वाच्यम्~। तस्य नित्यत्वे भूतले, घटानयनानन्तरमपि घटाभावबुद्धिप्रसङ्गात् , घटाभावस्य तत्र सत्त्वात् ,
तस्य च नित्यत्वात् , अन्यथा देशान्तरेऽपि घटाभावप्रतीतिर्न स्यात्~। वैशिष्ट्यस्य च तत्र सत्त्वात् , मम तु घटे पाकरक्ततादशायां श्यामरूपस्य नष्टत्वान्न तद्वत्ताबुद्धिः~।
वैशिष्ट्यस्याऽनित्यत्वे त्वनन्तवैशिष्ट्यकल्पने तवैव गौरवम्~। एवं च तत्तत्कालीनं तत्तद्भूतलादिकमेव तत्तदभावानां सम्बन्धः ॥११॥
॥इति समवायनिरूपणम्॥
अभावस्तु द्विधा संसर्गान्योन्याभावभेदतः~।
अभावं विभजते - अभावस्त्विति~। अभावत्वं - द्रव्यादिषट्कान्योऽन्याभाववत्त्वम्~। संसर्गेति~। संसर्गाभावान्योन्याभावभेदादित्यर्थः~। अन्योन्याभावस्यैकविधत्वात् तद्विभागाभावात्
संसर्गाभावं विभजते-प्रागभाव इति~। संसर्गाभावत्वमन्योन्याभावभिन्नाभावत्वम्~। अन्योन्याभावत्वं तादात्म्यसम्बन्धावच्छिन्नप्रतियोगिताकाभावत्वम्~।
विनाश्यभावत्वम् प्रागभावत्वम्~। जन्याभावत्वं ध्वंसत्वम्~। नित्यसंसर्गाभावत्वमत्यन्ताभावत्वम्~। यत्र तु भूतलादौ घटादिकमपसारितं, पुनरानीतं च, तत्र
घटकालस्यसम्बन्धाघटकत्वादत्यन्ताभावस्य नित्यत्वेऽपि घटकाले न घटात्यन्ताभावबुद्धिः~। तत्रोत्पादविनाशशाली चतुर्थोऽयमभाव इति केचित्~।
अत्र ध्वंसप्रागभावयोरधिकरणे नाऽत्यन्ताभाव इति प्राचाम् मतम् , श्यामघटे रक्तो नास्ति, रक्तघटे श्यामो नाऽस्तीति धीश्च प्रागभावं ध्वंसं चाऽवगाहते, न तु
तदत्यन्ताभावं, तयोर्विरोधात् नव्यास्तु - तत्र विरोधे मानाभावात् ध्वंसादिकालावच्छेदेनाऽप्यत्यन्ताभावो वर्तत इत्याहुः~।
प्रागभावस्तथा ध्वंसोऽप्यत्यन्ताभाव एव च ॥१२॥
एवं त्रैविध्यमापन्नः संसर्गाभाव इष्यते~।
सप्तानामपि साधर्म्यं ज्ञेयत्वादिकमुच्यते ॥१३॥
नन्वस्त्वभावानामधिकरणात्मकत्वं, लाघवादिति चेन्न~। अनन्ताधिकरणात्मकत्वकल्पनापेक्षयाऽतिरिक्तत्वकल्पनाया एव लघीयस्त्वात्~। एवं चाऽऽधाराधेयभावोऽप्युपपद्यते~।
एवं च तत्तच्छब्दरसगन्धाद्यभावानां प्रत्यक्षत्वमप्युपपद्यते~। अन्यथा तत्तदधिकरणानां तत्तदिन्द्रियाग्राह्यत्वात् प्रत्यक्षत्वं न स्यात्~।
एतेन ज्ञानविशेष - कालविशेषाद्यात्मकत्वमत्यन्ताभावस्येति प्रत्युक्तम् , अप्रत्यक्षत्वापत्तेः।
॥इत्यभावपदार्थविभागग्रन्थः॥
इदानीं पदार्थानां साधर्म्यं वैधर्म्यं च वक्तुं प्रक्रमते - सप्तानामिति~। समानो धर्मो येषां ते सधर्माणः, तेषां भावः साधर्म्यं, समानो धर्म इति फलितोऽर्थः~। एवं विरुद्धो
धर्मो येषां ते विधर्माणः, तेषां भावो वैधर्म्यं, विरुद्धो धर्म इति फलितोऽर्थः~। ज्ञेयत्वं ज्ञानविषयता, सा च सर्वत्रैवाऽस्ति, ईश्वरज्ञानविषयतायाः केवलान्वयित्वात्~।
द्रव्यादयः पञ्च भावा अनेके समवायिनः~।
एवमभिधेयत्व - प्रमेयत्वादिकं बोध्यम् ॥१३॥
॥इति सप्तपदार्थसाधर्म्यकथनम्॥
द्रव्यादय इति~। द्रव्यगुणकर्मसामान्यविशेषाणां साधर्म्यमनेकत्वं समवायित्वं च~। यद्यप्यनेकत्वमभावेऽप्यस्ति, तथाऽप्यनेकत्वे सति भावत्वं पञ्चानां साधर्म्यम् ,
तथा चाऽनेकभाववृत्तिपदार्थविभाजकोपाधिमत्त्वमिति फलितोऽर्थः, तेन प्रत्येकं घटादावाकाशादौ च नाऽव्याप्तिः~।
समवायित्वं च समवायसम्बन्धेन सम्बन्धित्वं, न तु समवायवत्त्वं, सामान्यादावभावात्~।
॥इति पञ्चपदार्थसाधर्म्यकथनम्॥
सत्तावन्तस्त्रयस्त्वाद्या गुणादिर्निर्गुणक्रियः ॥१४॥
सत्तावन्त इति~। द्रव्यगुणकर्मणां सत्तावत्त्वमित्यर्थः ॥
॥इत्याद्यपदार्थत्रयसाधर्म्यकथनम्॥
गुणादिरिति~। यद्यपि गुणक्रियाशून्यत्वमाद्यक्षणे घटादावतिव्याप्तं, क्रियाशून्यत्वञ्च गगनादावतिव्याप्तम् , तथाऽपि गुणवदवृत्तिधर्मवत्त्वं कर्मवदवृत्ति
पदार्थविभाजकोपाधिमत्त्वञ्च च तदर्थः~। नहि घटत्वादिकं द्रव्यत्वादिकं वा गुणवदवृत्ति, कर्मवदवृत्ति वा, किन्तु गुणत्वादिकं तथा, आकाशत्वादिकन्तु न पदार्थविभाजकोपाधिः ॥१४॥
॥इति गुणादिषट्‍पदार्थसाधर्म्यकथनम्॥
सामान्यपरिहीनास्तु सर्वे जात्यादयो मताः~।
पारिमाण्डल्यभिन्नानां कारणत्वमुदाहृतम् ॥१५॥
सामान्येति~। सामान्यानधिकरणत्वं सामान्यादीनामित्यर्थः ॥
॥इति सामान्यादिपदार्थचतुष्टयसाधर्म्यकथनम्॥
पारिमाण्डल्येति~। परिमाण्डल्यमणुपरिमाणम् , कारणत्वं तद्भिन्नानामित्यर्थः~। अणुपरिमाणं तु न कस्याऽपि कारणम्~। तद्धि स्वाश्रयारब्धद्रव्यपरिमाणारम्भकं भवेत्~।
तच्च न सम्भवति~। परिमाणस्य स्वसमानजातीयस्वोत्कृष्ट-परिमाणजनकत्वनियमात् , महदारब्धस्य महत्तरत्ववदणुजन्यस्याऽणुतरत्वप्रसङ्गात्~। एवं परममहत्परिमाणमतीन्द्रियसामान्यं
विशेषाश्च बोध्याः~।
इदमपि योगिप्रत्यक्षे विषयस्य न कारणत्वं, ज्ञायमानं सामान्यं न प्रत्यासत्तिः ज्ञायमानं लिङ्गं नाऽनुमितिकरणमित्यभिप्रायेणोक्तम्~। आत्ममानसप्रत्यक्षे
आत्मपरममहत्त्वस्य कारणत्वात् परममहत्परिमाणमाकाशादेर्बोध्यम्~। तस्याऽपि न कारणत्वमित्याचार्याणामाशय इत्यन्ये~। तन्न~। ज्ञानातिरिक्तं प्रत्येवाऽकारणताया
आचार्यैरुक्तत्वात् ॥१५॥
॥इति पारिमाण्डल्यभिन्नपदार्थसाधर्म्यकथनम्॥
ननु कारणत्वं किमत आह अन्यथासिद्धीति~। तस्य कारणत्वस्य ॥१६-१७॥
॥इति प्रासङ्गिककारणत्वनिरूपणम्॥
अन्यथासिद्धिशून्यस्य नियता पूर्ववर्तिता~।
कारणत्वं भवेत्तस्य त्रैविध्यं परिकीर्तितम्॥१६॥
समवायिकारणत्वं ज्ञेयमथाऽप्यसमवायिहेतुत्वम्~।
एवं न्यायनयज्ञैस्तृतीयमुक्तं निमित्तहेतुत्वम्॥१७॥
यत्समवेतं कार्यं भवति ज्ञेयं तदत्र समवायि~।
तत्राऽऽसन्नं जनकं द्वितीयमाभ्यां परं तृतीयं स्यात् ॥१८॥
तत्रेति~। समवायिकारणे प्रत्यासन्नं कारणं द्वितीयं असमवायिकारणमित्यर्थः~।अत्र यद्यपि तुरीतन्तुसंयोगे पटासमवायिकारणत्वं स्यात् , वेगादीनामभिघाताद्यसमवायिकारणत्वं स्यात् ,
एवं ज्ञानादीनामपीच्छाद्यसमवायिकारणत्वं स्यात् , तथाऽपि पटासमवायिकारणलक्षणे तुरीतन्तुसंयोगभिन्नत्वं देयम्~। तुरीतन्तुसंयोगस्तु तुरीपटसंयोगं
प्रत्यसमवायिकारणं भवत्येव~। एवं वेगादिकमपि वेगस्पन्दाद्यसमवायिकारणम् भवत्येवेति तत्तत्कार्यासमवायिकारणलक्षणे तत्तद्भिन्नत्वं देयम्~। आत्मविशेषगुणानां तु
कुत्राऽप्यसमवायिकारणत्वं नाऽस्ति~। तेन तद्भिन्नत्वं सामान्यलक्षणे देयमेव~। अत्र समवायिकारणे प्रत्यासन्नं द्विविधं, कार्यैकार्थप्रत्यासत्त्या, कारणैकार्थप्रत्यासत्त्या च~।
आद्यं यथा - घटादिकं प्रति कपालसंयोगादिकमसमवायिकारणम्~। तत्र कार्येण घटेन सह कारणस्य कपालसंयोगस्य एकस्मिन् कपाले प्रत्यासत्तिरस्ति~।
द्वितीयं यथा - घटरूपं प्रति कपालरूपमसमवायिकारणम्~। स्वगतरूपादिकं प्रति समवायिकारणं घटः तेन सह कपालरूपस्यैकस्मिन् कपाले प्रत्यासत्तिरस्ति~। तथा च क्वचित् समवायसम्बन्धेन,
क्वचित् स्वसमवायिसमवेतत्वसम्बन्धेनेति फलितोऽर्थः, इत्यं च कार्यैकार्थ-कारणैकार्थान्यतरप्रत्यासत्या समवायिकारणे प्रत्यासन्नं कारणं ज्ञानादिभिन्नमसमवायिकारणमिति
सामान्यलक्षणं पर्यवसितम्~।
॥इत्यसमवायिकारणत्वनिरूपणम्॥
आभ्यां--समवायिकारणासमवायिकारणाभ्यां, परं--भिन्नं कारणं, तृतीयं--
निमित्तकारणमित्यर्थः॥१८॥
॥इति निमित्तकारणत्वनिरूपणम्॥
येन सह पूर्वभावः, कारणमादाय वा यस्य~।
इदानीमन्यथासिद्धत्वमेव कियतामत आह - येनेति~। यत्कार्यं प्रति कारणस्य पूर्ववृत्तिता येन रूपेण, गृह्यते, तत्कार्यं प्रति तद्रूपमन्यथासिद्धमित्यर्थः~। यथा - घटं प्रति दण्डत्वमिति~।
॥इति प्रथमान्यथासिद्धनिरूपणम्॥
द्वितीयमन्यथासिद्धमाह--कारणमिति~। यस्य स्वातन्त्र्येणान्वयव्यतिरेकौ न स्तः, किन्तु कारणमादायैवान्वयव्यतिरेकौ गृह्येते, तदन्यथासिद्धम्~। यथा-दण्डरूपम्~।
॥इति द्वितीयान्यथासिद्धनिरूपणम्॥
अन्यं प्रति पूर्वभावे ज्ञाते यत्पूर्वभावविज्ञानम्॥१९॥
तृतीयमाह - अन्यं प्रतीति~। अन्यं प्रति पूर्ववृत्तित्वं गृहीत्वैव यस्य यत्कार्यं प्रति पूर्ववृत्तित्वं गृह्यते, तस्य तत्कार्यं प्रत्यन्यथासिद्धत्वम्~। यथा - घटादिकं प्रत्यकाशस्य~।
तस्य हि~। घटादिकं प्रति कारणत्वमाकाशत्वेनैव स्यात् , तद्धि शब्दसमवायिकारणत्वम् , एवं च तस्य शब्दं प्रति कारणत्वं गृहीत्वैव घटादिकं प्रति जनकत्वं ग्राह्यमतस्तदन्यथासिद्धम्~।
ननु शब्दाश्रयत्वेन तस्य कारणत्वे काऽन्यथासिद्धिरिति चेत् , पञ्चमीति गृहाण~। नन्वाकाशस्य शब्दं प्रति जनकत्वे किमवच्छेदकमिति चेत् , कवत्त्वादिकं,
विशेषपदार्थो वेति ॥१६॥
॥इति तृतीयान्यथासिद्धनिरूपणम्॥
जनकं प्रति पूर्ववृत्तितामपरिज्ञाय न यस्य गृह्यते~।
अतिरिक्तमथाऽपि यद्भवेन्नियतावश्यकपूर्वभाविनः॥२०॥
चतुर्थमन्यथासिद्धमाह - जनकं प्रतीति~। यत्कार्यजनकं प्रति पूर्ववृत्तित्वं गृहीत्वैव यस्य यत्कार्यं प्रति पूर्ववर्तित्वं गृह्यते, तस्य तत्कार्यं प्रत्यन्यथासिद्धत्वम्~। यथा
कुलालपितुर्घटं प्रति~। तस्य हि कुलालपितृत्वेन घटं प्रति जनकत्वेऽन्यथासिद्धिः, कुलालत्वेन रूपेण जनकत्वे त्विष्टापत्तिः, कुलालमात्रस्य घटं प्रति जनकत्वात्~।
॥इति चतुर्थान्यथासिद्धनिरूपणम्॥
पञ्चमान्यथासिद्धमाह - अतिरिक्तमिति~। अवश्यक्लृप्तनियतपूर्ववर्तिन एव कार्यसम्भवे तद्भिन्नमन्यथासिद्धमित्यर्थः~। अत एव प्रत्यक्षे महत्त्वं कारणम् ,
अनेकद्रव्यवत्त्वमन्यथासिद्धम् , तत्र हि महत्त्वमवश्यं क्लृप्तम् , तेनानेकद्रव्यवत्त्वमन्यथासिद्धम्~। नच वैपरीत्ये किं विनिगमकमिति वाच्यम्~। महत्त्वत्वजातेः कारणतावच्छेकत्वे
लाघवात्॥२०-२१॥
एते पञ्चाऽन्यथासिद्धा दण्डत्वादिकमादिमम्~।
घटादौ दण्डरूपादि द्वितीयमपि दर्शितम्॥२१॥
तृतीयं तु भवेद्व्योम कुलालजनकोऽपरः~।
पञ्चमो रासभादिः स्यादेतेष्वावश्यकस्त्वसौ॥२२॥
रासभादिरिति~। यद्यपि यत्किञ्चिद्घटव्यक्तिं प्रति रासभस्य नियतपूर्ववृत्तित्वमस्ति~। तथाऽपि घटजातीयं प्रति सिद्धकारणभावैर्दण्डादिभिरेव तद्व्यक्तेरपि सम्भवे
रासभोऽन्यथासिद्ध इति भावः~। एतेषु - पञ्चस्वन्यथासिद्धेषु मध्ये, पञ्चमोऽन्यथासिद्ध आवश्यकः, तेनैव परेषां चरितार्थत्वात्~। तथा हि - दण्डादिभिरवश्यकलृप्तनियतपूर्ववर्तिभिरेव
कार्यसम्भवे दण्डत्वादिकमन्यथासिद्धम्~। न च वैपरीत्ये किं विनिगमकमिति वाच्यम्~। दण्डत्वस्य कारणत्वे दण्डघटितायाः परम्परायाः सम्बन्धत्वकल्पने गौरवात्~।
एवमन्येषामप्यनेनैव चरितार्थत्वं सम्भवति॥२२॥
॥इति पञ्चविधान्यासिद्धनिरूपणम्॥
समवायिकारणत्वं द्रव्यस्यैवेति विज्ञेयम्~।
गुणकर्ममात्रवृत्ति ज्ञेयमथाप्यसमवायिहेतुत्वम्॥२३॥
समवायीति~। स्पष्टम्~।
गुणेति~। असमवायिकारणत्वं गुणकर्मभिन्नानां वैधर्म्यं, न तु गुणकर्मणोः साधर्म्यमित्यत्र तात्पर्यम्~।
अथ वासमवायिकारणवृत्तिसत्ताभिन्नजातिमत्त्वं तदर्थः~। तेन ज्ञानादीनामसमवायिकारणत्वविरहेऽपि नाऽव्याप्तिः॥२३॥
अन्यत्र नित्यद्रव्येभ्य आश्रितत्वमिहोच्यते~।
॥इति समवाय्यसमवायिकारणत्वरूपसाधर्म्यद्वयकथनम्॥
अन्यत्रेति~। नित्यद्रव्याणि - परमाण्वाकाशादीनि विहायाऽऽश्रितत्वं साधर्म्यमित्यर्थः~। आश्रितत्वं तु समवायादिसम्बन्धेन वृत्तिमत्वं, विशेषणतया नित्यानामपि
कालादौ वृत्तेः~।
॥इति नित्यद्रव्यातिरिक्तसाधर्म्यकथनम्॥
क्षित्यादीनां नवानां तु द्रव्यत्वं गुणयोगिता॥१४॥
इदानीं द्रव्यस्यैव विशिष्य साधर्म्यं वक्तुमारभते - क्षित्यादीनामिति~। स्पष्टम्॥२४॥
॥इति द्रव्यमात्रसाधर्म्यकथनम्॥
क्षितिर्जलं तथा तेजः पवनो मन एव च~।
क्षितिरिति~। पृथिव्यप्तेजोवायुमनसां परत्वापरत्ववत्त्वं मूर्तत्वं, वेगवत्त्वं, कर्मवत्त्वं च साधर्म्यम्~। न च यत्र घटादौ परत्वमपरत्वं वा नोत्पन्नं, तत्राऽव्याप्तिरिति
वाच्यम्~। परत्वादिसमानाधिकरण-द्रव्यत्वव्याप्य-जातिमत्त्वस्य विवक्षितत्वात्~। मूर्तत्वम्-अपकृष्टपरिमाणवत्त्वम् , तच्चैषामेव, गगनादिपरिमाणस्य कुतोऽप्यपकृष्टत्वाभावात्~।
पूर्ववत् कर्मवत्त्वं कर्मसमानाधिकरणद्रव्यत्वव्याप्यजातिमत्त्वं, वेगवत्त्वं वेगवद्वृत्तिद्रव्यत्वव्याप्यजातिमत्त्वं च बोध्यम् ॥२५॥
॥इति क्षित्यप्तेजोवायुमनसां साधर्म्यकथनम्॥
परापरत्वमूर्तत्व क्रियावेगाश्रया अमी॥२५॥
कालखात्मदिशां सर्वगतत्वं परमं महत्~।
कालेति~। कालाकाशात्मदिशां सर्वगतत्वम्~। सर्वगतत्वं सर्वमूर्तसंयोगित्वम् , परममहत्त्वं च~। परममहत्त्वत्वं जातिविशेषः, अपकर्षानाश्रयपरिमाणवत्त्वं वा~।
॥इति कालाकाशात्मदिशां साधर्म्यकथनम्~।
क्षित्यादि पञ्च भूतानि चत्वारि स्पर्शवन्ति हि॥१६॥
क्षित्यादीति~। पृथिव्यप्तेजोवाय्वाकाशानां भूतत्वम्~। तच्च बहिरिन्द्रियग्राह्यविशेषगुणवत्त्वम्~। अत्र ग्राह्यत्वं लौकिकप्रत्यक्षस्वरूपयोग्यत्वं बोध्यम्~। तेन ज्ञातो घट
इति प्रत्यक्षे ज्ञानस्याप्युपनीतभानविषयत्वात्तद्वत्यात्मनि नाऽतिप्रसङ्गः~। न वा लौकिकप्रत्यक्षाविषयरूपादिमति परमाण्वादावव्याप्तिः, तस्याऽपि स्वरूपयोग्यत्वात्~।
महत्त्वलक्षणकारणान्तरासन्निधानाच्च न प्रत्यक्षत्वम्~। अथ वाऽऽत्मावृत्तिविशेषगुणवत्त्वम् तत्त्वम्॥
॥इति क्षित्यप्तेजोवाय्वाकाशसाधर्म्यकथनम्॥
चत्वारीति~। पृथिव्यप्तेजोवायूनां स्पर्शवत्त्वं॥
द्रव्यारम्भ इति~। पृथिव्यप्तेजोवायुषु चतुर्षु द्रव्यारम्भकत्वम्~। न च द्रव्यानारम्भके घटादावव्याप्तिः~। द्रव्यसमवायिकारणवृत्ति-द्रव्यत्वव्याप्य-जातिमत्त्वस्य विवक्षितत्वात्~।
॥इति क्षित्यप्तेजोवायूनां साधर्म्यकथनम्॥
अथाकाशेति~। आकाशात्मनामव्याप्यवृत्तिक्षणिकविशेषगुणवत्त्वं साधर्म्यमित्यर्थः~। आकाशस्य विशेषगुणः शब्दः~। स चाव्याप्यवृत्तिः, यदा किञ्चिदवच्छेदेन
शब्द उत्पद्यते, तदान्यावच्छेदेन तदभावस्यापि सत्त्वात्~। क्षणिकत्वं च तृतीयक्षणवृत्तिध्वंसप्रतियोगित्वम्~। योग्यविभुविशेषगुणानां स्वोत्तरवृत्तिविशेषगुणनाश्यत्वात्
प्रथमशब्दस्य द्वितीयशब्देन नाशः~। एवं ज्ञानादीनामपि~। ज्ञानादिकं यदाऽऽत्मनि विभौ शरीरावच्छेदेनोत्पद्यते, तदा घटाद्यवच्छेदेन तदभावोऽस्त्येव~। एवं ज्ञानादिकमपि
क्षणद्वयावस्थाथि~। इत्थं चाऽव्याप्यवृत्तिविशेषगुणवत्वं, क्षणिकविशेषगुणवत्त्वं चाऽर्थः~। पृथिव्यादौ रूपादिर्विशेषगुणोऽस्तीत्यतोऽव्याप्यवृत्तीत्युक्तम्~। पृथिव्यादावव्याप्यवृत्तिः
संयोगादिरस्तीति विशेषगुणेत्युक्तम्~। न च रूपादीनामपि कदाचित् तृतीयक्षणे नाशसम्भवात् क्षणिकविशेषगुणवत्त्वं क्षित्यादावतिव्याप्तमिति वाच्यम्~।
चतुःक्षणवृत्तिजन्यावृत्तिजातिमद्विशेषगुणवत्त्वस्य तदर्थत्वात्~। अपेक्षाबुद्धिः क्षणत्रयं तिष्ठति, रूपत्वादिकं तु क्षणचतुष्टयस्थायिन्यपि रूपादौ वर्तते इति तद्व्युदासः~।
ईश्वरज्ञानस्य चतुःक्षणवृत्तित्वात् ज्ञानत्वस्य तद्वृत्तित्वाज्जन्येत्युक्तम्~। यद्याकाशजीवात्मनोः साधर्म्यं, तदा जन्येति न देयं, द्वेषत्वादिकमादायाऽऽत्मनि लक्षणसमन्वयात्~।
परममहत्त्वस्य तादृशगुणत्वाद् , चतुर्थक्षणे द्वित्वादीनां नाशाभ्युपगमाद् द्वित्वादीनामपि तथात्वात् तद्वारणाय - विशेषेति~। त्रिक्षणवृत्तित्वं वा वक्तव्यम्~। इच्छात्वादिकमादायाऽऽत्मनि
लक्षणसमन्वयात्॥२६॥२७॥
॥इत्याकाशात्मनोः साधर्म्यकथनम्॥
द्रव्यारम्भश्चतुर्षु स्यादथाऽऽकाशशरीरिणाम्~।
अव्याप्यवृत्तिक्षणिको विशेषगुण इष्यते॥२७॥
रूप-द्रवत्व-प्रत्यक्ष-योगिनः प्रथमास्त्रयः~।
गुरुणी द्वे रसवती द्वयोर्नैमित्तिको द्रवः॥२८॥
रूपद्रवत्वेति~। पृथिव्यप्तेजसां रूपवत्त्वं, द्रवत्ववत्त्वं, प्रत्यक्षविषयत्वं चेत्यर्थः~। न च चक्षुरादीनां भर्जनकपालस्थवह्नेः ऊष्मणश्च रूपवत्त्वे किं मानमिति वाच्यम्~।
तत्रापि तेजस्त्वेन रूपानुमानात्~। एवं वाय्वानीतपृथिवीजलतेजोभागानामपि पृथिवीत्वादिना रूपानुमानं बोध्यम्~।
न च घटादौ द्रुतसुवर्णादिभिन्ने तेजसि च द्रवत्ववत्त्वमव्याप्तमिति वाच्यम्~। द्रवत्ववद्वृत्ति --द्रव्यत्वव्याप्य - जातिमत्त्वस्य विवक्षितत्वात्~। घृतजतुप्रभृतिषु पृथिवीषु
जलेषु द्रुतसुवर्णादौ तेजसि च द्रवत्वसत्त्वात्तत्र च पृथिवीत्वादिसत्त्वात्तदादाय सर्वत्र लक्षणसमन्वयः~। न च प्रत्यक्षविषयत्वं परमाण्वादावव्याप्तम्  , अतिव्याप्तं च
रूपादाविति वाच्यम्~। चाक्षुषलौकिकप्रत्यक्षविषयवृत्ति-द्रव्यत्वव्याप्यजातिमत्त्वस्य विवक्षितत्वात्~। आत्मन्यतिव्याप्तिवारणाय चाक्षुषेति~।
इति पृथिव्यप्तेजसां साधर्म्यकथनम्~।
गुरुणी इति~। गुरुत्ववत्त्वं रसवत्त्वं च पृथिवीजलयोरित्यर्थः~। न च घ्राणेन्द्रियादीनां वाय्वानीतपृथिव्यादिभागानां च रसादिमत्त्वे किं मानमिति वाच्यम्~। तत्रापि
पृथिवीत्वादिना तदनुमानात्~।
इति पृथिवीजलयोः साधर्म्यकथनम्~।
द्वयोरिति-पृथिवी तेजसोरित्यर्थः~। न च नैमित्तिकं द्रवत्वं घटादौ वह्न्यादौ चाऽव्याप्तमिति वाच्यम्~। नैमित्तिकद्रवत्वसमानाधिकरण-द्रव्यत्वव्याप्यजातिमत्त्वस्य
विवक्षितत्वात् ।
॥इति पृथिवीतेजसोः साधर्म्यकथनम्॥
आत्मान इति~। पृथिव्यप्तेजोवाय्वाकाशात्मनां विशेषगुणवत्त्वमित्यर्थः~।
इति भूतात्मनोः साधर्म्यकथनम्॥
यदुक्तमिति~। ज्ञेयत्वादिकं विहायेति बोध्यम्~। तत्तु न कस्याऽपि वैधर्म्यं, केवलान्वयित्वात्॥२६॥
॥इति वैधर्म्यनिरूपणम्॥
आत्मानो भूतवर्गाश्च विशेषगुणयोगिनः~।
यदुक्तं यस्य साधर्म्यं वैधर्म्यमितरस्य तत्॥२९॥
स्पर्शादयोऽष्टौ वेगाख्यः संस्कारो मरुतो गुणाः~।
स्पर्शाद्यष्टौ रूपवेगौ द्रवत्वं तेजसो गुणाः॥३०॥
स्पर्शादयोऽष्टौ वेगश्च गुरुत्वं च द्रवत्वकम्~।
रूपं रसस्तथा स्नेहो वारिण्येते चतुर्दश॥३१॥
स्नेहहीना गन्धयुताः क्षितावेते चतुर्दश~।
बुद्ध्यादिषट्कं सङ्ख्यादिपञ्चकं भावना तथा॥३२॥
धर्माधर्मौ गुणा एते आत्मनः स्युश्चतुर्दश~।
सङ्ख्यादिपञ्चकं कालदिशोः शब्दश्च ते च खे॥३३॥
सङ्ख्यादिपञ्चकं बुद्धिरिच्छा यत्नोऽपि चेश्वरे~।
परापरत्वे सङ्ख्याद्याः पञ्च वेगश्च मानसे॥३४॥
स्पर्शादय इति॥ते च - पञ्च सङ्ख्यादयः~। खे - आकाशे॥३०॥३१॥३२॥३३॥३४॥
॥इति सामान्यतो द्रव्यगुणकथनम्॥




तत्र क्षितिर्गन्धहेतुः नानारूपवती मता~।
साधर्म्यवैधर्म्ये निरूप्य सम्प्रति प्रत्येकं पृथिव्यादिकं निरूपयति-तत्रेति~। गन्धहेतुरिति~। गन्धसमवायिकारणमित्यर्थः~। यद्यपि गन्धवत्त्वमात्रं पृथिव्या लक्षणमुचितं,
तथाऽपि पृथिवीत्वजातौ प्रमाणोपन्यासाय कारणत्वमुपन्यस्तम्~। तथा हि पृथिवीत्वं हि गन्धसमवायिकारणतावच्छेदकतया सिद्धयति, अन्यथा गन्धत्वावच्छिन्नस्याऽऽकस्मिकत्वापत्तेः~।
न च पाषाणादौ गन्धाभावात्गन्धवत्त्वमव्याप्तमिति वाच्यम्~। तत्राऽपि गन्धसत्त्वात्, अनुपलब्धिस्त्वनुत्कटत्वेनाऽप्युपपद्यते~। कथमन्यथा तद्भस्मनि गन्ध
उपलभ्यते~? भस्मनो हि पाषाणध्वंसजन्यत्वात् पाषाणोपादानोपादेयत्वं सिद्ध्यति, यद्द्रव्यं यद्द्रव्यध्वंंसजन्यं, तत् तदुपादानोपादेयमिति व्याप्तेः~। दृष्टं चेतत् खण्डपटे
महापटध्वंसजन्ये~। इत्थंं च पाषाणपरमाणोः पृथिवोत्वात् तज्जन्यपाषाणस्याऽपि पृथिवीत्वं~। तथा च तस्याऽपि गन्धवत्त्वे बाधकाभावः~।
तत्र क्षितिर्गन्धहेतुः नानारूपवती मता~।
षड्विधस्तु रसस्तत्र गन्धस्तु द्विविधो मतः॥३५॥
स्पर्शस्तस्यास्तु विज्ञेयो ह्यनुष्णाशीतपाकजः~।
नानेति~। शुक्लनीलादिभेदेन नानाजातीयं रूपं पृथिव्यामेव वर्तते, न तु जलादौ, तत्र शुक्लस्यैव सत्त्वात्, पृथिव्यां त्वेकस्मिन्नपि धर्मिणि पाकवशेन
नानारूपसम्भवात्~। न च यत्र नानारूपं नोत्पन्नं, तत्राऽव्याप्तिरिति वाच्यम्~। रूपद्वयवद्वृत्तिद्रव्यत्वव्याप्यजातिमत्त्वस्य विवक्षितत्वात्, रूपनाशवद्वृत्ति-द्रव्यत्वव्याप्य-जातिमत्त्वस्य
वा वाच्यत्वात्~। वैशेषिकनये पृथिवीपरमाणौ रूपनाशस्य रूपान्तरस्य च सत्त्वात्, न्यायनये घटादावपि तत्सत्त्वाल्लक्षणसमन्वयः~।
षड्विध इति~। मधुरादिभेदेन यः षड्विधो रसः, स पृथिव्यामेव~। जले मधुर एव रसः~। अत्राऽपि पूर्ववद्रसद्वयवद्वृत्ति-द्रव्यत्वव्याप्य-जातिमत्त्वं लक्षणार्थोऽवसेयः~।
"द्विविधः" इति वस्तुस्थितिमात्रं, न तु द्विविधगन्धवत्त्वं लक्षणं, द्विविधत्वस्य व्यर्थत्वात्~। द्वैविध्यं च सौरभासौरभभेदेन बोध्यम्~।
स्पर्श इति~। तस्याः-पृथिव्याः~। अनुष्णाशीतस्पर्शवत्त्वं वायावपि वर्तत इत्युक्तं पाकज इति~। इत्थं च पृथिव्याः स्पर्शोऽनुष्णाशीत इति ज्ञापनार्थं तदुक्तं, पाकजस्पर्शवत्त्वमात्रं
तु लक्षणम्, अधिकस्य वैयथ्र्र्यात्~। यद्यपि पाकजस्पर्शः पटादौ नाऽस्ति, तथाऽपि पाकजस्पर्शवद्वृत्ति-द्रव्यत्वव्याप्यजातिमत्त्वमर्थो बोध्यः॥इति पृथिवीनिरूपणे पृथिवीलक्षणकथनम्॥
नित्याऽनित्या च सा द्वेधा नित्या स्यादणुलक्षणा॥३६॥
अनित्या तु तदन्या स्यात् सैवाऽवयवयोगिनी~।
नित्येति सा - पृथिवी द्विविधा, नित्याऽनित्या चेत्यर्थः~।
अणुलक्षणा-परमाणुरूपा पृथिवी नित्या॥३६॥
तदन्या परमाणुभिन्ना पृथिवी द्व्यणुकादि सर्वाप्यनित्येत्यर्थः~। सैव अनित्या पृथिव्येवावयववतीत्यर्थः~।
नन्ववयविनि किं मानं~? परमाणुपुञ्जैरेवोपपत्तेः~। न च परमाणूनामतीन्द्रियत्वाद्घटादीनां प्रत्यक्षं न स्यादिति वाच्यम्~। एकस्य परमाणोरप्रत्यक्षत्वेऽपि तत्समूहस्य
प्रत्यक्षत्वापरभवात्~। मभैकस्य केशस्य दूरेऽप्रत्यक्षत्वेऽपि तत्समूहस्य प्रत्यक्षत्वम्~। न चैको घटः स्थूल इति बुद्धेरनुपपत्तिरिति वाच्यम्~। एको महान् धान्यराशिरितिवदुपपत्तेः~।
मैवम्~। परमाणोरतीन्द्रियत्वेन तत्समूहस्याऽपि प्रत्यक्षवाऽयोगात्, दूरस्थकेशस्तु नाऽतीन्द्रियः, सन्निधाने तस्यैेव प्रत्यक्षत्वात्~। न च तदानीं दृश्यपरमाणुपुञ्जस्य
उत्पन्नत्वान्न प्रत्यक्षत्वेऽपि विरोध इति वाच्यम्~। अदृश्यस्य दृश्यानुपादानत्वात्, अन्यथा चक्षुरूष्मादिसन्ततेः कदाचिद्दृश्यत्वप्रसङ्गात्~। न चाऽतितप्ततैलादौ
कथमदृश्यदहनसन्ततेर्दृश्यदहनोत्पत्तिरिति वाच्यम्~। तत्र तदन्तःपातिभिर्दृश्यदहनावयवैः स्थूलदहनोत्पत्तेरुपगमात्~। न चाऽदृश्येन द्व्यणुकेन कथं दृश्यत्रसरेणोरुत्पत्तिरिति
वाच्यम्~। यतो न दृश्यत्वमदृश्यत्वं वा कस्यचित् स्वभावादाचक्षमहे, किन्तु महत्त्वोद्भूतरूपादिकारणसमुदायवतो दृश्यत्वं, तदभावे चाऽदृश्यत्वम्~। तथा च त्रसरणोर्महत्त्वात्
प्रत्यक्षत्वं, न तु द्व्यणुकादेः, तदभावात~। न हि त्वन्मतेऽपि सम्भवतीदं, परमाणौ महत्त्वाभावात्~।
इत्थं चाऽवयविसिद्धौ तेषामुत्पादविनाशयोः प्रत्यक्षसिद्धत्वादनित्यत्वम्~। इत्यवयव्यनुमानम्॥
तेषां चाऽवयवावयवधाराया अनन्तत्वे मेरुसर्षपयोरपि साम्यप्रसङ्गः~। अतः क्वचिद्विश्रामो वाच्यः~। यत्र तु विश्रामः, तस्याऽनित्यत्वेऽसमवेत
(भाव)कार्योत्पत्तिप्रसङ्गात् तस्य नित्यत्वम्~। महत्परिमाणतारतम्यस्य गगनादौ विश्रान्तत्वमिवाऽणुपरिमाणतारतम्यस्याऽपि क्वचिद्विश्रान्तत्वमस्तीति तस्य परमाणुत्वसिद्धिः~।
न च त्रसरेणावेव विश्रामोऽस्त्विति वाच्यम्~। त्रसरेणुः सावयवः चाक्षुषद्रव्यत्वात् घटवदित्यनुमानेन तदवयवसिद्धौ, त्रसरेणोरववयवाः सावयवाः महदारम्भकत्वात्
कपालवदित्यनुमानेन तदवयवसिद्धेः~। न चेदमप्रयोजकम्, अपकृष्टमहत्त्वं प्रत्यनेकद्रव्यवत्त्वस्य प्रयोजकत्वात्~। न चैवंक्रमेण तदवयवधाराऽपि सिद्ध्येदिति वाच्यम्~।
अनवस्थाभयेन तदसिद्धेरिति~। इति परमाणुसाधनम्॥
सा च त्रिधा भवेद्देहमिन्द्रियं विषयस्तथा॥३७॥
योनिजादिर्भवेद्देह इन्द्रियं घ्राणलक्षणम्~।
सा चेति~। सा-कार्यरूपापृथिवी त्रिविधा, शरीरेन्द्रियविषयभेदादित्यर्थः~। तत्र देहमुदाहरति-योनिजादिरिति~। योनिजमयोनिजं चेत्यर्थः~।
योनिजमपि द्विविधं-जरायुजमण्डलं चेति~। जरायुजं मानुषादीनाम्~। अण्डजं-सर्पादीनाम्~। अयोनिजं स्वेदजोद्भिञ्जादिकम्~।
स्वेदजाः- कृमिदंशाद्याः उद्भिज्जाः-तरुगुल्माद्याः~। नारकिणां शरीरमप्ययोनिजम्~।
न च मानुषादिशरीराणां पार्थिवत्वे किं मानमिति वाच्यम्~। गन्धादिमत्त्वस्यैव प्रमाणत्वात्~। न च क्लेदोष्मादेरुपलम्भादाप्यत्वादिकमपि स्यादिति वाच्यम्~। तथा
सति जलत्वपृथिवीत्वादिना सङ्करप्रसङ्गात्~। न च तर्हि जलीयत्वादिकमेवाऽस्तु, न तु पार्थिवत्वमिति वाच्यम्~। क्लेदादीनां विनाशेऽपि शरीरत्वेन प्रत्यभिज्ञानात्,
गन्धाद्युपलब्धेश्च पृथिवीत्वसिद्धेः~। तेन पार्थिवादिशरीरे जलादीनां निमित्तत्वमात्रं बोध्यम्~।
शरीरत्वं न जातिः, पृथिवीत्वादिना साङ्कर्यात्~। किन्तु चे~।चेष्टाश्रयत्वम्~। वृक्षादीनामपि चेष्टाश्रयत्वान्नाऽव्याप्तिः~। न च वृक्षादेः शरीरत्वे किं मानमिति वाच्यम्~।
आध्यात्मिकवायुसम्बन्धस्य प्रमाणत्वात्~। तत्रैव किं मानमिति चेत्, भग्नक्षतसंरोहणादिना तदुन्नयनात्~। यदि हस्तादौ शरीरव्यवहारो न भवति, तदाऽन्त्यावयवित्वेन
विशेषणीयम्~। न च यत्र शरीरे चेष्टा न जाता तत्राऽव्याप्तिरिति वाच्यम्~। तादृशे प्रमाणाभावात्~। अथ वा चेष्टावदन्त्यावयविवृत्ति-द्रव्यत्वव्याप्य-जातिमत्त्वम्, अन्त्यावयविमात्रवृत्ति-
चेष्टावद्वृत्ति-जातिमत्त्वं वा तत्, मानुषत्वचैत्रत्वादिजातिमादाय लक्षणसमन्वयः~। न च नृसिंहशरीरे कथं लक्षणसमन्वयः~। तत्र नृसिंहत्वस्यैकव्यक्तिवृत्तितया जातित्वाभावात्,
जलीयतैजसा शरीरवृत्तितया देवत्वस्यापि जातित्वाभावादिति वाच्यम्~। कलपभैदेन नृसिंहशरीरस्य नानात्वेन नृसिंहत्वजात्या लक्षणसमन्वयात्~।
इति पार्थिवशरीरनिरूपणम्॥
इन्द्रियमिति~। घ्राणेन्द्रियं पार्थिवमित्यर्थः~। पार्थिवत्वं कथमिति चेदित्थम्~। घ्राणेन्द्रियं पार्थिवं रूपादिषु मध्ये गन्धस्यैवाऽभिव्यञ्जकत्वात् कुङ्कुमगन्धाभिव्यञ्जकगोघृतवन्~।
न च दृष्टान्ते स्वीयरूपादि व्यञ्जकत्वाद्दसिद्धिरिति वाच्यम्~। परकीयरूपाद्यव्यञ्जकत्वस्य तदर्थत्वात्~। न च नवशराव (गत)गन्धव्यञ्जकजलेऽनैकान्तिकत्वमिति
वाच्यम्~। तस्य सक्तुरसाभिव्यञ्जकत्वात्~। यद्वा परकीयेति न देयम्, वायूपनीतसुरभिभागानां दृष्टान्तत्वसम्भवात्~। न च घ्राणेन्द्रियसन्निकर्षस्य गन्धमात्राभिव्यञ्जकत्वात्तत्र
व्यभिचार इति वाच्यम्~। द्रव्यत्वे सतीति विशेषणात्~।
इति पार्थिवेन्द्रियनिरूपणम्॥
विषयमाह-विषय इति~। उपभोगसाधनं विषयः~। सर्वमेव कार्यजातमदृष्टाधीनं, यत् कार्यं यददृष्टाधीनं, तत् तदुपभोगं साक्षात् परम्परया वा जनयत्येव, न हि
बीजप्रयोजनाभ्यां विना कस्यचिदुत्पत्तिरस्ति, तेन द्व्यणुकादि ब्रह्माण्डान्तं सर्वमेव विषयो भवति~। शरीरेन्द्रिययोविषयत्वेऽपि प्रकारान्तरेणोपन्यासः शिष्यबुद्धिवैशद्यार्थः॥
३६॥३७॥३९॥
इति पार्थिवविषयनिरूपणम्॥इति पृथिवीग्रन्थः॥
विषयो द्व्यणुकादिश्च ब्रह्माण्डान्त उदाहृतः॥३८॥
वर्णः शुक्लो रस-स्पर्शौ जले मधुर-शीतलौ~।
जलं निरूपयति-वर्णः शुक्ल इति~। स्नेहसमवायिकारणतावच्छेदकतया जलत्वजातिसिद्धिः~।
यद्यपि स्नेहत्वं नित्यानित्यवृत्तितया न कार्यतावच्छेदकं, तथाऽपि जन्यस्नेहत्वं तथा बोध्यम्~।
अथ परमाणौ जलत्वं न स्यात्, तत्र जन्यस्नेहाभावात्, तस्य च नित्यस्य स्वरूपयोग्यत्वे फलावश्यम्भावनियमादिति चेन्न~। जन्यस्नेहजनक
तावच्छेदिकाया जन्यजलत्वजातेः सिद्धौ, तदवच्छिन्नजनकतावच्छेदकतया जलत्वजातिसिद्धिः~।
शुक्लरूपमेव जलस्येति दर्शयितुमुक्तं-वर्णः शुक्ल इति, न तु शुक्लरूपवत्त्वं लक्षणम्~। अथ वा नैमित्तिकद्रवत्ववदवृत्ति-रूपवद्वृत्ति-द्रव्यत्व-साक्षाद्व्याप्य-
जातिमत्त्वं, अभास्वरशुक्लेतररूपासमानाधिकरण-रूपवद्वृत्ति-द्रव्यत्वसाक्षाद्व्याप्यजातिमत्त्वं वा तदर्थः, तेन स्फटिकादौ नाऽतिव्याप्तिः~।
रसस्पर्शाविति~। जलस्य मधुर एव रसः~। शीत एव स्पर्शः~। तिक्तरसवद्वृत्ति-मधुरवद्वृत्ति-द्रव्यत्वसाक्षाद्व्याप्य-जातिमत्त्वं तदर्थः, तेन शर्करादौ नाऽतिव्याप्तिः~।
(शीतेतरस्पर्शवदवृत्ति-स्पर्शवद्वृत्ति-द्रव्यत्व साक्षाद्व्याप्य-जातिमत्त्वं तदर्थः)
ननु शुक्लरूपमेवेति कुतः~? कालिन्दीजलादौ नीलिमोपलब्धेरिति चेन्न~। नीलजनकतावच्छेदिकायाः पृथिवीत्वजातेरभावाज्जले नीलरूपासम्भवात्, कालिन्दीजले
नीलत्वप्रतीतिस्त्वाश्रयौपाधिकी~। अत एव वियति विक्षेपे धवलिमोपलब्धिः~।
अथ जले माधुर्ये किं मानम्~? न हि प्रत्यक्षेण कोऽपि रसस्तत्राऽनुभूयते, न च नारिकेलजलादौ माधुर्यमुपलभ्यत एवेति वाच्यम्~। तस्याऽऽश्रयौपाधिकत्वात्,
अन्यथा जम्बीरजलादावम्लादिरसोपलब्धेरम्लादिमत्त्वमपि स्यादिति चेन्न~। हरीतक्यादिभक्षणस्य जलरसाभिव्यञ्जकत्वात्~। न च हरीतक्यामेव जलोष्मसंयोगाद्रसान्तरोत्पत्तिरिति
वाच्यम्~। कल्पनागौरवात्, पृथिवात्वस्याऽम्लादिजनकतावच्छेदकत्वाच्च जले नाऽम्लादिकम्~। जन्बीररसादौ त्वाश्रयौपाधिकी तथा प्रतीतिः~।
स्नेहस्तत्र द्रवत्वं तु सांसिद्धिकमुदाहृतम्॥३९॥
एवं जन्यशीतस्पर्शजनकतावच्छेदकं जन्यजलत्वं, तदवच्छिन्नजनकतावच्छेदकं जलत्वं बोध्यम्~। घृष्टचन्दनादौ तु शैत्योपलब्धिश्चन्दनान्तर्वर्तिशीततरसलिलस्यैव~।
तेजःसंयोगाज्जले उष्णप्रतीतिरौपाधिकी स्फुटैव, तत्र पाकासम्भवात्~।
स्नेहस्तत्रेति~। घृतादावपि तदन्तर्वर्तिजलस्यैव स्नेहः, जलस्य स्नेहसमवायिकारणत्वात्~। तेन जल एव स्नेह इति मन्तव्यम्~।
द्रवत्वमिति~। सांसिद्धिकद्रवत्वत्वं जातिविशेषः प्रत्यक्षसिद्धः तदवच्छिन्नजनकतावच्छेदकमपि तदेवेति भावः~। तैलादावपि जलस्यैव द्रवत्वं, स्नेहप्रकर्षेण च
दहनानुकूल्यमिति वक्ष्यति॥३६॥
॥इति जलनिरूपणे जललक्षणकथनम्॥
नित्यतादि प्रथमवत्, किन्तु देहमयोनिजम्~।
इन्द्रियं रसनं सिन्धुर्हिमादिर्विषयो मतः॥४०॥
प्रथमवदिति~। पृथिव्या इवेत्यर्थः~। तथा हि - जलं द्विविधं नित्यमनित्यं च, परमाणुरूपं नित्यं, दव्यणुकादि सर्वमनित्यमवयवसमवेतं च~। अनित्यमपि त्रिविधं-
शरीरेन्द्रियविषयभेदात्~। पृथिवीतो यो विशेषस्तमाह-किन्त्विति~। देहमयोनिजं-अयोनिजमेवेत्यर्थः~। जलीयं शरीरमयोनिजं वरुणलोके प्रसिद्धम्~।
॥इति जलीयशरीरनिरूपणम्॥
इन्द्रियमिति~। जलीयमित्यर्थः~। तथा हि - रसनं जलीयं, गन्धाद्यव्यञ्जकत्वे सति रसव्यञ्जकत्वात् सक्तुरसाभिव्यञ्जकोदकवत्~। रसनेन्द्रियसन्निकर्षव्यभिचारवारणाय
द्रव्यत्वं देयम्~।
॥इति जलीयेन्द्रियनिरूपणम्॥
विषयं दर्शयति-सिन्धुरिति~। सिन्धुः-समुद्रः, हिमं-तुषारः, आदिपदात् सरित्कासारकरकादिः सर्वोऽपि ग्राह्यः~। न च हिमकरकयोः कठिनत्वात् पार्थिवत्वमिति
वाच्यम्~। ऊष्मणा विलीनस्य तस्य जलत्वस्य प्रत्यक्षसिद्धत्वात्, यद्द्रव्यं यद्द्रव्यध्वंसजन्यमिति व्याप्तेर्जलोपादानोपादेयत्वसिद्धेः~।
अदृष्टविशेषेण द्रवत्वप्रतिरोधात् करकायां काठिन्यप्रत्ययस्य भ्रान्तित्वात्॥४०॥इति जलीयविषयनिरूपणम्॥इति जलग्रन्थः॥
उष्णः स्पर्शस्तेजसस्तु स्याद्रूपं शुक्लभास्वरम्~।
तेजो निरूपयति उष्ण इति~। उष्णत्वं-स्पर्शनिष्ठो जातिविशेषः प्रत्यक्षसिद्धः~। इत्थं च जन्योष्णस्पर्शसमवायिकारणतावच्छेदकं तेजस्त्वं जातिविशेषः, तस्य
परमाणुवृत्तित्वं जलत्वस्येवाऽनुसन्धेयम्~। न चोष्णस्पर्शवत्त्वं-चन्द्रकिरणादावव्याप्तमिति वाच्यम्~। तत्राऽप्युष्णस्पर्शस्य सत्त्वात्, किन्तु तदन्तः पातिजलस्पर्शेनाऽभिभवादग्रहः~।
एवं रत्नकिरणादौ च पार्थिवस्पर्शेनाऽभिभवात्, चक्षुरादौ चाऽनुद्भूतत्वादग्रहः~।
रूपमित्यादि~। वैश्वानरे मरकतकिरणादौ च पार्थिवरूपेणाऽभिभवात् शुक्लरूपाग्रहः~। अथ तद्रूपाग्रहे धर्मिणोऽपि चाक्षुषत्वं न स्यादिति चेन्न~। अन्यदीयरूपेणैव
धर्मिणो ग्रहसम्भवात् शङ्खस्येव पित्तपीतिम्ना~। वह्नेस्तु शुक्लं रूपं नाऽभिभूतं; किन्तु तदीयं शुक्लत्वमभिभूतमित्यन्ये~।
नैषित्तिकं द्रव्त्वं तुनित्यतादि च पूर्ववत्~।
नैमित्तिकमिति~। सुवर्णादिरूपे तेजसि तत्सत्त्वात्~। न च नैमित्तिकद्रवत्ववत्त्वं दहनादावव्याप्तं, घृतादावतिव्याप्तं चेति वाच्यम्~। पृथिव्यवृत्ति-नैमित्तिकद्रवत्ववद्वृत्ति-
द्रव्यत्वसाक्षाद्व्यप्य-जातिमत्त्वस्य विवक्षितत्वात्~।
॥इति तेजोनिरूपणे तेजोलक्षणकथनम्॥
पूर्ववदिति~। जलस्येवेत्यर्थः~। तथाहि-तद्द्विविधं, नित्यमनित्यं च, नित्यं परमाणुरूपं, तदन्यदनित्यमवयवि च~। तच्चत्रिधा शरीरेन्द्रियविषयभेदात्~।
शरीरमयोनिजमेव~। तच्च सूर्यलोकादौ प्रसिद्धम्॥४२॥
इति तैजसशरीरनिरूपणम्॥
नैमित्तिकं द्रवत्वं तु नित्यतादि च पूर्ववत्॥४१॥
इन्द्रियं नयनं वह्निः स्वर्णादिर्विषयो मतः~।
अत्र यो विशेषस्तमाह-इन्द्रियमिति~। ननु चक्षुषस्तैजसत्वे किं मानमिति चेत्, चक्षुस्तैजसं परकीयस्पर्शाद्यव्यञ्जकत्वे सति परकीयरूपव्यञ्जकत्वात् प्रदीपवत्~।
प्रदीपस्य स्वीय पर्शव्यञ्जकत्वादत्र दृष्टान्तेव्याप्तिवारणाय प्रथमं परकीयेति~। घटादेः स्वीयरूपव्यञ्जकत्वाद्व्यभिचारवारणाय द्वितीयं परकीयेति~।
अथवा प्रभाया दृष्टान्तत्वसम्भवादाद्यं परकीयेति न देयम्~। चक्षुःसन्निकर्षे व्यभिचारवारणाय द्रव्यत्वं देयम्~। इति तैजसेन्द्रियनिरूपणम्॥
विषयं दर्शयति-वह्निरिति~। ननु सुवर्णस्य तैजसत्वे किं मानमिति चेन्न~। सुवर्णं तैजसम् असति प्रतिबन्धकेऽत्यन्तानलसंयोगेऽप्यनुच्छिद्यमान(जन्य) द्रुतवत्वा
(धिकरणत्वा)त्, यन्नैवं तन्नैवं, यथा पृथिवीति~। न चाऽप्रयोजकं, पृथिवीद्रवत्वस्य जन्यजलद्रवत्वस्य चाऽत्यन्ताग्निसंयोगनाश्यत्वात्~। ननु पीतिमगुरुत्वाश्रयस्याऽपि तदानीं
द्रुतत्वात्तेन व्यभिचार इति चेन्न~। जलमध्यस्थमसीक्षोदवत् तस्याऽद्रुतत्वात्~।
अपरे तु पीतिमाश्रयस्याऽत्यन्ताग्निसंयोगेऽपि पूर्वरूपापरावृत्तिदर्शनात् तत्प्रतिबन्धकं विजातीयद्रवद्रव्यं कल्प्यते~। तथा हि-अत्यन्ताग्निसंयोगीे पीतिमगुरुत्वाश्रयः
विजातीयरूपप्रतिबन्धकद्रवद्रव्यसंयुक्तः अत्यन्ताग्निसंयोगे सत्यपि पूर्वरूपविजातीयरूपानधिकरणपार्थिवत्वात् जलमध्यस्थपीतपटवत्~। तस्य च पृथिवीजलभिन्नस्य तेजस्त्वनियमात्~।
॥इति तैजसविषयनिरूपणम्॥इति तेजोग्रन्थः॥
अपाकजोऽनुष्णाशीतस्पर्शस्तु पवने मतः॥४२॥
तिर्यग्गमनवानेष ज्ञेयः स्पर्शादिलिङ्गकः~।
वायुं निरूपयति-अपाकज इति~। अनुष्णाशीतस्पर्शस्य पृथिवीव्यामपि सत्त्वादुक्तमपाकज इति~। अपाकजस्पर्शस्य जलादावपि सत्त्वादुक्तमनुष्णाशीतेति~। एतेन
वायवीयो विजातीयः स्पर्शो दर्शितः, तज्जनकतावच्छेदकं वायुत्वमिति भावः॥४२॥इति वायुनिरूपणे वायुत्वजातौ प्रमाणकथनम्॥
एषः-वायुः, स्पर्शादिलिङ्गकः-वायुर्हि स्पर्श-शब्द-धृति-कम्पैरनुमीयते, विजातीयस्पर्शेन विलक्षणशब्देन तृणादीनां धृत्या शाखादीनां कम्पनेन पूर्ववदिति~। वायुर्द्धिविधः-
नित्योऽनित्यश्च, परमाणुरूपो नित्यः, तदन्योऽनित्योऽवयवसमवेतश्च, सोऽपि त्रिविधः-शरीरेन्द्रियविषयभेदात्~।
पूर्ववन्नित्यत्यताद्युक्तं देहव्यापि त्वगिन्द्रियम्॥४३॥
च वायोरनुमानात्~। यथा च वायुर्न प्रत्यक्षस्तथाऽग्रे वक्ष्यते~।
॥इति वायुनिरूपणे वायौ प्रमाणकथनम्॥
तत्र शरीरमयोनिजं पिशाचादीनाम्, परन्तु जलीय-तैजस-वायवीयशरीराणां पार्थिवभागोपष्टम्भादुपभोगक्षमत्वं, जलादीनां प्राधान्याज्जलीयत्वादिकमिति~।॥इति
वायवीयशरीरनिरूपणम्॥
अत्र यो विशेषस्तमाह - देहव्यापीति~। शरीरव्यापकं स्पर्शग्राहकमिन्द्रियंत्वक्, तच्च वायवीयं रूपादिषु मध्ये स्पर्शस्यैवाऽभिव्यञ्जकत्वात् अङ्गसत्वक्, तच्च
वायवीयं रूपादिषु मध्ये स्पर्शस्यैवाऽभिव्यञ्जकत्वात् अङ्गसङ्गिसलिलशैत्याभिव्यञ्जकव्यजनपवनवत्॥४३॥इति वायवीयेन्द्रियनिरूपणम्~।
विषयं दर्शयति-प्राणादिरिति~। यद्यप्यनित्यो वायुश्चतुर्विधः, तस्य चतुर्थी विधा प्राणादिरित्युक्तमाकरे, तथाऽपि सङ्क्षेपादेव त्रैविध्यमुक्तम्~। प्राणस्त्वेेक एव
हृदादिनानास्थानवशान्मुखनिर्गमादिनानाक्रियाभेदाच्च नाना संज्ञां लभत इति~।
इति वायवीयविषयनिरूपणम्~।॥इति वायुग्रन्थः॥
प्राणादिस्तु महावायुपर्यन्तो विषयो मतः~।
आकाशस्य तु विज्ञेयः शब्दो वैशेषिको गुणः॥४४॥
आकाशं निरूपयति - आकाशस्येति~। आकाशकालदिशामेकैकव्यक्तित्त्वादाकाशत्वादिकं न जातिः, किन्तु आकाशत्वं शब्दाश्रयत्वम्~। "वैशेषिकः" इति कथनं तु
विशेषगुणान्तरव्यवच्छेदाय~।
एतेन प्रमाणमपि दर्शितम्~। तथाहि--शब्दो गुणः चक्षुग्र्रहणायोग्य-बहिरिन्द्रियग्राह्यजातिमत्त्वात् स्पर्शवत्, शब्दो द्रव्यसमवेतः गुणत्वात् संयोगवदित्यनुमानेन शब्दस्य
द्रव्यसमवेतत्वे सिद्धे, शब्दो न स्पर्शवद्विशेषगुणः अग्निसंयोगासमवायिकारणकत्वाभावे सति अकारणगुणपूर्वकप्रत्यक्षत्वात् , सुखवत्~। पाकजरूपादौ व्यभिचारवारणाय
सत्यन्तम्~। पटरूपादौ व्यभिचारवारणायाऽकारणगुणपूर्वकेति~। जलपरमाणुरूपादौ व्यभिचारवारणाय प्रत्यक्षेति~। शब्दो न दिक्कालमनसां गुणः विशेषगुणत्वात् रूपवत्~।
नाऽऽत्मविशेषगुणः बहिरिन्द्रिययोग्यत्वात् रूपवत्~। इत्थञ्च शब्दाधिकरणं नवमं द्रव्यं सिद्ध्यति~। न च वाय्ववयवेषु सूक्ष्मशब्दक्रमेण वायौ कारणगुणपूर्वकः शब्द
उत्पद्यतामिति वाच्यम्~। अयावद्द्रव्यभावित्वेन वायोर्विशेषगुणत्वाभावात्॥४४॥इत्याकाशे प्रमाणकथनम्॥
तत्र च शरीरस्य विषयस्य चाऽभावादिन्द्रियं दर्शयति-इन्द्रियमिति~। नन्वाकाशं लाघवादेकं सिद्धं, श्रोत्रं पुनः पुरुषभेदाद्भिन्नं कथमाकाशं स्यादिति चेत्तत्राऽऽह-एक
इति~। आकाश एकः सन्नपि उपाधेःकर्णशष्कुल्या भेदाद्भिन्नं श्रोत्रात्मकं भवतीत्यर्थः~।
॥इत्याकाशेन्द्रियनिरूपणम्॥इत्याकाशग्रन्थः॥
कालं निरूपयति-जन्यानामिति~। इति कालनिरूपणे काललक्षणकथनम्॥तत्र प्रमाणं दर्शयितुमाह-जगतामाश्रय इति~। (तथा हि) इदानीं घट इत्यादिप्रतीतिः
सूर्यपरिस्पन्दादिकं यदा विषयीकरोति, तदा सूर्यपरिस्पन्दादिना घटादेः सम्बन्धघटकः कल्प्यते~। इत्थं च तस्याऽऽश्रयत्वमेव सम्यक्॥४५॥
प्रमाणान्तरं दर्शयति-परापरत्वेति~। परत्वापरत्वादिबुद्धेरसाधारणं निमित्तं काल एव, परत्वापरत्वयोरसमवायिकारणसंयोगाश्रयो लाघवादतिरिक्तः कालः कल्प्यत
इति भावः~। इति काले प्रमाणकथनम्॥
नन्वेकस्य कालस्य सिद्धौ क्षण-दिन-मास-वर्षादिसमयभेदो न स्यादित्यत आह-क्षणादिः स्यादुपाधित इति~। कालस्त्वेकोऽप्युपाधिभेदात् क्षणादिव्यवहारविषयः~।
उपाधिस्तु स्वजन्यविभागप्रागभावावच्छिन्नं कर्म, पूर्वसंयोगनाशावच्छिन्नविभागो वा, पूर्वसंयोगनाशावच्छिन्नोत्तरसंयोगानन्तरं क्षणव्यवहारो न स्यादिति वाच्यम्~।
कर्मान्तरस्याऽपि सत्त्वादिति~। महाप्रलये क्षणादिव्यवहारो यद्यस्ति, तदा ध्वंसेनैवोपपादनीय इति~। दिनादिव्यवहारस्तु तत्तत्क्षणकूटैरिति॥इति कालस्यैकत्वव्यवस्थापनम्॥
इति कालग्रन्थः॥
दिशं निरूपयति-दूरान्तिकेति~। दूरत्वमन्तिकत्वं च दैशिकं परत्वमपरत्वं बोध्यम्~। तद्बुद्धेरसाधारणं बीजं दिगेव~। दैशिकपरत्वापरत्वयोरसमवायिकारणसंयोगश्रयतया
लाघवादेका दिक् सिद्ध्यतीति भावः॥४६॥
इति दिशि प्रमाणकथनम्॥
इन्द्रयन्तु भवेच्छोत्रमेकः सन्नप्युपाधितः~।
जन्यानां जनकः कालः जगतामाश्रयो मतः॥४५॥
परापरत्वधीहेतुः, क्षणादिः स्यादुपाधितिः~।
दूरान्तिकादिधीहेतुरेका नित्या दिगुच्यते॥४६॥
ननु यद्येकैव दिक्, तदा प्राचीप्रतीच्यादिव्यवहारः कथमुपपद्यत इत्यत आह-उपाधिमेदादिति~। यत्पुरुषस्योदयगिरिसन्निहिता या दिक् सा तत्पुरुषस्य प्राची,
एवमुदयगिरिव्यवहिता या दिक् सा प्रतीची~। एवं यत्पुरुषस्य सुमेरुसन्निहिता या दिक्, सोदीची, तद्व्यवहिता त्ववाची~। "सर्वेषामेव वर्षाणां मेरुरुत्तरतः स्थितः-" इति दिश
एकत्वव्यवस्थापनम्॥इति दिग्ग्रन्थः॥
आत्मानं निरूपयति-आत्मेन्द्रियाथधिष्ठातेति~। आत्मत्वजातिस्तु सुखदुःखादिसमवायिकारतावच्छेदकतया सिद्धयति~। ईश्वरेऽपि सा जातिरस्त्येव, अदूष्टादिरूपकारणाभावन्न
सुखदुःखाद्युत्पत्तिः, नित्यस्य स्वरूपयोग्यस्य फलावश्यम्भावनियम इत्यस्याऽप्रयोजकत्वात्~।
परे तु ईश्वरे सा जातिर्नास्त्येव, प्रमाणाभावात्~। न च दशमद्रव्यत्वापत्तिः, ज्ञानत्त्वेन विभजनादित्याहुः~। इत्यात्मत्वजातौ प्रमाणकथनम्॥
इन्द्रियाद्यधिष्ठाता-इन्द्रियाणां शरीरस्य च परंम्परया चैतन्यसम्पादकः~।
उपाधिभेदादेकाऽपि प्राच्यादिव्यपदेशभाक्
आत्मेन्द्रियाद्यधिष्ठाता करणं हि सकर्तृकम्॥४७॥
शरीरस्य न चैतन्यं मृतेषु व्यभिचारतः~।
यद्यप्यात्मनि अहं सुखी अहं दुःखीत्यक्षविषयत्वमसत्येव, तथाऽपि विप्रतिपन्नं प्रति प्रथमत एव शरीरादिभिन्नस्तत्प्रतीतिगोचर इति प्रतिपादयितुं न शक्यते, इत्यतः
प्रमाणान्तरं दर्शयति-करणमिति~। वाम्यादीनां च्छिदादिकरणानां कत्र्तरमन्तरेण फलानुपधानं दृष्टम्, एवं चक्षुरादीनां ज्ञानकरणानामपि फलोपधानं कर्तारमन्तरेण नोपपद्यत
इत्यतिरिक्तः कर्ता कल्प्यते॥४७॥
इत्यात्मनि प्रमाणकथनम्॥
ननु शरीरस्य कर्तृत्वमस्त्वत आह-शरीरस्येति~। ननु चैतन्यं ज्ञानादिक, मेव, मुक्तात्मनां त्वन्मत इव मृतशरीराणामपि तद्भावे का क्षतिः~? प्राणाभावेन
ज्ञानाभावस्य सिद्धेरिति चेन्न~। शरीरस्य चैतन्ये बाल्ये विलोकितस्य स्थाविरे स्मरणानुपपत्तेः, शरीराणामवयवोपचयापचयैरुत्पादविनाशशालित्वात्~। न च पूर्वशरीरोत्पन्नसंस्कारेण
द्वितीयशरीरे संस्कार उत्पाद्यत इति वाच्यम्~। अनन्तसंस्कारकल्पने गौरवात्~। एवंशरीरस्य चैतन्ये बालकस्य स्तन्यपाने प्रवृत्तिर्नस्यात्, इष्टसाधनताज्ञानस्य तद्धेतुत्वात्,
तदानीमिष्टसाधनतास्मारकाभावात्~। मन्मते तु जन्मान्तरानुभूतेष्टसाधनत्वस्य तदानीं स्मरणादेव प्रवृत्तिः~। न च जन्मान्तरानुभूतमन्यदपि स्मर्थतामिति वाच्यम्~। उद्बोधकाभावात्,
अत्र त्वनायत्या जीवनादृष्टमेवोद्बोधकं कल्प्यते~। इत्यञ्च संसारस्याऽनादितया आत्मनोऽनादित्वसिद्धावनादिभावस्य नाशासम्भवान्नित्यत्वं सिद्धयतीति(१)बोध्यम्॥
॥इति शरीरात्मवादिचार्वाकमतखण्डनम्॥
ननु चक्षुरादीनामेव ज्ञानादिकं प्रति करणत्वं कर्तृत्वं चाऽस्तु, विरोधे साधका (१) भावादत आह-तथात्वमिति~। तथात्वं-चैतन्यम्~। उपघातेनाशे सति,
अर्थात्चक्षुरादीनामेव~। कथं स्मृतिः~? पूर्वं चक्षुषाऽनुभूतानां चक्षुरभावे स्मरणं न स्यात्, अनुभवितुरभावात् अन्येनाऽनुभूतस्याऽन्येन स्मरणासम्भवात्, अनुभवस्मरणयोः
सामानाधिकरण्येन कार्यकारणभावादिति भावः (१)॥४॥इतीन्द्रियात्मवादिमतखण्डनम्॥
ननु चक्षुरादीनां चैतन्यं माऽस्तु, मनसस्तु नित्यस्य चैतन्यं स्यादत आह - मनोऽपीति~। न तथा-न चेतनम्~। ज्ञानाद्यनध्यक्षं तदा भवेत्, मनसोऽणुत्वात्प्रत्यक्षे
महत्त्वस्य हेतुत्वात् मनसि ज्ञानसुखादिसत्त्वे तत्प्रत्यक्षानुपपत्तिरित्यर्थः (१)~। यथा च मनसोऽणुत्वं, तथाऽग्रे वक्ष्यते~।
॥इति मनात्मवादिमतखण्डम्॥
नन्वस्तु विज्ञानमेवाऽऽत्मा, तस्य स्वतःप्रकाशरूपत्वाच्चेतनत्वम्~। ज्ञानसुखादिकन्तु तस्यैवाऽऽकारविशेषः~। तस्याऽपि भावत्वादेव क्षणिकत्वं, (१)
पूर्वपूर्वविज्ञानस्योत्तरोत्तरविज्ञाने हेतुत्वात् सुषुप्त्यवस्थायामप्यालयविज्ञानधारा निराबाधैव, मृगमदवासनावासितवसन इव पूर्वपूर्वविज्ञानजनितसंस्काराणमुत्तरविज्ञाने
हेतुत्वान्नाऽनुपपत्तिः स्मरणादेरिति चेन्न~। तस्य जगद्विषयकत्वे सर्वज्ञत्वापत्तिः, यत्किञ्चिद्विषयकत्वे विनिगमनाविरहः, सुषुप्तावपि विषयावभासप्रसङ्गाच्च ज्ञानस्य
सविषयत्वात्~। तदानीं निराकारा चित्सन्ततिरनुवर्तत इति चेन्न~। तस्याः प्रकाशत्वे प्रमाणाभावात्, अन्यथा घटादीनामपि ज्ञानत्वापत्तिः~। न चेष्टापत्तिः,
विज्ञानव्यतिरिक्तवस्तुनोऽभावा(१) दिति वाच्यम्~। घटादेरनुभूयमानस्याऽपलपितुमशक्यत्वात्~। आकारविशेष एवाऽयं विज्ञानस्येति चेन्न~। किमयमाकारोऽतिरिच्यते विज्ञानात्~?
तर्हि समायातं विज्ञानव्यतिरिक्तेन~। नाऽतिरिच्यते चेत्, तर्हि समूहालम्बने नीलाकारोऽपि पीताकारः स्यात्, स्वरूपतो विज्ञानस्याऽविशेषात्~। अपोहरूपो नीलत्वादिर्विज्ञानधर्म
इति चेन्न~। नीलत्वादीनां विरुद्धानामेकस्मिन्नसमावेशात्, इतरथा विरोधावधारणस्यैव दुरुपपादत्वात्~। नवा वासनासङ्क्रमः सम्भवति, मातृपुत्रयोरपि वासनासङ्क्रमप्रसङ्गात्~।
न चोपादानोपादेयभावो नियामक इति वाच्यम्~। वासनायाः सङ्क्रमासम्भवात्~। उत्तरस्मिन्नुत्पत्तिरेव सङ्क्रम इति चेन्न~। तदुत्पादकाभावात्, चितामेवोत्पादकत्वे संस्कारानन्त्यप्रसङ्गः~।
क्षणिकविज्ञानेष्वतिशयविशेषः कल्प्यत इति चेन्न~। मानाभावात्, कल्पनागौरवात्~।
॥इति क्षणिकविज्ञानात्मवादिबौद्धमतखण्डनम्॥
एतेन क्षणिकशरीरेष्वेव चैतन्यं प्रत्युक्तं, गौरवात्, अतिशये मानाभावाच्च~। बीजादावपि सहकारिसमवधानासमवधानाभ्यामेवोपपत्तेः कुर्वद्रूपत्वाकल्पनात्(१)॥
इति क्षणिकशरीरात्मवादिमतखण्डनम्॥
अस्तु तर्हिक्षणिकविज्ञाने गौरवात् नित्यविज्ञानमेवाऽऽत्मा, ""अविनाशी वा अरेऽयमात्मा, सत्यं ज्ञानमनन्तं ब्रह्म-"" इत्यादिश्रुतेश्चेति चेन्न~। तस्य सविषयकत्वासम्भवस्य
दर्शितत्वात्, निर्विषयस्य ज्ञानत्वे मानाभावात्, अतो विज्ञानादिभिन्नो नित्य आत्मेति सिद्धम्~। ""सत्यं ज्ञानं-"" इति तु ब्रह्मपरं, जीवे तु नोपयुज्यते, ज्ञानाज्ञान
सुखि(त्वदुःखि) त्वादिभिर्जीवानां भेदसिद्धौ सुतरामीश्वरभेदः, सिद्ध्यति~। अन्यथा बन्धमोक्षव्यवस्थाऽनुपपत्तेः~। योऽपीश्वराभेदबोधको वेदः, सोऽपि तदभेदेन तदीयत्वं
प्रतिपादयँस्तौति, अभेेदभावनयैव यतितव्यमिति वदति~। अत एव ""सर्वे आत्मानः समर्पिताः"" इति श्रूयते~। मोक्षदशायामज्ञाननिवृत्तावभेदो जायते, इत्यपि न, भेदस्य
नित्यत्वेन नाशासम्भवात्, भेदनाशेऽपि व्यक्तिद्वयं स्थास्यत्येव~। न च द्वित्वमपि नश्यतीति वाच्यम्~। तव निर्धर्मके ब्रह्मणि सत्यत्वाभावेऽपि सत्यत्वरूपं तदितिवत्
द्वित्वाभावेऽपि व्यक्तिद्वयात्मकौ ताविति सुवचत्वात्~। मिथ्यात्वाभावोऽधिकरणात्मकस्तत्र सत्यत्वमिति चेत्, एकत्वाभावो व्यक्तिद्वयात्मको द्वित्वमित्यप्युच्यताम्, प्रत्येकमेकत्वेऽपि
पृथिवीजलयोर्न गन्ध इतिवदुभयं नैकमित्यस्य सर्वजनसिद्धत्वात्~। योऽपि तदानीमभेदप्रतिपादको वेदः, सोऽपि निर्दुःखत्वादिना साम्यं प्रतिपादयति, सम्पदाधिक्ये पुरोहितोऽयं
राजा संवृत्त इतिवत्~। अत एव ""निरञ्चनः परमं साम्यमुपैति"" इति श्रूयते~। ईश्वरोऽपि न ज्ञानसुखात्मा, किन्तु ज्ञानाद्याश्रयः, ""नित्यं विज्ञानमानन्दं ब्रह्म"" इत्यादौ
विज्ञानपदेन ज्ञानाश्रय एवोक्तः, ""यः सर्वज्ञः स सर्ववित्"" इत्याद्यनुरोध्~। "आनन्दम्" इत्यस्याऽप्यानन्दवदित्यर्थः~। अर्शआदित्वान्मत्वर्थीयोऽच्प्रत्ययः, अन्यथा पुँल्लिङ्गत्वापत्तेः~।
आनन्दोऽपि दुःखाभावे उपचर्यते, भाराद्यपगमे सुखी संवृत्तोऽहमिति वत्~। अस्तु वा तस्मिन्नानन्दः, नत्वसावानन्दः, ""असुखम्"" इति श्रुतेः~। न विद्यते सुखं यस्येति कुतो
नाऽर्थ इति चेन्न~। क्लिष्टकल्पनापत्तेः, प्रकरणविरोधात्, "आनन्दम्" इत्यत्र मत्वर्थीयाच्प्रत्यथविरोधाच्चेति सङ्क्षेपः॥
॥इति नित्यविज्ञानात्मवादिवेदान्तिमतखण्डनम्॥
एतेन-प्रकृतिः कत्र्री, पुरुषस्तु पुष्करपलाशवन्निर्लेपः, किन्तु चेतनः, कार्यकारणयोरभेदात् कार्यनाशे सति कार्यरूपतया तन्नाशोऽपि न स्यादित्यकारणत्वं तस्य,
बुद्धिगतचैतन्याभिमानान्यथानुपपत्त्या तत्कल्पनम्~। बुद्धिश्च प्रकृतेः परिणामः, सैव महत्तत्वमन्तःकरणमित्युच्यते, तत्सत्त्वासत्वाभ्यं पुरुषस्य संसारापवगौ, तस्या एवेन्द्रियप्रणालिकया
परिणतिज्र्ञानरूपा घटादिना सम्बन्धः, पुरुषे कर्तृत्वाभिमानो बुद्धौ चैतन्याभिमानश्च भेदाग्रहात्, ममेदं कर्तव्यमिति मदंशः पुरुषोपरागो बुद्धेः स्वच्छतया तत्प्रतिबिम्बादतात्त्विकः,
दर्पणस्येव मुखोपरागः, इदमिति विषयोपराग इन्द्रियप्रणालिकया परिणतिभेदस्तात्त्विकः, निःश्वासाभिहतदर्पणस्येव मलिनिमा, कर्तव्यमिति व्यापारांशः, तेनांशत्रयवती बुद्धिः,
तत्परिणामेन ज्ञानेन पुरुषस्याऽतात्त्विकः सम्बन्धः, दर्पणस्य मलिनिम्नेव मुखस्योपलब्धिरुच्यते~। ज्ञानवत् सुखदुःखेच्छाद्वेषधर्मधर्मा अपि बुद्धेरेव, कृतिसामानाधिकरण्येन
प्रतीतेः~। न च बुद्धिश्चेतना, परिणामित्वात्, इति मतमपास्तम्~।
कृत्यदृष्टभोगानामिव चैेतन्यस्याऽपि सामानाधिकरण्यप्रतीतेः, तद्भिन्ने मानाभावाच्च~। चेतनोऽहं करोमीति प्रतीतिश्चैतन्यांशे भ्रम इति चेत्, कृत्यंशेऽपि किं न
नेष्यते~? अन्यथा बुद्धेर्नित्यत्वे मोक्षाभावः, अनित्यत्वे तत्पूर्वमसंसारापत्तिः~। नन्वचेतनायाः प्रकृतेः कार्यत्वात् बुद्धेरचेतनत्वम्, कार्यकारणयोस्तादात्म्यादिति चेन्न~। असिद्धेः,
कर्तुर्जन्यत्वे मानाभावात्~। वीतरागजन्मादर्शनादनादित्वम्, अनादेर्नाशासम्भवान्नित्यत्वम् तत् किं प्रकृत्यादिकल्पनेन~? न च-
""प्रकृतेः क्रियमाणानि गुणैः कर्माणि सर्वशः~।
अहङ्कारविमूढात्मा कर्ताऽहमिति मन्यते-"" (भगवद्गीता) इत्यनेन विरोध इति वाच्यम्~। प्रकृतेः- अदृष्टस्य, गुणैः - अदृष्टजन्यैरिच्छादिभिः, कर्ताऽहमिति-
कर्ताऽहमेवेत्यस्य तदर्थत्वात्,
""तत्रैवं सति कर्तारमात्मानं केवलं तु यः-"" (भगवद्गीता) इत्यादि वदता भगवता प्रकटीकृतोऽयमुपरिष्टादाशय इति सङ्क्षेपः~।
॥इति साङ्ख्यमतखण्डनम्॥
धर्माधर्मेति~। आत्मेत्यनुषज्यते~। शरीरस्य तदाश्रयत्वे देहान्तरकृतकर्मणां देहान्तरेण भोगानुपपत्तेः~। विशेषेति~। योग्यविशेषगुणस्य ज्ञानसुखादेः सम्बन्धेनाऽऽत्मनः
प्रत्यक्षत्वं सम्भवति, न त्वन्यथा, अहं जानेऽहं करोमीत्यादिप्रतीतेः॥४६॥इत्यात्मनि प्रमाणकथनम्॥
तथात्वं चेदिन्द्रियाणामुपघाते कथं स्मृतिः॥४८॥
मनोऽपि न तथा, ज्ञानाद्यनध्यक्षं तदा भवेत्~।
धर्माधर्माश्रयोऽध्यक्षो विशेषगुणयोगतः॥४९॥
""प्रकृतेः क्रियमाणानि गुणैः कर्माणि सर्वशः~।
अहङ्कारविमूढात्मा कर्ताऽहमिति मन्यते"" (भगवद्गीता)
इत्यनेन विरोध इति वाच्यम्~। प्रकृतेः-अदृष्टस्य
प्रवृत्त्याद्यनुमेयोऽयं रथगत्येव सारथिः~।
अहङ्कारस्याऽऽश्रयोऽयं मनोमात्रस्य गोचरः॥५०॥
विभुर्बुद्ध्यादिगुणवान् बुद्धिस्तु द्विविधा मता॥
प्रवृत्त्येति~। अयमात्मा परदेहादौ प्रवृत्त्यादिनाऽनुमीयते~। प्रवृत्तिरत्र चेष्टा, ज्ञानेच्छाप्रयत्नादीनां देहेऽभावस्योक्तप्रायत्वात्, चेष्टायाश्च प्रयत्नसाध्यत्वात्, चेष्टया
प्रयत्नवानात्माऽनुमीयत इति भावः~।
अत्र दृष्टान्तमाह - रथेति~। यद्यपि रथकर्म चेष्टा न भवति, तथाऽपि तेन कर्मणा सारथिर्यथाऽनुमीयते, तथा चेष्टात्मकेन कर्मणा परात्माऽनुमीयत इति भावः~।
इति परदेहादावात्मनि प्रमाणकथनम्॥
अहङ्कारस्येति~। अहङ्कारः- अहमिति प्रत्ययः, तस्याऽऽश्रयः -विषयः आत्मा, न शरीरादिरिति~।
मन इति~। मनोभिन्नेन्द्रियजन्यप्रत्यक्षाविषयः, मानसप्रत्यक्षविषयश्चेत्यर्थः, रूपाद्यभावेनेन्द्रियान्तरायोग्यत्वात्॥५०॥
विभुरिति~। विभुत्वं - परममहत्परिमाणवत्त्वम्~। तच्च पूर्वमुक्तमपि स्पष्टार्थमुक्तम्~। बुद्ध्यादिगुणवानिति~। बुद्धिसुखदुःखेच्छादयश्चतुर्दश गुणाः पूर्वमुक्ता वेदितव्याः~।
इत्यात्मस्वरूपकथनम्॥
अनुभूतिः स्मृतिश्च स्यात् अनुभूतिश्चतुर्विधा॥५१॥
अत्रैव प्रसङ्गात् कतिपयं बुद्धेः प्रपञ्चं दर्शयति~। द्वैविध्यं व्युत्पादयति अनुभूतिरिति~। आसां चतसृणां करणानि चत्वारि "प्रत्यक्षानुमानोपमानशब्दाः प्रमाणानि"इति
सूत्रोक्तानि वेदितव्यानि~। इन्द्रियजन्यं ज्ञानं प्रत्यक्षम्~। यद्यपि मनोरूपेन्द्रियजन्यं सर्वमेव ज्ञानं, तथाऽपीन्द्रियत्वेन रूपेणेन्द्रियाणां यत्र ज्ञाने करणत्वं, तत्प्रत्यक्षमिति
विवक्षितम्~। ईश्वरप्रत्यक्षं तु न लक्ष्यम्, "इन्द्रियार्थसन्निकर्षोत्पन्नं ज्ञानमव्यपदेश्यमव्यभिचारि व्यवसायात्मकं प्रत्यक्षं" इति सूत्रे तथैवोक्तत्वात्~।
अथवा ज्ञानाकरणकं ज्ञानं प्रत्यक्षम्~। अनुमितौ व्याप्तिज्ञानस्य, उपमितौ सादृश्यज्ञानस्य, शाब्दबोधे पदज्ञानस्य, स्मृतावनुभवस्य करणत्वात्तत्र नाऽतिव्याप्तिः~। इदं
लक्षणमीश्वरप्रत्यक्षसाधारणम्~। परामर्शजन्यं ज्ञानमनुमितिः~। यद्यपि परामर्शप्रत्यक्षादिकं परामर्शजन्यं, तथाऽपि परामर्शजन्यं हेत्वविषयकं ज्ञानं तदेवाऽनुमितिः~। न च
कादाचित्कहेतुविषयकानुमितावव्याप्तिरिति वाच्यम्~। तादृशज्ञानवृत्त्यनुभवत्वव्याप्यजातिमत्त्वस्य विवक्षितत्वात्~।
अथवा व्याप्तिज्ञानकरणकं ज्ञानमनुमितिः~।
एवं सादृश्यज्ञानकरणकं ज्ञानमुपमितिः~।
पदज्ञानकरणकं ज्ञानं शाब्दबोधः~।
प्रत्यक्षमप्यनुमितिस्तथोपमितिशब्दजे~।
घ्राणजादिप्रभेदेन प्रत्यक्षं षड्विधं मतम्॥५२॥
वस्तुतो यां काञ्चिदनुमितिव्यक्तिमादाय तद्व्यक्तिवृत्तित्वे सति यां काञ्चित् प्रत्यक्षव्यक्तिमादाय तदवृत्तिजातिमत्त्वमनुमितित्वम्~। एवं यत्किञ्चित्प्रत्यक्षादि
कमादाय तद्व्यक्तिवृत्त्यनुमित्यवृत्तिजातिमत्त्वं प्रत्यक्षत्वादिकं वाच्यमिति~।
॥इति प्रत्यक्षादिप्रमालक्षणकथनम्॥
जन्यप्रत्यक्षं विभजते - घ्राणजादीति~। घ्राणजं, रासनं, चाक्षुषं, स्पार्शनं, श्रोत्रं, मानसमिति षड्विधं प्रत्यक्षम्~। न चेश्वरप्रत्यक्षस्याऽविभजनान्न्यूनत्वं, जन्यप्रत्यक्षस्यैव
निरूपणीयत्वात्, उक्तसूत्रानुसारात्॥४१॥४२॥
गोचर इति~। ग्राह्य इत्यर्थः~। गन्धत्वादिरित्यादिपदात् सुरभित्वादिपरिग्रहः~। गन्धस्य प्रत्यक्षत्वात् तद्वृत्तिजातिरपि प्रत्यक्षा, गन्धाश्रयग्रहणे तु घ्राणस्याऽसामथ्र्यमिति
बोध्यम्~।
तथा रस इति~। रसत्वादिसहित इत्यर्थः~।
तथा शब्दोऽपि-शब्दत्वादिसहितः~। गन्धो रसश्चोद्भूतो बोध्यः॥५३॥उद्भूतरूपमिति~। ग्रीष्मोष्मादावनुद्भूतं रूपमिति न तत्प्रत्यक्षम्~। तद्वन्ति--उद्भूतरूपवन्ति~।
योग्येति~। पृथक्त्वादिकमपि योग्यव्यक्तिवृत्तितया बोध्यम्~। तादृशंयोग्यव्यक्तिवृत्तिम्~। चक्षुर्योग्यत्वमेव कथं~? तदाह - गृह्णातीति~। आलोकसंयोग उद्भूतरूपं च
चाक्षुषप्रत्यक्षे कारणम्~। तत्र द्रव्यचाक्षुषं प्रति तयोः समवायसम्बन्धेन कारणत्वं, द्रव्यसमवेतरूपादिप्रत्यक्षे स्वाश्रयसमवेतसमवायसम्बन्धेनेति॥५४॥५७॥
घ्राणस्य गोचरो गन्धो गन्धत्वादिरपि स्मृतः~।
तथा रसो रसज्ञायाः तथा शब्दोऽपि च श्रुतेः॥५३॥
उद्भूतरूपं नयनस्य गोचरो द्रव्याणि तद्वन्ति पृथक्त्वसङ्ख्ये
विभागसंयोगपरापरत्वस्नेहद्रवत्वं परिमाणयुक्तम्॥५४॥
क्रियां जातिं योग्यवृतिं्त समवायं च तादृशम्~।
गृह्णाति चक्षुःसम्बन्धादालोकोद्भूतरूपयोः॥५५॥
उद्भूतस्पर्शवत् द्रव्यं गोचरः सोऽपि च त्वचः~।
रूपान्यच्चक्षुषो योग्यं रूपमत्राऽपि कारणम्॥५६॥
उद्भूतेति~। उद्भूतस्पर्शवद्द्रव्यं त्वचो गोचरः~। सोऽपि-उद्भूतस्पर्शोऽपि, स्पर्शत्वादिसहितः~। रूपान्यदिति~। रूपभिन्नं रूपत्वादिभिन्नं यच्चक्षुषो योग्यं, तत्
त्वगिन्द्रियस्याऽपि ग्राह्यम्~। तथा च पृथक्त्वसङ्ख्यादयो ये चक्षुग्र्राह्या उक्ताः, एवं क्रियाजा(तिप्रभृ)तयो योग्यवृत्तयस्त्वचो ग्राह्या इत्यर्थः~।
अत्राऽपि -त्वगिन्द्रियजन्यद्रव्यप्रत्यक्षे रूप कारणम्~। तथा च बहिरिन्द्रियजन्यद्रव्यप्रत्यक्षे रूपं कारणम्~।
नवीनस्तु--त्वगिन्द्रियजन्यद्रव्यप्रत्यक्षमात्रे न रूपं कारणं, प्रमाणाभावात्, किन्तु चाक्षुषप्रत्यक्षे रूपं, स्पार्शनप्रत्यक्षे स्पर्शः कारणं, अन्वयव्यतिरेकात्~।
बहिरिन्द्रियजन्यद्रव्यप्रत्यक्षमात्रे किं कारणमिति चेत्, न किञ्चित्, आत्मावृत्तिश्ब्दभिन्नविशेषगुणवत्त्वं वा प्रयोजकमस्तु~। रूपस्य कारणत्वे लाघवमिति चेन्न~।
वायोस्त्वगिन्द्रियेणाऽग्रहणप्रसङ्गात्~। इष्टापत्तिरिति चेत्~। उद्भूतस्पर्श एव लाघवात् कारणमस्तु~। प्रभाया अप्रत्यक्षत्वे त्विष्टापत्तिरेव किं नेष्यते~? तस्मात् प्रभां
पश्यामीतिवत् वायुं स्पृशामीति प्रत्ययस्य सत्त्वाद्वायोरपि प्रत्यक्षत्वं सम्भवत्येव, बहिरिन्द्रियजन्यद्रव्यप्रत्यक्षमात्रे न रूपस्य, न वा स्पर्शस्य हेतुत्वम्~। वायुप्रभयोरेकत्वं गृह्यत
एव, क्वचित् द्वित्वादिकमपि, क्वचित् सङ्ख्यापरिमाणाद्यग्रहो दोषादित्याहुः॥५६॥इति चाक्षुषादिप्रत्यक्ष-चक्षुर्विषयादिकथनम्॥
द्रव्याध्यक्षे त्वचो योगो मनसा ज्ञानकारणम्~।
त्वचो योग इति~। त्वङ्मनःसंयोगो ज्ञानसामन्ये कारणमित्यर्थः~। किन्तत्र प्रमाणं~? सुषुप्तिकाले त्वचं त्यक्त्वा पुरीतति वर्तमानेन मनसा ज्ञानाजननमिति~। ननु
सुषुप्तिकाले किं ज्ञानं भविष्यति~? अनुभवरूपं स्मरणरूपं वा~? नाऽऽद्यः, अनुभवसामग्र्यभावात्, तथा हि - चाक्षुषादिप्रत्यक्षे चक्षुरादिना सह मनः संयोगस्य हेतुत्वात्
तदभावादेव न चाक्षुषादिप्रत्यक्षम्~। ज्ञानादेरभावादेव न मानसं प्रत्यक्षं, ज्ञानाद्यभावे च आत्मनोऽपि न प्रत्यक्षमिति~। व्याप्तिज्ञानाभावादेव नाऽनुमितिः, सादृश्यज्ञानाभावान्नोपमितिः,
पदज्ञानाभावान्न शाब्दबोधः, इत्यनुभवसामग्र्यभावान्नाऽनुभवः~। उद्बोधकाभावाच्च न स्मरणम्~। मैवम्~। सुषुप्तिप्राक्कालोत्पन्नेच्छादिव्यक्तेस्तत्सम्बन्धेनाऽऽत्मनश्च प्रत्यक्षत्वप्रसङ्गात्,
तदतीन्द्रियत्वे मानाभावात्, सुषुप्तिप्राक्काले निर्विकल्पकमेव नियमेन जायते इत्यत्रापि प्रमाणभावात्~। अथ ज्ञानमात्रे त्वङ्मनः संयोगस्य यदि कारणमत्वं, तदा रासन
चक्षुषादिप्रत्यक्षकाले त्वाचप्रत्यक्षं स्यात्, विषयत्वक्संयोगस्य त्वङ्मनःसंयोगस्य च (तत्र) सत्त्वात्, परस्परप्रतिबन्धादेकमपि वा न स्यादिति~। अत्र केचित् - पूर्वोक्तयुक्या
त्वङ्मनःसंयोगस्य ज्ञानहेतुत्वे सिद्धे चाक्षुषादिसामग्न्याः स्पार्शनादिप्रतिबन्धकत्वमनुभवानुरोधात् कल्प्यत इति~। अन्ये तु सुषुप्त्यनुरोधेन चर्ममनःसंयोगस्य ज्ञानहेतुत्वं
कल्प्यते, चाक्षुषादिप्रत्यक्षकाले त्वङ्मनःसंयोगाभावान्न त्वाचप्रत्यक्षमिति॥
॥इति त्वङ्मनःसंयोगस्य ज्ञानहेतुत्वव्यवस्थापनम्॥
मनोग्राह्यं सुखं दुःखमिच्छा द्वेषो मतिः कृतिः॥५७॥
मनोग्राह्यमिति~। मनोजन्यप्रत्यक्षविषय इत्यर्थः~। मतिः-ज्ञानम्~। कृतिः-प्रयत्नः, किन्तु "मनोमात्रस्य गोचरः" इत्यनेन पूर्वमुक्तत्वादत्र नोक्तः॥५७॥
॥इति मनोग्राह्यविषयनिरूपणम्॥
ज्ञानं यन्निर्विकल्पाख्यं तदतीन्द्रियमिष्यते~।
चक्षुःसंयोगाद्यनन्तरं घट इत्याकारकं घटत्वादिविशिष्टविषयकज्ञानं न सम्भवति, पूर्वं विशेषणस्य घटत्वादेज्र्ञानाभावात्, विशिष्टबुद्धौ विशेषणज्ञानस्य कारणत्वात्
~। तथा च प्रथमतो घटघटत्वयोर्वैशिष्ट्यानवगाह्येव ज्ञानं जायते, तदेव निर्विकल्पकम्~। तच्च न प्रत्यक्षम्~। तथा हि-वैशिष्ट्यानवगाहिज्ञानस्य प्रत्यक्षं न भवति, घटमहं
जानामीति प्रत्ययात्, तत्राऽऽत्मनि ज्ञानं प्रकारीभूय भासते, ज्ञाने घटः, तत्र घटत्वं, यः प्रकारः, स एव
विशेषणमित्युच्यते~। विशेषणे यद्विशेषणं तद्विशेषणं तद्विशेषणतावच्छेदकमित्युच्यते~।
विशेषणतावच्छेदकप्रकारकं ज्ञानं विशिष्टवैशिष्ट्यज्ञाने कारणम्~। निर्विकल्पके च घटत्वादिकं न प्रकारः, तेन घटत्वादिविशिष्टघटादिवैशिष्ट्यभानं ज्ञाने न
सम्भवति, घटत्वाद्यप्रकारकं च घटादिविशिष्टज्ञानं न सम्भवति, जात्यखण्डोपाध्यतिरिक्तपदार्थज्ञानस्य किञ्चिद्धर्मप्रकारकत्वनियमात्~।
॥इति निर्विकल्पकस्याऽतीन्द्रियत्वनिरूपणम्॥
महत्त्वं षड्विधे हेतुः, इद्रियं करणं मतम्॥५८॥
महत्त्वमिति~। द्रव्यप्रत्यक्षे महत्त्वं समवायसम्बन्धेन कारणम्~। द्रव्यसमवेतानां गुणकर्मसामान्यानां प्रत्यक्षे स्वाश्रयसमवायसम्बन्धेन, द्रव्यसमवेतानां गुणत्वकर्मत्वादीनां
प्रत्यक्षे स्वाश्रयसमवेतसमवायसम्बन्धेन कारणमिति॥इति महत्त्वस्य प्रत्यक्षहेतुत्वव्यवस्थापनम्॥
इन्द्रियमिति~। अत्राऽपि "षड्विधे" इत्यनुषज्यते~। इन्द्रियत्वं न जातिः, पृथिवीत्वादिना साङ्कर्यात्, किन्तु शब्देतरोद्भूतविशेषगुणानाश्रयत्वे सति
ज्ञानकारणमनःसंयोगाश्रयत्वमिन्द्रियत्वम्~। आत्मादिवारणाय सत्यन्तम्~। उद्भूतविशेषगुणस्य श्रोत्रे सत्त्वाच्छब्देतरेति~। विशेषगुणस्य रूपादेश्चक्षुरादावपि सत्त्वादुद्भूतेति~।
उद्भूतत्वं न जातिः, शुक्लत्वादिना सङ्करात्~। न च शुक्लत्वादिव्याप्यं नानैवोद्भूतत्वमिति वाच्यम्~। उद्भूतरूपत्वादिना चाक्षुषादौ जनकत्वानुपपत्तेः, किन्तु शुक्लत्वादिव्याप्यं
नानैवाऽनुद्भूतत्वं, तदभावकूटश्चोद्भूतत्वम्~। संयोगादावप्यास्ति~। तथा च शब्देतरोद्भूतगुणः संयोगादिः चक्षुरादावप्यस्त्यतो विशेषेति~। कालादिवारणाय विशेष्य दलम्~।
इन्द्रियावयवविषयसंयोगस्याऽपि प्राचां मते प्रत्यक्षजनकत्वादिन्द्रियावयववारणाय, नवीनमते कालादौ रूपाभावप्रत्यक्षे सन्निकर्षघटकतया कारणीभूतचक्षुःसंयोगाश्रयस्य
कालादेर्वारणाय मनःपदम्~। ज्ञानकारणमित्यपि तद्वारणायैव~। करणमिति~। असाधारणं कारणं करणम्~। असाधारणत्वं व्यापारवत्त्वम्॥५८॥
॥इतीन्द्रियनिरूपणम्॥
विषयेन्द्रियसम्बन्धो व्यापारः सोऽपि षड्विधः~।
द्रव्यग्रहस्तु संयोगात् संयुक्तसमवायतः॥५९॥
द्रव्येषु समवेतानां तथा तत्समवायतः~।
तत्राऽपि समवेतानां शब्दस्य समवायतः॥६०॥
तद्वृत्तीनां समवेतसमवायेन तु ग्रहः~।
प्रत्यक्षं समवायस्य विशेषणतया भवेत्॥६१॥
विशेषणतया तद्वदभावानां ग्रहो भवेत्~।
यदि स्यादुपलभ्येतेत्येवं यत्र प्रसज्यते॥६२॥
विषयेन्द्रियसम्बन्ध इति~। व्यापारः-सन्निकर्षः~।
षड्विधं सन्निकर्षमुदाहरणाद्वारा प्रदर्शयति-द्रव्यग्रह इति~। द्रव्यप्रत्यक्षमिन्द्रियसंयोगजन्यम्~। द्रव्यसमवेतप्रत्यक्षमिन्द्रियसंयुक्तसमवायजन्यम्~। एवमग्रेऽपि~।
वस्तुतस्तु द्रव्यचाक्षुषं प्रति चक्षुःसंयोगः कारणं, द्रव्यसमवेतचाक्षुषं प्रति चक्षुःसंयुक्तसमवायः कारणं, द्रव्यसमवेतसमवेतचाक्षुषं प्रति चक्षुःसंयुक्तसमवेत
समवायः, एवमन्यत्राऽपि विशिष्यैव कार्यकारणभावः~। परन्तु पृथिवीपरमाणुनीले नीलत्वं, पृथिवीपरमाणौ पृथिवीत्वं च चक्षुषा कथं न गृह्यते~? तत्र परम्परयोद्भूतरूपसम्बन्धस्य
महत्त्वसम्बन्धस्य च सत्त्वात्~। तथा हि नीलत्वजातिरेकैव घटनीले परमाणुनीले च वर्तते~। तथा च महत्त्वसम्बन्धः घटनीलमादाय वर्तते~। उद्भूतरूपसम्बन्धस्तूभयमादायैव
वर्तते, एवं पृथिवीत्वेऽपि घटादिकमादाय महत्त्वसम्बन्धो बोध्यः~। एवं वायौ तदीयस्पर्शादौ च सत्तायाश्चाक्षुषप्रत्यक्षं स्यात्~। तस्मादुद्भूतरूपावच्छिन्न-महत्त्वावच्छिन्न-
चक्षुःसंयुक्तसमवायस्य द्रव्यसमवेतचाक्षुषे, तादृशचक्षुःसंयुक्तसमवेतसमवायस्य द्रव्यसमवेतसमवेतचाक्षुषे कारणत्वं वाच्यम्~। इत्थं च परमाणुनीलादौ न नीलत्वादिग्रहः,
परमाणौ चक्षुःसंयोगस्य महत्त्वावच्छिन्नत्वाभावात्~। एवं वाय्वादौ न सत्तादिचाक्षुषं, तत्र चक्षुःसंयोगस्य रूपावच्छिन्नत्वाभावात्~। एवं यत्र घटस्य पृष्ठावच्छेदेनाऽऽलोकसंयोगः
चक्षुःसंयोगस्त्वग्रावच्छेदेन, तत्र घटप्रत्यक्षाभावादालोकसंयोगावच्छिन्नत्वं चक्षुःसंयोगे विशेषणं देयम्~।
एवं द्रव्यस्पार्शनप्रत्यक्षे त्वक्संयोगः कारणम्, द्रव्यसमवेतस्पार्शनप्रत्यक्षे त्वक्संयुक्तसमवायः, द्रव्यसमवेतसमवतेस्पार्शनप्रत्यक्षे त्वक्संयुक्तसमवेतसमवायः~। अत्राऽपि
महत्त्वावच्छिन्नत्वमुद्भूतस्पर्शावच्छिन्नत्वं पूर्ववदेव बोध्यम्~।
एवं गन्धप्रत्यक्षे घ्राणसंयुक्तसमवायः, गन्धसमवेतस्य घ्राणजप्रत्यक्षे घ्राणसंयुक्तसमवेतसमवायः~।
रासनप्रत्यक्षे रसनासंयुक्तसमवायः, रससमवेतरासनप्रत्यक्षे रसनासंयुक्तसमवेतसमवायः कारणम्~।
शब्दप्रत्यक्षे श्रोत्रावच्छिन्नसमवायः, शब्दसमवेतश्रावणप्रत्यक्षे श्रोत्रावच्छिन्नसमवेतसमवायः कारणम्~।
अत्र सर्वं प्रत्यक्षं लौकिकं बोध्यम्~। वक्ष्यमाणमलौकिकप्रत्यक्षमिन्द्रियसंयोगादिकं विनाऽपि सम्भवति~।
एवमात्मप्रत्यक्षे मनःसंयोगः, आत्मसमवेतमानसप्रत्यक्षे मनःसंयुक्तसमवायः, आत्मसमवेतसमवेतमानसप्रत्यक्षे मनःसंयुक्तसमवेतसमवायः कारणम्~।
अभावप्रत्यक्षे समवायप्रत्यक्षे चेन्द्रियसम्बद्धविशेषणता हेतुः~। वैशेषिकमते तु समवायो न प्रत्यक्षः~। अत्र यद्यपि विशेषणता नानाविधा, तथा हि-भूतलादौ
घटाद्यभावः संयुक्तविशेषणतया गृह्यते~। सङ्ख्यादौ रूपाद्यभावः संयुक्तसमवेतविशेषणतया, संङ्ख्यात्वादौ रूपाद्यभावः संयुक्तसमवेतसमवेतविशेषणतया, शब्दाभावः
केवलश्रोत्रावच्छिन्नविशेषणतया, कादौ खत्वाद्यभावः श्रोत्रावच्छिन्नसमवेतविशेषणविशेषणतया, एवं कत्वाद्यवच्छिन्नभावे गत्वाभावादिकं श्रोत्रावच्छिन्नविशेषणतया, घटाभावादौ
पटाभावश्चक्षुःसंयुक्तविशेषणविशेषणतया~। एवमन्यत्राऽप्यूह्यं, तथाऽपि विशेषणतात्वेन एकैव सा गण्यते, अन्यथा षोढा सन्निकर्ष इति प्राचां प्रवादो व्याहन्येतेति॥इति
लौकिकसन्निकर्षनिरूपणम्॥
अत्राभावप्रत्यक्षे योग्यानुपलब्धिः कारणम्~। तथा हि-भूतलादौ घटादिज्ञाने जाते घटाभावादिकं न ज्ञायत~। तेनाऽभावोपलम्भे प्रतियोग्युपलम्भाभावः कारणम्~। तत्र
योग्यताऽप्यपेक्षिता~। सा च प्रतियोगिसत्त्वप्रसञ्जनप्रसञ्जितप्रतियोगिकत्वरूपा, तदर्थश्च प्रतियोगिनो घटादेः सत्त्वप्रसक्त्या प्रसञ्जित उपलम्भरूपः प्रतियोगी यस्य
सोऽभावप्रत्यक्षे हेतुः तथाहि-यत्राऽऽलोकसंयोगादिकं वर्तते तत्र यद्यत्र घटः स्यात्तर्हि उपलभ्येतेत्यापादयितुं शक्यते~। तत्र घटाभावादेः प्रत्यक्षं भवति अन्धकारे तु
नाऽऽपादयितुं शक्यते तेन घटाभावादेरन्धकारे न चाक्षुषप्रत्यक्षं स्पार्शनप्रत्यक्षन्तु भवत्येव आलोकसंयोगादिकं विनाऽपि स्पार्शनप्रत्यक्षस्याऽऽपादयितुं शक्यत्वात्~।
गुरुत्वादिकं यदयोग्यं तदभावस्तु न योग्यः तत्र गुरुत्वादिप्रत्यक्षस्याऽऽपादयितुमशक्यत्वात् वायावुद्भूतरूपाभावः; पाषाणे सौरभाभावः, गुडे तिक्ताभावः, वह्नावनुष्णत्वाभावः,
श्रोत्रे शब्दाभावः, आत्मनि सुखाद्यभावः, एवमादयस्तत्तदिन्द्रियैर्गृह्यन्ते तत्तत्प्रत्यक्षस्याऽऽपादयितुं शक्यत्वात्~। संसर्गाभावप्रत्यक्षे प्रतियोगिनो योग्यता, अन्योन्याभावप्रत्यक्षेऽधिकरण
योग्यताऽपेक्षिता अतः स्तम्भादौ पिशाचादिभेदोऽपि चक्षुषा गृह्यते ५९॥॥६०॥६१॥६२॥
अलौकिकस्तु व्यापारस्त्रिविधः परिकीर्तितः~।
सामान्यलक्षणो ज्ञानलक्षणो योगजस्तथा॥६३॥
आसत्तिराश्रयाणां तु सामान्यज्ञानमिष्यते~।
तदिन्द्रियजतद्धर्मबोधसामग्र्यपेक्ष्यते॥६४॥
एवं प्रत्यक्षं लौकिकालौकिकभेदेन द्विविधं, तत्र लौकिकप्रत्यक्षे षोढा सन्निकर्षा वर्णिताः~।
अलौकिकसन्निकर्षस्त्विदानीमुच्यते अलौकिकस्त्विति~। व्यापारः सन्निकर्षः~। सामान्यलक्षण इति~। सामान्यं लक्षणं यस्येत्यर्थः~। तत्र लक्षणपदेन यादौ स्वरूपमुच्यते,
तदा सामान्यस्वरूपा प्रत्यासत्तिरित्यर्थो लभ्यते~। तच्चेन्द्रियसम्बद्धविशेष्यकज्ञानप्रकारीभूतं बोध्यं, तथा-हि यत्रेन्द्रियसंयुक्तो धूमादिः, तद्विशेष्यकं धूम इति ज्ञानं यत्र जातं,
तत्र ज्ञाने धूमत्वं प्रकारः, तत्र धूमत्वेन सन्निकर्षेण धूमा इत्येवं रूपं सकलधूमविषयकं ज्ञानं जायते~। अत्र यदीन्द्रियसम्बद्धप्रकारीभूतमित्येवोच्यते, तदा धूलीपटले
धूमत्वभ्रमानन्तरं सकसधूमविषयकं ज्ञानं न स्यात्, तत्र धूमत्वेन सह इन्द्रियसम्बन्धाभावात्~। मन्मते त्विन्द्रियसम्बद्धं धूलीपटलं, तद्विशेष्यकं धूम इति ज्ञानं, तत्र प्रकारीभूतं
धूमत्वं प्रत्यासत्तिः~। इन्द्रियसम्बन्धश्च लौकिको ग्राह्यः~। इदं च बहिरिन्द्रियस्थले, मानसस्थले तु ज्ञानप्रकारीभूतं सामान्यं प्रत्यासत्तिः॥६३॥
परन्तु समानानां भावः सामान्यम् , तच्च क्वचिन्नित्यं धूमत्वादि, क्वचिच्चाऽनित्यं घटादि~। यत्रैको घटः संयोगेन भूतले समवायेन कपाले वा ज्ञातः, तदनन्तरं
सर्वेषामेव तद्घटवतां भूतलादीनां कपालादीनां वा ज्ञानं भवति, तत्रेदं बोध्यम्~। परन्तु सामान्यं येन सम्बन्धेन ज्ञायते, तेन सम्बन्धेनाऽधिकरणानां प्रत्यासत्तिः~। किन्तु यत्र
तद्घटनाशानन्तरं तद्घटवतः स्मरणं जातं तत्र सामान्यलक्षणया सर्वेषां तद्घटवतां भानं न स्यात्, सामान्यस्य तदानीमभावात्~। किञ्चेन्द्रियसम्बद्धविशेष्यकं घट इति
ज्ञानं यत्र जातं, तत्र परदिने इन्द्रियसम्बन्धं विनाऽपि तादृशज्ञानप्रकारीभूतसामान्यस्य सत्त्वात् तादृशज्ञानं कुतो न जन्यते~? तस्मात् सामान्यविषयकं ज्ञानं प्रत्यासत्तिः, न
तु सामान्यमित्याह - आसत्तिरिति~। आसत्तिः-प्रत्यासत्तिरित्यर्थः~। तथा च "सामान्यलक्षणः" इत्यत्र "लक्षण"शब्दस्य विषयोऽर्थः~। तस्मात् सामान्यविषयकं ज्ञानं प्रत्यासत्तिरित्यर्थो
लभ्यते~। ननु चक्षुःसंयोगादिकं विनाऽपि सामान्यज्ञानं यत्र वर्तते, तत्र सकलघटादीनां चाक्षुषप्रत्यक्षं स्यादत आह-तदिति~। अस्याऽर्थः-यदा बहिरिन्द्रियेण सामान्यलक्षणया
ज्ञानं जननीयं, तदा यत्किञ्चिद्धर्मिणि तत्सामान्यस्य तदिन्द्रियजन्यज्ञानस्य सामग्र्यपेक्षिता~। सा च सामग्री चक्षुःसंयोगालोकसंयोगादिकम्~। तेनाऽन्धकारादौ चक्षुरादिना
तादृशं ज्ञानं न जन्यते॥६४॥
॥इति सामान्यलक्षणप्रत्यासत्तिनिरूपणम्॥
विषयी यस्य तस्यैव व्यापारो ज्ञानलक्षणः~।
ननु ज्ञानलक्षणा प्रत्यासत्तिर्यदि ज्ञानरूपा, सामान्यलक्षणाऽपि ज्ञानरूपा, तदा तयोर्भेदो न स्यादत आह विषयीति~। सामान्यलक्षणाप्रत्यासत्तिर्हि तदाश्रयस्य ज्ञानं
जनयति~। ज्ञानलक्षणाप्रत्यासत्तिस्तु यद्विषयकं ज्ञानं तस्यैव प्रत्यासत्तिः~। अत्राऽयमर्थः-प्रत्यक्षे सन्निकर्षं विना भानं न सम्भवति तथा च सामान्यलक्षणां विना धूमत्वेन
सकलधूमानां, वह्नित्वेन सकलवह्नीनां च भानं कथं भवेत्~? तदर्थं सामान्यलक्षणा स्वीक्रियते~। न च सकलवह्निधूमभानाभावे का क्षतिरिति वाच्यम्~। प्रत्यक्षधूमे
वह्निसम्बन्धस्य गृहीतत्वात् अन्यधूमस्य चाऽनुपस्थितत्वात् धूमो वह्निव्याप्यो न वेति संशयानुपपत्तेः~। मन्मते तु सामान्यलक्षणया सकलधूमोपस्थितौ कालान्तरीयदेशान्तरीयधूमे
वह्निव्याप्यत्वसन्देहः सम्भवति~। न च सामान्यलक्षणास्वीकारे प्रमेयत्वेन सकलप्रमेयज्ञाने जाते सार्वज्ञापत्तिरिति वाच्यम्~। प्रमेयत्वेन सकलप्रमेयज्ञाने जातेऽपि विशिष्य
सकलपदार्थानामज्ञातत्वेन सार्वज्ञ्याभावात्~।
एवं ज्ञानलक्षणाया अस्वीकारे सुरभि चन्दनमिति ज्ञाने सौरभस्य भानं कथं स्थात्~? यद्यपि सामान्यलक्षणयाऽपि सौरभस्य भानं सम्भवति, तथाऽपि सौरभत्वस्य
भानं ज्ञानलक्षणया~। एवं यत्र धूमत्वेन धूलीपटलं ज्ञातं, तत्र धूलीपटलस्याऽनुव्यवसाये भानं ज्ञानलक्षणया~।
॥इति ज्ञानलक्षणप्रत्यासत्तिनिरूपणम्॥
योगजो द्विविधः प्रोक्तो युक्तयुञ्जानमेदतः॥६५॥
युक्तस्य सर्वदा भानं, चिन्तासहकृतोऽपरः~।
योगज इति~। योगाभ्यासजनितो धर्मविशेषः श्रुतिपुराणादिप्रतिपाद्य इत्यर्थः~। युक्तयुञ्जानरूपयोगिद्वैविध्याद्धर्मस्याऽपि द्वैविध्यमिति भावः॥६५॥
युक्तस्येति~। युक्तस्य तावद्योगजधर्मसहायेन मनसा आकाशपरमाण्वादिनिखिलपदार्थगोचरं ज्ञानं सर्वदैव भवितुमर्हति, द्वितीयस्य चिन्ताविशेषोऽपि सहकारीति॥
इति योगजप्रत्यासत्तिनिरूपणम्॥
॥इति श्रीविश्वनाथपञ्चाननभट्टाचार्यविरचितायां न्यायसिद्धान्तमुक्तावल्यां प्रत्यक्षपरिच्छेदः समाप्तः॥
अथानुमानखण्डः[सम्पाद्यताम्]
व्यापारस्तु परामर्शः करणं व्याप्तिधीर्भवेत्॥६६॥
अनुमितिं व्युत्पादयति~। व्यापारस्त्विति~। अनुमायाम्-अनुमितौ व्याप्तिज्ञानं करणं, परमर्शो व्यापारः~।
तथाहि-येन पुरुषेण महानसादौ धूमे वह्निव्याप्तिर्गृहीता, पश्चात् स एव पुरुषः क्वचित् पर्वतादावविच्छिन्नमूलां धूमरेखां पश्यति, तदनन्तरं धूमो वह्निव्याप्य
इत्येवं रूपं व्याप्तिस्मरणं तस्य भवति, पश्चाच्च वह्निव्याप्यधूमवानयमिति ज्ञानं, स एव परामर्श इत्युच्यते~। तदनन्तरं पर्वतो वह्निमानित्यनुमितिर्जायते॥
इत्यनुमितिकारणनिरूपणम्॥
अनुमायां ज्ञायमानं लिङ्गं तु करणं न हि~।
अनागतादिलिङ्गेन न स्यादनुमितिस्तदा॥६७॥
व्याप्यस्य पक्षवृत्तित्वधीः परामर्श उच्यते॥
अत्र प्राचीनास्तु व्याप्यत्वेन ज्ञायमानं लिङ्गमनुमितिकरणमिति वदन्ति~। तत् दूषयति--ज्ञायमानमिति~। लिङ्गस्याऽनुमित्यकरणत्वे युक्तिमाह--अनागतादीति~।
यद्यनुमितौ लिङ्गं करणं स्यात्, तदाऽनागतेन लिङ्गेन, विनष्टेन चानुमितिर्न स्यात्, अनुमितिकरणस्य लिङ्गस्य तदानीमभावादिति॥६६॥॥६७॥इति
ज्ञायमानलिङ्गस्याऽनुमितिहेतुत्वखण्डनम्॥
व्याप्यस्येति~। व्याप्तिविशिष्टस्य पक्षेण सह वैशिष्ट्यावगाहिज्ञानमनुमितौ जनकं, तच्च पक्षे व्याप्य इति ज्ञानं पक्षो व्याप्यवानिति ज्ञानं वा~। अनुमितिस्तु पक्षे
व्याप्य इति ज्ञानात् पक्षे साध्यमित्याकारिका, पक्षो व्याप्यवानिति ज्ञानात् पक्षः साध्यवानित्याकारिका~।
द्विविधादपि परामर्शात् पक्षः साध्यवानित्येवाऽनुमितिरित्यन्ये~।
ननु वह्निव्याप्यधूमवान् पर्वत इति ज्ञानं विनाऽपि यत्र पर्वतो धूमवानिति प्रत्यक्षं, ततो वह्निव्याप्यो धूम इति स्मरणं, तत्र ज्ञानद्वयादेवाऽनुमितेर्दर्शनात्
व्याप्तिविशिष्टवैशिष्ट्यावगाहिज्ञानं न सर्वत्र कारणं, किन्तु व्याप्यतावच्छेदकप्रकारकपक्षधर्मताज्ञानत्वेन कारणत्वस्याऽऽवश्यकत्वात् तत्र विशिष्टज्ञानकल्पने गौरवाच्चेति
चेन्न~। व्याप्यतावच्छेदकाज्ञानेऽपि वह्निव्याप्यवानिति ज्ञानादनुमित्युत्पत्तेः लाघवाच्च व्याप्तिप्रकारकपक्षधर्मताज्ञानत्वेन हेतुत्वम्~।
किञ्च धूमवान् पर्वत इति ज्ञानादनुमित्यापत्तिः, व्याप्यतावच्छेदकीभूतधूमत्वप्रकारकस्य पक्षधर्मताज्ञानस्य सत्त्वात्~।
न च गृह्यमाणव्याप्यतावच्छेदकप्रकारकपक्षधर्मताज्ञानस्य हेतुत्वमिति वाच्यम्~। चैत्रस्य व्याप्तिग्रहे मैत्रस्य पक्षधर्मताज्ञानात् अनुमितिः स्यात्~।
यदि तत्पुरुषीयगृह्यमाणव्याप्यतावच्छेदकप्रकारकं तत्पुरुषीयपक्षधर्मताज्ञानं तत्पुरुषीयानुमितौ हेतुरित्युच्यते, तदाऽनन्तकार्यकारणभावः~। मन्मते तु समवायसम्बन्धेन
व्याप्तिप्रकारकपक्षधर्मताज्ञानं समवायसम्बन्धेनाऽनुमिति जनयतीति नाऽनन्तकार्यकारणभावः~।
यदि तु व्याप्तिप्रकारकं ज्ञानं पक्षधर्मताज्ञानं च स्वतन्त्रं कारणमित्युच्यते तदा कार्यकारणभावद्वयं, वह्निव्याप्यो धूम आलोकवांश्च पर्वत इति ज्ञानादप्यनुमितिः
स्यात्~। इत्थं च यत्र ज्ञानद्वयं, तत्राऽपि विशिष्टज्ञानं कल्पनीयं, फलमुखगौरवस्याऽदोषत्वात्॥इति परामर्शनिरूपणम्॥
व्याप्तिः साध्यवदन्यस्मिन्नसम्बन्ध उदाहृतः॥६८॥
व्याप्यो नाम व्याप्त्याश्रयः, तत्र व्याप्तिः केत्यत आह व्याप्तिरिति~। वह्निमान् धूमादित्यादौ साध्यो वह्निः, साध्यवान् महानसादिः, तदन्यो जलह्रदादिः, तद्वृृत्तित्वं
धूमस्येति लक्षणसमन्वयः पिण्डादौ वह्नेः सत्त्वान्नाऽतिव्याप्तिः~।
अत्र येन सम्बन्धेन साध्यं, तेन सम्बन्धेन साध्यवान् बोध्यः, अन्यथा समवायसम्बन्धेन वह्निमान् वह्नेरवयवः, तदन्यो महानसादिः, तत्र धूमस्य
विद्यमनत्वादव्याप्तिप्रसङ्गात्~।
साध्यवदन्यश्च साध्यवत्त्वावच्छिन्नप्रतियोगिताकभेदवान् बोध्यः~। तेन यत्किञ्चिद्वह्निमतो महानसादेर्भिन्ने पर्वतादौ धूमसत्त्वेऽपि न क्षतिः~।
येन सम्बन्धेन हेतुता तेनैव सम्बन्धेन साध्यवदन्यावृत्तित्वं बाध्यम्~। तेन साध्यवदन्यस्मिन् धूमावयवे धूमस्य समवायसम्बन्धेन सत्त्वेऽपि न क्षतिः~।
साध्यवदन्यावृत्तित्वं च साध्यवदन्यवृत्तित्वावच्छिन्नप्रतियोगिताकाभावः, तेन धूमवान् वह्नेरित्यत्र साध्यवदन्यजलह्रदादिवृत्तित्वाभावेऽपि नाऽतिव्याप्तिः~।
अत्र यद्यपि द्रव्यं गुणकर्मान्यत्वविशिष्टसत्त्वादित्यादौ विशिष्टसत्तायाः शुद्धसत्तायाश्चैक्यात् साध्यवदन्यस्मिन् गुणादाववृत्तित्वं नाऽस्ति, तथाऽपि
हेतुतावच्छेदकरूपेणाऽवृत्तित्वं वाच्यं, हेतुतावच्छेदकं वृत्तितानवच्छेदकमिति फलिताऽर्थः॥६८॥इति व्याप्तिनिरूपणे पूर्वपक्षग्रन्थः॥
अथवा हेतुमन्निष्ठविरहाप्रतियोगिना~।
साध्येन हेतोरैकाधिकरण्यं व्याप्तिरुच्यते॥६९॥
ननु केवलान्वयिनि ज्ञेयत्वादौ साध्ये साध्यवदन्यस्याऽप्रसिद्धत्वादव्याप्तिः~। किञ्च सत्तावान् जातेरित्यादौ साध्यवदन्यस्मिन् सामान्यादौ हेतुतावच्छेदकसम्बन्धेन
समवायेन वृत्तेरप्रसिद्धत्वादत आह-अथ वेति~। हेतुमति निष्ठा-वृत्तिर्यस्य स तथा विरहः-अभावः~। तथा च हेत्वधिकरणवृत्तिर्योऽभावस्तदप्रतियोगिना साध्येन सह हेतोः
सामानाधिकरण्यं व्याप्तिरुच्यते~।
अत्र यद्यपि वह्निमान् धूमादित्यादौ हेत्वधिकरणपर्वतादिवृत्त्यभावप्रतियोगित्वं तत्तद्वह्न्यादेरस्तीत्यव्याप्तिः~। न च समानाधिकरणवह्निधूमयोरेव व्याप्तिरिति
वाच्यम्~। तत्तद्वह्न्यादेरप्युभयाभावसत्त्वात्, एकसत्त्वेऽपि द्वयं नाऽस्तीति प्रतीतेः~। गुणवान् द्रव्यत्वादित्यादावव्याप्तिश्च~।
तथाऽपि प्रतियोगितानवच्छेदकं यत् साध्यतावच्छेदकं तदवच्छिन्नसामानाधिकरण्यं व्याप्तिरिति वाच्यम्~।
ननु रूपत्वव्याप्यजातिमत्त्वान् पृथिवीत्वादित्यादौ साध्यतावच्छेदिका रूपत्वव्याप्यजातयः, तासां च शुक्लत्वादिस्वरूपाणां
नीलघटादिवृत्त्यभावप्रतियोगितावच्छेदकत्वमस्तीत्यव्यप्तिरिति चेन्न~।
तत्र परम्परया रूपत्वव्याप्यजातित्वस्यैव साध्यतावच्छेदकत्वात्~। न हि तादृशधर्मावच्छिन्नाभावः काऽपि पृथिव्यामस्ति, रूपत्वव्याप्यजातिमान्नास्तीति बुद्ध्यापत्तेः~।
साध्यादिभेदेन व्याप्र्तेर्भेदात् तादृशस्थले साध्यतावच्छेदकतावच्छेदकं प्रतियोगितावच्छेदकतानवच्छेदकमित्येव लक्षणघटकमित्यपि वदन्ति~।
हेत्वधिकरणं हेतुतावच्छेदकविशिष्टाधिकरणं वाच्यम्~। तेन द्रव्यं गुणकर्मान्यत्वविशिष्टसत्त्वादित्यादौ शुद्धसत्ताधिकरणगुणादिनिष्ठाभावप्रतियोगित्वेऽपि द्रव्यत्वस्य
नाऽव्याप्तिः~।
हेतुतावच्छेदकसम्बन्धेन हेत्वधिकरणं बोध्यम्~। तेन समवायेन धूमाधिकरणतदवयवनिष्ठाभावप्रतियोगित्वेऽपि वह्नेर्नाऽव्याप्तिः~।
अभावश्च प्रतियोगिव्यधिकरणो बोध्यः~। तेन कपिसंयोग्येतद्वृक्षत्वादित्यादौ मूलावच्छेदेनैतद्वृक्षवृत्तिकपिसंयोगाभावप्रतियोगित्वेऽपि कपिसंयोगस्य नाऽव्याप्तिः~।
न च प्रतियोगिव्यधिकरणत्वं यदि प्रतियोग्यनधिकरणवृत्तित्वं, तदा तथैवाऽव्याप्तिः, प्रतियोगिनः कपिसंयोगस्याऽनधिकरणे गुणादौ वर्तमानो योऽभावस्तस्यैव
वृक्षे मूलावच्छेदेन सत्त्वात्~। यदि तु प्रतियोग्यधिकरणावृत्तित्वं, तदा संयोगी सत्त्वादित्यादावतिव्याप्तिः, सत्ताधिकरणे गुणादौ यः संयोगाभावस्तस्य प्रतियोग्यधिकरणद्रव्यवृत्तित्वादिति
वाच्यम्~। हेत्वधिकरणे प्रतियोग्यनधिकरणवृत्तित्वविशिष्टस्य विवक्षितत्वात्~। स्वप्रतियोग्यनधिकरणीभूतहेत्वधिकरणवृत्त्यभावेति निष्कर्षः~।
प्रतियोग्यनधिकरणत्वं च प्रतियोगितावच्छेदकावच्छिन्नानधिकरणत्वं वाच्यम्~। तेन विशिष्टसत्तावान् जातेरित्यादौ जात्यधिकरणगुणादौ
विशिष्टसत्ताभावप्रतियोगिसत्ताधिकरणत्वसत्त्वेऽपि न क्षतिः~।
अत्र साध्यतावच्छेदकसम्बन्धेन प्रतियोग्यनधिकरणत्वं बाध्यम्, तेन ज्ञानवान् सत्त्वादित्यादौ सत्ताधिकरणघटादेर्विषयतया ज्ञानाधिकरणत्वेपि न क्षतिः~।
इत्थञ्च वह्निमान् धूमादित्यादौ धूमाधिकरणे समवायेन वह्निविरहसत्त्वेऽपि न क्षतिः~।
ननु प्रतियोगितावच्छेदकावच्छिन्नस्य यस्य कस्यचित्प्रतियोगिनोऽनधिकरणत्वं, तत्सामान्यस्य वा, यत्किञ्चिच्प्रतियोगितावच्छेदकावच्छिन्नानधिकरणत्वं वा विवक्षितम्~?
आद्ये कपिसंयोगी एतद्वृक्षत्वादित्यत्र तथेवाऽव्याप्तिः, कपिसंयोगाभावप्रतियोगितावच्छेदकावच्छिन्नो वृक्षावृत्तिकपिसंयोगोऽपि भवति, तदनधिकरणं वृक्ष इति~।
द्वितीये तु प्रतियोगिव्यधिकरणाभावाप्रसिद्धिः, सर्वस्यैवाऽभावस्य पूर्वक्षणवृत्तित्वविशिष्टस्वाभावात्मकप्रतियोगिसमानाधिकरणत्वात्~। न च वह्निमान् धूमादित्यादौ
घटाभावादेः पूर्वक्षणवृत्तित्वविशिष्टस्वाभावात्मकप्रतियोग्यनधिकरणत्वं यद्यपि पर्वतादेः, तथापि साध्यतावच्छेदकसम्बन्धेन तत्प्रतियोग्यनधिकरणत्वमस्त्येवेति कथं
प्रतियोगिव्यधिकरणाभावाप्रसिद्धिरिति वाच्यम्~। घटाभावे यो वह्न्यभावः, तस्य घटाभावात्मकतया घटाभावस्य वह्निरपि प्रतियोगी, तदधिकरणं च पर्वतादिरित्येवंक्रमेण
प्रतियोगिव्यधिकरणस्याऽप्रसिद्धत्वात्~। यदि च घटाभावादौ वन्ह्यभावादिर्भिन्न इत्युच्यते, तथाऽपि धूमाभाववान् वह्न्यभावादित्यादावव्याप्तिः, तत्र साध्यतावच्छेदकसम्बन्धः
स्वरूपसम्बन्धः, तेन च सम्बन्धेन सर्वस्यैवाऽभावस्य पूर्वक्षणवृत्तित्वविशिष्टस्वाभावात्मकप्रतियोग्यधिकरणत्वं हेत्वधिकरणस्येति~।
तृतीये तु कपिसंयोगाभाववानात्मत्वादित्यादावव्याप्तिः, तत्राऽऽत्मवृत्तिकपिसंयोगाभावाभावः कपिसंयोगः, तस्य च गुणत्वात् तत्प्रतियोगितावच्छेदकं
गुणसामान्याभावत्वमपि, तदवच्छिन्नानधिकरणत्वं हेत्वधिकरणस्याऽऽत्मन इति~।
मैवम्~। यादृशप्रतियोगितावच्छेदकावच्छिन्नानधिकरणत्वं हेतुमतः, तादृशप्रतियोगितानवच्छेदकत्वस्य विवक्षितत्वात्~।
ननु कालो घटवान् कालपरिमाणादित्यत्र प्रतियोगिव्यधिकरणाभावाप्रसिद्धिः, हेत्वधिकरणस्य महाकालस्य जगदाधारतया सर्वेषामेवाऽभावानां साध्यतावच्छेदकसम्बन्धेन
कालिकविशेषणतया प्रतियोग्यधिकरणत्वात्~।
अत्र केचित्~। महाकालभेदविशिष्टघटाभावस्तत्र प्रतियोगिव्यधिकरणः, महाकालस्य घटाधारत्वेऽपि महाकालभेदविशिष्टघटानाधारत्वात्, महाकाले महाकालभेदाभावात्~।
वस्तुतस्तु प्रतियोगितावच्छेदकसम्बन्धेन प्रतियोग्यनधिकरणीभूतहेत्वधिकरणवृत्त्यभावप्रतियोगितासामान्ये यत्सम्बन्धावच्छिन्नत्व-यद्धर्मावाच्छिन्नत्वोभयाभावः, तेन
सम्बन्धेन तद्धर्मावच्छिन्नस्य तद्धेतुव्यापकत्वं बोध्यम्~। इत्थं च कालो घटवान् कालपरिमाणादित्यादौ संयोगसम्बन्धेन घटाभावप्रतियोगिनोऽपि घटस्याऽनधिकरणे
हेत्वधिकरणे महाकाले वर्तमानः स एव संयोगेन घटाभावः, तस्य प्रतियोगितायां कालिकसम्बन्धावच्छिन्नत्व - घटत्वावच्छिन्नत्वोभयाभावसत्त्वान्नाऽव्याप्तिः~।
ननु प्रमेयवह्निमान् धूमादित्यादौ प्रमेयवह्नित्वावच्छिन्नत्वमप्रसिद्धं, गुरुधर्मस्याऽनवच्छेदकत्वादिति चेन्न~। कम्बुग्रीवादिमान्नाऽस्तीति प्रतीत्या
कम्बुग्रीवादिमत्त्वावच्छिन्नप्रतियोगिताविषयीकरणेन गुरुधर्मस्याऽप्यवच्छेदकत्वस्वीकारादिति सङ्क्षेपः॥६६॥इति व्याप्तिनिरूपणे सिद्धान्तग्रन्थः॥
सिषाधयिषया शून्या सिद्धिर्यत्र न तिष्ठति~।
स पक्षस्तत्र वृत्तित्वज्ञानादनुमितिर्भवेत्॥७०॥
पक्षवृत्तित्वमित्यत्र पक्षत्वं किं~? तदाह - सिषाधयिषयेति~।
सिषाधयिषाविरहविशिष्टसिद्ध्यभावः पक्षता, विनाऽपि सिषाधयिषां घनगर्जितेन मेघानुमानात्~। अत एव साध्यसन्देहोऽपि न पक्षता, विनाऽपि साध्यसन्देहं
तदनुमानात्~। सिद्धौ सत्यामपि सिषाधयिषासत्त्वेऽनुमितिर्भवत्येवा अतः सिषाधयिषाविरहविशिष्टत्वं सिद्धौ विशेषणम्~। तथा च यत्र सिद्धिर्नाऽस्ति, तत्र सिषाधयिषायां
सत्यामसत्यामपि पक्षता~। यत्र सिषाधयिषाऽस्ति, तत्र सिद्धौ सत्यामसत्यामपि पक्षता~। यत्र सिद्धिरस्ति, सिषाधयिषा च नाऽस्ति, तत्र न पक्षता, सिषाधयिषाविरहविशिष्टसिद्धेः
सत्त्वात्~।
ननु यत्र परामर्शानन्तरं सिद्धिः, ततः सिषाधयिषा, तत्र सिषाधयिषाकाले परामर्शनाशान्नाऽनुमितिः, यत्र सिद्धि-परामर्श-सिषाधयिषाः क्रमेण भवन्ति, तत्र
सिषाधयिषाकाले सिद्धेर्नाशात् प्रतिबन्धकाभावादेवाऽनुमितिः, यत्र सिषाधयिषा-सिद्धि-परामर्शास्तत्र परामर्शकाले सिषाधयिषैेव नाऽस्ति, एवमन्यत्रापि सिद्धिकाले परामर्शकाले
च न सिषाधयिषा, योग्यविभुविशेषगुणानां यौगपद्यनिषेधात्, तत् कथं सिषाधयिषाविरहविशिष्टत्वं सिद्धेर्विशेषणमिति चेन्न~।
यत्र वह्निव्याप्यधूमवान् पर्वतो वह्निमानिति प्रत्यक्षं, स्मरणं वा ततः सिषाधयिषा, तत्र पक्षतासम्पत्तये तद्विशेषणस्याऽऽवश्यकत्वात्~।
इदं तु बोध्यम्~। यादृशयादृशसिषाधयिषासत्त्वे सिद्धिसत्त्वे सल्लिङ्गकानुमितिः, तादृशसिषाविरहविशिष्टसिद्ध्यभावः यल्लिङ्गकानुमितौ पक्षता~। तेन सिद्धिपरामर्शसत्त्वे
यत्किञ्चिज्ज्ञानं जायतामितीच्छायामपि नाऽनुमितिः, वह्निव्याप्यधूमवान् पर्वतो वह्निमानितिप्रत्यक्षसत्त्वे प्रत्यक्षातिरिक्तं ज्ञानं जायतामितीच्छायां तु भवत्येव~। एवं
धूमपरामर्शसत्त्वे आलोकेन वह्निमनुमिनुयामितीच्छायामपि नाऽनुमितिः~।
सिषाधयिषाविरहकाले यादृशसिद्धिसत्त्वे नाऽनुमितिस्तादृशी सिद्धिर्विशिष्यैव तत्तदनुमितिप्रतिबन्धिका वक्तव्या~। तेन पर्वतस्तेजस्वी, पाषाणमयो
वह्निमानितिज्ञानसत्त्वेऽप्यनुमितेर्न विरोधः~। परं तु पक्षतावच्छेदकसामानाधिकरण्येन साध्यसिद्धावपि तदवच्छेदेनाऽनुमितेर्दर्शनात् पक्षतावच्छेदकावच्छेदेनाऽनुमितिं प्रति
पक्षतावच्छेदकावच्छेदेन साध्यसिद्धिरेव प्रतिबन्धिका, पक्षतावच्छेदकसामानाधिकरण्येनाऽनुमितिं प्रति तु सिद्धिमात्रं विरोधि~।
इदं तु बोध्यम्~। यत्राऽयं पुरुषो न वेति संशयानन्तरं पुरुषत्वव्याप्यकरादिमानयमिति ज्ञानं, तत्राऽसत्यामनुमित्सायां पुरुषस्य प्रत्यक्षं भवति, न त्वनुमितिः~।
अतोऽनुमित्साविरहविशिष्टसमानविषयकप्रत्यक्षसामग्री कामिनीजिज्ञासादिवत् स्वातन्त्र्येण प्रतिबन्धिका, एवं परामर्शानन्तरं विना प्रत्यक्षेच्छां पक्षादेः प्रत्यक्षानुत्पत्तेः
प्रत्यक्षेच्छाविरहविशिष्टानुमितिसामग्री भिन्नविषयकप्रत्यक्षे प्रतिबन्धिका॥७०॥इति पक्षतानिरूपणम्~।
अनैकान्तो विरुद्धश्चाऽप्यसिद्धः प्रतिपक्षितः~।
कालात्ययापदिष्टश्च हेत्वाभासास्तु पञ्चधा॥७१॥
हेतुप्रसङ्गाद्धेत्वाभासान् विभजते अनैकान्त इत्यादि~। तल्लक्षणं तु यद्विषयकत्वेन ज्ञानस्याऽनुमितिविरोधित्वं तत्त्वम्~। तथाहि-व्यभिचारादिविषयकत्वेन
ज्ञानस्याऽनुमितिविरोधित्वात्ते दोषाः~।
यद्विषयकत्वं च यादृशविशिष्टविषयकत्वम्~। तेन बाधभ्रमस्याऽनुमितिविरोधित्वेऽपि न क्षतिः~। तत्र पर्वतो वह्न्यभाववानिति विशिष्टस्याऽप्रसिद्धत्वान्न हेतुदोषः~।
न च वह्न्यभावव्याप्यपाषाणमयत्ववानिति परामर्शकाले वह्निव्याप्यधूमस्याऽऽभासत्वं न स्यात्, तत्र वह्न्यभावव्याप्यवान् पक्ष इति विशिष्टस्याऽप्रसिद्धत्वादिति
वाच्यम्~। इष्टापत्तेः~। अन्यथा बाधस्याऽनित्यदोषत्वापत्तेः~। तस्मात्तत्र वह्न्यभावव्याप्यपाषाणमयत्ववानिति परामर्शकाले वह्निव्याप्यधूमस्य नाऽऽभासत्वं, भ्रमादनुमितिप्रतिबन्धमात्रं,
हेतुस्तु न दुष्टः~। इत्थं च साध्याभाववद्वृत्तिहेतुत्वादिकं दोषः~। तद्वत्त्वं च हेतौ येन केनापि सम्बन्धेनेति नव्याः~।
परे तु यद्विषयकत्वेन ज्ञानस्याऽनुमितिविरोधित्वं, तद्वत्त्वं हेत्वाभासत्वम् , सत्प्रतिपक्षे विरोधिव्याप्त्यादिकमेव तथा~। तद्वत्त्वं च हेतोज्र्ञानरूपसम्बन्धेन~। न चैवं
वह्निमान् धूमादित्यादौ पक्षे बाधभ्रमस्य साध्याभावविषयकत्वेनाऽनुमितिविरोधित्वात् ज्ञानरूपसम्बन्धेन तद्वत्त्वास्याऽपि सत्त्वात् सद्धेतोरपि बाधितत्वापत्तिरिति वाच्यम्~।
तत्र ज्ञानस्य सम्बन्धत्वाकल्पनात्~। अत्र सत्प्रतिपक्षित इति व्यवहारेण तत्कल्पनात् , तत्र बाधित इति व्यवहाराभावादित्याहुः~।
अनुमितिविरोधित्वं चाऽनुमितितत्करणान्यतरविरोधित्वम्~। तेन व्यभिचारिणि नाऽव्याप्तिः~।
दोषज्ञानं च यद्धेतुविषयकं, तद्धेतुकानुमितौ प्रातबन्धकम्~। तेनैकहेतौ व्यभिचारज्ञाने हेत्वन्तरेणाऽनुमित्युत्पत्तेः, तदभावाद्यनवगाहित्वाच्च
व्यभिचारज्ञानस्याऽनुमितिविरोधित्वाभावेऽपि न क्षतिरिति सङ्क्षेपः~।
यादृशसाध्यपक्षहेतौ यावन्तो दोषास्तावदन्यान्यत्वं तत्र हेत्वाभासत्वम्~। पञ्चत्वकथनं तु तत्सम्भवस्थलाभिप्रायेण~।
एवं च साधारणाद्यन्यतमत्वमनैकान्तिकत्वम्~।
साधारणः साध्यवदन्यवृत्तिः, तेन च व्याप्तिज्ञानप्रतिबन्धः क्रियते~।
असाधारणः साध्यासमानाधिकरणो हेतुः~। तेन साध्यसामानाधिकरण्यग्रहः प्रतिबद्ध्यते~। (यथा शब्दो नित्यः शब्दत्वादित्यादावसाधारण्यं, शब्दोऽनित्यः शब्दत्वादित्यादौ
त्वसाधारण्यभ्रमः~।)
अन्ये तु सपक्षावृत्तिरसाधारणः~। सपक्षश्च निश्चितसाध्यवान्~। तथाच शब्दोऽनित्यः शब्दत्वादित्यादौ यदा पक्षे साध्यनिश्चयस्तदा नाऽसाधारण्यं, तत्र हेतोर्निश्चयादिति
वदन्ति~।
अनुपसंहारी च अत्यन्ताभावाप्रतियोगिसाध्यकादिः~। अनेन व्यतिरेकव्याप्तिज्ञानप्रतिबन्धः क्रियते~।
विरुद्धस्तु साध्यव्यापकीभूताभावप्रतियोगी~। अयं साध्याभावग्रहसामग्रीत्वेन प्रतिबन्धकः~। सत्प्रतिपक्षे तु प्रतिहेतुः साध्याभावसाधकः अत्र तु हेतुरेवेति विशेषः~।
साध्याभावसाधकः एव हेतुः साध्यसाधकत्वेनोपन्यस्त इत्यशक्तिविशेषोपस्थापकत्वाच्च विशेषः~।
सत्प्रतिपक्षः साध्याभावव्याप्यवान् पक्षः~।
अगृहीताप्रामाण्यकसाध्यव्याप्यवत्त्वोपस्थितिकालीनागृहीताप्रामाण्यकतदभावव्याप्य वत्त्वोपस्थितिविषयस्तथेत्यन्ये~। अत्र च परस्पराभावव्याप्यवत्ताज्ञानात्
परस्परानुमितिप्रतिबन्धःफलम्~।
अत्र केचित्~। यथा घटाभावव्याप्यवत्ताज्ञानेऽपि घटचक्षुःसंगोगे सति घटवत्ताज्ञानं जायते, यथा च शङ्खे सत्यपि पीतत्वाभावव्याप्यशङ्खत्ववत्ताज्ञाने सति
पित्तादिदोषे पीतः शङ्ख इति धीः, एवं कोटिद्वयव्याप्यदर्शनेऽपि कोटिद्वयस्य प्रत्यक्षरूपः संशयो भवति तथा सत्प्रतिपक्षस्थले संशयरूपाऽनुमितिर्भवत्येव~। यत्र
चैककोटिव्याप्यदर्शनं तत्राऽधिकबलतया द्वितीयकोटिभानप्रतिबन्धान्न संशयः~। फलबलेन चाऽधिकबल-समबलभावः कल्प्यत इत्याहुः~।
तन्न~। तदभावव्याप्यवत्ताज्ञाने सति तदुपनीतभानविशेषशाब्दबोधादेरनुदयाल्लौकिकसन्निकर्षाजन्यदोषविशेषाजन्यज्ञानमात्रे तस्य प्रतिबन्धकता, लाघवात्, न
तूपनीतभानविशेषे शाब्दबोधे च पृथक् प्रतिबन्धकता, गौरवात्~। तथा च प्रतिबन्धकसत्त्वात् कथमनुमितिः~? न हि लौकिकसन्निकर्षस्थले प्रत्यक्षमिव सत्प्रतिपक्षस्थले
संशयाकाराऽनुमितिः प्रामाणिकी, येनाऽनुमितिभिन्नत्वेनाऽपि विशेषणीयम्~। यत्र च कोटिद्वयव्याप्यवत्ताज्ञानं, तत्रोभयत्राऽप्रामाण्यज्ञानात् संशयः, नाऽन्यथा, अगृहीताप्रामाण्यकस्यैव
विरोधिज्ञानस्य प्रतिबन्धकत्वादिति~।
असिद्धिस्तु आश्रयासिद्धाद्यन्यतमत्वम्~।
आश्रयासिद्धिः पक्षे पक्षतावच्छेदकस्याभावः, यत्र काञ्चनमयः पर्वतो बह्निमानिति साध्यते, तत्र पर्वतो न काञ्चनमय इति ज्ञाने विद्यमाने काञ्चनमये पर्वते
परामर्शप्रतिबन्धः फलम्~। तद्धर्मिकतदभावनिश्चये लौकिकसन्निकर्षाजन्यदोषविशेषाजन्यतद्धर्मिक-तज्ज्ञानमात्रे विरोधीति~।
न तु संशयसाधारणं पक्षे साध्यसंसृष्टत्वज्ञानमनुमितिकारणं, तद्विरोधितया च बाधसत्प्रतिपक्षयोर्हेत्वाभावत्वमिति युक्तम्~। अप्रसिद्धसाध्यकानुमित्यनापत्तेः,
साध्यसंशयादिकं विनाऽप्यनुमित्युत्पत्तेश्च~।
एवं साध्याभावज्ञाने प्रमात्वज्ञानमपि न प्रतिबन्धकं, मानाभावात्, गौरवाच्च~। अन्यथा सत्प्रतिपक्षादावपि तदभावव्याप्यवत्ताज्ञाने प्रमात्वविषयकत्वेन प्रतिबन्धकतापत्तेः,
किन्तु भ्रमत्वज्ञानानास्कन्दितबाधबुद्धेः प्रतिबन्धकता, तत्र भ्रमत्वशङ्काविघटनेन प्रामाण्यज्ञानं क्वचिदुपयुज्यते~।
न च बाधस्थले पक्षे हेतुसत्त्वेव्यभिचारः, पक्षे हेत्वभावे स्वरूपासिद्धिरेव दोष इति वाच्यम्~। बाधज्ञानस्य व्यभिचारज्ञानादेर्भेदात्~।
किं च यत्र परामर्शानन्तरं बाधबुद्धिः, तत्र व्यभिचारज्ञानादेरकिञ्चित्करत्वात्बाधस्याऽनुमितिप्रतिबन्धकत्वं वाच्यम्~। एवं यत्रोत्पत्तिक्षणावच्छिन्ने घटादौ
गन्धव्याप्यपृथिवीेत्ववत्ताज्ञानं तत्र बाधस्यैव प्रतिबन्धकत्वं वाच्यम्~। न च पक्षे घटे गन्धसत्त्वात् कथं बाध इति वाच्यम्~। पक्षतावच्छेदकदेशकालावच्छेदेनाऽनुमितेरनुभवसिद्धत्वादिति~।
बाधतद्व्याप्यभिन्ना ये हेत्वाभासाः, तद्व्याप्या अपि तन्मध्य एवान्तर्भवन्ति~। अन्यथा हेत्वाभासाधिक्यप्रसङ्गात्~। बाधव्याप्यसत्प्रतिपक्षो भिन्न एव, स्वतन्त्रेच्छेन मुनिना
पृथगुपदेशात्~। सत्प्रतिपक्षव्याप्यस्तु न प्रतिबन्धक इति प्रघट्टकार्थः॥७३॥
॥इति हेत्वाभाससामान्यनिरूपणम्॥
आद्यः साधारणस्तु स्यादसाधारणकोऽपरः~।
तथैवाऽनुपसंहारी त्रिधाऽनैकान्तिको भवेत्॥७२॥
यः सपक्षे विपक्षे च भवेत्साधारणस्तु सः~।
यस्तूभयस्माद्वयावृत्तः सत्त्वसाधारणो मतः॥७३॥
यः सपक्ष इति~। सपक्ष-विपक्ष-वृत्तिः साधारण इत्यर्थः~। सपक्षःनिश्चितसाध्यवान्~। विपक्षः-साध्यवद्भिन्नः~। विरुद्धवारणाय सपक्षवृत्तित्वमुक्तम्~।
वस्तुतो विपक्षवृत्तित्वमेव वाच्यं, विरुद्धस्य साधारणत्वेऽपि दूषकताबीजस्य भिन्नतया तस्य पार्थक्यात्॥इति साधारणनिरूपणम्~।
यस्तूभयस्मादिति~। सपक्ष-विपक्ष-व्यावृत्त इत्यर्थः~। सपक्षः-साध्यवत्तया निश्चितः~। विपक्षः - साध्यशून्यतया निश्चितः~। शब्दोऽनित्यः शब्दत्वादित्यादौ यदा
शब्देऽनित्यत्वस्य सन्देहः, तदा सपक्षत्वं विपक्षत्वं च घटादीनामेव, तदव्यावृत्तं च शब्दत्वमिति तदा तदसाधारणम्~। यदा तु शब्देऽअनित्यत्वनिश्चयः, तदा नाऽसाधारण्यम्~।
इदं तु प्राचां मतम्~। नवीनमतं तु पूर्वमुक्तम्॥७२-७३॥इत्यसाधारणनिरूपणम्॥
तथैवाऽनुपसंहारी केवलान्वयिपक्षकः~।
केवलान्वयीति~। सर्वमभिधेयं प्रमेयत्वादित्यादौ सर्वस्यैव पक्षत्वात् सामानाधिकरण्यग्रहस्थलान्तराभावान्नाऽनुमितिः~। इदं तु न सम्यक्, पक्षैकदेशे सहचारग्रहेऽपि
क्षतेरभावात्~। अस्तु वा सहचाराग्रहः, तावताऽप्यज्ञानरूपाऽसिद्धिरेव, न तु हेत्वाभासत्वं तस्य, तथाऽपि केवलान्वयिसाध्यकत्वं तत्त्वमित्युक्तम्॥
॥इत्यनुपसंहारिनिरूपणम्॥
यः साध्यवतीति~।
यः साध्यवति नैवाऽस्ति स विरुद्ध उदाहृतः॥७४॥
आश्रयासिद्धिराद्या स्यात्, स्वरूपासिद्धिरप्यथ~।
व्याप्यत्वासिद्धिरपरा स्यादसिद्धिरतस्त्रिधा॥७५॥
पक्षासिद्धिर्यत्र पक्षो भवेन्मणिमयो गिरिः~।
ह्रदो द्रव्यं धूमवत्त्वादत्राऽसिद्धिरथाऽपरा॥७६॥
व्याप्यत्वासिद्धिरपरा नीलधूमादिके भवेत्~।
एवकारेण साध्यवत्त्वावच्छेदेन हेत्वभावो बोधितः~। तथा च साध्यव्यापकीभूताभावप्रतियोगित्वं तदर्थः॥७४॥
॥इति विरुद्धनिरूपणम्॥
असिदिं्ध विभजते - आश्रयासिद्धिरित्यादि॥७५॥
पक्षासिद्धिरिति~। आश्रयासिद्धिरित्यर्थः~।
अपरेति~। स्वरूपासिद्धिरित्यर्थः॥७६॥
नीलधूमादिक इति~। नीलधूमत्वादिकं गुरुतया न हेतुतावच्छेदकं, स्वसमानाधिकरण-व्याप्यतावच्छेदक-धर्मान्तराघटितस्यैववच्छदकत्वात् व्याप्यतां~। धूमप्रागभावत्वसङ्ग्रहाय
स्वसमानाधिकरणेति॥इत्यसिद्धिनिरूपणम्॥
विरुद्धयोः परामर्शे हेत्वोः सत्प्रतिपक्षता॥७७॥
विरुद्धयोरिति~। कपिसंयोग-तदभावव्याप्यवत्तापरामर्शेऽपि न सत्प्रतिपक्षत्वमत उक्तं विरुद्धयोरिति~। तथा च स्वसाध्र्याविरुद्धसाध्याभावव्याप्यवत्ता
परामर्शकालीनसाध्यव्याप्यवत्तापरामर्शविषय इत्यर्थः॥७७॥
॥इति सत्प्रतिपक्षनिरूपणम्॥
साध्यशून्यो यत्र पक्षस्त्वसौ बाध उदाहृतः~।
उत्पत्तिकालीनघटे गन्धादिर्यत्र साध्यते॥७८॥
साध्यशून्य इति~। पक्षः-पक्षतावच्छेदकविशिष्ट इत्यर्थः~। तेन घटे गन्धसत्वेऽपि न क्षतिः~। एवं मूलावच्छिन्नो वृक्षः कपिसंयोगीत्यत्राऽपि बोध्यम्॥...॥
॥इति बाधमिरूपणम्॥
॥इति श्रोविश्वनाथपञ्चाननभट्टाचार्यविरचितायां सिद्धान्तमुक्तावल्या मनुमानखण्डम्~।
अथ उत्पमितिखण्डम्~।
ग्रामीणस्य प्रथमतः पश्यतो गवयादिकम्~।
सादृश्यधीर्गवादीनां या स्यात् सा करणं मतम्॥७९॥
वाक्यार्थस्याऽतिदेशस्य स्मृतिव्र्यापार उच्यते~।
गवयादिपदानां तु शक्तिधीरुपमाफलम्॥८०॥
साध्यशून्य इति~। पक्षः-पक्षतावच्छेदकविशिष्ट इत्यर्थः~। तेन घटे गन्धसत्त्वेेऽपि न क्षतिः~। एवं मूलावच्छिन्नो वृक्षः कपिसंयोगीत्यत्राऽपि बोध्यम्॥७८॥
॥इति बाधमिरूपणम्॥
॥इति श्रोविश्वनाथपञ्चाननभट्टाचार्यविरचितायां सिद्धान्तमुक्तावल्यामनुमानखण्डम्~।
उपमितिं व्युत्पादयति~। ग्रामीणस्येति~। यत्राऽऽरण्यकेन केनचित् ग्रामीणायोक्तं गोसदृशो गवयपदवाच्य इति, पश्चाच्च ग्रामीणेन क्वचिदरण्यादौ गवयो दृष्टः, तत्र
गोसादृश्यदर्शनं यज्जातं, तदुपमितिकरणम्~। तदनन्तरं गोसदृशो गवयपदवाच्य इत्यतिदेशवाक्यार्थस्मरणं यज्जायते, तदेव व्यापारः~। तदनन्तरं गवयो गवयपदवाच्य इति
ज्ञानं यज्जायते, तदुपमितिः~। न त्वयं गवयपदवाच्य इत्युपमितिः, गवयान्तरे शक्तिग्रहाभावप्रसङ्गात्॥७६॥
॥इति श्रीविश्वनाथपञ्चाननभट्टाचार्यविरचितायां सिद्धान्तमुक्तावल्यामुपमानखण्डम्॥
शब्दखण्डः[सम्पाद्यताम्]
पदज्ञानं करणं तु द्वारं तत्र पदार्थधीः~।
शब्दबोधः फलं तत्र शक्तिधीः सहकारिणी॥८१॥
शाब्दबोधप्रकारं दर्शयति-पदज्ञानं त्विति~। न तु ज्ञायमानं पदं करणम्~। पदाभावेऽपि मौनिश्लोकादौ शाब्दबोधात्~।
पदार्थधीरिति~। पदजन्यपदार्थस्मरणं व्यापारः~। अन्यथा पदज्ञानवतः प्रत्यक्षादिना पदार्थोपस्थितावपि शाब्दबोधापत्तेः~। तत्राऽपि वृत्त्या पदजन्यत्वं बोध्यम्~। अन्यथा
घटादिपदात् समवायसम्बन्धेन आकाशस्मरणे जाते आकाशस्याऽपि शाब्दबोधापत्तेः~। वृत्तिश्च शक्तिलक्षणान्यतरसम्बन्धः~। अत्रैव शक्तिज्ञानस्योपयोगः~। शक्तिग्रहाभावे
पदज्ञानेऽपि तत्सम्बन्धेन तत्स्मरणानुपपत्तेः~। पदज्ञानस्य हि एकसम्बन्धिज्ञानविधयाऽर्थस्मारकत्वम्~।
शक्तिश्च पदेन सह पदार्थस्य सम्बन्धः~। सा चाऽस्माच्छब्दादयमर्थो बोद्धव्य इतीश्वरेच्छारूपः~। आधुनिके नाम्नि शक्तिरस्त्येव~। ""एकादशेऽहनि पिता नाम कुर्यात्,
इतीश्वरेच्छायाः सत्त्वात्~। आधुनिकसङ्केतिते तु न शक्तिरिति सम्प्रदायः~।
नव्यास्तु-ईश्वरेच्छा न शक्तिः, किन्त्विच्छैव~। तेनाऽऽधुनिकसङ्केतितेऽपि शक्तिरस्त्येवेत्याहुः~।
शक्तिग्रहस्तु व्याकरणादितः~। तथाहि-
""शक्तिग्रहं व्याकरणोपमानकोशाप्तवाक्याद्व्यवहारतश्च~।
वाक्यस्य शेषाद्विवृतेर्वदन्ति सान्निध्यतः सिद्धपदस्य वृद्धाः~।""
धातुप्रकृतिप्रत्ययादीनां शक्तिग्रहो व्याकरणाद्भवति~। क्वचित् सति बाधके त्यज्यते~। यथा वैयाकरणैराख्यातस्य कर्तरि शक्तिरुच्यते~। चैत्रः पचतीत्यादौ कत्र्रा सह
चैत्रस्याऽभेदान्वयः~। तच्च गौरवात् त्यज्यते~। किन्तु कृतौ शक्तिः, लाघवात्~। कृतिश्चैत्रादौ प्रकारीभूय भासते~।
न च कर्तुरनभिधानाच्चैत्रादिपदोत्तरं तृतीया स्यादिति वाच्यम्~। कर्तृसङ्ख्यानभिधानस्य तत्र तन्त्रत्वात्~।
सङ्ख्याभिधानयोग्यश्च कर्मत्वाद्यनवरुद्धः प्रथमान्तपदोपस्थाप्यः~। कर्मत्वादीत्यस्येतरविशेषणत्वतात्पर्याविषयत्वमर्थः~। तेन चैत्र इव मैत्रो गच्छतीत्यादौ न चैत्रे
सङ्ख्यान्वयः~। यत्र कर्मादौ न विशेषणत्वे तत्पर्यं तद्वारणाय प्रथमान्तेति~।
यद्वा धात्वर्थातिरिक्ताविशेषणत्वं प्रथमदलार्थः~। तेन चैत्र इव मैत्रो गच्छतीत्यत्र चैत्रादेर्वारणम्~। स्तोकं पचतीत्यादौ स्तेकादेर्वारणाय द्वितीयदलम्~। तस्य
द्वितीयान्तपदोपस्थाप्यत्वाद्वारणमिति~।
एवं व्यापारेऽपि न शक्तिः, गौरवात्~। रथो गच्छतोत्यादौ तु व्यापारे, आश्रयत्वे वा लक्षणा~। जानातीत्यादौ त्वाश्रयत्वे, नश्यतीत्यादौ तु प्रतियोगित्वे निरूढलक्षणा~।
उपमानाद्यथा शक्तिग्रहस्तथोक्तम्~।
एवं कोशादपि शक्तिग्रहः~। सति बाधके क्वचित् त्यज्यते~। यथा नीलादिपदानां नीलरूपादौ नीलविशिष्टे च शक्तिः कोशेन व्युत्पादिता, तथाऽपि लाघवान्नीलादावेव
शक्तिः नीलादिविशिष्टे तु लक्षणेति~।
एवमाप्तवाक्यादपि~। यथा कोकिलः पिकशब्दवाच्य इत्यादिशब्दात् पिकादिशब्दानां कोकिले शक्तिग्रहः~।
एवं व्यवहारादपि~। यथा प्रयोजकवृद्धेन घटमानयेत्युक्तं, तच्छØत्वा प्रयोज्यवृद्धेन घट आनीतः, तदवधार्य पाश्र्वस्थो बालो घटानयनरूपं कार्यं
घटमानयेतिशब्दप्रयोज्यमित्यवधारयति~। ततश्च घटं नय, गामानयेत्यादिवाक्यादावापोद्वापाभ्यां घटादिपदानां कार्यान्वितघटादौ शकिं्त~।
इत्थञ्च भूतले नीलो घट इत्यादिवाक्यान्न शाब्दबोधः, घटादिपदानां कार्यान्वितघटादिबोधे सामथ्र्यावधारणात्~। कार्यताबोधं प्रति च लिङ्ादीनां सामथ्र्यात्तदभावान्न
शाब्दबोध इति केचित्~।
तन्न~। प्रथमतः कार्यान्वितघटादौ शक्त्यवधारणेऽपि लाघवेन पश्चात्तस्य परित्यागौचित्यात्~। अत एव चैत्र ! पुत्रस्ते जातः, कन्या ते गर्भिणीत्यादौ मुखप्रसाद्-
मुखमालिन्याभ्यां सुखदुःखे अनुमाय तत्कारणत्वेन परिशेषाच्छाब्दबोधं निर्णीय तद्धेतुतया तं शब्दमवधारयति~। तथा च व्यभिचारान्न कार्यान्विते शक्तिः~। न च तत्र तं
पश्येत्यादिशब्दान्तरमध्याहार्थं, मानाभावात्~। चैन्न! पुत्रस्ते जातो मृतश्चेत्यादौ तदभावाच्च~।
इत्थं च लाघवादन्वितघटेऽपि शकिं्त त्यक्त्वा घटपदस्य घटमात्रे शक्तिमवधारयति~।
एवं वाक्यशेषादपि शक्तिग्रहः~। यथा "यवमयश्चरुर्भवति" इत्यत्र यवपदस्य दोर्घशूकविशेषे आर्याणां प्रयोगः, कङ्गौ तु म्लेच्छानाम्~। तत्र हि ""यदान्या ओषधयो
म्लायन्ते, अथैते मोदमानास्तिष्ठन्ति""
""वसन्ते सर्वसस्यानां जायते पत्रशातनम्~।
मोदमानाश्च तिष्ठन्ति यवाः कणिशशालिनः""
इति वाक्यशेषाद्दीर्र्घशूके शक्तिर्निणीयते~। कङ्गौ तु शक्तिभ्रमात् प्रयोगः, नानाशक्तिकल्पने गौरवात्~। हर्र्यादिपदे तु विनिगमकाभावान्नानाशक्तिकल्पनम्~।
एवं विवरणादपि शक्तिग्रहः~। विवरणं तु तत्समानार्थकपदान्तरेण तदर्थकथनम्~। यथा घटोऽस्तीत्यस्य कलशोऽस्तीत्यनेन विवरणात् घटपदस्य कलशे शक्तिग्रहः~।
एवं पचतीत्यस्य पाकं करोतीत्यनेन विवरणादाख्यातस्य यत्नार्थकत्वं कल्प्यते~।
एवं प्रसिद्धपदस्य सान्निध्यादपि शक्तिग्रहः~। यथा-इह सहकारतरौ मधुरं पिको रौतोत्यादौ पिकशब्दस्य कोकिले शक्तिग्रह इति~।
॥इति शक्तिग्रहोपायनिरूपणम्॥
तत्र जातावेव शक्तिग्रहः, न तु व्यक्तौ, व्यभिचारात्, आनन्त्याच्च~। व्यकिं्त विना जातिभानस्याऽसम्भवाद्व्यक्तेरपि भानमिति केचित्~।
तन्न~। शकिं्त विना व्यक्तिभानानुपपत्तेः~। न च व्यक्तौलक्षणा, अनुपपत्तिप्रतिसन्धानं विनाऽपि व्यक्तिबोधात्~। न च व्यक्तिशक्तावानन्त्यम्~। सकलव्यक्तावेकस्या एव
शक्तेः स्वीकारात्~। न चाऽननुगमः गोत्वादेरेवाऽनुगमकत्वात्~। किञ्च गौः शक्येति शक्तिग्रहो यदि, तदा व्यक्तौ शक्तिः~। यदि तु गोत्वं शक्यमिति शक्तिग्रहः, तदा
गोत्वप्रकारकपदार्थस्मरणं शाब्दबोधश्च न स्यात्, समानप्रकारकत्वेन शक्तिज्ञानस्य पदार्थस्मरणं, शाब्दबोधं प्रति हेतुत्वात्~। किञ्च गोत्वे यदि शक्तिः, तदा गोत्वत्वं
शक्यतावच्छेदकं वाच्यम्~। गोत्वत्वं तु गवेतरासमवेतत्वे सति सकलगोसमवेतत्वम्~। तथा च गोव्यक्तीनां शक्यतावच्छेदकेऽनुप्रवेशात् तवैव गौरवम्~। तस्मात्
तत्तज्जात्याकृतिविशिष्टतत्तद्व्यक्तिबोधानुपपत्त्य कल्प्यमाना शक्तिर्जात्याकृतिविशिष्टव्यक्तावेव विश्राम्यतीति॥इति जातिशक्तिवादः॥
शक्तं पदम्~। तच्चतुर्विधम्~। क्वचिद्यौगिकं, क्वचिद्रूढं, क्वचिद्योगरूढं क्वचिद्यौगिकरूढम्~।
तथाहि--यत्राऽवयवार्थ एव बुद्ध्यते तद्यौगिकम्~। यथा पाचकादिपदम्~।
यत्राऽवयवशक्तिनैरपेक्ष्येण समुदायशक्तिमात्रेण बुद्ध्यते, तद्रूढम्~। यथा गो-मण्डला (पा) दिपदम्~।
यत्र तु अवयवशक्तिविषये समुदायशक्तिरप्यस्ति, तद्योगरूढम्~। यथा पङ्कजादिपदम्~। तथाहि-पङ्कजपदमवयवशक्त्या पङ्कजनिकतृत्वरूपमर्थं बोधयति,
समुदायशक्त्या च पद्मत्वेन रूपेण पद्मं बोधयति~। न च केवलया अवयवशक्त्या कुमुदे प्रयोगः स्यादिति वाच्यम्~। रूढिज्ञानस्य केवलयौगिकार्थज्ञानप्रतिबन्धकत्वादिति
प्राञ्चः~।
वस्तुतस्तु समुदायशक्त्युपस्थितपद्मेऽवयवार्थपङ्कजनिकर्तुरन्वयो भवति, सान्निध्यात्~।
यत्र तु रूढ्यर्थस्य बाधः प्रतिसन्धीयते, तत्र लक्षणया कुमुदादेर्बोधः~। यत्र तु कुमुदत्वेन रूपेण बोधे न तात्पर्यज्ञानं, पद्मत्वस्य च बाधः, तत्र चाऽवयवशक्तिमात्रेण
निर्वाह इत्याहुः~। यत्र तु स्थलपद्मादाववयवार्थबाधः, तत्र समुदायशक्त्या पद्मत्वेन रूपेण बोधः~। यदि तु स्थलपङ्कजं विजातीयमेव, तदा लक्षणयैव~।
यत्र तु यौगिकार्थरूढ्यर्थयोः स्वातन्त्र्येण बोधः, तद्यौगिकरूढम्~। यथोद्भिदादिपदम्~। तत्र हि उद्भेदनकर्ता तरुगुल्मादिरपि बुद्ध्यते, यागविशेषोऽपीति॥८२॥इति
पदनिरूपणम्॥
लक्षणा शक्यसम्बन्धः तात्पर्यानुपपत्तितः~।
लक्षणा शक्यसम्बन्ध इति~। गङ्गायां घोष इत्यादौ गङ्गापदस्य शक्यार्थे प्रवाहरूपे घोषस्थाऽन्वयानुपपत्तिस्तात्पर्यानुपपत्तिर्वा यत्र प्रतिसन्धीयते, तत्र लक्षणया
तीरस्य बोध इति~। सा च शक्यसम्बन्धरूपा~। तथाहि-प्रवाहरूपशक्यार्थसम्बन्धस्य तीरे गृहीतत्वात् तीरस्य स्मरणम्, ततः शाब्दबोधः~। परन्तु यद्यन्वयानुपपत्तिर्लक्षणाबीजं
स्यात्, तदा यष्टीः प्रवेशयेत्यादौ लक्षणा न स्यात्~। यष्टिषु प्रवेशान्वयस्याऽनुपपत्तेरभावात्~। तेन तत्प्रवशे भोजनतात्त्वपर्यानुपपत्त्या यष्टिधरेषु लक्षणा~। एवं काकेभ्यो
दधिरक्ष्यतामित्यादौ काकपदस्य दध्युपघातके लक्षणा, सर्वतो दधिरक्षायास्तात्पर्यविषयत्वात्~। एवं छत्रिणो यान्तीत्यादौ छत्रिपदस्यैकसार्थवाहित्वे लक्षणा~। इयमेवाऽजहत्स्वार्था
लक्षणेत्युच्यते, एकसार्थवाहित्वेन रूपेण छत्रि-तदन्ययोर्बोधात्~। यदि वाऽन्वयानुपपत्तिर्लक्षणाबीजं स्यात्, तदा क्वचित् गङ्गापदस्य तीरे, क्वचित् घोषपदस्य मत्स्यादौ
लक्षणेति नियमो न स्यात्~।
इदं तु बोध्यम्~। शक्यार्थसम्बन्धो यदि तीरत्वेन रूपेण गृहीतः, तदा तीरत्वेन तीरबोधः~। यदि तु गङ्गातीरत्वेन रूपेण गृहीतः, तदा तेनैव रूपेण स्मरणम्~। अत
एव लक्ष्यतावच्छेदके न लक्षणा, तत्प्रकारबोधस्य तत्र लक्षणां विनाऽप्युपपत्तेः~। परन्त्वेवंक्रमेण शक्यतावच्छेदकेऽपि शक्तिर्न स्यात्, तत्प्रकारकशक्यार्थस्मरणं प्रति
तत्पदस्य सामथ्र्यमित्यस्य सुवचत्वादिति विभावनीयम्~।
यत्र तु शक्यार्थस्य परम्परासम्बन्धरूपा लक्षणा, सा लक्षितलक्षणेत्युच्यते~। यथा द्विरेफादिपदाद्रेफद्वयसम्बन्धो भ्रमरपदे ज्ञायते, भ्रमरपदस्य च सम्बन्धो भ्रमरे
ज्ञायते, तत्र लक्षितलक्षणा~।
किन्तु लाक्षणिकं पदं नाऽनुभावकम्~। लाक्षणिकार्थस्य शाब्दबोधे तु पदान्तरं कारणम्~। शक्तिलक्षणान्यतरसम्बन्धेनेतरपदार्थन्वितस्वशक्यार्शाब्दबोधं प्रति पदानां
सामथ्र्यावधारणात्~।
वाक्ये तु शक्तेरभावात् शक्यसम्बन्धरूपा लक्षणाऽपि नाऽस्ति~। यत्र गभीरायां नद्यं घोष इत्युक्तं, तत्र नदीपदस्य नदीतीरे लक्षणा~। गभीरपदार्थस्य नद्या
सहाऽभेदान्वयः~। क्वचिदेकदेशान्वयस्याऽपि स्वीकृतत्वात्~। यदि तत्रैकदेशान्वयो न स्वीक्रियते. तदा नदीपदस्य गभीरनदीतीरे लक्षणा~। गभीरपदं तात्पर्यग्राहकम्~।
बहुव्रीहावप्येवम्~। तत्र हि चित्रगुपदादौ यदैकदेशान्वयः स्वीक्रियते, तदा गोपदस्य गोमति लक्षणा, गवि चित्राऽभेदान्वयः~। यदि त्वेकदेशान्वयो न स्वीक्रियते, तदा
गोपदस्य चित्रगोस्वामिनि लक्षणा~। चित्रपदं तात्पर्यग्राहकम्~। एवमारूढवानरो वृक्ष इत्यत्र वानरपदस्य वानरारोहणकर्मणि लक्षणा~। आरूढपदं च तात्पर्यग्राहकम्~।
एवमन्यत्राऽपि~।
तत्पुरुषे तु पूर्वपदे लक्षणा~। तथाहि राजपुरुषादिपदे राजपदार्थेन पुरुषादिपदार्थस्य साक्षान्नाऽन्वयः, निपातातिरिक्तनामार्थयोर्भेदेनान्वयबोधस्याव्युत्पन्नत्वात्~। अन्यथा
राजा पुरुषः इत्यत्रापि तथाऽन्वयबोधः स्यात्~। घटः पटो नेत्यादौ घटपटाभ्यां नञः साक्षादेवाऽन्वयान्निपातातिरिक्तेति~। नीलो घट इत्यादौ नामार्थयोरभेदसम्बन्धेनाऽन्वयाद्भेदेनेति~।
न च राजपुरुष इत्यादौ लुप्तविभक्तेः स्मरणं कल्प्यमिति वाच्यम्~। अस्मृतविभक्तेरपि ततो बोधोदयात्~। तस्मात् राजपदादौ राजसम्बन्धिनि लक्षणा~। तस्य च पुरुषेण
सहाऽभेदान्वयः~।
द्वन्द्वे तु धवखदिरौ छिन्धीत्यादौ धवः खदिरश्च विभक्त्यर्थद्वित्वप्रकारेण बुद्धयेते, तत्र न लक्षणा~। न च साहित्ये लक्षणेति वाच्यम्~। साहित्यशून्ययोरपि द्वन्द्वदर्शनात्~।
न चैकक्रियान्वयित्वरूपं साहित्यमस्तीति वाच्यम्~। क्रियाभेदेऽपि धवखदिरौ पश्य छिन्धीत्यादिदर्शनात्, साहित्यस्याऽननुभवाच्च~। अत एव ""राजपुरोहितौ सायुज्यकामौ
यजेयाताम्"" इत्यत्र लक्षणाभावात् द्वन्द्व आश्रीयते~। तस्मात् साहित्यादिकं नाऽर्थः~। किन्तु वास्तवो भेदो यत्र तत्र द्वन्द्वः न च नीलघटयोरभेद इत्यादौ कथमिति वाच्यम्~। तत्र
नीलपदस्य नीलत्वे, घटपदस्य घटत्वे लक्षणा~। अभेद इत्यस्य चाऽऽश्रयाभेद इत्यर्थात्~।
समाहारद्वन्द्वे तु यदि समाहारोऽप्यनुभूयत इत्युच्यते, तदाऽहिनकुलमित्यादौ परपदेऽहिनकुलसमाहारे लक्षणा, पूर्वपदं तात्पर्यग्राहकम्~। न च भेरिमृदङ्गं वादयेत्यत्र
कथं समाहारस्याऽन्वयः~? अपेक्षाबुद्धिविशेषरूपस्य तस्य वादनासम्भवादिति वाच्यम्~। परम्परासम्बन्धेन तदन्वयात्~। एवं पञ्चमूलीत्यादावपि~।
परे त्वहिनकुलमित्यादावहिर्नकुलश्च बुद्धयते~। प्रत्येकमेकत्वान्वयः~। समाहारसञ्ज्ञा च यत्रैकत्वं नपुंसकत्वं च "प्राणितूर्र्य - (२~।४~।२)" इत्यादिसूत्रेणोक्तं तत्रैव,
अन्यत्रैकवचनमसाध्विति वदन्ति~।
पितरौ श्वशुरावित्यादौ पितृपदे जनकदम्पत्योः, श्वशुरपदे स्त्रीजनकदम्पत्योर्लक्षणा~। एवमन्यत्रापि~। घटा इत्यादौ तु न लक्षणा~। घटत्वेन रूपेण नानाघटोपस्थितिसम्भवात्~।
कर्मधारयस्थले तु नीलोत्पलमित्यादावभेदसम्बन्धेन नीलपदार्थ उत्पलपदार्थे प्रकारः~। तत्र च न लक्षणा~। अत एव ""निषादस्थपतिं याजयेत्"" इत्यत्र न तत्पुरुषः,
लक्षणापत्तेः, किं तु कर्मधारय;, लक्षणाभावात्~। न च निषादस्य सङ्करजातिविशेषस्य वेदानधिकाराद्याजनासम्भव इति वाच्यम्~। निषादस्य विद्याप्रयुक्तेस्तत एव
कल्पनात्~। लाघवेन मुख्यार्थस्याऽन्वये तदनुपपत्त्या तत्कल्पनायाः फलमुखगौरवतयाऽदोषत्वादिति~।
उपकुम्भमद्र्धपिप्पलीत्यादौ परपदे तत्सम्बन्धिनि लक्षणा, पूर्वपदार्थप्रधानतया चाऽन्वयबोध इति~।
इत्थञ्च समासे न क्वाऽपि शक्तिः~। पदशक्त्यैव निर्वाहादिति॥इति लक्षणाग्रन्थः॥
आसत्तिर्र्योग्यताकाङ्क्षातात्पर्यज्ञानमिष्यते~।८२॥
कारणं सन्निधानं तु पदस्याऽऽसत्तिरुच्यते~।
आसत्तिरित्यादि~। आसत्तिज्ञानं, योग्यताज्ञानं, आकांक्षाज्ञानं, तात्पर्यज्ञानं च शाब्दबोधे कारणम्॥इति शाब्दबोधकारणकथनम्॥
तत्राऽऽसत्तिपदार्थमाह~। सन्निधानं त्विति~। अन्वयप्रतियोग्यनुयोगिपदयोरव्यवधानमासत्तिः~। तज्ज्ञानं शाब्दबोधे कारणम्, क्वचिद्व्यवहितेऽप्यव्यवधानभ्रमाच्छाब्दबोधादिति
केचित्~। वस्तुतस्तु अव्यवधानज्ञानस्याऽनपेक्षितत्वात् यत्पदार्र्थेस्य यत्पदार्थनाऽन्वयोऽपेक्षितस्तयोरव्यवधानेनोपस्थितिः शाब्दबोधे कारणम्~। तेन गिरिर्भुक्तमाग्निमान्
देवदत्तेनेत्यादौ न शाब्दबोधः~। तात्पर्यगर्भा चाऽऽसत्तिः~। नीलो घटो द्रव्यं पट इत्यादावासत्तिभ्रमाच्छाब्दबोधः~। आसत्तिभ्रमात् शाब्दभ्रमाभावेऽपि न क्षतिः~।
ननु यत्र छत्री कुण्डली वासस्वी देवदत्त इत्युक्तं, तत्रोत्तरपदस्मरणेनपूर्वपदस्मरणस्य नाशादव्यवधानेन तत्तत्पदस्मरणासम्भव इति चेन्न~।
प्रत्येकपदानुभवजन्यसंस्कारैश्चरमस्य तावत्पदविषयकस्मरणस्याऽव्यवधानेनोत्पत्तेः~। नानासन्निकर्षेरैरेकप्रत्यक्षस्येव नानासंस्कारेकस्मरणोत्पत्तेरपि सम्भवात्~।
तावत्पदसंस्कारसहितचरमवर्ण (१) ज्ञानस्योद्बोधकत्वात्~। कथमन्यथा नानावर्णैरेक पदस्मरणम्~?
परंतु तावत्पदार्थानां स्मरणादेकदैव खले कपोतन्यायात् तावत्पदार्थानां क्रियाकर्मभावेनाऽन्वयबोधरूपः शाब्दबोधो भवतीति केचित्~।
""वृद्धा युवानः शिशवः कपोताः खले यथाऽमी युगपत् पतन्ति~।
तथैव सर्वे युगपत् पदार्थाः परस्परेणाऽन्वयिनो भवन्ति॥""
अपरे तु-
""यद्यदाकाङ्क्षितं योग्यं सन्निधानं प्रपद्यते~।
तेन तेनोऽन्वितः स्वार्थः पदैरेवाऽगम्यते॥""
तथा च खण्डवाक्यार्थबोधानन्तरं तथैव पदार्थस्मृत्या महावाक्यार्थबोध इत्यप्याहुः~।
एतेन तावद्वर्णाभिव्यङ्ग्यः पदस्फोटोऽपि निरस्तः~। तत्तद्वर्णसंस्कारसहितचरमवर्णोपलम्भेन तद्व्यञ्जकेनैवोपपत्तेरिति~।
इदं तु बोध्यम्~। यत्र द्वारमित्युक्तं, तत्र पिधेहीति पदस्य ज्ञानादेव बोधः, न तु पिधानादिरूपार्थज्ञानात्, पदजन्यतत्तत्पदार्थोपस्थितेस्तत्तच्छाब्दबोधे हेतुत्वात्~। किञ्च
क्रियाकर्मपदानां तेन तेनैव रूपेणाऽऽकाङ्क्षितत्वात् तेन क्रियाप्देनविना कथं बोधः स्यात्~। तथा पुष्पेभ्यः इत्यादौ स्पृहयतीत्मादिपदाध्याहारं विना चतुथ्र्थनुपपत्तेः
पदाध्याहार आवश्यकः~। इत्यासत्तिनिरूपणम्~।
पदार्थे तत्र तद्वत्ता योग्यता परिकीर्तिता॥८३॥
योग्यतां निर्वक्ति-पदार्थ इति~। एकपदार्थेऽपरपदार्थसम्बन्धो योग्यतेत्यर्थः~। तज्ज्ञानाभावाच्च वह्निना सिञ्चतीत्यादौ न शाब्दबोधः~। नन्वेतस्या योग्यताया ज्ञानं
शाब्दबोधात् प्राक् सर्वत्र न सम्भवति, वाक्यार्थस्याऽपूर्वत्वादिति चेन्न~। तत्तत्पदार्थस्मरणे सति क्वचित् संशयरूपस्य, क्वचिन्निश्चयरूपस्य योग्यताज्ञानस्य सम्भवात्~।
नव्यास्तु योग्यताज्ञानं न शाब्दबोधहेतुः~। वह्निना सिञ्चतीत्यादौ सेके वह्निकरणकत्वाभावरूपायोग्यतानिश्चयेन प्रतिबन्धान्न शाब्दबोधः~। तदभावनिश्चयस्य
लौकिकसन्निकर्षाजन्यदोषविशेषाजन्यतज्ज्ञानमात्रे प्रतिबन्धकत्वाच्छाब्दबोधं प्रत्यपि प्रतिबन्धकत्वं सिद्धम्~। योग्यताज्ञानविलम्बाच्च शाब्दबोधविलम्बोऽसिद्ध इत्याहुः॥८२॥८३॥
इति योग्यतानिरूपणम्॥
यत्पदेन विना यस्याऽननुभावकता भवेत्~।
आकाङ्क्षां निर्वक्ति यत्पदेनेत्यादि~। येन पदेन विना यत्पदस्याऽन्वयाननुभावकत्वं, तेन पदेन सह तस्याऽऽकाङ्क्षेत्यर्थः~। क्रियापदं विना कारकपदं नाऽन्वयबोधं
जनयतीति तेन तस्याऽऽकाङ्क्षा~।
वस्तुतस्तु क्रियाकारकपदानां सन्निधानमासत्त्या चरितार्थम्~। परंतु घटकर्मताबोधं प्रति घटपदोत्तरद्वितीयारूपाकाङ्क्षाज्ञानं कारणम्~। तेन घटः कर्मत्वमानयनं
कृतिरित्यादौ न शाब्दबोधः~। अयमेति पुत्रो राज्ञः पुरुषोऽपसार्यतामित्यादौ तु पुत्रेण सह राजपदस्य तात्पर्यग्रहसत्त्वात् तेनैवाऽन्वयबोधः~। पुरुषेण सह तात्पर्यग्रहे तु तेन
सहाऽन्वयबोधः स्यादेव~।
॥इत्याकाङ्क्षानिरूपणम्॥
आकाङक्षा वक्तुरिच्छा तु तात्पर्यं परिकीर्तितम्॥८४॥
तात्पर्यं निर्वक्ति-वक्तुरिच्छेति~। यदि तात्पर्यज्ञानं कारणं न स्यात्, तदा सौन्धवमानयेत्यादौ क्वचिदश्वस्य क्वचिल्लवणस्य बोध इति न स्यात्~। न च
तात्पर्यग्राहकाणां प्रकरणादीनां शाब्दबोधे कारणत्वमस्त्विति वाच्यम्~। तेषामननुगमात्~। तात्पर्यज्ञानजनकत्वेन तेषामनुगमे तु तात्पर्यज्ञानमेव लाघवात् कारणमस्तु~।
इत्थञ्च वेदस्थलेऽपि तात्पर्यज्ञानार्थमीश्वरः कल्प्यते~। न च तत्राऽध्यापकतात्पर्यज्ञानं कारणमिति वाच्यम्~। सर्गादावध्यापकाभावात्~। न च प्रलय एव नाऽस्ति,
कुत्र सर्गादिरिति वाच्यम्~। प्रलयस्याऽऽगमेषु प्रतिपाद्यत्वात्~।
इत्थञ्च शुकवाक्येऽपीश्वरीयतात्पर्यज्ज्ञानं~। विसंवादिशुकवाक्ये तु शिक्षयितुरेव तात्पर्यज्ञानं कारणं वाच्यम्~।
अन्ये तु - नानार्थादौ क्वचिदेव तात्पर्यज्ञानं कारणम्~। तथा च शुकवाक्ये विनैव तात्पर्यज्ञानं शाब्दबोधः~। वेदे त्वनादिमीमांसापरिशोधिततर्कैरर्थावधारणमित्याहुः॥
इति तात्पर्यनिरूपणम्॥
इति श्रीविश्वनाथपञ्चाननभट्टाचार्यविरचितायां सिद्धान्तमुक्तावल्यां शब्दखण्डम्॥४॥
अथ स्मरणनिरूपणम्~।
पूर्वमनुभवस्मरणभेदाद्बुद्धेद्र्वैविध्यमुक्तम्~। तत्राऽनुभवप्रकारा दर्शिताः~। स्मरणं तु सुगमतया न दर्शितम्~। तत्र हि पूर्वानुभवः कारणम्~।
अत्र केचित्~। अनुभवत्वेन न कारणत्वं, किन्तु ज्ञानत्वेनैव~। अन्यथा स्मरणोत्तरं स्मरणं न स्यात्~। समानप्रकारकस्मरणेन पूर्वसंस्कारस्य विनष्टत्वात्~। मन्मते तु
तेनैव स्मरणेन संस्कारान्तरद्वारा स्मरणान्तरं जन्यत इत्याहुः~। तन्न~। यत्र समूहालम्बनोत्तरं घटपटादीनां क्रमेण स्मरणमजनिष्ट, सकलविषयकस्मरणं तु नाभूत्, तत्र फलस्य
संस्कारनाशकत्वाभावात् कालस्य, रोगस्य, चरमफलस्य वा सर्वत्र संस्कारनाशकत्वं वाच्यम्~। तथा च न क्रमिकस्मरणानुपपत्तिः~। न च पुनः पुनः स्मरणाद्दृढतरसंस्कारानुपपत्तिरिति
वाच्यम्~। झटित्युद्बोेधकसमवधानस्यैव दाढ्र्यपदार्थत्वात्~। नच विनिगमनाविरहादेव ज्ञानत्वेनाऽपि जनकत्वं स्यादिति वाच्यम्~। विशेषधर्मेण व्यभिचाराज्ञाने
सामान्यधर्मेणान्यथासिद्धत्वात्~। कथमन्यथा दण्डस्य भ्रमिद्वारा द्रव्यत्वेन रूपेण न कारणत्वम्~। न चाऽऽन्तरालिकस्मरणानां संस्कारनाशकत्वसंशयात् व्यभिचारसंशय इति
वाच्यम्~। अनन्तसंस्कारतन्नाशकल्पनापेक्षया चरमस्मरणस्यैव संस्कारनाशकत्वकल्पनेन व्यभिचारसंशयाभावात्॥इति स्मृतिकारणनिरूपणम्॥इति स्मृतिप्रक्रिया॥
साक्षात्कारे सुखादीनां करणं मन उच्यते~।
अयौगपद्याज्ज्ञानानां तस्याणुत्वमिहोच्यते॥८५॥
इदानीं क्रमप्राप्तं मनो निरूपयितुमाह-साक्षात्कारे इति~। एतेन मनसि प्रमाणं दर्शितम्~। तथाहि-सुखसाक्षात्कारः सकरणकः जन्यसाक्षात्कारत्वाच्चाक्षुषसाक्षात्कारवदित्यनुमानन
मनसः करणत्वसिद्धिः~। न चैवं दुःखादिसाक्षात्कारणामपि करणान्तराणि स्युरिति वाच्यम्~। लाघवादेकस्यैव तादृशसकलसाक्षात्कारकरणतया सिद्धेः~। एवं
सुखादीनामसमवायिकारणसंयोगाश्रयतया मनसः सिद्धिर्बोद्धव्या~।
तत्र मनसाऽणुत्वे प्रमाणमाह-अयौगपद्यादिति~। ज्ञानानां चाक्षुषरासनादीनां अयौगपद्यमेककालोत्पत्तिर्नास्तीत्यनुभवसिद्धम्~। तत्र नानेन्द्रियाणां सत्यपि विषयसन्निधाने
यत्सम्बन्धादेकेनेन्द्रियेण ज्ञानमुत्पद्यते यदसम्बन्धाच्च परैज्र्ञानं नोत्पाद्यते तन्मनसो विभुत्वे चासन्निधानं न सम्भवतीति न विभु मनः~। न च तदानीमदृष्टविशेषाद्बोधकविलम्बादेव
तज्ज्ञानविल इति वाच्यम्~। तथा सति चक्षुरादीनामपि अकल्पनापत्तेः~। न च दीर्घशष्कुलीभक्षणादौ नानावधानभाजाञ्च कथमेकदा नानेन्द्रियजन्यज्ञानमिति वाच्यम्~।
मनसोऽतिलाघवात् त्वरया नानेन्द्रियसम्बन्धान्नानाज्ञानोत्पत्तेः~। उत्पलशतपत्रभेदादाविव यौगपद्यप्रत्ययस्य भ्रान्तत्वात्~। न च मनसः सङ्कोचविकासशालित्वादुभयोपपत्तिरस्त्विति
वाच्यम्~। नानावयवतन्नाशादिकल्पने गौरवाल्लाघवान्निरवयवस्याणुरूपस्यैव मनसः कल्पनादिति सङ्क्षेपः॥इति मनोनिरूपणम्॥इति सिद्धान्तमुक्तावल्यां द्रव्यपदार्थव्याख्या
समाप्ता॥
अथ गुणनिरूपणम्~।
अथ द्रव्याश्रिता ज्ञेया निर्गुणा निष्क्रिया गुणाः~।
द्रव्यं निरूप्य गुणान्निरूपयति-अथेत्यादिना~। गुणत्वजातौ किं मानमिति चेत्, द्रव्यकर्मभिन्ने सामान्यवति या कारणता सा किञ्चिद्धर्मावच्छिन्ना कारणतात्वात्
निरवच्छिन्नकारणताया असम्भवात्~। नहि रूपत्वादिकं सत्ता वा तत्रावच्छेदिका, न्यूनाधिकदेशवृत्तित्वात्~। अतश्चतुर्विंशत्यनुगतं किञ्चिद् वाच्यं तदेव गुणत्वमिति सिद्धम्~।
इति गुणत्वजातिसाधनम्~।
द्रव्याश्रितः इति~। यद्यपि द्रव्याश्रितत्वं न लक्षणम् कर्मादावतिव्याप्तेस्तथापि द्रव्यत्वव्यापकतावच्छेकसत्ताभिन्नजातिमत्वं तदर्थः~। भवति हि गुणत्वं द्रव्यत्वव्यापकतावच्छेदकं
तद्वत्ता च गुणानामिति~। कर्मत्वं द्रव्यत्वं वा न द्रव्यत्वव्यापकतावच्छेदकं गगनादौ द्रव्यकर्मणोरभावात्~। द्रव्यत्वत्वं सामान्यत्वादिकं वा न जातिरिति तद्व्युदासः~।
निर्गुणा इति~। यद्यपि निर्गुणत्वं कर्मादावपि, तथापि सामान्यवत्त्वे सति कर्मान्यत्वे च सति निर्गुणत्वं बोध्यम्~। जात्यादीनां न सामान्यवत्त्वं कर्मणो न कर्मान्यत्वं
द्रव्यस्य न निर्गुणत्वमिति तत्र नातिव्याप्तिः~।
निष्क्रिया इति स्वरूपकथनं, न तु लक्षणं गगनादावतिव्याप्तेः~।
इति गुणसामान्यलक्षणकथनम्~।
रूपं रसः स्पर्शगन्धौ परत्वमपरत्वकम्॥८६॥
द्रवो गुरुत्वं स्नेहश्च वेगो मूर्तगुणा अमी~।
(वेगा इति~। वेगेन स्थितिस्थापकोऽप्युपलक्षणीयः)
मूर्तगुणा इति~। अमूर्तेषु न वर्तन्त इत्यर्थः~। (लक्षणं तु तावदन्यान्यत्वम्~। एवमग्रेऽपि~।
इति मूर्तगुणकथनम्॥
धर्माधर्मौ भावना च शब्दो बुद्ध्यादयोऽपि च॥८७॥
एतेऽमूर्तगुणाः सर्वे विद्वद्भिः परिकीर्तिताः~।
सङ्ख्यादयो विभागान्ता उभयेषां गुणा मताः॥८८॥
अमूर्तगुणा इति~। मूर्तेषु न वर्तन्ते इत्यर्थः~। इत्यमूर्तगुणकथनम्~।
उभयेषामिति~। मूर्तामूर्तगुणा इत्यर्थः~।
इति मूर्तामूर्तगुणकथनम्॥८६॥८७॥८८॥
संयोगश्च विभागश्च सङ्ख्या द्वित्वादिकास्तथा~।
द्विपृथक्त्वादयस्तद्वदेतेऽनेकाश्रिता गुणाः॥८९॥
अतः शेषगुणाः सर्वे मता एकैकवृत्तयः~।
अनेकाश्रिता इति~। संयोगविभागद्वित्वादीनि द्विवृत्तीनि; त्रित्वचतुष्ट्वादिकं त्रिचतुरादिवृत्तीेति बोध्यम्॥इत्यनेकाश्रितगुणकथनम्~।
अत इति रूपरसगन्धस्पर्शैकत्वपरिमाणैकपृथक्त्वपरत्वापरत्वबुद्धिसुखदुःखेच्छाद्वेषप्रयत्नगुरुत्वस्नेहसंस्कारादृष्टशब्दा इत्यर्थः~। इत्येकैकवृत्तिगुणकथनम्~।
बुद्ध्यादिषट्कं स्पर्शान्तः स्नेहः सांसिद्धिको द्रवः॥९०॥
अदृष्टभावनाशब्दा अमी वैशेषिका गुणाः~।
बुद्ध्यादीति~।बुद्धिसुखदुःखेच्छाद्वेषप्रयत्ना इत्यर्थः~। स्पर्शान्ताः - रूपरसगन्धस्पर्शा-इत्यर्थः~। द्रवः-द्रवत्वम्~। वैशेषिकाः-विशेषा एव वैशेेषिकाः स्वार्थे ठक् विशेषगुणा
इत्यर्थः~। इति विशेषगुणकथनम्~।
सङ्ख्यादिरत्वान्तो द्रवोऽसांसिद्धिकस्तथा॥९१॥
गुरुत्ववेगौ सामान्यगुणा एते प्रकीर्तिताः~।
सङ्ख्यादिरपरत्वान्तो द्रवत्वं स्नेह एव च॥९२॥
सङ्ख्यादिरिति~।सङ्ख्यापरिमाणपृथक्त्वसंयोगविभागपरत्वापरत्वानीत्यर्थः॥६२॥
इति सामान्यगुणनिरूपणम्~।
एते तु द्वीन्द्रियग्राह्या अथ स्पर्शान्तशब्दकाः~।
बाह्यैकैकेन्द्रियग्राह्याः गुरुत्वादृष्टभावनाः॥९३॥
द्विन्द्रियेति~। चक्षुषा त्वचापि ग्रहणयोग्यत्वात्
इति द्वीन्द्रियग्राह्यगुणनिरूपणम्~।
बाह्येति~। रूपादीनां चक्षुरादिग्राह्यत्वात्॥९३॥
इति एकैकेन्द्रियग्राह्यगुणनिरूपणम्~।
विभूनां तु ये स्युर्वैशेषिका गुणाः~।
अकारणगुणोत्पन्ना एते तु परिकीर्तिताः॥९४॥
विभूनमिति~। बुद्धिसुखदुःखेच्छाद्वेषप्रयत्नधर्माधर्मभावनाशब्दा इत्यर्थः~। कारणगुणेन कार्ये ये गुणा उत्पाद्यन्ते ते कारणगुणपूर्वका रूपादयो वक्ष्यन्ते, बुद्ध्यादयस्तु न
तादृशाः, आत्मादेः कारणाभावात्॥इत्यकारणगुणोत्पन्नगुणकथनम्॥
अपाकजास्तु स्पर्शन्ता द्रवत्वं च तथाविधम्~।
स्नेहवेगगुरुत्वैकपृथक्त्वपरिमाणकम्॥९५॥
स्थितिस्थापक इत्येते स्युः कारणगुणोद्भवाः~।
अपाकजास्त्विति~। पाकजरूपादीनां कारणगुणपूर्वकत्वाभावादपाकजा इत्युक्तम्~। तथाविधं - अपाकजम्~। तथैकत्वमपि बोध्यम्~। स्पष्टम्~।
इति कारणगणोत्पन्नगुणकथनम्॥
संयोगश्च विभागश्च वेगश्चैते तु कर्मजाः॥९६॥
संयोगश्चेति~। कर्मजन्यत्वं यद्यपि न साधर्म्यं, घटादावतिव्याप्तेः, संयोगजसंयोगेऽव्याप्तेश्च, तथाऽपि कर्मजन्यवृत्तिगुणत्वव्याप्यजातिमत्त्वं बोध्यम्~।
स्पर्शान्तपरिमाणैकपृथक्त्वस्नेहशब्दके~।
भवेदसमवायित्वं, अथ वैशेषिके गुणे॥९७॥
एवमन्यत्राप्यूह्यम्॥९५॥९६॥
॥इति कर्मजगुणकथनम्॥
स्पर्शान्तेति~। स्पर्शोऽत्राऽनुष्णो ग्राह्यः~। पृथक्त्व इत्यत्र त्वप्रत्ययस्य प्रत्येकमन्वयादेकत्वं पृथक्त्वं च ग्राह्यम्~। "पृथकत्वपदेन चैकपृथक्त्वम्~।
भवेदसमवायित्वमिति~। घटादिरूपरसगन्धस्पर्शाः कपालादिरूपरसगन्धस्पर्शेभ्यो भवन्ति~। एवं कपालादिपरिमाणादीनां घटादिपरिमाणाद्यसमवायिकारणत्वम्~। शब्दस्याऽपि
द्वितीयशब्दं प्रति~। एवं स्थितिस्थापकैकपृथक्त्वयोरपि ज्ञेयम्~।
॥इत्यसमवायिकारणगुणकथनम्॥
आत्मनः स्यान्निमित्तत्वं, उष्णास्पर्शगुरुत्वयोः~।
निमित्तत्वमिति~। बुद्ध्यादीनामिच्छादिनिमित्तत्वादिति भावः~।
॥इति निमित्तकारणगुणकथनम्॥
वेगेऽपि च द्रवत्वे च संयोगादिद्वये तथा॥९८॥
द्विधैव कारणत्वं स्यात्, अथ प्रादेशिको भवेत्~।
वैशेषिको विभुगुणः संयोगादिद्वयं तथा॥९९॥
द्विधैवेति~। असमवायिकारणत्वं निमित्तकारणत्वं च~। तथाहि-उष्णस्पर्श उष्णस्पर्शस्याऽसमवायिकारणं, पाकजे निमित्तम्~। वेगो वेगस्पन्दयोरसमवायिकारणम्,
अभिघाते निमित्तम्~। द्रवत्वं द्रवत्वस्पन्दयोरसमवायि, सङ्ग्रहे निमित्तम्~। भेेरीदण्डसंयोगः शब्दे निमित्तं, भेर्याकाशसंयोगोऽसमवायी~। वंशदलद्वयविभागः शब्दे निमित्तं,
वंशदलाकाशविभागोऽसमवायीति~।
॥इति द्विविधकारणगुणकथनम्॥
चक्षुग्र्राह्यं भवेद्रूपं, द्रव्यादेरुपलम्भकम्~।
प्रादेशिकः-अव्याप्यवृत्तिः॥इत्यव्याप्यवृत्तिगुणकथनम्॥
चक्षुरिति~। रूपशब्दप्रयोगः, तथाऽपि नीलपूतादिध्वनुगतजातिविशेऽनुभवसिद्ध एव, रूपशब्दाप्रयोदेऽपि नीलो वर्णः पीतो वर्ण इति वर्णविशेषोल्लेखिनी प्रतीतिरस्त्येव~।
एवं नीलत्वादिकमपि प्रत्यक्षसिद्धम्~।
॥इति रूपत्वजातिसाधनम्॥
न चैकैका एव नीलारुणादि व्यक्तय इत्येकव्यक्तिवृत्तित्वान्नीलत्वादिकं न जातिरिति वाच्यम्~। नीलो न~।अटो रक्त उत्पन्न इत्यादिप्रतीतेर्नीलादेरुत्पाद
विनाशशालितया नानात्वात्~। अन्यथा एकनीलनाशे जगदनीलतामापद्येत~। न च नीलसमवायरक्तसमवाययोरेवोत्पादविनाशविषयकोऽसौ प्रत्यय इति वाच्यम्~। प्रतीत्या
समवायानुल्लेखात्~। न च स एवाऽयं नील इति प्रत्यक्षबलाल्लाघवाच्चैक्यमिति वाच्यम्~। प्रत्यक्षस्य तज्जातीयविषयत्वात्. सैवेयं गुर्जरीतिवत्~। लाघवं तु प्रत्यक्षबाधितम्~।
अन्यथा घटादीनामप्यैक्यप्रसङ्गात्~। उत्पादविनाशबुद्धेः समवायालम्बन त्वापत्तेरिति~। एतेन रसादिकमपि व्याख्यातम्॥इति नीलादिरूपाणामैक्यखण्डनम्॥
चक्षुषः सहकारि स्यात्, शुक्लादिकमनेकधा॥१००॥
चक्षुग्र्राह्यमिति~। चक्षुग्र्राह्यविशेषगुण इत्यर्थः~। एवमग्रेऽपि
द्रव्यादेरिति~। उपलम्भकम्-उपलब्धिकारणम्~। इदमेव विवृणोति--चक्षुष इति~। द्रव्यगुणकर्मसामान्यानां चाक्षुषप्रत्यक्षं प्रति उद्भूतरूपं कारणम्~। इति रूपलक्षणग्रन्थः॥
शुक्लादिकमनेकधेति~। तच्च रूपंनीलपीतरक्तहरितकपिशकर्बुरादिभेदादनेकप्रकरकं भवति॥इति रूपविभागः॥
ननु कथं कर्बुरमतिरिक्तं रूपं भवति~? इत्थम्-नीलपीताद्यवयवारब्धोऽवयवी न तावन्नीरूपः, अप्रत्यक्षत्वप्रसङ्गात्~। नाऽपि व्याप्यवृत्ति नीलादिरूपमुत्पद्यते,
पीतावच्छेदेनाऽपि नीलोपलब्धिप्रसङ्गात्~। नाऽप्यव्याप्यवृत्तिनीलादिकमुत्पद्यते, व्याप्यवृत्तिजातोयगुणानामव्याप्यवृत्तित्वे विरोधात्~। तस्मान्नानाजातीयरूपैरवयविनि विजातीयं
चित्रंं रूपमारभ्यते~। अत एव एकं चित्ररूपमित्यनुभवोऽपि नानारूपकल्पने गोरवात्~।
इत्थं च नीलादीनां पीताद्यारम्भे प्रतिबन्धकत्वकल्पनादवयविनि न पीताद्युत्पत्तिः~।
एतेन स्पर्शोऽपि व्याख्यातः~।
रसादिकमपि नाऽव्याप्यवृत्ति~। किं तु नानाजातीयरसवदवयवैरारब्धेऽवयविनि रसाभावेऽपि न क्षंतिः, तत्र रसनयाऽवयवरस एव गृह्यते, रसनेन्द्रियादीनां द्रव्यग्रहे
सामथ्र्र्याभावात् अवयविनो नीरसत्वेऽपि क्षतेरभावात्~।
नव्यास्तु तत्राऽव्याप्यवृत्त्येव नानारूपं, नीलादेः पीतादिप्रतिबन्धकत्वकल्पने गौरवात्~। अत एव-
""लोहितो यस्तु वर्णेन मुखे पुच्छे च पाण्डुरः~।
श्वेतः खुरविषाणाभ्यां स नीलो वृष उच्यते-""
इत्यादिशास्त्रमप्युपपद्यते~। न च व्याप्याव्याप्यवृत्तिजातीयद्वयोर्विरोधः, मानाभावात्~। न च लाघवादेकं रूपं, अनुभवविरोधात्~। अन्यथा घटादेरपि लाघवादैक्यं स्यात्~।
एतेन स्पर्शादिकमपि व्याख्यातमिति वदन्ति~।. १००॥इति चित्ररूपनिरूपणम्~।
जलादिपरमाणौ तन्नित्यमन्यत् सहेतुकम्~।
जलादीति~। जलपरमाणौ तेजःपरमाणौ च रूपं नित्यं, पृथिवीपरमाणुरूपं तु न नित्यं, तत्र पाकेन रूपान्तरोत्पत्तेः~। न हि घटस्य पाकानन्तरं तदवयवोऽपक्व
उपलभ्यते~। न हि रक्तकपालस्य कपालिका नीलवयवा भवति~। एवं क्रमेण परमाणावपि पाकसिद्धेः~। अन्यत् - जलतेजःपरमाणुरूपभिन्नं रूपं सहेतुकम् - जन्यम्~। इति
रूपग्रन्थः~।
रसस्तु रसनाग्राह्यो मधुरादिरनेकधा॥१०१॥
सहकारी रसज्ञाया नित्यतादि च पूर्ववत्~।
घ्राणग्राह्यो भवेद्गन्धो घ्राणस्यैवोपकारकः॥१०२॥
सौरभं चाऽसौरभं च स द्वेधा परिकीर्तितः~।
स्पर्शस्त्वगिन्द्रियग्राह्यस्त्वचः स्यादुपकारकः॥१०३॥
अनुष्णाशीतोशीताष्णभेदात् स त्रिविधो मतः~।
काठिन्यादि क्षितावेव नित्यतादि च पूर्ववत्॥१०४॥
रसं निरूपयति~। रसस्त्विवति~। सहकारीति~। रासनज्ञाने रसः कारणमित्यर्थः~। पूर्ववदिति~। जलपरमाणौ रसो नित्यः, अन्यः सर्वोऽपि रसोऽनित्य इत्यर्थः~। इति
रसग्रन्थः॥
गन्धं निरूपयति~। घ्राणग्राह्य इति~। उपकारक इति~। घ्राणजन्यज्ञाने सहकारी स इत्यर्थः~। सर्वोऽपि गन्धोऽनित्य एव~। इति गन्धग्रन्थः॥
स्पर्शं निरूपयति~। स्पर्श इति~। उपकारक इति~। स्पार्शनप्रत्यक्षे स्पर्शः कारणमित्यर्थः~। अनुष्णाशीतेति~। पृथिव्यां वायौ च स्पर्शोऽनुष्णाशीतः~। जले शीतः~।
तेजस्युष्णः~। कठिनसुकुमारस्पर्शै पृथिव्यामेवेत्यर्थः~। कठिनत्वादिकं तु न संयोगगतो जातिविशेषः, चक्षुग्र्राह्यत्वापत्तेः~। पूर्ववदिति~। जलतेजोवायुपरमाणुस्पर्शा नित्याः,
तद्भिन्नास्त्वनित्या इत्यर्थः॥१०१-१०४॥इति स्पर्शग्रन्थः॥
एतेषां-रूपरसगन्धस्पर्शानाम्~। नाऽन्यत्रेति~। पृथिव्यां हि रूपरसगन्धस्पर्शपरावृत्तिरग्निसंयोगादुपलभ्यते~। न हि शतधाऽपि ध्मायमाने जले रूपादिकं परिवर्तते~। नीरे
सौरभमौष्ण्यं चाऽन्वयव्यतिरेकाभ्यामौपाधिकमेवेति निर्णीयते, पवनपृथिव्योः शीतस्पर्शादिवत्~।
तत्राऽपि-पृथिवीष्वपि मध्ये, परमाणावेव रूपादीनां पाक इति वैशेषिका वदन्ति~। तेषामयमाशयः-अवयविनाऽवष्टब्धेष्ववयवेषु पाको न सम्भवति~। परन्तु
वह्निसंयोगेनाऽवयविषु विनष्टेषु स्वतन्त्रेषु परमाणुषु पाकः~। पुनश्च पक्वपरमाणुसंयोगात् द्रव्यणुकादिक्रमेण पुनर्महावयविपर्यन्तमुत्पत्तिः~। तेजसामतिशयितवेगवशात्
पूर्वव्यूहनाशो झटिति व्यूहान्तरोत्पत्तिश्चेति~।
अत्र द्रव्यणुकादिविनाशमारभ्य कतिभिः क्षणैः पुनरुत्पत्त्या रूपादिमद्भवतीति शिष्यबुद्धिवैशद्यार्थे क्षणप्रक्रिया~।
तत्र विभागजविभागानङ्गीकारे नवक्षणा~।
तदङ्गीकारे तु विभागः किञ्चित्सापेक्षो विभागं जनयेत्~। निरपेक्षस्य तत्त्वे कर्मत्वं स्यात्~। "संयोगविभागयोरनपेक्षं कारणं कर्म"-इति हि वैशेषिकसूत्रम्~।
स्वोत्तरोत्पन्नभावानपेक्षत्वं तस्याऽर्थः~। अन्यथा कर्मणोऽप्युत्तरसंयोगोत्पत्तो पूर्वसंयोगनाशापेक्षणादव्याप्तिः स्यादिति~। तत्र यदि द्रव्यारम्भकसंयोगविनाशविशिष्टं कालमपेक्ष्य
विभागजविभागः स्यात्, तदा दशक्षणा~।
अथ द्रव्यनाशविशिष्टं कालमपेक्ष्य विभागजविभागः स्यात् , तदैकादशक्षणा~। तथा हि -
अथ नवक्षणा -
वह्निसंयोगात् परमाणो कर्मं , ततः परमाण्वन्तरेण विभागः, तत आरम्भतसंयोगनाशः, ततो झणुकनाशः १, ततः परमाणौ श्यामादिनाशः २, ततो रक्ताद्युत्पत्तिः ३,
ततो द्रव्यारम्भानुगुणा क्रिया ४, ततो विभागः ५, ततः पूर्वसंयोगनाशः ६, ततः आरम्भकसंयोगः ७, ततो द्व्यणुकोत्पत्तिः ८, ततो रूपाद्युत्पत्तिः~। इति नवक्षणाः~।
ननु श्यामादिनाशक्षणे रक्तोत्पत्तिक्षणे वा परमाणो द्रव्यारम्भानुगुणा क्रियाऽस्त्विति चेन्न~। अग्रिसंयुक्ते परमाणौ यत् कर्म, तद्विनाशमन्तरेण, गुणोत्पत्तिमन्तरेण च
परमाणौ क्रियान्तराभावात्, कर्मवति कर्मान्तरानुत्पत्तेः, निर्गुणे द्रव्ये द्व्र्यारम्भानुगुणक्रियानुपपत्तेश्च~। तथाऽपि परमाणो श्यामादि निवृत्तिसमकालं रक्ताद्युत्पत्तिः स्यादति
चेन्न~। पूर्वरूपादिध्वंसस्याऽपिरबपान्तरे हेतुत्वादिति~।
अथ दशक्षणा~।
सा चाऽऽरम्भकसंयोगविनाशविशिष्टं कालमपेक्ष्य विभागेन विभागजनने सति स्यात्~। तथाहि-वह्निसंयोगात् द्व्यणुकारम्भके परमाणौ कर्म, ततो विभागः, ततः
आरम्भकसंयोगनाशः, ततो द्रव्यणुकनाश-विभागजविभागौ १, ततः श्यामनाश-पूर्वसंयोगनाशौ २, ततो रक्तोत्पत्त्युत्तरसंयोगौ ३, ततो वह्निनोदनजन्यपरमाणुकर्मणो
नाशः४, ततोऽदृष्टवदात्मसंयोगात् द्रव्यारम्भानुगुणा क्रिया५, ततो विभागः ६, ततश्च पूर्वसयोगनाशः७, तत आरम्भकसंयोगः८, ततो द्रव्यणुकोत्पत्तिः६, ततो रक्तोत्पत्तिः
१०-इति दशक्षणा~। अथैकादशक्षणा-
वह्निसंयोगात् परमाणौ कर्म, ततो विभागः, ततो द्रव्यारम्भकसंयोगनाशः ततो द्वयणुकनाशः १, ततो द्रव्यणुकनाशविशिष्टं कालमपेक्ष्य विभागजविभाग-
श्यामनाशौ २, ततः पूर्वसंयोगनाश-रक्तोत्पत्ती ३, तत उत्तरदेशसंयोगः ४, ततो वह्निनोदनजन्यपरमाणुकर्मनाशः४, ततोऽदृष्टवदात्मसंयोगाद्द्रव्यारम्भानुगुणा क्रिया ६,
ततो विभागः ७, ततः पूर्वसंयोगनाशः ८, ततो द्रव्यारम्भकोत्तरसंयोगः ६, ततो द्व्यणुकोत्पत्तिः १०, ततो रक्ताद्युत्पत्तिः ११॥इत्येकादशक्षणा~।
मध्यमशब्दवदेकस्मादग्निसंयोगान्न रूपनाशोत्पादौ, तावत्कालमेकस्याऽग्नेरस्थिरत्वात्~। किं च नाशक एव यद्युत्पादकस्तदा नष्टे रूपादाववÐग्ननाशे नीरूपश्चिरं
परमाणुः स्यात्,
उत्पादकश्चेन्नाशकः, तदा रक्तोत्पत्तौ तदग्नेर्नाशे रक्ततरता न स्यात्~।
अथ परमाण्वन्तरे कर्मचिन्तनात् पञ्चमादिक्षणेऽपि गुणोत्पत्तिः~। तथाहि एकत्र परमाणौ, कर्म, ततो विभागः, तत् आरम्भकसंयोगनाश-परमाण्वन्तरकर्मणीे, ततो
द्रव्यणुकनाशः परमाण्वन्तरकर्मजन्यविभाग इत्येकः कालः १, ततः श्यामादिनाशो विभागाश्च पूर्वसंयोगनाश्चश्चेत्येकः कालः २, तता ेरक्तोत्पत्तिः द्रव्यारम्भकसंयोग
इत्येकः कालः ३, अथ द्वयणुकोत्पत्तिः ४, ततो रक्तोत्पत्तिः~। इति पञ्चक्षणा॥
द्रव्यनाशनसमकालं परमाण्वन्तरे कर्मचिन्तनात् षष्ठे गुणोत्पत्तिः~। तथाहि-परमाणौ कर्म, तत आरम्भकसंयोगनाशः, ततो झणुकनाशः १, ततः श्यामनाशः २, ततो
रक्तोत्पत्ति-परमाण्वन्तरकर्मणी ३, ततः परमाण्वन्तरकर्मजविभागः ४, ततः पूर्वसंयोगनाशः ५, ततः परमाण्वन्तरसंयोगः ६, ततो झणुकोत्पत्तिः ७, अथ रक्तोत्पत्तिः
८-इत्यष्टक्षणा॥१०२॥इति पीलुपाकवादनिरूपणम्~।
नैयायिकानां तु नये द्रव्यणुकादावपीष्यते~।
नैयाभिकानामिति~। नैयायिकानां मते द्रव्यणुकादाववभवियपि पाको भवति~।
तेषामयमाशयः, अवयविनां सच्छिद्रत्वाद्वह्रेः सूक्ष्मावयवैरन्तः प्रविष्टैरवयवेष्ववष्टब्धेष्वपि पाको न विरुद्ध्यते~। अनन्तावयवितन्नाशकल्पने गौरवम्~। इत्थं च सोऽयं
घट इत्यादिप्रत्यभिज्ञाऽपि सङ्गच्छते~। यत्र तु न प्रत्यभिज्ञा, तत्राऽवयविनाशोऽपि स्वीक्रियत इति~। इति पिठरपाकवादनिरूपणम्~।
गणानाव्यवहारे तु हेतुः सङ्ख्याऽभिधीयते॥१०६॥
सङ्ख्यां निरूपयितुमाह~। गणनेति~। गणनाव्यवहारासाधारणं कारणं सङ्ख्येत्यर्थः॥१०६॥
नित्येषु नित्यमेकत्वमनित्येऽनित्यमिष्यते~।
नित्येष्विति~। नित्येषु-परमाण्वादिषु एकत्वं नित्यम्~। अनित्ये-घटादावेकत्वमनित्यमित्यर्थः~।
द्वित्वादयः परार्धान्ता अपेक्षाबुद्धिजा मताः॥१०७॥
द्वित्वादय इति~। द्वित्वादयो व्यासज्यवृत्तिसङ्ख्या अपेक्षाबुद्धिजन्याः॥१०३॥
अनेकाश्रयपर्याप्ता एते तु परिकीर्तितः~।
अनेकेति~। यद्यपि द्वित्वादिसमवायः प्रत्येकं घटादावपि वर्र्तते, तथाऽपि एको द्वाविति प्रत्ययाभावात् एको न द्वाविति प्रत्ययसद्भावाच्च द्वित्वादीनां पर्याप्तिलक्षणः
कश्चित् सम्बन्धोऽनेकाश्रयोऽभ्युपेयते~।
अपेक्षाबुद्धिनाशाच्च नाशस्तेषां निरूपितः॥१०८॥
प्रथममपेक्षाबुद्धिः, ततो द्वित्वोत्पत्तिः, ततो विशेषणज्ञानं द्वित्वत्वनिर्विकल्पात्मकं, ततो द्वित्वत्वाविशिष्टप्रत्यक्षमपेक्षाबुद्धिनाशश्च, ततो द्वित्वनाश इति~।
यद्यपि ज्ञानानां द्विक्षणमात्रस्थायित्वम्, योग्यविभुविशेषगुणानां स्वोत्तरवृत्तिगुणनाश्यत्वात्, तथाऽप्यपेक्षाबुद्धेस्त्रिक्षणावस्थायित्वं कल्प्यते~। अन्यथा
निर्विकल्पककालेऽपेक्षाबुद्धिनाशानन्तरं द्वित्वस्यैव नाशः स्यात्, न तु द्वित्वप्रत्यक्षं, तदानीं विषयाभावात्, विद्यमानस्यैव चक्षुरादिना ज्ञानजननोपगमात्~। तस्मात्
द्वित्वप्रत्यक्षादिकमपेक्षाबुद्धेर्नाशकं कल्प्यते~।
न चाऽपेक्षाबुद्धिनाशात् कथं द्वित्वनाश इति वाच्यम्~। कालान्तरे द्वित्वप्रत्यक्षाभावात् अपेक्षाबुद्धिस्तदुत्पादिका, तन्नाशात्तन्नाश इति कल्पनात्~। अत एव
तत्पुरुषीयापेक्षाबुद्धिजन्यद्वित्वादिकं तेनैव गृह्यत इति कल्प्यते~। न चाऽपेक्षाबुद्धेर्द्वित्वप्रत्यक्षे कारणत्वमस्त्विति वाच्यम्~। लाघवेन द्वित्वं प्रत्येव कारणत्वस्योचितत्वात्~।
अतीन्द्रिये द्व्यणुकादावपेक्षाबुद्धिर्योगिनां, सर्गादिकालीनपरमाण्वादावीश्वरीयापेक्षाबुद्धिः, ब्रह्माण्डान्तरवर्तियोगिनामपेक्षाबुद्धिर्वा द्वित्वादिकारणमिति॥१०८॥
अनेकैकत्वबुद्धिर्या साऽपेक्षाबुद्धिरिष्यते~।
अपेक्षाबुद्धिः केत्यत आह अनेकैकत्वेति~। अयमेकोऽयमेक इत्याकारिकेत्यर्थः~। इदं तु बोध्यम्~। यत्राऽनियतैकत्वज्ञानं, तत्र त्रित्वादिभिन्ना बहुत्वसङ्ख्योेत्पद्यते, यथा
सेनावनादाविति कन्दलीेकारः~।
आचार्यास्तु त्रित्वादिकमेव बहुत्वं मन्यन्ते~। तथा च त्रित्वादिव्यापिका बहुत्वत्वजातिर्नाऽतिरिच्यते~। सेनावनादावुत्पन्नेऽपि त्रित्वादौ त्रित्वाद्यग्रहो दोषात्~। इत्थं चेतो
बहुतरेयं सेनेति प्रतीतिरुपपद्यते~। बहुत्वस्य सङ्ख्यान्तरत्वे तु तत्तारतम्याभावान्नोपपद्येतेत्यवधेयम्~। इति सङ्ख्यानिरूपणम्~।
परिमाणं भवेन्मानव्यवहारस्य कारणम्॥१०९॥
अणु दीर्घं महद्ध्रस्वमिति तद्भेद ईरितः
परिमाणं निरूपयति-परिमाणमिति~। परिमितिव्यवहारासाधारणं कारणं परिमाणमित्यर्थः~। तच्चतुर्विधम्, अणु, महत्, दीर्र्घं, हस्वं चेति~।
अनित्ये तदनित्यं स्यान्नित्ये नित्यमुदाहृतम्॥११०॥
सङ्ख्यातः परिमाणाच्च प्रचयादपि जायते~।
अनित्यं, द्व्यणुकादौ तु संख्याजन्यमुदाहृतम्॥१११॥
तत्-परिमाणाम्~। "नित्यं" इत्यत्र परिमाणमित्यनुषज्यते~। "जायते" इत्यत्राऽपि परिमाणमित्यनुवर्तते~। अनित्यमिति पूर्वेणान्वितम्~। तथा चाऽनित्यपरिमाणं
सङ्ख्याजन्यं, परिमाणजन्यं, प्रचयजन्यं चेत्यर्थः~।
तत्र सङ्ख्याजन्यमुदाहरति-द्व्यणुकादाविति~। द्व्यणुकस्य त्रसरेणोश्च परिमाणं प्रति परमाणुपरिमाणं द्व्यणुकपरिमाणं वा न कारणम्, परिमाणस्य
स्वसमानजातीयोत्कृष्टपरिमाणजनकत्वनियमात्~। द्व्यणुकस्याऽणुपरिमाणं तु परमाण्वणुत्वापेक्षया नोत्कृष्टम्~। त्रसरेणुपरिमाणं तु न सजातीयम्, अतः परमाणौ द्वित्वसङ्ख्या
द्व्यणुकपरिमाणस्य, द्व्यणुके त्रित्वसङ्ख्या च त्रसरेणुपरिमाणस्याऽसमवायिकारणमित्यर्थः॥१०९॥११०॥१११॥
परिमाणं घटादौ तु परिमाणजमुच्यते~।
परिमाणजन्यं परिमाणमुदाहरति---परिमाणं घटादाविति~। घटादिपरिमाणं कपालादिपरिमाणजन्यम्~।
प्रचयः शिथिलाख्यो यः संयोगस्तेन जन्यते॥११२॥
प्रचयजन्यमुदाहर्तुं प्रचयं निर्वक्ति~। प्रचय इति~।
परिमाणं तूलकादौ, नाशस्त्वाश्रयनाशतः~।
परिमाणं चाऽऽश्रयनाशादेव नश्यतीत्याह~। नाश इति~। अर्थात् परिमाणस्यैव~। न चाऽवयविनाशः कथं परिमाणनाशकः~? सत्यप्यवयविनि त्रिचतुरादिपरमाणुविश्लेषे
तदुपचये वाऽवयविनः प्रत्यभिज्ञानेऽपि परिमाणान्तरस्य प्रत्यक्षसिद्धत्वादिति वाच्यम्~। परमाणुविश्लेषे हि द्व्यणुकस्य नाशोऽवश्यमभ्युपेयः~। तन्नाशे च त्र्यणुकनाशः~। एवं
क्रमेण महावयविनो नाशस्याऽऽवश्यकत्वात्~। सति च नाशकेऽनभ्युपगममात्रेण नाशस्याऽपलपितुमशक्यत्वात्~। शरीरादाववयवोपचयेऽसमवायि
कारणनाशस्याऽऽवश्यकत्वादवयविनाश आवश्यकः~। न च पटाद्यनाशेऽपि तन्त्विन्तरसंयोगात् परिमाणाधिक्यं न स्यादिति वाच्यम्~। तत्राऽपि
वेमाद्यभिघातेनाऽसमवायिकारणतन्तुसंयोगनाशात् पटनाशस्याऽऽवश्यकत्वात्~। किंच तन्त्वन्तरस्य तत्पटावयवत्वे पूर्वं तत्पट एव न स्यात्, तन्त्वन्तररूपकारणभावात्~।
तन्त्वन्तरस्याऽवयवत्वाभावे च न तेन परिमाणाधिक्यं, संयुक्तद्रव्यान्तरवत्~। तस्मात्तत्र तन्त्वन्तरसंयोगे सति पूर्वं पटनाशस्ततः पटान्तरोत्पत्तिरित्यवश्यं स्वीकार्यम्~।
अवयविनः प्रत्यभिज्ञानं तु साजात्येन, दीपकलिकादिवत् न च पूर्वतन्तव एव तन्त्वन्तरसहकारात् पूर्वपटे सत्येव पटान्तरमारभन्तामिति वाच्यम्~। मूर्तयोः समानदेशताविरोधात्
तत्र पटद्वयासम्भवात्, एकदा नानाद्रव्यस्य तत्रोपलम्भस्य बाधितत्वच्च~। तस्मात् पूर्वस्य द्रव्यस्य प्रतिबन्धकस्य विनाशे द्रव्यान्तरोत्पत्तिरित्यस्याऽवश्यमभ्युपेयत्वात्॥इति
परिमाणनिरूपणम्॥
संयोगं निरूपयति~। अप्राप्तयोरिति~।११२
तं विभजते~। कीर्तित इति~। एषः - संयोगः~।
संख्यावत्तु पृथक्त्वं स्यात् पृथक्प्रत्ययकारणम्॥११३॥
अन्योन्याभावतो नाऽस्य चरितार्थत्वमिष्यते~।
अस्मात् पृथगिदं नेति प्रतीतिर्हि विलक्षणा॥११४॥
पृथक्त्वं निरूपयति~। सङ्ख्यावदिति~। पृथक्प्रत्ययासाधारणकारणं पृथक्त्वम्~। तन्नित्यतादिकं सङ्ख्यावत्~। तथाहि नित्येष्वेकत्वं नित्यमनित्येष्वनित्यम्~। अनित्यमेकत्वं
तु आश्रयद्वितीयक्षणे चोत्पद्यते~। आश्रयनाशान्नश्यति~। तथैकपृथक्त्वं~। द्वित्वादिवच्च द्विपृथक्त्वादिकमपीत्यर्थः~।
नन्वयमस्मात् पथगित्यादावन्योन्याभावो भासते, तत् कथं पृथक्त्वं गुणान्तरं स्वीक्रियते~? न चाऽस्तु पृथक्त्वं, न त्वन्योन्याभाव इति वाच्यम्~। रूपं न घट इति
प्रतीत्यनापत्तेः~। न हि रूपे घटावधिकं पृथक्त्वं गुणान्तरमस्ति न वा घटे घटावधिकं पृथक्त्वमस्ति, येन परम्परासम्बम्धः कल्प्यत इत्यत आह~। अस्मादिति~। ननु
शब्दवैलक्षण्यमेव, न त्वर्थवैलक्षण्यमिति चेन्न~। विनाऽर्थभेदं घटात् पृथगितिवद्घटो न पट इत्यत्राऽपि पञ्चमीप्रसङ्गात्~। तस्माद्यदर्थयागे पञ्चमी सोेऽर्थो नञर्थान्योन्याभावतो
भिन्नो गुणान्तरं कल्प्यत इति॥११२॥११३~।११४॥इति प्रथक्त्वनिरूपणम्॥
अप्राप्तयोस्तु या प्राप्तिः सैव संयोग ईरितः~।
कीर्तितस्त्रिविघस्त्वेष आद्योऽन्यतरकर्मजः॥११५॥
तथोभयस्पन्दजन्यो भवेत् संयोगजोऽपरः~।
आदिमः श्येनशैलादिमंयोगः परिकीर्तितः॥११६॥
मेषयोः सन्निपातो यः स द्वितीय उदाहृतः~।
कपालतरुसंयोगात् संयोगस्तरुकुम्भयोः॥११७॥
सन्निपातः-संयोगः~। द्वितीयः - उभयकर्मजः~।
तृतीयः स्यात् कर्मजोऽपि द्विधैव परिकीर्तितः~।
अभिघातो नोदनं च शब्दहेतुरिहादिमः॥११८॥
तृतीय इति~। संयोगजसंयोग इत्यर्थः तृतीयः स्यात् " इति पूर्वेणाऽन्वितम्~।
आदिमः-अभिघातः~। द्वितीयो नोदनाख्यः इति~।. इति संयोगनिरूपणम्~।
शब्दाहेतुर्द्वितीयः स्यात् विभागोऽपि त्रिधा भवेत्~।
एककर्मोद्भवस्त्वाद्यो द्वयकर्मोद्भवोऽपरः॥११९॥
विभागजस्तृतीयः स्यात् तृतीयोऽपि द्विधा भवेत्~।
हेतुमात्रविभागोत्थोे हेत्वहेतुविभागजः॥१२०
विभक्तप्रत्ययासाधारणकारणं विभागं निरूपयति~। विभाग इति~।
एककर्मेति~। तदुदाहरणं तु श्येनशैलविभागादिकं पूर्ववद्बोध्यम्~।
तृतीयोऽपि-विभागजविभागः कारणामात्रविभागजन्यः कारणकारणविभागजन्यश्चेति द्विविधः, आद्यस्तावद्यत्रैककपाले कर्म, ततः कपालद्वयविभागः, ततो
घटारम्भकसंयोगनाशः, ततो घटनाशः, ततस्तेनैव कपालविभागेन सकर्मणः कपालस्याऽऽकाशविभागो जन्यते~। तत आकाशसंयोगनाशः, तत उत्तरदेशसंयोगः, ततः
कर्मनाश इति~। न च तेन कर्मणैव कथं देशान्तरविभागो न जन्यत इति वाच्यम्~। णकस्य कर्मण आरम्भकसंयोगप्रतिद्वन्द्विविभागजनकत्वस्याऽनारम्भकसंयोगप्रतिद्वन्द्विविभागजनकत्वस्य
च विरोधात्~। अन्यथा विकसत्कमलकुड्मलभङ्गप्रसङ्गात्~। तस्माद्यदीदमनारम्भकसंयोगप्रतिद्वन्द्विविभागं जनयेत्, तदाऽऽरम्भकसंयोगप्रतिद्वन्द्विविभागं न जनयेत्~। न च
कारणविभागेनैव द्रव्यनाशात् पूर्वं कुतो देशान्तरविभागो न जन्यत इति वाच्यम्~। आरम्भकसंयोगप्रतिद्वन्द्विविभागवतोऽवयवस्य सति द्रव्ये देशान्तरविभागासम्भवात्~।
द्वितीयस्तावद्यत्र हस्तक्रियया हस्ततरुविभागस्ततः शरीरेऽपि विभक्तप्रत्ययो भवति, तत्र शरीरतरुविभागे हस्तक्रिया न कारणं, व्यधिकरणत्वात्~। शरीरे तु क्रिया नास्त्वेव,
अवयविकर्मणे यावदवयवकर्मनियतत्वात्~। अतस्तत्र कारणकारणाविभागेन कार्याकार्यविभागो जन्यत इति~। अत एव विभागो गुणान्तरम्~। अन्यथा शरीरे विभक्तप्रत्ययो
न स्यात्~। अतः संयोगनाशेन विभागो नाऽन्यथासिद्धो भवति॥११६॥॥११७॥११८॥११९॥१२०॥॥इति विभागनिरूपणम्॥
परत्वं चाऽपरत्वं च द्विविधं परिकीर्तितम्~।
दैशिकं कालिकं चाऽपि मूर्त एव तु दैशिकम्॥१२१॥
परत्वं मूर्तसंयोगभूयस्त्वज्ञानतो भवेत्~।
अपरत्वं तदल्पत्वबुद्धितः स्यादितीरितम्॥१२२॥
तयोरसमवायी तु दिक्संयोगस्तदाश्रये~।
दिवाकरपरिस्पन्दभूयस्त्वज्ञानतो भवेत्॥१२३॥
परत्वमपरत्वं तु तदीयाल्पत्वबुद्धितः॥
अत्र त्वसमवायी स्यात् संयोगः कालपिण्डयोः॥१२४॥
अपेक्षाबुद्धिनाशेन नाशस्तेवां निरूपितः~।
परापरव्यवहारनिमित्ते परत्वापरत्वे निरूपयति~। परत्वञ्चापरत्त्वं चेति~। दैशिकमिति~। दैशिकपरत्वं बहुतरमूर्तसंयोगान्तरितत्वज्ञानादुत्पद्यते~। एवं
तदल्पीयस्त्वज्ञानादपरत्वमुत्पद्यते~। अत्राऽवधित्वार्थं पञ्चम्यपेक्षा, यथा "पाटलिपुत्रात् काशीमपेक्ष्य प्रयागः परः" पाटलिपुत्रात् कुरुक्षेत्रमपेक्ष्य प्रयागोऽपर इति॥१२१॥
१२२॥
तयोः-दैशिकपरत्वापरत्वयोः~। असमवायी-असमवायिकारणम्~। तदाश्रये--दैशिकपरत्वापरत्वाश्रये~। दिवाकरेति~। अत्र परत्वापरत्वं कालिकं ग्राह्यम्~। यस्य सूर्यपरिस्पन्दापेक्षया
यस्य सूर्यपरिसम्पन्दोऽधिकः, स ज्येष्ठः यस्य न्यूनः स कनिष्ठः~। कालिकपरत्वापरत्वे जन्यद्रव्ये (१) एव~। अत्र-कालिकपरत्वापरत्वयोः~। तेषां-
कालिकदैशिकपरत्वापरत्वानाम्॥१२३॥१२४॥
॥इति परत्वापरत्वनिरूपणम्॥
बुद्धेः प्रपञ्चः प्रागेव प्रायशो विनिरूपितः॥१२५॥
अथाऽवशिष्टोऽप्यपरः प्रकारः परिदश्र्यते~।
क्रमप्राप्तं बुदिं्ध निरूपयति~। बुद्धेरिति~।
अप्रमा च प्रमा चेति ज्ञानं द्विविधमिष्यते॥१२६॥
तच्छून्ये तन्मतिर्या स्यादप्रमा सा निरूपिता~।
तत्प्रपञ्चो विपर्यासः संशयोऽपि प्रकीर्तितः॥१२७॥
तत्र अप्रमां निरूपयति~। तच्छून्य इति~। तदभाववति तत्प्रकारकं ज्ञानं भ्रम इत्यर्थः~। तत्प्रपञ्चः - अप्रमाप्रपञ्चः॥१२५॥१२६॥१२७॥
आद्यो देहेष्वात्मबुद्धिः शङ्खादौ पीततामतिः~।
भवेन्निश्चयरूपा या संशयोऽथ प्रदश्र्यते॥१२८॥
किंस्विन्नरो वा स्थाणुर्वेत्यादिबुद्धिस्तु संशयः~।
तदभावाप्रकारा धीस्तत्प्रकारा तु निश्चयः॥१२९॥
आद्य इति~। विपर्यय इत्यर्थः~। शरीरादौ निश्चयरूपं यदात्मत्वप्रकारकं ज्ञानं निश्चयरूपं तद्भ्रम इति~। किंस्विदिति वितर्के~। इति भ्रमप्रमानिरूपणम्॥
निश्चयस्य लक्षणमाह~। तदभावेति~। तदभावाप्रकारकं तत्प्रकारकं ज्ञानं निश्चयः॥१२८॥१२६॥इति निश्चयनिरूपणम्॥
स संशयो मतिर्या स्यादेकत्राऽभावभावयोः~।
संशयं लक्षयति~। स संशय इति~। एकधर्मिकविरुद्धभावाभावप्रकारकं ज्ञानं संशय इत्यर्थः~।
साधारणादिधर्मस्य ज्ञानं संशयकारणम्॥१३०॥
साधारणेति~। उभयसाधारणो यो धर्मस्तज्ज्ञानं संशयकारणम्~। यथा उच्चैस्तरत्वं स्थाणुपुरुषसाधारणं ज्ञात्वाऽयं स्थाणुर्न वेति सन्दिग्धे~। एवमसाधारणधर्मज्ञानं
कारणम्~। यथा शब्दत्वस्य नित्यानित्यव्यावृत्तत्वेन शब्दे गृहीतत्वाच्छब्दो नित्यो न वेति सन्दिग्धे~। विप्रत्तिपत्तिस्तु शब्दो नित्यो न वेत्यादिशब्दात्मिका न संशयकारणम्,
शब्दव्याप्तिज्ञानादीनां निश्चयमात्रजनकत्वस्वभावात्~। किंतु तत्र शब्देन कोटिद्वयज्ञानं जन्यते~। संशयस्तु मानस एवेति~। एवं ज्ञाने प्रामाण्यसंशयाद्विषयसंशय इति~। एवं
व्याप्यसंशयादपि व्यापकसंशय इत्यादिकं बोध्यम्~। किंतु संशये धर्मिज्ञानं धर्मीन्द्रियसन्निकर्षो वा कारणमिति॥१३०॥इति संशयनिरूपणम्॥
दोषोऽप्रमाया जनकः प्रमायास्तु गुणो भवेत्~।
पित्तदूरत्वादिरूपो दोषो नानाविधो मतः॥१३१॥
दोष इति~। अप्रमां प्रति दोषः कारणं, प्रमां प्रति गुणः कारणम्~। तत्राऽपि पित्तादिरूपा दोषा अननुगताः~। तेषां कारणत्वमन्वयव्यतिरेकाभ्यां सिद्धम्~। गुणस्य
प्रमाजनकत्वं त्वनुमानात् सिद्धम्~। यथा प्रमा ज्ञानसामान्यकारणभिन्नकारणजन्या, जन्यज्ञानत्वात्, अप्रमावत्~। न च दोषाभाव एव कारणमस्त्विति वाच्यम्~। पीतः शङ्ख
इतिज्ञानस्थलेऽपि पित्तरूपदोषसत्त्वाच्छङ्खत्वप्रमानुत्पत्तिप्रसङ्गात्, विनिगमनाविरहादनन्तदोषाभावकारणत्वमपेक्ष्य गुणस्य कारणतायान्याय्यत्वात्~। न च गुणसत्त्वेऽपि
पित्तेन प्रतिबन्धाच्छङ्खे न श्वैत्यज्ञानम्, अतः पित्तादिदोषाभावानां कारणत्वमवश्यं वाच्यम्~। तथा च किं गुणस्य हेतुत्वकल्पनेनेति वाच्यम्~। तथाऽप्यन्वयव्यतिरेकाभ्यां
गुणस्याऽपि हेतुत्वसिद्धेः~। एवं भ्रमं प्रति गुणाभावः कारणमित्यस्यापि सुवचत्वम्~।
तत्र दोषाः क इत्याकाङ्क्षायामाह~। पित्तेति~। क्वचित् पीतादिभ्रमे पित्तं दोषः, क्वचिच्चन्द्रादेः स्वल्पपरिमाणभ्रमे दूरत्वं दोषः, क्वचिच्च वंशोरगभ्रमे मण्डूकवसाञ्जनमित्येवंरूपा
दोषा अननुगता एव भ्रान्तिजनका इत्यर्थः॥१३१॥
प्रत्यक्षे तु विशेष्येण विशेषणवता समम्~।
सन्निकर्षो गुणस्तु स्यात्, अथ त्वनुमितौ पुनः॥१३२॥
पक्षे साध्यविशिष्टे तु परामर्शो गुणो भवेत्~।
शक्ये सादृश्यबुद्धिस्तु भवेदुपमितौ गुणः॥१३३॥
शाब्दबोधे योग्यतायास्तात्पर्यस्याऽथवा प्रमा~।
गुणः स्यात्, भ्रमभिन्नं तु ज्ञानमत्रोच्यते प्रमा॥१३४॥
अथ के गुणा इत्याकाङ्क्षायां प्रत्यक्षादौ क्रमेण गुणान् दर्शयति~। प्रत्यक्षेत्विति~। प्रत्यक्षे विशेषणवद्द्विशेष्यसन्निकर्षो गुणः~।
अनुमितौ साध्यवति साध्यव्याप्यवैशिष्ट्यज्ञानं गुणः~। एवमग्रेऽपि ज्ञेयम्~।
॥इति प्रामाण्यवाद उत्पत्तिवादः॥
अथवा तत्प्रकारं यज्ज्ञानं तद्वद्विशेष्यकम्~।
तत्प्रमा, न प्रमा नाऽपि भ्रमः स्यान्निर्विकल्पकम्॥१३५॥
प्रकारतादिशून्यं हि सम्बन्धानवगाहि तत्~।
प्रमां निरूपयति~। भ्रमभिन्नमिति~।
ननु यत्र शुक्तिरजतयोरिमे रजते इति ज्ञानं जातं, तत्र रजतांशेऽपि प्रमा न स्यात्~। तज्ज्ञानस्य भ्रमभिन्नत्वाभावादत् आह~। अथवेति~। तद्वद्विशेष्यकत्वे सति तत्प्रकारकं
ज्ञानं प्रमेत्यर्थः~। अथैवं स्मृतेरपि प्रमात्वं स्यात्~। ततः किमिति चेत्, तथा सति तत्करणस्याऽपि प्रमाणान्तरत्वं स्यादिति चेन्न~। यथार्थानुभवकरणस्यैव प्रमाणत्वेन
विवक्षितत्वात्~।
इदं तु बोध्यम्~। येन सम्बन्धेन यद्वत्ता तेन सम्बन्धेन तद्वद्विशेष्यकत्वं तेन सम्बन्धेन तत्प्रकारकत्वं च वाच्यम्~। तेन कपालादौ संयोगादिना घटादिज्ञाने
नाऽतिव्याप्तिः~।
एवं सति निर्विकल्पकं प्रमा न स्यात्, तस्य सप्रकारकत्वाभावादत आह~। न प्रमेति~। ननु वृक्षे कपिसंयोगज्ञानं भ्रमः प्रमा च स्यादिति चेन्न~।
प्रतियोगिव्यधिकरणकपिसंयोगाभाववति संयोगज्ञानस्य भ्रमत्वात्~। न च वृक्षे कपिसंयोगाभावावच्छेदेन संयोगज्ञानं भ्रमो न स्यात्, तत्र संयोगाभावस्य प्रतियोगिसमानाधिकरणत्वादिति
वाच्यम्~। तत्र संयोगाभावावच्छेदेन संयोगज्ञानस्य भ्रमत्वात्~। लक्ष्यस्याऽननुगमाल्लक्षणस्याऽननुगमेऽपि न क्षतिः~।
॥इति प्रमालक्षणम्॥
प्रमात्वं न स्वतो ग्राह्यं संशयानुपपत्तितः॥१३६॥
प्रमात्वमिति~। मीमांसका हि प्रमात्वं स्वतोग्राह्यमिति वदन्ति~। तत्र गुरूणां मते ज्ञानस्य स्वप्रकाशत्वात्तज्ज्ञानप्रामाण्यं तेनैव गृह्यते~। भट्टानां मते ज्ञानमतीन्द्रियम्,
ज्ञानजन्यज्ञातता प्रत्यक्षा तया च ज्ञानमनुमीयते~। मुरारिमिश्राणां मतेऽनुव्यवसायेन ज्ञानं गृह्यते~। सर्वेषामपि मते तज्ज्ञानविषयकज्ञानेन तज्ज्ञानप्रामाण्यं गृह्यते~।
विषयनिरूप्यं हि ज्ञानमतो ज्ञानवित्तिवेद्यो विषयः~। तन्मतं दूषयति~। न स्वतो ग्राह्यमिति~। संशयेति~। यदि ज्ञानस्य प्रामाण्यं स्वतो ग्राह्यं स्यात्, तदाऽनभ्यासदशापन्नज्ञाने
प्रामाण्यसंशयो न स्यात्~। तत्र हि यदि ज्ञानं ज्ञातं, तदा त्वन्मते प्रामाण्यं ज्ञातमेवेति कथं संशयः~? यदि तु ज्ञानं न ज्ञातं, तर्हि धर्मिज्ञानाभावात् कथं संशयः~? तस्माद् ज्ञाने
प्रामाण्यमनुमेयम्~। तथाहि इदं ज्ञानं प्रमा, संवादिप्रवृत्तिजनकत्वात्, यन्नैवं तन्नैवं, यथाऽप्रमा~। इदं पृथिवीत्वप्रकारकं ज्ञानं प्रमा, स्नेहवति जलत्वप्रकारकज्ञानत्वात्~। न च
हेतुज्ञानं कथं जातमिति वाच्यम्~। पृथिवीत्वप्रकारकत्वस्य स्वतोग्राह्यत्वात्, तत्र गन्धग्रहेण गन्धवद्विशेष्यकत्वस्याऽपि सुग्रहत्वात्, तत्प्रकारकत्वावच्छिन्नतद्विद्विशेष्यकत्वं परं
न गृह्यते, संशयानुरोधात्॥इति प्रामाण्यवादे ज्ञप्तिवादः॥
ननु सर्वेषां ज्ञानानां यथार्थत्वात् प्रमालक्षणे तद्वद्विशेष्यकत्वं विशेषणं व्यर्थम्~। न च रङ्गे रजतार्थिनः प्रवृत्तिभ्र्रमजन्या न स्यात्, तव मते भ्रमस्याऽभावादिति
वाच्यम्~। तत्र हि दोषाधीनस्य पुरोर्त्तिनि स्वतन्त्रोपस्थितरजतभेदाग्रहस्य हेतुत्वात्~। सत्यरजतस्थले तु विशिष्टज्ञानस्य सत्त्वात्तदेव कारणम्, अस्तु वा तत्राऽपि रजतभेदाग्रहः~।
स एव कारणमिति~। न चाऽन्यथाख्यातिः सम्भवति~। रजतप्रत्यक्षकारणस्य रजतेन्द्रियसन्निकर्षस्याऽभावात्~। रङ्गे रजतबुद्धेरनुपपत्तेरिति चेन्न~। सत्यरजतस्थले प्रवृर्तिं प्रति
विशिष्टज्ञानस्य हेतुतायाः क्लृप्तत्वात् अन्यत्राऽपि तत्कल्पनात्~। न च संवादिप्रवृत्तौ तत्कारणं, विसंवादिप्रवृत्तौ च भेदाग्रहः कारणमिति वाच्यम्~। लाघवेन प्रवृत्तिमात्रे
तस्य हेतुत्वकल्पनात्~। इत्थं च रङ्गे रजतत्व विशिष्टबुद्ध्यनुरोधेन ज्ञानलक्षणप्रत्यासत्तिकल्पनेऽपि न क्षतिः~। फलमुखगौरवस्यादोषत्वात्~। किंच रङ्गरजतयोरिमे रजते
रङ्गे वेति ज्ञानं यत्र जातं तत्र न कारणबाधोऽपि~। अपि च यत्र रङ्गरजतयोरिमे रजतरङ्गे इति ज्ञानं तत्रोभयत्र युगपत्प्रवृत्तिनिवृत्ती स्याताम्~। रङ्गे रङ्गभेदग्रहे रजते
रजतभेदग्रहे चान्यथाख्यातिभयात्, त्वन्मते दोषादेव रङ्गे रजतभेदाग्रहस्य रजते रङ्गभेदाग्रहस्य च सत्त्वात्~। किंचानुमितिं प्रति भेदाग्रहस्य हेतुत्वे जलह्रदे
वह्निव्याप्यधूमवद्भेदाग्रहादनुमितिर्निराबाधा~। यदि च विशिष्टज्ञानं कारणं तदाऽयोगोलके वह्निव्याप्यधूमज्ञानमनुमित्यनुरोधादापतितम्~। सेयमुभयतःपाशारज्जुः~। इत्थञ्चान्यथाख्यातौ
प्रत्यक्षमेव प्रमाणं, रङ्गं रजततया जानामीत्यनुभवादिति संक्षेपः॥१३६॥इत्यन्यथाख्यातिवादः॥
व्यभिचारस्याग्रहोऽपि सहचारग्रहस्तथा~।
हेतुव्र्याप्तिग्रहे, तर्कः क्वचिच्छङ्कानिवर्तकः॥१३७॥
पूर्वं व्याप्तिरुक्ता तद्ग्रहोपायस्तु न दर्शित इत्यतस्तं दर्शयति व्यभिचारस्येति~। व्यभिचाराग्रहः सहचारग्रहश्च व्याप्तिग्रहे कारणमित्यर्थः~। व्यभिचारग्रहस्य व्याप्तिग्रहे
प्रतिबन्धकत्वात् तदभावः कारणम्~। एवमन्वयव्यतिरेकाभ्यां सहचारग्रहस्यापि हेतुता~। भूयोदर्शनं तु न कारणं, व्यभिचारास्फूर्र्तौै सकृद्दर्शनेऽपि क्वचिद् व्याप्तिग्रहात्,
क्वचिद्व्यभिचारशङ्काविधूननद्वारा भूयोदर्शनमुपयुज्यते॥इति व्याप्तिग्रहोपायनिरूपणम्॥
यत्र तु भूयोदर्शनादपि शङ्का नापैति तत्र विपक्षे बाधकस्तर्कोऽपेक्षितः~। तथाहि वह्निविरहिण्यपि धूमः स्यादिति यद्याशङ्का भवति तदा सा वह्निधूमयोः
कार्यकारणभावस्य प्रतिसन्धानान्निवर्र्तते~। यद्ययं वह्निमान् न स्यात्तदा धूमवान् न स्यात्, कारणं विना कार्यानत्पत्तेः~। यदि क्वचित् कारणं विनापि भविष्यत्यहेतुक एव
भविष्यतीति तत्राप्याशङ्का भवेत् तदा सा व्याघातादपसारणीया~। यदि हि कारणं विना कार्यं स्यात् तदा धूमार्थे वह्नेस्तृप्त्यर्थं भोजनस्य वा नियमत उपादानं तवैव न
स्यादिति~। यत्र स्वत एव शङ्क नावतरति तत्र न तर्कापेक्षापीति~। तदिदमुक्तम्-तर्कः क्वचिदिति॥१३७॥
॥इति तर्कनिरूपणम्॥
साध्यस्य व्यापको यस्तु हेतोरव्यापकस्तथा~।
स उपाधिर्भवेत् तस्य निष्कर्षोऽयं प्रदश्र्यते॥१३८॥
इदानीं (परकीय) व्याप्तिग्रहप्रतिबन्धार्थमुपार्धिं निरूपयति~। साध्यस्येति~। साध्यत्वाभिमतव्यापकत्वे सति साधनत्वाभिमताव्यापक इत्यर्थः॥
ननु स श्यामो मित्रातनयत्वादित्यत्र शाकपाकजत्वं नोपाधिः स्यात्, तस्य साध्यव्यापकत्वाभावात्~। श्यामत्वस्य घटादावपि सत्त्वात्~। एवं वायुः प्रत्यक्षः
प्रत्यक्षस्पर्शाश्रयत्वादित्यत्रोद्भूद्रूपवत्त्वं नोपाधिः स्यात्, प्रत्यक्षत्वस्यात्मादावपि सत्त्वात्~। तत्र च रूपाभावात्~। एवं ध्वंसो विनाशी जन्यत्वादित्यत्र भावत्वं नोपाधिः
स्यात्~। विनाशित्वस्य प्रागभावेऽपि सत्त्वात्~। तत्र च भावत्वाभावादिति चेन्न~। सद्धर्मावच्छिन्नसाध्यव्यापकत्वं तद्धर्मावच्छिन्नसाधनाव्यापकत्वमित्यर्थे तात्पर्यात्~।
मित्रातनयत्वावच्छिन्नश्यामत्वस्य व्यापकं शाकपाकजत्वं, तदवच्छिन्नसाधनाव्यापकं च~। एवं पक्षधर्मबहिद्र्रव्यत्वावच्छिन्नप्रत्यक्षत्वस्य व्यापकमुद्भूतरूपवत्त्वं
बहिद्र्रव्यत्वावच्छिन्नसाधनस्याव्यापकं च~। एवं ध्वंसो विनाशी जन्यत्वादित्यत्र जन्यत्वावचच्छिन्नसाध्यव्यापकं भावत्वं बोध्यम्~। सद्धेतोस्त्वेेतादृशो धर्मो नास्ति यदवच्छिन्नस्य
साध्यस्य व्यापकं तदवच्छिन्नस्य साधनस्य चाव्यापकं किञ्चित् स्यात्~। व्यभिचारिणि त्वन्तत उपाध्यधिकरणं यत्साध्याधिकरणं यच्चोपाधिशून्यं साध्यव्यभिचारनिरूपकमधिकरणं
तदन्यतरत्वावच्छिन्नस्य साध्यस्य व्यापकत्वं साधनस्य चाव्यापकत्वमुपाधेः सम्भवतीति॥१३८॥
सर्वे साध्यसमानाधिकरणाः स्युरुपाधयः~।
हेतोरेकाश्रये येषां स्वसाध्यव्यभिचारिता॥१३९॥
अत एव लक्ष्यमपि उपाधिरूपमेतदनुसारेण दर्शयति~। सर्व इति~। स्वसाध्येति~। स्वमुपाधिः~। स्वं च साध्यं च स्वसाध्ये, तयोव्र्यभिचारितेत्यर्थः॥१३९॥
व्यभिचारस्यानुमानमुपाधेस्तु प्रयोजनम्~।
उपाधेर्दूषकताबीजमाह~। व्यभिचारस्येति~। उपाधिव्यभिचारेण साध्यव्यभिचारानुमानमुपाधेः प्रयोजनमित्यर्थः~। तथाहि यत्र शुद्धसाध्यव्यापक उपाधिस्तत्र शुद्धेनोपाधिव्यभिचारेण,
साध्यव्यभिचारानुमानम्~। यथा धूमवान् वह्नेरित्यादौ वह्निर्धूमव्यभिचारी आद्र्रेन्धनव्यभिचारित्वादिति~। व्यापकव्यभिचारिणो व्याप्यव्यभिचारावश्यकत्वात्~। यत्र तु
किञ्चिद्धर्मावच्छिन्नसाध्यव्यापक उपाधिस्तत्र तद्धर्मवत्युपाधिव्यभिचारेण साध्यव्यभिचारानुमानम्~। यथा स श्यामो मित्रातनयत्वादित्यत्र मित्रातनयत्वं श्यामत्वव्यभिचारि
मित्रातनये शाकपाकजत्वव्यभिचारित्वादिति~। बाधानुन्नीतपक्षेतरस्तु साध्यव्यापकताग्राहकप्रमाणाभावात् स्वव्याघाताकत्वाच्च नोपाधिः~। बाधोन्नीतस्तुपक्षेतर उपाधिर्भवत्येव,
यथा वह्निरनुश्ष्णः कृतकत्वादित्यादौ प्रत्यक्षेण वह्नेरुष्णत्वग्रहे वह्नितरत्वमुपाधिः~। यस्य तूपाधेः साध्यव्यापकत्वादिकं सन्दिह्यते स सन्दिग्धोपाधिः~। पक्षेतरस्तु
सन्दिग्घोपाधिरपि नोद्भावनीयः, कथकसम्प्रदायानुरोधादिति~।
केचित्तु सत्प्रतिपक्षोत्थापनमुपाधिफलम्~। तथाहि--अयोगोलकं धूमवद्वह्नेरित्यादावयोगोलकं धूमाभाववत् आद्र्रेन्धनाभावादिति सत्प्रतिपक्षसम्भवात्~। इत्थं च
साधनव्यापकोऽपि क्वचिदुपाधिः~। यथा करका पृथिवी कठिनसंयोगवत्त्वादित्यादावनुष्णाशीतस्पर्शवत्त्वम्~। न चात्र स्वरूपासिद्धिरेव दूषणमिति वाच्यम्~। सर्वत्रोपाधेर्दूषणान्तरसाङ्कर्यात्~।
अत्र च साध्यव्यापकः पक्षावृत्तिरुपाधिरिति वदन्ति~। इत्युपाधिनिरूपणम्॥
शब्दोपमानयोर्नैव पृथक्प्रामाण्यमिष्यते॥१४०॥
अनुमानगतार्थत्वादिति वैशेषिकं मतम्॥
शब्दोपमानयोरिति~। वैशेषिकाणां मते प्रत्यक्षमनुमानं च प्रमाणम्~। शब्दोपमानयोस्त्वनुमानविधयैव प्रामाण्यम्~। तथाहि दण्डेन गामभ्याजेत्यादिपदानि वैदिकपदानि वा
तात्पर्यविषयस्मारितपदार्थसंसर्गज्ञानपूर्वकाणि, आकांक्षादिमत्पदकदम्बत्वात्, घटमानयेति पदकदम्बवत्~।
यद्वैते पदार्था मिथः संसर्गवन्तः, योग्यतादिमत्पदोपस्थापितत्वात्, तादृशपदार्थवत्~। दृष्टान्तेऽपि दृष्टान्तान्तरेण साध्यसिद्धिरिति~। एवं गवयव्यक्तिप्रत्यक्षानन्तरं गवयपदं
गवयत्वप्रवृत्तिनिमित्तकम्, असति वृत्त्यन्तरे वृद्धैस्तत्र प्रयुज्यमानत्वात्~। असति च वत्त्यन्तरे यद्यत्र वृद्धैः प्रयुज्यते तत्तत्प्रवृत्तिनिमित्तकं, यथा गोपदं गोत्वप्रवृत्तिनिमित्तकम्~।
यद्वा गवयपदं सप्रवृत्तिनिमित्तकं , साधुपदत्वादित्यनुमानेन पक्षधम्र्मताबलाद् गवयत्वप्रवृत्तिनिमित्तकत्वं भासते~।
तन्न सम्यग्विना व्याप्तिबोधं शाब्दादिबोधतः॥१४१॥
तन्मतं दूषयति~। तन्न सम्यगिति~। व्याप्तिज्ञानं विनापि शाब्दबोधस्यानुभवसिद्धत्वात्~। न हि सर्वत्र शब्दश्रवणाद्यनन्तरं व्याप्तिज्ञाने मानमस्तीति~। किं च सर्वत्र
शाब्दस्थले यदि व्याप्तिज्ञानं कल्प्यते तदा सर्वत्रानुमितिस्थले पदज्ञानं कल्पयित्वा शाब्दबोध एव किं न स्वीक्रियत इति ध्येयम्॥१४०-१४१॥
इति शब्दोपमानयोः पृथक्प्रामाण्यनिरूपणम्॥
त्रैविध्यमनुमानस्य केवलान्वयिभेदतः~।
द्वैविध्यं तु भवेद् व्याप्तेरन्वयव्यतिरेकतः॥१४२॥
त्रैविध्यमिति~। अनुमानं हि त्रिविधं -केवलान्वयिकेवलव्यतिरेक्यन्वयव्यतिरेकिभेदात्~। तत्रासद्विपक्षः केवलान्वयी~। यथा ज्ञेयमभिधेयत्वादित्यादौ~। तत्र हि सर्वस्यैव
ज्ञेयत्वाद् विपक्षासत्त्वम्~। (ननु सर्वेषां धर्माणां व्यावृत्तत्वात् केवलान्वस्य सिद्धिरिति चेन्न, व्यावृत्तत्वस्य सर्वसाधारण्ये तस्यैव केवलान्वयित्वात्~। किंच वृत्तिमदत्यन्ताभावाप्रतियोगित्वं
केवलान्वयित्वम्~। तच्च गगनाभावादौ प्रसिद्धम्)~। असत्सपक्षः केवलव्यतिरेकी~। यथा पृथिवीतरेभ्यो भिद्यते, गन्धवत्त्वादित्यादौ~। तत्र हि जलादित्रयोदशभेदस्य पूर्वमसिद्धतया
निश्चितसाध्यवतः सपक्षस्याभाव इति~। सत्सपक्षविपक्षोऽन्वयव्यतिरेकी, यथा वह्निमान् धूमादित्यादौ~। तत्र सपक्षस्य महानसादर्विपक्षस्य जलह्रदादेश्च सत्त्वादिति॥१४२॥
॥इत्यनुमानत्रैविध्यनिरूपणम्॥
अन्वयव्याप्तिरुक्तैव व्यतिरेकादिहोच्यतं~।
साध्याभावव्यापकत्वं हेत्वभावस्य यद्भवेत्॥१४३
तत्र व्यतिरेकिणि व्यतिरेकव्याप्तिज्ञानं कारणं, तदर्थं व्यतिरेकव्याÏप्त निर्वक्ति~। साध्याभावव्यापकत्वमिति~। साध्याभावव्यापकीभूताभावप्रतियोगित्वमित्यर्थः~। अत्रेदं
बोध्यम्~। यत्सम्बन्धेन यदवच्छिन्नं प्रति येन सम्बन्धेन येन रूपेण व्यापकता गृह्यते तत्सम्बन्धावच्छिन्नतद्झर्मावच्छिन्नाभाववत्ताज्ञानात्
तत्सम्बन्धावच्छिन्नप्रतियोगिताकतद्धर्मावच्छिन्नाभावस्य सिद्धिः~। इत्थं च यत्र विशेषणतादिसम्बन्धेनेतरत्वव्यापकत्वं गन्धाभावे गृह्यते तत्र गन्धाभावाभावेनेतरत्वात्यन्ताभावः
सिद्ध्यति~। यत्र तु तादात्म्यसम्बन्धेनेतरव्यापकता गृह्यते तत्र तादात्म्यसम्बन्धेनेतरस्याभावः सिद्ध्यति~। स एवान्योन्याभावः~। एवं यत्र संयोगसम्बन्धेन धूमं प्रति
संयोगसम्बन्धेन वह्नेव्र्यापकता गृह्यते तत्र संयोगसम्बन्धावच्छिन्नप्रतियोगिताकवह्यभावेन जलह्रदे संयोगसम्बन्धावच्छिन्नप्रतियोगिताकधूमाभावः सिद्ध्यति~।
अत्र च व्यतिरेकव्याप्तिग्रहे व्यतिरेकसहचारज्ञानं कारणम्~।
केचितु व्यतिरेकसहचारेणान्वयव्याप्तिरेव गृह्यते न तु व्यतिरेकव्याप्तिज्ञानमपि कारणम्~। यत्र व्यतिरेकसहचाराद्व्याप्तिग्रहस्तत्र व्यतिरेकीत्युच्यते~। साध्यप्रसिद्धिस्तु
घटादावेव जाता, पश्चात् पृथिवीत्वावच्छेदेन साध्यत इति वदन्ति॥१४३॥
॥इति व्यतिरेकव्याप्तिनिरूपणम्॥
अर्थापत्तिस्तु नैवेह प्रमाणान्तरमिष्यते~।
व्यतिरेकव्याप्तिबुद्धया चरितार्था हि सा यतः॥१४४॥
अर्थापत्तिस्त्विति~। अर्थापतिं्त प्रमाणान्तरं केचन मन्यन्ते~। तथाहि यत्र देवदतस्य शतवर्षजीवित्वं ज्योतिः शास्त्रादवगतं जीविनो गृहासत्त्वं च प्रत्यक्षादवगतं तत्र
शतवर्षजीवित्वान्यथानुपपत्त्या बहिःसत्त्वं कल्प्यते~। तदप्यनुमानेन गतार्थत्वान्नेष्यते~। तथाहि यत्र जीवित्वस्य बहिःसत्त्वगृहसत्त्वान्यतरव्याप्यत्वं गृहीतं तत्रान्यतरसिद्धौ
जायमानायां गृहसत्त्वबाधाद्बहिःसत्त्वमनुमितौ भासते~। एवं पीनो देवदत्तो दिवा न भुङ्क्ते इत्यादौ पीनत्वस्य भोजनव्याप्यत्वावगमाद्भोजनं सिद्ध्यति~। दिवामोचनबाधे च
रात्रिभोजनं सिद्धयति~।
अभावप्रत्यक्षस्याऽऽनुभविकत्वादनुपलम्भोऽपि न प्रमाणान्तरम्~। किञ्चानुपलम्भस्याज्ञातस्य हेतुत्वे ज्ञानाकरणत्वात् प्रत्यक्षत्वात्, ज्ञातस्य हेतुत्वे तु
तत्प्राप्यनुपलम्भान्तरापेक्षेत्यनवस्था॥
एवं चेष्टापि न प्रमाणान्तरम्~। तस्याः सङ्केतग्राहकशब्दस्मारकत्वेन लिप्यादिसमशीलत्वाच्छब्द एवान्तर्भावात्~। यत्र तु व्याप्त्यादिग्रहस्तत्रानुमितिरेवेति॥१४४~।.
इत्यर्थापत्त्यादिनिराकरणम्~।
सुखं तु जगतामेव काम्यं धर्मेण जायते~।
अधर्मजन्यं दुःखं स्यात् प्रतिकूलं सचेतसाम्॥१४५॥
सुखं निरूपयति~। सुखं तु जगतामेवेति~। काम्यम्-अभिलाषविषयः~। धर्मेणेति~। धर्मत्वेन सुखत्वेन कार्यकारणभाव इत्यर्थः॥इति सुखनिरूपणम्॥
दुःखं निरूपयति~। अधर्मेति~। अथधर्मत्वेन दुःखत्वेन कार्यकारणभाव इत्यर्थः~। प्रतिकूलमिति~। दुःखत्वज्ञानादेव सर्वेषां स्वाभाविकद्वेषविषय इत्यर्थः॥१४५॥
॥इति दुःखनिरूपणम्॥
निर्दुःखत्वे सुखे चेच्छा तज्ज्ञानादेव जायते~।
इच्छा तु तदुपाये स्यादिष्टोपायत्वधीर्यदि॥१४६॥
इच्छां निरूपयति~। निर्दुःखत्व इति~। इच्छा हि फलविषयिणी उपायविषयिणी च~। फलं तु सुखं दुःखाभावश्च~। तत्र फलेच्छां प्रति फलज्ञानं कारणम्~। अत एव
पुरुषार्थः सम्भवति, यज्ज्ञातं सत् स्ववृत्तितयेष्यते स पुरुषार्थ इति तल्लक्षणात्~। इतरेच्छानधीनेच्छाविषयत्वं फलितोऽर्थः~।
उपायेच्छां प्रतीष्टसाधनताज्ञानं कारणम्~।
चिकीर्षा कृतिसाध्यत्वप्रकारेच्छा तु या भवेत्~।
तद्धेतुः कृतिसाध्येष्टसाधनत्वमतिर्भवत्॥१४७॥
बलवद्द्विष्टहेतुत्वमतिः स्यात्प्रतिबन्धिका~।
चिकीर्षेति~। कृतिसाध्यविषयिणीच्छा चिकीर्षा~। पाकं कृत्या साधयामीति तदनुभवात्~। चिकीर्षां प्रति कृतिसाध्यताज्ञानमिसाधनताज्ञानं च कारणम्~। तद्धेतुरिति~।
अत एव वृष्ट्यादौ कृतिसाध्यताज्ञानाभावान्न चिकीर्षा॥१४७॥
बलवदिति~। बलवदिद्द्वष्टसाधनताज्ञानं तत्र प्रतिबन्धकमतो मधुविषसम्पृक्तान्नभोजने न चिकीर्षा~। बलवद्द्वेषः प्रतिबन्धक इत्यन्ये~।
तदहेतुत्वबुद्धेस्तु हेतुत्वं कस्यचिन्मते॥१४८॥
द्विष्टसाधनताबुद्धिर्भवेद् द्वेषस्य कारणम्~।
तदहेतुत्वेति~। बलवदनिष्टाजनकत्वज्ञानं तत्र कारणमित्यर्थः~। (कृतिसाध्यताज्ञानादिमतो बलवदनिष्टसाधनताज्ञानशून्यस्य बलवदनिष्टाजनकत्वज्ञानं विनापि
चिकीर्षायां विलम्बाभावात् कस्यचिन्मत इत्यस्वरसो दर्शितः)॥१४८॥
॥इति इतीच्छानिरूपणम्॥
द्वेषं निरूपयति~। द्विष्टसाधनतेति~। दुःखोपायविषयकं द्वेषं प्रति द्विष्टसाधनताज्ञानं कारणमित्यर्थः~। बलवदिष्टसाधनताज्ञानं च प्रतिबन्धकम्~। तेन नान्तरीयकदुःखजनके
पाकादौ न द्वेषः॥इति द्वेषनिरूपणम्॥
प्रवृत्तिश्च निवृत्तिश्च तथा जीवनकारणम्॥१४९॥
एवं प्रयत्नत्रैविध्यं तान्त्रिकैः परिकीर्तितम्~।
चिकीर्षाकृतिसाध्येष्टसाधनत्वमतिस्तथा॥१५०॥
यत्नं निरूपयति~। प्रवृत्तिश्चेति~। प्रवृत्तिनिवृत्तिजीवनयोनियत्नभेेदाद् प्रयत्नस्त्रिविध इत्यर्थः॥१४६॥
चिकीर्षेत्यादि~। मधुविषसम्पृक्तान्नभोजनादौ बलवदनिष्टानुबन्धित्वेन चिकार्षाभावान्न प्रवृत्तिरिति भावः~। कृतिसाध्यताज्ञानादिवद्बलवदनिष्टाननुबन्धित्वज्ञानमपि
स्वतन्त्रान्वयव्यतिरेकाभ्यां प्रवृत्तौ कारणमित्यपि वदन्ति~।
कार्यताज्ञानं प्रवर्तकमिति गुरवः~। तथाहि-ज्ञानस्य प्रवृत्तौ जननीयायां चिकीर्षातिरिक्तं नाऽपेक्षितमस्ति~। सा च कृतिसाध्यताज्ञानसाध्या~। इच्छायाः
स्वप्रकारकधीसाध्यत्वनियमात्~। चिकीर्षा हि कृतिसाध्यत्वप्रकारकेच्छा~। तत्र कृतिसाध्यत्वं प्रकार~। तत्प्रकारकज्ञानं चिकीर्षायां कारणम्~। तद्द्वारा प्रवृत्तौ च हेतुः~। न
त्विष्टसाधनताज्ञानं तत्र हेतुः~। नित्ये तदभावात्~। कृत्यसाध्येऽपि प्रवृत्त्यापत्तेः~। कृत्यसाध्यताज्ञानं प्रतिबन्धकमिति चेन्न~। तदभावापेक्षया कृतिसाध्यताज्ञानस्य लघुत्वात्~। न
च द्वयोरेव हेतुत्वम्, गौरवात् ननु त्वन्मतेऽपि मधुविषसम्पृक्तान्नभोजने चैत्यवन्दने च प्रवृत्त्यापत्तिः, कार्यताज्ञानसत्त्वादिति चेन्न~। स्वविशेषणवत्ताप्रतिसन्धा
नजन्यकार्यताज्ञानस्य प्रवर्तकत्वात्~। काम्ये हि यागपाकादौ कामना स्वविशेषणम्~। ततश्च बलवदनिष्टाननुबन्धिकाम्यसाधनताज्ञानेन कार्यताज्ञानम्, ततः प्रवृत्तिः~।
तृप्तश्च भोजने न प्रवर्ततेे, तदानीं कामनायाः पुरुषविशेषणत्वाभावात्~। नित्ये शौचादिकं पुरुषविशेषणम्, तेन शौचादिज्ञानाधीनकृतिसाध्यताज्ञानात्तत्र प्रवृत्तिः~। ननु
तदपेक्षयां लाघवेन बलवदनिष्टाननुबन्धीष्टसाधनताज्ञानविशिष्टकार्यताज्ञानमेव हेतुरस्तु~। बलवदनिष्टाननुबन्धित्वं चेष्टोत्पत्तिनान्तरीयकदुः
खाधिकदुःखाजनकत्वम्, बलवद्द्वेषविषयदुःखाजनकत्वं वेति चेन्न~। इष्टसाधनत्वकृति साध्यत्वयोर्युगपज्ज्ञातुमशक्यत्वात्, साध्यत्वसाधनत्वयोर्विरोधात्~। असिद्धस्य हि
साध्यत्वं सिद्धस्य च साधनत्वम्~। न चैकमेकेनैकदा सिद्धमसिद्धं च ज्ञायते~। तस्मात् कालभेदादुभयं ज्ञायत इति~।
मैवम्~। लाघवेन बलवदनिष्टाननुबन्धीष्टसाधनत्वे सति कृतिसाध्यताज्ञानस्य हेतुत्वात्~। न च साध्यत्वसाधनत्वयोर्विरोधः~। यदाकदाचित् साध्यत्वसाधनत्वयोरविरोधात्,
एकदा साध्यत्वसाधनत्वयोश्चाऽज्ञानात्~। नव्यास्तु, ममेदं कृतिसाध्यमिति ज्ञानं न प्रवर्र्तकम्, अनागते तस्य ज्ञातुमशक्यत्वात्~। किन्तु यादृशस्य पुंसः कृतिसाध्यं यद्दृष्टं
तादृशत्वं स्वस्य प्रतिसन्धाय तत्र प्रवर्र्तते~। तेनौदनकामस्य तत्साधनताज्ञानवतस्तदुपकरणवतः पाकः कृतिसाध्यस्तादृशश्चाहमिति प्रतिसन्धाय पाके प्रवृत्तिरित्याहुः~।
तन्न, स्वकल्पितलिप्यादिप्रवृत्तौ यौवने कामोद्भेदादिना सम्भोगादौ च प्रवृत्तौ तदभावात्~।
इदं तु बोध्यम्~। इदानीन्तनेष्टसाधनत्वादिज्ञानं प्रवर्र्तकं, तेन भावियौवराज्ये बालस्य न प्रवृत्तिः तदानीं कृतिसाध्यत्वाज्ञानात्~। एवं तृप्तो भोजने न प्रवर्तते,
तदानीमिष्टसाधनत्वाज्ञानात्~। प्रवर्तते च रोषदूषितचित्तो विषादिभक्षणे, तदानीं बलवनिष्टाननुबन्धित्वज्ञानात्~। न चाऽऽस्तिकस्याऽगम्यागमने शत्रुवधादिप्रवृत्तौ च कथं
बलवदनिष्टाननुबन्धित्वबुद्धिः~? नरकसाधनत्वज्ञानादिति वाच्यम्~। उत्कटरागादिना नरकसाधनताधीतिरोधानात्~। वृष्ट्यादौ तु कृतिसाध्यताज्ञानाभावान्न चिकीर्षाप्रवृत्ती~।
किन्त्विष्टसाधनताज्ञानादिच्छामात्रम्~। कृतिश्च प्रवृत्तिरूपा बोध्या~। तेन जीवनयोनियन्नसाध्ये प्राणपञ्चकसञ्चारे न प्रवृत्तिः~।
इत्थं च प्रवर्तकत्वानुरोधाद्विधेरपीष्टसाधनत्वादिकमेवाऽर्थः~। इत्थं च विश्वजिता यजेतेत्यादौ यत्र फलं न श्रूयते, तत्रापि स्वर्गः फलं कल्प्यते~।
नन्वहरहः सन्ध्यामुपासीतेत्यादौ इष्टानुपपत्तेः कथं प्रवृत्तिः~? न चाऽऽर्थवादिकं ब्रह्मलोकावाप्तिः प्रत्यवायाभावो वा फलमिति वाच्यम्~। तथा सति काम्यत्वे
नित्यत्वहान्यापत्तेः~। कामनाभावेऽकरणापत्तेः~। इत्थं च यत्र फलश्रुतिस्तत्राऽर्थवादमात्रमिति चेन्न~। ग्रहणश्राद्धादौ नित्यत्वनैमित्तिकत्वयोरिव नित्यत्वकाम्यत्वयोरप्यविरोधात्~।
न च कामनाभावेऽकरणापत्तिः, त्रिकालस्तवपाठादाविव कामनासद्भावस्यैव कल्पनात्~।
ननु वेदबोधितकार्यताज्ञानात् प्रवृत्तिः सम्भवयेवेति चेन्न~। इष्टसाधनत्वमविज्ञाय तादृशकार्यताज्ञानसहस्रेणाऽपि प्रवृत्तेरसम्भवात्~।
यदपि पण्डापूर्वं फलमिति, तदपि न~। कामनाऽभावेऽकरणापत्तेस्तौल्यात्~। कामनाकल्पने त्वार्थवादिकफलमेव रात्रिसत्रन्यायात् कल्प्यते~। अन्यथा प्रवृत्त्यनुपपत्तेः~।
तेनानुत्पत्तिमेवाऽन्ये प्रत्यवायस्य मन्वते~। एवं
"सन्ध्यामुपासते ये तु सततं शंसितव्रताः~।
विधूतपापास्ते यान्ति ब्रह्मलोकमनामयम्॥"
एवं-
"दद्यादहरहः श्राद्धं पितृभ्यः प्रीतिमावहन्~।-
इत्यादिववचनप्रतिपादितब्रह्मलोकादिकमेव फलमस्तु~। न च पितृप्रीतिः कथं फलं~? व्यधिकरणत्वादिति वाच्यम्~। गयाश्राद्धादाविवोद्देश्यत्वसम्बन्धेनैव फलजनकत्वस्य
क्वचित्कल्पनात्~। अत एवोक्तं शास्त्रदर्शितं फलमनुष्ठानकर्तरीत्युत्सर्ग इति~। पितृणां मुक्तत्वे तु स्वस्य स्वर्गादिफलम्, यावन्नित्यनैमित्तिकानुष्ठानस्य सामान्यतः
स्वर्गफलकल्पनात्~। पण्डापूर्वार्थं प्रवृत्तिश्च न सम्भवति~। न हि तत् सुखं, तस्य स्वतः पुरुषार्थत्वाभावेन फलत्वाभावात्~। न वा तत्साधनम्~। प्रत्यवायानुत्पत्तौ कथं
प्रवृत्तिरिति चेत्, इत्थम्-तथाहि नित्ये कृते प्रत्यवायाभावस्तिष्ठति तदभावे तदभावः~। एवं प्रत्यवायाभावस्य सत्त्वे दुःखप्रागभावसत्त्वं तदभावे तदभाव इति योगक्षेमसाधारणकारणताया
दुःखप्रागभावं प्रत्यपि सुवचत्वात्~। एवमेव प्रायश्चित्तस्यापि दुःखप्रागभावहेतुत्वमिति~।
ननु न कलञ्जं भक्षयेदित्यत्र विध्यर्थे कथं नञर्थान्वयः~? इष्टसाधनत्वाभावस्य कृतिसाध्यत्वाभावस्य च बोधयितुमशक्यत्वादिति चेन्न~। तत्र बाधादिष्टसाधनत्वं
कृतिसाध्यत्वं च न विध्यर्थः, किन्तु बलवदनिष्टाननुबन्धित्वमात्रम्~। तदभावश्च नञा बोध्यते~। अथवा बलवदनिष्टाननुबन्धीष्टसाधनत्वे सति कृतिसाध्यत्वं विध्यर्थस्तदभावश्च
नञा बोध्यमानो विशिष्टाभावो विशेष्यवति विशेषणाभावे विश्राम्यति~।
ननु श्येनेनाऽभिचरन् यजेतेत्यत्र कथं बलवदनिष्टाननुबन्धित्वं विध्यर्थः~? श्येनस्य मरणानुकूलव्यापारस्य हिंसात्वेन नरकसाधनत्वात्~। न च वैधत्वान्न निषेध इति
वाच्यम्~। अभिचारे प्रायश्चित्तोपदेशात्~।
न च मरणानुकूलव्यापारमात्रं यदि हिंसा, तदा खड्गकारस्य कूपकर्तुश्च हिंसकत्वापत्तिः, गललग्नान्नभक्षणजन्यमरणे स्वात्मवधित्वापतिश्चेति वाच्यम्~। मरणोद्देश्यकत्वस्याऽपि
विशेषणत्वात्~। अन्योद्देश्यशक्षिप्तनाशचह्रतब्राह्मणस्य तु वाचनिकं प्रायश्चित्तमिति चेन्न, तत्र बलवदनिष्टाननुबन्धित्वस्य विध्वर्थत्वाभावात्~। वस्तुतः श्येनवारणायाऽदृष्टाद्वारकत्वेन
विशेषणीयम्~। अत एव काशीमरणाद्यर्थं कृतशिवपूजादेरपि न हिंसात्वम्~। न च साक्षान्मरणजनकस्यैव हिंसात्वं, श्येनस्तु न तथा, किं तु तज्जन्यापूर्वमिति वाच्यम्~।
खड्गाघातेन ब्राह्मणे व्रणपाकपरम्परया मृते हिंसात्वानापत्तेः~।
केचित्तु श्येनस्य हिंसा फलं, न तु मरणम्~। तेन श्येनजन्यखड्गाघातादिरूपा हिंसाऽभिचारपदार्थः~। तस्य च पापजनकत्वमतः श्येनस्य वैधत्वात् पापाजनकत्वेऽप्य
ग्रिमपापं प्रतिसन्धाय सन्तो न प्रवर्तन्ते इत्याहुः~।
आचार्यास्तु आप्ताभिप्रायो विध्यर्थः~। पाकं कुय्र्या इत्यादावाज्ञादिरूपेच्छावाचित्ववल्लिङ्मात्रस्येच्छावाचित्वम्, लाघवात्~। एवं च स्वर्गकामो यजेतेत्यादौ यागः
स्वर्गकामकृतिसाध्यतया आप्तेष्ट इत्यर्थः~। ततश्चाप्तेष्टत्वेनेष्टसाधनत्वादिकमनुमाय प्रवर्र्तन्ते~। कलञ्जभक्षणादौ तदभावान्न प्रवर्र्तते~। यस्तु वेदे पौरुषेयत्वं नाऽभ्युपैति तं
प्रति विधिरेव तावद्गर्भ इव श्रुतिकुमार्याः पुंयोगे मानम्~। न च कत्र्रस्मरणं बाधकम्~। कपिलकणादादिकमारभ्याद्यपर्यन्तं कर्तृस्मरणस्यैव प्रतीयमानत्वात्~। अन्यथा
स्मृतीनामप्यकर्तृकत्वापत्तेः~। तत्रैव कर्तृस्मरणमस्तीति चेत्, वेदेऽपि ""छन्दांसि जज्ञिरे तस्मात्-"" इत्यादिकर्तृस्मरणमस्त्येव~। एवं -
"प्रतिमन्वन्तरं चैषा श्रुतिरन्या विधीयते"- इत्यपि द्रष्टव्यम्~।
""स्वयम्भूरेष भगवान् वेदो गीतस्त्वया पुरा~।
शिवाद्या ऋषिपर्यन्ताः स्मर्तारोऽस्य न कारकाः॥""
इति तु वेदस्य स्तुतिमात्रम्~। न च पौरुषेयत्वे भ्रमादिसम्भवादप्रामाण्यं स्यादिति वाच्यम्~। नित्यसर्वज्ञत्वेन निर्दोषत्वात्~। अत एव पुरुषान्तरस्य भ्रमादिसम्भवान्न
कपिलादेः कर्तृत्वं वेदस्य~। किञ्च वर्णानामनित्यत्वस्य वक्ष्यमाणत्वात् सुतरां तत्सन्दर्भस्य वेदस्यानित्यत्वमिति॥१५०॥
उपादानस्य चाऽध्यक्षं प्रवृत्तौ जनकं भवेत्~।
निवृत्तिस्तु भवेद् द्वेषाद् द्विष्टोपायत्वधीयदि॥१५१
यत्नो जीवनयोनिस्तु सर्वदाऽतीन्द्रियो भवेत्~।
शरीरे प्राणसञ्चारे कारणं तत् प्रकीर्तितम्॥१५२॥
उपादानस्येति-उपादानस्य-समवायिकारणस्य, अध्यक्षं प्रत्यक्षं च प्रवृत्तकारणमति~।
निवृत्तिरिति~। द्विष्टसाधनताज्ञानस्य दुःखसाधनविषयकनिवृतिं्त प्रति जनकत्वमन्वयव्यतिरेकादवधारितमिति भावः॥१५१॥
यत्न इति~। जीवनयोनियत्नो यावज्जीवनमनुवर्तते~। स चातीन्द्रियः~। तत्र प्रमाणमाह~। शरीर इति~। प्राणसञ्चारो ह्यधिकश्वासादिः प्रयत्नसाध्यः~। इत्थं च प्राणसञ्चारस्य
सर्वस्य यत्नसाध्यत्वमनुमानात्~। प्रत्यक्षयत्नस्य वाधाच्चातीन्द्रययत्नसिद्धिः~। स एव जीवनयोनिः प्रयत्नः॥१५२॥इति यत्ननिरूपणम्॥
अतीन्द्रियं गुरुत्वं स्यात् पृथिव्यादिद्वये तु तत्~।
अनित्ये तदनित्यं स्यान्नित्ये नित्यमुदाहृतम्॥१५३॥
तदेवाऽसमवायि स्यात् पतनाख्ये तु कर्मणि~।
सांसिद्धिकं द्रवत्वं स्यान्नैमित्तिकमथाऽपरम्॥१५४॥
सांसिद्धिकं तु सलिले द्वितीयं क्षितितेजसोः~।
परमाणौ जले नित्यमन्यत्राऽनित्यमिष्यते॥१५५॥
नैमित्तिकं वह्नियोगात् तपनीयघृतादिषु~।
गुरुत्वं निरूपयति अतीन्द्रियमिति~।
अनित्ये-द्रव्यणुकादौ~। तत् - गुरुत्वमनित्यं, नित्ये - परमाणौ नित्यं, गुरुत्वमित्यनुवर्तते॥१५३॥
तत्-गुरुत्वम्, असमवायि-असमवायिकारणम्~। पतने-आद्यपतन इत्यर्थः~।
॥इति गुरुत्वनिरूपणम्॥
द्रवत्वं निरूपयति~। सांसिद्धिकमिति~। द्रवत्वं द्विविधं, सांसिद्धिकं नैमित्तिकं चेत्यर्थः~।
परमाणाविति~। जलपरमाणौ द्रवत्वं नित्यमन्यत्र पृथिवीपरमाण्वादौ जलद्व्यणुकादौ च द्रवत्वमनित्यम्॥१२२॥
कुत्रचित्तेजसि कुत्रचित्पृथिव्यां च नैमित्तिकं द्रवत्वम्~। तत्र को वा नैमित्तिकार्थस्तद्दर्शयति~। नैमित्तिकमिति~। वह्नीतिपदं तेजोऽर्थकम्~। तथा च तेजःसंयोगजन्यं
नैमित्तिकं द्रवत्वम्~। तच्च सुवर्णादिरूपे तेजसि, घृतजतुप्रभृतिपृथिव्यां च वर्तते इत्यर्थः~।
द्रवत्वं स्यन्दने हेतुर्निमित्तं सङ्ग्रहे तु तत्~।१५६॥
स्नेहो जले स नित्योऽणावनित्योऽवयविन्यसौ~।
तैलान्तरे तत्प्रकर्षाद् दहनस्याऽनुकूलता॥१७॥
संस्कारभेदो वेगोऽथ स्थितिस्थापकभावने~।
मूर्तमात्रे तु वेगः स्यात् कर्मजो वेगजः क्वचित्॥१५८॥
द्रवत्वं स्यन्दने हेतुरिति~। असमवायिकारणमित्यर्थः~। संग्रहे-सक्त्वादिसंयोगविशेषे~। तत्-द्रवत्वम्, स्नेहसहितमिति बोद्धव्यम्~। तेन द्रुतसुरवर्णादिना न संग्रहः॥१५६॥
इति द्रवत्वनिरूपणम्॥
स्नेहं निरूपयति~। स्नेहो जल इति~। जल एवेत्यर्थः~। असौ-स्नेहः~। ननु पृथिव्यामपि तैले स्नेह उपलभ्यते~। न चाऽसौ जलीयः~। तथा सति दहनप्रातिकूल्यप्रसङ्गादत
आह~। तैलान्तर इति~। तत्प्रकर्षात्-स्नेहप्रकर्षात्~। तैल उपलभ्यमानः स्नेहोऽपि जलीय एव, तस्य प्रकृष्टत्वादग्नेरानुकूल्यम्~। अपकृष्टस्नेहं हि जलं वह्नि नाशयतीति
भावः॥१५७॥इति स्नेहनिरूपणम्॥
संस्कारं निरूपयति~। संस्कारेति~। वेगस्थितिस्थापकभावनाभेदात् संस्कारस्त्रिविध इत्यर्थः~। मर्तमात्र इति~। कर्मजवेगजभेदाद्वेगो द्विविध इत्यर्थः~। शरादौ हि
नोदनजनितेन कर्मणा वेगो जन्यते~। तेन च पूर्वकर्मनाशः~। तत उत्तरकर्म~। एवमग्रेऽपि~। विना च वेगं कर्मणः कर्मप्रतिबन्धकत्वात् पूर्वकर्मनाश उत्तरकर्मोत्पत्तिश्च न
स्यात्~। यत्र वेगवता कपालेन जनिते घटे वेगो जन्यते स वेगजो वेगः॥१५८॥
स्थितिस्थापकसंस्कारः क्षितौ केचिच्चतुष्र्वपि~।
अतीन्द्रियोऽसौ विज्ञेयः क्वचित् स्पन्देऽपि कारणम्॥१५९॥
स्थितिस्थापकेति~। आकृष्टशाखादीनां परित्यागे पुनर्गमनस्य स्थितिस्थापकःसाध्यत्वात्~। केचिदिति~। चतुर्ष-क्षित्यादिषु स्थितिस्थापकं केचिन्मन्यन्ते तदप्रमाणमिति
भावः~। असौ-स्थितिस्थापकः~। क्वचित्-आकृष्टशाखादौ॥१५६॥
भावनाख्यस्तु संस्कारो जीववृत्तिरतीन्द्रियः~।
उपेक्षानात्मकस्तस्य निश्चयः कारणं भवेत्॥१६०॥
भावनाख्य इति~। तस्य-संस्कारस्य उपेक्षात्मकज्ञानात् संस्कारानुपपत्तेरुपेक्षानात्मक इत्युक्तम्~। तत्संशयात् संस्कारानुत्पत्तेर्निश्चय इत्युक्तम्~। तेनोपेक्षान्यनिश्चयत्वेन
हेतुत्वम्~। तेनोपेक्षादिस्थले न स्मरणम्~। इत्थं च संस्कारं प्रति ज्ञानत्वेनैव हेतुताऽस्त्विति चेन्न~। विनिगमनाविरहेण संस्कारं प्रत्यप्युपेक्षान्यनिश्चयत्वेन हेतुतायाः
सिद्धत्वात्~। किं चोपेक्षास्थले संस्कारकल्पनाया गुरुत्वात् संस्कारं प्रति चोपेक्षान्यनिश्चयत्वेन हेतुतायाः सिद्धत्वात्॥१६०॥
स्मरणे प्रत्यभिज्ञायामप्यसौ हेतुरुच्यते~।
तत्र प्रमाणं दर्शयति~। स्मरण इति~। असौ-संस्कारो यतः स्मरणं प्रत्यभिज्ञानं च जनयत्यतः संस्कारः कल्प्यते~। विना व्यापारं पूर्वानुभवस्य स्मरणादिजननासामात्थ्र्यात्,
स्वस्वव्यापारान्यतराभावे कारणत्वासम्भवात्॥न च प्रत्यभिज्ञां प्रति तत्तत्संस्कारस्य हेतुत्वे प्रत्यभिज्ञायाः संस्कारजन्यत्वेन स्मृतित्वापत्तिरिति वाच्यम्~। अप्रयोजकत्वात्~।
परे त्वनुद्बुद्धसंस्कारात् प्रत्यभिज्ञानुदयादुद्बुद्धसंस्कारस्य हेतुत्वापेक्षया तत्तत्स्मरणस्यैव प्रत्यभिज्ञां प्रति हेतुत्वं कल्प्यत इत्याहुः~।
॥इति संस्कारनिरूपणम्॥
धर्माधर्मावदृष्टं स्याद्, धर्मः स्वर्गादिकारणम्॥१६१॥
गङ्गास्नानादियागादिव्यापारः स तु कीर्तितः~।
अदृष्टं निरूपयति~। धर्माधर्माविति~। स्वर्गादिसकलसुखानां स्वर्गसाधनीभूतशरीरादीनां च साधन धर्म इत्यर्थः॥१६१॥
तत्र प्रमाणं दर्शयितुमाह यागादिति~। यागादिव्यापारतया हि धर्मः कल्प्यते~। अन्यथा यागादीनां चिरविनष्टतया निव्र्यापारतया च कालान्तरभाविस्वर्गजनकत्वं न
स्यात्~। तदुक्तमाचार्यैः-
"चिरध्वस्तं फलायाऽलं न कर्मातिशयं विना"-इति~।
ननु यागध्वंस एव व्यापारः स्यात्~। न च प्रतियोगिध्वंसयोरेकत्राऽजनकत्वम्, सर्वत्र तथात्वे मानाभावात्~। न च त्वन्मते फलानन्त्यं, मन्मते चरमफलस्याऽपूर्वनाश
कत्वान्न तथात्वमिति वाच्यम्~। कालविशेषस्य सहकारित्वादित्यत आह~। गङ्गास्नानेति~। गङ्गास्नानस्य हि स्वर्गजनकत्वेऽनन्तानां जलसंयोगध्वंसानां व्यापारत्वमपेक्ष्यैकमेवाऽपूर्वं
कल्प्यते, लाघवादिति भावः~।
कर्मनाशाजलस्पर्शादिना नाश्यस्त्वसौ मतः॥१६२॥
ननु ध्वंसोऽपि न व्यापारोऽस्तु~। न च निव्र्यापारस्य चिरध्वस्तस्य कथं कारणत्वमिति वाच्यम्~। अनन्यथासिद्धनियतपूर्ववर्तित्वस्य तत्रापि सत्त्वात्~। अव्यवहितपूर्ववर्त्तित्वं
हि चक्षुःसंयोगादेः कारणत्वं, न तु सर्वत्र, कार्यकालवृत्तित्वमिव समवायिकारणस्य कारणत्व इत्यत आह~। कर्मनाशेति~। यदि ह्यपूर्वं न स्यात्तदा कर्मनाशाजलस्पर्शादिना
नाश्यत्वं धर्मस्य न स्यात्~। न हि तेन यागादिनाशः प्रतिबन्धो वा कत्र्तुं शक्यते, तस्य पूर्वमेव वृत्तत्वादिति~।
एतेन देवताप्रीतिरेव फलमित्यपास्तम्~। गङ्गास्नानादौ सर्वत्र देवताप्रीतेरसम्भवात्~। देवतायाश्चेतनत्वेऽपि तत्प्रीतेरनुद्देश्यत्वात्~। प्रीतेः सुखस्वरूपत्वेन
विष्णुप्रीत्यादौ तदसम्भवात्~। जन्यसुखादेस्तत्राभावात्~। तेन विष्णुप्रीतिजन्यत्वेन पराभिमतस्वर्गादिरेव विष्णुप्रीतिशब्देन कथ्यते॥१६२॥
अधर्मो नरकादीनां हेतुर्निन्दितकर्मजः~।
प्रायश्चित्तादिनाश्योऽसौ जीववृत्ती त्विमौ गुणौ॥१६३॥
अधर्मो नरकादीनामिति~। नरकदुःखादिसकलदुःखानां नारकीयशरीरादीनां च साधनमधर्म इत्यर्थः~। प्रमाणमाह~। प्रायश्चित्तेति~। यदि, ह्यधर्मो न स्यात् तदा
प्रायश्चित्तादिना नाश्यत्वमधर्मस्य न स्यात्~। न हि तेन ब्रह्महननादीनां नाशः प्रतिबन्धो वा विधातुं शक्यते, तस्य पूर्वमेव विनष्टत्वादिति भावः~। जीवेति~। ईश्वरस्य
धर्माधर्माभावादिति भावः॥१६३॥
इमौ तु वासनाजन्यौ ज्ञानादपि विनश्यतः~।
शब्दो ध्वनिश्च वर्णश्च मृदङ्गादिभवो ध्वनिः॥१६४॥
कण्ठसंयोगादिजन्या वर्णास्ते कादयो मताः
सर्वः शब्दो नभोवृत्तिः श्रोत्रोत्पन्नस्तु गृह्यते॥१६५॥
वीचीतरङ्गन्यायेन तदुत्पत्तिस्तु कीर्तिता~।
कदम्बगोलकन्यायादुत्पत्तिः कस्यचिन्मते॥१६६॥
उत्पन्नः को विनष्टः क इति बुद्धेरनित्यता~।
सोऽयं क इति बुद्धिस्तु साजात्यमवलम्बते॥१६७॥
इमौ-धर्माघर्मौं~। वासनेति~। अतो ज्ञानिना कृते अपि सुकृतदुप्कृतकर्मणी न फलायाऽलमिति भावः~। ज्ञानादपीत्यपिना भोगपरिग्रहः~। ननु तत्त्वज्ञानस्य कथं
धर्माधर्मनाशकत्वं~? ""नाभुक्तं क्षीयते कर्म कल्पकोटिशतैरपि"" इतिवचनविरोधात्~। इत्थं च तत्त्वज्ञानिनां झटिति कायव्यूहेन सकलकर्मणां भोगेन क्षय इति चेन्न~। तत्र
भोगस्य वेदबोधितनाशकोपलक्षकत्वात्~। कथमन्यथा प्रायश्चित्तादिना कर्मणां नाशः~? तदुक्तम्, ""ज्ञानाग्निः सर्वकर्माणि भस्मसात्कुरुतेऽर्जुन"" इत्यादिना~। श्रूयते च-
""क्षीयन्ते चास्य कर्माणि तस्मिन् दृष्टे परावरे~।"" इति~।
ननु तत्त्वज्ञानिनस्तर्हि शरीरावस्थानं सुखदुःखादि च न स्यात् ज्ञानेन सर्वेषां कर्मणां नाशादिति चेन्न~। प्रारब्धेतरकर्मणामेव नाशात्~। तत्तच्छरीरभोगजनकं हि यत्कर्म
तत् प्रारब्धम्~। तदभिप्रायमेव नाऽभुक्तमिति वचनमिति~। इत्यदृष्टनिरूपणम्॥
शब्दं निरूपयति~। शब्दो ध्वनिश्चेति॥१६३~।
नभोेवृत्तिः आकाशसमवेतः~। दूरस्थशब्दस्याऽग्रहणादाह~। श्रोत्रेति॥१६५॥
ननु मृदङ्गाद्यवच्छेदेनोत्पन्ने शब्दे श्रोत्रे कथमुत्पत्तिरत आह~। वीचीति~। आद्यशब्देन बहिर्दशदिगवच्छिन्नेऽन्यः शब्दस्तेनैव शब्देन जन्यते तेन चापरस्तद्व्यापकः~।
एवंक्रमेण श्रोत्रोत्पन्नो गृह्यते इति~। कदम्बेति~। आद्यशब्दादृशसु दिक्षु दश शब्दा उत्पद्यन्ते~। ततश्चान्ये दश शब्दा उत्पद्यन्त इति भावः~। अस्मिन् कल्पे कल्पनागौरवादुक्तं
कस्यचिन्मत इति॥१६६॥
ननु शब्दस्य नित्यत्वादुत्पत्तिः कथमत आह~। उत्पन्न इति~। शब्दानामुत्पादविनाशशालित्वादनित्यत्वमित्यर्थः~। ननु स एवायं ककार इत्यादिप्रत्यभिज्ञानाच्छब्दानां
नित्यत्वम्~। इत्थं चोत्पादनाशबुद्धिभ्र्रमरूपैवेत्यत आह~। सोऽयं क इति~। तत्र प्रत्यभिज्ञानस्य तत्सजातीयत्वं विषयो न तु तद्व्यक्त्यभेदो विषयः, उक्तप्रतीतिविरोधात्~। इत्थं
च द्वयोरपि बुद्ध्योर्न भ्रमत्वमिति॥१६७॥
ननु सजातीयत्वं सोऽयमिति प्रत्यभिज्ञायां भासते इति कुत्र दृष्टमित्यत आह~।
तदेवौषधमित्यादौ सजातीयेऽपि दर्शनात्~।
तस्मादनित्या एवेति वर्णाः सर्वे मतं हि नः॥१६८॥
इति श्रीविश्वनाथपञ्चाननकृता कारिकावली समाप्ता~।
तदेवेति~। यदौषधं मया कृतं तदेवान्योनापि कृतमित्यादिदर्शनादिति भावः~।
इति सिद्धान्तमुक्तावल्यां गुणनिरूपणम्॥
॥इति शब्दनिरूपणम्॥
इति श्रीमहामहोपध्यायविद्यानिवासभट्टाचार्यपुत्रश्रीयुतविश्वनाथपञ्चानन
भट्टाचार्यविरचिता न्यायसिद्धान्तमुक्तावली सम्पूर्णा~।
\end{document}